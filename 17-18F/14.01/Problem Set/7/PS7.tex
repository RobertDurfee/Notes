\documentclass{article}
\usepackage{tikz}
\usepackage{float}
\usepackage{enumerate}
\usepackage{amsmath}
\usepackage{bm}
\usepackage{indentfirst}
\usepackage{siunitx}
\usepackage[utf8]{inputenc}
\usepackage{graphicx}
\graphicspath{ {Images/} }
\usepackage{float}
\usepackage{mhchem}
\usepackage{chemfig}
\allowdisplaybreaks

\title{ 14.01 Problem Set 7 }
\author{ Robert Durfee }
\date{ November 30, 2017 }

\begin{document}

\maketitle

\section{ True or False }

\begin{enumerate}[i.]
    \item \textbf{True.} As long as the person is maximizing their present
        value, they will prefer option two.
        
        Calculating the present value for option one:
        \begin{align*}
            PV_{1} &= \$1200 + \frac{ \$120 }{ 1 + r } + \frac{ \$120 }{ ( 1 + r
        )^{ 2 } } + \frac{ \$120 }{ ( 1 + r )^{3} }
        \end{align*}

        Calculating the present value for option two:
        \begin{align*}
            PV_{2} &= \$360 + \frac{ \$240 }{ 1 + r } + \frac{ \$240 }{ ( 1
        + r )^{2} } + \frac{ \$240 }{ ( 1 + r )^{3} }
        \end{align*}
        
    \item \textbf{False.} People can be risk loving, averse, or neutral
        depending on their utility function. As a result, a risk loving person
        could prefer the \$50 risky investment, just because it involves risk.

        Using a risk-loving utility function:
        $$ U(x) = x^{2} $$

        Consider a possible (extremely) risky investment:
        $$ E_{1}( x ) =  0.99 \cdot 0 + 0.01 \cdot 5000 = 50 $$
        $$ E_{1}( U( x ) ) = 0.99 \cdot 0^{2} + 0.01 \cdot
        5000^{2} = 250,000 $$

        Now the safe money:
        $$  E_{2}( x ) = 1.0 \cdot 100 = 100 $$
        $$ E_{2}(U( x )) = 1.0 \cdot 100^{2} = 10,000 $$

        Therefore, since people maximize expected utility, not expected return,
        this person would prefer the risky investment.

    \item \textbf{False.} As we saw in the last problem set, ther are utility
        function for which a rise in interest rate has no effect on the
        consumption in the first period.
        $$ U( c_{1}, c_{2} ) = \ln( c_{1} ) + \ln( c_{2} ) $$

        For this intertemporal utility function, a rise in interest will not
        effect $c_{1}$.

    \item \textbf{False.} As long as there is an end to the prisoners dilemma
        (it is finite), there will be incentive to cheat on the last round, if
        cooperation exists. Once this fact is established, the first person to
        cheat gets the most beneft. As a result, they will both cheat on the
        first round.

\end{enumerate}

\section{ International Trade }

\begin{enumerate}[i.]
    \item Both France and Spain have an absolute advantage in cheese.  However,
        France has a comparative advantage in cheese and Spain has a comparative
        advantage in wine. France is 1.5 times better at producing cheese than
        Spain, and only 1.1 times better at producing wine. Conversely,
        Spain is 1.5 times worse at producing cheese than France is, but only
        1.1 times worse at producing wine.

    \item Yes, it is in both countries best interest to trade (just looking at
        efficiency, ignoring job displacement). Both countries can consume more
        of each good becuase the joint PPF surpasses each individual PPF. France
        may be better at producing each good, however, it is not better at
        producing each good equally. As a result, there is still incentive to
        trade.

\end{enumerate}

\section{ Decision-Making Under Uncertainty }

\begin{enumerate}[i.]
    \item Expected value for government bonds purchased at a price of \$100:
        $$ E_{B} = \frac{ 1 }{ 4 } \cdot \frac{ 103 }{ 1 - 0.01 } + \frac{ 1 }{
        2 } \cdot \frac{ 103 }{ 1 + 0.02 } + \frac{ 1 }{ 4 } \cdot \frac{ 103 }{
        1 + 0.05 } = 101.024 $$

        Expected value for treasury inflation protected securities purchased at
        a price of \$102:
        $$ E_{T} = \frac{ 1 }{ 4 } \cdot \frac{ 103 \cdot ( 1 - 0.01 ) }{ 1 -
        0.01 } + \frac{ 1 }{ 2 } \cdot \frac{ 103 \cdot ( 1 + 0.02 ) }{ 1 + 0.02
        } + \frac{ 1 }{ 4 } \cdot \frac{ 103 \cdot ( 1 + 0.05 ) }{ 1 + 0.05 } $$

        Canceling inflation terms:
        $$ E_{T} = 103 $$

    \item Original utility:
        $$ U( 100 ) = 10 $$
        
        Expected utility of the payoff of government bonds:
        $$ E_{B}[ U( x ) ] = \frac{ 1 }{ 4 } \sqrt{ \frac{ 103 }{ 1 - 0.01 } }
        + \frac{ 1 }{ 2 } \sqrt{ \frac{ 103 }{ 1 + 0.02 } } + \frac{ 1 }{ 4 }
        \sqrt{ \frac{ 103 }{ 1 + 0.05 } } = 10.0505$$

        Expected utility of the payoff of treasury inflation protected
        securities (with inflation terms cancelled):
        $$ E_{T}[ U( x ) ] = \frac{ 1 }{ 4 } \sqrt{ \frac{ 103  }{ 1 + 0.02 } }
        + \frac{ 1 }{ 2 } \sqrt{ \frac{ 103 }{ 1 + 0.02 } } + \frac{ 1 }{ 4 }
        \sqrt{ \frac{ 103 } { 1 + 0.02 } } = 10.0489$$

        To maximize expected utility, the best investment is the government
        bond. This is surprising because, at first glance, the phrase "inflation
        protected" seems more secure and the expected value seems to be higher.
        However, the premium payed causes the TIPS to cost more and reduces the
        payoff.

    \item Setting expected utility of the payoff of government bonds equal to
        the expected utility of the payoff of treasury inflation protected
        securities:
        $$ \sqrt{ \frac{ 103 }{ 1 + P } } = 10.0505 $$

        The maximum premium is:
        $$ P = 0.01967 $$

    \item Original uttility:
        $$ U( 100 ) = 10 $$

        Expected utility of the payoff of government bonds:
        $$ E_{B}[ U( x ) ] = \frac{ 1 }{ 4 } \sqrt{ \frac{ 103 }{ 1 - 0 } }
        + \frac{ 1 }{ 2 } \sqrt{ \frac{ 103 }{ 1 + 0.05 } } + \frac{ 1 }{ 4 }
        \sqrt{ \frac{ 103 }{ 1 + 0.30 } } = 9.71467$$

        Expected utility of the payoff of treasury inflation protected
        securities (with inflation terms cancelled):
        $$ E_{T}[ U( x ) ] = \frac{ 1 }{ 4 } \sqrt{ \frac{ 103  }{ 1 + 0.02 } }
        + \frac{ 1 }{ 2 } \sqrt{ \frac{ 103 }{ 1 + 0.02 } } + \frac{ 1 }{ 4 }
        \sqrt{ \frac{ 103 } { 1 + 0.02 } } = 10.0489$$

        Setting expected utility of the payoff of government bonds equal to
        the expected utility of the payoff of treasury inflation protected
        securities:
        $$ \sqrt{ \frac{ 103 }{ 1 + P } } = 9.71467 $$

        The maximum premium is:
        $$ P = 0.091393 $$

\end{enumerate}

\section{ Diversification }

\begin{enumerate}[i.]
    \item Original utility:
        $$ U( 10,000 ) = 100 $$

        Expected utility of buying bonds:
        $$ E_{B}[ U( x ) ] = 1.0 \cdot \sqrt{ 10,000 \cdot ( 1 + 0.01 )} = 100.499 $$

        Expected utility of buying Amazon stock:
        $$ E_{A}[ U( x ) ] = 0.5 \cdot \sqrt{ 10,000 \cdot 4 } + 0.5 \cdot \sqrt{ 0 }
        = 100 $$

        Oliver prefers to buy bonds over Amazon stock.

    \item There are four possibilities for this strategy: 

        \begin{itemize}
        
            \item Amazon increases and Walmart decreases with a worth of
                \$20,000 (payoff of \$10,000) at a probability of 25\%.  
        
            \item Walmart increases and Amazon decreases with a worth of
                \$20,000 (payoff of \$10,000) and a probability of 25\%. 
        
            \item Both Amazon and Walmart increase with a worth of \$40,000
                (payoff of \$30,000) and a probability of 25\%.  
                
            \item Both Amazon and Walmart decrease with a worth of \$0 (payoff
                of -\$10,000) and a  probability of 25\%.

        \end{itemize}

    \item The expected value of this strategy:
        $$ E = 0.25 \cdot 10,000 \cdot 4 + 0.50 \cdot 5,000 \cdot 4 + 0,25 \cdot
        0 = 20,000$$

        The expected utility of this strategy:
        $$ E[ U( x) ) ] = 0.25 \cdot \sqrt{ 10,000 \cdot 4 } + 0.5 \cdot \sqrt{
        5,000 \cdot 4 } + 0.25 \cdot \sqrt{ 0 } =  120.711$$

        Yes, Oliver prefers this strategy over buying bonds. This is because of
        diversification where investing in multiple, uncorrelated stocks will
        reduce risk.

    \item It depends on what Oliver's timeline is. If he needs the money in the
        near future, less risky options are better because the value only
        approaches the expected value over a long period of time. In general, it
        is probably best for Oliver to invest in bonds as well, from the money
        he needs short term.

\end{enumerate}

\section{ European Potatoes }

\begin{enumerate}[i.]
    \item Yes, the European producers will still sell to European consumers, but
        not as much as they have before. \textit{Figure 1} and \textit{Figure 2}
        show the supply and demand in Europe along with the world price before
        the subsidy and the perceived world price after the subsidy (the green
        line in both graphs).  The x-axis is quantity and the y-axis is price.
        The red line is supply and the blue line is demand with Europe.
        Importantly, the European consumers can still consume at the world
        price, they will import the difference.

    \item Exports will increase as a result of the subsidy as the producers will
        receive a higher, subsidized price for exporting. This is shown
        graphically in \textit{Figure 2} as the length of the green line between
        the supply and demand curves, which has increased from \textit{Figure
        1}.

        \begin{figure}[H]
            \centering
            \includegraphics[scale=0.40]{"Before Subsidy"}
            \caption{Before Subsidy}
        \end{figure}

        \begin{figure}[H]
            \centering
            \includegraphics[scale=0.40]{"After Subsidy"}
            \caption{After Subsidy}
        \end{figure}

    \item Initial consumer surplus (under demand
        curve, above world price):
        $$ CS_{i} = \frac{ 1 }{ 2 } \cdot ( 1000 - 950 ) \cdot ( 25 - 0 ) = 625 $$

        Since consumers can still consume at the world price, through imports,
        their surplus remains the same after the subsidy.
        $$ \Delta CS = 0 $$

    \item Initial producer surplus (above supply curve, below world price):
        $$ PS_{i} = \frac{ 1 }{ 2 } \cdot ( 950 - 400 ) \cdot ( 55 - 0 ) = 15,125 $$

        Final producer surplus:
        $$ PS_{f} = \frac{ 1 }{ 2 } \cdot ( 990 - 400 ) \cdot ( 59 - 0 ) = 17,405 $$

        Change in producer surplus:
        $$ \Delta PS = 17,405 - 15,125 = 2,280 $$

    \item The cost of the subsidy is given by the amount of exports multiplied
        by the unit subsidy where the amount of exports is the quantity supplied
        at the perceived world price:
        $$ C = ( 59 - 0 ) \cdot ( 990 - 950 ) = 2,360 $$

    \item The initial total welfare is:
        $$ W_{i} = 625 + 15,125 = 15,750 $$

        The final total welfare is:
        $$ W_{f} = 625 + 17,405 -2,360 = 15,670 $$

        Change in total welfare is:
        $$ \Delta W = 15,670 - 15,750 = -80 $$

\end{enumerate}

\end{document}
