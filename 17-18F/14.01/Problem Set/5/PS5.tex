\documentclass{article}
\usepackage{enumerate}
\usepackage{bm}
\usepackage{amsmath}
\usepackage[utf8]{inputenc}

\title{14.01 Problem Set 5}
\author{Robert Durfee}
\date{November 3, 2017}

\begin{document}

\maketitle

\section*{Problem 1}

\begin{enumerate}
    \item \textbf{False}. In a Nash equilibrium in a two-player game, both
        players do not have to select a dominate strategy.
    \item \textbf{False}. Under perfect price discrimination, each consumer pays
        their willingness to pay. As a result, there is no consumer surplus.
        Although total surplus is higher, it is all given to the producer where
        as in uniform price it is given to both consumer and producer, but there
        is deadweight loss.
    \item \textbf{False}. Without the prisoners talking to one another, they
        will always choose the dominant non-cooperative strategy.
    \item \textbf{False}. If a monopoly has a uniform price, there can be
        deadweight loss associated with the price set by the monopoly. In this
        case, giving the monopoly's profit to consumers will transfer welfare,
        but the deadweight loss will still exist.
    \item \textbf{False}. Nash equilibrium is defined as the point where neither
        player wants to deviate unilaterally. As a result, there can be a Nash
        equilibrium at a point where the sum of both players is less than
        another point. For example:
        \begin{center}
            \begin{tabular}{c c c}
                 & C & D \\
                C & (2,2) & (0,3) \\
                D & (3,0) & (1,1)
            \end{tabular}
        \end{center}
    \item \textbf{False}. As the quantity sold increases, the price of the good
        decreases. In a perfectly competitive market, firms will enter and exit
        until a single, minimum price is reached. If he charges above that
        price, he will have no revenue as consumers will go to the less
        expensive firms. If he charges below that price, he will have to
        shutdown.
\end{enumerate}

\section*{Problem 2}

\begin{enumerate}
    \item 
    \begin{enumerate}[i]
        \item There are \textbf{no dominant strategies}.
        \item The pure strategy Nash equilibria are \textbf{(Down, Left)} and
            \textbf{(Up, Right)}.
    \end{enumerate}
    \item
    \begin{enumerate}[i]
        \item Player 1's dominant strategy is \textbf{Up}. Player 2 has
            \textbf{no dominant strategy}.
        \item The pure strategy Nash equilibrium is \textbf{(Up, Left)}.
    \end{enumerate}
    \item
    \begin{enumerate}[i]
        \item There are \textbf{no dominant strategies}.
        \item There is \textbf{no pure strategy Nash equilibrium}.
    \end{enumerate}
    
\end{enumerate}

\section*{Problem 3}

\begin{enumerate}
    \item There are \textbf{no dominant strategies}. There is \textbf{no pure
        strategy Nash equilibrium}.
    \item (Scissors, Rock) becomes a pure strategy Nash equilibrium when
        $\bm{\delta \geq 1}$.
\end{enumerate}

\section*{Problem 4}

\begin{enumerate}
    \item Price elasticity of demand is defined as:
    \begin{align*}
        \varepsilon(P)&=\frac{\partial Q_{D}}{\partial P}\frac{P}{Q_{D}}
    \intertext{Substituting in the demand function:}
        \varepsilon(P)&=\frac{\partial(10-P)}{\partial P}\frac{P}{10-P}\\ 
        \bm{\varepsilon(P)}&\bm{=-\frac{P}{10-P}}
    \end{align*}
    \item A firm's revenue is given by:
    \begin{align*}R&=Q_{D}\cdot P
    \intertext{Taking the partial derivative with respect to quantity give the
    marginal revenue equation:}
        MR&=\frac{\partial R}{\partial Q_{D}}\\
        MR&=P+\frac{\partial P}{\partial Q_{D}}Q_{D}
    \intertext{Knowing the definition for price elasticity of demand, this
    equation can be simplified to:}
        MR&=P\left(1+\frac{\partial P}{\partial Q_{D}}\frac{Q_{D}}{P}\right)\\
        MR&=P\left(1+\frac{1}{\varepsilon}\right)
    \intertext{Any given firm will set marginal cost equal to marginal revenue
    to maximize profits:}
        MC&=P\left(1+\frac{1}{\varepsilon}\right)\\
    \bm{\frac{P-MC}{P}}&\bm{=-\frac{1}{\varepsilon}}
    \end{align*}
    \item To calculate the monopoly's profit maximizing quantity, set marginal
        cost equal to marginal revenue:
    \begin{align*}
        Q+1&=\frac{\partial((10-Q)(Q))}{\partial Q}\\
        Q+1&=10-2Q\\ 
        \bm{Q}&\bm{=3}
    \intertext{To calculate the monopoly's price, plug the quantity into the
    demand function:}
        P&=10-3\\
        \bm{P}&\bm{=7}
    \end{align*}
    \item If the demand becomes more inelastic, the curve will become steeper.
        As a result, the price markup will increase and thus the market price.
        This makes sense as an inelastic demand means there is very low
        substitutability with other goods. Therefore, the monopoly can get away
        with charging a higher price because there are fewer close alternatives
        to their product. However, the firm must also worry about firm entry if
        they set the price too high.
    
\end{enumerate}

\section*{Problem 5}

\begin{enumerate}
    \item There are two different consumers for Amtrak, businesspeople and
        vacationers, with different demand functions:
    \begin{align*}
        Q_{B}&=100-P_{B}\\
        Q_{V}&=40-\frac{1}{2}P_{V}
    \end{align*}
    There are also unequal amounts of both types of consumers, 50 businesspeople
    and 20 vacationers. Therefore, each quantity demanded should be multiplied
    by the number of people who it applies to:
    \begin{align*}
        Q_{BT}&=5000-50P_{B}\\
        Q_{VT}&=800-10P_{V}
    \end{align*}
    Calculating the inverse demand functions for each type of consumer:
    \begin{align*}
        P_{B}&=100-\frac{Q_{BT}}{50}\\
        P_{V}&=80-\frac{Q_{VT}}{10}
    \end{align*}
    The marginal revenue Amtrak will receive for each type of consumer:
    \begin{align*}
        MR_{B}&=100-\frac{Q_{BT}}{25}\\
        MR_{V}&=80-\frac{Q_{VT}}{5}
    \end{align*}
    The marginal cost faced by Amtrak can be written in terms of each quantity
    demanded by each consumer:
    \begin{align*}
        MC&=0.1Q_{T}\\
        MC&=0.1(Q_{BT}+Q_{VT})
    \end{align*}
    To calculate Amtrak's profit maximizing quantity for each type of consumer,
    set the marginal cost equal to marginal revenue for each consumer.
    \begin{align*}
        0.1(Q_{BT}+Q_{VT})&=100-\frac{Q_{BT}}{25}\\
        0.1(Q_{BT}+Q_{VT})&=80-\frac{Q_{VT}}{5}
    \end{align*}
    Solving the system of equations results in:
    \begin{align*}
        Q_{BT}&=687.5\\
        Q_{VT}&=37.5
    \end{align*}
    These quantities are for total number of trips purchased by each group of
    consumers so to determine the number of trips, divide by the number of
    consumers in each category:
    \begin{align*}
        Q_{B}&=\frac{Q_{BT}}{N_{B}}=\frac{687.5}{50}=\bm{13.75}\\
        Q_{V}&=\frac{Q_{VT}}{N_{V}}=\frac{37.5}{20}=\bm{1.875}
    \end{align*}
    \item To calculate Amtrak's price for each type of consumer, plug the
        quantity into each inverse demand function. 
    \begin{align*}
        P_{B}&=100-\frac{687.5}{50}=\bm{86.25}\\
        P_{V}&=80-\frac{37.5}{10}=\bm{76.25}
    \end{align*}
    
\end{enumerate}

\end{document}
