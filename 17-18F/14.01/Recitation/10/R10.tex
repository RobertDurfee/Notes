\documentclass{article}
\usepackage{tikz}
\usepackage{float}
\usepackage{enumerate}
\usepackage{amsmath}
\usepackage{bm}
\usepackage{indentfirst}
\usepackage{siunitx}
\usepackage[utf8]{inputenc}
\usepackage{graphicx}
\graphicspath{ {Images/} }
\usepackage{float}
\usepackage{mhchem}
\usepackage{chemfig}
\allowdisplaybreaks

\title{ 14.01 Recitation 10 }
\author{ Robert Durfee }
\date{ December 8, 2017 }

\begin{document}

\maketitle

\section{ True or False }

In the following economy, the US never trades.
\begin{center}
    \begin{tabular}{ c c c }
        & US & Spain \\
        Cost of Apples & 3 & 6 \\
        Cost of Oranges & 11 & 12 \\
    \end{tabular}
\end{center}

\textbf{False}. Although US has an absolute advantage in both, Spain has
a comparative advantage in apples.

\bigbreak

If the cost of producing Oranges in the US reduces to 6, then there is
no absolute advantage for Spain and the US has no incentive to trade.

\section{ Consumption Problem }

There is one individual, Shelby, who cares only about two goods ($T$, $M$). Her
utility function is:
$$ U = 2 \cdot \sqrt{ T \cdot M } $$

The cost for tea is \$4 and the cost of oatmeal is \$9. Shelby's income is
\$288. 

\begin{enumerate}[1.]
    \item Draw the indifference curve.

        \begin{figure}[H]
            \centering
            \includegraphics[scale=0.25]{"Indifference Curves"}
            \caption{Indifference Curves}
        \end{figure}

    \item Find Shelby's optimal consumption bundle ($T^{*}$, $M^{*}$).

        The slope of the indifference curve:
        $$ - \frac{ \partial U / \partial T}{ \partial U / \partial M } = - \frac{
        M}{ \sqrt{ M \cdot T } } \frac{ \sqrt{ T \cdot M } }{ T } = - \frac{ M }{
        T}$$

        Setting this equal to the slope of the budget constraint:
        $$ -\frac{ M }{ T } = -\frac{ P_{T} }{ P_{M} } $$
        $$ \frac{ M }{ T } = \frac{ 4 }{ 9 }$$

        Substituting into the budget constraint:
        $$ M^{*} = 16$$
        $$ T^{*} = 36$$
        $$ U^{*} = 48 $$

\end{enumerate}

\section{ Production Problem }

Consider a producer with the production function:
$$ F( K,L ) = 4K + L$$

The factor prices are:
$$ w = 3,\ r = 6 $$

\begin{enumerate}[1.]
    \item Does the production function exhibit increasing, decreases, or
        constant returns to scale?
        $$ F( 2K, 2L ) = 4( 2K ) + 2L = 2 F( K, L ) $$

        As a result, this is constant returns to scale.

    \item Write the cost function in terms of capital, labor, return to capital
        and wage.

        $$ C( K, L ) = rK + wL$$

    \item Find optimal choice of capital and labor in terms of $q$. 

        Slope of production function:
        $$ -\frac{ \partial F / \partial K }{ \partial F / \partial L } = 4 $$

        Slope of budget constraint:
        $$ -\frac{ r }{ w } = 2 $$

        This shows that there will be no labor consumed and only capital.
        $$ L^{*} = 0,\ K^{*} = q/4 $$

        The isoquant:
        $$ q = 4K + L $$

        The isocost:
        $$ C = 3L + 6K $$

        These curves only intercept at the K-axis. This is what causes the firm
        to only purchase capital.

    \item TBD

    \item Find long-term supply curve.

        $$ C( q ) = r K( q ) = \frac{ 6 q }{ 4 } = 1.5 q $$
        $$ MC( q ) = 1.5 $$

    \item Suppose the demand is given by $P = 11.5 - 2Q$. Find the equilibrium
        P, Q in a competitive market. 

        We know that the price has to be $1.5$ because the market is perfectly
        competitive. 

        Setting supply equal to demand:
        $$ 1.5 = 11.5 - 2Q $$
        $$ Q^{*} = 5,\ P^{*} = 1.5 $$

    \item Suppose the market were run by a monopolist. How many units will he
        monopolist sell?

        Determine the revenue:
        $$ R( Q ) = Q( 11.5 - 2Q ) $$
        $$ MR( Q ) = 11.5 - 4Q $$

        Set marginal revenue equal to marginal cost:
        $$ 1.5 = 11.5 - 4Q $$
        $$ Q^{*} = 10/4 $$

        Plugging this back into the demand function for price:
        $$ P = 11.5 - 2Q  $$
        $$ P^{*} = 6.5 $$

    \item How much would a firm be willing to pay the government for monopoly
        rights?

        The firm will be willing to pay up to its profit to allow the right to
        act as a monopoly. This is what their profits would be in a perfectly
        competitve market. 

        $$ \pi = R - C $$
        $$ \pi = 2.5 ( 6.5 - 1.5 ) = 12.5 $$
\end{enumerate}

\end{document}

