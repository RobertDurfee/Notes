\documentclass{article}
\usepackage{tikz}
\usepackage{float}
\usepackage{enumerate}
\usepackage{amsmath}
\usepackage{bm}
\usepackage{indentfirst}
\usepackage{siunitx}
\usepackage[utf8]{inputenc}
\usepackage{graphicx}
\graphicspath{ {Images/} }
\usepackage{float}
\usepackage{mhchem}
\usepackage{chemfig}
\allowdisplaybreaks

\title{ 14.01 Lecture 19 }
\author{ Robert Durfee }
\date{ November 13, 2017 }

\begin{document}

\maketitle

\section{ Increasing Savings }

Remember, savings used to be a bad. But in fact, in a macro sense, savings is a
indicator of growth. If people save, capital supply goes up. This leads to a
lower interest rate. If the interest rate goes down, for any possible
investment, the NPV goes up. Investment opportunities become better and then
there is more investment.

If savings is a bad, but good for the economy, how do we get people to save?
\textbf{Tax subsities to retirement savings} is a method to achieve this. If you
put money in the bank, your interest is taxed. This is because it is earned
income. This lowers the returns of savings to individuals. Therefore there is
less savings. If the substitution effects dominate, the taxation will lead to
less savings. \textbf{If the income effects dominate, the taxation will lead to
more savings.}

We can combat this substitution-dominated effect by introducing pensions where
the money is not taxed. This is the same as a 401(k), but you can control this.
There are also individual retirement accounts. People think these things are
great because you do not get taxed, but this is wrong. You only get taxed when
you take the money out. The tax is only \textbf{deferred}. Why does this work?
This is becuase the interest compounds. You would much rather pay taxes later.

\subsection{Couple Earnings Example}

\begin{figure}[H]
    \centering
    \includegraphics[scale=0.70]{"Figure 1"}
    \caption{Tax Subsidy to Savings}
\end{figure}

Looking at \textit{Figure 1}. This couple earns  \$100. They can put their money
in the bank or in a retirement account. If the interest rate is 10\%, the
regular acount will have less withdrawn at the end than the IRA. It is better to
pay taxes later. Over 30 years, the IRA will have over twice as much as the
regular account withrawn.

\subsection{Money Market Fund}

This pays a very low interest rate, but is totally safe. This is guaranteed.
There is no way to lose money.

\subsection{Corporate Bonds}

You loan money to companies. This is done through a fund, not directly. This is
about 4-5\%. The corporate bonds cannot guarantee your return. These are riskier
as you might not get your money back.

\subsection{Stock}

You can buy a share of a company. The return is about 7\%. This is quite
riskier, the stock can go up or down.

\subsection{Risk-Return Tradeoff}

These three elements highlight the \textbf{risk-return tradeoff}. You diversify
your account. You keep the money you need right now in a low risk fund. But, any
money that you don't need right now, you should put in a higher risk fund.
\textbf{Diversification} is important because if you take a risk, they should be
completely uncorrelated.

The most stupid thing you can do is buy stock in the company you work. If the
company goes bankrupt, you will lose your job and your investments. This little
sidetrack is important in the field of \textbf{finance}. 

\section{International Trade}

\subsection{Roses Example}

Roses on Valentines day. This day falls in winter, and there aren't a lot of
roses then. Now we fly roses in from Columbia. There, roses grow year round. Is
this better? Roses cost a lot less. However, a bunch of rose growers lost their
jobs. This is the fundamental problem of international trade. 

\bigbreak

International trade is described by exports and imports. \textbf{Exports} are
the value of the goods a country sells to the rest of the world.
\textbf{Imports} are the value of the goods a countr buys from the rest of the
world. The US has a trade deficit of about \%800 billion. This causes people to
be upset. This is wrong. This number is meaningless.

\subsection{Pokemon Example}

You collect Pokemon. You have two of one and none of the other. Your friend has
two of the other and none of one. You trade and now you both have one of each.
What happened to your deficit of one? You now have a deficit. This is not the
point, the point is are trades happening that make both parties better off?

\subsection{US-Bangladash Dollars-Sweaters Example}

Take the US and a poor nation. They have a surplus of cheap sweaters and we want
sweaters. Now we are happy to send our dollars, and they are happy to receive
our dollars. They don't need sweaters.

But this also involves a sweater producer in the US. That is what makes
international trade difficult. How do we resolve this tension?

\section{Production Possibility Frontier}

\begin{figure}[H]
    \centering
    \includegraphics[scale=0.70]{"Figure 2a"}
    \caption{Linear PPF}
\end{figure}

A \textbf{PPF} shows maximum combination of outputs that can be produced from a
given set of inputs. Looking at \textit{Figure 2}. You, as a firm, can produce
two outputs: problem sets and exams. The more time you spend on one results in
less time spent on the other. 

\begin{figure}[H]
    \centering
    \includegraphics[scale=0.70]{"Figure 2b"}
    \caption{PPF with Economies of Scope}
\end{figure}

This is not correct because doing work on exams helps with problem sets and vice
versa. This is \textbf{economies of scope}, shown in \textit{Figure 3} as a
company does more different things, their overall production goes up. They are
better off at not specializing. If you do both problem sets and exams, you are
better off. 

\begin{figure}[H]
    \centering
    \includegraphics[scale=0.70]{"Figure 2c"}
    \caption{PPF with Diseconomies of Scope}
\end{figure}

\textbf{Diseconomies of scope} are goods where
doing one will make it harder to do another, for example squash and tennis. This
is shown in \textit{Figure 4}.

\subsection{US-Columbia Computers-Roses Example}

Look at countries US and Columbia and use two goods computers and roses. Assume
it is easier to produce computers in the US and roses in Columbia. Opportunity
cost of a rose in terms of computers is higer in the US. You have to give of
more computers to get roses in the US. This opportunity cost is lower in
Columbia (and vice versa). \textbf{Comparative advantage} is when you have a
lower opportunity cost of producing that good than someone else. This is
different than \textbf{absolute advantage}, which means you are simply better at
something. The US has a comparative advantage in computers and Columbia has a
comparative advantage in roses.

\begin{figure}[H]
    \centering
    \includegraphics[scale=1]{"Figure 3a"}
    \caption{US PPF}
\end{figure}

\begin{figure}[H]
    \centering
    \includegraphics[scale=1]{"Figure 3b"}
    \caption{Columbia PPF}
\end{figure}

Looking at \textit{Figures 5 and 6}. The US and Columbian PPFs are shown.
Imagine that the tastes are such that US consumers want $C_{US}$ and that
Columbian consumers want $C_{CO}$. \textbf{Autarky} means that there is no
trading. Under this assumption, 1,500 roses are produced. This production is
shown in \textit{Figure 7}.

\begin{figure}[H]
    \centering
    \includegraphics[scale=0.95]{"Figure 3c"}
    \caption{Production and Consumption in Autarky}
\end{figure}

\begin{figure}[H]
    \centering
    \includegraphics[scale=0.75]{"Figure 4"}
    \caption{Joint Production PPF}
\end{figure}

Look at \textit{Figure 8}. In the world as a whole, if all we produced was
computers, there would be 3,000 and the same if we only produced roses. If the
US only made computers and Columbia only made roses. This is shown in
\textit{Figures 9 and 10}. Then the world would have 2,000 roses and computers. This
increased the world's supply of both roses and computers. This is as a result of
specialization. Now people in the US and Columbia can consume more. We made this
majic happen by taking advantage of \textbf{specialization}. Trade frees up
production factors to go where they are most productive. This is shown in
\textit{Figure 11}.

\begin{figure}[H]
    \centering
    \includegraphics[scale=1]{"Figure 5a"}
    \caption{US Consumption}
\end{figure}

\begin{figure}[H]
    \centering
    \includegraphics[scale=1]{"Figure 5b"}
    \caption{Columbia Consumption}
\end{figure}

\begin{figure}[H]
    \centering
    \includegraphics[scale=0.95]{"Figure 5c"}
    \caption{Production and Consumption with Trade}
\end{figure}

\end{document}

