\documentclass{article}
\usepackage{tikz}
\usepackage{float}
\usepackage{enumerate}
\usepackage{amsmath}
\usepackage{bm}
\usepackage{indentfirst}
\usepackage{siunitx}
\usepackage[utf8]{inputenc}
\usepackage{graphicx}
\graphicspath{ {Images/} }
\usepackage{float}
\usepackage{mhchem}
\usepackage{chemfig}
\allowdisplaybreaks

\title{ 14.01 Lecture 23 }
\author{ Robert Durfee }
\date{ November 29, 2017 }

\begin{document}

\maketitle

\section{ Taxation in US }

\begin{figure}[H]
    \centering
    \includegraphics[scale=0.30]{"Revenue by Type of Tax"}
    \caption{Revenue by Type of Tax}
\end{figure}

What does the US tax and how does it do it? \textit{Figure 1} shows the division
of different types of both state and federal level. Income taxes are shown in
\textit{Figure 2}. \textbf{Marginal} and \textbf{average} taxes are different.
For marginal taxes, you pay the 10\% on the first income you earn. You pay the
15\% on the next dollar you make out of that tax bracket. Thus, the marginal tax
reaches average taxes asymptotically. For example, Bill Gates' average and
marginal taxes are about the same. 

We use the marginal tax in the efficiency side of things and average tax in the
equity side of things. We use a \textbf{progressive} tax system which causes
average income tax to increase as you earn more.

There are other taxes, like social security, which are flat charged to both your
employer and you. There are corporate taxes, too difficult to discuss today.
Consumption tax is based on sales and on certain goods, \textbf{excise} taxes.
And lastly, you pay property tax which is based on the value of your home. 

\begin{figure}[H]
    \centering
    \includegraphics[scale=0.35]{"Marginal Tax Rates"}
    \caption{Marginal Tax Rates}
\end{figure}

\subsection{ Tax Incidence }

\begin{figure}[H]
    \centering
    \includegraphics[scale=0.50]{"Burden of Producer Gas Tax"}
    \caption{Burden of Producer Gas Tax}
\end{figure}

\textbf{Tax incidence} means that people who pays the tax may not be the group
that supports the economic burden of the tax. This is because market respond.
Looking at \textit{Figure 3} and the market for gas. Suppose the suppliers have
to pay a \$0.50 tax. You may assume this harms the producers, but instead, with
the upward shift of the supply curve, there is an increase in price and a
decrease in supply. Overall, there is a \$0.30 burden to consumers and \$0.20
burden to producers. The tax burden is divided between the consumer and
producer. This is because, with the tax, prodcers have a \$0.30 higher price
than before. Consumers just get the \$0.30 raise in the price. 

Even though producers send the check to the government, the consumers are the
ones that feel the larger economic burden from the tax. It doesn't matter if you
put this tax on producers or consumers. 

\begin{figure}[H]
    \centering
    \includegraphics[scale=0.35]{"Burden of Consumer Gas Tax"}
    \caption{Burden of Consumer Gas Tax}
\end{figure}

Looking at \textit{Figure 4}, where everyone who buys a gallon of gas has to pay
\$0.50. This is a tax on the consumers. Since gas is essentially more expensive,
you are going to demand less, shifting the demand curve downward. The price will
decrease and the quantity will decrease. Consumers send \$0.50 to the
government, but they save \$0.20. Producers received \$0.20 less sales. 

\subsection{ Elasticity }

The more inelastic, the more they have the burden of taxes, and the more
elastic, the less they have the burden of taxes. Let's say that the demand for
gas is completely inelastic. If the supply curve shifts up from a producer tax,
the consumers have all the burden. Now, if demand is completely elastic, where
you only drive if the gas is a certain price. If the supply is shifted upward,
the price still stays the same. Thus, producers have all of the tax. 

\begin{figure}[H]
    \centering
    \includegraphics[scale=0.50]{"Inelastic Demand Producer Tax Burden"}
    \caption{Inelastic Demand Producer Tax Burden}
\end{figure}

\begin{figure}[H]
    \centering
    \includegraphics[scale=0.50]{"Elastic Demand Producer Tax Burden"}
    \caption{Elastic Demand Producer Tax Burden}
\end{figure}

In the first case, the producer has all the power. They have more negotiating
power. Therefore, they can avoid the tax. In the second case, the consumer has
all the power. They can avoid the tax because they have more negotiating power. 

Consider the effect of tax on hotel. There is a city hotel tax. Now there is
Airbnb. Airbnb can avoid the tax. As a result, consumer's demand has be come
more elastic. Now the tax has more of a burden on the hotels than the consumers.

\subsection{ Tax Base }

What should we tax? In Europe, they tax everything we do, but the consumption
tax is much larger. They have the \textbf{value added tax}, which is a higher
tax on consumption.

$$ Y = C + S $$

When we shift tax from income to consumption, this reduces the taxation on
savings. An income tax taxes both. If we went from income to consumption tax,
savings will go from taxed to untaxed. Assuming substitution effects dominate,
taxes on savings reduces savings. This is bad for the economy because the cost
of borrowing is higher. More investment in Europe.

We don't only tax comsumption because it is regressive because those who have
less money, spend more. Rich people save and poor people do not save. Tax on
consumption is more efficient, because it leads to more growth, but it is
unfair.

You can try to limit this effect by not applying the tax to necessities such as
food. How fair of a tax depends on the market. 

\section{ Transfers in US }

You are worse off if people aren't working and just collecting welfare. We are
best off when people:

$$ \rm{MVL} = \rm{MP_{L}} $$

Welfare breaks this equation because now you have marginal value of leisure and
the welfare equal the marginal product of labor. 

$$ \rm{MVL} + \rm{Welfare} = \rm{MP_{L}} $$

This doesn't mean welfare is bad, just that there is a tradeoff as there is a
deadweight loss. 

\subsection{ Categorical Welfare }

This is essential money you get for being poor and having something else going
on. \textbf{TANF} is a welfare given to single families who are poor.
\textbf{SSI} is when you are unable to work because you are disabled. We do this
because when people are poor, they can be skilled or unskilled. Then we shift
money from skilled to unskilled. But, we don't want to give things to people who
are lazy. This can get rid of deadweight loss because these people cannot work
anyway. But this only works if the money doesn't go to the lazy.

\subsection{ Moral Hazard }

We are assuming that disabled and single people are that way beyond their
control. But what if there targeting causes the problem itself. \textbf{Moral
hazard} is when we have more people becoming single moms because we created the
categorical. With single moms, this is completely not the case. No evidence.
However, giving money to disabled people causes more people to be disabled.
The advantage is you can target the group, but then you risk the result of
making that group larger.

\subsection{ In-Kind Benefits }

Why do these two cash welfare programs at up to \$80 billion, but
\textbf{in-kind} benefits make up \$800 billion. If you give people money,
everyone will pretend to get the money. But if you give them goods, people are
less likely to pretend. Only those who are truly needed will suffer through
shitty public housing or food stamps. 

\subsection{ Conditional Cash Transfer }

The \textbf{earned income tax credit} is how the US implements this. Looking at
\textit{Figure 6}, the x-acis is what you earn in wages and salary, and the
y-axis is the check that you get from the United States. For every dollar you
earn, up to about \$14,000, you will receive a \$0.40 check from the government.
This will induce them to work harder, not less hard. But there is a trade off:
you don't want to giver this money to everyone. Once you reach \$18,000, you
will start to phase out the benefit. Now there is a tax. Now the person would
want to work less. 

The evidence is positive. This induces millions to get into the labor force and
has little effect on how much people are willing to work. This is complicated,
so people just don't think about the penalization. The left proposes that the
EITC should be increased significantly.

\begin{figure}[H]
    \centering
    \includegraphics[scale=0.30]{"Earned Income Tax Credit"}
    \caption{Earned Income Tax Credit}
\end{figure}

\end{document}

