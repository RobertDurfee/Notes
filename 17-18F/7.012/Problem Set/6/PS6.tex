\documentclass{article}
\usepackage{tikz}
\usepackage{float}
\usepackage{enumerate}
\usepackage{amsmath}
\usepackage{bm}
\usepackage{indentfirst}
\usepackage{siunitx}
\usepackage[utf8]{inputenc}
\usepackage{graphicx}
\graphicspath{ {Images/} }
\usepackage{float}
\usepackage{mhchem}
\usepackage{chemfig}
\allowdisplaybreaks

\title{ 7.012 Problem Set 6 }
\author{ Robert Durfee }
\date{ November 15, 2017 }

\begin{document}

\maketitle

\section*{ Question 1 }

\begin{enumerate}[A.]
    \item 
        \begin{enumerate}[i.]
            \item If the SRP is inhibited, then the ribosome cannot stop
                translation and bring the mRNA to the ER to be synthesized. As a
                result, the protein will be synthesized in the \textbf{cytosol}.
            \item If the signal sequence does not exist, then the SRP will not
                stop translation and bring the mRNA to the ER to be synthesized.
                As a reults, the protein will be synthesized in the
                \textbf{cytosol}.
            \item Since the protein will go into the secretory pathway, it will
                be blocked by the inhibitor of ER-to-golgi transport. As a
                result, the protein will not be secreted and will be stuck in
                the \textbf{ER}.
            \item The protein was not originally meant to localize to the
                nucleus therefore the protein will continue along the secretory
                pathway and be \textbf{secreted outside the cell}.
        \end{enumerate}
    \item 
        \begin{enumerate}[i.]
            \item Parts \textbf{a, e, i, and m} are inside the lumen of the ER
                and eventually the outside of the cell. Therefore, these parts
                of the protein can bind to a ligand.
            \item Parts \textbf{c, g, k, and o} are outside the lumen of the ER
                and eventually the inside of the cell. Therefore, these parts of
                the protein can bind to the G protein.
            \item Parts \textbf{b, d, f, h, j, l, and n} are within the membrane
                of the ER and eventually the plasma membrane. Therefore, these
                parts of the protein will have hydrophobic amino acid residues.
        \end{enumerate}
\end{enumerate}

\section*{Question 2}

\begin{enumerate}[A.]
    \item
        \begin{enumerate}[i.]
            \item \textbf{Step 2.}
            \item \textbf{Step 2.}
            \item \textbf{Step 1.}
            \item \textbf{Step 4.}
        \end{enumerate}
    \item \textbf{Condition i}: Since the G$\rm{\alpha}$ proteins are stuck in
        the GTP bound conformation, and this is an active conformation, the cell
        will be constantly signalling.
    \item \textbf{Condition ii}: the G$\rm{\alpha}$ proteins will not be able to
        switch from inactive to active because it cannot trade a GDP for a GTP.
        Therefore, the signal will be stopped at step 2. 
        
        \textbf{Condition iii}: a conformational change doesn't occur when the
        ligand is bound thus the receptor is not activited. Therefore, the
        signal will be stopped at step 1. 
        
        \textbf{Condition iv}: ATP cannot be converted to cAMP which is
        necessary to activate PKA. Therefore, the signal will be stopped at step
        4.
    \item The G$\rm{\alpha}$ proteins will be stuck in an active state and since
        this occurs in a step after the mutation, the signal will result in PKA
        being \textbf{activated}.
    \item Although the G$\rm{\alpha}$ proteins will be stuck in an active state,
        the mutation occurs in the ATP to cAMP conversion which is downstream.
        As a result, PKA will \textbf{not be activated}.
    \item PKA will \textbf{not be activated} because both the mutation and the
        toxin prevent activation of the signal. However, the toxin will have no
        effect because it affects a step downstream from the mutation.
    \item PKA will \textbf{not be activated} because both the mutation and the
        toxin prevent activation of the signal. However, the mutation will have
        no effect because it occurs downstream of the toxin-affected step.
    \item A possible advantage to the adrenaline signaling pathway over one that
        involves transcription factors is response time. The adrenaline
        signaling pathway can respond very rapidly whereas a pathway involving
        transcription factors will take longer to respond.
\end{enumerate}

\section*{Question 3}

\begin{enumerate}[A.]
    \item Embryonic stem cells are pluripotent, meaning that they will
        differentiate into every type of cell that will make up an adult mouse.
        Therefore, a gene alteration in this type of cell will trickle down into
        every other type of cell, knocking out the gene in all cells. (After a
        homozygous mouse is breeded.)
    \item Scientists don't use pluripotent stem cells but rather
        multi-/oligopotent stem cells that differentiate within a certain
        tissue. This is because the cells the scientists wish to fix are limited
        to a certain tissue (bone marrow, blood).
    \item
        \begin{enumerate}[a.]
            \item The first cell division (i) is unique to a stem cell because it
                has to replenish itself as well as create a differentiated cell.
                As a result, one cell is a duplicate and the other is
                differentiated.
            \item The second cell division (ii) occurs in a terminally
                differentiated cell because they only have to ability to
                replicate themselves and cannot turn into other cell types.
        \end{enumerate}
    \item Transit amplifying cells protect the genome of stem cells by limiting
        the number of successive division that a stem cell passes during its
        lifetime. In the instestinal tract, stem cells are kept at the bottom of
        wells or crypts that contain mucus that prevents contact with harmful
        materials.
\end{enumerate}

\section*{Question 4}

\begin{enumerate}[A.]
    \item Mutation 1 will prevent the cell from signalling when exposed to EGF.
        However, mutation 4 occurs downstream from mutation 1 and will cause the
        cell to be stuck constantly signalling. As a result, regardless of the
        presence of EGF, the cell \textbf{will undergo proliferation}. 
    \item Mutation 2 will result in a constantly signalling cell, however,
        mutation 3, occurring downstream from mutation 2, will prevent half of
        the proteins from continuing the signal, but the other half can. As a
        result, the cell \textbf{will undergo proliferation}.
    \item Mutation 3 will prevent the cell from signalling when exposed to EGF.
        However, mutation 4 occurs downstream which results in half the proteins
        constantly signalling. As a result, the cell \textbf{will undergo
        proliferation}.
    \item Both mutation 4 and 5 cause the cell to be constantly signalling. As a
        result, the cell \textbf{will undergo proliferation}.
    \item Mutation 5 will result in a constantly signalling cell, however,
        mutation 6 occurs downstream and prevents the signal from continuing. As
        a result, the cell \textbf{will not undergo proliferation}.
    \item Mutation 1 will prevent the cell from signalling, however, mutation 5
        occurs downstream resulting in half the proteins constantly signalling.
        As a result, the cell \textbf{will undergo proliferation}.
\end{enumerate}

\end{document}

