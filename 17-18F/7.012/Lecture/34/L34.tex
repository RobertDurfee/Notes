\documentclass{article}
\usepackage{tikz}
\usepackage{float}
\usepackage{enumerate}
\usepackage{amsmath}
\usepackage{bm}
\usepackage{indentfirst}
\usepackage{siunitx}
\usepackage[utf8]{inputenc}
\usepackage{graphicx}
\graphicspath{ {Images/} }
\usepackage{float}
\usepackage{mhchem}
\usepackage{chemfig}
\allowdisplaybreaks

\title{ 7.012 Lecture 34 }
\author{ Robert Durfee }
\date{ December 11, 2017 }

\begin{document}

\maketitle

\section{ Molecular Evolution }

In our geneome, there is junk DNA and coding sequences. There can be mutations
that occur in the coding strands that form changes in phenotype or there can be
no changes in phenotype. There are some proteins that are highly conservative
taht do not change very much over many years. This can be the DNA for histones
to compact DNA or the Pax6 gene which specifies the formation of the eye in
mammals. 

Over a period of time, gene sequences randomly mutate and drift apart, or
\textbf{diverge}, over time. Using the divergence between certain sequences, you
can determine the organizational tree for different species and how they are
connected. 

How can this be done within a species? We must limit our search to sequences of
our geneome that cannot recombine. The mitochondrial DNA does not recombine.
When a sperm joins with an egg, only the egg's mitochondria exist. As a result,
all of your mitochondria com from your mother. In addition, the Y chromosome
also rarely recombines. They only come from the fathers. 

In populations of constant size, then, on average, after about 4 generations,
only 25\% of mtDNA and Y DNA will survive. After 8 generations, only 5.6\%.
These types of populations still exist in some places in the world. They were
more popular in the earlier years. 

We can determine the age of a species by looking at the heterogeneity. If there
are only a small number of individuals, most of the previously accumulated
genetic heterogeneity will be lost. When looking at two chimps, there is
significantly more differences between them than between two humans. This
suggests that humans' bottleneck was relatively recent. 

Looking at the genetic distances of human populations across the globe shows
that humans in Africa are much more diverse than humans in other parts of the
world. This suggests that human evolution occured in Africa until relatively
recently. 

During the trek out of Africa, when the Neanderthal were present, there was
interbreeding. As a result, many Europeans have Neanderthal DNA. This also shows
that there can be interbreeding without loss of fertility, as is the case with
horse and donkey offspring. 

Using the genomes of feas, we can determine how long people have been wearing
clothes. There are two types of fleas: head lice and body lice (under clothes).
Looking at the common ancestors between the two species, we can determine that
we have been wearing clothes 70,000 some years.

\end{document}

