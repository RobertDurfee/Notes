\documentclass{article} 
\usepackage{indentfirst} 
\usepackage[utf8]{inputenc}

\title{7.012 Lecture 22 - Cell Signaling} 
\author{Robert Durfee} 
\date{November 6, 2017}

\begin{document}

\maketitle

\section{Intercommunication}

We talk about how to make complex organisms. The architecture of complex tissues
requires that the constituent cells continuously talk with one another through
\textbf{intercommunication}. Without this ongoing communication, there is chaos. 

Cell-surface receptors transmit signals through the membrane. Cells express a
diverse array of transmembrane proteins in their plasma membranes.

\section{Receptors}

These transmembrane proteins transduce signals from outside the cell into the
cytoplasm of the cell. The cell then responds with appropriate signaling. There
are downstream responses, but we won't discuss these in class today.

In the relaxed state of receptors, the receptor binds to GDP. This is only there
to provide as a marker. This labels the receptor as inactive.

A \textbf{ligand} is something whose presence causes a receptor to activate.
When a ligand binds to the ectodomain, GTP is kicked off and floats into the
cytoplasm. Now the GDP can bind. This is not used as energy, it only allows the
receptor to become active.

After some time, the GTP will be hydrolyzed back to GDP. This resets the
receptor to its relaxed state. This is, again, not to store energy. This is a
way to create a binary switch.

Receptors are divided into \textbf{ectodomain}, \textbf{transmembrane domain},
and \textbf{cytoplasmic domains}. There are \textbf{kinase} domains inside the
cytoplasmic domain which is responsible for signaling when the ectodomain
receives a signal.

In general, kinase takes the phosphate off of ATP and transfers it to amino acid
side chains of various substrate proteins.

Transmembrane receptors can move laterally in the plane of the plasma membrane.
Ligand receptors will bind causing \textbf{dimerization} where two receptors
come together within the plane of the membrane. This then brings the cytoplasmic
domains together. This causes \textbf{transphosphorylation}. 

Receptor transphosphorylation activates different docking sites. As a result,
multi-protein cytoplasmic signals can be formed.

Negative feedback is necessary to shut down positive signals after a certain
amount of time.

A cell cannot decided whether or not to start proliferating. This is decided by
its neighbors. This is called an \textbf{autocrine loop}.

\subsection{Cell Proliferation Example}

We put some cells at the bottom of a petri dish with a medium. These cells will
do nothing. If you add plasma, still nothing happens. When you add serum, they
multiply rapidly. Somehow the serum contains materials that are lacking in the
plasma. Nutrients by themselves are not enough to persuade a cell to
proliferate.

Plasma exists before clotting, but after clotting, serums exists with the blood
clot. When you have a wound, the surviving cells are stimulated to grow by a
growth factor \textbf{PDGF}. This will cause a wound to proliferate immediately
after the wound is incurred. This is done through platelet
\textbf{degranulation}.

Growth factor receptors are responsive to the PDGF that is released in the
degranulation. They respond by emitting signals into the cytoplasm that
stimulate the cell to proliferate. 

Tyrosine kinases are the kinases related to growth factor receptors. This kinase
can have many different substrates that result in different intracelluar
signals.

There are many different ectodomains of receptors designed to bind to a variety
of different growth factor ligands. These all have different downstream
substrates.

\subsection{Cancer Cell Example}

The GTP-ase is inhibited. Therefore the proliferation signal is stuck in the
active state for an indefinite period of time.

Also, mutant receptors can allow ligand-independent firing. The ectodomain can
either be mutated or even borken off.

\subsection{Extracellular Matrix Example}

A cell will bind to its extracellular matrix on receptors called
\textbf{integrins}. This assures the cell that proliferation is appropriate. If
the cell is disconnected, it will not proliferate.

\end{document}
