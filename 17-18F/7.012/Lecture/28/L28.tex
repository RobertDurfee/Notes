\documentclass{article}
\usepackage{tikz}
\usepackage{float}
\usepackage{enumerate}
\usepackage{amsmath}
\usepackage{bm}
\usepackage{indentfirst}
\usepackage{siunitx}
\usepackage[utf8]{inputenc}
\usepackage{graphicx}
\graphicspath{ {Images/} }
\usepackage{float}
\usepackage{mhchem}
\usepackage{chemfig}
\allowdisplaybreaks

\title{ 7.012 Lecture 28 }
\author{ Robert Durfee }
\date{ November 27, 2017 }

\begin{document}

\maketitle

\section{ Introduction }

This is a huge topic that has many different areas. There isn't enough time to
talk about it in that much depth. However, we will cover some basics. We will
look at how a neuron is able to function.

\section{ Neurons and Connection }

\textbf{Neurons} are the name we give to your nerve cells. There are about
$10^{12}$ neurons with about $10^{3}$ connections each. Therefore, there are
about $10^{15}$ different pathways all throughout the body.

A \textbf{receptor} is a cell that can detect signals. Some examples of
receptors are \textbf{light receptors} in your eyes and \textbf{sound receptors}
in your ears. There are many more, of course, which contribute to he major
senses. 

The \textbf{cell body} has many \textbf{dendrites} which seek out many different
connections. Within the cell body, there is the \textbf{nucleus}. A
\textbf{synapse} will form and send an electrical signal along the \textbf{axon}
and end at the \textbf{nerve terminals} and release \textbf{neurotransmitters}
to another cell or a muscle.

\section{ Transmitting a Signal }

We will be studying the axon as it if were a long cylinder. Originally,
scientists used a squids huge eye axon to study how it works.

In the axon, there is a voltage potential difference between the inside of the
axon and the outside. There is more negativity in the inside than the outside.
This potential is about $50\ \si{ mV }$. This occurs over a membrane of $3\ \si{
nm}$. Thus, the electric field has a magnitude of $200,000\ \si{ V\ cm^{-1} }$.
This is a huge value.

If you change the electric field in a small way, you can get huge results. When
an electric signal is sent down the axon, the inside of the axon changes from
being negative to positive. If we inject a small amount of current to raise the
inside potential to $-50\ \si{ mV }$, there will be a huge jump into positive
potential. This will cause an \textbf{action potential} which cascades down the
axon. 

The part where the action potential goes from negative to positive is
\textbf{depolarization} and the part where it goes from positive to negative is
\textbf{polarization}. This all happens within one millisecond.

One the outisde of the cell, there is more sodium and calcium, and the inside of
the cell, there is more potassium. These ions have very different concentrations
between inside and outside.

\section{ Membrane Proteins }

These proteins take sodium out and put potassium in. This is done simply by
swapping ions. It is important to note that no change occurs, only concentration
differences. This goes against the concetration gradient which violates entropy.
Thus, the protein must pay in ATP. This is a \textbf{sodium potassium ATP-driven
pump}.

There is also a calcium out pump. There are no ions exchanged, like before. This
pump must also pay in ATP. But this changes the charge. (Ignore this for now.)

Now, we introduce a channel that allows potassium to leave freely. After
spending all this energy to get potassium, why would we want it to just leave
freely? Some of the potassium will leave, but not all of it. Since it gets to
leave, the charge will get more positive on the outside. At some point, the
concentration gradient says "go out", but the electrical gradient says "go in".
At $-70\ \si{ mV }$, these two gradients will balance out each other. This is
accomplished through a tiny exchange in potassium.

Now, we introduce a channel that prevents sodium from flowing freely at $-70\
\si{ mV }$. This channel will open when the internal potential goes to $-50\
\si{ mV }$. This is called a \textbf{voltage-gated sodium channel}. When this
gate opens, since there is a huge concentration outside, the ions want to come
in. In addition, it is more negative inside the cell. So the ions still want to
come in. Thus the sodium ions want to come in for two reasons. The electrical
and conectration gradients then balance at $-50\ \si{ mV }$. Then, the potential
will come down. But why will it come down?

Let's introduce a \textbf{votage-gated potassium channel} which will open when
the volatge inside the cell reaches $+30\ \si{ mV }$. Now the potassium flows
in. But both are now flowing. This won't work. As a result, we must close and
inactivate the voltage-gated sodium channel. This is done through a built in
timer which shuts down the gate for a short period of time.

\section{ Propogation of Action Potential }

At the very end of the axon, assume there is an action potential created. When
the small bit becomes positive, and the negatives are pulled closer. Then, the
next little bit will increase its potential and become more positive. Once this
is reduced to $-50\ \si{ mV }$, the action potential shoots up. It only goes one
way because the sodium channel has a timer that inactivates the gate.

\section{ Speeding Up }

Let's think about bringing in an electrical insulator. Wrap the the insulator
around the axon with little opening inbetween. This would result in a jumping
action potential between the gaps. But what is this insulator? The insulator is
a cell that has wrapped itself around the axon and squeezed out all of its
cytoplasm. This is the \textbf{myclin sheeth}. And the process that causes the
transfer is \textbf{saltatory conduction}.

\end{document}

