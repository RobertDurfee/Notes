\documentclass{article}
\usepackage{tikz}
\usepackage{float}
\usepackage{enumerate}
\usepackage{amsmath}
\usepackage{bm}
\usepackage{indentfirst}
\usepackage{siunitx}
\usepackage[utf8]{inputenc}
\usepackage{graphicx}
\graphicspath{ {Images/} }
\usepackage{float}
\usepackage{mhchem}
\usepackage{chemfig}
\allowdisplaybreaks

\title{ 7.012 Lecture 25 }
\author{ Robert Durfee }
\date{ November 15, 2017 }

\begin{document}

\maketitle

\section{ What is Cancer? }

Normal tissue is very well ordered, but a cancerous, tumor cell will be chaotic.
Normal cells will ask their neighbors if it is appropriate to proliferate.
Cancerous cells do not do this, they just proliferate continuously. 

Tumors are \textbf{monoclonal}, meaning that they are all descendents of a single
ancestor. \textbf{Polyclonal} tumors are very rare. 

\subsection{Timeline}

Cancer also takes a long time to develope. As a result, it is a disease of old
age. There are several constructs within our cells which stalls or prevents
cancer. As we start living older, there will be more cancer. This means that
none of us are at more risk for cancer, rather, we are simply living older into
the time where cancer is highly prevalent.

Invasive tumors \textbf{metastasize} and more throughout the body and grow
elsewhere. This is what is responsible for 90\% of cancer mortality. 

\section{What Causes Cancer?}

\subsection{Carcinogens}

\textbf{Carcinogens} are the risk factors that are connected with and possibly
responsible for the development of cancer. A scientist painted rabbit ears will
coal tar. This was the first time a cancer was experimentally induced in a
subject. This identified coal tar as a carcinogen. The constituents of the tar
were then later purified and identified.

But this does not explain how these carcinogens are able to cause cancer. The
speculation was that the carcinogens are also mutangenetic. A scientist designed
a test that allowed people to test for the mutagenetic potency of carcinogens.
He did this by making a mutant whose growth could be controlled with a neccesary
supplement. Without it, the cell couldn't grow. Thus, the scientists can look
for the cells exposed to the carcinogen allowing the cell to grow without the
supplement. There proved to be a strong correlation between high mutagenetic
compounds and high carcinogenic compounds.

Since it was likely that the cancer cells are also mutant cells, they must carry
a mutant genome with them. As a result, specific tests were created to
determine if \textbf{tumorgenic} cells are also mutagenic. This was done by
extracting DNA from a tumor and injecting them into normal cells and observing
the results. As a final test, these cells are then injected into a host to prove
that the tumor-looking cells actually create tumors.

\subsection{Oncogenes}

The gene that is mutated that causes cancer is called an \textbf{oncogene}. By
recombining a normal gene and an oncogene, it is possible to identify the
specific mutation that causes the cancer. DNA sequencing wasn't done because
there could be other differences between the gene that don't cause cancer. Also,
DNA sequencing technology was not available back then.

This specific example causes a glycene to become a valine in the ras protein, a
G-protein. This glycene was responsible for shutting off the signal after it is
activated. As a result, the cell will be constantly signalling to proliferate.

\subsection{Communicable Cancer}

An experiment was done in chickens which found that there is a subcellular
particle (like a virus) that, when transferred to another host, can cause cancer
in that host. 

Typical viruses, after they infect a host cell, are \textbf{cytopathic}, meaning
that they rupture and kill the host cell. But a virus can infect a cell, and
spare the host. This is \textbf{temperate} infection where the virus will force
the cell to push viruses out of the plasma membrane. A third way a virus can
infect a cell is by \textbf{transforming} the cell and becoming a
\textbf{focus}. The cell is not killed, rather undergoes a strong shift in its
phenotype. The virus also injects information which causes the cell to become
cancerous. 

These cells also inherit an oncogene that is created by the virus. Thus cell
descendents are also cancerous.

\end{document}

