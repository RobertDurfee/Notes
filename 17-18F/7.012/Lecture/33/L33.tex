\documentclass{article}
\usepackage{tikz}
\usepackage{float}
\usepackage{enumerate}
\usepackage{amsmath}
\usepackage{bm}
\usepackage{indentfirst}
\usepackage{siunitx}
\usepackage[utf8]{inputenc}
\usepackage{graphicx}
\graphicspath{ {Images/} }
\usepackage{float}
\usepackage{mhchem}
\usepackage{chemfig}
\allowdisplaybreaks

\title{ 7.012 Lecture 33 }
\author{ Robert Durfee }
\date{ December 8, 2017 }

\begin{document}

\maketitle

\section{ Cancer }

Cancer is when cells grow and divide without end in a multicellular organism. It
is bad for an individual cell to keep growing in a multicellular organism. There
are many different regulation techniques to keep them in check. When this all
breaks down, you get a \textbf{neoplast}. This is essentially a tumor. Tumors
can be \textbf{benign} or \textbf{malignent}. Benign tumors simply mean that
they don't spread. Usually this means it is safe, but it could press against
something. This can be fixed by removing the tumor. Malignant, or
\textbf{metastatic}, cancer moves and invades surrounding tissue.  This causes
\textbf{metastasis} where the cancer is spread all around the body.

Cancer is very common. There are 12 million who have it in the US. Half of those
who get it will die. This accounts for one quarter of all deaths. The common
forms are lung, stomach, breast, prostate, colon, etc. Stomach cancer is more
common in Asia. 

Cancers arise from mutations in the genome. This can come from diet, exposure to
harmful elements, predisposition, etc. In you lifetime, you have $10^{16}$
different cell divisions. In a given gene, there is a $10^{-6}$ chance of
getting mutated. This sayes that each gene in your body is mutated $10^{10}$
times in your lifetime. These are usually harmless because the mutation knocks
out only one copy of the gene. 

You can increase the chance of mutation by exposing to radiation. This is one of
the best ways to mutate DNA. Another way is through smoking. Strong sunlight for
long periods of time. Charred meats have lots of mutagens. Most mutagens are
carcinogens as well. 

Looking at the graph of onset of cancer, it matches about $x^{5}$. This suggests
that is requires multiple events to cause cancer over lifetime. However,
different cancers requires different mutations. There are also predispositions. 

In order to detect cancer on an x-ray, you need over $10^{8}$ cancer cells. To
feel the cancer, you need over $10^{9}$ cells. At $10^{12}$ cancer cells, the
patient is dead. 

\section{ Regulation of Cell Growth }

Well behaved cells grow and then stop at some point. They do this by recieving
growth signals causing them to grow. When they stop getting these signals, they
should stop growing. If they don't they are cancerous. 

On the surface of cells, there are growth factor receptors. In the presence of a
growth factor, two half-growth factor receptors come together. Inside the cell,
there are tyrosine residues which become phosphorolyzed in the presence of the
other growth factor receptor. These phosphoralized tyrosines are recognized by
the Grb2 adapter. This then binds to the Sos molecule which "tickles" the Ras
protein and activates it.

\section{ Ras Protein }

Ras can bind to either GDP or GTP. When Ras binds to GTP, there is a
conformational change that causes the Ras to be active. GDP-bound Ras is
inactive. Ras is a \textbf{GTPase}. \textbf{GTPase Activating Proteins} can
stimulate Ras to inactive. \textbf{Guanine Exchange Factors} then stimulate Ras
to become active. 

\section{ Ras Signaling }

After Ras undergoes a conformational change after being activated by the Sos
GEF. This activates RAF which is a kinase. This activates NEK, which activates,
ERK, which activates a whole bunch on different things which are involving in
proliferation (such as transcription factors).

\section{ Cancer-Causing Mutations }

\subsection{ Receptor Always Active }

This can occur by the overproduction of growth factor by a cell (which then
signals itself to proliferate). This causes \textbf{autocrine stimulation}.

Another way is through \textbf{ligand-independent dimerization}. THis can occur
through amino acid changes or there is a deletion of the receptor part of the
growth factor receptor. This can also happen through the overproduction of the
growth factor receptor.

\subsection{ Ras Always Active}

This can occur by a mutation in Ras that simply causes sticking to GTP, thus
always being active. This can also occur through the deletion of GAP, which
prevents the exchange of GTP for GDP. 

\subsection{ Constituently Active Kinase }

These kinases could each have a mutation that results in an always-active
kinases. This can occur in RAF, NEK, and ERK.

\end{document}
