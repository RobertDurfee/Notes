\documentclass{article}
\usepackage{tikz}
\usepackage{float}
\usepackage{enumerate}
\usepackage{amsmath}
\usepackage{bm}
\usepackage{indentfirst}
\usepackage{siunitx}
\usepackage[utf8]{inputenc}
\usepackage{graphicx}
\graphicspath{ {Images/} }
\usepackage{float}
\usepackage{mhchem}
\usepackage{chemfig}
\allowdisplaybreaks

\title{ 7.012 Recitation 21 }
\author{ Robert Durfee }
\date{ December 7, 2017 }

\begin{document}

\maketitle

\section{ Cholesterol }

\textbf{Cholesterols} integrate into the cell membran and regulates the fluidity
and tension of the membrane. Cholesterol is a hydrophobic molecule so it will
likely integrate into the hydrophobic region of the lipid bilayer.

Cholesterol is a precursor for many other important molecules in the body. These
include: steroids, hormones, vitamins, and bile acids.

\subsection{Cholesterol Uptake }

Cholesterol can be produced by the body. This synthesis utilizes the enzyme
\textbf{HMG-CoA reductase}. This enzyme catalyzes the first regulated step of
cholesterol synthesis.

Cholesterol can also be consumed in foods like meat, eggs, etc.

\subsection{ Cholesterol Transport }

Since cholesterol is a hydrophobic molecule, it is not soluble in the blood--an
aqueous solution--so it must be packaged. Another reason cholesterol is packaged
is to allow the regulation of uptake of cholesterol within cells. 

The packaging mechanism uses \textbf{lipoproteins}. The different types of
these proteins form different packages: LDL (low density), HDL (high density),
etc. These proteins can help regulated the uptake of cholesterol. They are
sandwiched between different lipid molecules forming a monolayer with a
hydrophobic region inside for the cholesterol. The LDL lipoproteins are much
larger than the HDL lipoproteins. 

After transportation, the cholesterol molecules end up in cells. In their plasma
membranes, there is nearly a one-to-one ratios of cholesterols and lipids. In
the liver, there is an even higher ratio as the liver is responsible for
producing bile acid. The uptake into a cell is dependent on \textbf{HDL/LDL
receptors} on the cells.

\subsection{ Bile Acid }

\textbf{Bile acids} help to dissolve hydrophobic molecules in your diet such as
fats. After bile acids do their job, they are reabsorbed from the gut after
digestion. This means that your body loses very little cholesterol.

\section{ Health Effects }

High concentrations of circulating LDL increases the risk of cardiovascular
disease. More specifically, LDL causes \textbf{athrosclerosis}. This is the
constricting and clogging of arteries as a result of plaque build up.

There is a genetic disorder that results in extremely high concentrations of
LDL. This genetic disorder, \textbf{familial hypercholesterolemia}, is caused by
mutations in LDL receptor that prevent LDL uptake from blood: leading to high
concentrations within the blood stream.

\section{ Treatments }

Most obviously, you can limit cholesterol levels in the body by simply consuming
less cholesterol. However, the body still produces cholesterol by itself from
other molecules (acetic acid).

Another method to treat high cholesterol are drugs called \textbf{statins}.
These drugs inhibit HMG-Coa reductase which prevents cholesterol synthesis. As a
result, much lower concentrations will be in the body, but you will still
consume cholesterol in your diet. 

The last important method of treatment is to prevent the reuse of bile acids in
the gut by sequestering. This will force the liver to consume more of the
cholesterol produced by the body.

\end{document}

