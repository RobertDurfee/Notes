\documentclass{article}
\usepackage[utf8]{inputenc}
\usepackage{indentfirst}

\title{7.012 Recitation 12}
\author{Robert Durfee}
\date{October 31, 2017}

\begin{document}

\maketitle

\section{Gene Regulation}

Multicellular organisms use gene regulation so they can have multiple different
cell types. As a result, they can respond to different conditions. There are
several different types of gene regulation.

\subsection{Transcriptional Gene Regulation}

\textbf{Transcriptional gene regulation} chooses whether or not to make the
mRNA. As a result, it happens before the mRNA is transcribed.

A \textbf{cis-regulatory element} is a type of transcriptional gene regulation
that has to be present in the same DNA molecule in order to be functional. Some
examples include operators and promoters.

A \textbf{trans-regulatory element} is a type of transcriptional gene regulation
that can be present on another chromosome but can still impact the gene being
regulated. This is possible by the creation of a specific protein. Some examples
include transcriptional factors and repressors.

\subsection{Post-Transcriptional Gene Regulation}

\textbf{post-transcriptional gene regulation} occurs after the mRNA has been
transcribed. There are several forms of this including mRNA splicing and
stability. There is also a form where you choose whether or not to translate the
mRNA as well.

\section{Operons}

Eukaryotic mRNA is always \textbf{monocistronic}. This means that each mRNA
contains instructions for only one gene. Prokaryotic mRNA, on the other hand, is
\textbf{polycistronic}. This means that each mRNA can contain instructions for
multiple genes. 

An \textbf{operon} is a cluster of genes that are all regulated by the same
promoter. The benefits to this method is that the genes will all be transcribed
at the same time. This is efficient to allow proteins in the same pathway to the
co-regulated.

\section{Single Nucleotide Polymorphism}

A \textbf{single nucleotide polymorphism} (SNP) is a base pair where there is a
variation within a population. These often appear in areas that don't encode
genes. For example, on a chromosome, there is a point where 60\% of the
population has an A/T pair and 40\% has a G/C pair. This is important to note as
an SNP can be correlated to a disease gene. However, if a disease arose in
multiple individuals at different times, the disease can be linked with multiple
SNPs. However, in families, this is almost strictly followed. 

This is possible because of \textbf{haplotypes}, which are clusters of loci on
the DNA that are tightly linked so they don't recombine often.

\end{document}
