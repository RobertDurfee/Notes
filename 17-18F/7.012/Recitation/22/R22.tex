\documentclass{article}
\usepackage{tikz}
\usepackage{float}
\usepackage{enumerate}
\usepackage{amsmath}
\usepackage{bm}
\usepackage{indentfirst}
\usepackage{siunitx}
\usepackage[utf8]{inputenc}
\usepackage{graphicx}
\graphicspath{ {Images/} }
\usepackage{float}
\usepackage{mhchem}
\usepackage{chemfig}
\allowdisplaybreaks

\title{ 7.012 Recitation 22 }
\author{ Robert Durfee }
\date{ December 12, 2017 }

\begin{document}

\maketitle

\section{ Molecular Evolution }

\textbf{Molecular evolution} is the change in the heritable characteristics of a
biological population over time. Molecular evolution in DNA, RNA, and protein
sequences can all be different manifestations of molecular evolution.
\textbf{Mutations} can come in three forms. Mutations can be beneficial, which
increases evolutionary fitness. They can be \textbf{deleterious}, which
decreases evolutionary fitness. Lastly, they can be neutral, which has no effect
on the evolutionary fitness of a population.

\subsection{Genetic drift}

This can occur when an allele is randomly lost in a population over a period of
many generations. This can happen if those with one allele is more fit than
those with the other allele during a specific time.  Genetic drift can also
occur through the result of a \textbf{bottleneck}. This is when there are sudden
events that result in random changes in the allele composition of a population.
Lastly, genetic drift can occur through the \textbf{founder effect}. This is
when the population splits can the allele composition of the new population is
different from that of the original population.

\subsection{Natural Selection}

\textbf{Natural selection} is when beneficial alleles increase as a proportion of
the population because bearers of that allele are more fit. This is a form of a
beneficial mutation.

\subsection{Phylogenetic Tree}

In a basic level, a \textbf{phylogenetic tree} is a diagram that depicts the
relatedness of sequences between species. The farther apart two organisms are in
the tree (in distance), the less related their sequences. Shared mutations
are indicated as a deviation from the same \textbf{reference sequence}.

\section{Drug Design}

A \textbf{drug} is a chemical that is used to treat some sort of disease or
condition, in the context of medicine. In order for a drug to be effective, the
drug needs to be specific in what it targets. It has to work under practical
conditions. The drug also needs to persist long enough in the organism to have a
meaningful effect.

\section{Imatinib (Gleevec)}

This is a cancer treating drug that treats chronic myelogenous leukemia (CML).
This targets the Philadelphia chromosome (chr9 and chr22). This results from the
fusion BCR and ABL. ABL is a tyrosine kinase that can drive proliferation. This
binding results in constituently active ABL kinase.

Gleevec works by inhibiting ABL. Note that this drug does not only target
cancerous cells, but all cells. This is typically acceptable because cancerous
cells are more dependent upon ABL than other cells which aren't constantly
dividing like cancer.

\end{document}
