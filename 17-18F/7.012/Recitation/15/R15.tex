\documentclass{article}
\usepackage{tikz}
\usepackage{float}
\usepackage{enumerate}
\usepackage{amsmath}
\usepackage{bm}
\usepackage{indentfirst}
\usepackage{siunitx}
\usepackage[utf8]{inputenc}
\usepackage{graphicx}
\graphicspath{ {Images/} }
\usepackage{float}
\usepackage{mhchem}
\usepackage{chemfig}
\allowdisplaybreaks

\title{ 7.012 Recitation 15 }
\author{ Robert Durfee }
\date{ November 9, 2017 }

\begin{document}

\maketitle

\section{ Stem Cells }

A \textbf{stem cell} is a cell that can self renew and is to some degree
undifferentiated that can be differentiated later into other cell types.
\textbf{Differentiation} is the process of changing into a more specialized cell
type. An example of a cell that is not very differentitated is an 
\textbf{embryonic stem cell}. \textbf{Potency} is referring to the range of cell
types that a stem cell can be differentiated into.

\subsection{Types of Cells}

\textbf{Totipotent cell} can be differentiated into any cell type. An example
would be zygotes. \textbf{Pluripotent cells}, like embryonic stem cells, can be
differentitated into any cells that actually become part of the organism.
Basically everything but the zygotes. \textbf{Multipotent cells} are cells that
can differentiate into many cell types, but they are confined to a specific
type. These would be, for example, hematopoietic stem cells and intestinal stem
cells. \textbf{Bipotent and unipotent cells} can differentiate into two or one
other cell type. A \textbf{terminally differentiated cell} is a cell that is
complete differentiated and has reached the end of the line in terms of
differentitation.

\subsection{Types of Division}

A \textbf{mother cell} can divide into cells that are the same the same as the
mother, two that are the same as each other, but not the mother, or they can
divide into two cells, one is the same as the mother and the other is different,
or the mother can divide into two cells that are different from each other and
not the same as the mother.

The first two are considered \textbf{symetric divisions} because both daughter
cells are the same as each other. The other two are considered
\textbf{asymetric} as the two daughters are different.

A stem cell can do symetrical division if both are the same as the stem cell,
because they must self renew. They then can also divide asymetrically if one is
the same as the stem cell.

A terminally differentiated cell can only divide symetrically with the same
daughters as the mother because it cannot turn into different cell types.

\subsection{ Stem Cell Niches }

A \textbf{stem cell niche} is a cellular environment that maintains the stem
cell's identity. Frequently, stem cell exist in environments that tell them to be
stem cells, if they leave, they will no longer be stem cells. The neighboring
cells secret proteins that tell the other cell to be a stem cell.

\section{Cellular Reprogramming }

How is a cell's identity determined? Via \textbf{transcriptomes}. There are a
whole different sets of transcription factors that will result in different sets
of proteins in cell of the same genome. \textbf{Reprogramming} is the process of
changing the set of proteins/transcription factors present to read the genome.
If you take a \textbf{sematic cell} (which is a non-gamete cell) and a
un/fertilized egg, both have a set of transcription factors that tell it the type
of cell it should be. Using \textbf{somatic cell nuclear transfer (SCNT)}, you
can take the nucleus from the sematic cell and inject it into an enucleated
fertilized egg, you can then have an organism which has the same genome of the
sematic cell.

\textbf{Induced pluripotent stem cells (IPSC)} are induced embryonic stem cells
that come from sematic cells that are transformed into equivalent embryonic stem
cells.

You can take these IPS cells and you can apply different factors and produce
differentiated cell types. 

\section{Cell Signalling}

\textbf{Consitutively active} means that a pathway is active without the
upstream signal. For example, if the ligand isn't present, nothing happens. If
there is a mutant the caused the G-protein was consitutively active, the
G-protein would be always active. What if you added another mutation that
prevents CAMP from working. Then the pathway will not work. If you have
something consitutively active downstream of a mutation, the pathway will always
be active.

\end{document}

