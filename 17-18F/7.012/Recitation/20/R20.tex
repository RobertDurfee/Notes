\documentclass{article}
\usepackage{tikz}
\usepackage{float}
\usepackage{enumerate}
\usepackage{amsmath}
\usepackage{bm}
\usepackage{indentfirst}
\usepackage{siunitx}
\usepackage[utf8]{inputenc}
\usepackage{graphicx}
\graphicspath{ {Images/} }
\usepackage{float}
\usepackage{mhchem}
\usepackage{chemfig}
\allowdisplaybreaks

\title{ 7.012 Recitation 20 }
\author{ Robert Durfee }
\date{ December 5, 2017 }

\begin{document}

\maketitle

\section{ Immunology }

The \textbf{immune system} is an organism's response to pathogens and altered
self. This system must be able to distinguish between self and non-self and
altered self. In addition, the organism must remember a pathogen that is has
seen before. 

The immune system is made up of two different parts. The first is the
\textbf{innate immune system} which is static and can always recognize patterns
that indicates the prescence of pathogens and response rapidly. This is also
considered the non-specific part. Examples include \textbf{macrophages} and 
\textbf{natural killer cells}.

The second part of the immune system is the \textbf{adaptive immune system}.
This part takes longer to respond to an infection, but it is specific to
different pathogens and not to pathogens in general. This enables immune memory. 

Within the adaptive immune system, there are two arms. The first is the
\textbf{humoral arm} which acts via soluble factors (antibodies). The second is
the \textbf{cellular arm} which acts via cells (cytotoxic T cells).

\section{ Adaptive Immune System }

There are two types of cell that make up the adaptive immune system which we are
concerned with. The \textbf{B cells} secrete antibodies. The \textbf{T cells}
have a range of different functions which are all involved in regulating
adaptive immune responses and mediating cellular response to altered self.

\subsection{ Antibodies }

There are four different peptides in an antibody: two heavy chains and two light
chains. Within these chains, there are constant and variable regions. The upper
segments of the Y-shaped antibody are the variable regions on both the light and
heavy chains. These site bind with the antigens. The constant regions allow the
antibody to interact with other parts of the immune system. These regions are
the same for all antibodies within a certain class.

The antibody will directly recognize pathogens (antigens). The antibody will
then recruit complement to lyse the pathogen. Since an antibody is bivalent, it
will form a clump of pathogens which will be consumed by macrophages and
degraded.

\subsection{ Antibody Diversity }

Using \textbf{VDJ recombination}, a B cell can rearrange its genome and create
diverse antibodies. In the genome, there are clumps of different V, D, and J
regions (along with the constant region). When producing an antibody, the cell
will grab different V, D, and J segments and combine them to form the varialbe
section of the heavy chain. For the light chain, the cell will undergo
\textbf{VJ recombination} as there is no D region. 

\subsection{ B Cells }

The prescence of a given antigen will tell a B cell (which has its own specific
antibody) to begin proliferating. The created cells are called \textbf{plasma
cells}. In addition, the B cell will produce memory cells which persist long
term (sometimes the entire lifetime of the organism) hidden away in the bone
marrow. This provides a quick response to a pathogen encountered before. 

\section{ Antibodies in Science }

Antibodies have the ability to recognize specific proteins, as a result, they
can be used in the purification of specific proteins. They can also be used to
tag certain parts of cell with flourescence for imaging. When doing gel
experiments, antibodies can be used to visualize the proteins. 

There are two different categories for antibodies in use. \textbf{Polyclonal
antibodies} are created when you inject an organism with a pathogen you wish the
organism to create antibodies against. They all respond to the pathogen, but
they are slightly different. \textbf{Monoclonal antibodies} are created by
taking a single B cell and, as a result, all the antibodies are identical. 

\section{ Vaccines }

A \textbf{vaccination} involves challenging an organism's immune system with an
antigen so that the immune system can have a rapid response to the pathogen in
the future when the pathogen is encountered again. This can be done through
inactivated form of the pathogen, a non-virolent form of the pathogen, broken
pieces of the pathogen, or simply the pathogen itself.

\end{document}

