\documentclass{article}
\usepackage{tikz}
\usepackage{float}
\usepackage{enumerate}
\usepackage{amsmath}
\usepackage{bm}
\usepackage{indentfirst}
\usepackage{siunitx}
\usepackage[utf8]{inputenc}
\usepackage{graphicx}
\graphicspath{ {Images/} }
\usepackage{float}
\usepackage{mhchem}
\usepackage{chemfig}
\allowdisplaybreaks

\title{ 7.012 Recitation 18 }
\author{ Robert Durfee }
\date{ November 28, 2017 }

\begin{document}

\maketitle

\section{ Cancer }

\textbf{Cancer} is s disease that is caused by aberrant proliferation of cells
and mutation of other tissues. This is caused by mutations in genes that
regulated proliferation. There are two major classes of cancers. First of which
is through \textbf{oncogenes} which are genes that upregular proliferation.
These result in \textbf{gain-of-function} mutations in cancer. The second of
which are through \textbf{tumor suppressor genes} which are genes that block
proliferation. These result in \textbf{loss-of-function} mutations. 

Loss-of-function mutations required \textbf{homozygous} mutations in order to
appear.  Gain-of-function mutations, on the other hand, can arise with only
\textbf{heterozygous} mutation.

Cancer typically arises in \textbf{somatic cells} (all cells except germ cells).
It is really difficult to get a homozygous mutation in a somatic cell. To get a
homozygous mutation, a somatic cell must undergo \textbf{loss of
heterozygosity}. This is when a heterozygous tumor suppressor mutation becomes
homozygous or hemizygous. This can occur in several ways like \textbf{homologous
recombination}, which results in homozygous, and \textbf{loss of wild-type
allele}, which results in hemizygous.

In the case of familial retnoblastoma, where there is an inherited mutation and
then a loss of homozygosity, and therefore cancer arises. 

\section{ Viruses }

A \textbf{retrovirus} is an RNA virus that goes through a DNA intermediate that
integrates into the host genome. In some cases, the virus can take some of the
cells genome with it during replication. In the case of RSV, the \textbf{Sarc
oncogene} was accidentally packaged. The cell was then transformed because Sarc
is an oncogene. With HPV, the viral genes enhance oncogenesis by inhibiting
tumor suppressing genes.

\textbf{Viruses} are parasites that take advantage of host machinery to
replicate themselves. There are several types of viruses such as RNA and DNA
viruses which can further be separated into double and single stranded. The RNA
viruses can be either plus or minus stranded as well. Furthermore, the virus can
be \textbf{enveloped} or not, meaning there may or may not be a membrane around
the capsid. 

DNA viruses always end up in the nucleus where the machinery exists to
replicated DNA. RNA viruses may end up in the nucleus or the cytoplasm.
Retroviruses (typically single, plus stranded) need to get into the nucleus to
integrate into the genome. RNA viruses that go from RNA to RNA (without DNA
intermediate) can just replicate in the cytoplasm. 

\subsection{ Quantification of Viruses }

Using \textbf{electron microscopy}, you can see the viral particles in an image
of tissue sample. Another method is to look for \textbf{plaque forming units}
where you extract the virus from the sample, then you dilute it several times
until you get to the point where you have very small number of viruses in your
sample. Then you put those viruses on a plate of cells. You will then end up
with patches of cells dying. Based on the number of patches, you can determine
the number of viral particles you had to begin with. Each viral particle that
causes these patches are called a \textbf{plaque forming unit}. The advantage of
this method is that there are many viruses that aren't properly replicated and
are not functional. Using the EM method, you are counting both functional and
nonfunctional viral particles.

\section{ Carcinogens }

A \textbf{carcinogen} is a compound that increases the risk of cancer. This is
gone by increasing the number of mutations you will get within your somatic
cells. This is either done directly or indirectly. If they are done directly,
they are called \textbf{mutagens}. If they are done indirectly, they are not
mutagens. 

The \textbf{Ames test} is a test for mutagenieity of compounds. This test
doesn't use human cells, rather, bacteria cells. The cells are monitored for
histidine synthesis (or something similar). They are placed in a histidine free
medium and one is treated with the compound and one is not. After the bacteria
has time to synthesis, you can compare the number of \textbf{revertants} between
the treated and untreated cells. The higher the number of revertants means the
more mutagenicity.

\section{ Cancer Therapy }

The main treatments of cancer include \textbf{chemotherapy}, \textbf{radiation},
\textbf{immunotherapy}, and \textbf{targeted druge therapies}. The first two is
to cause errors during DNA replication. This is worse for cancer cells because
they replicate a lot more frequently than normal cells. They don't specifically
target cancer cells, they just happen to be more harmful to cancer cells. 

Immunotherapy tells the immune system to specifically target the cancer cells.
Targetd drug therapy deals with drugs that are specifically targeting the
specific proteins that drive proliferation in a given cancer. This drug could
also inhibit the protein in normal cells as well. Therefore, this tries to only
target mutated version of the protein, if possible.

\end{document}

