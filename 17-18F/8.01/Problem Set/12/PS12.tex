\documentclass{article}
\usepackage{tikz}
\usepackage{float}
\usepackage{enumerate}
\usepackage{amsmath}
\usepackage{bm}
\usepackage{indentfirst}
\usepackage{siunitx}
\usepackage[utf8]{inputenc}
\usepackage{graphicx}
\graphicspath{ {Images/} }
\usepackage{float}
\usepackage{mhchem}
\usepackage{chemfig}
\allowdisplaybreaks

\title{ 8.01 Problem Set 12 }
\author{ Robert Durfee }
\date{ December 8, 2017 }

\begin{document}

\maketitle

\section{ Two Oscillating Springs }

\subsection*{ Part A }

Since the object is at equilibrium, the forces acting on it sum to zero. Since
there are only two (important) forces, they must be equal. Using Hooke's law:
$$ k l_{e} = 2k( l - l_{e} ) $$
$$ l_{e} = \frac{ 2l }{ 3 } $$

\subsection*{ Part B }

Suppose the block is offset to the right. (Which makes sense in the context of
the problem considering the initial velocity is negative.) Then the force
applied to the block by the left spring will be a restoring force equal to:
$$ \vec{ F }_{1} = -k \left( \frac{ 2l }{ 3 } + x \right) \hat{ i } $$

The force applied to the block by the right spring will also be a restoring
force, but in the opposite direction:
$$ \vec{ F }_{2} = 2k \left( \frac{ l }{ 3 } - x \right) \hat{ i } $$

Using Newton's second law:
$$ m \frac{ d^{2}x }{ dt^{2} } =  -k \left( \frac{ 2l }{ 3 } + x \right) $$
$$ m \frac{ d^{2}x }{ dt^{2} } =  2k \left( \frac{ l }{ 3 } - x \right) $$

Combining these two expressions results in:
$$ m \frac{ d^{2}x }{ dt^{2} } = -3kx $$

\subsection*{ Part C }

A solution to the differential equation above, as shown in class:
$$ x( t ) = A \cos( \omega_{0} t ) + B \sin( \omega_{0} t )$$
$$ v( t ) = -A \omega_{0} \sin( \omega_{0} t ) + B \omega_{0} \cos( \omega_{0} t ) $$

From the differential equation, we know $\omega_{0}$
$$ \omega_{0} = \sqrt{ \frac{ 3k }{ m } } $$

Solving for $A$ using the initial position:
$$ x_{0} = A \cdot 1 + B \cdot 0 $$
$$ A = x_{0} $$

Solving for $B$ using the initial velocity:
$$ -v_{0} = A \omega_{0} \cdot 0 + B \omega_{0} \cdot 1 $$
$$ B = -\frac{ v_{0} }{ \omega_{0} } = -v_{0} \sqrt{ \frac{ m }{ 3k } } $$

Plugging these values into the equation for position:
$$ x( t ) = x_{0} \cos \left( \sqrt{ \frac{ 3k }{ m } } t \right) - v_{0} \sqrt{
\frac{ m }{ 3k } } \sin \left( \sqrt{ \frac{ 3k }{ m } } t \right) $$

Plugging these values into the equation for velocity:
$$ v( t ) = -x_{0} \sqrt{ \frac{ 3k }{ m } } \sin \left( \sqrt{ \frac{ 3k }{ m }
} t \right) - v_{0} \cos \left( \sqrt{ \frac{ 3k }{ m } } t \right) $$

\section{ Journey to the Center of the Earth }

\subsection*{ Part A }

The force acting on the mass as it moves through the center of the Earth is
pointed towards the center of the Earth. Its magnitude is proportional to the
amount of mass contained within the smaller radius $r$.

\bigbreak

Given that the mass of the Earth is considered uniform:
$$ \rho = \frac{ M_{E} }{ 4/3 \pi R_{E}^{3} } $$

Converting this to the mass enclosed within the smaller radiu:
$$ m_{e} = \rho V_{e} = \frac{ M_{E} r^{3} }{ R_{E}^{3} } $$

Now we can use the gravitaional force equation to calculate the magnitude of the
force at any given radius $r$:
$$ \vec{ F } = -\frac{ G m m_{e} }{ r^{2} } \hat{ r } = -\frac{ G m M_{E} r }{
R_{E}^{3} } \hat{ r } $$

Using the work-kinetic energy theorem (and the knowledge that gravitational
force is conservative):
$$ U_{f} - U_{i} = - \int F dx $$

Substituting force expression and eliminating $U_{i}$ by setting $r_{i} = 0$
where potential energy is zero.
$$ U( r ) = \int \limits_{0}^{r} \frac{ G m M_{E} r }{ R_{E}^{3} } dr = \frac{ G
m M_{E} r^{2} }{ 2 R_{E}^{2} }$$

\subsection*{ Part B }

Using Netwon's second law, the force can be rewritten in terms of acceleration:
$$ m \frac{ d^{2}r }{ dt^{2} } = - \frac{ G m M_{E} }{ R_{E}^{3} } r $$

This forms the standard differential equation utilized in SHM.

\bigbreak

This can also be seen by comparing the potential energy equation to a spring's
potential energy equation which is also proportional the the squared
displacement from the equilibrium:
$$ \frac{ 1 }{ 2 } k x^{2} $$
$$ \frac{ 1 }{ 2 } \frac{ G m M_{E} }{ R_{E}^{3} } r^{2} $$ 

\subsection*{ Part C }

Calculating the initial energy (all potential):
$$ PE = \frac{ G m M_{E} R_{E}^{2} }{ 2 R_{E}^{3} } = \frac{ G m M_{E} }{ 2 R_{E} } $$

Setting this equal to final energy (all kinetic):
$$ \frac{ 1 }{ 2 } m v^{2} = \frac{ G m M_{E} }{ 2 R_{E} } $$

Solving for final velocity:
$$ v = \sqrt{ \frac{ G M_{E} }{ R_{E} } } $$

\section{ Physical Pendulum Pivoted Ring }

\subsection*{ Part A }

The energy is conserved throughout swinging because we assume only gravity (a
conservative force) acts on the pendulum. Potential energy:
$$ U( \theta ) = 2 m g R ( 1 - \cos \theta ) $$

Kinetic energy:
$$ K( \theta ) = \frac{ 1 }{ 2 } m v^{2} = 2 m R^{2} \left( \frac{ d\theta }{ dt }
\right)^{2} $$

Expressing total mechanical energy:
$$ E =  2 m g R ( 1 - \cos \theta ) + 2 m R^{2} \left( \frac{ d\theta }{ dt }
\right)^{2} $$

Since there is no change in mechanical energy:
$$ 0 = 2 m g R \sin \theta \frac{ d\theta }{ dt } + 4 m R^{2} \frac{ d\theta }{
dt} \frac{ d^{2}\theta }{ dt^{2} } $$
$$ 0 = 4 m R^{2} \frac{ d\theta }{ dt } \left( \frac{ d^{2}\theta }{ dt^{2} } +
\frac{ g }{ 2 R } \sin \theta \right) $$

Using the small angle approximation ($\sin \theta \approx \theta$):
$$ \frac{ d^{2}\theta }{ dt^{2} } = -\frac{ g }{ 2 R } \theta $$

Extracting $\omega_{0}$ from SHM:
$$ \omega_{0} = \sqrt{ \frac{ g }{ 2 R } } $$

Using the ratio of angular frequency and period:
$$ T = 2 \pi \frac{ 1 }{ \omega_{0} } $$
$$ T = 2 \pi \sqrt{ \frac{ 2 R }{ g } } $$

Equating kinetic and potential energies equal:
$$ 2 m g R ( 1 - \cos \theta ) = 2 m R^{2} \left( \frac{ d\theta }{ dt }
\right)^{2} $$

Solving for angular speed:
$$ \frac{ d\theta }{ dt } = \sqrt{ \frac{ g }{ R } ( 1 - \cos \theta ) } $$

Using small angle approximation ($\cos \theta \approx 1 - \theta / 2$):
$$ \frac{ d\theta }{ dt } = \sqrt{ \frac{ g }{ 2 R } } \theta $$

\subsection{ Part B }

Without the small angle approximation, we just go back one simplification step
from the last part:
$$ \frac{ d\theta }{ dt } = \sqrt{ \frac{ g }{ R } ( 1 - \cos \theta ) } $$

\section{ Small Oscillations }

\subsection*{ Part A }

Plugging the given force equation into the provided integral:
$$ U( r ) - U( \infty ) = - \int \limits_{\infty}^{r} \left( -\frac{ G m_{1}
m_{2} }{ r^{2} } + \frac{ C }{ r^{3} } \right) dr $$

Solving this definite integral yields:
$$ U( r ) = -\frac{ G m_{1} m_{2} }{ r } + \frac{ C }{ 2 r^{2} } $$

\subsection*{ Part B }

Taking the derivative of the previous equation to find minimum:
$$ U'( r ) = \frac{ G m_{1} m_{2} }{ r^{2} } - \frac{ C }{ r^{3} } $$

Setting derivative equal to zero and solving for $r$:
$$ 0 = \frac{ G m_{1} m_{2} }{ r^{2} } - \frac{ C }{ r^{3} }  $$
$$ r_{0} = \frac{ C }{ G m_{1} m_{2} } $$

Plugging $r_{0}$ back into potential energy function:
$$ U( r_{0} ) = -\frac{ ( G m_{1} m_{2} )^{2} }{ 2 C } $$

\subsection*{ Part C }

Taking the second derivative of the potential energy equation:
$$ U''( r ) = -\frac{ 2 G m_{1} m_{2} }{ r^{3} } + \frac{ 3 C }{ r^{4} } $$

Substituting our value for $r_{0}$ to get the $k$ constant:
$$ k \approx U''( r_{0} ) = \frac{ G^{4} m_{1}^{4} m_{2}^{4} }{ C^{3} } $$

Using the relation to angular frequency:
$$ \omega_{0} = \sqrt{  \frac{ k }{ m } } $$

Since $m_{1} \ll m_{2}$, only $m_{1}$ oscillates:
$$ \omega_{0} = \sqrt{ \frac{ G^{4} m_{1}^{3} m_{2}^{4} }{ C^{3} } } $$

\subsection*{ Part D }

At the extreme locations closest and furthest apart, there is no kinetic energy.
As a result, the total mechanical energy is equal to the potential energy:
$$ E = -\frac{ G m_{1} m_{2} }{ r } + \frac{ C }{ 2 r^{2} } $$

Solving this for $r$ yeilds two solutions (for when $U( r ) < E < 0$):
$$ r_{1} = - \frac{ \sqrt{ 2 C E + G^{2} m_{1}^{2} m_{2}^{2} } + G m_{1} m_{2}
}{ 2 E } $$
$$ r_{2} = \frac{ \sqrt{ 2 C E + G^{2} m_{1}^{2} m_{2}^{2} } - G m_{1} m_{2}
}{ 2 E } $$

\section{ MICA Cube Experiment }

\begin{enumerate}[1.]
    \item One of our group members will hold the cube in her hand and spin her
        arms and pull her arms in. 

        \begin{enumerate}[a.]
            \item A typical rate of angular speed for spinning with arms
                extended is about 2 revolutions per second. For spinning with
                arms pulled in, a typical angular speed is 24 revolutions per
                second. Converting to radians per second:
                $$ \omega_{min} \approx 4 \pi\ \si{ rad\ s^{-1} } $$
                $$ \omega_{max} \approx 48 \pi\ \si{ rad\ s^{-1} } $$

            \item Sketch of expected angular speed (x-axis is time and y-axis is
                angular speed):

                \begin{figure}[H]
                    \centering
                    \includegraphics[scale=0.60]{"Sketch"}
                    \caption{Approximate Expected Angular Speed}
                \end{figure}
                
        \end{enumerate}

    \item Link submitting via the Google form: https://youtu.be/6pC5VcIHj9Q

    \item Actual data collected by MICA cube:

        \begin{figure}[H]
            \centering
            \includegraphics[scale=0.5]{"Experiment"}
            \caption{Experiment}
        \end{figure}

        The first point, a, represents the initial speed. Point b shows the
        angular speed increasing her arms are drawn inwards. Point c shows the
        maximum speed with arms fully drawn in. point d shows her slowing down.

    \item Since angular momentum is nearly conserved (thanks to the ice), since
        our group member's moment of inertia is decreasing as her arms are
        pulled in, he angular speed must increase to cancel the change.

\end{enumerate}

\end{document}
