\documentclass{article}
\usepackage{tikz}
\usepackage{float}
\usepackage{enumerate}
\usepackage{amsmath}
\usepackage{bm}
\usepackage{indentfirst}
\usepackage{siunitx}
\usepackage[utf8]{inputenc}
\usepackage{graphicx}
\graphicspath{ {Images/} }
\usepackage{float}
\usepackage{mhchem}
\usepackage{chemfig}
\allowdisplaybreaks

\title{ 5.111 Lecture 35 }
\author{ Robert Durfee }
\date{ December 11, 2017 }

\begin{document}

\maketitle

\section{ Reaction Mechanism }

A reaction mechanism is a series of steps, or \textbf{elementary reactions} that
take place to convert reactants to products. We will address which reactions are
fast and slow and whether the reaction mechanisms are consistent with
experimental data. 

$$ \ce{ 2NO + O_{2} <=> 2NO_{2} } $$

The observed rate of formation is: $\ce{ NO_{2} } = k[ \ce{ NO } ]^{2}[ \ce{
O_{2} } ] $. However, there is a very low chance that these molecules combine in
one step. Proposed:
$$ \ce{ NO + NO <=> N_{2}O_{2} } $$
$$ \ce{ N_{2}O_{2} + O_{2} -> NO_{2} } $$

Rate of formation of the first reaction: $\ce{ N_{2}O_{2} } = k_{1} [ \ce{ NO }
]^{2}$. The rate of decomposition of the first reaction is $ \ce{ N_{2}O_{2} } =
k_{-1} [ \ce{ N_{2}O_{2} } ]$. The rate of consumption in the second reaction is
$ \ce{ N_{2}O_{2} } = k_{2}[ \ce{ N_{2}O_{2} } ][ \ce{ O_{2} } ]$. The rate of
formation of $\ce{ NO_{2} }$ is $2k_{2}[ \ce{ N_{2}O_{2} } ][ \ce{ O_{2} } ]$. The overall
net rate of change of $\ce{ N_{2}O_{2} }$ is $k_{1}[ \ce{ NO } ]^{2} - k_{-1}[
\ce{ N_{2}O_{2} } ] - k_{2}[ \ce{ N_{2}O_{2} } ][ \ce{ O_{2} } ]$.

We then assume that there is not change in the concentration of the
intermediate. Then we can solve for $\ce{ N_{2}O_{2} }$ concentration. 
$$ [ \ce{ N_{2}O_{2} } ] = \frac{ 2 k_{2}k_{1} [ \ce{ NO } ]^{2} }{ k_{-1} +
k_{2} [ \ce{ O_{2} } ] } $$

Plugging this into the equation for $[ \ce{ NO_2 } ]$:
$$ \mathrm{ Rate\ of\ formation\ of\ } \ce{ NO_{2} } = \frac{ 2 k_{2} k_{1} [
\ce{ NO } ]^{2} [ \ce{ O_{2} } ] }{ k_{-1} + k_{2}[ \ce{ O_{2} } ] }$$

But, this is not what was observed. This is because the above equation doesn't
take into account the fast and slow steps. If $k_{-1} \gg k_{2}$, then we can
simplify this equation:
$$ \mathrm{ Rate\ of\ formation\ of } \ce{ NO_{2} } = \frac{ k_{2}k_{1} [ \ce{
NO } ]^{2}[ \ce{ O_{2} } ] }{ k_{-1} } $$

This is consistent with what was described before. 

\end{document}

