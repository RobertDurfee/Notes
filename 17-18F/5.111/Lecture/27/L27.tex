\documentclass{article}
\usepackage{tikz}
\usepackage{float}
\usepackage{enumerate}
\usepackage{amsmath}
\usepackage{bm}
\usepackage{indentfirst}
\usepackage{siunitx}
\usepackage[utf8]{inputenc}
\usepackage{graphicx}
\graphicspath{ {Images/} }
\usepackage{float}
\usepackage{mhchem}
\usepackage{chemfig}
\allowdisplaybreaks

\title{ 5.111 Lecture 27 }
\author{ Robert Durfee }
\date{ November 17, 2017 }

\begin{document}

\maketitle

\section{ Cell Potential }

This is the potential difference betwee nthe two sides of the electrochemical
cell. It is the measure of the electron pulling (reduction) power of the
reaction.

$$1\ \si{J} = 1\ \si{C} \cdot \ \si{V}$$
$$\Delta G_{cell}^o = - n F E^o_{cell}$$

Where the standard states are gasses at 1 bar, solutes at 1 M, and liquids and
solids are pure.

$$E^o_{cell} = E_{right}^o - E_{left}^o$$

This equation allows us to look at half reactions and calculate the $E_{cell}$
for any combination of the half-reactions. Note that the left refers to the
anode/oxidation reaction and the right refers to the cathode/reduction reaction.
Also, these are reduction potentials only.

A large, positive reduction potential means that the compound really wants to be
reduced. (This makes it a good oxidizing agent as well.)

\section{Equilibrium Cells }

An exhausted battery is a sign that the cell reaction has reached equilibrium.
At equilibrium, a cell has zero potential difference across its electrodes.
Under what conditions does the cell reach equilibrium? $\Delta G$ changes as the
concentrations of the components change until the equilibrium is reach, when
$\Delta G = 0$.

$$\Delta G = \Delta G^o + RT\ln(Q)$$
$$\Delta G^o = -nFE_{cell}^o$$

Combining these equations results in the Nernst equation:

$$E_{cell} = E_{cell}^o -\frac{RT}{nF}\ln(Q)$$

Knowing that $\Delta G$ is zero, the condition for equilibrium can be solved for
as well as is shown below.
$$\ln(K) = \frac{nFE_{cell}^o}{RT}$$

\end{document}

