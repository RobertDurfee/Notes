\documentclass{article}
\usepackage{tikz}
\usepackage{float}
\usepackage{enumerate}
\usepackage{amsmath}
\usepackage{bm}
\usepackage{indentfirst}
\usepackage{siunitx}
\usepackage[utf8]{inputenc}
\usepackage{graphicx}
\graphicspath{ {Images/} }
\usepackage{float}
\usepackage{mhchem}
\usepackage{chemfig}
\allowdisplaybreaks

\title{ 5.111 Lecture 26 }
\author{ Robert Durfee }
\date{ November 15, 2017 }

\begin{document}

\maketitle

\section{Electrochemistry }

\textbf{Electrochemistry} is the study of redox reactions at electrodes. We can
obtain electricity from a spontaneous reaciton, or we can apply electricity to
push forward a nonspontaneous reaction.

A \textbf{galvanic cell} is one where extract electricity from a spontaneous
reaction. A \textbf{electrolytic cell} is a cell where we apply electricity to
causes a nonspontaneous reaction to occur.

\subsection{Components}

There are two \textbf{electrodes} used in each types of electrolytic cell. The
\textbf{anode} is oxidized and electrons flow out. The \textbf{cathode} is
reduced and electrons is used. (This can be rememberd by "an ox" and "red cat".)
Thus, the electrons flow through the wire from the anoode to the cathode. In a
galvanic cell, we seperate the parts of the reaction into two parts and connect
them electrically. The \textbf{salt bridge} allows the cations to flow to the
cathode. 

The special part of a galvanic cell is that we can extract the energy from the
electron transport by splitting the reaction into two half-reactions.

Electrodes configuration, whether they are anodes or cathodes, depends on the
other electrode they are paired with.

\subsection{Notation}

$$\ce{Zn\ \vert\ Zn^{2+}\ \vert\vert\ Cu^{2+}\ \vert\ Cu}$$

The anode reaction is on the left and the cathode reaction is on the right. The
electrodes used are on the far left and right and the salt bridge transformation
is what appears in between the electrodes. The reaction proceeds from the left
to the right as the direction of electrons transferred.

\subsection{Faraday's Law}

The amount of product formed or reatant consumed is stoichiometrically equal to
the amount of electrons transferred. Farrady's constant allows us to convert
between number of moles of electrons and charge transfered. 

You can also use inert electrodes, such as platinum, and in conjunction with
another solution. A hydrogen electrode can be constructred with a platinum
electrode.

\subsection{Cell Potential}

The ability of an electrochemical reaction to move electrons is called the cell
potential or electromotive force.
$$\Delta G_{cell} = nFE_{cell} = W_c$$

This turns into the following at standard conditions:
$$\Delta G_{cell}^{\circ} = n F E_{cell}^{\circ}$$

The standard reduction potential of a cell can be determined by:
$$\Delta E^{\circ}_{cell} = E_{red}^{\circ}(\rm{cathode}) -
E_{red}^{\circ}(\rm{anode})$$

If this reduction potential is positive, the reaction is spontaneous. If this
value is negative, the reaction is nonspontaneous.

\end{document}

