\documentclass{article}
\usepackage{tikz}
\usepackage{float}
\usepackage{enumerate}
\usepackage{amsmath}
\usepackage{bm}
\usepackage{indentfirst}
\usepackage{siunitx}
\usepackage[utf8]{inputenc}
\usepackage{graphicx}
\graphicspath{ {Images/} }
\usepackage{float}
\usepackage{mhchem}
\usepackage{chemfig}
\allowdisplaybreaks

\title{ 5.111 Lecture 36 }
\author{ Robert Durfee }
\date{ December 13, 2017 }

\begin{document}

\maketitle

\section{ Catalysis }

A \textbf{catalyst} is a substance that takes part in a chemical reaction and
speeds it up, but doesn't undergo any permanent change itself. Catalysts,
therefore, do not appear in the overall balanced equation. Catalysts act by
reducing the $E_{a}$ for the forward and reverse reactions. In other words, the
catalysts stabilize the transition state. It is also important to note that the
catalysts have no effect on the Gibb's free energy of the reaction. Since Gibb's
free energy is a state function, there is no dependence upon the path taken. As
a result, there is no change in the equilibrium constant.

\textbf{Homogeneous} catalysts are catalysts that are in the same phase as the
reactants. \textbf{Heterogeneous} catalysts are catalysts that are in different
phases than the reactants.

\section{Enzymes}

These are catalysts that occur naturally in biology. An \textbf{enzyme} is a
large protein molecule made up of amino acids. Using \textbf{peptide bonds},
these amino acids are combined into complex, compact structures.

Reactant molecules involved with a catalyzed biological reaction are called
\textbf{substrates}. Substrates bind to an active site on the enzyme.

$$ \ce{ E + S <=> ES} $$
$$ \ce{ ES -> E + P } $$

There are several rates:
\begin{center}
  \begin{tabular}{ c c }
    Rate of ES formation: & $k_{1}[\ce{E}][\ce{S}]$\\
    Rate of ES consumption: & $k_{-1}[ \ce{ES}]$\\
    Rate of ES consumption: & $k_{2}[\ce{ES}]$\\
  \end{tabular}
\end{center}

The overall reaction rate is (assuming the steady-state approximation):
$$ 0 \approx k_{1}[\ce{E}][\ce{S}] - k_{-1}[\ce{ES}] - k_{2}[\ce{ES}] $$

$$ [\ce{ES}] = \frac{ k_{1}[E]_{0}[S] }{ k_{1}[S] +k_{-1} + k_{2} } $$

Notice that $[E]$ is replaced with $[E]_{0} - [ES]$. We can further simplify
this reaction by combining the rate constants:
$$ [ES] = \frac{ [E]_{0}[S] }{ [S] + K_{m} } $$

Converting to the rate of product formation results in the
\textbf{Michaelis-Menten} equation:
$$ \frac{ d[P] }{ dt } = \frac{ k_{2} [E]_{0}[S] }{ [S] + K_{m} } $$

The large value of $K_{m}$ denotes the substrate affinity for the enzyme.  High
substrate affinity is determined by a small $K_{m}$. For a very low $K_{m}$, the
less $[S]$ is require to reach the maximum rate: $k_{2}[E]_{0} = V_{max}$.

If there is a significantly lower [S] than $K_{m}$, then adding more substrate
increases the rate significantly. If there is more [S] than $K_{m}$, then adding
more substrate does not noticeably increase the rate. If $[S] = K_{m}$, then the
rate is half of the maximum rate.

The \textbf{turnover frequency}, $k_{cat}$, are the number of catalytic cycles
performed by an enzyme per unit time.
$$ k_{cat} = k_{2} = \frac{ V_{max} }{ [E]_{0} } $$

\textbf{Catalytic efficiency}:
$$ \eta = \frac{ k_{cat} }{ K_{m} } $$

When $k_{2} \gg k_{-1}$:
$$ \eta \approx k_{1} $$

\section{Inhibitors}

\textbf{Inhibitors} are the opposite of enzymes. They increase the activation
energy. They act similarly to the substrates and bind to the enzyme. This
prevents the actual substrate from binding to the enzyme and getting catalyzed.
\end{document}

