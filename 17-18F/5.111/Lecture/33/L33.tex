\documentclass{article}
\usepackage{tikz}
\usepackage{float}
\usepackage{enumerate}
\usepackage{amsmath}
\usepackage{bm}
\usepackage{indentfirst}
\usepackage{siunitx}
\usepackage[utf8]{inputenc}
\usepackage{graphicx}
\graphicspath{ {Images/} }
\usepackage{float}
\usepackage{mhchem}
\usepackage{chemfig}
\allowdisplaybreaks

\title{ 5.111 Lecture 33 }
\author{ Robert Durfee }
\date{ December 6, 2017 }

\begin{document}

\maketitle

\section{ Radioactive Decay }

In nucleal chemistry, there are changes that occur in the nucleus. 
$$ {}^{A}_{Z}X_{N}$$

In the above diagram, $A$ is the total number of nucleons (atomic mass). $Z$ is
the number of protons. $N$ is the number of neutrons. The minimum amount of
specificity is simply $A$ and $X$. 

An isotope is defined by $Z$ and the total number of nucleons, $A$. 

\subsection{ Spontaneous Nuclear Decay }

This does not occur for stable elements. Atomic mass and atomic number can both
change. Both mass an charge must balance in a nuclear decay reaction. The
reaction must be spontaneous, even if it is very slow.

The first type of nuclear decay is alpha decay.
$$ \ce{ {}^{A}_{Z}X -> {}^{A - 4}_{Z - 2}Y + {}^{4}_{2} \alpha^{} + 2e- } $$

This is exclusive to heavy elements with $Z > 83$. These alpha particles are
easily absorbed and cannot travel very far.

The second type is beta decay:
$$ \ce{ {}^{A}_{Z}X -> {}^{A}_{Z + 1}Y + {}_{-1}^{0} \beta } $$

These occur where there is a high difference in number of neutrons and protons.

The last is gamma radiation:
$$ \ce{ {}^{A}_{Z}X -> {}^{A}_{Z} + {}_{0}^{0} \gamma } $$

This occurs after beta or alpha decay leaves the particle in an excited state.
This type of radiation goes very far and doesn't really interact much with
compounds. 

\subsection{ Radioactive Kinetics }

The number of nuclei, $N$, decays with first order kinetics. The rate of this
decay is measured using a \textbf{Geiger counter}.
$$ N = N_{0}e^{-kt} $$

Nuclear decay rate is also called the \textbf{activity}. 
$$ A = -\frac{ dN }{ dt } = kN $$
$$ A( t ) = A_{0}e^{-kt} $$

The half-life for radioactive materials is given as:
$$ t_{1/2} = \frac{ N_{0} \ln 2 }{ A_{0} } $$

Initial activity and half lives are inversely related. 

\section{ Second Order Rate Laws }

The rate law is defined as:
$$ k[ \ce{ A } ]^{2} $$

The second order integrated rate law is:
$$ \frac{ 1 }{ [ \ce{ A } ] } = kt + \frac{ 1 }{ [ \ce{ A } ]_{0} } $$

The half-life is, then:
$$ t_{1/2} = \frac{ 1 }{ k [ \ce{ A } ]_{0} } $$

The higher the concentration, the shorter the half-life. To determine the order
of a reaction, plot the natural log and the inverse and see which has a linear
relation.

\section{ Realtion Between Rate Constant and Equilibrium Constant }

$$ \ce{ A + B <=> C + D } $$

For the forward reaction:
$$ k_{1} [ \ce{ A } ] [ \ce{ B } ] $$

For the reverse reaction:
$$ k_{-1} [ \ce{ C } ] [ \ce{ D } ]$$

This shows the following relation:
$$ K_{eq} = \frac{ k_{1} }{ k_{-1} } $$

\section{ Elementary Steps and Molecularity }

Reactions do not typically occur in one step, but proceed through a series of
steps. Each step is called an \textbf{elementary reaction}. The overall
reaciton, the order and the rate law cannot be derived from the stoichiometry of
the balanced reaction. But the elementary reactions can as they occur exactly as
they are written. 

The number of reactant molecules that come together to form a product is the
\textbf{molecularity}. \textbf{Unimolecular} deals with a single reactant.
\textbf{Bimolecular} deals with two reactants, etc.

\end{document}

