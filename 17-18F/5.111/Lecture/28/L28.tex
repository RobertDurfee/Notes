\documentclass{article}
\usepackage{tikz}
\usepackage{float}
\usepackage{enumerate}
\usepackage{amsmath}
\usepackage{bm}
\usepackage{indentfirst}
\usepackage{siunitx}
\usepackage[utf8]{inputenc}
\usepackage{graphicx}
\graphicspath{ {Images/} }
\usepackage{float}
\usepackage{mhchem}
\usepackage{chemfig}
\allowdisplaybreaks

\title{ 5.111 Lecture 28 }
\author{ Robert Durfee }
\date{ November 20, 2017 }

\begin{document}

\maketitle

\section{ Transition Metals }

These are the elements in groups 3-12. They have many different uses. Some are
used for hydrogen fuel, some are used for coins, some defined the bronze and
iron ages, some make up paint pigments, and even natural, biological functions. 

The different electron configurations fill the d-orbitals. In these, it is
important to note that the fourth and ninth positions are skipped and fill the
upper s-orbitals instead. There are five different d-orbitals. $d_{z^2}$,
$d_{x^2,y^2}$, $d_{xy}$, $d_{yz}$, and $d_{xz}$.

These transition metals have many different oxidation states. Thus there are a
lot of different combinations that can be made with other elements. These give
the metals many of their unique properties. First they lose s-orbital electrons,
thn they lose d-oribtal electrons.

The transition metals with high oxidation states are strong oxidizing agents
(example given was manganese). Strong reducing agents, left of periodic table,
strong oxidizing agents, right of periodic table. 

Transition metals, becasue they can form many different complexes, make
excellent catalysts. 

\subsection{Coordination Compounds} 

These metals have strong abilities to form complexes with small molecules and
ions. \textbf{Donor atoms} or \textbf{ligands} are Lewis bases that donate
electrons. \textbf{Acceptor atoms} are transition metals which are Lewis acids
taht accept lone pair electrons.

\textbf{Coordination number} represents the number of ligands that bind to the
transition metal. The coordination number of 6 is the most common.

\textbf{Coordination complex notation} demonstrated the difference between
ligands and counter ions. The ligands are included in brackets. These are bound
to the metal ion directly.

Some ligands can have more than one donor atom. They have multiple lone pairs of
electrons and are large enough to wrap around a metal ion at more than one
place. These are \textbf{polydentates} or \textbf{chelating} agents. 

The \textbf{Chelate effect} is the high stability of metal chelates due to the
favorable entropic factor accompanying release of non-chelating ligands (usually
$\ce{H_2O}$) from the coordination sphere. 

\textbf{Geometric isonmers} are coordination compounds with the same ligands
which occur at different locations with respect to each other.

\textbf{Optical isomers} are non-superimposable mirror images of each other. A
\textbf{chiral} molecule has different properties in chiral environments (such
as a human body).

The number of electrons available for coordinating with ligands is the
\textbf{d-electron count}. This is equal to the group number minus the oxidation
number.

\end{document}

