\documentclass{article}
\usepackage{tikz}
\usepackage{float}
\usepackage{enumerate}
\usepackage{amsmath}
\usepackage{bm}
\usepackage{indentfirst}
\usepackage{siunitx}
\usepackage[utf8]{inputenc}
\usepackage{graphicx}
\graphicspath{ {Images/} }
\usepackage{float}
\usepackage{mhchem}
\usepackage{chemfig}
\allowdisplaybreaks

\title{ 5.111 Review }
\author{ Robert Durfee }
\date{ December 14, 2017 }

\begin{document}

\maketitle

\section{ Transition Metals }

These are all the elements in the d-block on the periodic table. These are the
elements that begin to fill the d-orbitals. The electron configurations are
pretty normal, except chromium and manganese which will fill half of the d-shell
because that is more favorable. Copper and zinc  will the entire d-shell.

When forming ions, you always lose the 4s electrons first. Note that none of
these ions are negative, they will always be positive. Additionally, all these
ions do not have any 4s electrons remaining.

\section{Coordination Complex}

A \textbf{coordination complex} is not a bunch of covalent bonds. The ligands are
Lewis bases and the transition metal cations are Lewis acids. The Lewis acids
accept the electrons and the Lewis bases donate the electrons.

\subsection{Geometry of Complexes}

For 6 ligands, the geometry is octahedral. For 5 ligands, the geometry is square
pyramidal. For 4 ligands, we will tell you if it is tetrahedral or square
planar.

\subsection{Isomers}

Cis molecules have 90 degrees between them. Trans molecules have 180 degrees
between them. The order of the ligands is important. Cis essentially also means
on the same side and trans essentially means opposite sides.

\subsection{Chirality}

These molecules reflect a molecule across a mirror plane. If the images are
non-superimposable, they are chiral.

\subsection{Shapes of D-Orbitals}

Note that for the $d_{x^{2}-y^{2}}$ orbital, the lobes are on the axes. For
$d_{xy}$, $d_{yz}$, $d_{xz}$ orbitals, the lobes are in between the axes.

\subsection{Energy Gap}

Certain orbitals have higher energy than others. This energy is determined by
the clash between the orbitals and the ligands. The closer the ligands are
brought to the orbitals, the higher the clash.

\subsection{Assinging Electrons}

First, assign the oxidation numbers. The oxidation number of an ion is its
charge. Charges must add up to overall charge. Then, calculate the d-count of
the metal: d-count = group number - oxidation number. Then draw the appropriate
splitting diagram. Lastly, fill in the electrons. If the ligand is strong, low
spin, match spin before filling orbitals. If the ligand is weak, high spin,
fill orbitals before matching spin. For the strong ligands, there is a high
energy gap and the weak ligands have a small energy gap.

\section{Kinetics}

To determine the rate order with respect to a certain reactant, hold one
concentration constant and adjust the other. If the rate increases directly,
then the reaction is first order. If the rate doesn't change, then the reaction
is zero order. If the rate increases by a power of 2, then the rate is second.

\section{Nuclear Chemistry}

$$ A = k N $$

Activity is rate of decomposition. The number of nuclei is the analogue to
concentration. For nuclear chemistry, the reactions follow the first order.
$$ A_{t} = A_{0}e^{-kt} $$

\section{Reaction Mechanisms}

You cannot determine the rate of a reaction simply by the overall chemical
equation. However, you can do this by using reaction mechanisms. For example:
$$\ce{2A + B <=> D}$$
$$ \ce{D + B -> E + F} $$
$$ \ce{F -> G} $$

You can determine the overall reaction by canceling terms on each side:
$$ \ce{2A + 2B -> E + G} $$

In this case, D and F are the \textbf{intermediates}. We can also determine the
rate of the overall reaction using the elementary steps.
$$ r_{1} = k_{1}[A]^{2}[B] $$
$$ r_{2} = k_{2}[D][B] $$
$$ r_{3} = k_{3}[F] $$

The first reaction has a double arrow. Thus, you need to incorporate the reverse
reaction.
$$ r_{-1} = k_{-1}[D] $$

To determine the rate of the overall reaction, you take the rate of the rate
determining step. This step is the slowest step.  For this reaction, this rate
will be:
$$ r = k_{2}[D][B] $$

In the overall rate, D is an intermediate. We cannot use this in the overall
rate equation. The \textbf{steady state approximation} states that the rate of
consumption equals the rate of formation. As a result:
$$ r_{1} = r_{-1} + r_{2} $$
$$ k_{1}[A]^{2}[B] = k_{-1}[D] + k_{2}[ D ][ B ]$$

Factoring out D:
$$ [D] = \frac{ k_{1}[A]^{2}[B] }{ k_{-1} + k_{2}[B] } $$

But this is not the rate law yet. You must plug this back into the equation
above:
$$  r = k_{2}[B] \frac{ k_{1}[A]^{2}[B] }{ k_{-1} + k_{2}[B] } $$

\section{Enzymes}

Enzyme catalysis reduced the activation energy required in an overall reaction.
The basic equation looks like:
$$ \ce{E + S <=> ES } $$
$$ \ce{ES -> E + P} $$

It is important to note that the enzyme is \textbf{not} consumed in the overall
reaction. This is the definition for the enzyme.
$$ K_{M} = \frac{ k_{-1} + k_{2} }{ k_{1} } $$

\section{Electrochemistry}

Loss electrons is oxidation. Oxidation occurs at the anode. Gain electrons is
reduction. Reduction occurs at the cathode. Electrons flow from the anode to
the cathode. The cathode gets plated and the anode gets corroded.

\section{Acid/Base Titrations}

Equivalency point is reached when the number of moles of base equals the number
of moles of base. The pH is determined by using the reverse reaction and an ICE
chart.

When you have a strong acid reacting with a strong base, you need to determine
which you will have more of. Once you know the number of moles more of one, you
can determine the concentration of either $\ce{OH-}$ or $\ce{H+}$.

\section{Molecular Orbital Diagrams}

Remember the energy axis. The more electronegative element is lower in energy.
Nitrogen to the left elements has the pi bonds below the sigma bonds. For oxygen
and right elements, the pi bonds are above the sigma bonds. The anti-bonding
sigma is always above the anti-bonding pi.

The bond order is calculated by taking the number of bonding electrons and
subtracting the anti-bonding electrons and dividing by two. Diamagnetic is when
there are no unpaired electrons. Paramagnetic is when there is at least one
unpaired electron.

\section{Valence Bond Theory}

Steric number represents the number of things attached to the central atom,
including lone pairs of electrons. To determine the hybridization, you only need
to know the steric number.

\begin{center}
  \begin{tabular}{ c c }
    SN & Hyb \\
    2 & $sp$ \\
    3 & $sp^{2}$ \\
    4 & $sp^{3}$ \\
  \end{tabular}
\end{center}

Double ($\pi$) bonds are formed either by two $p_{x}$ orbitals or two $p_{y}$
orbitals.  By convention, the axis of the bond is the z-axis.

\section{Photoelectric Effect}

Classical physics predicted that the number of electrons ejected from a metal
surface would be unaffected by increasing the intensity of light. However,
experiments showed that an increase number of electrons were ejected as the
intensity of light was increased. This is because the intensity is the number of
photons per second. By increasing the number of photons, you increase the number
of electrons expelled.
$$ E = h\nu = \frac{ h c }{ \lambda } $$

The intensity of light is the number of photons per second. Only by increasing
the energy of light will you be able to increase the kinetic energy of the
electrons.

The threshold frequency is the minimum frequency of light needed for ejection of
an electron. It depends on the metal because different metals binds the
electrons with different strength.
$$ \lambda = \frac{h}{p} $$

\section{Wavefunctions}

A wavefunction is spherically symmetric if the wavefunction only depends on $r$.
This implies that the wavefunction is an s orbital. If there is a $\theta$ term,
then this has to be a  p or d orbital. The general equation for a wavefunction
is:
$$ \Psi = R(r) \cdot Y(\theta, \phi) $$
$$ \Psi = \mathrm{constant} \cdot \mathrm{polynomial} \cdot \mathrm{constant}
\cdot f(\theta, \phi)$$

A node is a point where there is zero probability of an electron being in that
position. To find the location of a radial node, set the polynomial equal to
zero. An angular node is located when the function of $\theta$ and $\phi$ is set
to zero. Note that $\theta$ is constrained between $0$ and $\pi$ and $\phi$ is
constrained between $0$ and $2\pi$.

\end{document}
