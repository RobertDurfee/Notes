\documentclass{article}
\usepackage{tikz}
\usepackage{float}
\usepackage{enumerate}
\usepackage{amsmath}
\usepackage{bm}
\usepackage{indentfirst}
\usepackage{siunitx}
\usepackage[utf8]{inputenc}
\usepackage{graphicx}
\graphicspath{ {Images/} }
\usepackage{float}
\usepackage{mhchem}
\usepackage{chemfig}
\allowdisplaybreaks

\title{ 5.111 Recitation 20 }
\author{ Robert Durfee }
\date{ November 28, 2017 }

\begin{document}

\maketitle

\section{ Topics }

Chemical Equilibria

Le Chatelier's Principle

Solubility

Acid/Base Definitions and Equilibria

pH, pOH, $\rm{ pK_{a} }$, $\rm{ pK_{b} }$, $\rm{ pK_{w} }$

Buffers, Henderson-Hasselbalch Equation

Titrations and Special Points

Redox Reactions

Electrochemical Cells

Nernst Equation

\section{ Buffers }

$$ \ce{ CH_{3}COOH + H_{2}O <=> CH_{3}COO- + H_{3}O+ } $$

In this situation, you put the acetate salt and the acetic acid into solution.
To solve for the pH, you use the henderson-hasselbalch equation.

$$ \rm{ pH } = \rm{ pK_{a} } + \log \left( \frac{ [ \ce{ A- } ] }{ [ \ce{ HA } ]
} \right) $$

To construct a buffer, you want the $\rm{ pK_{a} } = \rm{ pH }$. This does not
need to be exact, just close. 

Another question they can ask is if you want the pH to be equal to a certain
value, what is the ratio of the two values in the logarithm.

\section{ Indicators }

An indicator should tell you when you reach the equivalency point. You will pick
and indicator with a $ \rm{ pK_{a} }$ close to the equivalency point.

\end{document}

