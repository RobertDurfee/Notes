\documentclass{article}
\usepackage{indentfirst}
\usepackage[utf8]{inputenc}
\usepackage[version=4]{mhchem}

\title{5.111 Recitation 14}
\author{Robert Durfee}
\date{November 2, 2017}

\begin{document}

\maketitle

\section{Regrade Policy}

Do \textbf{not} write on the exam. If you have a complaint, attach a note to the
exam detailing the complaint and return to us by Monday, November 6, after
lecture.

\section{Exam Statistics}

My score was 87\%.

\begin{center}
\begin{tabular}{c c}
    Average & 83.3\% \\
    Standard Deviation & 9.9\% \\
    Median & 86.0\%
\end{tabular}
\end{center}

\section{Exam Notes}

\subsection*{Question 1}

In the AXE configuration, the 'A' represents the central atom. This is not the
steric number.

\subsection*{Question 2}

Be sure to draw the labelled, vertical energy axis on molecular orbital energy
diagrams. Additionally, be sure to number the molecular subshells.

\subsection*{Question 3}

Read the instructions about significant figures. Be more careful when adding
numbers and including units.

\subsection*{Question 4}

Careful of units. By using the molar mass, you are compounding Avogadro's
number, therefore, your answer is not in units per mole, rather, units per mole
squared.

\section{Equilibrium}

\subsection{Gas Reaction Quotient}

The general reaction equation is:
\begin{gather*}
\ce{aA_{(g)} + bB_{(g)} <=> cC_{(g)} + dD_{(g)}}
\end{gather*}

The \textbf{reaction quotient} is a ratio of the partial pressures (measured in
bars) of the gas reactants and products:
\begin{gather*}
\ce{Q=\frac{(P_{C})^{c}(P_{D})^{d}}{(P_{A})^{a}(P_{B})^{b}}}
\end{gather*}

And the equation that relates the reaction quotient to Gibbs' free energy is:
\begin{gather*}
\ce{\Delta G=\Delta G^{o} + RT\ln(Q)}
\end{gather*}

Where all pressures are measured in bar.

\subsection{Aqueous Solution Reaction Quotient}

The general reaction equation becomes:
\begin{gather*}
\ce{aA_{(aq)} + bB_{(aq)} <=> cC_{(aq)} + dD_{(aq)}}
\end{gather*}

And the \textbf{reaction quotient} becomes a ratio of the concentrations
(measured in moles per liter) of the aqueous reactants and products:
\begin{gather*}
\ce{Q=\frac{[C]^{c}[D]^{d}}{[A]^{a}[B]^{b}}}
\end{gather*}

\subsection{Equilibrium Constant}

The equilibrium constant is a special reaction quotient. This reaction quotient
is where the \ce{\Delta G} is equal to zero. As a result, the condition for the
reaction equilibrium constant, K, is: 
\begin{gather*}
\ce{\Delta G^{o}=RT\ln(K)}\\
\ce{K=e^{-\Delta G^{o}/RT}}\\
\ce{R=8,314\ J/mol\cdot K}
\end{gather*}

\end{document}
