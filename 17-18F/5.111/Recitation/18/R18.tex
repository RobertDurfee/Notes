\documentclass{article}
\usepackage{tikz}
\usepackage{float}
\usepackage{enumerate}
\usepackage{amsmath}
\usepackage{bm}
\usepackage{indentfirst}
\usepackage{siunitx}
\usepackage[utf8]{inputenc}
\usepackage{graphicx}
\graphicspath{ {Images/} }
\usepackage{float}
\usepackage{mhchem}
\usepackage{chemfig}
\allowdisplaybreaks

\title{ 5.111 Recitation 18 }
\author{ Robert Durfee }
\date{ November 16, 2017 }

\begin{document}

\maketitle

\section{ Redox Reactions }

Redox stands for reduction-oxidation. This specifices whether a compound is
gaining or losing electrons. Oxidation means losing electrons and reduction
means gaining electrons. "Leo says ger." Reducing agent is the thing that is
oxidized and that causes reduction. Oxidizing agent is the thing that is reduced
and that causes oxidation.

$$\ce{2MnO_4^- + 5 C_2O_4^{2-} + 16H+ -> 2Mn^{2+} + 10CO_2 + 8H_2O}$$

Manganese oxidation number is +7 and carbon is +3. On the right, Manganese is +2
and carbon in +4. The oxidation number of manganese went down, thus it gained
electrons, thus is is reduced, thus it is the oxidizing agent. The oxidation
number of carbon went up, thus it lost electrons, thus it is oxidized, thus it
is the reducing agent.

\subsection{ How to Find Oxidation Numbers}

There are several rules:

\begin{itemize}
    \item If the element is in its natural state, then the oxidation number is
        0. This applies to diatomic molecules as well where the whole
        compound has an oxidation number of 0.
    \item If you are in group 1, the oxidation number is +1.
    \item If you are in group 2, the oxidation number is +2.
    \item Oxygen is probably -2. Almost always except in the case of a
        peroxide, then it is -1, a superperoxide, then it is -1/2, or an
        ozonide, then it is -1/3.
    \item Hydrogen is +1 if it is bonded to a nonmetal and is -1 if bonded to a
        metal.
    \item Flourine is always -1. Other halogens are usually -1 and the exception
        is if they are bonded to oxygen.
\end{itemize}

You can also fall back on lewis structures and calculate the oxidation number
similar to formal charge. However, the more electronegative atom in a bond will
take all the electrons from that bond. If they have the same electronegatively,
they each pull half of the electrons. Once again, the sum of oxidation numbers
equals the charge on the compound.

\subsection{Balancing Example}

$$\ce{MnO_4^- + C_2O_4^{2-} -> Mn^{2+} + CO_2}$$

Separate into two half-reactions.

$$\ce{MnO_4^- -> Mn^{2+}}$$
$$\ce{C_2O_4^{2-} -> CO_2}$$

Now balance everything except hydrogen and oxygen and charge.

$$\ce{MnO_4^- -> Mn^{2+}}$$
$$\ce{C_2O_4^{2-} -> 2CO_2}$$

Balance oxygen by adding $\ce{H_2O}$

$$\ce{MnO_4^- -> Mn^{2+} + 4H_2O}$$
$$\ce{C_2O_4^{2-} -> CO_2}$$

Balance hydrogen by adding $\ce{H+}$ (assuming acidic)

$$\ce{MnO_4^- + 8H+ -> Mn^{2+} + 4H_2O}$$
$$\ce{C_2O_4^{2-} -> CO_2}$$

Now all the mass is balanced. Now we balanced charge by adding electrons.

$$\ce{MnO_4^- + 8H+ + 5e- -> Mn^{2+} + 4H_2O}$$
$$\ce{C_2O_4^{2-} -> CO_2 + 2e-}$$

Multiply so that the number of electrons cancel.

$$\ce{2MnO_4^- + 16H+ + 10e- -> 2Mn^{2+} + 8H_2O}$$
$$\ce{5C_2O_4^{2-} -> 5CO_2 + 10e-}$$

Add the two together.

$$\ce{2MnO_4^- + 5 C_2O_4^{2-} + 16H+ -> 2Mn^{2+} + 10CO_2 + 8H_2O}$$

If in base, neutralize.

$$\ce{2MnO_4^- + 5 C_2O_4^{2-} + 8H_2O -> 2Mn^{2+} + 10CO_2 + 16OH-}$$

There is a faster way which involves calculating the oxidation numbers for the
two half reactions. Calculate the number of electrons released or acquired in
each reaction and multiply until they equal. Then add the equations together and
balanced oxygen and then hydrogen until balanced.

\section{Electrochemical Cells}

There are two different types: galvanic cells and electrolytic cells. Galvanic
cells are spontaneous and release electricity and electrolytic cells are
nonsponstaneous and require an electric current.

For a galvanic cell, there is an anode on the left (typically) and a cathode on
the right (typically). These elctrodes are connected with a wire and a load.

Lets say we have a zinc anode and a copper cathode.
$$\ce{Zn -> Zn^{2+} + 2e-}$$
$$\ce{Cu^{2+} + 2e- -> Cu}$$

In order to prevent a build up of charge in either beaker, there is a salt
bridge that connects the two solutions. These have unreactive ions that travel
to balance the charge build up in the solutions.

Electrons go from the anode to the cathode. A salt bridge is necessary for
charge neutrality. This can be remembered through "an ox" and "red cat".

There is also a shorthand notation:
$$\ce{Zn\ \vert\ Zn^{2+}\ \vert\vert\ Cu^{2+}\ \vert\ Cu}$$

The anode is on the left adn the cathode is on the right. The lines are phase
changes. The reaction always goes from the left to the right.

If your anode had chromium in solution and platinum as a cataylst, a comma would
be used beause there is no phase change:
$$\ce{Pt\ \vert\ Cr^{2+} ,,\ Cr^{3+}\ \vert\vert\ ...}$$

Standard H electrode (SHE) anode and cathode:
$$\ce{Pt\ \vert\ H_2\ \vert\ H+\ \vert\vert\ ...}$$
$$\ce{...\ \vert\vert\ H+\ \vert\ H_2\ \vert\ Pt}$$

Relating the electrochemical cell to Gibbs free energy:

$$\Delta G = -n F \Delta E_{cell}$$

Where $F$ is Faraday's constant with a value of 96485 $\si{C\ mol^{-1}}$.

$$\Delta E_{cell} = E_{\rm{cathode}} - E_{\rm{anode}}$$

Where $E$ is always the reduction potential, not oxidation potential.

\end{document}
