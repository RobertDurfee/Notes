\documentclass{article}
\usepackage{tikz}
\usepackage{float}
\usepackage{enumerate}
\usepackage{amsmath}
\usepackage{bm}
\usepackage{indentfirst}
\usepackage{siunitx}
\usepackage[utf8]{inputenc}
\usepackage{graphicx}
\graphicspath{ {Images/} }
\usepackage{float}
\usepackage{mhchem}
\usepackage{chemfig}
\allowdisplaybreaks

\title{ 5.111 Problem Set 8 }
\author{ Robert Durfee }
\date{ November 15, 2017 }

\begin{document}

\maketitle

\section{ pH, pOH, and pK }

Writing out the deprotonation chemcial equation for lactic acid:
$$\ce{C_3H_6O_3 <=> H+ + C_3H_5O_3-}$$

Convert the $\rm{pK_a}$ to a $\rm{K_a}$:
\begin{align*}
    \rm{K_a} &= 10^{-\rm{pK_a}} \\
    \rm{K_a} &= 8.317 \cdot 10^{-4}
\end{align*}

Writing out the equilibrium chart:

\begin{center}
    \begin{tabular}{c c c c}
         & $\ce{C_3H_6O_3}$ & $\ce{H+}$ & $\ce{C_3H_5O_3-}$ \\
        Initial & $1.5 \cdot 10^{-3}$ & 0 & 0 \\
        Change & $-x$ & $+x$ & $+x$ \\
        Equilibrium & $1.5 \cdot 10^{-3} - x$ & x & x
    \end{tabular}
\end{center}

Set equalibrium equal to $\rm{K_a}$:
\begin{align*}
    8.317 \cdot 10^{-4} &= \frac{x^2}{1.5 \cdot 10^{-3} - x} \\
    1.2476 \cdot 10^{-6} &= x^2 \\
    1.1170 \cdot 10^{-3} &= x
\end{align*}

The 5\% assumption doesn't hold. Using the quadratic equation:
\begin{align*}
    x &= \frac{8.317 \cdot 10^{-4} - \sqrt{ (-8.317 \cdot 10^{-4} )^2 -
    4(-1)(1.24755 \cdot 10^{-6})}}{2(-1)} \\
    x &= 7.7599 \cdot 10^{-4}\ \si{M}
\end{align*}

Thus, this value of $x$ equals the concentration of hydrogen ion:
\begin{align*}
    \rm{pH} &= -\log([\ce{H+}]) \\
    \rm{pH} &= -\log(7.7 \cdot 10^{-4}) \\
    \rm{pH} &= 3.11
\end{align*}

Solving for the pOH:
\begin{align*}
    \rm{pOH} &= 14 - \rm{pH} \\
    \rm{pOH} &= 14 - 3.11 \\
    \rm{pOH} &= 10.9
\end{align*}

Solving for the percent deprotonation:
$$\frac{7.7599 \cdot 10^{-4}}{1.5 \cdot 10^{-3}} = 52.7 \%$$

\section{Buffer Problems}

$\ce{KH_2PO_4}$ is the acid and $\ce{Na_2HPO_4}$ is the conjugate base.
$$\ce{H_2PO_4- <=> H+ + HPO_4^{2-} }$$

\begin{enumerate}[(a)]
    \item Calculating moles of $\ce{H_2PO_4-}$:
        $$(0.1\ \si{L})(0.100\ \si{M}) = 0.01\ \si{mol}$$

        Calculating moles of $\ce{HPO_4^{2-}}$:
        $$(0.1\ \si{L})(0.150\ \si{M}) = 0.015\ \si{mol}$$

        Solving for initial pH using the Henderson-Hasselbalch equation:
        \begin{align*}
            \rm{pH} &= \rm{pK_a} + \log \left( \frac{[\ce{A-}]}{[\ce{HA}]}
            \right) \\
            \rm{pH} &= 7.21 + \log \left( \frac{0.150}{0.100} \right) \\
            \rm{pH} &= 7.39
        \end{align*}

        Solving for moles of added base:
        $$(0.0800\ \si{L})(0.0100\ \si{M}) = 0.0008\ \si{mol}$$
        
        Setting up an ICE chart:
        \begin{center}
            \begin{tabular}{c c c c}
                & $\ce{H_2PO_4-}$ & $\ce{H+}$ & $\ce{HPO_4^{2-}}$ \\
                Initial & 0.01 &  & 0.015 \\
                Change & -0.0008 &  & +0.0008 \\
                Equilibrium & 0.0092 & x & 0.0158
            \end{tabular}
        \end{center}

        Solving for final pH using the Henderson-Hasselbalch equation:
        \begin{align*}
            \rm{pH} &= \rm{pK_a} + \log \left( \frac{[\ce{A-}]}{[\ce{HA}]}
            \right) \\
            \rm{pH} &= 7.21 + \log \left( \frac{0.0158}{0.0092} \right) \\
            \rm{pH} &= 7.44
        \end{align*}

        Change in pH:
        \begin{align*}
            \Delta \rm{pH} &= \rm{pH}_f - \rm{pH}_i \\
            \Delta \rm{pH} &= 7.44 - 7.39 \\
            \Delta \rm{pH} &= 0.05
        \end{align*}

    \item Solving for moles of added acid:
        $$(0.010\ \si{L})(1.0\ \si{M}) = 0.010\ \si{mol}$$
        
        Setting up an ICE chart:
        \begin{center}
            \begin{tabular}{c c c c}
                & $\ce{H_2PO_4-}$ & $\ce{H+}$ & $\ce{HPO_4^{2-}}$ \\
                Initial & 0.01 & 0.010 & 0.015 \\
                Change & +0.010 & -0.010 & -0.010 \\
                Equilibrium & 0.020 & x & 0.005
            \end{tabular}
        \end{center}

        Solving for final pH using the Henderson-Hasselbalch equation:
        \begin{align*}
            \rm{pH} &= \rm{pK_a} + \log \left( \frac{[\ce{A-}]}{[\ce{HA}]}
            \right) \\
            \rm{pH} &= 7.21 + \log \left( \frac{0.005}{0.020} \right) \\
            \rm{pH} &= 6.61
        \end{align*}

        Change in pH:
        \begin{align*}
            \Delta \rm{pH} &= \rm{pH}_f - \rm{pH}_i \\
            \Delta \rm{pH} &= 6.61 - 7.39 \\
            \Delta \rm{pH} &= -0.78
        \end{align*}

\end{enumerate}

\section{Titration Curves}
\begin{figure}[H]
    \centering
    \includegraphics[scale=1]{"Figure 1"}
\end{figure}

The pH of equivalency should be higher than 7.

\section{Titration Problem}

\begin{enumerate}[(a)]
    \item Deprotonation of benzoic acid:
        $$\ce{C_6H_5COOH + H_2O <=> H_3O+ + C_6H_5COO-}$$

        Setting up ICE chart:
        \begin{center}
            \begin{tabular}{c c c c}
                & $\ce{C_6H_5COOH}$ & $\ce{H_3O+}$ & $\ce{C_6H_5COO-}$ \\
                Initial & 0.15 M & 0 M & 0 M \\
                Change & $-x$ & $+x$ & $+x$ \\
                Final & $0.15 - x$ & $x$ & $x$
            \end{tabular}
        \end{center}

        Equating to $\rm{K_a}$
        \begin{align*}
            6.5 \cdot 10^{-5} &= \frac{x^2}{0.15 - x} \\
            x &= 3.12 \cdot 10^{-3}
        \end{align*}

        Solving for pH:
        \begin{align*}
            \rm{pH} &= -\log(3.12 \cdot 10^{-3}) \\
            \rm{pH} &= 2.51
        \end{align*}

    \item Deprotonation of benzoic acid with addition of $\ce{OH-}$:
        $$\ce{C_6H_5COO- + H_2O <=> OH- + C_6H_5COOH}$$

        Calculate moles of NaOH added:
        \begin{align*}
            n_{\ce{NaOH}} &= (0.0014\ \si{L})(0.30\ \si{M}) \\
            n_{\ce{NaOH}} &= 0.00042\ \si{mol}
        \end{align*}

        Calculate moles of $\ce{C_6H_5COOH}$:
        \begin{align*}
            n_{\ce{C_6H_5COOH}} &= (0.015\ \si{L})(0.15\ \si{M}) \\
            n_{\ce{C_6H_5COOH}} &= 0.00225\ \si{mol}
        \end{align*}

        Setting up ICE chart:
        \begin{center}
            \begin{tabular}{c c c c}
                & $\ce{C_6H_5COO-}$ & $\ce{OH-}$ & $\ce{C_6H_5COOH}$ \\
                Initial & 0 mol & 0.00042 mol  & 0.00225 mol \\
                Change & +0.00042 & -0.00042 & -0.00042 \\
                Final & 0.00042 & 0 & 0.00183
            \end{tabular}
        \end{center}

        Using the Henderson-Hasselbalch equation:
        \begin{align*}
            \rm{pH} &= \rm{pK_a} + \log \left( \frac{[\ce{A-}]}{[\ce{HA}]}
            \right) \\
            \rm{pH} &= 4.19 + \log \left( \frac{0.00042}{0.00183} \right) \\
            \rm{pH} &= 3.55
        \end{align*}

    \item At the half-equivalency point, half the number of moles of the given
        acid will be neutralized by the added base.
        $$\frac{n_{\ce{C_6H_5COOH}}}{2} = 0.001125\ \si{mol}$$

        Calculating the volume of NaOH needed to reach 0.001125 moles:
        \begin{align*}
            V &= \frac{0.001125\ \si{mol}}{0.30\ \si{M}} \\
            V &= 3.75\ \si{mL}
        \end{align*}

    \item The pH at half-equivalency is the value of $\rm{pK_a}$.
        \begin{align*}
            \rm{pH} &= \rm{pK_a} \\
            \rm{pH} &= 4.19
        \end{align*}

    \item The number of moles of added base will equal the number of moles of
        the acid.
        \begin{align*}
            V_{base} &= \frac{(M_{acid}(V_{acid})}{M_{base}} \\
            V_{base} &= 7.5\ \si{mL}
        \end{align*}

    \item Calculate the concentration of the conjugate base by using the total
        number of moles and the new volume at equivalency.
        \begin{align*}
            [\ce{C_6H_5COO-}] &= \frac{n_{\ce{C_6H5COO-}}}{V_{acid} + V_{base}}
            \\
            [\ce{C_6H_5COO-}] &= \frac{0.00225}{0.015 + 0.0075} \\
            [\ce{C_6H_5COO-}] &= 0.1\ \si{M}
        \end{align*}

        Setting up ICE chart:
        \begin{center}
            \begin{tabular}{c c c c}
                & $\ce{C_6H_5COO-}$ & $\ce{OH-}$ & $\ce{C_6H_5COOH}$ \\
                Initial & 0.1 M & 0 M  & 0 M \\
                Change & $-x$ & $+x$ & $+x$ \\
                Final & 0.01 - $x$ & $x$ & $x$
            \end{tabular}
        \end{center}

        Convert $\rm{K_a}$ to $\rm{K_b}$:
        \begin{align*}
            \rm{K_b} &= \frac{\rm{K_w}}{\rm{K_a}} \\
            \rm{K_b} &= \frac{10^{-14}}{6.5 \cdot 10^{-5}} \\
            \rm{K_b} &= 1.538 \cdot 10^{-10}
        \end{align*}
        
        Equating to $\rm{K_b}$:
        \begin{align*}
            1.538 \cdot 10^{-10} &= \frac{x^2}{0.1 - x} \\
            x &= 3.922 \cdot 10^{-6}
        \end{align*}

        Convert to pH:
        \begin{align*}
            \rm{pH} &= 14 + \log (3.922 \cdot 10^{-6}) \\
            \rm{pH} &= 8.59
        \end{align*}

    \item Calculate number of moles added past equivalency:
        \begin{align*}
            n_{\ce{NaOH}} &= (0.0045\ \si{L})(0.30\ \si{M}) \\
            n_{\ce{NaOH}} &= 0.00135\ \si{mol}
        \end{align*}

        Calculate the concentration of the $\ce{OH-}$ added past equivalency:
        \begin{align*}
            [\ce{OH-}] &= \frac{0.00135\ \si{mol}}{0.0045\ \si{L} + 0.015\
            \si{L} + 0.0075\ \si{L}} \\
            [\ce{OH-}] &= 0.05\ \si{M}
        \end{align*}

        Convert to pH:
        \begin{align*}
            \rm{pH} &= 14 + \log(0.05) \\
            \rm{pH} &= 12.70
        \end{align*}

\end{enumerate}

\end{document}

