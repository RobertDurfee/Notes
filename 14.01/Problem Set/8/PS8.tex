\documentclass{article}
\usepackage{tikz}
\usepackage{float}
\usepackage{enumerate}
\usepackage{amsmath}
\usepackage{bm}
\usepackage{indentfirst}
\usepackage{siunitx}
\usepackage[utf8]{inputenc}
\usepackage{graphicx}
\graphicspath{ {Images/} }
\usepackage{float}
\usepackage{mhchem}
\usepackage{chemfig}
\allowdisplaybreaks

\title{ 14.01 Problem Set 8 }
\author{ Robert Durfee }
\date{ December 1, 2017 }

\begin{document}

\maketitle

\section{ True or False }

\begin{enumerate}[1.]
    \item \textbf{False.} The EIFC will encourage those who earn very little to
        earn more money, however, it doesn't give credit to those who are
        wealthy. As a result, this must be phased out at a certain income where
        there is a dimishing EIFC, thus not encouraging to work more.

    \item \textbf{True.} The more elastic the curve, the more the burden of a
        tax will be felt. For example with completely inelastic demand (and not
        completely inelastic supply), a producer tax will shift the supply
        curve, but the quantity will remain the same and the price change will
        be felt completely by the consumer. If the demand were completely
        elastic, there would be no price change and thus the consumers will
        experience no tax burden.

    \item 
\end{enumerate}

\end{document}

