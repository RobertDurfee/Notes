\documentclass{article}
\usepackage{tikz}
\usepackage{float}
\usepackage{enumerate}
\usepackage{amsmath}
\usepackage{bm}
\usepackage{indentfirst}
\usepackage{siunitx}
\usepackage[utf8]{inputenc}
\usepackage{graphicx}
\graphicspath{ {Images/} }
\usepackage{float}
\usepackage{mhchem}
\usepackage{chemfig}
\allowdisplaybreaks

\title{ 14.01 Problem Set 8 }
\author{ Robert Durfee }
\date{ December 1, 2017 }

\begin{document}

\maketitle

\section{ True or False }

\begin{enumerate}[1.]
    \item \textbf{False.} The EIFC will encourage those who earn very little to
        earn more money, however, it doesn't give credit to those who are
        wealthy. As a result, this must be phased out at a certain income where
        there is a dimishing EIFC, thus not encouraging to work more.

    \item \textbf{True.} The more elastic the curve, the more the burden of a
        tax will be felt. For example with completely inelastic demand (and not
        completely inelastic supply), a producer tax will shift the supply
        curve, but the quantity will remain the same and the price change will
        be felt completely by the consumer. If the demand were completely
        elastic, there would be no price change and thus the consumers will
        experience no tax burden.

    \item \textbf{False.} When Alan has significantly more potatoes than Brian,
        the utilitarian government will prefer such a transfer:

        Consider Alan has 1000 potatoes and Brian has 10. Social welfare will
        be:
        $$ SW_{i} = \sqrt{ 1000 } + \sqrt{ 10 } = 34.79 $$

        After the transfer, Alan has 990 potatoes and Brian has 15. Social
        welfare then becomes:
        $$ SW_{f} = \sqrt{ 990 } + \sqrt{ 15 } = 35.34 $$

        As a result, the social welfare is higher in the second case.

    \item \textbf{False.} Both positive and negative externalities result in
        inefficient outcomes.

\end{enumerate}

\section{ Tax Incidence }

\begin{enumerate}[1.]
    \item Setting quantites supplied and demanded equal:
        $$ 40 - 3P = 2P - 20 $$
        $$ P^{*} = 12 $$

        Substituting this price into quantity demanded equation:
        $$ Q = 40 - 3P $$
        $$ Q^{*} = 4 $$

        In \textit{Figure 1}, the y-axis is the price, the x-axis is the
        quantity. The red line is demand and the blue lines is supply.

        \begin{figure}[H]
            \centering
            \includegraphics[scale=0.5]{"Supply and Demand for Sugary Drinks"}
            \caption{Supply and Demand for Sugary Drinks}
        \end{figure}

    \item Incorporating the tax into the marginal cost function (supply curve):
        $$ 40 - 3P = 2 ( P - \tau ) - 20 $$
        $$ P^{*} = \frac{ 60 + 2t }{ 5 } $$

        Substituting this price into quantity demanded equation:
        $$ Q = 40 - 3P $$
        $$ Q^{*} = \frac{ 20 - 6 \tau }{ 5 } $$

        In \textit{Figure 2}, the y-axis is the price, the x-axis is the
        quantity. The red line is demand, the blue lines is supply, and the
        yellow line is the new supply.

        \begin{figure}[H]
            \centering
            \includegraphics[scale=0.5]{"Supply and Demand for Sugary Drinks
            After Producer Tax"}
            \caption{Supply and Demand for Sugary Drinks After Producer Tax}
        \end{figure}

    \item Incoporating the tax into the marginal benefit (demand curve):
        $$ 40 - 3 ( P + \gamma ) = 2P - 20 $$
        $$ P^{*} = \frac{ 60 - 3\gamma }{ 5 } $$

        Substituting this price into quantity supplied equation:
        $$ Q = 2P - 20 $$
        $$ Q^{*} =  \frac{ 20 - 6\gamma }{ 5 }$$

        In \textit{Figure 3}, the y-axis is the price, the x-axis is the
        quantity. The red line is demand, the blue lines is supply, and the
        black line is the new demand.

        \begin{figure}[H]
            \centering
            \includegraphics[scale=0.50]{"Supply and Demand for Sugary Drinks
            After Consumer Tax"}
            \caption{Supply and Demand for Sugary Drinks After Consumer Tax}
        \end{figure}

    \item If $\gamma = \tau$, both the consumer and producer taxes will have the
        same effect as the quantities consumed/produced will be the same in
        either case as well as relative burdens between producers and consumers.

\end{enumerate}

\section{ Taxes and Redistribution }

\begin{enumerate}[1.]
    \item 
        \begin{enumerate}[a.]
            \item The budget constraints will follow the pattern:
                $$ c = w_{j} ( 24 - l ) $$

                Substituting in respective wages:
                $$ c_{r} = 360 - 15l_{r} $$
                $$ c_{p} = 120 - 5l_{p} $$

                In \textit{Figure 4}, the y-axis is consumption, the x-axis is
                hours of leisure. The red line is the rich budget constraint and
                the blue line is the poor budget constraint.

                \begin{figure}[H]
                    \centering
                    \includegraphics[scale=0.5]{"Rich and Poor Budget
                    Constraints"}
                    \caption{Rich and Poor Budget Constraints}
                \end{figure}

            \item Solving for the slope of the utility curve:
                $$ -\frac{ \partial U / \partial l }{ \partial U / \partial c }
                = -\frac{ 24 - l }{ 1 }$$

                Equating to wages for poor individual to calculate hours of
                leisure:
                $$ 24 - l_{p} = 5 $$
                $$ l_{p}^{*} = 19 $$

                Converting hours of leisure to hours of labor:
                $$ 24 - l_{p} = h_{p} $$
                $$ h_{p}^{*} = 5 $$

                Plugging hours of leisure into budget constraint to get
                consumption:
                $$ c_{p} = 120 - 5l_{p} $$
                $$ c_{p}^{*} =  25$$

                Equating to wages for rich individual to calculate hours of
                leisure:
                $$ 24 - l_{r} = 15 $$
                $$ l_{r}^{*} = 9 $$

                Converting hours of leisure to hours of labor:
                $$ 24 - l_{r} = h_{r} $$
                $$ h_{r}^{*} = 15 $$

                Plugging hours of leisure into budget constraint to get
                consumption:
                $$ c_{r} = 360 - 15 l_{r} $$
                $$ c_{r}^{*} = 225 $$

        \end{enumerate}

    \item 
        \begin{enumerate}[a.]
            \item The budget constraints will follow the pattern:
                $$ c = \frac{ 4 }{ 5 } w_{j} ( 24 - l ) $$

                Substituting in respective wages:
                $$ c_{r} = 288 - 12l_{r} $$
                $$ c_{p} = 96 - 4l_{p} $$

                In \textit{Figure 5}, the y-axis is consumption, the x-axis is
                hours of leisure. The red line is the new rich budget constraint and
                the blue line is the new poor budget constraint.

                \begin{figure}[H]
                    \centering
                    \includegraphics[scale=0.50]{"Rich and Poor Budget
                    Constraints After Tax"}
                    \caption{Rich and Poor Budget Constraints After Tax}
                \end{figure}

            \item The slope of the utility curve is the same:
               $$ -\frac{ \partial U / \partial l }{ \partial U / \partial c }
                = -\frac{ 24 - l }{ 1 }$$

                Equating to wages for poor individual to calculate hours of
                leisure:
                $$ 24 - l_{p} = \frac{ 4 \cdot 5 }{ 5 } $$
                $$ l_{p}^{*} = 20 $$

                Converting hours of leisure to hours of labor:
                $$ 24 - l_{p} = h_{p} $$
                $$ h_{p}^{*} = 4 $$

                Plugging hours of leisure into budget constraint to get
                consumption:
                $$ c_{p} = 96 - 4l_{p} $$
                $$ c_{p}^{*} =  16$$

                Equating to wages for rich individual to calculate hours of
                leisure:
                $$ 24 - l_{r} = \frac{ 4 \cdot 15 }{ 5 } $$
                $$ l_{r}^{*} = 12 $$

                Converting hours of leisure to hours of labor:
                $$ 24 - l_{r} = h_{r} $$
                $$ h_{r}^{*} = 12 $$

                Plugging hours of leisure into budget constraint to get
                consumption:
                $$ c_{r} = 288 - 12 l_{r} $$
                $$ c_{r}^{*} = 144 $$

            \item For each poor individual, the government collects:
                $$ R_{p} = \frac{ h_{p}^{*} \cdot w_{p} }{ 5 } = 4 $$

                For each rich individual, the government collects:
                $$ R_{r} = \frac{ h_{r}^{*} \cdot w_{r} }{ 5 } = 36 $$

                Overall tax collected per capita:
                $$ R =  \frac{ 1 }{ 7 } \cdot 36 + \frac{ 6 }{ 7 } \cdot 4 =
                8.57$$

        \end{enumerate}
        
    \item 
        \begin{enumerate}[a.]
            \item 
            \item 
        \end{enumerate}

    \item 
\end{enumerate}

\section{ Externalities }

\begin{enumerate}[1.]
    \item 
    \item 
    \item 
    \item 
\end{enumerate}

\end{document}

