\documentclass{article}
\usepackage{tikz}
\usepackage{float}
\usepackage{enumerate}
\usepackage{amsmath}
\usepackage{bm}
\usepackage{indentfirst}
\usepackage{siunitx}
\usepackage[utf8]{inputenc}
\usepackage{graphicx}
\graphicspath{ {Images/} }
\usepackage{float}
\usepackage{mhchem}
\usepackage{chemfig}
\allowdisplaybreaks

\title{ 14.01 Problem Set 6 }
\author{ Robert Durfee }
\date{ November 17, 2017 }

\begin{document}

\maketitle

\section{ True or False }

\begin{enumerate}
    \item \textbf{False}. In general, the price markup depends on the inverse
        number of firms in the market so the price markup will decrease as the
        number of firms increase. However, price markup also depends on the
        elasticity of demand. Thus, if the demand is perfectly elastic, there
        will be no price markup independent from the number of firms.
    \item \textbf{False}. Whether the firms in oligopoly act cooperatively or
        competitively, consumers will be worse off, so long as demand is not
        perfectly elastic. In general, the price markup will be lessened by the
        increase in number of firms, reaching zero with infinite firms (perfect
        competition).     
    \item 
        \begin{enumerate}[a.]
            \item \textbf{False}. If one cost is sufficiently high, the quantity
                produced will become negative. However, a negative quantity
                doesn't make sense. This means the firm will shutdown.
            \item \textbf{True}. For the linear cost curves given, subject to
                the same linear demand, the equilibrium quantity for the lower
                cost firm will be higher than the higher cost firm.
        \end{enumerate}
    \item \textbf{False}. Depending on the strength of the substitution and
        income effects, the market labor supply curve can be upward or downward
        sloping. For example, if someone is working only to purchase a single
        item of fixed cost, an increase in wages will result in a decrease in
        working time (income effect dominates).
    \item \textbf{False}. Many firms exercise a monopsony over their employees
        and thus pay them less than what they are worth to the company. As a
        result, there is already a deadweight loss, in many cases. If the
        minimum wage is increased within this range, the deadweight loss would
        decrease and an improvement for workers.
    \item \textbf{True}. Like in question 1.4, there are both income and
        substitution effects to be considered with an increase in interest
        rate. 
\end{enumerate}

\section{Cournot Duopoly}

\begin{enumerate}
    \item
    \item
    \item
\end{enumerate}

\section{Cournot with $N$ Firms}

\begin{enumerate}
    \item
    \item
\end{enumerate}

\section{Labor Market}

\begin{enumerate}
    \item
    \item
    \item
    \item
\end{enumerate}

\section{Intertemporal Choice}

\begin{enumerate}
    \item
    \item
    \item
    \item
\end{enumerate}

\end{document}

