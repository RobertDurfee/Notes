\documentclass{article}
\usepackage{tikz}
\usepackage{float}
\usepackage{enumerate}
\usepackage{amsmath}
\usepackage{bm}
\usepackage{indentfirst}
\usepackage{siunitx}
\usepackage[utf8]{inputenc}
\usepackage{graphicx}
\graphicspath{ {Images/} }
\usepackage{float}
\usepackage{mhchem}
\usepackage{chemfig}
\allowdisplaybreaks

\title{ 14.01 Problem Set 6 }
\author{ Robert Durfee }
\date{ November 17, 2017 }

\begin{document}

\maketitle

\section{ True or False }

\begin{enumerate}
    \item \textbf{False}. In general, the price markup depends on the inverse
        number of firms in the market so the price markup will decrease as the
        number of firms increase. However, price markup also depends on the
        elasticity of demand. Thus, if the demand is perfectly elastic, there
        will be no price markup independent from the number of firms.
    \item \textbf{False}. Whether the firms in oligopoly act cooperatively or
        competitively, consumers will be worse off, so long as demand is not
        perfectly elastic. In general, the price markup will be lessened by the
        increase in number of firms, reaching zero with infinite firms (perfect
        competition).     
    \item 
        \begin{enumerate}[a.]
            \item \textbf{False}. If one cost is sufficiently high, the quantity
                produced will become negative. However, a negative quantity
                doesn't make sense. This means the firm will shutdown.
            \item \textbf{True}. For the linear cost curves given, subject to
                the same linear demand, the equilibrium quantity for the lower
                cost firm will be higher than the higher cost firm.
        \end{enumerate}
    \item \textbf{False}. Depending on the strength of the substitution and
        income effects, the market labor supply curve can be upward or downward
        sloping. For example, if someone is working only to purchase a single
        item of fixed cost, an increase in wages will result in a decrease in
        working time (income effect dominates).
    \item \textbf{False}. Many firms exercise a monopsony over their employees
        and thus pay them less than what they are worth to the company. As a
        result, there is already a deadweight loss, in many cases. If the
        minimum wage is increased within this range, the deadweight loss would
        decrease and an improvement for workers.
    \item \textbf{True}. Like in question 1.4, there are both income and
        substitution effects to be considered with an increase in interest
        rate. 
\end{enumerate}

\section{Cournot Duopoly}

\begin{enumerate}
    \item The inverse demand function is
        $$P(Q) = 200 - Q$$
        Where $Q = q_F + q_G$. The cost functions for the two firms are
        $$C_F(q_F) = 5 q_F$$
        $$C_G(q_G) = \frac{1}{2} q_G^2$$
        The firms' profits are then given by:
        $$\pi_F(q_F, q_G) = q_F (200 - (q_F + q_G)) - 5 q_F$$
        $$\pi_G(q_F, q_G) = q_G (200 - (q_F + q_G)) - \frac{1}{2} q_G^2$$
        Maximizing profits individually:
        $$\frac{\partial \pi_F}{\partial q_F} = 195 - 2 q_F - q_G = 0$$
        $$\frac{\partial \pi_G}{\partial q_G} = 200 - q_F - 3 q_G = 0$$
        Solving this system of equations: 
        $$q_F^* =  77,\ q_G^* = 41$$
        Summing these individual quantities yields the total quantity:
        $$Q_T^* = 118$$
        Plugging this quantity back into the inverse demand function yields the
        market equilibrium price:
        $$P^* = 82$$
        In the following graph, the red curve is the best response for Ford
        and the blue curve is the best response for General Motors with $q_F$ on
        the horizontal axis and $q_G$ on the vertical axis. The equilibrium is
        the selected intersection of the two curves.
        \begin{figure}[H]
            \centering
            \includegraphics[scale=0.8]{"Figure 1"}
        \end{figure}
    \item The firms' joint profit is given by:
        $$\pi_T = 195 q_F + 200 q_G - 2 q_F q_G - q_F^2 - \frac{3}{2} q_G^2$$
        Maximizing joint profits:
        $$\frac{\partial \pi_T}{\partial q_F} = 195 - 2 q_G - 2 q_F = 0$$
        $$\frac{\partial \pi_T}{\partial q_G} = 200 - 2 q_F - 3 q_G = 0$$
        Solving this system of equations:
        $$q_F^* = 92.50,\ q_G^* = 5.00$$
        Summing these individual quantities yields the total quantity:
        $$Q_T^* = 97.50$$
        Plugging this quantity back into the inverse demand function yields the
        market equilibrium price:
        $$P^* = 102.50$$
        This quantity is lower than Cournot equilibrium and thus the price is
        higher. This is because the two firms can work together, assessing which
        has the lower cost and use that company to produce more of the output at
        lower cost. In addition, by working together, the two firms can act more
        like a monopoly, increasing price markup and avoiding the poisoning
        effect by producing less.
    \item Calculating each firms' marginal cost:
        $$MC_F(q_F) = 5$$
        $$MC_G(q_G) = q_G$$
        Setting each marginal cost equal to the inverse demand function:
        $$5 = 200 - (q_F + q_G)$$
        $$q_G = 200 - (q_F+q_G)$$
        Solving this system of equations:
        $$q_F^* = 190.00,\ q_G^* = 5.00$$
        Summing these individual quantities yeilds the total quantity:
        $$Q_T^* = 195.00$$
        Plugging this quantity back into the inverse demand function yields the
        market equilibrium price: 
        $$P^* = 5.00$$
        This quantity is higher than the Cournot competition and collusion cases
        because the government essentially sets the price for the two firms.
        Thus the firms cannot charge a markup and there is no poisoning effect.
        Since General motors is initially more efficient, it will provide the
        first 5 cars. Then, Ford becomes more efficient and will provide the
        remaining amount.
\end{enumerate}

\section{Cournot with $N$ Firms}

\begin{enumerate}
    \item Firm $i$'s profits:
        $$\pi_i(q_1, ..., q_N) = q_i \left(200 - \sum
        \limits^N q_j\right) - 5 q_i$$
        Maximizing profits:
        $$\frac{ \partial \pi_i(q_1, ..., q_N)}{\partial q_i} = \left(200 -
        \sum \limits^N q_j\right) - q_i - 5 = 0$$
        Solving for $q_i$ to get best response:
        $$ q_i(q_1, ..., q_{i-1}, q_{i+1}, ..., q_N) = 97.5 - \frac{1}{2} \sum
        \limits_{ j \neq i}^{N} q_j $$
    \item Setting each firms quantity equal to $q_i$:
        $$q_i^* = \frac{195}{N+1}$$
        Multiplying this individual quantity by the number of firms:
        $$Q_T^* = \frac{195 N}{N + 1}$$
        Plugging this quantity back in inverse demand to find price:
        $$P^* = \frac{5(N + 40)}{N + 1}$$
        The quantity value will increase as $N$ increases until perfect
        competition is met ($N = \infty$) where quantity produced will be 195.
        The price value decreases as $N$ increases until perfect competition is
        met ($N = \infty$)  where the price will be 5.
\end{enumerate}

\section{Labor Market}

\begin{enumerate}
    \item The worker's budget constraint in terms of leisure is:
        $$ 16 w = c + w l$$
        Leisure can be related to work in the following equation assuming if you
        aren't working, you are leisuring:
        $$ l = 16 - L $$
        Thus, the budget constraint can be rewritten as:
        $$ 16 w = c + w (16 - L)$$
    \item Solving for the slope of the budget curve:
        $$-\frac{P_l}{P_c} = -\frac{w}{1} = -w$$
        Solving for the slope of the utility function:
        $$-\frac{\partial U / \partial l}{\partial U / \partial c} =
        -\frac{c}{l}$$
        Equating slopes:
        $$w = \frac{c}{l}$$
        Plugging into the budget constraint to get leisure:
        $$16 w = w l + w l$$
        $$l^* = 8$$
        Plugging into the budget constraint to get consumption:
        $$16 w = c + c$$
        $$c^* = 8 w$$
        Converting leisure to work:
        $$L = 16 - 8$$
        $$L^* = 8$$
    \item Each laborer supplies $L^*$ units of labor, thus for 100 laborers,
        the labor supply curve will be:
        $$S_L(w) = 100 \cdot 8 = 800$$
        Each firm will produce $\sqrt{L}$ units:
        $$Y = \sqrt{L}$$
        The demand for labor is given by $MP_L \cdot P$ where $P=1$:
        $$D_L(w) = \frac{1}{4 w^2}$$
        Multiply this by 100 firms:
        $$D_L(w) = \frac{25}{w^2}$$
        Equating the supply and demand functions:
        $$w = 0.177$$
    \item Given that the labor supply curve is completely inelastic and does not
        depend on $w$, there will be no change in the equilibrium. Also, the tax
        is felt by the labor supplier, not the demander thus the demand curve
        will not change either.
\end{enumerate}

\section{Intertemporal Choice}

\begin{enumerate}
    \item The price of \$1 of consumption in the second period is \$($1+r$) in
        terms of the first period. This is because the a dollar saved will gain
        interest.
    \item The budget constraint in the first period is given by:
        $$1,000,000 = c_1 + s$$
        The budget constraint in the second period is given by:
        $$(1 + r)s = c_2$$
        Summing these two equations and eliminating $s$:
        $$1,000,000 = c_1 + \frac{c_2}{1 + r}$$
    \item Solving for the slope of the budget curve:
        $$-\frac{P_{c1}}{P_{c2}} = -(r+1)$$
        Solving for the slope of the utility function:
        $$-\frac{\partial U / \partial c_1}{\partial U / \partial c_2} = -
        \frac{c_2}{c_1}$$
        Equating slopes: 
        $$\frac{c_2}{c_1} = r+1$$
        Plugging into the constraint to get consumption in period 1:
        $$1,000,000 = c_1 + \frac{c_1(1+r)}{1+r}$$
        $$c_1^* = 500,000$$
        Plugging into the constraint to get consumption in period 2:
        $$1,000,000 = \frac{c_2}{r+1} + \frac{c_2}{r+1}$$
        $$c_2^* = 500,000(r+1)$$
        Thus, the income consumed will be \$500,000 and the income saved will be
        \$500,000.
    \item The consumption in period one is independent of the interest rate. As
        a result, the amount of income saved and spent in period one will be the
        same irrespective of the interest rate. However, the amount of
        consumption in the second period will increase as the interest rate
        increases. For $r=0.50$, $c_2=750,000$. For $r=0.60$, $c_2=800,000$

        In the following graphs, the orange curve is the indiference curve and
        the blue curve is the budget constraint with $c_1$ on the horizontal
        axis and $c_2$ on the vertical axis. The equilibrium is the selected
        interection of the two curves. The first graph shows $r=0.5$ and the
        second graph shows $r=0.6$.

        \begin{figure}[H]
            \centering
            \includegraphics[scale=0.8]{"Figure 2"}
        \end{figure}

        \begin{figure}[H]
            \centering
            \includegraphics[scale=0.8]{"Figure 3"}
        \end{figure}

\end{enumerate}

\end{document}
