\documentclass{article}
\usepackage{tikz}
\usepackage{float}
\usepackage{enumerate}
\usepackage{amsmath}
\usepackage{bm}
\usepackage{indentfirst}
\usepackage{siunitx}
\usepackage[utf8]{inputenc}
\usepackage{graphicx}
\graphicspath{ {Images/} }
\usepackage{float}
\usepackage{mhchem}
\usepackage{chemfig}
\allowdisplaybreaks

\title{ 14.01 Recitation 8 }
\author{ Robert Durfee }
\date{ November 17, 2017 }

\begin{document}

\maketitle

\section{ True and False}

\begin{enumerate}
    \item \textbf{Consider a risky investment with an expected payoff of \$100.
        No individual with well defined utility will ever prefer the risky
        investment to a \$500 bill.}

        False. There are some people who have an attitude for risk which is
        displayed through their utility function.

    \item \textbf{Consider the following game:}
        \begin{center}
            \begin{tabular}{c c c}
                & C & D \\
                C & (1, 1) & (0, 2) \\
                D & (2, 0) & (0, 0)
            \end{tabular}
        \end{center}
        \textbf{(D, D) can be avoided if the game is played repeatedly for
        finite (but very large) number of periods.}

        False. In an infinite number of games, cooperation can be accomplished,
        even though the Nash equilibrium is (D, D) in a single period. However,
        in a finite period, the last round, someone is going to cheat therefore
        they might as well start earlier.

\end{enumerate}

\section{Investment}

Suppose Alex's current wealth is \$400 and he faces a risky option (or
lottery). He can earn \$500 with a probability $\pi_A = 1 / 5$ and lose \$300
with a probability of $\pi_C = 1 / 5$ and he can gain \$0 with a probability of
$\pi_B = 3 / 5$. Suppose his utility is 
$$U(W)=\sqrt{w}$$
Where $W$ is wealth today and $w$ is wealth tomorrow.

\begin{enumerate}
    \item \textbf{What is the expected value of the risky investment?}

        $$E[L] = \frac{1}{5} \$500 - \frac{1}{5} \$300 + \frac{3}{5} \$0$$
        $$E[L] = \$40$$

    \item \textbf{Which utility level would Alex get if he received the expected
        values of this lottery for sure?}

        $$U(W) = \sqrt{\$400 + \$40} = \sqrt{\$440}$$

    \item \textbf{What is Alex's expected utility?}

        $$E[U] = \frac{1}{5} \sqrt{\$400 + \$500} + \frac{1}{3} \sqrt{\$400 -
        \$300} + \frac{3}{5} \sqrt{\$400}$$
        $$E[U] = \frac{\sqrt{\$900} + \sqrt{\$100} + 3\sqrt{\$400}}{5}$$
        $$E[U] = \$20$$

\end{enumerate}

\section{Investment}

Suppose Dave has an endowment of \$10k and he wants to invest in Amazon and
Bitcoin and each option is worth \$100 today and will be worth \$140 in one year
with probability $\frac{1}{2}$ or will stay at \$100 with probability
$\frac{1}{2}$.

Also note that this evolution of Amazon and Bitcoin are independent. This means
that both of them are two lotteries, and they are completely uncorrelated.

Suppose Dave's utility is $\sqrt{w}$.

\begin{enumerate}
    \item \textbf{What is Dave's utility from not investing?}

        $$U = \sqrt{\$10\rm{k}} = \$100$$

    \item \textbf{What is Dave's expected utility investing solely in Bitcoin?}

        $$E[B] = \frac{\$10\rm{k} + \$14\rm{k}}{2} = \$12\rm{k}$$

        $$E[U] = \frac{\sqrt{\$14\rm{k}} + \sqrt{\$10\rm{k}}}{2} = \$109.16$$

    \item \textbf{Suppose Dave wants to diversify between Amazon and Bitcoin in
        50-50.}

        For this problem, there are four cases for the output. Bitcoin goes up,
        Amazon goes down; Bitcoin goes down, Amazon goes up; Bitcoin and Amazon
        go down; Bitcoin and Amazon go up.

        $$E[U] = \frac{\sqrt{\$14k} + 2\sqrt{\$12k} + \sqrt{\$10k}}{4} = 109.35$$

\end{enumerate}

\end{document}

