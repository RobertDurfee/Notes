\documentclass{article}
\usepackage{tikz}
\usepackage{float}
\usepackage{enumerate}
\usepackage{amsmath}
\usepackage{bm}
\usepackage{indentfirst}
\usepackage{siunitx}
\usepackage[utf8]{inputenc}
\usepackage{graphicx}
\graphicspath{ {Images/} }
\usepackage{float}
\usepackage{mhchem}
\usepackage{chemfig}
\allowdisplaybreaks

\title{ 14.01 Lecture 21 }
\author{ Robert Durfee }
\date{ November 20, 2017 }

\begin{document}

\maketitle

\section{ Decision Making Under Uncertainty }

We assumed that everyone knows everything, but that isn't true. We face a lot of
decisions under uncertainty. We have a set of tools for addressing uncertainty
which is basically an extension to the consumer model. These tools are
\textbf{expected utility theory}. 

The \textbf{expected value} is the probability that you lose tiems the value of
losing plus the probability that you win times the value of winning. A
\textbf{fair gamble} has an expected value of zero. However, expected value is
not how people make decisions, they use \textbf{expected utility} because people
are \textbf{risk averse}. As a result, people do not like bets that are fair.

$$E[U] = P_L \cdot U_L + P_W \cdot U_W$$

This is different because of \textbf{diminishing marginal utility}. The next
unit doesn't make you as happy as losing that same about will make you sad. The
fact that you care about getting poorer than being richer leads people to be
risk averse.

\subsection{Example}

$$U = \sqrt{C}$$

We start with utility of $10$ and and original wealth of \$100. The expected
utility of the previous gamble is:

$$U_L = \sqrt{0}$$
$$U_W = \sqrt{225}$$
$$E[U] = 0.5 \cdot  \sqrt{0} + 0.5 \cdot \sqrt{225} = 7.5$$

This is why you would not take the bet. Risk aversion is defined as people not
willing to take fair bets.

\begin{figure}[H]
    \centering
    \includegraphics[scale=0.60]{"Risk Aversion"}
    \caption{Risk Aversion}
\end{figure}

Looking at \textit{Figure 1}. The curve is utiltiy. At point A, you originally
had before the gamble. You then take the linear average between the two points
of ulitity after the bet. The concavity of utility is the reason for risk
aversion. Beyond risk aversion, we hav \textbf{loss aversion} which means we
just feel terrible about losing what we have.

Suppose now you are forced to take this bet. How much are you willing to pay the
person forcing you to take the bet to not have to take the bet? You will the
difference such that your wealth is along the utility curve with the expected
value. Your utility will have to win \$300 more dollars such that your expected
utility is higher than what you started with.

\section{Extensions}

Now, what if you lose \$10 or win \$12.5. 

$$E[U] = 0.5 \cdot \sqrt{90} + 0.5 \cdot \sqrt{112.5} = 10.5$$

Now your utility function's curve means less and as a result you are less risk
averse. In \textit{Figure 2}, you are risk neutral. This means that you don't
care about winning or losing, you just care about the money. 

New utility function:
$$U = \frac{C^2}{1000}$$

\begin{figure}[H]
    \centering
    \includegraphics[scale=0.60]{"Risk Neutrality"}
    \caption{Risk Neutrality}
\end{figure}

The initial conditions are the same, but by taking the gamble, your utility
nearly doubles. This type of utility is \textbf{risk loving} where there is no
diminishing marginal utility. Now, this person is willing to take an unfair bet.
This person would pay to take an unfair bet. This type of utility function is
shown in \textit{Figure 3}. 

\begin{figure}[H]
    \centering
    \includegraphics[scale=0.60]{"Risk Loving"}
    \caption{Risk Loving}
\end{figure}

\section{Applications}

\subsection{Insurance}

We spend nearly 10\% of our economy on insurance. This is because of risk
aversion. Take a single 25 year old male. Your only risk is getting hit by a
car. Does this person get health insurance? The risk of getting hit by a car is
1\% and your income is \$40,000. If you get hit, your medical bills are
\$30,000. Thus the expected value every year is \$300. Say that your utility is
$U=\sqrt{C}$. How much will they pay for insurance? You pay at the beginning
year and pay regardless of if you get hit. If you have insurance, you don't have
to pay for the medical bills. 

\textbf{I will pay as much that will leave me at least as well of as not having
insurance.}

$$E[U] = 0.99 \cdot \sqrt{40,000} + 0.01 \cdot \sqrt{10,000} = 199$$

What if I buy insurance?

$$E[U] = 0.99 \cdot \sqrt{40,000 - x} + 0.01 \cdot \sqrt{40,000 - x}$$
$$\sqrt{40,000 - x} = 199$$
$$x^* = 399$$

Thus, I am willing to pay \$399 for insurance. How is this related to the
expected loss of getting hit by a car? \$300. Why is this lower? The difference
between this value and the insurance is the \textbf{risk premium}. In this case,
the risk premium is \$99. 

As the size of the loss rises, your risk premium rises. This is because of the
concavity of the utility function. As your income increases, then you are less
likely to pay a risk premium. This is because your utility is essentially
linear. Think Bill Gates in this model.

\subsection{Lottery}

This is the opposite of insurance. You expected value is half of what you pay.
So basically, this is a total rip off, yet they are wildly popular. How can it
be possible? Maybe people are actually risk loving. This is most likely not
true. The second possibility is that people are both risk loving and averse, or
\textbf{Friedman-Savage}.

\begin{figure}[H]
    \centering
    \includegraphics[scale=0.60]{"Risk Averse and Risk Loving"}
    \caption{Risk Averse and Risk Loving}
\end{figure}

In \textit{Figure 4}, what is people are locally risk averse, but globally risk
loving. Lets do some examples. 50-50 chance and you start at point $b^*$. You
will either go to $W_1$ or $W_3$. You will be risk averse. You start at $f^*$
and you get $W_1$ or $W_5$. You will take this bet as you will be take to point
$f$.

But the big bucks in lottery is not mega bowl, but in little scratch off
tickets. This would not make a situation like described above. So what is wrong?
Friedman-Savage can't exaplain it either. Now there is \textbf{gambling as
entertainment}. Thus, your utility also include the thrill of taking risk. With
a scratchoff, you are locally neutral, but you get the boost of winning.

Another theory is that people are misinformed or \textbf{misjudge} the deal. The
lottery is a tax on those who are bad at math. These last two are most
plausible, but they are opposing. Entertainment means this is an volutary tax
that results lower overall taxes and public services. If it were the misinformed
model, then we should hate lotteries and the poorest people use the lottery the
most. 

If you put new casinos in an area, there are more jobs and then more
bankruptcies.

\end{document}

