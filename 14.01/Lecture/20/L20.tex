\documentclass{article}
\usepackage{tikz}
\usepackage{float}
\usepackage{enumerate}
\usepackage{amsmath}
\usepackage{bm}
\usepackage{indentfirst}
\usepackage{siunitx}
\usepackage[utf8]{inputenc}
\usepackage{graphicx}
\graphicspath{ {Images/} }
\usepackage{float}
\usepackage{mhchem}
\usepackage{chemfig}
\allowdisplaybreaks

\title{ 14.01 Lecture 20}
\author{ Robert Durfee }
\date{ November 15, 2017 }

\begin{document}

\maketitle

\section{ Gains from Trade }

Ths intuition is that as long as there exists comparative advantage,
specialization yeilds gains from trade. This drives understanding of
international trade. In our past example, US has a comparative advantage in
producing computers and Columbia has a comparative advantage in roses. If the
countries do not trade, the consumers have to consumer at some point on the PPF
of each curve. Neither country can specialize. In \textit{Figure 1}, you can see
a convex PPF can be generated with economies of scope.

\begin{figure}[H]
    \centering
    \includegraphics[scale=0.8]{"Figure 1"}
    \caption{Joint PPF}
\end{figure}

\subsection{Comparative Advantage}

Where does comparative advantage come from? For example, Canada has a
comparative advantage in lumber. Therefore, the first type of comparative
advantage is \textbf{factor endowments}. Clothing comes from China because China
has a lot of cheap labor. The other source of comparative advantage is
\textbf{technology}. Japan is a major exporter of automobiles. They had
superior technology for producing technologies. This gives them a comparative
advantage. This, unlike factor endowments, technology can change as people can
learn. As a result, technology can be trasnferred. Now China is producing cars
as well.

\section{Welfare Implications of Trade}

Lets consider the market for roses in \textit{Figure 2}. Lets imagine there is
no trade initially and there is demand and supply for roses. There is an
equilibrium in Autarky. 

\begin{figure}[H]
    \centering
    \includegraphics[scale=0.9]{"Figure 2"}
    \caption{Welfare in Autarky}
\end{figure}

Now we allow international trade. We model trade is we add a world price in
\textit{Figure 3}. As a long as there are producers in the world that are more
efficient, this price will be lower than the US domestic price. Here we are assuming
the US's supply doesn't affect the world price. What this means is at the lower
price, US consumers want a higher quanty. US producers only want to produce
$Q_T$. The difference is filled by imports. 

Looking at \textit{Figure 4}, what is the welfare for the US? For consumers,
surplus used to be area $W$. Now their surplus is area $W+X+Z$. Producers used
to get $X+Y$, now they only get $Y$. Americans are better off by amount $Z$. The
total surplus in the US has gone up. That is why we like trade. This happens
because more trade is better. 

\begin{figure}[H]
    \centering
    \includegraphics[scale=0.9]{"Figure 3"}
    \caption{US Domestic Rose Market with Imports}
\end{figure}

\begin{figure}[H]
    \centering
    \includegraphics[scale=0.9]{"Figure 4"}
    \caption{Impact of Imports on Domestic Welfare}
\end{figure}

However, we don't only import, we also export shown in \textit{Figure 5}. Lets
assume the rest of the world in inefficient for computers. Now the world price
is higher. With the higher price, US consumers don't want as many computers.
More computers for world, less for US. Producers now want to produce $Q_T$, but
consumers only want $C_T$. This gap is filled by exports.

\begin{figure}[H]
    \centering
    \includegraphics[scale=0.9]{"Figure 5"}
    \caption{US Domestic Rose Market with Exports}
\end{figure}

\begin{figure}[H]
    \centering
    \includegraphics[scale=0.9]{"Figure 6"}
    \caption{Impact of Exports on Domestic Welfare}
\end{figure}

Looking at \textit{Figure 6}, the total welfare has gone up. Consumers go from
area $W+X$. Now they only have area $X$. Producers used to get area $Y$. Now
they get $X+Y+Z$. Society is better off because it has gained area $Z$.
Producers are better off, consumers are worse off, but we only care about total
welfare right now. Thus, trade makes us better off, whether importing or
exporting.

\section{Trade Policy}

Opposition to international trade says it is a job killer. When talking about
textiles, this is true because textile jobs shifted to China. We can stop import
of textiles from China by limiting the quantity, this is a \textbf{quota}. You
can also impose a \textbf{tariff} which increases the price of imports.

\begin{figure}[H]
    \centering
    \includegraphics[scale=0.9]{"Figure 7"}
    \caption{Tariff}
\end{figure}

In \textit{Figure 7}, the US imposes a tariff on roses. This tariff incrases the
price people from $P_W$ to $P_T$. This is done through a \textbf{flat tariff}.
Now at this new price, demand decreases and supply increases. As a result, the
gap between demand and supply decreases and imports are limited. Now more jobs
are kept in the US. 

The welfare implications are shown in \textit{Figure 8}. Consumers lose area
$A+B+C+D$ and producers gain area $A$. Thus, consumers are worse off. In addition,
the government gets income, which is the area $C$. Overall, society is worse off
by area $B+D$. This is why tariffs are a bad idea because the loss to consumers
is larger than the gain to producers.

In converse, you can ban exports, back in \textit{Figure 6}. The gain to
consumers will be less than the loss to producers. We are worse off stopping
trade in both ways, imports and exports.

\begin{figure}[H]
    \centering
    \includegraphics[scale=0.9]{"Figure 8"}
    \caption{Impact of Tariff on Domestic Welfare}
\end{figure}

If we impose a tariff there are other additional costs. Since we don't act in a
vacuum, a tariff can introduce a \textbf{trade war} where we have to deal with
an impose tariff on the products we export. Thus our producers will be harmed.
Another reason is that we \textbf{care about the rest of the world}. If we shut
down trade, we are worse off and those that would've traded with us would be
worse off.

Trade for 30-40 years has been shiting towards freer trade. But now it has
stopped. What are the problems? Our models assumes that we can
\textbf{compensate the losers}. For example, you can tax the consumers for
cheaper goods and transfer this money to those who lost their jobs. Since there
is a surplus, consumers will still be better off and producers will be just as
well off. In \textit{Figure 4}, all that would be neccesary is for consumers to
give producers area $X$.

There was a study done in areas with China-competing firms after China took
over, such as in the South with textiles. There was a massive decrease in jobs.
The losers are much louder than the winners.

Another problem is that there is \textbf{destructive comparative advantage}. Why
does China have cheap labor? They exploit their workers. China has comparative
advantage in areas that are harmful to environment and their child labor. A sort
of win-win solution is to raise labor standards and environmental standards in
order to be part of agreements like NAFTA.

\end{document}

