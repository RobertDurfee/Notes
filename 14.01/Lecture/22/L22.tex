\documentclass{article}
\usepackage{tikz}
\usepackage{float}
\usepackage{enumerate}
\usepackage{amsmath}
\usepackage{bm}
\usepackage{indentfirst}
\usepackage{siunitx}
\usepackage[utf8]{inputenc}
\usepackage{graphicx}
\graphicspath{ {Images/} }
\usepackage{float}
\usepackage{mhchem}
\usepackage{chemfig}
\allowdisplaybreaks

\title{ 14.01 Lecture 22 }
\author{ Robert Durfee }
\date{ November 27, 2017 }

\begin{document}

\maketitle

\section{ Efficiency vs Equity }

Before, we have only focuesd on maximizing the efficiency, but we didn't really
care how the "pie" was divided up. We have talked about the deadweight loss and
adding consumer and producer surplus to get total surplus. To think about this,
we can compare perfect competition and perfect price discrimination. The
triangles are the same size, but there are very different divisions. What if, in
order to make society more fair, we have to give something up. 

\textbf{Okun's Leaky Bucket} is a thought process we can use to think about
this. Lets assume that we think the rich should give money to the poor. This
money is put into a bucket and dumped in front of a poor person. What if, as you
carry the bucket to the poor person, you lose a certain percentage which is
wasted. At what point do say it is not worth it? Typically, it is not painless
to give money to the poor. How much leakage do we have to deal with?

We have to look at different parts to come to a solution. Firstly, how do we
value transfers? What is the distribution to begin with and end with? Where is
this leakage? And, lastly, what does this look like in practice?

\section{ Socially Optimal Allocation }

You can't get away from the fact you have to decided how society feels when you
transfer goods. To do this, we introduce a \textbf{social welfare function}.
This is a function of the utility of all the people involved. Essentiall,y it is
a utility function for society. We will deal with them the same way: we will
draw \textbf{isowelfare curves} which are the same as indifference curves. 

\textit{Figure 1} shows all combinations of utility of Homer and Ned. Holding
all of society resources constant, how is society indifferent between the
utility of multiple individuals.

\begin{figure}[H]
    \centering
    \includegraphics[scale=0.60]{"Isowelfare Curves"}
    \caption{Isowelfare Curves}
\end{figure}

\subsection{ Utilitarian }

What isowelfare curve function can we use? The \textbf{utilitarian isowelfare
curve} simply adds all the utility of all individuals. This says that we are
indifferent over who has the next unit of utility. This implies a massive shift
of wealth from the rich to the poor. This is because of \textbf{diminishing
marginal utility}. The effect of wealth on richer is less on utility than for
poorer. This also assumes no leaking. This is a centerist view.

$$ SWF = U_{1} + U_{2} + ... + U_{N} $$

\subsection{ Rawlsian }

A more liberal view is the \textbf{Rawlsian social welfare function}. Image that
you are just born and you know nothing. What would you want society to look like
"behind the vale of ignorance". This says that we would be better off destroying
any amount wealth to increase utility of the poorest. This essential means that
we should transfer no matter how much leakage.

$$ SWF = min( U_{1}, U_{2}, ..., U_{N} ) $$

\subsection{ Nozick }

The conservative view is that everyone has eqaul opportinuty. This is the
\textbf{Nozick social welfare}. We only care that everyon starts off equally,
and then let the "chips roll". The first problem is what means starting off
equal? The notion of equalizing opportunity is very difficult. Also, a lot of
succes comes from luck. Why would we want to reward this? This essentially says
that all wealth is achived through hard work.

\subsection{ Commodity Egalitarianism }

A midway view is the \textbf{Commodity Egalitarianism}. Lets say we are in a
society where everyone is in a decent standard. If one person is way better off,
that doesn't hurt you. This only cares about reaching a certain threshold and
not caring beyond this point.

\section{ Inequality }

\begin{figure}[H]
    \centering
    \includegraphics[scale=0.45]{"Income Distribution"}
    \caption{Income Distribution}
\end{figure}

\textit{Figure 2} shows income distribution. If everything were equal, all these
rows ( except the last ) would be 20\%. There has never been a society where
everything is equal. \textit{Figure 3} shows the amount of income held by the
top 1%. 

\textit{Figure 4} shows the comparison of income distribution between countries.
The US has the second worst income inequality behind Mexico ( among major
economies ). 

\textit{Figure 5} shows the inequality in Baltimore. Roland Park adn Sandtown
are 3 miles apart, yet their life expectancy varies dramatically. One is above
US average, and the other is below North Korea. This is also the case for
poverty rate and average income. 

\begin{figure}[H]
    \centering
    \includegraphics[scale=0.45]{"Share of Top 1 Percent"}
    \caption{Share of Top 1\%}
\end{figure}

Maybe we shouldn't care about relative equality, rather, use \textbf{absolute
deprivation}. We do this using the poverty line which says that people below
which cannot live in the United States. We accomplished this my multiply the
cost of a meal and multiply it by 3. In \textit{Figure 6}, the poverty line is
shown for families of different sizes. This is quite a small number.

In \textit{Figure 7}, the percent of different age groups below the poverty line
has gone down. Essentially, the poor have not gotten worse or better while the
rich got way richer. All of the social welfare curves show that we should be
redistribute more.

\begin{figure}[H]
    \centering
    \includegraphics[scale=0.50]{"Income Distribution Between Countries"}
    \caption{Income Distribution Between Countries}
\end{figure}

\begin{figure}[H]
    \centering
    \includegraphics[scale=0.45]{"Baltimore Life Expectancy"}
    \caption{Baltimore Life Expectancy}
\end{figure}

\begin{figure}[H]
    \centering
    \includegraphics[scale=0.60]{"Poverty Line"}
    \caption{Poverty Line}
\end{figure}

\begin{figure}[H]
    \centering
    \includegraphics[scale=0.45]{"Poverty Rates Over Time"}
    \caption{Poverty Rates Over Time}
\end{figure}

\section{ Leakage }

Where do the leakages arise in practice? The proce of taking money from the
rich and giving to the poor causes both to work less. This is \textbf{behavioral
response to tax and transfer system}.

\begin{figure}[H]
    \centering
    \includegraphics[scale=0.55]{"Impact of Tax and Transfer"}
    \caption{Impact of Tax and Transfer}
\end{figure}

In \textit{Figure 8}, everyone makes \$20. We introduce a transfer system where
everyone has to get \$10,000. 
$$ T = max( 0, 10000 - Y ) $$

To finance this, there will also be a tax. This tax will be 20\% of whatever is
earned above \$20,000. 

For point $C$, the budget constraint has been lowered because of the tax. This
only starts at \$20,000. In other words, the opportunity cost of leisure has
been lowered. Assuming upward sloping labor supply, you will work less.

\begin{figure}[H]
    \centering
    \includegraphics[scale=0.55]{"Labor Market"}
    \caption{Labor Market}
\end{figure}

For point $A$, they can work less and consume more. They can quit and get more
of both. For point $B$, he is willing to give up a little consumption to get a
lot more leisure. We have now reduced the labor supply of the rich and the poor.
If people work less, society is poorer, as shown in \textit{Figure 9}. There is
now a deadweight loss as we reduced the size of the economy. This is the
\textbf{equity efficiency trade-off}. We have made society more fair, but we
have also created deadweight loss. How much are we willing to shrink the pie to
redistribute wealth.

\end{document}
