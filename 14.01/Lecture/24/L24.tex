\documentclass{article}
\usepackage{tikz}
\usepackage{float}
\usepackage{enumerate}
\usepackage{amsmath}
\usepackage{bm}
\usepackage{indentfirst}
\usepackage{siunitx}
\usepackage[utf8]{inputenc}
\usepackage{graphicx}
\graphicspath{ {Images/} }
\usepackage{float}
\usepackage{mhchem}
\usepackage{chemfig}
\allowdisplaybreaks

\title{ 14.01 Lecture 24 }
\author{ Robert Durfee }
\date{ December 4, 2017 }

\begin{document}

\maketitle

\section{ Externality Theory }

Typically, governments control 20-40\% of economies. Is this all bad? No, there
are other reasons why government should be involved in regulating economies.
There is a study of the broad feild of \textbf{market failures} which is why
the private market may be unfair or inefficient. One example is monopolies and
\textbf{externalities}. This is whenever the action of one party affects another
party, yet that first party doesn't fare the benefits or costs of that effect.
If I do something that makes someone better off, but I don't received any
consequences, this is a \textbf{positive externality}. If I do something that
makes someone worse off, and receive no consequences, this is a \textbf{negative
externality}.

\subsection{ Negative Production Externality }

Suppose a steel factory on a river produces one unit of sludge per unit of
steel. Now this sludge kills the fish which fishermen fish for down the river.
There is no reason that the steel company should care. This creates the negative
externality. 

\begin{figure}[H]
    \centering
    \includegraphics[scale=0.63]{"Negative Production Externality"}
    \caption{Negative Production Externality}
\end{figure}

Looking at \textit{Figure 1}. The market for steel with private equilibrium at
point $A$. Say there is a marginal damage of \$100 through the death of fish.
The supply curve is now \textbf{private marginal cost} and the higher supply is
the \textbf{social marginal cost} which includes the harmful effects of the
death of fish. For the steel company, only their costs matter. But for society,
there is also the negative effect of the death of fish. 

The social marginal cost intercepts the demand at a lower production. If social
marginal cost is higher than private marginal cost, then the social marginal
benefit is higher. Then we want less steel because of diminishing marginal
benefit. This overproduction creates a deadweight loss of triangle $ABC$. This
is why it is called a market failure. The steel producer should take into
consideration the death of fish in his marginal cost.

\subsection{ Negative Consumption Externality }

\begin{figure}[H]
    \centering
    \includegraphics[scale=0.63]{"Negative Consumption Externality"}
    \caption{Negative Consumption Externality}
\end{figure}

In the case of smoking, your consumption of cigarettes makes others worse off.
Looking at \textit{Figure 2} and the market for cigarettes. We assume the
private marginal cost equals the social marginal cost. The private marginal
benefit (thrill of smoking) is higher than the social marginal benefit (others
around get pain). The optimal consumption of cigarettes is lower because the
consumer is ignoring the costs to others. This creates a deadweight loss.

\subsection{ Positive Consumption Externality }

There are piles of dirt in my neighbor's yard. If he gets rid of them, he is
\$800 happier. But the cost to remove them is \$1000. Then he won't remove the
dirt. These dirt piles make me sad as well (\%500). Overall social benefit is
\$1300. The neighbor will still not remove the dirt because he ignores the
social benefit. 

\bigbreak

Why can't I just pay my neighbor the difference? I can't: my neighbor would
think I'm a total asshole. Even if people were economically rational, should I
pay my benefit or the difference between? These negotiations are hard. The most
important reason is that most externalities are negative like the past two. How
do you compensate for second hand smoke and who do you compensate? In theory,
private market can solve, in practice, it cannot.

\section{ Government Solutions }

\subsection{ Quantity Regulation }

The government can force the steel to produce $Q_{1}$. But, this requires a
significant amount of information. The demand and supply curves (and social
marginal cost) are needed. This is not preferred.

\subsection{ Corrective Taxation }

\begin{figure}[H]
    \centering
    \includegraphics[scale=0.63]{"Corrective Taxation"}
    \caption{Corrective Taxation}
\end{figure}

For every unit you produce, I'll levvy a tax on each unit, shown in
\textit{Figure 3}. Now, his private marginal cost will become the social
marginal cost. The externality has been internalized and no longer an
externality. This is more effective than regulating quantity produced because
you only need to know the marginal damage in the form of a price.

This can also be done consumer side in the case of cigarettes as well by taxing
the cigarettes. The lower social benefit will then be internalized.

In the case of nuclear plants, we don't just want to tax after spill. This may
cause them to go out of business and still cause a spill. In this case, you will
accept the quantity regulation and its drawbacks. 

\subsection{ Corrective Subsidies }

For positive externalities, instead of taxing, you would subsidize. 

\bigbreak

Remember, we are now talking about efficiency right now. In the case of taxing
cigarettes, this is a tax on the poor. Now we have a question over equality. If
we both tax cigarettes and give the money to poor people, then we have a way to
deal with both equity and efficiency.

\section{ Policy Issues }

\subsection{ Health Externalities }

\begin{figure}[H]
    \centering
    \includegraphics[scale=0.75]{"Cigarette Taxation and Consumption"}
    \caption{Cigarette Taxation and Consumption}
\end{figure}

Drunk driving is a classic negative externality where you get all the benefit of
getting drunk and none of the consequences of killing others. Looking at
\textit{Figure 4} and the consumption of cigarettes versus the taxation on
cigarettes. This has massively lowered the consumption of cigarettes in the US.

However, we never stated that we should ban smoking. Why, then, do we have
illegal drugs? This model never suggests to ban anything. The negative
externalities come from making drugs illegal: the police force and violence. We
ban these drugs because we think that there is a break down of even this model.
We ban things because people don't know what they are doing (in the case of
underage smoking). This is a problem in the standard choice model.

\subsection{ Environmental Externalities }

Greenhouse gases have increased as the result of man-made production. Even
though some gases are neccesary, we are going well beyond the natural level. The
world will be 10\% poorer as the result of production loss from global warming. 

We don't think about this when we fill up with gas. This causes greenhouse gases
to be an externality. We should drive less because of the cost to others and the
environment. We could use this marginal damage to impose a tax, which has been
done in Europe (where the price is nearly double). Instead, we tried to go the
regulatory route (such as the Paris Accord and Kyoto Agreement) by limiting the
amout of greenhouse gases produced. This is done voluntarily. 

\end{document}

