\documentclass{article}
\usepackage{tikz}
\usepackage{float}
\usepackage{enumerate}
\usepackage{amsmath}
\usepackage{bm}
\usepackage{indentfirst}
\usepackage{siunitx}
\usepackage[utf8]{inputenc}
\usepackage{graphicx}
\graphicspath{ {Images/} }
\usepackage{float}
\usepackage{mhchem}
\usepackage{chemfig}
\allowdisplaybreaks

\title{ 14.01 Lecture 26 }
\author{ Robert Durfee }
\date{ December 11, 2017 }

\begin{document}

\maketitle

\section{ Behavioral Economics }

In the past, we have made a lot of assumptions about how people act in most
situtations. But there are some times when our assumptions break down. For
example, we often overlook human psychology. The more you enrich a model, the
more you can explain, but the more complicated it becomes. We try to minimally
enrich our models and we ask if there are certain policy issues we can address.

\subsection{Time Inconsistency}

This is a term that says everything we have modeled so far is right with an
additional wrinkle: humans have \textbf{self control problems}. This says that
people can know and maximize their utility and make plans, but they have issues
carrying this out tomorrow. This happens when an action has short-term costs,
but long-term benefits. For example, quitting smoking makes you miserable, but
you will live longer if you quit. You know the correct option is known to you,
but you lack to self-control to complete this. This is extensible to many other
areas as well such as exercise and dieting. 

An experiment was done on employees in Amsterdam, They were given a list of
snacks they could eat. About half of the people chose healthy and the others
chose unhealthy. When it came time to actually get the snacks, they changed
their minds.

There is also a huge demand for \textbf{commitment devices}. This is where you
essentially tie your own hands. In our models, we like to keep all options open.
But when quitting smoking, you try to make yourself miserable while smoking. 

Memberships at sports clubs: people would be better off if they pay every time
they go. But if you pay in advanced, they feel more guilty and force themselves
to go. 

We have incorporated this into our models by changing discounting.
\textbf{Discounting} is that, in life, we are impatient. All else equal, we want
money today, not tomorrow (assuming no inflation). This results in a new
indifference curve:
$$ U = \sum_{i = t}^{T} C_{i} \cdot \delta^{1 - t} $$
$$ \delta < 1 $$

This model doesn't work either. For example, the ratio between two years
of values. People want money now, versus two years, but versus 8 and 10 years,
you will take more money later.
$$ U = C + \beta \sum_{i = t + 1}^{T} C_{i} \cdot \delta^{i - t} $$
$$ \beta < 1 $$

This will resolve the issue expressed about the last utility function. This
simple change will allow the representation of a whole new area of economics.

\subsection{Loss Aversion}

This is not the same as risk aversion. You are afraid of losing what you already
have. For a small bet, your risk aversion comes into play. In a bet where 50\%
will win \$110, or lose \$100. If our preferences are the same as expressed
before with risk aversion, you would turn down the between \$1000 loss and
infinite win. 

\subsection{Unstable Preferences} 

A professor asked "How much will you pay me to sing?" and "How much will you pay
me not to sing?" He was able to draw demand curves for both, but this is
inconsistent because you should want to pay for both.

\subsection{Endowment Effect} 

\subsection{Statistical Biases} 

If you ask people if they are better than the average driver. 80\% say that they
are. That is impossible. Smokers are pretty good at knowing by how much smoking
will effect people's lives, they will accurately know. But they say there is no
effect on their personal lives.

\subsection{Intrinsic vs Extrinsic}

Consider taking a drug for a disease. Without insulin, you will die. But many
diabetics will not take their drugs regularly. If you pay them five dollars,
they will start to take their medication regularly.

\subsection{Defaults and Presentations Matter}

A company offered a 401(k) to their workers. Most workers don't sign up right
away. When the firm hired you, they changed from an opt in to an opt out, more
people were enrolled in the plan. The change was from 12\% to 80\%. This is not
rational behavior. 

\section{Policy}

\subsection{Government Provided Commitment} 

This addressed the time inconsistency problem. People can sufferer from
\textbf{internalities} which are damages people do to themselves. Time
consistent people will want to quit smoking. However, people cannot quit. The
government can provide you a method to help you quit: taxes. There is no legal
way to get around that. We have shown that higher taxes lead to less smoking.
This supports our utility function from before by taxing the current cost. 

As long as cigarettes are somewhat elastic, government taxes should not make
people happier. However, in the real world, people were happier after the
government raised taxes. This is inconsistent with the standard model.

\subsection{Nudges}

This is a way to make people happier, without making anything worse. This is
also called \textbf{benign paternalism}. An example of this would be the 401(k).
The \textbf{save more tomorrow plan} leads you to commit to saving your raise in
the future. In the standard model, this would be unappealling. However, in the
second model, this would be very attractive to people. Turns out, the second
model rules in this case.

\end{document}

