\documentclass{article}
\usepackage{indentfirst}
\usepackage{graphicx}
\graphicspath{ {Images/} }
\usepackage{float}
\usepackage[utf8]{inputenc}

\title{14.01 Lecture 16 - Factor Markets}
\author{Robert Durfee}
\date{November 1, 2017}

\begin{document}

\maketitle

\section{Introduction}

Producers face some cost of labor and cost of capital. Where do those costs come
from? In the past we assumed that they were given. But just as before when we
could determine the price of a good, we can determine the price of an input.
These prices are also determined from market equilibrium. 

\section{Factor Demand}

For our factor demands, we will first assume perfectly competitive market. This
assumption means that there are many buyers and many sellers. Therefore, no one
monopolizes or oligopolizes the market.

\subsection{Short Run Labor Demand}

What is the marginal cost and the marginal benefit for one more unit of labor?
The \textbf{marginal cost} is equal to the wages. That is, to hire one more
worker, we simply have to pay them $w$.

$$MC=w$$

The \textbf{marginal benefit} is equal to the marginal product of labor times
the price of their product. The firm doesn't solely care about the marginal
product of labor, they care about what that marginal product of labor brings
them in terms of profits.

$$MB=MP_{L}\cdot P$$

Determining the equilibrium position, we want to hire more workers until the
value that they produce is equal to what they cost to produce it.


$$MC=MB\Rightarrow w=MP_{L}\cdot P$$

\begin{figure}
    \centering
    \includegraphics[scale=0.33]{"Figure 16-1"}
    \caption{Firm's Labor Inputs}
\end{figure}

The slope of the marginal revenue product of labor shown in \textit{Figure 1} is
downward sloping because of diminishing product of labor which is included in
the calculation of marginal revenue product of labor as shown above. We also
assume the supply curve for labor is flat because this is the supply that one
given firm faces.

It is also important to note that firms are price takers for the labor market.
This may seem incorrect in the real world, however, this is the case with
perfect competition. 

\subsubsection{National Basketball Association Example}

Let's say we are an owner of a basketball team in the NBA. We are deciding how
much to pay our players. We would like to maximize wins, or the amount of money
we receive from those wins. We need to evaluate how much does each player
contributes to a given win. Multiply this times the value of a given win and we
can determine the salary of a given player.

LeBron James is arguably the best player in the NBA. His marginal product is
enormous. Therefore, his salary is quite high: \$31 million. This ignores salary
limitation and other minutia, however, the logic still holds.

Nate Robinson, one of the shortest players in the NBA, was originally a very
good player, winning several slam dunk contests. However, near the end of this
career, his ability began to wane. Therefore his marginal product was quire low
and his salary was only \$2 million. Then he moved to Israel and his marginal
product skyrocketed. He became a superstar, but his overall salary went down.
This was because the value of each given win was significantly less in Israel
than the United States NBA.

\subsection{Long Run Capital Demand}

Here we have a similar analysis to that of short run labor demand. We want to
employ machines until the cost of the next machine is equal to the value of what
that next machine produces. How much does it cost to use a machine? This is the
\textbf{marginal cost} of the machine $r$.

$$MC=r$$

Add another machine, how much does you product function go up? This is the
\textbf{marginal benefit} of the capital.

$$MB=MP_{K}\cdot P$$

Setting these two equal determines the equilibrium for the amount of capital
used.

$$MP_{K}\cdot P=r$$

It can help to think of all machines as being rented, the same as when we
thought of people being hired only for a day. 

\subsection{Long Run Labor Demand}

This has the same intuition as before, however capital is now variable. As a
result, long run demand will be more elastic than in short run because you can
substitute a machine for a person when your capital is not fixed. We can see
this in \textit{Figure 2}. Remember, elasticity is all about substitutability. 

\begin{figure}[H]
    \centering
    \includegraphics[scale=0.33]{"Figure 16-2"}
    \caption{Short and Long Run Labor Demand}
\end{figure}

\section{Factor Supply}

\subsection{Labor Supply}

Labor supply is interesting to talk about because we deal heavily with income
and substitution effects with changes in wages. Now, Giffen goods are not just
theoretical exercises, they actually exist.

When trying to model the labor supply, we are essentially trying to figure out
how much someone wants to work. However, most normal people (unlike those found
at MIT) don't like work. As a result, the trade-off is the more you work, the
more stuff you can buy, but the less time you will have to enjoy what you buy.
It is very difficult (or impossible) to model trading a good for a bad.
Therefore, we don't model the bad, we model the good: leisure time. Then, when
calculating the labor, we just invert leisure.

$$L^{*}=24-N^{*}$$

$N$ is the number of hours of leisure time and $L$ is the number of hours of
labor. $24$ represents the limit of 24 hours in a day.

\begin{figure}[H]
    \centering
    \includegraphics[scale=0.33]{"Figure 16-3"}
    \caption{Labor-Leisure Trade-Off}
\end{figure}

On the x-axis of \textit{Figure 3} we show increasing leisure to the right and
increasing work to the left. On the y-axis of \textit{Figure 3} we choose a
bundle of goods and set its price equal to one. This allows us to talk about
demand for all things. As the slope of the budget constraint is the price of
good x over price of good y, the slope of this budget constraint is the wage. 

$$\frac{P_{x}}{P_{y}}=\frac{w}{1}$$

This makes sense as wages are the opportunity cost of leisure, the good on the
x-axis.

Then there is an indifference curve which specifies how individuals prefer work
and leisure. The point of tangency is the optimum bundle of work and leisure.

\begin{figure}[H]
    \centering
    \includegraphics[scale=0.33]{"Figure 16-4"}
    \caption{Income and Substitution Effects for Labor Supply}
\end{figure}

In \textit{Figure 4}, an increase in wages makes leisure more expensive through
opportunity cost. As a result, leisure dropped. When wages go up, you are
relatively richer. The income effect will naturally go the "wrong" way for a
normal good.

\begin{figure}
    \centering
    \includegraphics[scale=0.33]{"Figure 16-5"}
    \caption{Income and Substitution Effects for Labor Supply}
\end{figure}

In \textit{Figure 5}, Giffen labor supply behavior is shown. This makes sense
intuitively as when you become richer, you want to buy more of everything,
including more time off. For example, if you were working to buy a car and you
were being paid \$10/hr. It would take you 2,000 hours to pay for that car.
However, if you were offered \$20/hr, and the only thing you wanted was that
car, you would only work 1,000 hours to get the car. As a result, the increase
in wages would cause you to work less.

Using the information from \textit{Figure 4} and \textit{Figure 5} labor supply
curves can be generated. In the case of \textit{Figure 4} the curve would be
sloping upward as you choose less leisure when wages increase and therefore more
work. In the case of \textit{Figure 5} the curve would be sloping downward as
you choose more leisure when wages increase and therefore less work.

\subsubsection{Evidence for Labor Supply}

Looking back into history to see when this behavior was exhibited will help
build intuition with the supply curves. Back in the 1970s, men worked and
married women typically didn't work. In that world, what happens when the wages
go up? For married men, there is nothing else to do at home. Therefore, they
have nothing to substitute for because everyone else is working.

\begin{center}
    \begin{tabular}{c c c c}
        & Substitution Effect & Income Effect & Labor Supply Slope \\
        Married Men & Small & Large & Negative \\
        Married Women & Large & Small & Positive 
    \end{tabular}
\end{center}

\end{document}
