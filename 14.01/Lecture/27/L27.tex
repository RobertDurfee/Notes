\documentclass{article}
\usepackage{tikz}
\usepackage{float}
\usepackage{enumerate}
\usepackage{amsmath}
\usepackage{bm}
\usepackage{indentfirst}
\usepackage{siunitx}
\usepackage[utf8]{inputenc}
\usepackage{graphicx}
\graphicspath{ {Images/} }
\usepackage{float}
\usepackage{mhchem}
\usepackage{chemfig}
\allowdisplaybreaks

\title{ 14.01 Lecture 27 }
\author{ Robert Durfee }
\date{ December 13, 2017 }

\begin{document}

\maketitle

\section{ Health Insurance }

Individuals who are risk averse will want insurance. There are two different
people: those who are sick and those who are healthy. When you are sick, you
have much lower consumption. This makes them sad. People will pay a higher
premium while they are healthy to avoid substantially lower consumption while
sick.

There is also the problem of asymmetric information. The insurers have much less
information. They are less willing to supply information and, as a result, there
is a market failure. We can solve this in several ways:

\subsection{Subsidies}

The healthy don't want to buy because it is a bad deal for them. If you
subsidize, you can make it better for them. A subsidy can be supplied to either
side of the market and, in the limit, they are essentially the same.

\subsection{Mandate}

This will force people to buy the insurance. There is no averse selection
problem that is introduced in the market failure.

\subsection{Public Provision}

The is when the government provides the insurance. As a result, everyone signs
up and everyone gets the insurance.

\bigbreak

Originally, nearly 60\% of people had employer supplied insurance. Workplaces
are not completely healthy or sick, so there is a predictable number of sick and
healthy people. This is much better for the insurers. Additionally, the
employers are given a subsidy. Raises are taxed, but increases in health
insurance is not. This makes it desirable to get paid in health insurance.

20\% of people receive health insurance from the government. This comes in two
flavors: \textbf{Medicare}, which is for the elderly, and \textbf{Medicaid},
which is for the poor. Notice, that the 'poor' refers to being poor and
something else.

The remaining 20\% of people are uninsured, for all intents and purposes. 15\%
are classified as uninsured by the government and 5\% are classified as
self-insured. However, this market suffers from drastic averse selection.
Insurance company sets up so that they only insure healthy people. They set up
the market so that they will not cover \textbf{pre existing conditions}. They
also charge higher premiums to those who the companies think are sick. This
doesn't really count because sick people really wouldn't be insured. There are
also limits that companies will place on their packages.

\subsection{Single Payer}

What do we do with the 20\% uninsured? One proposition was \textbf{single payer}
insurance where the government supplies insurance to everyone. This loses the
excess costs of advertising, etc. Also, this is simpler for hospitals and other
agencies. But this never got passed. There are several political barriers.

One is 80\% of people are pretty alright with insurance. If my employer pays
your insurance, this is essentially like an \textbf{implicit tax}. You get paid
less because you get health insurance. If you introduce single payer, everyone
should get a raise. This will happen in a perfectly competitive labor market. If
you pay less, they will go somewhere else. However, the single payer will
increase taxes. If wage and tax increase at the same amount, there is no net
change.

There is also the problem of money in the political system. The private health
insurance market makes billions of dollars. If single payer is introduced, then
they lose all their income. They will throw their money at politicians to make
sure that their market does not go away.

Another problem is the limited supply of doctors. If 20\% more people have
health insurance, then people have to wait longer. Are we willing to make this
trade-off for more lives saved?

\subsection{Free Subsidies}

If we just throw subsidies at insurers to provide insurance to people. This will
still fail. The insurance companies will still discriminate. They won't want to
provide insurance no matter what.

\subsection{Affordable Care Act}

Mitt Romney came up with a way. They wouldn't touch the people who were happy
with the insurance. But, for everyone else, insurance companies would be unable
to discriminate between sick and healthy. The second leg was to mandate that
everyone has to have insurance. But you can't force people to do something they
cannot afford. Lastly, the government would provide subsidies to those who
couldn't afford.

This was a huge success. There was a two thirds majority of favorable sentiment.
The mandate stated that if you didn't get insurance, you would have to pay a tax
penalty. As a result, for those that insurance was too expensive, they would
just pay this penalty. Other states tried to implement this, but failed. They
couldn't afford the subsidies.

When the Massachusetts program was implemented nationwide, in addition to
subsidies, the poorest members received public insurance. This added the public
provision insurance.

\bigbreak

In summary, the ACA was a policy success, but political failure. Everyone has
been healthier in almost every way. Politically, the supporters cannot get
through the fact that the policy is working over those who don't support the
law. The healthier people were more upset because they had to pay slightly more
for insurance to cover the mandate. If there is a partisan program, which
creates losers, there will be a political failure. There will be a loud
opposition created from the losers.

\end{document}

