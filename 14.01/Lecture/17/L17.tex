\documentclass{article}
\usepackage{indentfirst}
\usepackage{graphicx}
\graphicspath{ {Images/} }
\usepackage{float}
\usepackage[utf8]{inputenc}

\title{14.01 Lecture 17}
\author{Robert Durfee}
\date{November 6, 2017}

\begin{document}

\maketitle

\section{Application of Labor Supply Model}

How this matters in the real world with a real world example. Here we will be looking at free trade in Vietnam, specifically child labor. There are significant child abuse labor. They are deprived of their ability to go to school and work at all ages. If we have free trade and more products will come from China and thus, more child labor. But this may not be the case.

\begin{figure}[H]
    \centering
    \includegraphics[scale=0.33]{"Figure 1"}
    \caption{Rice Export Quotas in Vietnam}
\end{figure}

In \textit{Figure 1}, we have the market for rice. Lets say that the demand for rice is $D_{w}$ and the price is $P_{w}$. Lets say that initially, Vietnam only wants rice local. Therefore, they put in a quota where rice can only be supplied to local consumers. This lowers demand and thus lowers price and quantity. We will not look at this as good or bad, just as a shock to the model. 

This restriction was in place, but in the late 1990s, they removed this restriction. What happens when you remove this restriction? People thought that this would be bad for the children. In \textit{Figure 2} this is the market for child labor in Vietnam. We start with $L_{1}$ child workers. Now we can sell to others, the demand shifts out to $D_{2}$ so child worker quantity to $L_{2}$. Now more children have to work. Why is this bad? Now the farm families get richer. They will want more of everything. They will say, "I don't want my children working in the rice fields anymore." Now there is a shift in the supply curve. They will take children to education. The supply curve will shift in to $S_{3}$, there is actually a decrease in child labor. If it only shifts to $S_{2}$, child labor will still increase.

\begin{figure}[H]
    \centering
    \includegraphics[scale=0.33]{"Figure 2"}
    \caption{Child Labor in Vietnam}
\end{figure}

What actually happens? Dartmouth did a study of areas close to ocean (big export) and areas near the interior (little export). Ocean had huge shocks, interior had very little effect. What they found was that child labor went down. Situation 4 in \textit{Figure 2} was what actually happened. This is a great example of counter-intuitive model of economics.

\section{Labor Market Equilibrium}

\begin{figure}[H]
    \centering
    \includegraphics[scale=0.33]{"Figure 3"}
    \caption{Labor Market Equilibrium}
\end{figure}

We wanted to know where $w$ comes from. This comes from labor market equilibrium in \textit{Figure 3}. Demand slope is downward sloping from diminishing marginal revenue of labor. Lets assume income effects dominate substitution. Equilibrium is then formed at point 1. This is where $w$ comes from.

\section{Minimum Wage Example}

This now allows us to study the phenomenon of minimum wage. This is the minimum amount a company has to pay. In a standard model, this has the same effect as a price ceiling. What would happen if a minimum wage was placed below equilibrium (90\%). This would have no effect as it doesn't effect the robust equilibrium.

What if in \textit{Figure 4} there is a minimum wage above equilibrium? Consumer surplus was $A+B+C$. Now,  only $L_{D}$ workers are hired. Now, consumer surplus is $A$. Remember, we are the producers, and the firms are consumers. The workers surplus has increased. $B$ is now completely transported to workers from firms. However, society is now worse off. There is a deadweight loss from trades that aren't being made. Higher wages lead to lower employment.

\begin{figure}[H]
    \centering
    \includegraphics[scale=0.33]{"Figure 4"}
    \caption{Labor Market with Minimum Wage}
\end{figure}

How big is this effect? Studies looked at minimum wage increases in adjacent states. This has shown no effect on employment. How can that be? There are three possible reasons:

\begin{enumerate}
    \item \textbf{Minumum wage is not binding.} 
    
    But this is not true because it must be raised.
    
    \item \textbf{Maybe labor demand is inelastic. }
    
    Shown in \textit{Figure 5}. This means that a firm wants only to hire $L_{1}$ workers and will always want this for any reasonable wage. This is probably unrealistic overall.
    
    \item \textbf{Noncompetitive labor markets.}
    
    This means that labor markets aren't perfectly competitive.
    
\end{enumerate}

\begin{figure}[H]
    \centering
    \includegraphics[scale=0.33]{"Figure 5"}
    \caption{Labor Market with Inelastic Demand}
\end{figure}

What makes a market perfectly competitive? Many sellers and buyers. However, workers may not have complete information about different jobs. Workers do not like to switch jobs, either. This means that firms can have \textbf{monopsony} where they can have market power over their workers. This means a firm can set the wages for labor.

Firms make some surplus from their workers through monopsony. This allows firms to pay workers less than their marginal revenue product:

$$w<MRP_{L}$$

This is parallel to the price being greater than marginal cost:

$$p>MC$$

If the minimum wage is less than the wage, no difference. If the minimum wage is higher than the marginal revenue product, then we go to a classic situation. If the minimum wage is in the range between wage and marginal revenue product, the deadweight loss can be reduced. This government can be welfare improving. This is the same analysis as with monopolies.

Is the minimum wage within this range? These past studies show that the minimum wage is in this range. However, Seattle doubled the minimum wage. Studies showed that there was a decline in labor. Workers are getting higher wages, but there are fewer workers. Is this better? We need to establish a framework in terms of equity or fairness.

\subsection{Company Towns}

There would be these huge towns isolated and you'd have to move and live there. They had you and could exploit you. If you get lower wage, you wouldn't travel hundreds of miles to go to a different town. As a result, monopsony power is very realistic

\section{Capital Input Markets}

What is capital? Do not think about what capital is, but where is comes from. Labor comes from taking leisure away from people. Capital comes from sacrificing consumption. This is the \textbf{diversion of current consumption towards future consumption.} Now we are talking about time. The stuff firms do and where do they get the money to do it.

When you are a farmer and harvest grain, you can eat all now or use some to replant next year. This is done by diverting current consumption. This allows for next years' consumption. With market economy, it isn't this direct. Our savings become capital. How does savings become capital? Through the capital market. Just as decisions on how long to work determines labor, household savings determine how much capital.

\begin{figure}[H]
    \centering
    \includegraphics[scale=0.33]{"Figure 6"}
    \caption{Equilibrium in Capital Markets}
\end{figure}

In \textit{Figure 6}, we have the capital equilibrium. The demand for capital is a firm's demand for capital. This comes from setting the marginal revenue of capital equal to the interest rate.

$$K^{*}=i^{*}$$

The demand is downward sloping because each machine is less productive. Where does supply curve come from? The more people are paid to save, the more they will save.

Capital supply, how much consumption am I willing to give up to ensure I have consumption in the future. Savings is a bad. This means you can't have stuff today. But we can't model bads. Therefore we model \textbf{second period consumption}. Today you can consume $C_{1}$ in the future you can consume $Y-C_{1}$.

\begin{figure}[H]
    \centering
    \includegraphics[scale=0.33]{"Figure 7"}
    \caption{Intertemporal Substitution}
\end{figure}

In \textit{Figure 7}, we model intertemporal substitution. If I save and receive 2\% interest, next year I have $1+2\%$ or $1+i$. This is the slope of the budget constraint. The intercepts are determined by income.

The price of consuming today is the interest rate. The opportunity cost of spending today is the ability to spend more tomorrow.

\subsection{Example}

I earn \$80,000. Next year I have an unpaid sybatical. Intercept on x-axis would mean I have consumed all and saved none. Intercept on the y-axis, I don't spend anything and next year I will have \$88,000. There should be an interior solution. How do I decide? There is a utility function. This utility function will intercept at tangency at optimal bundle of savings and consumption.


\subsection{Change in Interest}

The interest rate doubles in \textit{Figure 8}. What does this do to my consumption and savings. We don't model savings, we model second period consumption where savings can be derived later. Now there is a higher opportunity cost in consuming in period one. Therefore, I will shift towards more consumption period two through the substitution effect. However, I am now richer and want even more and this include consumption in period one.

\begin{figure}[H]
    \centering
    \includegraphics[scale=0.33]{"Figure 8"}
    \caption{Intertemporal Substitution Through Increased Interest}
\end{figure}

Sometimes the income effect can offset the substitution and result in downward sloping supply.

\end{document}
