\documentclass{article}
\usepackage{tikz}
\usepackage{float}
\usepackage{enumerate}
\usepackage{amsmath}
\usepackage{bm}
\usepackage{indentfirst}
\usepackage{siunitx}
\usepackage[utf8]{inputenc}
\usepackage{graphicx}
\graphicspath{ {Images/} }
\usepackage{float}
\usepackage{mhchem}
\usepackage{chemfig}
\allowdisplaybreaks

\title{ 14.01 Lecture 25 }
\author{ Robert Durfee }
\date{ December 6, 2017 }

\begin{document}

\maketitle

\section{ Why Social Insurance? }

This is one of the single biggest things that the United States does. This
provides security over risks because people do not like risk. This is a big
business in the US. Despite this huge business, the private market leaves us
under-insured. This comes from \textbf{information asymmetry}. This is a type of
market failure that arises from the fact that one party may know more than the
other. 

Looking at the market for used cars. There was no way to know the car's history.
You were dealing with an uncertain deal. You would be worried about getting a
lemon. The fact that someone is willing to sell something, means there must be
something wrong with it. Some people may not deal with this market simply
because of this risk. 

If I have a car in perfect condition for \$5000. Someone is willing to pay
\$6000. But, the average amount of cars need \$2000 in repairs. So, he is only
willing to pay \$4000. The trade doesn't happen because there is impefect
information sharing because I cannot convince the guy my car is in perfect
condition.

In this case, the seller has more information than the buyer. In the insurance
market, the buyer has more information than the seller. If you want to buy
health insurance, the insurance companies are skeptical.

Imagine you want to supply health insurance for recent MIT grads. Suppose there
are 100 grads that you are looking to insure. In an average year, 90 are healthy
and 10 are sick. The healthy guys have a 90\% chance of \$0 and a 10\% of
\$10,000. The expected cost is thus \$1,000. For the sick people, there is a
50\% chance of \$0 and a 50\% chance of \$10,000. The expected cost is thus
\$5,000. 

If everyone buys health insurance, the expected value of payout is $ 0.90 \cdot
1,000 + 0.10 \cdot 5,000 = 1,400$. The insurance company will charge \$1,500 per
person and make a \$100 profit. This is wrong because we assumed that everyone
has to buy insurance. for the sick people, this is a great deal. They only pay
\$1,500 when it would've cost them \$5,000. But for the healthy, this is not a
good deal.

Assume that only half of the sick people buy the insurance (given from their
risk behavior). Now the payout is $ 10 \cdot 5,000 + 45 \cdot 1000 = 95,000 $
This is losing money. This is the effect of \textbf{adverse selection}. Since
the company loses money, they won't sell to anyone. This prevents people who
this policy would've helped from getting the policy. The problem arises from
charging one price to two different types of goods. 

\subsection{ Subsidies }

Any MIT grad who buys insurance gets a \$500 tax credit. Then, everyone buys the
insurance. This solves the problem. However, this cost money and raises taxes
which leads to inefficiency. 

\subsection{ Mandate }

Everyone has to have health insurance. Everyone will buy the insurance and the
insurance company makes money. THis solves the problem without causing an
increase in money. This works by making those who don't want to buy have to buy.
The burden switches from taxpayers to those who wouldn't've bought insurance.
Worker's compensation is mandated by US.

\subsection{ Better Information }

If everyone knew what was going to go wrong, would insurance be sold? No, there
would be no insurance sold. Think of genetic testing providing total
information.

\subsection{ Provide Insurance }

This removes adverse selection because everyone gets insurance for free. But
this is an even higher cost. Traditional methods utilize this for disability,
etc. The Affordable Care Act provides the poor tax credits for buying insurance.

\section{ Social Insurance Tradeoff }

The provides \textbf{consumption smoothing} which pools people together to deal
with extreme cases (no one starves). But this introduces \textbf{moral hazard}
which says if you insure people for adverse behavior, they may gravitate toward
that adverse behavior. 

Looking at \textbf{worker's compensation}. If you get hurt at work, you get a
payment which pays about two thirds of what you would've earned. There is an
adverse selection problem because the employer will be skepitcal of why you
insist you need worker's comp. 

It is also difficult to determine whether or not there is true injury. People
will pretend to just get worker's compensation. This does happen in practice. A
prison guard claimed he couldn't work, but there are videos of him teach karate
with a "bad back". Different states have different worker's comp benefits. if a
state raises worker's comp, there are more injuries in those states. 

\subsection{ Efficiency Cost }

You work the amount such that wage is equal to marignal value of leisure. When
worker's comp is introduced, there is marginal value of leisure and worker's
compensation. There is now a lower value of marginal leisure. So more leisure is
taken and less labor is supplied to the market. 

\subsection{ Taxes }

The more people that take worker's comp, there is higher taxes payed. This will
lead other's to work less hard because of increased taxes. 

\section{ Social Security }

The largest social insurance in the United States. One of the most significant
risks you want to insure is the risk you will not be able to work when you get
older. When you retire, you get money and when you die, your family gets money.
This is a \$400 billion a year program. When you turn 62, you are eligible to
get social security pays half of what you earned before you retire. But it is
progressive (Bill Gates gets nothing, poor get more than what they worked). 

The moral hazard is inducing people to retire. It turns out that both social
security is benefiting people and causing earlier retirement. SS has decreased
the poverty rate in the elderly. 

There is also evidence that people are leaving the workforce early. In the
Netherlands, the SS replaces 90\% of what you earn. There is no actuarial
adjustment to make you indifferent between retiring today and tomorrow. At your
retirement age, you get the full amount. If you wait, you just get one more
check. If you work, you lose money to pay the tax. This lead to no one working
over the retirement age. This is a very strong moral hazard effect. 
the workforce early.
\end{document}

