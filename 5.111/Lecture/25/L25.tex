\documentclass{article}
\usepackage{tikz}
\usepackage{float}
\usepackage{enumerate}
\usepackage{amsmath}
\usepackage{bm}
\usepackage{indentfirst}
\usepackage{siunitx}
\usepackage[utf8]{inputenc}
\usepackage{graphicx}
\graphicspath{ {Images/} }
\usepackage{float}
\usepackage{mhchem}
\usepackage{chemfig}
\allowdisplaybreaks

\title{ 5.111 Lecture 25 }
\author{ Robert Durfee }
\date{ November 13, 2017 }

\begin{document}

\maketitle

\section{ Oxidation Reduction Reactions }

These reactions are all about electron transfer. Examples include combustion
reactions with metals, such as in fireworks. 
$$\ce{2Mg + O_2 <=> 2MgO}$$

In this reaction, magnesium is giving electrons to oxygen. Therefore, magnesium
is oxidized. Oxygen has gained electrons and therefore is reduced.

We can determine what is oxidied or reduced by using \textbf{oxidation numbers}.
In natural states, compounds have 0 oxidation number. In the compound, magnesium
has a positive oxidation number and oxygen has a negative oxidation number.

\textbf{Oxidation}: loss of electrons, oxidation number increases, reducing
agent. \textbf{Reduction}: gain of electrons, oxidation number decreases,
oxidizing agent.

Note that you do not need oxygen for all oxidation reactions or reduction
reactions. 
$$\ce{Mg + Cl_2 <=> MgCl_2}$$

This is an example of another reation that doesn't use oxygen.

\subsection{Assigning Oxidation Numbers}

Free elements have an oxidation number of zero. Ions composed of only one atoms
have an oxidation number of the ion. Group 1 has +1, group 2 has +2, and
aluminum has an oxidation number of +3. These are universal rules. 

Oxygen in most reactions has an oxidation number of -2. In peroxides, oxygen has
-1. In superoxides, oxygen has -1/2. and in ozonides, oxygen has -1/3.

Hydrogen is always +1. This makes it very stable in all compounds.

Flourine is also always -1. However, the other elements in the period can have
positive or negative oxidation numbers, depending on the compound.

The sum of all oxidation numbers in an ion should equal the net charge. If the
compound is neutral, then the sum should also be zero.

With $\ce{NH_4+}$, hydrogen has +1, and nitrogen has -3.

\subsection{Identifying Reducing and Oxidizing Agents}

Oxidizing agents are reduced and reducing agents are oxidized.
$$\ce{Mg + 2HCl <=> H_2 + MgCl_2}$$

In this reaction, magnesium is the reducing agent because it is oxidized.
$\ce{HCl}$ is then the oxidizing agent because it is reduced.

\subsection{Disproportionation Reaction}

A reactant element in one oxidation state is both oxidized and reduced.
$$\ce{2H_2O_2 <=> 2H_2O + O_2}$$

This is the thermal decomposition of hydrogen peroxide. In this case, oxygen
goes from -1 to -2 and 0. As a result, hydrogen peroxide is both oxidized and
reduced.

\subsection{Balancing Redox Reactions}
$$\ce{Fe^{2+} + Cr_2O_7^{2-} <=> Cr^{3+} + Fe^{3+}}$$

This reaction is not balanced. \textbf{You are free to add water, $\ce{OH-}$,
and $\ce{H+}$}. Look at the half-reactions. Balance half-reactions separately.
balanced all elements in the half-reaction except oxygen and hydrogen. 
$$\ce{Cr_2O_7^{2-} <=> Cr^{3+}}$$

This reaction is reduced. Add water to balanced oxygen and hydrogen to the left
to balanced the water added. Now add electrons to balanced out.
$$\ce{Fe^{2+} <=> Fe^{3+}}$$

Multiply this half-reaction until it cancels the electrons added.
$$\ce{14H+ + Cr_2O_7^{2-} + 6Fe^{2+} <=> 2Cr^{3+} + 6Fe^{3+} + 7H_2O}$$

This is how to balance in an acidic situtation, in an abundance of $\ce{H+}$.

\end{document}

