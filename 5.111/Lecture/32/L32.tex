\documentclass{article}
\usepackage{tikz}
\usepackage{float}
\usepackage{enumerate}
\usepackage{amsmath}
\usepackage{bm}
\usepackage{indentfirst}
\usepackage{siunitx}
\usepackage[utf8]{inputenc}
\usepackage{graphicx}
\graphicspath{ {Images/} }
\usepackage{float}
\usepackage{mhchem}
\usepackage{chemfig}
\allowdisplaybreaks

\title{ 5.111 Lecture 32 }
\author{ Robert Durfee }
\date{ December 4, 2017 }

\begin{document}

\maketitle

\section{ Chemical Kinetics }

When considering a chemical reaction, we must consider if a reaction occurs
spontaneously and then we must consider how quickly it will proceed. The first a
thermodynamics question and the second is a \textbf{kinetics} question. 

When referring to \textbf{stability}, we refer to the spontaneous tendency.
\textbf{Labile} refers to the rate at which this tendency is realized. A
chemical kinetics experiment measures the rate at which the concentration of a
substance taking part in a chemical reaction changes with time. The reaction
rate is sensitive to the initial concentration and temperature. 

The average reaction rate is:
$$ \frac{ \rm{ \Delta Concentration } }{ \rm{ \Delta Time } } $$

The instantaneous reaction rate is:
$$ \lim_{\Delta t \rightarrow 0} \frac{ \Delta \rm{ Concentration } }{ \Delta t }$$

The initial rate is the instantaneous rate when $t=0$. When we rate "rate" in
problem set and textbooks, we typically refer to the instantaneous rate, not the
average. For \textbf{rate expression}, the reactants have negative rates and the
products have positive rates. 
$$ \ce{ aA + bB -> cC + dD } $$

Assuming no intermediate species and the concentrations are independent of time:
$$ -\frac{ 1 }{ a }\frac{ dA }{ dt } = -\frac{ 1 }{ b }\frac{ dB }{ dt } =
\frac{ 1 }{ c }\frac{ dC }{ dt } = \frac{ 1 }{ d }\frac{ dD }{ dt } $$

\section{ Rate Laws }

The instantaneous rate is related to the concentration at that time by a
proportionality constant, $k$, called the \textbf{rate constant}. 
$$ k [ \ce{ A } ]^{m}[ \ce{ B } ]^{n} $$

Where $m$ and $n$ are the rate order's of A and B. This is for the initial rate,
far from equilibrium. For closer to equilibrium, the net rate is the difference
between the forward and reverse reaction rates.

Rate law is only determined through experimentation. Also, the rate order is not
by default the coefficient of the chemical equation. Additionally, $m$ and $n$
can be more than just integers, they can be fractions, etc. 

\subsection{ First Order }

$$ k[ \ce{ A } ] $$

This means that double the concentration means double the rate.

\subsection{ Second Order }

$$ k[ \ce{ A } ]^{2} $$

This means that double the concentration means four times the rate.

\subsection{ Inverse Order }

$$ k[ \ce{ A } ]^{-1} $$

This means double the concentation means half the rate.

\subsection{ Half Order }

$$ k[ \ce{ A } ]^{1/2} $$

This means double the concentration means 1.4 times the rate

\subsection{ Zero Order}

$$ k[ \ce{ A } ]^{0} $$

This means that the concentration is independent of the reaction rate.

\section{ Rate Laws }

The overall reaction rate is the sume of all the orders of individual
components. Also, the units for $k$ vary depending on the number of species and
their orders. The end result must be concentration over time.

\section{ Integrated Rate Law }

You can separate the time and concentration terms and then integrate. This will
show the concentration after a certain amount of time given the initial
concentration. for first order:
$$ \ln[ \ce{ A } ]_{t} = -kt + \ln[ \ce{ A } ]_{0} $$
$$ [ \ce{ A } ]_{t} = [ \ce{ A } ]_{0} \cdot e^{-kt} $$

\section{ Half-Life of First-Order }

\textbf{Half-life}  is the time it takes for the original concentration to be
reduced by half ($t_{1/2}$).
$$ t_{1/2} = \frac{ \ln( 2 ) }{ k } $$

Importantly, this does not depend on the initial concentration of the solution.

\end{document}

