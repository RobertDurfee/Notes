\documentclass{article}
\usepackage{tikz}
\usepackage{float}
\usepackage{enumerate}
\usepackage{amsmath}
\usepackage{bm}
\usepackage{indentfirst}
\usepackage{siunitx}
\usepackage[utf8]{inputenc}
\usepackage{graphicx}
\graphicspath{ {Images/} }
\usepackage{float}
\usepackage{mhchem}
\usepackage{chemfig}
\allowdisplaybreaks

\title{ 5.111 Lecture 34 }
\author{ Robert Durfee }
\date{ December 8, 2017 }

\begin{document}

\maketitle

\section{ Effects of Temperature }

For the gas-phase, a qualitative observation is that reaction rates tend to
increase with increased temperature. Quantitatively, we used the Arrhenius
equation. This equation relates the temperature and the activation energy.
$$ k = Ae^{- E_{a} / RT } $$

This shows that rate constants vary exponentially with the inverse of
temperature. What is the meaning of the pre-exponential constant $A$? Well, this
factor is not dependent upon the temperature. The activation energy, $E_{a}$, is
largely temperature independent. This is because the slope of the Arrhenius
equation is quite constant. The physical meaning of $E_{a}$ is the sensitivity
of rates to temperature. The steeper the slope, the higher the activation
energy. 

$$ \ln \frac{ k_{2} }{ k_{1} } = -\frac{ E_{a} }{ R } \left( \frac{ 1 }{ T_{2} }
- \frac{ 1 }{ T_{1} }\right) $$

\section{ Reaction Coordinate and Activated Complex }

Two molecules collide to form a product but not every two molecules that collide
will form a product. Only when the collision energy is greater than some
critical energy (activation energy) will a reaction result. This activation
energy is required because as molecules approach, their potential energy
increases as the bonds distort. The encounter results in the formation of an
activated complex or \textbf{transition state} transition state which is a
combination of molecules that can either go onto form products or fall apart
again into unchanged reactants. 

At lower temperature, only a small amount of molecules will have enough kinetic
energy. At higher temperatures, there will be a larger fraction that will have
the necessary kinetic energy. This is shown with the \textbf{Maxwell-Boltzmann
distribution}. 

\section{ Elementary Reactions }

The reaction barrier is always positive. Therefore, the temperature increases
the rate of an elementary reaction. For overall reactions, increasing
temperature can decrease or increase the overall rate. This is caused by the
equilibrium constant which either increases or decreases with increase in
temperature depending on whether a reaction is exothermic or endothermic.

When $E_{a}$ is a small positive number, the rate constant increases only a
little with T. When $\Delta H$ is a large negative number, the equilibrium
constant will decrease a lot with T. As a result, the overall reaction will slow
down with increased temperature.

\end{document}

