\documentclass{article}
\usepackage{tikz}
\usepackage{float}
\usepackage{enumerate}
\usepackage{amsmath}
\usepackage{bm}
\usepackage{indentfirst}
\usepackage{siunitx}
\usepackage[utf8]{inputenc}
\usepackage{graphicx}
\graphicspath{ {Images/} }
\usepackage{float}
\usepackage{mhchem}
\usepackage{chemfig}
\allowdisplaybreaks

\title{ 5.111 Lecture 30 }
\author{ Robert Durfee }
\date{ November 27, 2017 }

\begin{document}

\maketitle

\section{ Crystal Field Theory }

In addition to the geometry of ligands, ligands also change the d-orbital
energies of the metal. The basic idea is that and the energy levels of the
metal's d-orbitals are altered from those in the free metal ions.
\textbf{Crystal field theory} describes the metal-ligand bond is ionic.
\textbf{Ligand field theory} uses covalent and ionic descriptions. This won't be
used in this class as much.

This theory considers ligands as negative point charges and repulsion between
hte negative point change and the d-orbitals. This will shift with the orbital
energy levels. You must combine the geometrical coordination complexes as well
as the d-orbital shapes. 

\subsection{ Octahedral Complex }

Ligands are point right at $d_{x^{2}}$ and $d_{x^{2}-y^{2}}$ and thus there is
large repulsion. These orbitals are \textbf{destabilized} more than $d_{xy}$,
$d_{yz}$, and $d_{xz}$. Thus these are \textbf{stabilized}. In each category,
destabilized and stabilized, they are all \textbf{degenerate} with respect to
each other. 

The degenerate destabilized orbitals have higher energy and the degenerate
stabilied orbitals have lower energy. They are separated by $\Delta_{O}$. The
size of $\Delta_{O}$ is determined by the \textbf{nature} of the ligand. 

\section{ Spectrochemical Series }

The relative abilities of common ligands to split the d-orbiatl energy levels
generate the \textbf{spectrochemical series}. Strong field ligands produce large
seperate between the d-orbitals and weak field ligands produce small speration
between the d-orbitals. Weak field ligands include $\ce{ I- }, \ce{ Br- }, \ce{
Cl- }, \ce{ F- }, \ce{ OH- }, \ce{ H_{2}O}$. Strong field ligands include $\ce{
NH_{3} }, \ce{ CO }, \ce{ CN- }$. In order of increasing strength. 

To draw the field splitting diagrams, first determine the oxidation number of
the center, then the d-count, and then draw the diagram by placing the
electrons. You can place either by singly filling before pairing, or pairing
first. This is decided based on whether $\Delta_{O}$ is greater or less than the
pairing energy: the energy of electron-electron repulsion. When $\Delta_{O}$ is
small, place singly first. This results in high spin electrons. When
$\Delta_{O}$ is large, place pairing first. This results in low spin electrons.

\subsection{ Light Absorbtion }

A substance absorbs photons if the nergies of the phonots match the energies
required to excite the electrons to higher energy levels. If low frequency is
absorbed, the wavelength is long. If high frequency is absorbed, the wavelength
is short. The color of transmitted light is \textbf{complementary} to the color
of absorbed light. 

Some coordination complexes are colorless when the d-d transitions are not in
the visible range. For example, this includes zinc and cadmium. 

\end{document}

