\documentclass{article}
\usepackage{caption}
\usepackage{subcaption}
\usepackage{tikz}
\usepackage{float}
\usepackage{enumerate}
\usepackage{amsmath}
\usepackage{bm}
\usepackage{indentfirst}
\usepackage{siunitx}
\usepackage[utf8]{inputenc}
\usepackage{graphicx}
\graphicspath{ {Images/} }
\usepackage{float}
\usepackage{mhchem}
\usepackage{chemfig}
\usepackage{modiagram}
\allowdisplaybreaks

\title{ 5.111 Problem Set 10 }
\author{ Robert Durfee }
\date{ December 8, 2017 }

\begin{document}

\maketitle

\section*{ Transition Metals and Coordination Compounds }

\begin{enumerate}[1.]
    \item 
        \begin{enumerate}[a.]
            \item The bond angles for square planar molecules are $90^{\circ}$.
                
                $$ \chemfig{ Pt( <:[ :25 ] NH_{3} ) ( <:[ :155 ] Cl ) ( <[ :-155
                ] Cl ) ( <[ :-25 ] NH_{3} ) } $$

            \item Cisplatin crystal field energy level diagram: 
                \begin{center}
                \begin{MOdiagram}[labels-fs = \tiny]
                    \AO{s}{1.75; pair}
                    \AO(20pt){s}{1.75; pair}
                    \AO(40pt){s}{1.75; pair}
                    \AO(60pt){s}{1.75; up}
                    \AO(80pt){s}{1.75; up}
                    \AO(110pt){s}[label={$\mathrm{d_{xz}}$}]{0}
                    \AO(130pt){s}[label={$\mathrm{d_{yz}}$}]{0}
                    \AO(120pt){s}[label={$\mathrm{d_{z^2}}$}]{0.75}
                    \AO(120pt){s}[label={$\mathrm{d_{xy}}$}]{2.5; up}
                    \AO(120pt){s}[label={$\mathrm{d_{x^2 - y^2}}$}]{4; up}
                    \connect{AO5 & AO6, AO5 & AO8, AO5 & AO9, AO5 & AO10}
                    \draw[->] (-0.5,0) -- (-0.5,4) node[pos=0.75, anchor=south
                    west]{$E$};
                \end{MOdiagram}
                \end{center}

            \item There are unpaired electrons. As a result, this will be
                paramagnetic. 
        \end{enumerate}
    \item
        \begin{enumerate}[a.]
            \item $\ce{ Co^{3+} }$
                \begin{figure}[H]
                    \centering
                    \begin{subfigure}{.5\textwidth}
                        \centering
                        \begin{MOdiagram}[labels-fs = \tiny]
                            \AO{s}{1; pair}
                            \AO(20pt){s}{1; up}
                            \AO(40pt){s}{1; up}
                            \AO(60pt){s}{1; up}
                            \AO(80pt){s}{1; up}
                            \AO(110pt){s}[label={$\mathrm{d_{xy}}$}]{0; pair}
                            \AO(130pt){s}[label={$\mathrm{d_{xz}}$}]{0; pair}
                            \AO(150pt){s}[label={$\mathrm{d_{yz}}$}]{0; pair}
                            \AO(120pt){s}[label={$\mathrm{d_{z^2}}$}]{2;}
                            \AO(140pt){s}[label={$\mathrm{d_{x^2 - y^2}}$}]{2;}
                            \connect{AO5 & AO6, AO5 & AO9}
                            \connect{AO5 & AO6, AO5 & AO9}
                            \draw [->] (-0.5,0) -- (-0.5,2) node[pos=0.75,
                            anchor=south west]{$E$};
                        \end{MOdiagram}
                        \caption{Strong (Low Spin)}
                    \end{subfigure}%
                    \begin{subfigure}{.5\textwidth}
                        \centering
                        \begin{MOdiagram}[labels-fs = \tiny]
                            \AO{s}{1; pair}
                            \AO(20pt){s}{1; up}
                            \AO(40pt){s}{1; up}
                            \AO(60pt){s}{1; up}
                            \AO(80pt){s}{1; up}
                            \AO(110pt){s}[label={$\mathrm{d_{xy}}$}]{0; pair}
                            \AO(130pt){s}[label={$\mathrm{d_{xz}}$}]{0; up}
                            \AO(150pt){s}[label={$\mathrm{d_{yz}}$}]{0; up}
                            \AO(120pt){s}[label={$\mathrm{d_{z^2}}$}]{2; up}
                            \AO(140pt){s}[label={$\mathrm{d_{x^2 - y^2}}$}]{2; up}
                            \connect{AO5 & AO6, AO5 & AO9}
                            \draw (AO8) node[anchor=west, xshift=10]{$\mathrm{t_{2g}}$};
                            \draw (AO10) node[anchor=west, xshift=10]{$\mathrm{e_{g}}$};
                        \end{MOdiagram}                        
                        \caption{Weak (High Spin)}
                    \end{subfigure}%
                \end{figure}
            \item $\ce{ Zn^{2+} }$
                \begin{figure}[H]
                    \centering
                    \begin{subfigure}{.5\textwidth}
                        \centering
                        \begin{MOdiagram}[labels-fs = \tiny]
                            \AO{s}{1; pair}
                            \AO(20pt){s}{1; pair}
                            \AO(40pt){s}{1; pair}
                            \AO(60pt){s}{1; pair}
                            \AO(80pt){s}{1; up}
                            \AO(110pt){s}[label={$\mathrm{d_{xy}}$}]{0; pair}
                            \AO(130pt){s}[label={$\mathrm{d_{xz}}$}]{0; pair}
                            \AO(150pt){s}[label={$\mathrm{d_{yz}}$}]{0; pair}
                            \AO(120pt){s}[label={$\mathrm{d_{z^2}}$}]{2; pair}
                            \AO(140pt){s}[label={$\mathrm{d_{x^2 - y^2}}$}]{2; up}
                            \connect{AO5 & AO6, AO5 & AO9}
                            \draw [->] (-0.5,0) -- (-0.5,2) node[pos=0.75,
                            anchor=south west]{$E$};
                        \end{MOdiagram}                        
                        \caption{Strong}
                    \end{subfigure}%
                    \begin{subfigure}{.5\textwidth}
                        \centering
                        \begin{MOdiagram}[labels-fs = \tiny]
                            \AO{s}{1; pair}
                            \AO(20pt){s}{1; pair}
                            \AO(40pt){s}{1; pair}
                            \AO(60pt){s}{1; pair}
                            \AO(80pt){s}{1; up}
                            \AO(110pt){s}[label={$\mathrm{d_{xy}}$}]{0; pair}
                            \AO(130pt){s}[label={$\mathrm{d_{xz}}$}]{0; pair}
                            \AO(150pt){s}[label={$\mathrm{d_{yz}}$}]{0; pair}
                            \AO(120pt){s}[label={$\mathrm{d_{z^2}}$}]{2; pair}
                            \AO(140pt){s}[label={$\mathrm{d_{x^2 - y^2}}$}]{2; up}
                            \connect{AO5 & AO6, AO5 & AO9}
                            \draw (AO8) node[anchor=west, xshift=10]{$\mathrm{t_{2g}}$};
                            \draw (AO10) node[anchor=west, xshift=10]{$\mathrm{e_{g}}$};
                        \end{MOdiagram}                        
                        \caption{Weak}
                    \end{subfigure}%
                \end{figure}
            \item $\ce{ Nb^{3+} }$
                \begin{figure}[H]
                    \centering
                    \begin{subfigure}{.5\textwidth}
                        \centering
                        \begin{MOdiagram}[labels-fs = \tiny]
                            \AO{s}{1; up}
                            \AO(20pt){s}{1; up}
                            \AO(40pt){s}{1; }
                            \AO(60pt){s}{1; }
                            \AO(80pt){s}{1; }
                            \AO(110pt){s}[label={$\mathrm{d_{xy}}$}]{0; up}
                            \AO(130pt){s}[label={$\mathrm{d_{xz}}$}]{0; up}
                            \AO(150pt){s}[label={$\mathrm{d_{yz}}$}]{0; }
                            \AO(120pt){s}[label={$\mathrm{d_{z^2}}$}]{2; }
                            \AO(140pt){s}[label={$\mathrm{d_{x^2 - y^2}}$}]{2; }
                            \connect{AO5 & AO6, AO5 & AO9}
                            \draw [->] (-0.5,0) -- (-0.5,2) node[pos=0.75,
                            anchor=south west]{$E$};
                        \end{MOdiagram}                        
                        \caption{Strong}
                    \end{subfigure}%
                    \begin{subfigure}{.5\textwidth}
                        \centering
                        \begin{MOdiagram}[labels-fs = \tiny]
                            \AO{s}{1; up}
                            \AO(20pt){s}{1; up}
                            \AO(40pt){s}{1; }
                            \AO(60pt){s}{1; }
                            \AO(80pt){s}{1; }
                            \AO(110pt){s}[label={$\mathrm{d_{xy}}$}]{0; up}
                            \AO(130pt){s}[label={$\mathrm{d_{xz}}$}]{0; up}
                            \AO(150pt){s}[label={$\mathrm{d_{yz}}$}]{0; }
                            \AO(120pt){s}[label={$\mathrm{d_{z^2}}$}]{2; }
                            \AO(140pt){s}[label={$\mathrm{d_{x^2 - y^2}}$}]{2; }
                            \connect{AO5 & AO6, AO5 & AO9}
                            \draw (AO8) node[anchor=west, xshift=10]{$\mathrm{t_{2g}}$};
                            \draw (AO10) node[anchor=west, xshift=10]{$\mathrm{e_{g}}$};
                        \end{MOdiagram}                        
                        \caption{Weak}
                    \end{subfigure}%
                \end{figure}

        \end{enumerate}
    \item 
        \begin{enumerate}[a.]
            \item $\ce{ [ Mn( CO)_{6} ]Cl_{3} }$ (two unpaired electrons)
                \begin{center}
                    \begin{MOdiagram}[labels-fs = \tiny]
                        \AO{s}{1; up}
                        \AO(20pt){s}{1; up}
                        \AO(40pt){s}{1; up}
                        \AO(60pt){s}{1; up}
                        \AO(80pt){s}{1; }
                        \AO(110pt){s}[label={$\mathrm{d_{xy}}$}]{0; pair}
                        \AO(130pt){s}[label={$\mathrm{d_{xz}}$}]{0; up}
                        \AO(150pt){s}[label={$\mathrm{d_{yz}}$}]{0; up}
                        \AO(120pt){s}[label={$\mathrm{d_{z^2}}$}]{2; }
                        \AO(140pt){s}[label={$\mathrm{d_{x^2 - y^2}}$}]{2; }
                        \connect{AO5 & AO6, AO5 & AO9}
                        \draw (AO8) node[anchor=west, xshift=10]{$\mathrm{t_{2g}}$};
                        \draw (AO10) node[anchor=west, xshift=10]{$\mathrm{e_{g}}$};
                        \draw [->] (-0.5,0) -- (-0.5,2) node[pos=0.75,
                        anchor=south west]{$E$};
                    \end{MOdiagram}
                \end{center}
            \item $\ce{ [ Ni( OH_{2} )_{4} ]^{2+} }$ (two upaired electrons)
                \begin{center}
                    \begin{MOdiagram}[labels-fs = \tiny]
                        \AO{s}{1; pair}
                        \AO(20pt){s}{1; pair}
                        \AO(40pt){s}{1; pair}
                        \AO(60pt){s}{1; up}
                        \AO(80pt){s}{1; up}
                        \AO(110pt){s}[label={$\mathrm{d_{xy}}$}]{2; pair}
                        \AO(130pt){s}[label={$\mathrm{d_{xz}}$}]{2; up}
                        \AO(150pt){s}[label={$\mathrm{d_{yz}}$}]{2; up}
                        \AO(120pt){s}[label={$\mathrm{d_{z^2}}$}]{0; pair}
                        \AO(140pt){s}[label={$\mathrm{d_{x^2 - y^2}}$}]{0; pair}
                        \connect{AO5 & AO6, AO5 & AO9}
                        \draw (AO8) node[anchor=west,
                        xshift=10]{$\mathrm{t_{2}}$};
                        \draw (AO10) node[anchor=west, xshift=10]{$\mathrm{e}$};
                        \draw [->] (-0.5,0) -- (-0.5,2) node[pos=0.75,
                        anchor=south west]{$E$};
                    \end{MOdiagram}
                \end{center}

                \bigbreak
                \bigbreak
                \bigbreak
                \bigbreak
                
            \item $\ce{ [ VCl_{6} ]^{3-} }$ (two unpaired electrons)
                \begin{center}
                    \begin{MOdiagram}[labels-fs = \tiny]
                        \AO{s}{1; up}
                        \AO(20pt){s}{1; up}
                        \AO(40pt){s}{1; }
                        \AO(60pt){s}{1; }
                        \AO(80pt){s}{1; }
                        \AO(110pt){s}[label={$\mathrm{d_{xy}}$}]{0; up}
                        \AO(130pt){s}[label={$\mathrm{d_{xz}}$}]{0; up}
                        \AO(150pt){s}[label={$\mathrm{d_{yz}}$}]{0; }
                        \AO(120pt){s}[label={$\mathrm{d_{z^2}}$}]{2; }
                        \AO(140pt){s}[label={$\mathrm{d_{x^2 - y^2}}$}]{2; }
                        \connect{AO5 & AO6, AO5 & AO9}
                        \draw (AO8) node[anchor=west, xshift=10]{$\mathrm{t_{2g}}$};
                        \draw (AO10) node[anchor=west, xshift=10]{$\mathrm{e_{g}}$};
                        \draw [->] (-0.5,0) -- (-0.5,2) node[pos=0.75,
                        anchor=south west]{$E$};
                    \end{MOdiagram}
                \end{center}
       \end{enumerate}
    \item For the first complex, $\ce{ CN- }$ is a strong ligand which results
        in the following crystal field splitting diagrams:

        \begin{figure}[H]
            \centering
            \begin{subfigure}{.5\textwidth}
                \centering
                \begin{MOdiagram}[labels-fs = \tiny]
                    \AO{s}{1; pair}
                    \AO(20pt){s}{1; pair}
                    \AO(40pt){s}{1; up}
                    \AO(60pt){s}{1; up}
                    \AO(80pt){s}{1; up}
                    \AO(110pt){s}[label={$\mathrm{d_{xy}}$}]{0; pair}
                    \AO(130pt){s}[label={$\mathrm{d_{xz}}$}]{0; pair}
                    \AO(150pt){s}[label={$\mathrm{d_{yz}}$}]{0; pair}
                    \AO(120pt){s}[label={$\mathrm{d_{z^2}}$}]{2; up}
                    \AO(140pt){s}[label={$\mathrm{d_{x^2 - y^2}}$}]{2; }
                    \connect{AO5 & AO6, AO5 & AO9}
                    \draw [->] (-0.5,0) -- (-0.5,2) node[pos=0.75,
                    anchor=south west]{$E$};
                \end{MOdiagram}                        
                \caption{$\ce{ [ Co( CN )_{6} ]^{4-} }$}
            \end{subfigure}%
            \begin{subfigure}{.5\textwidth}
                \centering
                \begin{MOdiagram}[labels-fs = \tiny]
                    \AO{s}{1; pair}
                    \AO(20pt){s}{1; up}
                    \AO(40pt){s}{1; up}
                    \AO(60pt){s}{1; up}
                    \AO(80pt){s}{1; up}
                    \AO(110pt){s}[label={$\mathrm{d_{xy}}$}]{0; pair}
                    \AO(130pt){s}[label={$\mathrm{d_{xz}}$}]{0; pair}
                    \AO(150pt){s}[label={$\mathrm{d_{yz}}$}]{0; pair}
                    \AO(120pt){s}[label={$\mathrm{d_{z^2}}$}]{2; }
                    \AO(140pt){s}[label={$\mathrm{d_{x^2 - y^2}}$}]{2; }
                    \connect{AO5 & AO6, AO5 & AO9}
                    \draw (AO8) node[anchor=west, xshift=10]{$\mathrm{t_{2g}}$};
                    \draw (AO10) node[anchor=west, xshift=10]{$\mathrm{e_{g}}$};
                \end{MOdiagram}                        
                \caption{$\ce{ [ Co( CN )_{6} ]^{3-} }$}
            \end{subfigure}%
        \end{figure}

        On ther other hand, we don't know the strength of bidentate chelating
        ligand. If it is a weak ligand, the observation that it remains
        paramagnetic can be explained by the following diagrams:

        \begin{figure}[H]
            \centering
            \begin{subfigure}{.5\textwidth}
                \centering
                \begin{MOdiagram}[labels-fs = \tiny]
                    \AO{s}{1; pair}
                    \AO(20pt){s}{1; pair}
                    \AO(40pt){s}{1; up}
                    \AO(60pt){s}{1; up}
                    \AO(80pt){s}{1; up}
                    \AO(110pt){s}[label={$\mathrm{d_{xy}}$}]{0; pair}
                    \AO(130pt){s}[label={$\mathrm{d_{xz}}$}]{0; pair}
                    \AO(150pt){s}[label={$\mathrm{d_{yz}}$}]{0; up}
                    \AO(120pt){s}[label={$\mathrm{d_{z^2}}$}]{2; up}
                    \AO(140pt){s}[label={$\mathrm{d_{x^2 - y^2}}$}]{2; up}
                    \connect{AO5 & AO6, AO5 & AO9}
                    \draw [->] (-0.5,0) -- (-0.5,2) node[pos=0.75,
                    anchor=south west]{$E$};
                \end{MOdiagram}                        
                \caption{$\ce{ [ Co( ox )_{3} ]^{4-} }$}
            \end{subfigure}%
            \begin{subfigure}{.5\textwidth}
                \centering
                \begin{MOdiagram}[labels-fs = \tiny]
                    \AO{s}{1; pair}
                    \AO(20pt){s}{1; up}
                    \AO(40pt){s}{1; up}
                    \AO(60pt){s}{1; up}
                    \AO(80pt){s}{1; up}
                    \AO(110pt){s}[label={$\mathrm{d_{xy}}$}]{0; pair}
                    \AO(130pt){s}[label={$\mathrm{d_{xz}}$}]{0; up}
                    \AO(150pt){s}[label={$\mathrm{d_{yz}}$}]{0; up}
                    \AO(120pt){s}[label={$\mathrm{d_{z^2}}$}]{2; up}
                    \AO(140pt){s}[label={$\mathrm{d_{x^2 - y^2}}$}]{2; up}
                    \connect{AO5 & AO6, AO5 & AO9}
                    \draw (AO8) node[anchor=west, xshift=10]{$\mathrm{t_{2g}}$};
                    \draw (AO10) node[anchor=west, xshift=10]{$\mathrm{e_{g}}$};
                \end{MOdiagram}                        
                \caption{$\ce{ [ Co( ox )_{3} ]^{3-} }$}
            \end{subfigure}%
        \end{figure}

    \item $\ce{ [ MnCl_{6} ]^{3-} }$ with four unpaired electrons. $\ce{ Cl- }$
        is a weak ligand with high spin.
        \begin{center}
            \begin{MOdiagram}[labels-fs = \tiny]
                \AO{s}{1; up}
                \AO(20pt){s}{1; up}
                \AO(40pt){s}{1; up}
                \AO(60pt){s}{1; up}
                \AO(80pt){s}{1; }
                \AO(110pt){s}[label={$\mathrm{d_{xy}}$}]{0; up}
                \AO(130pt){s}[label={$\mathrm{d_{xz}}$}]{0; up}
                \AO(150pt){s}[label={$\mathrm{d_{yz}}$}]{0; up}
                \AO(120pt){s}[label={$\mathrm{d_{z^2}}$}]{2; up}
                \AO(140pt){s}[label={$\mathrm{d_{x^2 - y^2}}$}]{2; }
                \connect{AO5 & AO6, AO5 & AO9}
                \draw (AO8) node[anchor=west, xshift=10]{$\mathrm{t_{2g}}$};
                \draw (AO10) node[anchor=west, xshift=10]{$\mathrm{e_{g}}$};
                \draw [->] (-0.5,0) -- (-0.5,2) node[pos=0.75,
                anchor=south west]{$E$};
            \end{MOdiagram}                        
        \end{center}

        $\ce{ [ Mn( CN )_{6} } ]^{3-}$ with two upaired electrons. $\ce{ CN- }$
        is a strong ligand with low spin.
        \begin{center}
            \begin{MOdiagram}[labels-fs = \tiny]
                \AO{s}{1; up}
                \AO(20pt){s}{1; up}
                \AO(40pt){s}{1; up}
                \AO(60pt){s}{1; up}
                \AO(80pt){s}{1; }
                \AO(110pt){s}[label={$\mathrm{d_{xy}}$}]{0; pair}
                \AO(130pt){s}[label={$\mathrm{d_{xz}}$}]{0; up}
                \AO(150pt){s}[label={$\mathrm{d_{yz}}$}]{0; up}
                \AO(120pt){s}[label={$\mathrm{d_{z^2}}$}]{2; }
                \AO(140pt){s}[label={$\mathrm{d_{x^2 - y^2}}$}]{2; }
                \connect{AO5 & AO6, AO5 & AO9}
                \draw (AO8) node[anchor=west, xshift=10]{$\mathrm{t_{2g}}$};
                \draw (AO10) node[anchor=west, xshift=10]{$\mathrm{e_{g}}$};
                \draw [->] (-0.5,0) -- (-0.5,2) node[pos=0.75,
                anchor=south west]{$E$};
            \end{MOdiagram}                        
        \end{center}
\end{enumerate}

\section*{ Introduction to Chemical Kinetics }

\begin{enumerate}[1.]
    \setcounter{enumi}{5}
    \item 
        \begin{enumerate}[a.]
            \item By plotting the natural logarithm of the known values, it is
                clear that there is a linear relationship. As a result, this
                reaction is first order governed by the integrated rate law:
                $$ \ln [ \ce{ A } ] = -kt + \ln [ \ce{ A } ]_{0} $$

                \begin{figure}[H]
                    \centering
                    \includegraphics[scale=0.75]{"Linear Trendline"}
                    \caption{Linear Trendline}
                \end{figure}

                Note: y-axis is $\ln [ A ]$ in molars and x-axis is $t$ in days.

            \item From the trendline, the slope is -0.3735. From the integrated
                rate law:
                $$ k = 0.3735\ \si{ s^{-1} } $$

                Using the equation for half-life for a first order reaction:
                $$ t_{1/2} = \frac{ \ln 2 }{ k } = 1.856\ \si{ s } $$

            \item Using the trendline equation:
                $$ [ \ce{ A } ](t) = e^{-kt + \ln [ \ce{ A } ]_{0}} $$
                $$ e^{-0.3735 \cdot 3 + \ln( 6.000 )} = 1.960 $$

        \end{enumerate}
    \item 
        \begin{enumerate}[a.]
            \item Using the first order rate law:
                $$ \frac{ 1 }{ 8 } = e^{-6.85 \cdot 10^{-2} \cdot t} $$

                Solving for $t$:
                $$ t = 30.36\ \si{ min } $$

            \item Using the first order rate law:
                $$ 0.10 = e^{-6.85 \cdot 10^{-2} \cdot t} $$

                Solving for $t$:
                $$ t = 33.61\ \si{ min } $$

            \item Using the first order rate law:
                $$ \frac{ 1 }{ 3 } = e^{-6.85 \cdot 10^{-2} \cdot t} $$

                Solving for $t$:
                $$ t = 16.04\ \si{ min } $$

        \end{enumerate}
\end{enumerate}

\end{document}

