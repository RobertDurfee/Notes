\documentclass{article}
\usepackage{tikz}
\usepackage{float}
\usepackage{enumerate}
\usepackage{amsmath}
\usepackage{bm}
\usepackage{indentfirst}
\usepackage{siunitx}
\usepackage[utf8]{inputenc}
\usepackage{graphicx}
\graphicspath{ {Images/} }
\usepackage{float}
\usepackage{mhchem}
\usepackage{chemfig}
\allowdisplaybreaks

\title{ 5.111 Problem Set 8 }
\author{ Robert Durfee }
\date{ November 14, 2017 }

\begin{document}

\maketitle

\section{ pH, pOH, and pK }

Writing out the deprotonation chemcial equation for lactic acid:
$$\ce{C_3H_6O_3 <=> H+ + C_3H_5O_3-}$$

Convert the $\rm{pK_a}$ to a $\rm{K_a}$:
\begin{align*}
    \rm{K_a} &= 10^{-\rm{pK_a}} \\
    \rm{K_a} &= 8.317 \cdot 10^{-4}
\end{align*}

Writing out the equilibrium chart:

\begin{center}
    \begin{tabular}{c c c c}
         & $\ce{C_3H_6O_3}$ & $\ce{H+}$ & $\ce{C_3H_5O_3-}$ \\
        Initial & $1.5 \cdot 10^{-3}$ & 0 & 0 \\
        Change & $-x$ & $+x$ & $+x$ \\
        Equilibrium & $1.5 \cdot 10^{-3} - x$ & x & x
    \end{tabular}
\end{center}

Set equalibrium equal to $\rm{K_a}$:
\begin{align*}
    8.317 \cdot 10^{-4} &= \frac{x^2}{1.5 \cdot 10^{-3} - x} \\
    1.2476 \cdot 10^{-6} &= x^2 \\
    1.1170 \cdot 10^{-3} &= x
\end{align*}

The 5\% assumption doesn't hold. Using the quadratic equation:
\begin{align*}
    x &= \frac{8.317 \cdot 10^{-4} - \sqrt{ (-8.317 \cdot 10^{-4} )^2 -
    4(-1)(1.24755 \cdot 10^{-6})}}{2(-1)} \\
    x &= 7.7599 \cdot 10^{-4}\ \si{M}
\end{align*}

Thus, this value of $x$ equals the concentration of hydrogen ion:
\begin{align*}
    \rm{pH} &= -\log([\ce{H+}]) \\
    \rm{pH} &= -\log(7.7 \cdot 10^{-4}) \\
    \rm{pH} &= 3.11
\end{align*}

Solving for the pOH:
\begin{align*}
    \rm{pOH} &= 14 - \rm{pH} \\
    \rm{pOH} &= 14 - 3.11 \\
    \rm{pOH} &= 10.9
\end{align*}

Solving for the percent deprotonation:
$$\frac{7.7599 \cdot 10^{-4}}{1.5 \cdot 10^{-3}} = 52.7 \%$$

\section{Buffer Problems}

$\ce{KH_2PO_4}$ is the acid and $\ce{Na_2HPO_4}$ is the conjugate base.
$$\ce{H_2PO_4- <=> H+ + HPO_4^{2-} }$$

\begin{enumerate}[(a)]
    \item Calculating moles of $\ce{H_2PO_4-}$:
        $$(0.1\ \si{L})(0.100\ \si{M}) = 0.01\ \si{mol}$$

        Calculating moles of $\ce{HPO_4^{2-}}$:
        $$(0.1\ \si{L})(0.150\ \si{M}) = 0.015\ \si{mol}$$

        Solving for initial pH using the Henderson-Hasselbalch equation:
        \begin{align*}
            \rm{pH} &= \rm{pK_a} + \log \left( \frac{[\ce{A-}]}{[\ce{HA}]}
            \right) \\
            \rm{pH} &= 7.21 + \log \left( \frac{0.150}{0.100} \right) \\
            \rm{pH} &= 7.39
        \end{align*}

        Solving for moles of added base:
        $$(0.0800\ \si{L})(0.0100\ \si{M}) = 0.0008\ \si{mol}$$
        
        Setting up an ICE chart:
        \begin{center}
            \begin{tabular}{c c c c}
                & $\ce{H_2PO_4-}$ & $\ce{H+}$ & $\ce{HPO_4^{2-}}$ \\
                Initial & 0.01 &  & 0.015 \\
                Change & -0.0008 &  & +0.0008 \\
                Equilibrium & 0.0092 & x & 0.0158
            \end{tabular}
        \end{center}

        Solving for final pH using the Henderson-Hasselbalch equation:
        \begin{align*}
            \rm{pH} &= \rm{pK_a} + \log \left( \frac{[\ce{A-}]}{[\ce{HA}]}
            \right) \\
            \rm{pH} &= 7.21 + \log \left( \frac{0.0158}{0.0092} \right) \\
            \rm{pH} &= 7.44
        \end{align*}

        Change in pH:
        \begin{align*}
            \Delta \rm{pH} &= \rm{pH}_f - \rm{pH}_i \\
            \Delta \rm{pH} &= 7.44 - 7.39 \\
            \Delta \rm{pH} &= 0.05
        \end{align*}

    \item Solving for moles of added acid:
        $$(0.010\ \si{L})(1.0\ \si{M}) = 0.010\ \si{mol}$$
        
        Setting up an ICE chart:
        \begin{center}
            \begin{tabular}{c c c c}
                & $\ce{H_2PO_4-}$ & $\ce{H+}$ & $\ce{HPO_4^{2-}}$ \\
                Initial & 0.01 & 0.010 & 0.015 \\
                Change & +0.010 & -0.010 & -0.010 \\
                Equilibrium & 0.020 & x & 0.005
            \end{tabular}
        \end{center}

        Solving for final pH using the Henderson-Hasselbalch equation:
        \begin{align*}
            \rm{pH} &= \rm{pK_a} + \log \left( \frac{[\ce{A-}]}{[\ce{HA}]}
            \right) \\
            \rm{pH} &= 7.21 + \log \left( \frac{0.005}{0.020} \right) \\
            \rm{pH} &= 6.61
        \end{align*}

        Change in pH:
        \begin{align*}
            \Delta \rm{pH} &= \rm{pH}_f - \rm{pH}_i \\
            \Delta \rm{pH} &= 6.61 - 7.39 \\
            \Delta \rm{pH} &= -0.78
        \end{align*}

\end{enumerate}

\section{Titration Curves}
\begin{figure}[H]
    \centering
    \includegraphics[scale=1]{"Figure 1"}
\end{figure}

The pH of equivalency should be higher than 7.

\section{Titration Problem}
$$\ce{C_6H_5OOH <=> H+ + C_6H_500-}$$

\begin{enumerate}[(a)]
    \item 
\end{enumerate}

\end{document}

