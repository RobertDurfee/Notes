\documentclass{article}
\usepackage[version=4]{mhchem}
\usepackage{amsmath}
\usepackage{indentfirst}
\usepackage[utf8]{inputenc}
\usepackage{siunitx}
\allowdisplaybreaks

\title{5.111 Problem Set 7}
\author{Robert Durfee}
\date{November 6, 2017}

\begin{document}

\maketitle

\section{Equilibrium Constant}

\subsection{Chloromethane}
\begin{gather*}
    \ce{2CH_{4(g)} + 2Cl_{2(g)} <=> 2CH_3Cl_{(g)} + 2HCl_{(g)}}\\\\
    \Delta G^{\circ}_{f}(\ce{CH_{3}Cl_{(g)}}) = 48.50\ \si{kJ\ mol^{-1}}\\
    \Delta G^{\circ}_{f}(\ce{CH_{4(g)}}) = -50.72\ \si{kJ\ mol^{-1}}\\
    \Delta G^{\circ}_{f}(\ce{Cl_{2(g)}}) = 0\ \si{kJ\ mol^{-1}}\\
    \Delta G^{\circ}_{f}(\ce{HCl_{g}}) = -95.30\ \si{kJ\ mol^{-1}}
\end{gather*}

\begin{enumerate}
    \item What is the Gibb's Free Energy $\Delta G^{\circ}$ in \si{kJ\
        mol^{-1}}?
    \begin{align*}
        \Delta G^{\circ}&=\sum\limits_{Products}\Delta G^{\circ}_{f} -
        \sum\limits_{Reactants} \Delta G^{\circ}_{f}\\
        \Delta
        G^{\circ}&=\left[2(48.50)+2(-95.30)\right]-\left[2(-50.72)+2(0)\right]\
        \si{kJ\ mol^{-1}}\\
        \Delta G^{\circ}&=7.84\ \si{kJ\ mol^{-1}}
    \end{align*}
    \item What is the equilibrium constant, K?
    \begin{align*}
        K &= e^{-\Delta G^{\circ}/RT}\\
        K &= e^{-(7840\ \si{J\ mol^{-1}})/((8.314\ \si{J\ mol^{-1}}\
    K^{-1})(298\ \si{K}))}\\
        K &= 0.0422
    \end{align*}
    
\end{enumerate}

\subsection{Nitric Acid}
\begin{gather*}
    \ce{3NO_{2(g)} + H_{2}O_{(l)} <=> 2HNO_{3(aq)} + NO_{(g)}}\\\\
    \Delta G^{\circ}_{f}(\ce{NO_{2(g)}}) = 51.33\ \si{kJ\ mol^{-1}}\\
    \Delta G^{\circ}_{f}(\ce{H_{2}O_{(l)}}) = -237.13\ \si{kJ\ mol^{-1}}\\
    \Delta G^{\circ}_{f}(\ce{HNO_{3(aq)}}) = -111.25\ \si{kJ\ mol^{-1}}\\
    \Delta G^{\circ}_{f}(\ce{NO_{(g)}}) = 86.55\ \si{kJ\ mol^{-1}}
\end{gather*}

\begin{enumerate}
    \item What is the Gibb's Free Energy $\Delta G^{\circ}$ in \si{kJ\
        mol^{-1}}?
    \begin{align*}
        \Delta G^{\circ}&=\sum\limits_{Products}\Delta G^{\circ}_{f} -
        \sum\limits_{Reactants} \Delta G^{\circ}_{f}\\
        \Delta G^{\circ}&=\left[2(-111.25)+(86.55)\right]-
        \left[3(51.33)+(-237.13)\right]\si{kJ mol^{-1}}\\
        \Delta G^{\circ}&=-52.81\ \si{kJ mol^{-1}}
    \end{align*}
    \item What is the equilibrium constant, $K$?
    \begin{align*}
        K &= e^{-\Delta G^{\circ}/RT}\\
        K &= e^{(52810\ \si{J\ mol^{-1}})/((8.314\ \si{J\ mol^{-1}\
K^{-1}})(300\ \si{K}))}\\
        \ce{K} &= 1.568\cdot 10^{9}
    \end{align*}
\end{enumerate}

\section{Methane Equilibrium Reaction}

At 1400 \si{K}, $K= 2.5\cdot10^{-3}$ for the reaction \ce{CH_{4(g)} +
2H_{2}S_{(g)} <=> CS_{2(g)} + 4H_{2(g)}}. A 18 \si{L} reaction vessel at 1400
\si{K} contains 4 \si{mol} of \ce{CH_{4}}, 5 \si{mol} of \ce{CS_{2}}, 22
\si{mol} of \ce{H_{2}}, and 16 \si{mol} of \ce{H_{2}S}. Assume $K$ was
calculated using the pressure in \si{atm}.

\begin{enumerate}
    \item Is the reaction mixture at equilibrium?
    \begin{enumerate}
        \item Solve for the partial pressures.
        $$P_{i} = \frac{n_{i} R T}{V}$$
        \begin{align*}
            P_{\ce{CH_{4}}} &= \frac{(4\ \si{mol})(8.314\ \si{J\ mol^{-1}\
    K^{-1}})(1400\ \si{K})}{18\ \si{L}}&=2586.58\ \si{Pa}\\
            P_{\ce{CS_{2}}} &= \frac{(5\ \si{mol})(8.314\ \si{J\ mol^{-1}\
    K^{-1}})(1400\ \si{K})}{18\ \si{L}}&=3233.22\ \si{Pa}\\
            P_{\ce{H_{2}}} &= \frac{(22\ \si{mol})(8.314\ \si{J\ mol^{-1}\
    K^{-1}})(1400\ \si{K})}{18\ \si{L}}&=14266.2\ \si{Pa}\\
            P_{\ce{H_{2}S}} &= \frac{(16\ \si{mol})(8.314\ \si{J\ mol^{-1}\
    K^{-1}})(1400\ \si{K})}{18\ \si{L}}&=10346.3\ \si{Pa}
        \end{align*} 
        \item Convert to \si{atm}.
        $$1\ \si{Pa} = 9.8692\cdot10^{-6}\ \si{atm}$$
        \begin{align*}
            P_{\ce{CH_{4}}} &= (2586.58\ \si{Pa})(9.8692\cdot10^{-6}\ \si{atm\
        Pa^{-1}}) &= 0.0255\ \si{atm}\\
            P_{\ce{CS_{2}}} &= (3233.22\ \si{Pa})(9.8692\cdot10^{-6}\ \si{atm\
        Pa^{-1}}) &= 0.0319\ \si{atm}\\
            P_{\ce{H_{2}}} &= (14266.2\ \si{Pa})(9.8692\cdot10^{-6}\ \si{atm\
        Pa^{-1}}) &= 0.1408\ \si{atm}\\
            P_{\ce{H_{2}S}} &= (10346.3\ \si{Pa})(9.8692\cdot10^{-6}\ \si{atm\
        Pa^{-1}}) &= 0.1021\ \si{atm}
        \end{align*} 
        \item Calculate the reaction quotient.
        \begin{align*}
            Q&=\frac{(P_{\ce{CS_{2}}})(P_{\ce{H_{2}}})^{4}}{(P_{\ce{CH_{4}}})(P_{\ce{H_{2}S}})^{2}}\\
            Q&=\frac{(0.0319\ \si{atm})(0.1408\ \si{atm})^{4}}{(0.0255\
        \si{atm})(0.1021\ \si{atm})^{2}}\\
            Q&=4.7\cdot10^{-2}
        \end{align*}
        \item Compare the reaction quotient to the equilibrium constant.
        $$K=2.5\cdot10^{-3} \ne Q=4.7\cdot10^{-2}$$
        The reaction is not in equilibrium.
    \end{enumerate}
    \item In which direction must the reaction mixture proceed to attain
        equilibrium?
    $$K=2.5\cdot10^{-3} < Q=4.7\cdot10^{-2}$$
    The reaction must proceed to the left to reach equilibrium.
    
\end{enumerate}

\section{Industrial Chemistry Reaction}

The industrially important (catalyzed) reaction shown below is used to produce
starting materials for polymer synthesis. In this reaction, hydrogen is
abstracted as \ce{H_{2(g)}} from ethane \ce{C_{2}H_{6}} to produce ethene
\ce{C_{2}H_{4}}.

\begin{gather*}
    \ce{C_{2}H_{6(g)} <=> H_{2(g)} + C_{2}H_{4(g)}}\\\\
    \Delta G^{\circ}_{f}(\ce{C_{2}H_{6(g)}}) = -32.82\ \si{kJ\ mol^{-1}}\\
    \Delta G^{\circ}_{f}(\ce{H_{2(g)}}) = 68.15\ \si{kJ\ mol^{-1}}\\
    \Delta G^{\circ}_{f}(\ce{C_{2}H_{4(g)}}) = 0.00\ \si{kJ\ mol^{-1}}
\end{gather*}

\subsection*{Part A}

\begin{enumerate}
    \item Calculate $\Delta G^{\circ}_{R}$ (in \si{kJ\ mol^{-1}}) at 312 \si{K},
        assuming the $\Delta G^{\circ}_{f}$ are measured at 312 \si{K}.
    \begin{align*}
        \Delta G^{\circ}&=\sum\limits_{Products}\Delta G^{\circ}_{f} -
        \sum\limits_{Reactants} \Delta G^{\circ}_{f}\\
        \Delta G^{\circ}&=\left[68.15+0.00\right]-\left[-32.82\right]\ \si{kJ\
    mol^{-1}}\\
        \Delta G^{\circ}&=100.97\ \si{kJ\ mol^{-1}}
    \end{align*}
    \item Calculate $K$ at 312 \si{K}.
    \begin{align*}
        K &= e^{-\Delta G^{\circ}/RT}\\
        K &= e^{-(100970\ \si{J\ mol^{-1}})/((8.314\ \si{J\ mol^{-1}\
K^{-1}})(312\ \si{K}))}\\
        \ce{K} &= 1.245\cdot 10^{-17}
    \end{align*}
    
\end{enumerate}

\subsection*{Part B}

Now we consider the reaction at 298 \si{K}. This reaction has
$K=2.00\cdot10^{-18}$. If the reaction is initiated by adding a catalyst to a
flask containing \ce{C_{2}H_{6}} at $P=4.38$ \si{bar}, what will the partial
pressure of \ce{H_{2}} and \ce{C_{2}H_{4}} be at equilibrium?
\begin{center}
\begin{tabular}{c c c c}
    & \ce{C_{2}H_{6}} & \ce{H_{2}} & \ce{C_{2}H_{4}} \\
    Initial & $4.38$ \si{bar} & $0.00$ \si{bar} & $0.00$ \si{bar} \\
    Change & $-x$ \si{bar} & $+x$ \si{bar} & $+x$ \si{bar} \\
    Final & $4.38-x$ \si{bar} & $x$ \si{bar} & $x$ \si{bar}
\end{tabular}
\end{center}
\begin{align*}
    K&=\frac{(P_{\ce{H_{2}}})(P_{\ce{C_{2}H_{4}}})}{(P_{\ce{C_{2}H_{6}}})}\\
    2.00\cdot10^{-18}&=\frac{x^{2}}{4.38\ \si{bar}-x}\\
    2.00\cdot10^{-18}&\approx\frac{x^{2}}{4.38\ \si{bar}}\\
    x&\approx\sqrt{2.00\cdot10^{-18}\cdot4.38\ \si{bar}}\\
    x&\approx2.960\cdot10^{-9}\ \si{bar}\\
    P_{\ce{H_{2}}}\approx P_{\ce{C_{2}H_{4}}} &\approx 2.960\cdot10^{-9}\
    \si{bar}
\end{align*}

\subsection*{Part C}

You are working as a chemist and are tasked with producing a higher reaction
yield. What should you do to the following components of the reaction to
increase the reaction yield?

\begin{enumerate}
    \item For the reactant \ce{C_{2}H_{6}}, we should add more \ce{C_{2}H_{6}}
        as Le Chatelier'S Principle states that an increase in reactants will
        cause a momentary increase in products. 
    
    \item For the products \ce{H_{2}} or \ce{C_{2}H_{4}}, we should remove
        \ce{H_{2}} or \ce{C_{2}H_{4}} as Le Chatelier's Principle states that an
        decrease in products will cause a momentary increase in products.
    
    \item The volume of the reaction container should be increased at constant
        pressure and temperature as Le Chatelier's Principle states that an
        increase in volume will favor the side of the chemical equation with
        more gas molecules.
    
    \item The pressure in the reaction should not be altered by adding an inert
        gas at constant volume and temperature as this will only increase total
        pressure under constant volume.
\end{enumerate}

\section{Nitric Oxide}
$$\ce{N_{2(g)} + O_{2(g)} <=> 2NO_{(g)}}$$

\begin{enumerate}
    \item The temperature dependence of the equilibrium constant can be
        expressed by the following expression. 
    $$\ln(K_{eq})=2.50-\frac{2.170\cdot10^{4} \si{K}}{T}$$
    What is $\Delta H^{\circ}_{r}$ in \si{kJ\ mol^{-1}} at 1264.4 \si{K}?
    $$\ln(K_{eq})=\frac{\Delta S^{\circ}}{R}-\frac{\Delta H^{\circ}}{RT}$$
    It is possible to identify the term containing $\Delta H^{\circ}$:
    \begin{align*}
        2.170\cdot10^4\ \si{K}&=\frac{\Delta H^{\circ}}{R}\\
        \Delta H^{\circ}&=(2.170\cdot19^{4}\ \si{K})(8.314\ \si{J\ mol^{-1}\
    K^{-1}})\\
        \Delta H^{\circ}&=180414\ \si{J\ mol^{-1}}\\
        \Delta H^{\circ}&=180.414\ \si{kJ\ mol^{-1}}
    \end{align*}
    
    \item Would the reaction proceed further toward the product side at high
        temperatures or at low temperatures?
    \begin{align*}
        K_{eq}=e^{2.50-2.170\cdot10^{4}\ \si{K}/T}
    \end{align*}
    
    As temperature increases, $K_{eq}$ increases, therefore favoring the
    products. As temperature decreases, $K_{eq}$ decreases, therefore favoring
    the reactants.
    
    \item An equimolar mixture of \ce{N_{2(g)}} and \ce{O_{2(g)}} was heated to
        a certain temperature until an equilibrium is established. The
        equilibrium mixture contains an equal amount in \si{mol} of
        \ce{N_{2(g)}}, \ce{O_{2(g)}}, and \ce{NO_{(g)}}. Assuming that the
        reaction vessel has a fixed volume, what was the temperature in \si{K}?
    \begin{align*}
        K_{eq}=\frac{(P_{\ce{NO}})^{2}}{(P_{\ce{N_{2}}})(P_{O_{2}})}
    \end{align*}
    Since the partial pressure of each is only dependent upon number of moles,
    all the partial pressures will be the same at equilibrium.
    \begin{align*}
        K_{eq}&=\frac{(P)^{2}}{(P)(P)}\\
        K_{eq}&=1
    \end{align*}
    Substituting $K_{eq}$ into the Van 't Hoff equation:
    \begin{align*}
        \ln(K_{eq})&=2.50-\frac{2.170\cdot10^{4} \si{K}}{T}\\
        2.50&=\frac{2.170\cdot10^{4} \si{K}}{T}\\
        T&=\frac{2.170\cdot10^{4} \si{K}}{2.50}\\
        T&=8680\ \si{K}
    \end{align*}
    
    \item An equimolar mixture of \ce{N_{2(g)}} and \ce{O_{2(g)}} with a total
        pressure of 11.5 \si{bar} is allowed to reach equilibrium at 9723.0
        \si{K}. What is the partial pressure in bar of each reactant and product
        in the equilibrium mixture?
    
    Calculate the $K_{eq}$ for $9723.0\ \si{K}$:
    \begin{align*}
        K_{eq}&=e^{2.50-(2.170\cdot10^{4} \si{K})/(9723.0\ \si{K})}\\
        K_{eq}&=1.30758
    \end{align*}
    \begin{center}
    \begin{tabular}{c c c c}
         & \ce{N_{2}} & \ce{O_{2}} & \ce{2NO} \\
        Initial & 5.75 bar & 5.75 bar & 0 bar \\
        Change & $-x$ & $-x$ & $+2x$ \\
        Final & $5.75-x$ & $5.75-x$ & $2x$
    \end{tabular}
    \end{center}
    \begin{align*}
        1.30758&=\frac{(2x)^2}{(5.75-x)^2}\\
        x&=2.0917
    \end{align*}
    Solving for the partial pressures:
    \begin{alignat*}{3}
        P_{\ce{N_{2}}}&=5.75-x&&=5.75-2.0927&&=3.658\ \si{bar}\\
        P_{\ce{O_{2}}}&=5.75-x&&=5.75-2.0927&&=3.658\ \si{bar}\\
        P_{\ce{NO}}&=2x&&=2(2.0927)&&=4.183\ \si{bar}\\
    \end{alignat*}
\end{enumerate}

\section{Hydrogen Sulfide Le Chatelier}

To improve air quality and obtain a useful product, sulfur is often removed from
coal and natural gas by treating the fuel contaminant hydrogen sulfide with
\ce{O_{2}}:
$$\ce{2H_{2}S_{(g)} + O_{2(g)} <=> 2S_{(g)} + 2H_{2}O_{(g)}}$$

\begin{enumerate}
    \item What happens to \ce{[H_{2}O]} if \ce{O_{2}} is added? Increase.
    
    \item What happens to \ce{[H_{2}S]} if \ce{O_{2}} is added? Decrease.
    
    \item What happens to \ce{[O_{2}]} if \ce{H_{2}S} is added? Decrease.
    
    \item What happens to \ce{[H_{2}S]} if \ce{S} is added? Increase.
\end{enumerate}

\section{Oxidation of Sulfur Dioxide}

The oxidation of sulfur dioxide (\ce{SO_{2}}) is a key step in the production of
sulfuric acid in the industry:
$$\ce{2SO_{2(g)} + O_{2(g)} <=> 2SO_{3(g)}}$$
\begin{align*}
    K_{eq}(298.0 \si{K})&=4.0\cdot10^{24}\\
    K_{eq}(500.0 \si{K})&=2.5\cdot10^{10}
\end{align*}

A mixture of 3.36 \si{mmol} \ce{SO_{2}} and 2.49 \si{mmol} \ce{O_{2}} in a 666.9
\si{mL} container was heated to 500.0 \si{k} and was allowed to reach
equilibrium. To which direction would the equilibrium shift if the mixture is
cooled to 298.0 \si{k}?

Since $K_{eq}(298.0\ \si{K}) > K_{eq}(500.0\ \si{K})$, and
$K_{eq}=[products]/[reactants]$, going from 500.0 \si{K} to 298.0 \si{K} will
result in an increase in products.

\section{Nitrogen Dissolution}

\ce{N_{2(g)}} dissolved in a sample of water in a partly filled, sealed
container has reached equilibrium with its partial pressure in the air above the
solution. What happens to the solubility of the \ce{N_{2(g)}} if:

\begin{enumerate}
    \item The partial pressure of \ce{N_{2(g)}} is increased by compressing the
        gas to one-fourth of its original volume?
    
    The solubility of \ce{N_{2}} will increase.
    
    \item The total pressure of the gas above the liquid is tripled by adding
        \ce{O_{2(g)}}?
    
    The solubility of \ce{N_{2}} will not change.
    
\end{enumerate}

\section{Dissolved Gas}

A soft drink is made by dissolving \ce{CO_{2}} gas at 2.61 \si{atm} in 652.2
\si{mL} of an aqueous, flavored solution at 20.0 \si{\degreeCelsius} and sealing
the resulting solution in a bottle. Supposing that various additives comprise
about 5.0\% of the soft drink’s volume and the remainder is water. The Henry's
Law constant for \ce{CO_{2}} in water is $k_{H}=0.023\ \si{mol\ L^{-1}\
atm^{-1}}$.

\begin{enumerate}
    \item What is the maximum number of moles of \ce{CO_{2}} that could be
        dissolved in 652.2 \si{mL} of the soft drink at 2.61 \si{atm} of
        \ce{CO_{2}} and 20.0 \si{\degreeCelsius}?
    \begin{align*}
        n_{1}&=k_{H}PV\\
        n_{1}&=(0.023\ \si{mol\ L^{-1}\ atm^{-1}})(2.61\ \si{atm})(0.6522\
        \si{L})(0.95)\\
        n_{1}&=0.0372\ \si{mol}
    \end{align*}
    
    \item How many moles of \ce{CO_{2}} would you expect to be released from
        this 652.2 \si{mL} solution if the conditions were shifted to a
        \ce{CO_{2}} partial pressure of 0.028 \si{atm} and 25.0
        \si{\degreeCelsius}? Under these conditions, $k_{H}=0.033$ \si{mol\
        L^{-1}\ atm^{-1}}.
    \begin{align*}
        n_{2}&=k_{H}PV\\
        n_{2}&=(0.033\ \si{mol\ L^{-1}\ atm^{-1}})(0.028\ \si{atm})(0.6522\
        \si{L})(0.95)\\
        n_{2}&=0.000573\ \si{mol}
    \end{align*}
    Calculate the difference between the number of mols dissolved:
    \begin{align*}
        \Delta n &= n_{1}-n_{2}\\
        \Delta n &= 0.0372 - 0.000573\\
        \Delta n &= 0.0366\ \si{mol}
    \end{align*}
    
\end{enumerate}

\end{document}
