\documentclass{article}
\usepackage{chemfig}
\usepackage{indentfirst}
\usepackage[version=4]{mhchem}
\usepackage[utf8]{inputenc}
\usepackage{siunitx}

\title{5.111 Recitation 15}
\author{Robert Durfee}
\date{November 2017}

\begin{document}

\maketitle

\section{Significant Figures}

For logarithms and exponents, keep the number of significant figures prior to the operation. If the question specifies, round to what is asked. Although this is not scientifically correct, this is the approximation that we will be using because anything else gets too complicated.

\section{Le Chatelier}

\subsection{Concentration/Partial Pressure}

What is the the value of $\Delta G$ when you are away from equilibrium.
$$\Delta G=\Delta G^{\circ}+RT\ln(Q)$$
$$\Delta G^{\circ}= -RT\ln(K)$$
$$\Delta G = RT\ln(Q/K)$$

If $Q>K$, then $\Delta G>0$. If $Q<K$, then $\Delta G <0$.
$$\ce{aA + bB <=> cC + dD}$$
$$Q=\frac{[C]^c[D]^d}{[A]^a[B]^b}$$

If you start at equilibrium, then $Q=K$. If you then add more $A$, the concentration A will increase. This causes $Q$ to decrease. This makes $Q<K$ and thus $\Delta G<0$. This causes the reaction to shift towards the products.

\subsubsection*{Example}
Chemical equation:
$$\ce{2SO_{2(g)} + O_{2(g)} <=> 2SO_{3(g)}}$$
Writing the expressiong for $Q$ for this reaction:
$$Q=\frac{(P_{\ce{SO_3}})^2}{(P_{\ce{SO_2}})^{2}(P_{O_2})}$$
If you add more $\ce{SO_{2}}$, you favor the products. If you add more $\ce{SO_3}$, you favor the reactants. If you remove either, it will cause the opposite effect.

\subsection{Temperature}

We can also change the temperature of the reaction which will also effect the equilibrium. Based on the enthalpy, we can place heat on one side of the reaction. Then we can treat it as a normal chemical.

\subsubsection*{Example (Continued)}
New chemical equation:
$$\ce{2SO_{2(g)} + O_{2(g)} <=> 2SO_{3(g)} + Heat}$$
$$\Delta H=-197.78\ \si{kJ\ mol^{-1}}$$
This reaction is exothermic. If we then increase the temperature, then we are increasing the amount of heat available, pushing the reaction towards the reactants.

\subsection{Inert Gas}
Inert gasses increase total pressure of the system, but this doesn't affect any of the partial pressures.

\subsubsection*{Example (Continued)}
New chemical equation:
$$\ce{N_{2(g)} + 2SO_{2(g)} + O_{2(g)} <=> N_{2(g)} + 2SO_{3(g)} + Heat}$$
If we add $\ce{N_2}$, there is no change to the equilibrium.

\section{Acids and Bases}

\textbf{Arrhenius acids} increase the concentration of $\ce{[H_3O+]}$ in water. \textbf{Arrhenius bases} increase the concentration of $\ce{[OH-]}$ in water. \textbf{Bronsted-Lowry acids} donate protons. \textbf{Bronsted-Lowry bases} accept the protons.

For Arrhenius, we assume that we place these in water. For example:
$$\ce{HCl + H_2O -> H_3O+ + Cl-}$$

For Bronsted-Lowry there doesn't have to be water, but $\ce{NH_3}$ would still be a base, for example:
$$\ce{NH_3 + CH_3COOH <=> NH_4+ + CH_3COO-}$$

\textbf{Lewis acids} accept electron pairs. \textbf{Lewis bases} donate electron pairs. For example, $\ce{NH_3}$ has a lone pair of electrons which can be donated and $\ce{BH_3}$ has an open orbital for electrons. Therefore, $\ce{NH_3}$ is a Lewis base and $\ce{BH_3}$ is a Lewis acid. When the electrons are donated, they form a bond.
$$\ce{NH_3 + BH_3 <=> NH_3BH_3}$$

Water is \textbf{amphoteric} which means that it can act as both and acid or a base, as shown below.
$$\ce{H_2O + H_2O <=> H_3O+ + OH-}$$
One water acts as an acid and the other acts as a base.

If you react and acid and a base, the acid goes to its conjugate base and the base goes to its conjugate acid.
$$\ce{CH_3COOH + H_2O <=> H_3O+ + CH_3COO-}$$

In this case, $\ce{CH_3COOH}$ is the acid, $\ce{H_2O}$ is the base. $\ce{H_3O+}$ is the conjugate acid, $\ce{CH_3COO-}$ is the conjugate base. This allows us to determine the relative strengths of acids and bases. Strong acids and bases go to completly useless conjugate acids and bases which do not contribute to pH at all. For example, $\ce{HCl}$ is a strong acid and $\ce{Cl-}$ is a completely useless conjugate base.

\subsection{Example}
$$\ce{HCOOH + H_2O <=> H_3O+ + HCOO-}$$
$$K_a=1.8\cdot10^{-4}=\frac{[\ce{H_3O+}][\ce{HCOO-}]}{[\ce{CHOOH}]}$$
The $K_a$ value shows us that the acid is not completely converted to the base. This would happen for a larger $K_a$ value.

\end{document}
