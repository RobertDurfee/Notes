\documentclass{article}
\usepackage{tikz}
\usepackage{float}
\usepackage{enumerate}
\usepackage{amsmath}
\usepackage{bm}
\usepackage{indentfirst}
\usepackage{siunitx}
\usepackage[utf8]{inputenc}
\usepackage{graphicx}
\graphicspath{ {Images/} }
\usepackage{float}
\usepackage{mhchem}
\usepackage{chemfig}
\allowdisplaybreaks

\title{ 5.111 Recitation 24 }
\author{ Robert Durfee }
\date{ December 12, 2017 }

\begin{document}

\maketitle

\section{ Arrhenius Equation }

$$ k = A e^{-E_{a} / RT} $$

This is the equation which shows the temperature dependency of the rate constant
with temperature and the activation energy. The $E_{a}$ term is the activation
energy for the reaction. This can be rearranged:
$$ \ln(k) = \ln(A) - \frac{ E_{a} }{ RT }$$

Using this equation, we can plot the natural logarithm of the rate constant and
the inverse time to get a linear graph with a slope containing the activation
energy. As the activation energy increase, the slope becomes steeper. As a
result, if temperature increases, the rate will be more responsive to
temperature changes. This is consistent with what we learned before:
$$ \Delta G = -RT \ln(K) $$
$$  K = e^{-\Delta G / RT} $$

Since the equilibrium is related to the speed of the forward and reverse
reactions, we can relate this to rate constants:
$$ k_{f} = Ae^{-E_{a} / RT},\ k_{r} = Ae^{-E_{a} - \Delta G / RT} $$

For a simple reaction at equilibrium,
$$ \ce{ A <=> B } $$
$$ \mathrm{rate} = k_{f}[ A ] - k_{r}[ B ] = 0$$
$$ \frac{ k_{f} }{ k_{r} } = \frac{ [B] }{ [A] } = K$$

Substituting back in:
$$  \frac{ k_{f} }{ k_{r} } = e^{-\Delta G / RT} = K $$

\subsection{Example}

$$ \ce{ BH_{4}- + NH_{4}+ -> BH_{3}NH_{3} + H_{2} } $$
$$ k = 1.94 \cdot 10^{-4}\ \si{ M^{-1}\ s^{-1} },\ T = 30.0\ \si{ C },\ \
E_{a} = 161\ \si{ kJ\ mol^{-1} },\ t_{1/2} = 1.00 \cdot 10^{4}\ \si{ s }$$

What is $k$ at $40^{\circ}$?
$$ 1.94 \cdot 10^{-4} = A e^{-161000 / (8.314) (303) } $$

Solving for $A$:
$$ A = 1.11 \cdot 10^{24} $$

Plugging in $A$ to solve for $k$ at new temperature:
$$ k = (1.1 \cdot 10^{24}) e^{-161000 / (8.314)(313)} = 1.49 \cdot 10^{-3} $$

What is the half-life at the new temperature?
$$ \frac{ t_{1/2, 40} }{ t_{1/2, 30} } = \frac{k_{30}}{k_{40}} $$
$$ t_{1/2, 40} = 1.30 \cdot 10^{3} $$

\section{Mechanisms}

Steps:
$$ \ce{ H_{2}O_{2} -> H_{2}O + O } $$
$$ \ce{ O + CF_{2}Cl_{2} -> ClO + CF_{2} + CF_{2}Cl } $$
$$ \ce{ ClO + O_{3} -> Cl + 2O_{2} } $$
$$ \ce{ Cl + CF_{2}Cl -> CF_{2}Cl_{2} } $$

Overall:
$$ \ce{ H_{2}O_{2} + O_{3} -> H_{2}O + 2O_{2} } $$

The \textbf{molecularity} represents the number of reactants. For example, the
first reaction is unimolecular, the second is bimolecular, etc. The
\textbf{reaction intermediates} are any reactants that are generates and then
consumed and the \textbf{catalysts} are any reactants/products that are supplied
and then are produced in the end. In this case, $\ce{ CF_{2}Cl_{2} }$ is the
catalyst.

Above, the slow reaction is the second. All the rest are fast. This is the
\textbf{rate-determining step}.

\section{Michaelis-Menten}

$$ \ce{ E + S <=> ES -> E + P } $$
There are two rates for the above reaction:
$$ \mathrm{rate} = k_{2}[ ES ] $$
$$ \frac{ d[\ce{ES}] }{ dt } = k_{1}[E][S] - k_{-1}[ES] - k_{2}[ES] = 0 $$
$$ [E_{TOT}] = [E] + [ES] $$
$$ \mathrm{rate} = \frac{ k_{2}[E_{TOT}][S] }{ K_{M} + [S] },\ K_{M} = \frac{
k_{-1} + k_{2} }{ k_{1} }$$

\end{document}

