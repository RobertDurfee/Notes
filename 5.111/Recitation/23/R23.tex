\documentclass{article}
\usepackage{tikz}
\usepackage{float}
\usepackage{enumerate}
\usepackage{amsmath}
\usepackage{bm}
\usepackage{indentfirst}
\usepackage{siunitx}
\usepackage[utf8]{inputenc}
\usepackage{graphicx}
\graphicspath{ {Images/} }
\usepackage{float}
\usepackage{mhchem}
\usepackage{chemfig}
\allowdisplaybreaks

\title{ 5.111 Recitation 23 }
\author{ Robert Durfee }
\date{ December 7, 2017 }

\begin{document}

\maketitle

\section{ Kinetics }

This is the study of how fast reactions occur. This is in contrast to
equilibrium, which assumes that you wait a long time for everything to be
converted. 

$$ \ce{ AuCl^{-}_{2} -> 2Au + AuCl^{-}_{4} + 2Cl- } $$

We want to find some way to track how fast this reaction is going. In this case,
only one molecule has a color: $\ce{ Au }$. Over time, we can track the
absorbtion wavelength, which will increase with the amount of $\ce{ Au }$
produced. 

$$ \Delta n_{\ce{ AuCl_{2}^{-} }} = -3 \cdot \Delta t \cdot V $$

This can then be related to the other species stoichiometrically. The rate of
the reaction can be written:
$$ \mathrm{ rate } = -\frac{ 1 }{ 3 } \frac{ \Delta n_{\ce{ AuCl_{2}^{-} }} }{ V
\cdot \Delta t} $$

This can be simplified into the following, which can relate rates of different
species to each other:
$$ \mathrm{ rate } = -\frac{ 1 }{ 3 } \frac{ \Delta [ \ce{ AuCl_{2}^{-} } ] }{
\Delta t } $$

To get the initial rates, use the following. Note: the exponent only corresponds
stoichiometrically when the reaction is elementary.
$$ \mathrm{ rate } = k[ \ce{ AuCl_{2}^{-} } ]^{3} $$

For a simple reaction:
$$ \ce{ A + B -> C }$$

The probability of the reaction is specified as:
$$ P( \mathrm{ reaction } ) = P( \mathrm{ collide } ) \cdot P( \mathrm{ react
\mid collide } ) \cdot P( A\ \mathrm{ in }\ \phi ) \cdot P( B\ \mathrm{ in }\
\phi )$$

Where $\phi$ is a small area. 
$$ P( A\ \mathrm{ in }\ \phi ) = [ \ce{ A } ] \phi N_{A} $$
$$ P( B\ \mathrm{ in }\ \phi ) = [ \ce{ B } ] \phi N_{A} $$ 

Substituting:
$$ P( \mathrm{ reaction } ) = P( \mathrm{ collide } ) \cdot P( \mathrm{ react
\mid collide } ) \cdot ( \phi N_{A} )^{2} \cdot [ \ce{ A } ] \cdot [ \ce{ B } ]
$$

Where all the terms prior to the concentration terms are incorporated into $k$.

\subsection{ Multiple Step Reactions }

$$ \ce{ 2A + B -> 2C } $$
$$ \ce{ A + C -> D } $$

Where the first reaction is slow and the second is fast. As a result, the
slowest reaction dictates, approximately, the overall reaction.
$$ \mathrm{ rate } = k [ \ce{ A } ]^{2} [ \ce{ B } ] $$

This rate changes dramatically depending on the rate determining step. 

\bigbreak

The \textbf{order} of the reaction is the sum of all the exponents. 

\section{ Integrated Rate Laws }

\subsection{ First Order }

$$ \ln [ \ce{ A } ] = -kt + \ln[ \ce{ A } ]_{0} $$

\subsection{ Second Order }

$$ \frac{ 1 }{ [ \ce{ A } ] } = kt + \frac{ 1 }{ [ \ce{ A } ]_{0} } $$
\end{document}
