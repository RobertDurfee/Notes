\documentclass{article}
\usepackage{tikz}
\usepackage{float}
\usepackage{enumerate}
\usepackage{amsmath}
\usepackage{bm}
\usepackage{indentfirst}
\usepackage{siunitx}
\usepackage[utf8]{inputenc}
\usepackage{graphicx}
\graphicspath{ {Images/} }
\usepackage{float}
\usepackage{mhchem}
\usepackage{chemfig}
\allowdisplaybreaks

\title{ 5.111 Recitation 17 }
\author{ Robert Durfee }
\date{ November 14, 2017 }

\begin{document}

\maketitle

\section{ Buffers }

$$\ce{H_2CO_3 + H_2O <=> HCO_3- + H_2O}$$

Say you are given you have a solution of 0.50 M $\ce{H_2CO_3}$ and 0.50 M
$\ce{HCO_3-}$.

$$\rm{pH} = \rm{pK_a} + \log \left( \frac{[\ce{HCO_3-}]}{[\ce{H_2CO_3}]}
\right)$$
$$\rm{pH} = 6.15$$

\section{Titration}

Four types of titrations that we look at: strong acid/strong base, strong
base/strong acid, weak acid/strong base, weak base/strong acid.

\subsection{Strong-Strong}

For a strong acid/strong base, the equivalence point will be where we have added
the same number of moles of base as acid. This is the theoretical value. Since
they are both strong, it should come to a pH of 7. For a strong base/strong
acid, the same results are shown, however the S-curve will start high rather
than low.

You can solve for the volume of base/acid added using dilution rules. 
$$V_{base} = \frac{M_{acid}}{M_{base}} V_{acid}$$

\subsection{Weak-Strong}

For a weak acid titrated with a strong base, there will initially be a strong
increase in pH. But then we enter the \textbf{buffer region} and the pH will
platteau for awhile. Then the equivalence point will be reached and surpassed.

The half-equivalence point is when you add half the amout of base/acid required
to neutralize the acid/base you started with. This point is important because it
is where the pH equals the $\rm{pK_a}$. This is because in the
Henderson-Hasselbalch equation, the logarithm will be zero.

For a weak base titrated with a strong acid, the same effects will exists, but
the graph will be flipped.

\end{document}

