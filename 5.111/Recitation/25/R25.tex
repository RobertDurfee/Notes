\documentclass{article}
\usepackage{tikz}
\usepackage{float}
\usepackage{enumerate}
\usepackage{amsmath}
\usepackage{bm}
\usepackage{indentfirst}
\usepackage{siunitx}
\usepackage[utf8]{inputenc}
\usepackage{graphicx}
\graphicspath{ {Images/} }
\usepackage{float}
\usepackage{mhchem}
\usepackage{chemfig}
\allowdisplaybreaks

\title{ 5.111 Recitation 25 }
\author{ Robert Durfee }
\date{ December 14, 2017 }

\begin{document}

\maketitle

\section{ Catalysis }

The goal of \textbf{catalysis} is to speed up reactions that normally take a
really long time. To get from reactants to the products, the reaction must
overcome an energy barrier, or activation energy, $E_{a}$. Catalysts lower this
activation energy. Looking at the Ahrennius Equation:
$$ k = a e^{-E_{a} / RT} $$

The exponent term is increased by decreasing $E_{a}$. However, it is important
to notice that the overall $\Delta G$ of the reaction remains unchanged, as it
is a state function. As a result, the equilibrium constant also remains the
same.

\section{Enzymes}

\textbf{Enzymes} are the biological catalysts in organisms. A \textbf{substrate}
is bound to the enzyme and the enzyme performs a conformational change to the
substrate causing it to release from the enzyme in its new state. The general
reaction is:
$$ \ce{ S + E <=> ES -> P + E } $$

The rate of this reaction is:
$$ \mathrm{rate} \frac{ d[P] }{ dt } = k_{2}[ES] $$

However, it is hard to know the concentration of ES. So we will alter this
equation into terms that we know: $[E]_{0},\ [S]$. To do this, we will assume
the \textbf{steady-state approximation} which says the intermediate
concentration is unchanged.
$$ \frac{ d[ES] }{ dt } = k_{1}[E][S] - k_{-1}[ES] - k_{2}[ES] = 0 $$
$$ [E]_{0} = [E] + [ES] $$
$$ \frac{ d[ES] }{ dt } = k_{1}[E]_{0}[S] - k_{1}[ ES ][ S ] -k_{-1}[ES] -
k_{2}[ES] = 0 $$
$$ [ES] = \frac{ k_{1}[E]_{0}[S] }{ k_{-1} + k_{2} + k_{1}[S] } $$

We define a constant to simplify this even more, $K_{M}$ and $V_{max}=k_{2}[ E
]_{0}$:
$$ K_{M} = \frac{ k_{-1} + k_{2} }{ k_{1} }$$

Then, the rate becomes:
$$ \mathrm{rate} = \frac{ V_{max}[S] }{ K_{M} + [S] } $$

We can do some qualitative analysis on this equation. If $[S]$ is really large,
then the rate will be equal to $V_{max}$. If $K_{M} = [S]$, then the rate will
be half the maximum rate. This gives a physical meaning to the quantity of
$K_{M}$: it is essentially the affinity of the enzyme to the substrate.

\subsection{Example}

$$  K_{M} = 5 \cdot 10^{-5}\ \si{ M },\ k_{2} = 2 \cdot 10^{3} $$
$$ [E]_{0} = 6 \cdot 10^{-7}\ \si{ M } $$

What is $V_{max}$?
$$ V_{max} = k_{2}[E]_{0} = (2 \cdot 10^{3})(6 \cdot 10^{-7}) = 1.2 \cdot 10^{-3} $$

What is the concentration of the substrate if the rate is half of $V_{max}$?
$$ [S] = K_{M} = 5 \cdot 10^{-5} $$

\section{Nuclear Chemistry}

\subsection{Introduction}

Everything we have talked about so far is different elements binding to each
other and exchanging electrons, but the elements remain the same during the
interaction. \textbf{Nuclear chemistry} is the study of reactions that change
the nucleus of an element.

Notation:
$$ {}^{A}_{Z}X $$

$A$ is the number of \textbf{nucleons}, $Z$ is the number of protons, and $X$ is
the elemental symbol.

An electron is written as: ${}_{-1}^{0}e^{-}$. A \textbf{positron} is written as
${}_{1}^{0}e^{+}$. A proton is written as ${}_{1}^{1}p^{+}$. A neutron is
written as ${}_{0}^{1}n$.

An \textbf{isotope} has the same number of protons, different number of
neutrons. An \textbf{isotone} has the same number of neutrons, different number
of protons. An \textbf{isobar} has the same total mass, but a different number
of protons and neutrons.

\subsection{Nuclear Decay}

\textbf{Nuclear decay} is the decomposition reaction, but for nuclear chemistry.
Most elements found in nature have stable decomposition reactions, meaning that
they will, not decay into another element spontaneously. This makes sense
because if the reaction was non-spontaneous, it would've decayed into something
else.

The graph of number of neutrons vs number of protons is skewed towards higher
neutrons. This means that as the number of protons increases, the higher the
number of neutrons is needed.

In \textbf{alpha emission}, the alpha particle, ${}_{2}^{4}\ce{He}$, is emitted.
This usually occurs for atomic numbers greater than 83. These particles are
easily blocked and don't travel very far.
$$ \ce{{}_{92}^{238}U -> {}_{2}^{4}He + {}_{90}^{234}Th} $$

In \textbf{beta emission}, an electron is emitted from a neutron to form a
proton. This typically happens for elements on the top left of the stability
line. You can also have positron emission. This is medium dangerous: it can be
blocked by aluminum foil.
$$ \ce{{}^{234}_{90}Th -> {}_{-1}^{0}e^{-} + {}_{91}^{234}Th} $$
$$  \ce{{}^{11}_{6}C -> {}_{1}^{0}e^{+} + {}^{11}_{5}B } $$

In \textbf{gamma emission}, a high energy light particle is release. There is no
change in atomic number of number of nucleons. These are very dangerous. This
needs to be blocked by lead or concrete.

\subsection{Energy Changes}
$$ \ce{{}^{1}_{0}n -> {}^{1}_{1}p^{+} + {}_{-1}^{0}e^{-} } $$

We can calculate the change in mass during this reaction. This results in a
small, non-zero change in the mass of the products and reactants. Using
Einstein's famous equation, we can find the change in energy:
$$ \Delta E = \Delta m c^{2} $$

\subsection{Kinetics}

This is a decomposition reaction in the first order. This means the reaction is
unimolecular, as well. All the usual kinetics equations apply here.

\end{document}

