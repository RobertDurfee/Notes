\documentclass{article} 
\usepackage{tikz} 
\usepackage{float}
\usepackage{enumerate} 
\usepackage{amsmath} 
\usepackage{bm}
\usepackage{indentfirst} 
\usepackage{siunitx} 
\usepackage[utf8]{inputenc}
\usepackage{graphicx} 
\graphicspath{ {Images/} } 
\usepackage{float}
\usepackage{mhchem} 
\usepackage{chemfig} 
\allowdisplaybreaks

\title{ 5.111 Recitation 16 } 
\author{ Robert Durfee } 
\date{ November 9, 2017}

\begin{document}

\maketitle

\section{Acids and Bases}

$$\ce{HCOOH_{(aq)} + H_2O_{(l)} <=> H_3)+ + HCOO-_{(aq)}}$$ 
We start with 0.1 M of \ce{HCOOH}, then
$$K_a=1.8\cdot10^{-4}=\frac{[\ce{H_3O+}][\ce{HCOO-}]}{[\ce{HCOOH}]}=\frac{(x)(x)}{0.1-x}$$
Assume that $x$ is smaller than 0.1. Therefore: 
$$K_a=\frac{x^2}{0.1}$$
$$x=0.0042\ \si{M}$$ 
The percent deprotonation would be: 
$$\frac{0.0042\ \si{M}}{0.1\ \si{M}}=4.2\% < 5\%$$

These numbers are very small and don't give a good sense of the differences.
Therefore, we define pH.

$$pH=-\log([\ce{H_3O+}])$$

For the above example, $pH=2.38$. This shows that our solution is quite acidic.
We can also calculate $pOH$ which is defined as 

$$pOH=-\log([\ce{OH-}])=14-pH$$

For the above example, $pOH=11.62$.

\subsection{Water}

The equilibrium constant of water is defined through the following equations:

$$\ce{2H_2O <=> H_3O+ + OH-}$$ $$K_w=1.0\cdot10^{-14}$$ $$K_w=K_aK_b$$
$$14=pK_w=pK_a+pK_b$$

\section{Relative Strengths}

We are looking at the relative values of $K_a$ or $pK_a$. The larger the $K_a$
value, the stronger the solution. This, in turn, means that the smaller the
$pK_a$ value is, the stronger the solution is. And what we mean by strong is if
the $pK_a$ of $pK_b$ value is less than one. Solutions are weak if $pK_a$ or
$pK_b$ is between 1 and 14. And they are ineffective if the $pK_a$ or $pK_b$ is
greater than 14.

\subsection{Example} 
\begin{center} 
        \begin{tabular} {c c c c} 
                Acids  & $pK_a$ & Bases & $pK_b$\\ 
                $\ce{H_2SO_4}$ & -2 & $\ce{HSO_4-}$ & 16\\ 
                $\ce{HSO_4-}$ & 1.92 & $\ce{SO_4^{2-}}$ & 12.08\\ 
                $\ce{CH_3COOH}$ & 4.75 & $\ce{CH_3COO-}$ & 9.25\\ 
                $\ce{CH_3OH}$ & 15 & $\ce{CH_3O-}$ & -1\\ 
         \end{tabular} 
 \end{center}

In this example, $\ce{H_2SO_4}$ would be the strong acid, the second two are
weak acids, and the last one is ineffective. This pattern also follows in the reverse
with the bases to the right.

\section{Salt Solutions}

Suppose wee put $\ce{NaCN}$ in water. There are two ions $\ce{Na+}$ and $\ce{CN-}$.

$$\ce{Na+ + H_2O <=> NaOH + H+}$$
$$\ce{CN- + H_2O <=> HCN + OH-}$$

The first reaction's  $pK_a$ value is larger than 14, therefore it has no
contribution to the $pH$ value. The second reaction has a $pK_b$ value between
1 and 14, meaning it is a weak base and will have some contribution to $pH$.
Although there are two different species with two equilibrium, but we only have
to worried about the second reaction when solving for $ph$.

Support we put 0.100 \si{g} of $\ce{NaCN}$ in 0.500 \si{L} of $\ce{H_2o}$. Then
you start with $4.08\cdot10^{-3}\ \si{M}$. Using an ICE chart, we can determine:

$$K_b=1.6\cdot10^{-5}=\frac{[\ce{HCN}][\ce{OH-}]}{[\ce{CN-}]}=\frac{x^2}{4.08\cdot10^{-3}-x}$$

Solving for $x$:

$$x=[\ce{OH-}]=2.5\cdot10^{-14}\ \si{M}$$

Solving for $pOH$:

$$pOH=3.61$$

This shows that the reaction is slightly basic. How do we do this without
calculations? For example, $\ce{NH_4ClO_2}$ which will decompose into
$\ce{NH_4+}$ and $\ce{ClO_2-}$. The cation will have a $pK_a$ of 9.25 and the
anion reation will have a $pK_b$ of 12.04. We would ask the question of whether
the solution will be acidic or basic. The lower value wins. As a result, this
example, we will expect an acidic solution becase $pK_a<pK_b$.

\section{Buffers}

We insert an acid or base along with the conjugate base or acid, respectively.
As a result, we will use the Henderson-Hasselbach equation.

$$pH=pK_a+\log_{10}\left(\frac{[A^-]}{[HA]}\right)$$

If we have the following buffer system (which is present in blood):

$$\ce{H_2CO_3 + H_2O <=> HCO_3- + H_3O+}\ K_a=4.3\cdot10^{-7}$$
$$K_a=\frac{[\ce{HCO_3-}][\ce{H_3O+}]}{[\ce{H_2CO_3}]}$$
$$[\ce{H_3O-}]=K_a\frac{[\ce{H_2CO_3}]}{[\ce{HCO_3-}]}$$
$$pH=-\log([\ce{H_3O+}])$$
$$pH=pK_a-\log\left(\frac{[\ce{H_2CO_3}]}{[\ce{HCO_3-}]}\right)$$
$$pH=pK_a+\log\left(\frac{[\ce{HCO_3-}]}{[\ce{H_2CO_3}]}\right)$$

Above is how we determine the Henderson-Hasselbach equation.

\end{document}

