\documentclass{article}
\usepackage{tikz}
\usepackage{float}
\usepackage{enumerate}
\usepackage{amsmath}
\usepackage{bm}
\usepackage{indentfirst}
\usepackage{siunitx}
\usepackage[utf8]{inputenc}
\usepackage{graphicx}
\graphicspath{ {Images/} }
\usepackage{float}
\usepackage{mhchem}
\usepackage{chemfig}
\allowdisplaybreaks

\title{ 5.111 Recitation 19 }
\author{ Robert Durfee }
\date{ November 21, 2017 }

\begin{document}

\maketitle

\section{Gibb's Free Energy }
$$\ce{Cu + 2 Ag+ -> Cu^{2+} + 2 Ag}$$
$$\ce{Cu^{2+} + 2e- -> Cu}$$
$$\ce{Ag+ + e- -> Ag}$$

Where the half reaction's potentials are as follows:
$$E^o_{\ce{Cu}} = +0.34\ \si{V}$$
$$E^o_{\ce{Ag}} = +0.80\ \si{V}$$

We will solve this cell's potential using Gibb's Free Energy. For copper half
reaction:
$$\Delta G = -nFE$$
$$\Delta G = -(2)(96485)(+0.34 V)$$
$$\Delta G = -65.5\ \si{kJ\ mol^{-1}}$$

For silver half reaction:
$$\Delta G = -nFE$$
$$\Delta G = -(1)(96485)(+0.80 V)$$
$$\Delta G = -77.2\ \si{kJ\ mol^{-1}}$$

Now we add these reactions together to get the original reaction:
$$\ce{Cu -> Cu^{2+} + 2e-}$$
$$\ce{Ag+ + e- -> Ag}$$

The resulting Gibb's Free Energy is:
$$\Delta G = 65.6 + 2(-77.2) = -88.8\ \si{kJ\ mol^{-1}}$$

\subsection{Example}

\begin{center}
    \begin{tabular}{c c c}
        $\ce{Cu+ + e- -> Cu}$ & $E^o = +0.52\ \si{V}$ & $\Delta G = -50\ \si{kJ\
        mol^{-1}}$ \\
        $\ce{Cu^{2+} + e- -> Cu+}$ & $E^o = +0.16\ \si{V}$ & $\Delta G = -15\
        \si{kJ\ mol^{-1}}$ \\
        $\ce{Cu^{2+} + 2e- -> Cu}$ & $E^o \neq +0.68\ \si{V}$ & $\Delta G = -65\
        \si{kJ\ mol^{-1}}$
    \end{tabular}
\end{center}

The first solution is incorrect. We can't just add the two half reactions
potential. This is because in each half reaction, one electron is consumed, but
in the end, two electrons are consumed. 

$$\Delta G^o = -nFE^o$$
$$+0.34 = -(2)(96485)E^o$$
$$E^o = +0.34$$

\subsection{Example}

Nernst Equation
$$E = E^o - \frac{RT}{nF}\ln(Q)$$

The cell's reduction potential away from standard conditions:
$$E = 0.249\ \si{V}$$

Balancing the overall cell reaction:
$$\ce{H_2 -> H+ + 2e-}$$
$$\ce{AgCl + e- -> Ag + Cl-}$$
$$\ce{2AgCl + H_2 + 2e- -> 2e- + 2Ag + 2Cl- + H+}$$

Where the cell reduction for hydrogen reaction is $E^o = 0.00\ \si{V}$ and the
cell reduction for silver reaction is $E^o = 0.22\ \si{V}$.

The overall reaction cell standard reduction potential is $E^o = 0.22\ \si{V}$.

The reaction quotient can be written as:
$$Q = \frac{[\ce{H+}]^2[\ce{Cl-}]^2}{(P_{\ce{H_2}})}$$

Then the Nernst equation can be written out by combining all of these
components.

\end{document}

