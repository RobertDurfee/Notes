\documentclass{article}
\usepackage{tikz}
\usepackage{float}
\usepackage{enumerate}
\usepackage{amsmath}
\usepackage{bm}
\usepackage{indentfirst}
\usepackage{siunitx}
\usepackage[utf8]{inputenc}
\usepackage{graphicx}
\graphicspath{ {Images/} }
\usepackage{float}
\usepackage{mhchem}
\usepackage{chemfig}
\allowdisplaybreaks

\title{ 5.111 Recitation 21 }
\author{ Robert Durfee }
\date{ December 5, 2017 }

\begin{document}

\maketitle

\section{ Coordination Complexes }

\subsection{ Oxidation Number of Center Metal }

Example compound:
$$ \ce{ [ W( OH_{2} )_{6} ]^{4+} } $$

Before, we could draw the Lewis structure, but this gets too complicated with
coordination complexes. Remember that the sum of oxidation numbers must equal
the total charge. We also know that the total charge for $\ce{ H_{2}O }$ is
zero, so, in this case, tungsten's oxidation number is $+4$.

We can then use the oxidation number to determine the \textbf{d-count}. The
d-count is the group number minus the oxidation number. This number represents
the number of electrons in the d-orbital. For tungsten, it is in group 6 and has
an oxidation number of 4. Thus, the d-count is 2. This means there are two
electrons in the d-orbital. This is written as $d^{2}$. 

\subsection{ Crystal Field Diagram }

There are 5 different d-orbitals: $d_{xy}$, $d_{xz}$, $d_{yz}$, $d_{z^{2}}$, and
$d_{x^{2}-y^{2}}$. Conventionally, the ligands are placed along the $x$, $y$,
and $z$ axes. One is on the negative and the other is on the positive. When
determining the respective energies of the orbitals, you look at the overlap of
the ligands and the orbitals. The $d_{z^{2}}$ and $d_{x^{2}-y^{2}}$ have a lot
of overlap with the ligands. Thus, they have larger energy. The $d_{xy}$,
$d_{yz}$, and $d_{xz}$ have lower energy as they are destabilized less. 

The difference between the upper orbitals and the lower orbitals is called
$\Delta_{O}$. This amount changes based on the ligand that is in use. When we
say strong or weak field ligands, we mean how much overlap occurs between the
ligands and the central atoms.
$$ \ce{ I- } < \ce{ Br- } < \ce{ Cl- } < \ce{ F- } < \ce{ OH- } < \ce{ OH_{2} }
< \ce{ NH_{3} } < \ce{ CO } < \ce{ CN- }$$

The large $\Delta_{O}$ are on the right and the small $\Delta_{O}$ are on the
left. The ligands to the right of $\ce{ OH_{2} }$ are strong and the ligands to
the left of $\ce{ OH_{2} }$ are weak. The weak ligands have poor overlap of wave
functions and the strong ligands have large overlap.

The strength of ligands also determines the \textbf{spin} of the ligand. This
lets us know how to fill in the orbitals. The \textbf{strong ligands have low
spin} and the \textbf{weak ligands have high spin}. With low spin, fill lower
orbitals first with electrons paired. With high spin, fill each orbital first
before pairing electrons. 

Where there are unpaired electrons, it is \textbf{paramagnetic}. Where there
are no unpaired electrons, it is \textbf{dimagnetic}. 

The energy difference $\Delta_{O}$ determines the color of the coordination
complex. When we excite an electron with light and it goes from the lower
d-orbital to the higher d-orbital, it absorbes the color with the same energy.
If you shine white light and it absorbs red, then it will appear green (the
complement of red). If the compound absorbs yellow, it will appear violet.

\end{document}

