%
% 6.S077 problem set solutions template
%
\documentclass[12pt,twoside]{article}

\newcommand{\name}{}

\usepackage{amssymb}
\usepackage{amsmath}
\usepackage{graphicx}
\usepackage{latexsym}
\usepackage{times,url}
\usepackage{cprotect}
\usepackage{listings}
\usepackage{graphicx}
\usepackage[table]{xcolor}
\usepackage[letterpaper]{geometry}
\usepackage{tikz-qtree}
\usepackage{enumerate}

\newcommand{\profs}{Mauricio Karchmer, Aleksander Madry, Bruce Tidor}
\newcommand{\subj}{6.046}
\newcommand{\ttt}[1]{{\tt\small #1}}

\definecolor{dkgreen}{rgb}{0,0.6,0}
\definecolor{gray}{rgb}{0.5,0.5,0.5}
\definecolor{mauve}{rgb}{0.58,0,0.82}

\lstset{
  language=Python,
  aboveskip=1pc,
  belowskip=1pc,
  basicstyle={\footnotesize\ttfamily},
  numbers=left,
  showstringspaces=false,
  numberstyle=\tiny\color{gray},
  keywordstyle=\color{blue},
  commentstyle=\color{dkgreen},
  stringstyle=\color{mauve},
}

\tikzset{
  % every node/.style={minimum width=2em,draw,circle},
  % level 1/.style={sibling distance=2cm},
  level distance=1cm,
  edge from parent/.style=
  {draw,edge from parent path={(\tikzparentnode) -- (\tikzchildnode)}},
}

\newif\ifHideSolutions
\newcommand{\solution}[1]{\color{dkgreen}\textbf{Solution: }#1\color{black}}
\newcommand{\rubric}[1]{\color{dkgreen}{\bf Rubric:} #1\color{black}}

% \HideSolutionsfalse
% \ifHideSolutions
%   \renewcommand{\solution}[1]{}
%   \renewcommand{\rubric}[1]{}
% \fi

\newlength{\toppush}
\setlength{\toppush}{2\headheight}
\addtolength{\toppush}{\headsep}

\newcommand{\htitle}[2]{\noindent\vspace*{-\toppush}\newline\parbox{6.5in}
{\textit{Design and Analysis of Algorithms}\hfill\name\newline
Massachusetts Institute of Technology \hfill #2\newline
\profs\hfill #1 \vspace*{-.5ex}\newline
\mbox{}\hrulefill\mbox{}}\vspace*{1ex}\mbox{}\newline
\begin{center}{\Large\bf #1}\end{center}}

\newcommand{\handout}[2]{\thispagestyle{empty}
 \markboth{#1}{#1}
 \pagestyle{myheadings}\htitle{#1}{#2}}

\newcommand{\lecture}[3]{\thispagestyle{empty}
 \markboth{Lecture #1: #2}{Lecture #1: #2}
 \pagestyle{myheadings}\htitle{Lecture #1: #2}{#3}}

\newcommand{\htitlewithouttitle}[2]{\noindent\vspace*{-\toppush}\newline\parbox{6.5in}
{\textit{Design and Analysis of Algorithms}\hfill#2\newline
Massachusetts Institute of Technology \hfill 6.046\newline
\profs\hfill Handout #1\vspace*{-.5ex}\newline
\mbox{}\hrulefill\mbox{}}\vspace*{1ex}\mbox{}\newline}

\newcommand{\handoutwithouttitle}[2]{\thispagestyle{empty}
 \markboth{Handout \protect\ref{#1}}{Handout \protect\ref{#1}}
 \pagestyle{myheadings}\htitlewithouttitle{\protect\ref{#1}}{#2}}

\newcommand{\exam}[2]{% parameters: exam name, date
 \thispagestyle{empty}
 \markboth{\hspace{1cm}\subj\ #1\hspace{1in}Name\hrulefill\ \ }%
          {\subj\ #1\hspace{1in}Name\hrulefill\ \ }
 \pagestyle{myheadings}\examtitle{#1}{#2}
 \renewcommand{\theproblem}{Problem \arabic{problemnum}}
}
\newcommand{\examsolutions}[3]{% parameters: handout, exam name, date
 \thispagestyle{empty}
 \markboth{Handout \protect\ref{#1}: #2}{Handout \protect\ref{#1}: #2}
% \pagestyle{myheadings}\htitle{\protect\ref{#1}}{#2}{#3}
 \pagestyle{myheadings}\examsolutionstitle{\protect\ref{#1}} {#2}{#3}
 \renewcommand{\theproblem}{Problem \arabic{problemnum}}
}
\newcommand{\examsolutionstitle}[3]{\noindent\vspace*{-\toppush}\newline\parbox{6.5in}
{\textit{Design and Analysis of Algorithms}\hfill#3\newline
Massachusetts Institute of Technology \hfill 6.046\newline
%Singapore-MIT Alliance \hfill SMA5503\newline
\profs\hfill Handout #1\vspace*{-.5ex}\newline
\mbox{}\hrulefill\mbox{}}\vspace*{1ex}\mbox{}\newline
\begin{center}{\Large\bf #2}\end{center}}

\newcommand{\takehomeexam}[2]{% parameters: exam name, date
 \thispagestyle{empty}
 \markboth{\subj\ #1\hfill}{\subj\ #1\hfill}
 \pagestyle{myheadings}\examtitle{#1}{#2}
 \renewcommand{\theproblem}{Problem \arabic{problemnum}}
}

\makeatletter
\newcommand{\exambooklet}[2]{% parameters: exam name, date
 \thispagestyle{empty}
 \markboth{\subj\ #1}{\subj\ #1}
 \pagestyle{myheadings}\examtitle{#1}{#2}
 \renewcommand{\theproblem}{Problem \arabic{problemnum}}
 \renewcommand{\problem}{\newpage
 \item \let\@currentlabel=\theproblem
 \markboth{\subj\ #1, \theproblem}{\subj\ #1, \theproblem}}
}
\makeatother


\newcommand{\examtitle}[2]{\noindent\vspace*{-\toppush}\newline\parbox{6.5in}
{\textit{Design and Analysis of Algorithms}\hfill#2\newline
Massachusetts Institute of Technology \hfill 6.046 Spring 2019\newline
%Singapore-MIT Alliance \hfill SMA5503\newline
\profs\hfill #1\vspace*{-.5ex}\newline
\mbox{}\hrulefill\mbox{}}\vspace*{1ex}\mbox{}\newline
\begin{center}{\Large\bf #1}\end{center}}

\newcommand{\grader}[1]{\hspace{1cm}\textsf{\textbf{#1}}\hspace{1cm}}

\newcommand{\points}[1]{[#1 points]\ }
\newcommand{\parts}[1]
{
  \ifnum#1=1
  (1 part)
  \else
  (#1 parts)
  \fi
  \ 
}

\newcommand{\bparts}{\begin{problemparts}}
\newcommand{\eparts}{\end{problemparts}}
\newcommand{\ppart}{\problempart}

%\newcommand{\lg} {lg\ }

\setlength{\oddsidemargin}{0pt}
\setlength{\evensidemargin}{0pt}
\setlength{\textwidth}{6.5in}
\setlength{\topmargin}{0in}
\setlength{\textheight}{8.5in}


\newcommand{\Spawn}{{\bf spawn} }
\newcommand{\Sync}{{\bf sync}}

\newcommand{\cif}[1]{\mbox{if $#1$}}
\newcommand{\cwhen}[1]{\mbox{when $#1$}}

\newcounter{problemnum}
\newcommand{\theproblem}{Problem \theproblemsetnum-\arabic{problemnum}}
\newenvironment{problems}{
        \begin{list}{{\bf \theproblem. \hspace*{0.5em}}}
        {\setlength{\leftmargin}{0em}
         \setlength{\rightmargin}{0em}
         \setlength{\labelwidth}{0em}
         \setlength{\labelsep}{0em}
         \usecounter{problemnum}}}{\end{list}}
\makeatletter
\newcommand{\problem}[1][{}]{\item \let\@currentlabel=\theproblem \textbf{#1}}
\makeatother

\newcounter{problempartnum}[problemnum]
\newenvironment{problemparts}{
        \begin{list}{{\bf (\alph{problempartnum})}}
        {\setlength{\leftmargin}{2.5em}
         \setlength{\rightmargin}{2.5em}
         \setlength{\labelsep}{0.5em}}}{\end{list}}
\newcommand{\problempart}{\addtocounter{problempartnum}{1}\item}

\newenvironment{truefalseproblemparts}{
        \begin{list}{{\bf (\alph{problempartnum})\ \ \ T\ \ F\hfil}}
        {\setlength{\leftmargin}{4.5em}
         \setlength{\rightmargin}{2.5em}
         \setlength{\labelsep}{0.5em}
         \setlength{\labelwidth}{4.5em}}}{\end{list}}

\newcounter{exercisenum}
\newcommand{\theexercise}{Exercise \theproblemsetnum-\arabic{exercisenum}}
\newenvironment{exercises}{
        \begin{list}{{\bf \theexercise. \hspace*{0.5em}}}
        {\setlength{\leftmargin}{0em}
         \setlength{\rightmargin}{0em}
         \setlength{\labelwidth}{0em}
         \setlength{\labelsep}{0em}
        \usecounter{exercisenum}}}{\end{list}}
\makeatletter
\newcommand{\exercise}{\item \let\@currentlabel=\theexercise}
\makeatother

\newcounter{exercisepartnum}[exercisenum]
%\newcommand{\problem}[1]{\medskip\mbox{}\newline\noindent{\bf Problem #1.}\hspace*{1em}}
%\newcommand{\exercise}[1]{\medskip\mbox{}\newline\noindent{\bf Exercise #1.}\hspace*{1em}}

\newenvironment{exerciseparts}{
        \begin{list}{{\bf (\alph{exercisepartnum})}}
        {\setlength{\leftmargin}{2.5em}
         \setlength{\rightmargin}{2.5em}
         \setlength{\labelsep}{0.5em}}}{\end{list}}
\newcommand{\exercisepart}{\addtocounter{exercisepartnum}{1}\item}


% Macros to make captions print with small type and 'Figure xx' in bold.
\makeatletter
\def\fnum@figure{{\bf Figure \thefigure}}
\def\fnum@table{{\bf Table \thetable}}
\let\@mycaption\caption
%\long\def\@mycaption#1[#2]#3{\addcontentsline{\csname
%  ext@#1\endcsname}{#1}{\protect\numberline{\csname 
%  the#1\endcsname}{\ignorespaces #2}}\par
%  \begingroup
%    \@parboxrestore
%    \small
%    \@makecaption{\csname fnum@#1\endcsname}{\ignorespaces #3}\par
%  \endgroup}
%\def\mycaption{\refstepcounter\@captype \@dblarg{\@mycaption\@captype}}
%\makeatother
\let\mycaption\caption
%\newcommand{\figcaption}[1]{\mycaption[]{#1}}

\newcounter{totalcaptions}
\newcounter{totalart}

\newcommand{\figcaption}[1]{\addtocounter{totalcaptions}{1}\caption[]{#1}}

% \psfigures determines what to do for figures:
%       0 means just leave vertical space
%       1 means put a vertical rule and the figure name
%       2 means insert the PostScript version of the figure
%       3 means put the figure name flush left or right
\newcommand{\psfigures}{0}
\newcommand{\spacefigures}{\renewcommand{\psfigures}{0}}
\newcommand{\rulefigures}{\renewcommand{\psfigures}{1}}
\newcommand{\macfigures}{\renewcommand{\psfigures}{2}}
\newcommand{\namefigures}{\renewcommand{\psfigures}{3}}

\newcommand{\figpart}[1]{{\bf (#1)}\nolinebreak[2]\relax}
\newcommand{\figparts}[2]{{\bf (#1)--(#2)}\nolinebreak[2]\relax}


\macfigures     % STATE

% When calling \figspace, make sure to leave a blank line afterward!!
% \widefigspace is for figures that are more than 28pc wide.
\newlength{\halffigspace} \newlength{\wholefigspace}
\newlength{\figruleheight} \newlength{\figgap}
\newcommand{\setfiglengths}{\ifnum\psfigures=1\setlength{\figruleheight}{\hruleheight}\setlength{\figgap}{1em}\else\setlength{\figruleheight}{0pt}\setlength{\figgap}{0em}\fi}
\newcommand{\figspace}[2]{\ifnum\psfigures=0\leavefigspace{#1}\else%
\setfiglengths%
\setlength{\wholefigspace}{#1}\setlength{\halffigspace}{.5\wholefigspace}%
\rule[-\halffigspace]{\figruleheight}{\wholefigspace}\hspace{\figgap}#2\fi}
\newlength{\widefigspacewidth}
% Make \widefigspace put the figure flush right on the text page.
\newcommand{\widefigspace}[2]{
\ifnum\psfigures=0\leavefigspace{#1}\else%
\setfiglengths%
\setlength{\widefigspacewidth}{28pc}%
\addtolength{\widefigspacewidth}{-\figruleheight}%
\setlength{\wholefigspace}{#1}\setlength{\halffigspace}{.5\wholefigspace}%
\makebox[\widefigspacewidth][r]{#2\hspace{\figgap}}\rule[-\halffigspace]{\figruleheight}{\wholefigspace}\fi}
\newcommand{\leavefigspace}[1]{\setlength{\wholefigspace}{#1}\setlength{\halffigspace}{.5\wholefigspace}\rule[-\halffigspace]{0em}{\wholefigspace}}

% Commands for including figures with macpsfig.
% To use these commands, documentstyle ``macpsfig'' must be specified.
\newlength{\macfigfill}
\makeatother
\newlength{\bbx}
\newlength{\bby}
\newcommand{\macfigure}[5]{\addtocounter{totalart}{1}
\ifnum\psfigures=2%
\setlength{\bbx}{#2}\addtolength{\bbx}{#4}%
\setlength{\bby}{#3}\addtolength{\bby}{#5}%
\begin{flushleft}
\ifdim#4>28pc\setlength{\macfigfill}{#4}\addtolength{\macfigfill}{-28pc}\hspace*{-\macfigfill}\fi%
\mbox{\psfig{figure=./#1.ps,%
bbllx=#2,bblly=#3,bburx=\bbx,bbury=\bby}}
\end{flushleft}%
\else\ifdim#4>28pc\widefigspace{#5}{#1}\else\figspace{#5}{#1}\fi\fi}
\makeatletter

\newlength{\savearraycolsep}
\newcommand{\narrowarray}[1]{\setlength{\savearraycolsep}{\arraycolsep}\setlength{\arraycolsep}{#1\arraycolsep}}
\newcommand{\normalarray}{\setlength{\arraycolsep}{\savearraycolsep}}

\newcommand{\hint}{{\em Hint:\ }}

% Macros from /th/u/clr/mac.tex

\newcommand{\set}[1]{\left\{ #1 \right\}}
\newcommand{\abs}[1]{\left| #1\right|}
\newcommand{\card}[1]{\left| #1\right|}
\newcommand{\floor}[1]{\left\lfloor #1 \right\rfloor}
\newcommand{\ceil}[1]{\left\lceil #1 \right\rceil}
\newcommand{\ang}[1]{\ifmmode{\left\langle #1 \right\rangle}
   \else{$\left\langle${#1}$\right\rangle$}\fi}
        % the \if allows use outside mathmode,
        % but will swallow following space there!
\newcommand{\paren}[1]{\left( #1 \right)}
\newcommand{\bracket}[1]{\left[ #1 \right]}
\newcommand{\prob}[1]{\Pr\left\{ #1 \right\}}
\newcommand{\Var}{\mathop{\rm Var}\nolimits}
\newcommand{\expect}[1]{{\rm E}\left[ #1 \right]}
\newcommand{\expectsq}[1]{{\rm E}^2\left[ #1 \right]}
\newcommand{\variance}[1]{{\rm Var}\left[ #1 \right]}
\renewcommand{\choose}[2]{{{#1}\atopwithdelims(){#2}}}
\def\pmod#1{\allowbreak\mkern12mu({\rm mod}\,\,#1)}
\newcommand{\matx}[2]{\left(\begin{array}{*{#1}{c}}#2\end{array}\right)}
\newcommand{\Adj}{\mathop{\rm Adj}\nolimits}

\newtheorem{theorem}{Theorem}
\newtheorem{lemma}[theorem]{Lemma}
\newtheorem{corollary}[theorem]{Corollary}
\newtheorem{xample}{Example}
\newtheorem{definition}{Definition}
\newenvironment{example}{\begin{xample}\rm}{\end{xample}}
\newcommand{\proof}{\noindent{\em Proof.}\hspace{1em}}
\def\squarebox#1{\hbox to #1{\hfill\vbox to #1{\vfill}}}
\newcommand{\qedbox}{\vbox{\hrule\hbox{\vrule\squarebox{.667em}\vrule}\hrule}}
\newcommand{\qed}{\nopagebreak\mbox{}\hfill\qedbox\smallskip}
\newcommand{\eqnref}[1]{(\protect\ref{#1})}

%%\newcommand{\twodots}{\mathinner{\ldotp\ldotp}}
\newcommand{\transpose}{^{\mbox{\scriptsize \sf T}}}
\newcommand{\amortized}[1]{\widehat{#1}}

\newcommand{\punt}[1]{}

%%% command for putting definitions into boldface
% New style for defined terms, as of 2/23/88, redefined by THC.
\newcommand{\defn}[1]{{\boldmath\textit{\textbf{#1}}}}
\newcommand{\defi}[1]{{\textit{\textbf{#1\/}}}}

\newcommand{\red}{\leq_{\rm P}}
\newcommand{\lang}[1]{%
\ifmmode\mathord{\mathcode`-="702D\rm#1\mathcode`\-="2200}\else{\rm#1}\fi}

%\newcommand{\ckt}[1]{\ifmmode\mathord{\mathcode`-="702D\sc #1\mathcode`\-="2200}\else$\mathord{\mathcode`-="702D\sc #1\mathcode`\-="2200}$\fi}
\newcommand{\ckt}[1]{\ifmmode \sc #1\else$\sc #1$\fi}

%% Margin notes - use \notesfalse to turn off notes.
\setlength{\marginparwidth}{0.6in}
\reversemarginpar
\newif\ifnotes
\notestrue
\newcommand{\longnote}[1]{
  \ifnotes
    {\medskip\noindent Note: \marginpar[\hfill$\Longrightarrow$]
      {$\Longleftarrow$}{#1}\medskip}
  \fi}
\newcommand{\note}[1]{
  \ifnotes
    {\marginpar{\tiny \raggedright{#1}}}
  \fi}


\newcommand{\reals}{\mathbbm{R}}
\newcommand{\integers}{\mathbbm{Z}}
\newcommand{\naturals}{\mathbbm{N}}
\newcommand{\rationals}{\mathbbm{Q}}
\newcommand{\complex}{\mathbbm{C}}

\newcommand{\oldreals}{{\bf R}}
\newcommand{\oldintegers}{{\bf Z}}
\newcommand{\oldnaturals}{{\bf N}}
\newcommand{\oldrationals}{{\bf Q}}
\newcommand{\oldcomplex}{{\bf C}}

\newcommand{\w}{\omega}                 %% for fft chapter

\newenvironment{closeitemize}{\begin{list}
{$\bullet$}
{\setlength{\itemsep}{-0.2\baselineskip}
\setlength{\topsep}{0.2\baselineskip}
\setlength{\parskip}{0pt}}}
{\end{list}}

% These are necessary within a {problems} environment in order to restore
% the default separation between bullets and items.
\newenvironment{normalitemize}{\setlength{\labelsep}{0.5em}\begin{itemize}}
                              {\end{itemize}}
\newenvironment{normalenumerate}{\setlength{\labelsep}{0.5em}\begin{enumerate}}
                                {\end{enumerate}}

%\def\eqref#1{Equation~(\ref{eq:#1})}
%\newcommand{\eqref}[1]{Equation (\ref{eq:#1})}
\newcommand{\eqreftwo}[2]{Equations (\ref{eq:#1}) and~(\ref{eq:#2})}
\newcommand{\ineqref}[1]{Inequality~(\ref{ineq:#1})}
\newcommand{\ineqreftwo}[2]{Inequalities (\ref{ineq:#1}) and~(\ref{ineq:#2})}

\newcommand{\figref}[1]{Figure~\ref{fig:#1}}
\newcommand{\figreftwo}[2]{Figures \ref{fig:#1} and~\ref{fig:#2}}

\newcommand{\liref}[1]{line~\ref{li:#1}}
\newcommand{\Liref}[1]{Line~\ref{li:#1}}
\newcommand{\lirefs}[2]{lines \ref{li:#1}--\ref{li:#2}}
\newcommand{\Lirefs}[2]{Lines \ref{li:#1}--\ref{li:#2}}
\newcommand{\lireftwo}[2]{lines \ref{li:#1} and~\ref{li:#2}}
\newcommand{\lirefthree}[3]{lines \ref{li:#1}, \ref{li:#2}, and~\ref{li:#3}}

\newcommand{\lemlabel}[1]{\label{lem:#1}}
\newcommand{\lemref}[1]{Lemma~\ref{lem:#1}} 

\newcommand{\exref}[1]{Exercise~\ref{ex:#1}}

\newcommand{\handref}[1]{Handout~\ref{#1}}

\newcommand{\defref}[1]{Definition~\ref{def:#1}}

% (1997.8.16: Victor Luchangco)
% Modified \hlabel to only get date and to use handouts counter for number.
%   New \handout and \handoutwithouttitle commands in newmac.tex use this.
%   The date is referenced by <label>-date.
%   (Retained old definition as \hlabelold.)
%   Defined \hforcelabel to use an argument instead of the handouts counter.

\newcounter{handouts}
\setcounter{handouts}{0}

\newcommand{\hlabel}[2]{%
\stepcounter{handouts}
{\edef\next{\write\@auxout{\string\newlabel{#1}{{\arabic{handouts}}{0}}}}\next}
\write\@auxout{\string\newlabel{#1-date}{{#2}{0}}}
}

\newcommand{\hforcelabel}[3]{%          Does not step handouts counter.
\write\@auxout{\string\newlabel{#1}{{#2}{0}}}
\write\@auxout{\string\newlabel{#1-date}{{#3}{0}}}}


% less ugly underscore
% --juang, 2008 oct 05
\renewcommand{\_}{\vrule height 0 pt depth 0.4 pt width 0.5 em \,}

\usepackage{enumitem}
\newcommand{\theproblemsetnum}{5}
\newcommand{\releasedate}{Friday, April 19}
\newcommand{\partaduedate}{Wednesday, May 1}

\title{14.32 Problem Set \theproblemsetnum}

\begin{document}

\handout{Problem Set \theproblemsetnum}{\releasedate}
\textbf{All parts are due {\bf \partaduedate} at {\bf 9:00AM}}.

\setlength{\parindent}{0pt}
\medskip\hrulefill\medskip

{\bf Name:} Robert Durfee

\medskip

{\bf Collaborators:} None

\medskip\hrulefill

\begin{problems}

\problem  % Problem 1

\begin{problemparts}

\problempart  % Problem 1a

There are a few reasons why $x_i$ might be endogenous. Here are a couple:

\begin{itemize}
    \item {\bf Omitted variable bias}: It is possible that the number of
    schools per student is correlated to the wealth of the area. Furthermore,
    the wealth of the area is likely correlated to how well students perform
    on exams.
    \item {\bf Reverse causality}: It is possible that an increase in student
    exam scores could cause an increase in number of schools in the region.
    For example, if an area experiences a shock resulting in an increase in
    student exam scores, some private/charter schools might come to the area
    looking to capitalize on the more intelligent student base.
\end{itemize}

\problempart  % Problem 1b

There are two criteria to consider for an instrument:

\begin{itemize}
    \item {\bf Relevance}: First we must consider if the instrument is
    correlated to the explanatory variable. In this case, we must convince
    ourselves that a major river running through a city will have an effect
    on the number of schools in the city. As a stream reduces the space for
    schools and other real estate, it is plausible that the presence of a
    stream would affect the number of schools. I think it is worth testing
    this condition as the number of student in the area might go down
    proportionally thereby not significantly decreasing the number of schools
    per 1,000 students.
    \item {\bf Exclusion}: Next we must consider if the instrument is
    correlated to the error term. In this case, we must convince ourselves
    that a major river running through a city will have no direct affect on
    the student's exam scores. This certainly seems plausible as proximity to
    a body of water shouldn't alter an individual's intelligence.
\end{itemize}

\problempart  % Problem 1c

In this case we are dealing with a just-identified regression. Since we only
have a single regressor, this can be written as,
$$ \hat{\beta}_1^{\mathrm{IV}} = \frac{\hat{\cov}(z_i, y_i)}{\hat{\cov}(z_i, x_i)} $$
In the more general case, this is equivalent to,
$$ \hat{\beta}_1^{\mathrm{IV}} = (Z^T X)^{-1} Z^T Y $$

\problempart  % Problem 1d

Starting with the estimator from Part C:
\begin{align*}
    \hat{\beta}_1^{\mathrm{IV}} &= \frac{\hat{\cov}(z_i, y_i)}{\hat{\cov}(z_i, x_i)} \\
    &= \frac{\hat{\mathbb{E}}[z_i y_i] - \hat{\mathbb{E}}[z_i] \cdot
    \hat{\mathbb{E}}[y_i]}{\hat{\mathbb{E}}[z_i x_i] - \hat{\mathbb{E}}[z_i] \cdot
    \hat{\mathbb{E}}[x_i]} \\
    &= \frac{p \cdot \hat{\mathbb{E}}[z_i y_i \mid z_i = 1] + (1 - p) \cdot
    \hat{\mathbb{E}}[z_i y_i \mid z_i = 0] - p \cdot \hat{\mathbb{E}}[y_i]}{p
    \cdot \hat{\mathbb{E}}[z_i x_i \mid z_i = 1] + (1 - p) \cdot
    \hat{\mathbb{E}}[z_i x_i \mid z_i = 0] - p \cdot \hat{\mathbb{E}}[x_i])} \\
    &= \frac{p \cdot \hat{\mathbb{E}}[y_i \mid z_i = 1] - p \cdot \left(p
    \cdot \hat{\mathbb{E}}[y_i \mid z_i = 1] + (1 - p) \cdot
    \hat{\mathbb{E}}[y_i \mid z_i = 0]\right)}{p \cdot \hat{\mathbb{E}}[x_i
    \mid z_i = 1] - p \cdot \left(p \cdot \hat{\mathbb{E}}[x_i \mid z_i = 1]
    + (1 - p) \cdot \hat{\mathbb{E}}[x_i \mid z_i = 0]\right)} \\
    &= \frac{p \cdot (1 - p) \cdot \left(\hat{\mathbb{E}}[y_i \mid z_i = 1] -
    \hat{\mathbb{E}}[y_i \mid z_i = 0]\right)}{p \cdot (1 - p) \cdot
    \left(\hat{\mathbb{E}}[x_i \mid z_i = 1] - \hat{\mathbb{E}}[x_i \mid z_i
    = 0]\right)} \\
    &= \frac{\hat{\mathbb{E}}[y_i \mid z_i = 1] - \hat{\mathbb{E}}[y_i \mid
    z_i = 0]}{\hat{\mathbb{E}}[x_i \mid z_i = 1] - \hat{\mathbb{E}}[x_i \mid
    z_i = 0]}
\end{align*}

\end{problemparts}

\newpage

\problem  % Problem 2

\begin{problemparts}

\problempart % Problem 2a

A very simple model (using only the current $r_t$ and $g_t$, but past terms
could be added) accounting for quarterly seasonality (quarter indicators
$q_1$, $q_2$, and $q_3$) and linear time trends (using time variable $t$)
could be,
$$ y_t = \beta_0 + \beta_1 r_t + \beta_2 g_t + \beta_3 q_1 + \beta_4 q_2 +
\beta_5 q_3 + \beta_6 t + u_t $$

\problempart % Problem 2b

Serial correlation makes OLS no longer BLUE. Although the coefficient
estimates will remain consistent and unbiased, the variance estimates will be
inconsistent and biased. As a result, any hypothesis tests or confidence
intervals will be incorrect. More often than not, the coefficient will appear
more precise than it actually is.

\problempart % Problem 2c

To test for simple, weakly dependent, serial correlation, one can use the
$\mathrm{AR}(1)$ test. To run this test, run OLS on the original model stated
in Part A. Then calculate the residuals $u_t$. Run the following regression
on the residuals:
$$ u_t = \gamma u_{t - 1} + \varepsilon_t $$
If the coefficient $\gamma$ is statistically significant (the confidence
interval does not contain zero), then there is serial correlation. If the
coefficient is not statistically significant, there is no serial correlation.

\problempart % Problem 2d

If there is serial correlation present, one should use an autoregressive
or moving average model (assuming the dependence is weak) or a first
differences model (if the dependence is strong).

Assuming the dependence is weak, for the autoregressive model, choose a $p$
and run the following regression,
$$ y_t = \beta_0 + \beta_1 r_t + \beta_2 g_t + \beta_3 q_1 + \beta_4 q_2 +
\beta_5 q_3 + \beta_6 t + \beta_{6 + 1} y_{t - 1} + \ldots + \beta_{6 + p}
y_{t - p} + u_t $$

\end{problemparts}

\newpage

\problem  % Problem 3

\begin{problemparts}

\problempart % Problem 3a

Let $\varepsilon_{it} = \alpha_i + \eta_{it}$. It is possible that $x_{it}$
is correlated with $\varepsilon_{it}$ if $x_{it}$ is correlated with
$\alpha_i$. Suppose that states' population and number of high-security
prisons have an inverse correlation. That is, more remote, less-populous
states have more high-security prisons than other, more-populous states. In
this situation, it is likely that prison guards in the less-populous states
would be more likely to join a union for collective bargaining to ensure they
aren't overworked, etc. Furthermore, these less-populated states would likely
provide higher pensions to attract more workers to their prisons (along with
higher wages, etc.). All this would cause a correlation of $x_{it}$ with
$\alpha_i$ from omitted variable bias and thus also with $\varepsilon_{it}$.

\problempart % Problem 3b

The first differences model is defined as,
\begin{align*}
    y_{it} - y_{it - 1} &= \beta_0 + \beta_1 x_{it} + \alpha_i + u_{it} - \beta_0 - \beta_1 x_{it - 1} - \alpha_i - u_{it-1} \\
    \Delta y_{it} &= \beta_1 \Delta x_{it} + \Delta u_{it} \\
\end{align*}
Using this model, the estimate for $\beta_1$ is given by,
\begin{align*}
    \plim \hat{\beta}_1 &= \plim\left(\frac{\hat{\cov}(\Delta x_{it}, \Delta
    y_{it})}{\hat{\var}(\Delta x_{it})}\right) \\
    &= \plim \left(\frac{\hat{\cov}(\Delta x_{it}, \beta_1 \Delta x_{it} +
    \Delta u_{it})}{\hat{\var}(\Delta x_{it})} \right)\\
    &= \plim \left(\frac{\beta_1 \hat{\var}(\Delta x_{it}) +
    \hat{\cov}(\Delta x_{it}, \Delta u_{it})}{\hat{\var}(\Delta x_{it})}
    \right)\\
    &= \plim \left(\beta_1 + \frac{\hat{\cov}(x_{it} - x_{it-1}, u_{it} -
    u_{it-1})}{\hat{\var}(x_{it} - x_{it-1})}\right) \\
    &= \plim \left(\beta_1 + \frac{\hat{\cov}(x_{it}, u_{it}) -
    \hat{\cov}(x_{it}, u_{it-1}) - \hat{\cov}(x_{it-1},u_{it}) +
    \hat{\cov}(x_{it-1},u_{it-1})}{\hat{\var}(x_{it} - x_{it-1})}\right) \\
    &= \beta_1 + \frac{\cov(x_{it}, u_{it}) - \cov(x_{it}, u_{it-1})
    -\cov(x_{it-1},u_{it}) + \cov(x_{it-1},u_{it-1})}{\var(x_{it} -
    x_{it-1})} \\
\end{align*}
If we assume the following (strict exogeneity),
$$\cov(u_{it}, x_{it}) = 0,\ \cov(u_{it-1}, x_{it}) = 0,\ \cov(u_{it},
x_{it-1}) = 0\quad \forall i, t $$
Then this simplifies to,
$$ \plim \hat{\beta}_1 = \beta_1 $$
Therefore, the estimate is consistent.

\problempart % Problem 3c

The fixed effects models is defined as,
\begin{align*}
    y_{it} - \bar{y}_i &= \beta_0 + \beta_1 x_{it} + \alpha_i + u_{it} -
    \beta_0 - \beta_1 \bar{x}_i - \bar{\alpha}_i - \bar{u}_i \\
    \tilde{y}_{it} &= \beta_1 \tilde{x}_{it} + \tilde{u}_{it}
\end{align*}
This comes from the fact that $\alpha_i = \bar{\alpha}_i$. Using this model,
the estimate for $\beta_1$ is given by,
\begin{align*}
    \plim \hat{\beta}_1 &= \plim \left( \frac{\hat{\cov}(\tilde{x}_{it},
    \tilde{y}_{it})}{\hat{\var}(\tilde{x}_{it})} \right) \\
    &= \plim \left(\frac{\hat{\cov}(\tilde{x}_{it}, \beta_1 \tilde{x}_{it} +
    \tilde{u}_{it})}{\hat{\var}(\tilde{x}_{it})}\right) \\
    &= \plim \left( \frac{\beta_1 \hat{\var}(\tilde{x}_{it}) +
    \hat{\cov}(\tilde{x}_{it},
    \tilde{u}_{it})}{\hat{\var}(\tilde{x}_{it})}\right) \\
    &= \plim \left(\beta_1 + \frac{\hat{\cov}(x_{it} - \bar{x}_i, u_{it} -
    \bar{u}_i)}{\hat{\var}(x_{it} - \bar{x}_i)}\right) \\
    &= \beta_1 + \frac{\cov(x_{it} - \mu_{x_i}, u_{it} -
    \mu_{u_i})}{\var(x_{it} - \mu_{x_i})} \\
    &= \beta_1 + \frac{\cov(x_{it}, u_{it})}{\var(x_{it})} \\
\end{align*}
If we assume the following (strict exogeneity),
$$ \cov(x_{it}, u_{it}) = 0\quad \forall i, t $$
Then this simplifies to,
$$ \plim \hat{\beta}_1 = \beta_1 $$
Therefore, the estimate is consistent.

\end{problemparts}

\end{problems}

\end{document}
