
% Default to the notebook output style

    


% Inherit from the specified cell style.




    
\documentclass[11pt]{article}

    
    
    \usepackage[T1]{fontenc}
    % Nicer default font (+ math font) than Computer Modern for most use cases
    \usepackage{mathpazo}

    % Basic figure setup, for now with no caption control since it's done
    % automatically by Pandoc (which extracts ![](path) syntax from Markdown).
    \usepackage{graphicx}
    % We will generate all images so they have a width \maxwidth. This means
    % that they will get their normal width if they fit onto the page, but
    % are scaled down if they would overflow the margins.
    \makeatletter
    \def\maxwidth{\ifdim\Gin@nat@width>\linewidth\linewidth
    \else\Gin@nat@width\fi}
    \makeatother
    \let\Oldincludegraphics\includegraphics
    % Set max figure width to be 80% of text width, for now hardcoded.
    \renewcommand{\includegraphics}[1]{\Oldincludegraphics[width=.8\maxwidth]{#1}}
    % Ensure that by default, figures have no caption (until we provide a
    % proper Figure object with a Caption API and a way to capture that
    % in the conversion process - todo).
    \usepackage{caption}
    \DeclareCaptionLabelFormat{nolabel}{}
    \captionsetup{labelformat=nolabel}

    \usepackage{adjustbox} % Used to constrain images to a maximum size 
    \usepackage{xcolor} % Allow colors to be defined
    \usepackage{enumerate} % Needed for markdown enumerations to work
    \usepackage{geometry} % Used to adjust the document margins
    \usepackage{amsmath} % Equations
    \usepackage{amssymb} % Equations
    \usepackage{textcomp} % defines textquotesingle
    % Hack from http://tex.stackexchange.com/a/47451/13684:
    \AtBeginDocument{%
        \def\PYZsq{\textquotesingle}% Upright quotes in Pygmentized code
    }
    \usepackage{upquote} % Upright quotes for verbatim code
    \usepackage{eurosym} % defines \euro
    \usepackage[mathletters]{ucs} % Extended unicode (utf-8) support
    \usepackage[utf8x]{inputenc} % Allow utf-8 characters in the tex document
    \usepackage{fancyvrb} % verbatim replacement that allows latex
    \usepackage{grffile} % extends the file name processing of package graphics 
                         % to support a larger range 
    % The hyperref package gives us a pdf with properly built
    % internal navigation ('pdf bookmarks' for the table of contents,
    % internal cross-reference links, web links for URLs, etc.)
    \usepackage{hyperref}
    \usepackage{longtable} % longtable support required by pandoc >1.10
    \usepackage{booktabs}  % table support for pandoc > 1.12.2
    \usepackage[inline]{enumitem} % IRkernel/repr support (it uses the enumerate* environment)
    \usepackage[normalem]{ulem} % ulem is needed to support strikethroughs (\sout)
                                % normalem makes italics be italics, not underlines
    \usepackage{mathrsfs}
    

    
    
    % Colors for the hyperref package
    \definecolor{urlcolor}{rgb}{0,.145,.698}
    \definecolor{linkcolor}{rgb}{.71,0.21,0.01}
    \definecolor{citecolor}{rgb}{.12,.54,.11}

    % ANSI colors
    \definecolor{ansi-black}{HTML}{3E424D}
    \definecolor{ansi-black-intense}{HTML}{282C36}
    \definecolor{ansi-red}{HTML}{E75C58}
    \definecolor{ansi-red-intense}{HTML}{B22B31}
    \definecolor{ansi-green}{HTML}{00A250}
    \definecolor{ansi-green-intense}{HTML}{007427}
    \definecolor{ansi-yellow}{HTML}{DDB62B}
    \definecolor{ansi-yellow-intense}{HTML}{B27D12}
    \definecolor{ansi-blue}{HTML}{208FFB}
    \definecolor{ansi-blue-intense}{HTML}{0065CA}
    \definecolor{ansi-magenta}{HTML}{D160C4}
    \definecolor{ansi-magenta-intense}{HTML}{A03196}
    \definecolor{ansi-cyan}{HTML}{60C6C8}
    \definecolor{ansi-cyan-intense}{HTML}{258F8F}
    \definecolor{ansi-white}{HTML}{C5C1B4}
    \definecolor{ansi-white-intense}{HTML}{A1A6B2}
    \definecolor{ansi-default-inverse-fg}{HTML}{FFFFFF}
    \definecolor{ansi-default-inverse-bg}{HTML}{000000}

    % commands and environments needed by pandoc snippets
    % extracted from the output of `pandoc -s`
    \providecommand{\tightlist}{%
      \setlength{\itemsep}{0pt}\setlength{\parskip}{0pt}}
    \DefineVerbatimEnvironment{Highlighting}{Verbatim}{commandchars=\\\{\}}
    % Add ',fontsize=\small' for more characters per line
    \newenvironment{Shaded}{}{}
    \newcommand{\KeywordTok}[1]{\textcolor[rgb]{0.00,0.44,0.13}{\textbf{{#1}}}}
    \newcommand{\DataTypeTok}[1]{\textcolor[rgb]{0.56,0.13,0.00}{{#1}}}
    \newcommand{\DecValTok}[1]{\textcolor[rgb]{0.25,0.63,0.44}{{#1}}}
    \newcommand{\BaseNTok}[1]{\textcolor[rgb]{0.25,0.63,0.44}{{#1}}}
    \newcommand{\FloatTok}[1]{\textcolor[rgb]{0.25,0.63,0.44}{{#1}}}
    \newcommand{\CharTok}[1]{\textcolor[rgb]{0.25,0.44,0.63}{{#1}}}
    \newcommand{\StringTok}[1]{\textcolor[rgb]{0.25,0.44,0.63}{{#1}}}
    \newcommand{\CommentTok}[1]{\textcolor[rgb]{0.38,0.63,0.69}{\textit{{#1}}}}
    \newcommand{\OtherTok}[1]{\textcolor[rgb]{0.00,0.44,0.13}{{#1}}}
    \newcommand{\AlertTok}[1]{\textcolor[rgb]{1.00,0.00,0.00}{\textbf{{#1}}}}
    \newcommand{\FunctionTok}[1]{\textcolor[rgb]{0.02,0.16,0.49}{{#1}}}
    \newcommand{\RegionMarkerTok}[1]{{#1}}
    \newcommand{\ErrorTok}[1]{\textcolor[rgb]{1.00,0.00,0.00}{\textbf{{#1}}}}
    \newcommand{\NormalTok}[1]{{#1}}
    
    % Additional commands for more recent versions of Pandoc
    \newcommand{\ConstantTok}[1]{\textcolor[rgb]{0.53,0.00,0.00}{{#1}}}
    \newcommand{\SpecialCharTok}[1]{\textcolor[rgb]{0.25,0.44,0.63}{{#1}}}
    \newcommand{\VerbatimStringTok}[1]{\textcolor[rgb]{0.25,0.44,0.63}{{#1}}}
    \newcommand{\SpecialStringTok}[1]{\textcolor[rgb]{0.73,0.40,0.53}{{#1}}}
    \newcommand{\ImportTok}[1]{{#1}}
    \newcommand{\DocumentationTok}[1]{\textcolor[rgb]{0.73,0.13,0.13}{\textit{{#1}}}}
    \newcommand{\AnnotationTok}[1]{\textcolor[rgb]{0.38,0.63,0.69}{\textbf{\textit{{#1}}}}}
    \newcommand{\CommentVarTok}[1]{\textcolor[rgb]{0.38,0.63,0.69}{\textbf{\textit{{#1}}}}}
    \newcommand{\VariableTok}[1]{\textcolor[rgb]{0.10,0.09,0.49}{{#1}}}
    \newcommand{\ControlFlowTok}[1]{\textcolor[rgb]{0.00,0.44,0.13}{\textbf{{#1}}}}
    \newcommand{\OperatorTok}[1]{\textcolor[rgb]{0.40,0.40,0.40}{{#1}}}
    \newcommand{\BuiltInTok}[1]{{#1}}
    \newcommand{\ExtensionTok}[1]{{#1}}
    \newcommand{\PreprocessorTok}[1]{\textcolor[rgb]{0.74,0.48,0.00}{{#1}}}
    \newcommand{\AttributeTok}[1]{\textcolor[rgb]{0.49,0.56,0.16}{{#1}}}
    \newcommand{\InformationTok}[1]{\textcolor[rgb]{0.38,0.63,0.69}{\textbf{\textit{{#1}}}}}
    \newcommand{\WarningTok}[1]{\textcolor[rgb]{0.38,0.63,0.69}{\textbf{\textit{{#1}}}}}
    
    
    % Define a nice break command that doesn't care if a line doesn't already
    % exist.
    \def\br{\hspace*{\fill} \\* }
    % Math Jax compatibility definitions
    \def\gt{>}
    \def\lt{<}
    \let\Oldtex\TeX
    \let\Oldlatex\LaTeX
    \renewcommand{\TeX}{\textrm{\Oldtex}}
    \renewcommand{\LaTeX}{\textrm{\Oldlatex}}
    % Document parameters
    % Document title
    \title{PS5P4}
    
    
    
    
    

    % Pygments definitions
    
\makeatletter
\def\PY@reset{\let\PY@it=\relax \let\PY@bf=\relax%
    \let\PY@ul=\relax \let\PY@tc=\relax%
    \let\PY@bc=\relax \let\PY@ff=\relax}
\def\PY@tok#1{\csname PY@tok@#1\endcsname}
\def\PY@toks#1+{\ifx\relax#1\empty\else%
    \PY@tok{#1}\expandafter\PY@toks\fi}
\def\PY@do#1{\PY@bc{\PY@tc{\PY@ul{%
    \PY@it{\PY@bf{\PY@ff{#1}}}}}}}
\def\PY#1#2{\PY@reset\PY@toks#1+\relax+\PY@do{#2}}

\expandafter\def\csname PY@tok@w\endcsname{\def\PY@tc##1{\textcolor[rgb]{0.73,0.73,0.73}{##1}}}
\expandafter\def\csname PY@tok@c\endcsname{\let\PY@it=\textit\def\PY@tc##1{\textcolor[rgb]{0.25,0.50,0.50}{##1}}}
\expandafter\def\csname PY@tok@cp\endcsname{\def\PY@tc##1{\textcolor[rgb]{0.74,0.48,0.00}{##1}}}
\expandafter\def\csname PY@tok@k\endcsname{\let\PY@bf=\textbf\def\PY@tc##1{\textcolor[rgb]{0.00,0.50,0.00}{##1}}}
\expandafter\def\csname PY@tok@kp\endcsname{\def\PY@tc##1{\textcolor[rgb]{0.00,0.50,0.00}{##1}}}
\expandafter\def\csname PY@tok@kt\endcsname{\def\PY@tc##1{\textcolor[rgb]{0.69,0.00,0.25}{##1}}}
\expandafter\def\csname PY@tok@o\endcsname{\def\PY@tc##1{\textcolor[rgb]{0.40,0.40,0.40}{##1}}}
\expandafter\def\csname PY@tok@ow\endcsname{\let\PY@bf=\textbf\def\PY@tc##1{\textcolor[rgb]{0.67,0.13,1.00}{##1}}}
\expandafter\def\csname PY@tok@nb\endcsname{\def\PY@tc##1{\textcolor[rgb]{0.00,0.50,0.00}{##1}}}
\expandafter\def\csname PY@tok@nf\endcsname{\def\PY@tc##1{\textcolor[rgb]{0.00,0.00,1.00}{##1}}}
\expandafter\def\csname PY@tok@nc\endcsname{\let\PY@bf=\textbf\def\PY@tc##1{\textcolor[rgb]{0.00,0.00,1.00}{##1}}}
\expandafter\def\csname PY@tok@nn\endcsname{\let\PY@bf=\textbf\def\PY@tc##1{\textcolor[rgb]{0.00,0.00,1.00}{##1}}}
\expandafter\def\csname PY@tok@ne\endcsname{\let\PY@bf=\textbf\def\PY@tc##1{\textcolor[rgb]{0.82,0.25,0.23}{##1}}}
\expandafter\def\csname PY@tok@nv\endcsname{\def\PY@tc##1{\textcolor[rgb]{0.10,0.09,0.49}{##1}}}
\expandafter\def\csname PY@tok@no\endcsname{\def\PY@tc##1{\textcolor[rgb]{0.53,0.00,0.00}{##1}}}
\expandafter\def\csname PY@tok@nl\endcsname{\def\PY@tc##1{\textcolor[rgb]{0.63,0.63,0.00}{##1}}}
\expandafter\def\csname PY@tok@ni\endcsname{\let\PY@bf=\textbf\def\PY@tc##1{\textcolor[rgb]{0.60,0.60,0.60}{##1}}}
\expandafter\def\csname PY@tok@na\endcsname{\def\PY@tc##1{\textcolor[rgb]{0.49,0.56,0.16}{##1}}}
\expandafter\def\csname PY@tok@nt\endcsname{\let\PY@bf=\textbf\def\PY@tc##1{\textcolor[rgb]{0.00,0.50,0.00}{##1}}}
\expandafter\def\csname PY@tok@nd\endcsname{\def\PY@tc##1{\textcolor[rgb]{0.67,0.13,1.00}{##1}}}
\expandafter\def\csname PY@tok@s\endcsname{\def\PY@tc##1{\textcolor[rgb]{0.73,0.13,0.13}{##1}}}
\expandafter\def\csname PY@tok@sd\endcsname{\let\PY@it=\textit\def\PY@tc##1{\textcolor[rgb]{0.73,0.13,0.13}{##1}}}
\expandafter\def\csname PY@tok@si\endcsname{\let\PY@bf=\textbf\def\PY@tc##1{\textcolor[rgb]{0.73,0.40,0.53}{##1}}}
\expandafter\def\csname PY@tok@se\endcsname{\let\PY@bf=\textbf\def\PY@tc##1{\textcolor[rgb]{0.73,0.40,0.13}{##1}}}
\expandafter\def\csname PY@tok@sr\endcsname{\def\PY@tc##1{\textcolor[rgb]{0.73,0.40,0.53}{##1}}}
\expandafter\def\csname PY@tok@ss\endcsname{\def\PY@tc##1{\textcolor[rgb]{0.10,0.09,0.49}{##1}}}
\expandafter\def\csname PY@tok@sx\endcsname{\def\PY@tc##1{\textcolor[rgb]{0.00,0.50,0.00}{##1}}}
\expandafter\def\csname PY@tok@m\endcsname{\def\PY@tc##1{\textcolor[rgb]{0.40,0.40,0.40}{##1}}}
\expandafter\def\csname PY@tok@gh\endcsname{\let\PY@bf=\textbf\def\PY@tc##1{\textcolor[rgb]{0.00,0.00,0.50}{##1}}}
\expandafter\def\csname PY@tok@gu\endcsname{\let\PY@bf=\textbf\def\PY@tc##1{\textcolor[rgb]{0.50,0.00,0.50}{##1}}}
\expandafter\def\csname PY@tok@gd\endcsname{\def\PY@tc##1{\textcolor[rgb]{0.63,0.00,0.00}{##1}}}
\expandafter\def\csname PY@tok@gi\endcsname{\def\PY@tc##1{\textcolor[rgb]{0.00,0.63,0.00}{##1}}}
\expandafter\def\csname PY@tok@gr\endcsname{\def\PY@tc##1{\textcolor[rgb]{1.00,0.00,0.00}{##1}}}
\expandafter\def\csname PY@tok@ge\endcsname{\let\PY@it=\textit}
\expandafter\def\csname PY@tok@gs\endcsname{\let\PY@bf=\textbf}
\expandafter\def\csname PY@tok@gp\endcsname{\let\PY@bf=\textbf\def\PY@tc##1{\textcolor[rgb]{0.00,0.00,0.50}{##1}}}
\expandafter\def\csname PY@tok@go\endcsname{\def\PY@tc##1{\textcolor[rgb]{0.53,0.53,0.53}{##1}}}
\expandafter\def\csname PY@tok@gt\endcsname{\def\PY@tc##1{\textcolor[rgb]{0.00,0.27,0.87}{##1}}}
\expandafter\def\csname PY@tok@err\endcsname{\def\PY@bc##1{\setlength{\fboxsep}{0pt}\fcolorbox[rgb]{1.00,0.00,0.00}{1,1,1}{\strut ##1}}}
\expandafter\def\csname PY@tok@kc\endcsname{\let\PY@bf=\textbf\def\PY@tc##1{\textcolor[rgb]{0.00,0.50,0.00}{##1}}}
\expandafter\def\csname PY@tok@kd\endcsname{\let\PY@bf=\textbf\def\PY@tc##1{\textcolor[rgb]{0.00,0.50,0.00}{##1}}}
\expandafter\def\csname PY@tok@kn\endcsname{\let\PY@bf=\textbf\def\PY@tc##1{\textcolor[rgb]{0.00,0.50,0.00}{##1}}}
\expandafter\def\csname PY@tok@kr\endcsname{\let\PY@bf=\textbf\def\PY@tc##1{\textcolor[rgb]{0.00,0.50,0.00}{##1}}}
\expandafter\def\csname PY@tok@bp\endcsname{\def\PY@tc##1{\textcolor[rgb]{0.00,0.50,0.00}{##1}}}
\expandafter\def\csname PY@tok@fm\endcsname{\def\PY@tc##1{\textcolor[rgb]{0.00,0.00,1.00}{##1}}}
\expandafter\def\csname PY@tok@vc\endcsname{\def\PY@tc##1{\textcolor[rgb]{0.10,0.09,0.49}{##1}}}
\expandafter\def\csname PY@tok@vg\endcsname{\def\PY@tc##1{\textcolor[rgb]{0.10,0.09,0.49}{##1}}}
\expandafter\def\csname PY@tok@vi\endcsname{\def\PY@tc##1{\textcolor[rgb]{0.10,0.09,0.49}{##1}}}
\expandafter\def\csname PY@tok@vm\endcsname{\def\PY@tc##1{\textcolor[rgb]{0.10,0.09,0.49}{##1}}}
\expandafter\def\csname PY@tok@sa\endcsname{\def\PY@tc##1{\textcolor[rgb]{0.73,0.13,0.13}{##1}}}
\expandafter\def\csname PY@tok@sb\endcsname{\def\PY@tc##1{\textcolor[rgb]{0.73,0.13,0.13}{##1}}}
\expandafter\def\csname PY@tok@sc\endcsname{\def\PY@tc##1{\textcolor[rgb]{0.73,0.13,0.13}{##1}}}
\expandafter\def\csname PY@tok@dl\endcsname{\def\PY@tc##1{\textcolor[rgb]{0.73,0.13,0.13}{##1}}}
\expandafter\def\csname PY@tok@s2\endcsname{\def\PY@tc##1{\textcolor[rgb]{0.73,0.13,0.13}{##1}}}
\expandafter\def\csname PY@tok@sh\endcsname{\def\PY@tc##1{\textcolor[rgb]{0.73,0.13,0.13}{##1}}}
\expandafter\def\csname PY@tok@s1\endcsname{\def\PY@tc##1{\textcolor[rgb]{0.73,0.13,0.13}{##1}}}
\expandafter\def\csname PY@tok@mb\endcsname{\def\PY@tc##1{\textcolor[rgb]{0.40,0.40,0.40}{##1}}}
\expandafter\def\csname PY@tok@mf\endcsname{\def\PY@tc##1{\textcolor[rgb]{0.40,0.40,0.40}{##1}}}
\expandafter\def\csname PY@tok@mh\endcsname{\def\PY@tc##1{\textcolor[rgb]{0.40,0.40,0.40}{##1}}}
\expandafter\def\csname PY@tok@mi\endcsname{\def\PY@tc##1{\textcolor[rgb]{0.40,0.40,0.40}{##1}}}
\expandafter\def\csname PY@tok@il\endcsname{\def\PY@tc##1{\textcolor[rgb]{0.40,0.40,0.40}{##1}}}
\expandafter\def\csname PY@tok@mo\endcsname{\def\PY@tc##1{\textcolor[rgb]{0.40,0.40,0.40}{##1}}}
\expandafter\def\csname PY@tok@ch\endcsname{\let\PY@it=\textit\def\PY@tc##1{\textcolor[rgb]{0.25,0.50,0.50}{##1}}}
\expandafter\def\csname PY@tok@cm\endcsname{\let\PY@it=\textit\def\PY@tc##1{\textcolor[rgb]{0.25,0.50,0.50}{##1}}}
\expandafter\def\csname PY@tok@cpf\endcsname{\let\PY@it=\textit\def\PY@tc##1{\textcolor[rgb]{0.25,0.50,0.50}{##1}}}
\expandafter\def\csname PY@tok@c1\endcsname{\let\PY@it=\textit\def\PY@tc##1{\textcolor[rgb]{0.25,0.50,0.50}{##1}}}
\expandafter\def\csname PY@tok@cs\endcsname{\let\PY@it=\textit\def\PY@tc##1{\textcolor[rgb]{0.25,0.50,0.50}{##1}}}

\def\PYZbs{\char`\\}
\def\PYZus{\char`\_}
\def\PYZob{\char`\{}
\def\PYZcb{\char`\}}
\def\PYZca{\char`\^}
\def\PYZam{\char`\&}
\def\PYZlt{\char`\<}
\def\PYZgt{\char`\>}
\def\PYZsh{\char`\#}
\def\PYZpc{\char`\%}
\def\PYZdl{\char`\$}
\def\PYZhy{\char`\-}
\def\PYZsq{\char`\'}
\def\PYZdq{\char`\"}
\def\PYZti{\char`\~}
% for compatibility with earlier versions
\def\PYZat{@}
\def\PYZlb{[}
\def\PYZrb{]}
\makeatother


    % Exact colors from NB
    \definecolor{incolor}{rgb}{0.0, 0.0, 0.5}
    \definecolor{outcolor}{rgb}{0.545, 0.0, 0.0}



    
    % Prevent overflowing lines due to hard-to-break entities
    \sloppy 
    % Setup hyperref package
    \hypersetup{
      breaklinks=true,  % so long urls are correctly broken across lines
      colorlinks=true,
      urlcolor=urlcolor,
      linkcolor=linkcolor,
      citecolor=citecolor,
      }
    % Slightly bigger margins than the latex defaults
    
    \geometry{verbose,tmargin=1in,bmargin=1in,lmargin=1in,rmargin=1in}
    
    

    \begin{document}
    
    
    
    

    
    \begin{Verbatim}[commandchars=\\\{\}]
{\color{incolor}In [{\color{incolor}1}]:} \PY{k+kn}{import} \PY{n+nn}{pandas} \PY{k}{as} \PY{n+nn}{pd}
        \PY{k+kn}{from} \PY{n+nn}{statsmodels}\PY{n+nn}{.}\PY{n+nn}{regression}\PY{n+nn}{.}\PY{n+nn}{linear\PYZus{}model} \PY{k}{import} \PY{n}{OLS}
        \PY{k+kn}{import} \PY{n+nn}{numpy} \PY{k}{as} \PY{n+nn}{np}
        \PY{k+kn}{from} \PY{n+nn}{linearmodels}\PY{n+nn}{.}\PY{n+nn}{iv} \PY{k}{import} \PY{n}{IV2SLS}
\end{Verbatim}

    \section*{14.32 Problem Set 5}\label{problem-set-5}

\subsection*{Problem 4}\label{problem-4}

Loading the data from the Current Population Survey Merged Outgoing
Rotation Groups for 2018 from the National Bureau of Economic Research.

    \begin{Verbatim}[commandchars=\\\{\}]
{\color{incolor}In [{\color{incolor}2}]:} \PY{n}{full\PYZus{}df} \PY{o}{=} \PY{n}{pd}\PY{o}{.}\PY{n}{read\PYZus{}stata}\PY{p}{(}\PY{l+s+s1}{\PYZsq{}}\PY{l+s+s1}{morg18.dta}\PY{l+s+s1}{\PYZsq{}}\PY{p}{)}
\end{Verbatim}

    We add a constant column to run OLS with a constant later on.

    \begin{Verbatim}[commandchars=\\\{\}]
{\color{incolor}In [{\color{incolor}3}]:} \PY{n}{full\PYZus{}df}\PY{p}{[}\PY{l+s+s1}{\PYZsq{}}\PY{l+s+s1}{\PYZus{}cons}\PY{l+s+s1}{\PYZsq{}}\PY{p}{]} \PY{o}{=} \PY{l+m+mf}{1.}
\end{Verbatim}

    \subsubsection*{Part A}\label{part-a}

In this problem, we are interested in determining if getting married has
an impact on one's weekly earnings. To answer this question, we will
regress the logarithm of weekly earnings on a binary indictor of married
vs un-married. There are a few possible confounding variables in this
case:

\begin{itemize}
\tightlist
\item
  The age of an individual is likely strongly correlated with both
  marital status and weekly earnings.
\item
  The education level of an individual is likely strongly correlated
  with both marital status and weekly earnings.
\item
  Whether an individual has children or not is likely strongly
  correlated to both marital status and weekly earnings.
\item
  A person's sex is likely strongly correlated to weekly earnings and it
  might possibly be weakly correlated to their marital status.
\item
  A person's race is likely strongly correlated to weekly earnings and
  it might possible be weakly correlated to their marital status.
\end{itemize}

By controlling for these factors, we might reduce some omitted variables
bias. To do so, we can extract the important columns:

    \begin{Verbatim}[commandchars=\\\{\}]
{\color{incolor}In [{\color{incolor}4}]:} \PY{n}{df} \PY{o}{=} \PY{n}{full\PYZus{}df}\PY{p}{[}\PY{p}{[}\PY{l+s+s1}{\PYZsq{}}\PY{l+s+s1}{\PYZus{}cons}\PY{l+s+s1}{\PYZsq{}}\PY{p}{,} \PY{l+s+s1}{\PYZsq{}}\PY{l+s+s1}{earnwke}\PY{l+s+s1}{\PYZsq{}}\PY{p}{,} \PY{l+s+s1}{\PYZsq{}}\PY{l+s+s1}{marital}\PY{l+s+s1}{\PYZsq{}}\PY{p}{,} \PY{l+s+s1}{\PYZsq{}}\PY{l+s+s1}{age}\PY{l+s+s1}{\PYZsq{}}\PY{p}{,} 
                      \PY{l+s+s1}{\PYZsq{}}\PY{l+s+s1}{ihigrdc}\PY{l+s+s1}{\PYZsq{}}\PY{p}{,} \PY{l+s+s1}{\PYZsq{}}\PY{l+s+s1}{ownchild}\PY{l+s+s1}{\PYZsq{}}\PY{p}{,} \PY{l+s+s1}{\PYZsq{}}\PY{l+s+s1}{race}\PY{l+s+s1}{\PYZsq{}}\PY{p}{,} \PY{l+s+s1}{\PYZsq{}}\PY{l+s+s1}{sex}\PY{l+s+s1}{\PYZsq{}}\PY{p}{]}\PY{p}{]}
\end{Verbatim}

    We are only going to consider people who have a positive weekly income.
Therefore, we drop all \texttt{NaN} incomes (this will also drop unknown
education levels in \texttt{ihigrdc}) and take only people with income
not equal to zero.

    \begin{Verbatim}[commandchars=\\\{\}]
{\color{incolor}In [{\color{incolor}5}]:} \PY{n}{df} \PY{o}{=} \PY{n}{df}\PY{o}{.}\PY{n}{dropna}\PY{p}{(}\PY{p}{)}
        \PY{n}{df} \PY{o}{=} \PY{n}{df}\PY{p}{[}\PY{n}{df}\PY{p}{[}\PY{l+s+s1}{\PYZsq{}}\PY{l+s+s1}{earnwke}\PY{l+s+s1}{\PYZsq{}}\PY{p}{]} \PY{o}{!=} \PY{l+m+mf}{0.}\PY{p}{]}
\end{Verbatim}

    We don't want to regress on the weekly wage directly, instead we use the
logarithm of the weekly wage.

    \begin{Verbatim}[commandchars=\\\{\}]
{\color{incolor}In [{\color{incolor}6}]:} \PY{n}{df}\PY{p}{[}\PY{l+s+s1}{\PYZsq{}}\PY{l+s+s1}{lgearnwke}\PY{l+s+s1}{\PYZsq{}}\PY{p}{]} \PY{o}{=} \PY{n}{np}\PY{o}{.}\PY{n}{log}\PY{p}{(}\PY{n}{df}\PY{p}{[}\PY{l+s+s1}{\PYZsq{}}\PY{l+s+s1}{earnwke}\PY{l+s+s1}{\PYZsq{}}\PY{p}{]}\PY{p}{)}
        \PY{n}{df} \PY{o}{=} \PY{n}{df}\PY{o}{.}\PY{n}{drop}\PY{p}{(}\PY{n}{columns}\PY{o}{=}\PY{p}{[}\PY{l+s+s1}{\PYZsq{}}\PY{l+s+s1}{earnwke}\PY{l+s+s1}{\PYZsq{}}\PY{p}{]}\PY{p}{)}
\end{Verbatim}

    The data set breaks marital status into several categories. We are only
interested in whether someone is married or not.

    \begin{Verbatim}[commandchars=\\\{\}]
{\color{incolor}In [{\color{incolor}7}]:} \PY{n}{df}\PY{p}{[}\PY{l+s+s1}{\PYZsq{}}\PY{l+s+s1}{married}\PY{l+s+s1}{\PYZsq{}}\PY{p}{]} \PY{o}{=} \PY{n}{df}\PY{p}{[}\PY{l+s+s1}{\PYZsq{}}\PY{l+s+s1}{marital}\PY{l+s+s1}{\PYZsq{}}\PY{p}{]}\PY{o}{.}\PY{n}{replace}\PY{p}{(}\PY{p}{\PYZob{}}
            \PY{l+m+mi}{1}\PY{p}{:} \PY{l+m+mf}{1.}\PY{p}{,}  \PY{c+c1}{\PYZsh{} Married Civilian Spouse Present}
            \PY{l+m+mi}{2}\PY{p}{:} \PY{l+m+mf}{1.}\PY{p}{,}  \PY{c+c1}{\PYZsh{} Married AF Spouse Present}
            \PY{l+m+mi}{3}\PY{p}{:} \PY{l+m+mf}{1.}\PY{p}{,}  \PY{c+c1}{\PYZsh{} Married Spouse Absent}
            \PY{l+m+mi}{4}\PY{p}{:} \PY{l+m+mf}{0.}\PY{p}{,}  \PY{c+c1}{\PYZsh{} Widowed}
            \PY{l+m+mi}{5}\PY{p}{:} \PY{l+m+mf}{0.}\PY{p}{,}  \PY{c+c1}{\PYZsh{} Divorced}
            \PY{l+m+mi}{6}\PY{p}{:} \PY{l+m+mf}{0.}\PY{p}{,}  \PY{c+c1}{\PYZsh{} Separated}
            \PY{l+m+mi}{7}\PY{p}{:} \PY{l+m+mf}{0.}\PY{p}{,}  \PY{c+c1}{\PYZsh{} Never Married}
        \PY{p}{\PYZcb{}}\PY{p}{)}
        \PY{n}{df} \PY{o}{=} \PY{n}{df}\PY{o}{.}\PY{n}{drop}\PY{p}{(}\PY{n}{columns}\PY{o}{=}\PY{p}{[}\PY{l+s+s1}{\PYZsq{}}\PY{l+s+s1}{marital}\PY{l+s+s1}{\PYZsq{}}\PY{p}{]}\PY{p}{)}
\end{Verbatim}

    I don't really care about the number of children, only whether or not a
child is present. Therefore, I simplify this variable to an indicator.

    \begin{Verbatim}[commandchars=\\\{\}]
{\color{incolor}In [{\color{incolor}8}]:} \PY{n}{df}\PY{p}{[}\PY{l+s+s1}{\PYZsq{}}\PY{l+s+s1}{children}\PY{l+s+s1}{\PYZsq{}}\PY{p}{]} \PY{o}{=} \PY{p}{(}\PY{n}{df}\PY{p}{[}\PY{l+s+s1}{\PYZsq{}}\PY{l+s+s1}{ownchild}\PY{l+s+s1}{\PYZsq{}}\PY{p}{]} \PY{o}{\PYZgt{}} \PY{l+m+mi}{0}\PY{p}{)}\PY{o}{.}\PY{n}{astype}\PY{p}{(}\PY{n}{np}\PY{o}{.}\PY{n}{float64}\PY{p}{)}
        \PY{n}{df} \PY{o}{=} \PY{n}{df}\PY{o}{.}\PY{n}{drop}\PY{p}{(}\PY{n}{columns}\PY{o}{=}\PY{p}{[}\PY{l+s+s1}{\PYZsq{}}\PY{l+s+s1}{ownchild}\PY{l+s+s1}{\PYZsq{}}\PY{p}{]}\PY{p}{)}
\end{Verbatim}

    There are many different categories of race represented in this data
set. To make things super simple, we are only going to indicate whether
someone is black.

    \begin{Verbatim}[commandchars=\\\{\}]
{\color{incolor}In [{\color{incolor}9}]:} \PY{n}{df}\PY{p}{[}\PY{l+s+s1}{\PYZsq{}}\PY{l+s+s1}{white}\PY{l+s+s1}{\PYZsq{}}\PY{p}{]} \PY{o}{=} \PY{p}{(}\PY{n}{df}\PY{p}{[}\PY{l+s+s1}{\PYZsq{}}\PY{l+s+s1}{race}\PY{l+s+s1}{\PYZsq{}}\PY{p}{]} \PY{o}{==} \PY{l+m+mi}{1}\PY{p}{)}\PY{o}{.}\PY{n}{astype}\PY{p}{(}\PY{n}{np}\PY{o}{.}\PY{n}{float64}\PY{p}{)}
        \PY{n}{df} \PY{o}{=} \PY{n}{df}\PY{o}{.}\PY{n}{drop}\PY{p}{(}\PY{n}{columns}\PY{o}{=}\PY{p}{[}\PY{l+s+s1}{\PYZsq{}}\PY{l+s+s1}{race}\PY{l+s+s1}{\PYZsq{}}\PY{p}{]}\PY{p}{)}
\end{Verbatim}

    The data set represent male as 1 and female as 2. To simplify this, we
transform the data such that male is 1 and female is 0.

    \begin{Verbatim}[commandchars=\\\{\}]
{\color{incolor}In [{\color{incolor}10}]:} \PY{n}{df}\PY{p}{[}\PY{l+s+s1}{\PYZsq{}}\PY{l+s+s1}{male}\PY{l+s+s1}{\PYZsq{}}\PY{p}{]} \PY{o}{=} \PY{p}{(}\PY{n}{df}\PY{p}{[}\PY{l+s+s1}{\PYZsq{}}\PY{l+s+s1}{sex}\PY{l+s+s1}{\PYZsq{}}\PY{p}{]} \PY{o}{==} \PY{l+m+mi}{1}\PY{p}{)}\PY{o}{.}\PY{n}{astype}\PY{p}{(}\PY{n}{np}\PY{o}{.}\PY{n}{float64}\PY{p}{)}
         \PY{n}{df} \PY{o}{=} \PY{n}{df}\PY{o}{.}\PY{n}{drop}\PY{p}{(}\PY{n}{columns}\PY{o}{=}\PY{p}{[}\PY{l+s+s1}{\PYZsq{}}\PY{l+s+s1}{sex}\PY{l+s+s1}{\PYZsq{}}\PY{p}{]}\PY{p}{)}
\end{Verbatim}

    Now our dataframe consists of the following columns.

    \begin{Verbatim}[commandchars=\\\{\}]
{\color{incolor}In [{\color{incolor}11}]:} \PY{n+nb}{list}\PY{p}{(}\PY{n}{df}\PY{o}{.}\PY{n}{columns}\PY{p}{)}
\end{Verbatim}

\begin{Verbatim}[commandchars=\\\{\}]
{\color{outcolor}Out[{\color{outcolor}11}]:} ['\_cons',
          'age',
          'ihigrdc',
          'lgearnwke',
          'married',
          'children',
          'white',
          'male']
\end{Verbatim}
            
    \subsubsection*{Part B}\label{part-b}

\paragraph{Part I}\label{part-i}

Our data has already been imported into \texttt{df} as in Part A. Here's
a preview:

    \begin{Verbatim}[commandchars=\\\{\}]
{\color{incolor}In [{\color{incolor}12}]:} \PY{n+nb}{print}\PY{p}{(}\PY{n}{df}\PY{p}{[}\PY{p}{:}\PY{l+m+mi}{10}\PY{p}{]}\PY{p}{)}
\end{Verbatim}

    \begin{Verbatim}[commandchars=\\\{\}]
    \_cons  age  ihigrdc  lgearnwke  married  children  white  male
2     1.0   52     12.0   6.805723      0.0       1.0    0.0   0.0
3     1.0   19     12.0   5.991465      0.0       0.0    0.0   0.0
5     1.0   22     12.0   4.248495      0.0       0.0    0.0   0.0
6     1.0   48     12.0   6.522093      0.0       0.0    1.0   1.0
17    1.0   59     12.0   6.684612      0.0       0.0    0.0   1.0
18    1.0   27     12.0   5.953243      0.0       0.0    0.0   1.0
19    1.0   30     12.0   6.514713      0.0       0.0    0.0   0.0
20    1.0   49     12.0   5.786897      0.0       0.0    0.0   0.0
28    1.0   48     15.0   7.025005      1.0       0.0    0.0   1.0
31    1.0   51     14.0   7.489110      1.0       0.0    0.0   1.0

    \end{Verbatim}

    \paragraph{Part II}\label{part-ii}

We can quickly look at the summary statistics of our data set:

    \begin{Verbatim}[commandchars=\\\{\}]
{\color{incolor}In [{\color{incolor}13}]:} \PY{n+nb}{print}\PY{p}{(}\PY{n}{df}\PY{o}{.}\PY{n}{describe}\PY{p}{(}\PY{p}{)}\PY{p}{)}
\end{Verbatim}

    \begin{Verbatim}[commandchars=\\\{\}]
          \_cons            age        ihigrdc      lgearnwke        married  \textbackslash{}
count  105108.0  105108.000000  105108.000000  105108.000000  105108.000000   
mean        1.0      41.815143      12.749990       6.421394       0.500590   
std         0.0      15.025220       2.309817       0.803723       0.500002   
min         1.0      16.000000       0.000000      -4.605170       0.000000   
25\%         1.0      29.000000      12.000000       6.004677       0.000000   
50\%         1.0      41.000000      12.000000       6.461468       1.000000   
75\%         1.0      54.000000      14.000000       6.915723       1.000000   
max         1.0      85.000000      18.000000       7.967145       1.000000   

            children          white           male  
count  105108.000000  105108.000000  105108.000000  
mean        0.303459       0.804439       0.530873  
std         0.459754       0.396634       0.499048  
min         0.000000       0.000000       0.000000  
25\%         0.000000       1.000000       0.000000  
50\%         0.000000       1.000000       1.000000  
75\%         1.000000       1.000000       1.000000  
max         1.000000       1.000000       1.000000  

    \end{Verbatim}

    \paragraph{Part III}\label{part-iii}

We can run an ordinary least squares regression of \texttt{lgearnwke} on
\texttt{married} with heteroskedastic robust standard errors without
controlling for anything and without using instrumental variables.

    \begin{Verbatim}[commandchars=\\\{\}]
{\color{incolor}In [{\color{incolor}14}]:} \PY{n}{simple\PYZus{}model} \PY{o}{=} \PY{n}{OLS}\PY{p}{(}\PY{n}{df}\PY{p}{[}\PY{l+s+s1}{\PYZsq{}}\PY{l+s+s1}{lgearnwke}\PY{l+s+s1}{\PYZsq{}}\PY{p}{]}\PY{p}{,} \PY{n}{df}\PY{p}{[}\PY{p}{[}\PY{l+s+s1}{\PYZsq{}}\PY{l+s+s1}{\PYZus{}cons}\PY{l+s+s1}{\PYZsq{}}\PY{p}{,} \PY{l+s+s1}{\PYZsq{}}\PY{l+s+s1}{married}\PY{l+s+s1}{\PYZsq{}}\PY{p}{]}\PY{p}{]}\PY{p}{)}
         \PY{n}{simple\PYZus{}results} \PY{o}{=} \PY{n}{simple\PYZus{}model}\PY{o}{.}\PY{n}{fit}\PY{p}{(}\PY{n}{cov\PYZus{}type}\PY{o}{=}\PY{l+s+s1}{\PYZsq{}}\PY{l+s+s1}{HC1}\PY{l+s+s1}{\PYZsq{}}\PY{p}{,} \PY{n}{use\PYZus{}t}\PY{o}{=}\PY{k+kc}{True}\PY{p}{)}
         \PY{n+nb}{print}\PY{p}{(}\PY{n}{simple\PYZus{}results}\PY{o}{.}\PY{n}{summary}\PY{p}{(}\PY{p}{)}\PY{p}{)}
\end{Verbatim}

    \begin{Verbatim}[commandchars=\\\{\}]
                            OLS Regression Results                            
==============================================================================
Dep. Variable:              lgearnwke   R-squared:                       0.061
Model:                            OLS   Adj. R-squared:                  0.061
Method:                 Least Squares   F-statistic:                     6779.
Date:                Wed, 01 May 2019   Prob (F-statistic):               0.00
Time:                        15:54:39   Log-Likelihood:            -1.2289e+05
No. Observations:              105108   AIC:                         2.458e+05
Df Residuals:                  105106   BIC:                         2.458e+05
Df Model:                           1                                         
Covariance Type:                  HC1                                         
==============================================================================
                 coef    std err          t      P>|t|      [0.025      0.975]
------------------------------------------------------------------------------
\_cons          6.2233      0.004   1761.417      0.000       6.216       6.230
married        0.3957      0.005     82.333      0.000       0.386       0.405
==============================================================================
Omnibus:                    32594.095   Durbin-Watson:                   1.898
Prob(Omnibus):                  0.000   Jarque-Bera (JB):           287844.443
Skew:                          -1.241   Prob(JB):                         0.00
Kurtosis:                      10.718   Cond. No.                         2.62
==============================================================================

Warnings:
[1] Standard Errors are heteroscedasticity robust (HC1)

    \end{Verbatim}

    Using the results from this regression, we can explicitly test our
hypothesis:

\[ H_0 : married = 0 \] \[ H_1 : married \neq 0 \]

    \begin{Verbatim}[commandchars=\\\{\}]
{\color{incolor}In [{\color{incolor}15}]:} \PY{n}{simple\PYZus{}married\PYZus{}t\PYZus{}test} \PY{o}{=} \PY{n}{simple\PYZus{}results}\PY{o}{.}\PY{n}{t\PYZus{}test}\PY{p}{(}\PY{l+s+s1}{\PYZsq{}}\PY{l+s+s1}{married = 0}\PY{l+s+s1}{\PYZsq{}}\PY{p}{)}
         \PY{n+nb}{print}\PY{p}{(}\PY{n}{simple\PYZus{}married\PYZus{}t\PYZus{}test}\PY{o}{.}\PY{n}{summary}\PY{p}{(}\PY{p}{)}\PY{p}{)}
\end{Verbatim}

    \begin{Verbatim}[commandchars=\\\{\}]
                             Test for Constraints                             
==============================================================================
                 coef    std err          t      P>|t|      [0.025      0.975]
------------------------------------------------------------------------------
c0             0.3957      0.005     82.333      0.000       0.386       0.405
==============================================================================

    \end{Verbatim}

    From this \(t\)-test, we can see the \(p\)-value is equal to zero or,
equivalently, that the confidence interval does not include zero. Thus,
it is safe to reject the null that marriage has no effect on weekly
wages. From the coefficient, and individual who is married makes
approximately 40\% more on average than an individual who is not
married.

    \paragraph{Part IV}\label{part-iv}

Now we can control for some other confounding factors.

\begin{itemize}
\tightlist
\item
  We would expect \(\mathrm{cov}(lgearnwke, age)\) and
  \(\mathrm{cov}(married, age)\) to both be strongly positive. This
  should cause overstatement of the coefficient on married.
\item
  We would expect \(\mathrm{cov}(lgearnwke, ihigrd)\) to be strongly
  positive and \(\mathrm{cov}(married, ihigrd)\) to be slightly
  positive. This should cause overstatement of the coefficient on
  married.
\item
  We would expect \(\mathrm{cov}(lgearnwke, children)\) to be slightly
  positive and \(\mathrm{cov}(married, children)\) to be strongly
  positive. This should cause overstatement of the coefficient on
  married.
\item
  We would expect \(\mathrm{cov}(lgearnwke, male)\) to be strongly
  positive and \(\mathrm{cov}(married, male)\) to be roughly zero. This
  may or may not have a noticeable effect.
\item
  We would expect \(\mathrm{cov}(lgearnwke, white)\) to be strongly
  positive and \(\mathrm{cov}(married, white)\) to be roughtly zero.
  This may or may not have a noticeable effect.
\end{itemize}

Running the controlled regression:

    \begin{Verbatim}[commandchars=\\\{\}]
{\color{incolor}In [{\color{incolor}16}]:} \PY{n}{controlled\PYZus{}model} \PY{o}{=} \PY{n}{OLS}\PY{p}{(}\PY{n}{df}\PY{p}{[}\PY{l+s+s1}{\PYZsq{}}\PY{l+s+s1}{lgearnwke}\PY{l+s+s1}{\PYZsq{}}\PY{p}{]}\PY{p}{,} \PY{n}{df}\PY{p}{[}\PY{p}{[}\PY{l+s+s1}{\PYZsq{}}\PY{l+s+s1}{\PYZus{}cons}\PY{l+s+s1}{\PYZsq{}}\PY{p}{,} \PY{l+s+s1}{\PYZsq{}}\PY{l+s+s1}{married}\PY{l+s+s1}{\PYZsq{}}\PY{p}{,} \PY{l+s+s1}{\PYZsq{}}\PY{l+s+s1}{age}\PY{l+s+s1}{\PYZsq{}}\PY{p}{,} 
                                                     \PY{l+s+s1}{\PYZsq{}}\PY{l+s+s1}{ihigrdc}\PY{l+s+s1}{\PYZsq{}}\PY{p}{,} \PY{l+s+s1}{\PYZsq{}}\PY{l+s+s1}{children}\PY{l+s+s1}{\PYZsq{}}\PY{p}{,} \PY{l+s+s1}{\PYZsq{}}\PY{l+s+s1}{male}\PY{l+s+s1}{\PYZsq{}}\PY{p}{,} 
                                                     \PY{l+s+s1}{\PYZsq{}}\PY{l+s+s1}{white}\PY{l+s+s1}{\PYZsq{}}\PY{p}{]}\PY{p}{]}\PY{p}{)}
         \PY{n}{controlled\PYZus{}results} \PY{o}{=} \PY{n}{controlled\PYZus{}model}\PY{o}{.}\PY{n}{fit}\PY{p}{(}\PY{n}{cov\PYZus{}type}\PY{o}{=}\PY{l+s+s1}{\PYZsq{}}\PY{l+s+s1}{HC1}\PY{l+s+s1}{\PYZsq{}}\PY{p}{,} \PY{n}{use\PYZus{}t}\PY{o}{=}\PY{k+kc}{True}\PY{p}{)}
         \PY{n+nb}{print}\PY{p}{(}\PY{n}{controlled\PYZus{}results}\PY{o}{.}\PY{n}{summary}\PY{p}{(}\PY{p}{)}\PY{p}{)}
\end{Verbatim}

    \begin{Verbatim}[commandchars=\\\{\}]
                            OLS Regression Results                            
==============================================================================
Dep. Variable:              lgearnwke   R-squared:                       0.209
Model:                            OLS   Adj. R-squared:                  0.209
Method:                 Least Squares   F-statistic:                     4170.
Date:                Wed, 01 May 2019   Prob (F-statistic):               0.00
Time:                        15:54:39   Log-Likelihood:            -1.1383e+05
No. Observations:              105108   AIC:                         2.277e+05
Df Residuals:                  105101   BIC:                         2.277e+05
Df Model:                           6                                         
Covariance Type:                  HC1                                         
==============================================================================
                 coef    std err          t      P>|t|      [0.025      0.975]
------------------------------------------------------------------------------
\_cons          4.4266      0.016    270.281      0.000       4.394       4.459
married        0.1652      0.005     33.159      0.000       0.155       0.175
age            0.0117      0.000     62.513      0.000       0.011       0.012
ihigrdc        0.0872      0.001     83.489      0.000       0.085       0.089
children       0.2082      0.005     43.890      0.000       0.199       0.217
male           0.3739      0.004     83.750      0.000       0.365       0.383
white          0.0599      0.005     11.157      0.000       0.049       0.070
==============================================================================
Omnibus:                    45012.175   Durbin-Watson:                   1.886
Prob(Omnibus):                  0.000   Jarque-Bera (JB):           665862.547
Skew:                          -1.668   Prob(JB):                         0.00
Kurtosis:                      14.871   Cond. No.                         316.
==============================================================================

Warnings:
[1] Standard Errors are heteroscedasticity robust (HC1)

    \end{Verbatim}

    Putting all the controls together drastically reduces the coefficient of
married. We can explicitly test the our hypothesis:

\[ H_0 : married = 0 \] \[ H_1 : married \neq 0 \]

    \begin{Verbatim}[commandchars=\\\{\}]
{\color{incolor}In [{\color{incolor}17}]:} \PY{n}{controlled\PYZus{}married\PYZus{}t\PYZus{}test} \PY{o}{=} \PY{n}{controlled\PYZus{}results}\PY{o}{.}\PY{n}{t\PYZus{}test}\PY{p}{(}\PY{l+s+s1}{\PYZsq{}}\PY{l+s+s1}{married = 0}\PY{l+s+s1}{\PYZsq{}}\PY{p}{)}
         \PY{n+nb}{print}\PY{p}{(}\PY{n}{controlled\PYZus{}married\PYZus{}t\PYZus{}test}\PY{o}{.}\PY{n}{summary}\PY{p}{(}\PY{p}{)}\PY{p}{)}
\end{Verbatim}

    \begin{Verbatim}[commandchars=\\\{\}]
                             Test for Constraints                             
==============================================================================
                 coef    std err          t      P>|t|      [0.025      0.975]
------------------------------------------------------------------------------
c0             0.1652      0.005     33.159      0.000       0.155       0.175
==============================================================================

    \end{Verbatim}

    From this \(t\)-test, we can see the \(p\)-value is still equal to zero
or, equivalently, that the confidence interval still does not include
zero. Thus, it is safe to reject the null that marriage has no effect on
weekly wages. From the coefficient, and individual who is married makes
approximately 17\% more on average than an individual who is not
married. However, this is significantly lower than the naive estimate
which suggests the effect of marriage on income is not nearly as strong
as it first appears.

Furthermore, there are likely more omitted variables present in this
model. Some quick Googling suggests the length of someone's marriage
could be a strong confounder. Unfortunately I couldn't test this using
the CPS data. Though it is kind of incorporated in the model using age.


    % Add a bibliography block to the postdoc
    
    
    
    \end{document}
