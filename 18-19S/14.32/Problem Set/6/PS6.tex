%
% 6.S077 problem set solutions template
%
\documentclass[12pt,twoside]{article}

\newcommand{\name}{}

\usepackage{amssymb}
\usepackage{amsmath}
\usepackage{graphicx}
\usepackage{latexsym}
\usepackage{times,url}
\usepackage{cprotect}
\usepackage{listings}
\usepackage{graphicx}
\usepackage[table]{xcolor}
\usepackage[letterpaper]{geometry}
\usepackage{tikz-qtree}
\usepackage{enumerate}

\newcommand{\profs}{Mauricio Karchmer, Aleksander Madry, Bruce Tidor}
\newcommand{\subj}{6.046}
\newcommand{\ttt}[1]{{\tt\small #1}}

\definecolor{dkgreen}{rgb}{0,0.6,0}
\definecolor{gray}{rgb}{0.5,0.5,0.5}
\definecolor{mauve}{rgb}{0.58,0,0.82}

\lstset{
  language=Python,
  aboveskip=1pc,
  belowskip=1pc,
  basicstyle={\footnotesize\ttfamily},
  numbers=left,
  showstringspaces=false,
  numberstyle=\tiny\color{gray},
  keywordstyle=\color{blue},
  commentstyle=\color{dkgreen},
  stringstyle=\color{mauve},
}

\tikzset{
  % every node/.style={minimum width=2em,draw,circle},
  % level 1/.style={sibling distance=2cm},
  level distance=1cm,
  edge from parent/.style=
  {draw,edge from parent path={(\tikzparentnode) -- (\tikzchildnode)}},
}

\newif\ifHideSolutions
\newcommand{\solution}[1]{\color{dkgreen}\textbf{Solution: }#1\color{black}}
\newcommand{\rubric}[1]{\color{dkgreen}{\bf Rubric:} #1\color{black}}

% \HideSolutionsfalse
% \ifHideSolutions
%   \renewcommand{\solution}[1]{}
%   \renewcommand{\rubric}[1]{}
% \fi

\newlength{\toppush}
\setlength{\toppush}{2\headheight}
\addtolength{\toppush}{\headsep}

\newcommand{\htitle}[2]{\noindent\vspace*{-\toppush}\newline\parbox{6.5in}
{\textit{Design and Analysis of Algorithms}\hfill\name\newline
Massachusetts Institute of Technology \hfill #2\newline
\profs\hfill #1 \vspace*{-.5ex}\newline
\mbox{}\hrulefill\mbox{}}\vspace*{1ex}\mbox{}\newline
\begin{center}{\Large\bf #1}\end{center}}

\newcommand{\handout}[2]{\thispagestyle{empty}
 \markboth{#1}{#1}
 \pagestyle{myheadings}\htitle{#1}{#2}}

\newcommand{\lecture}[3]{\thispagestyle{empty}
 \markboth{Lecture #1: #2}{Lecture #1: #2}
 \pagestyle{myheadings}\htitle{Lecture #1: #2}{#3}}

\newcommand{\htitlewithouttitle}[2]{\noindent\vspace*{-\toppush}\newline\parbox{6.5in}
{\textit{Design and Analysis of Algorithms}\hfill#2\newline
Massachusetts Institute of Technology \hfill 6.046\newline
\profs\hfill Handout #1\vspace*{-.5ex}\newline
\mbox{}\hrulefill\mbox{}}\vspace*{1ex}\mbox{}\newline}

\newcommand{\handoutwithouttitle}[2]{\thispagestyle{empty}
 \markboth{Handout \protect\ref{#1}}{Handout \protect\ref{#1}}
 \pagestyle{myheadings}\htitlewithouttitle{\protect\ref{#1}}{#2}}

\newcommand{\exam}[2]{% parameters: exam name, date
 \thispagestyle{empty}
 \markboth{\hspace{1cm}\subj\ #1\hspace{1in}Name\hrulefill\ \ }%
          {\subj\ #1\hspace{1in}Name\hrulefill\ \ }
 \pagestyle{myheadings}\examtitle{#1}{#2}
 \renewcommand{\theproblem}{Problem \arabic{problemnum}}
}
\newcommand{\examsolutions}[3]{% parameters: handout, exam name, date
 \thispagestyle{empty}
 \markboth{Handout \protect\ref{#1}: #2}{Handout \protect\ref{#1}: #2}
% \pagestyle{myheadings}\htitle{\protect\ref{#1}}{#2}{#3}
 \pagestyle{myheadings}\examsolutionstitle{\protect\ref{#1}} {#2}{#3}
 \renewcommand{\theproblem}{Problem \arabic{problemnum}}
}
\newcommand{\examsolutionstitle}[3]{\noindent\vspace*{-\toppush}\newline\parbox{6.5in}
{\textit{Design and Analysis of Algorithms}\hfill#3\newline
Massachusetts Institute of Technology \hfill 6.046\newline
%Singapore-MIT Alliance \hfill SMA5503\newline
\profs\hfill Handout #1\vspace*{-.5ex}\newline
\mbox{}\hrulefill\mbox{}}\vspace*{1ex}\mbox{}\newline
\begin{center}{\Large\bf #2}\end{center}}

\newcommand{\takehomeexam}[2]{% parameters: exam name, date
 \thispagestyle{empty}
 \markboth{\subj\ #1\hfill}{\subj\ #1\hfill}
 \pagestyle{myheadings}\examtitle{#1}{#2}
 \renewcommand{\theproblem}{Problem \arabic{problemnum}}
}

\makeatletter
\newcommand{\exambooklet}[2]{% parameters: exam name, date
 \thispagestyle{empty}
 \markboth{\subj\ #1}{\subj\ #1}
 \pagestyle{myheadings}\examtitle{#1}{#2}
 \renewcommand{\theproblem}{Problem \arabic{problemnum}}
 \renewcommand{\problem}{\newpage
 \item \let\@currentlabel=\theproblem
 \markboth{\subj\ #1, \theproblem}{\subj\ #1, \theproblem}}
}
\makeatother


\newcommand{\examtitle}[2]{\noindent\vspace*{-\toppush}\newline\parbox{6.5in}
{\textit{Design and Analysis of Algorithms}\hfill#2\newline
Massachusetts Institute of Technology \hfill 6.046 Spring 2019\newline
%Singapore-MIT Alliance \hfill SMA5503\newline
\profs\hfill #1\vspace*{-.5ex}\newline
\mbox{}\hrulefill\mbox{}}\vspace*{1ex}\mbox{}\newline
\begin{center}{\Large\bf #1}\end{center}}

\newcommand{\grader}[1]{\hspace{1cm}\textsf{\textbf{#1}}\hspace{1cm}}

\newcommand{\points}[1]{[#1 points]\ }
\newcommand{\parts}[1]
{
  \ifnum#1=1
  (1 part)
  \else
  (#1 parts)
  \fi
  \ 
}

\newcommand{\bparts}{\begin{problemparts}}
\newcommand{\eparts}{\end{problemparts}}
\newcommand{\ppart}{\problempart}

%\newcommand{\lg} {lg\ }

\setlength{\oddsidemargin}{0pt}
\setlength{\evensidemargin}{0pt}
\setlength{\textwidth}{6.5in}
\setlength{\topmargin}{0in}
\setlength{\textheight}{8.5in}


\newcommand{\Spawn}{{\bf spawn} }
\newcommand{\Sync}{{\bf sync}}

\newcommand{\cif}[1]{\mbox{if $#1$}}
\newcommand{\cwhen}[1]{\mbox{when $#1$}}

\newcounter{problemnum}
\newcommand{\theproblem}{Problem \theproblemsetnum-\arabic{problemnum}}
\newenvironment{problems}{
        \begin{list}{{\bf \theproblem. \hspace*{0.5em}}}
        {\setlength{\leftmargin}{0em}
         \setlength{\rightmargin}{0em}
         \setlength{\labelwidth}{0em}
         \setlength{\labelsep}{0em}
         \usecounter{problemnum}}}{\end{list}}
\makeatletter
\newcommand{\problem}[1][{}]{\item \let\@currentlabel=\theproblem \textbf{#1}}
\makeatother

\newcounter{problempartnum}[problemnum]
\newenvironment{problemparts}{
        \begin{list}{{\bf (\alph{problempartnum})}}
        {\setlength{\leftmargin}{2.5em}
         \setlength{\rightmargin}{2.5em}
         \setlength{\labelsep}{0.5em}}}{\end{list}}
\newcommand{\problempart}{\addtocounter{problempartnum}{1}\item}

\newenvironment{truefalseproblemparts}{
        \begin{list}{{\bf (\alph{problempartnum})\ \ \ T\ \ F\hfil}}
        {\setlength{\leftmargin}{4.5em}
         \setlength{\rightmargin}{2.5em}
         \setlength{\labelsep}{0.5em}
         \setlength{\labelwidth}{4.5em}}}{\end{list}}

\newcounter{exercisenum}
\newcommand{\theexercise}{Exercise \theproblemsetnum-\arabic{exercisenum}}
\newenvironment{exercises}{
        \begin{list}{{\bf \theexercise. \hspace*{0.5em}}}
        {\setlength{\leftmargin}{0em}
         \setlength{\rightmargin}{0em}
         \setlength{\labelwidth}{0em}
         \setlength{\labelsep}{0em}
        \usecounter{exercisenum}}}{\end{list}}
\makeatletter
\newcommand{\exercise}{\item \let\@currentlabel=\theexercise}
\makeatother

\newcounter{exercisepartnum}[exercisenum]
%\newcommand{\problem}[1]{\medskip\mbox{}\newline\noindent{\bf Problem #1.}\hspace*{1em}}
%\newcommand{\exercise}[1]{\medskip\mbox{}\newline\noindent{\bf Exercise #1.}\hspace*{1em}}

\newenvironment{exerciseparts}{
        \begin{list}{{\bf (\alph{exercisepartnum})}}
        {\setlength{\leftmargin}{2.5em}
         \setlength{\rightmargin}{2.5em}
         \setlength{\labelsep}{0.5em}}}{\end{list}}
\newcommand{\exercisepart}{\addtocounter{exercisepartnum}{1}\item}


% Macros to make captions print with small type and 'Figure xx' in bold.
\makeatletter
\def\fnum@figure{{\bf Figure \thefigure}}
\def\fnum@table{{\bf Table \thetable}}
\let\@mycaption\caption
%\long\def\@mycaption#1[#2]#3{\addcontentsline{\csname
%  ext@#1\endcsname}{#1}{\protect\numberline{\csname 
%  the#1\endcsname}{\ignorespaces #2}}\par
%  \begingroup
%    \@parboxrestore
%    \small
%    \@makecaption{\csname fnum@#1\endcsname}{\ignorespaces #3}\par
%  \endgroup}
%\def\mycaption{\refstepcounter\@captype \@dblarg{\@mycaption\@captype}}
%\makeatother
\let\mycaption\caption
%\newcommand{\figcaption}[1]{\mycaption[]{#1}}

\newcounter{totalcaptions}
\newcounter{totalart}

\newcommand{\figcaption}[1]{\addtocounter{totalcaptions}{1}\caption[]{#1}}

% \psfigures determines what to do for figures:
%       0 means just leave vertical space
%       1 means put a vertical rule and the figure name
%       2 means insert the PostScript version of the figure
%       3 means put the figure name flush left or right
\newcommand{\psfigures}{0}
\newcommand{\spacefigures}{\renewcommand{\psfigures}{0}}
\newcommand{\rulefigures}{\renewcommand{\psfigures}{1}}
\newcommand{\macfigures}{\renewcommand{\psfigures}{2}}
\newcommand{\namefigures}{\renewcommand{\psfigures}{3}}

\newcommand{\figpart}[1]{{\bf (#1)}\nolinebreak[2]\relax}
\newcommand{\figparts}[2]{{\bf (#1)--(#2)}\nolinebreak[2]\relax}


\macfigures     % STATE

% When calling \figspace, make sure to leave a blank line afterward!!
% \widefigspace is for figures that are more than 28pc wide.
\newlength{\halffigspace} \newlength{\wholefigspace}
\newlength{\figruleheight} \newlength{\figgap}
\newcommand{\setfiglengths}{\ifnum\psfigures=1\setlength{\figruleheight}{\hruleheight}\setlength{\figgap}{1em}\else\setlength{\figruleheight}{0pt}\setlength{\figgap}{0em}\fi}
\newcommand{\figspace}[2]{\ifnum\psfigures=0\leavefigspace{#1}\else%
\setfiglengths%
\setlength{\wholefigspace}{#1}\setlength{\halffigspace}{.5\wholefigspace}%
\rule[-\halffigspace]{\figruleheight}{\wholefigspace}\hspace{\figgap}#2\fi}
\newlength{\widefigspacewidth}
% Make \widefigspace put the figure flush right on the text page.
\newcommand{\widefigspace}[2]{
\ifnum\psfigures=0\leavefigspace{#1}\else%
\setfiglengths%
\setlength{\widefigspacewidth}{28pc}%
\addtolength{\widefigspacewidth}{-\figruleheight}%
\setlength{\wholefigspace}{#1}\setlength{\halffigspace}{.5\wholefigspace}%
\makebox[\widefigspacewidth][r]{#2\hspace{\figgap}}\rule[-\halffigspace]{\figruleheight}{\wholefigspace}\fi}
\newcommand{\leavefigspace}[1]{\setlength{\wholefigspace}{#1}\setlength{\halffigspace}{.5\wholefigspace}\rule[-\halffigspace]{0em}{\wholefigspace}}

% Commands for including figures with macpsfig.
% To use these commands, documentstyle ``macpsfig'' must be specified.
\newlength{\macfigfill}
\makeatother
\newlength{\bbx}
\newlength{\bby}
\newcommand{\macfigure}[5]{\addtocounter{totalart}{1}
\ifnum\psfigures=2%
\setlength{\bbx}{#2}\addtolength{\bbx}{#4}%
\setlength{\bby}{#3}\addtolength{\bby}{#5}%
\begin{flushleft}
\ifdim#4>28pc\setlength{\macfigfill}{#4}\addtolength{\macfigfill}{-28pc}\hspace*{-\macfigfill}\fi%
\mbox{\psfig{figure=./#1.ps,%
bbllx=#2,bblly=#3,bburx=\bbx,bbury=\bby}}
\end{flushleft}%
\else\ifdim#4>28pc\widefigspace{#5}{#1}\else\figspace{#5}{#1}\fi\fi}
\makeatletter

\newlength{\savearraycolsep}
\newcommand{\narrowarray}[1]{\setlength{\savearraycolsep}{\arraycolsep}\setlength{\arraycolsep}{#1\arraycolsep}}
\newcommand{\normalarray}{\setlength{\arraycolsep}{\savearraycolsep}}

\newcommand{\hint}{{\em Hint:\ }}

% Macros from /th/u/clr/mac.tex

\newcommand{\set}[1]{\left\{ #1 \right\}}
\newcommand{\abs}[1]{\left| #1\right|}
\newcommand{\card}[1]{\left| #1\right|}
\newcommand{\floor}[1]{\left\lfloor #1 \right\rfloor}
\newcommand{\ceil}[1]{\left\lceil #1 \right\rceil}
\newcommand{\ang}[1]{\ifmmode{\left\langle #1 \right\rangle}
   \else{$\left\langle${#1}$\right\rangle$}\fi}
        % the \if allows use outside mathmode,
        % but will swallow following space there!
\newcommand{\paren}[1]{\left( #1 \right)}
\newcommand{\bracket}[1]{\left[ #1 \right]}
\newcommand{\prob}[1]{\Pr\left\{ #1 \right\}}
\newcommand{\Var}{\mathop{\rm Var}\nolimits}
\newcommand{\expect}[1]{{\rm E}\left[ #1 \right]}
\newcommand{\expectsq}[1]{{\rm E}^2\left[ #1 \right]}
\newcommand{\variance}[1]{{\rm Var}\left[ #1 \right]}
\renewcommand{\choose}[2]{{{#1}\atopwithdelims(){#2}}}
\def\pmod#1{\allowbreak\mkern12mu({\rm mod}\,\,#1)}
\newcommand{\matx}[2]{\left(\begin{array}{*{#1}{c}}#2\end{array}\right)}
\newcommand{\Adj}{\mathop{\rm Adj}\nolimits}

\newtheorem{theorem}{Theorem}
\newtheorem{lemma}[theorem]{Lemma}
\newtheorem{corollary}[theorem]{Corollary}
\newtheorem{xample}{Example}
\newtheorem{definition}{Definition}
\newenvironment{example}{\begin{xample}\rm}{\end{xample}}
\newcommand{\proof}{\noindent{\em Proof.}\hspace{1em}}
\def\squarebox#1{\hbox to #1{\hfill\vbox to #1{\vfill}}}
\newcommand{\qedbox}{\vbox{\hrule\hbox{\vrule\squarebox{.667em}\vrule}\hrule}}
\newcommand{\qed}{\nopagebreak\mbox{}\hfill\qedbox\smallskip}
\newcommand{\eqnref}[1]{(\protect\ref{#1})}

%%\newcommand{\twodots}{\mathinner{\ldotp\ldotp}}
\newcommand{\transpose}{^{\mbox{\scriptsize \sf T}}}
\newcommand{\amortized}[1]{\widehat{#1}}

\newcommand{\punt}[1]{}

%%% command for putting definitions into boldface
% New style for defined terms, as of 2/23/88, redefined by THC.
\newcommand{\defn}[1]{{\boldmath\textit{\textbf{#1}}}}
\newcommand{\defi}[1]{{\textit{\textbf{#1\/}}}}

\newcommand{\red}{\leq_{\rm P}}
\newcommand{\lang}[1]{%
\ifmmode\mathord{\mathcode`-="702D\rm#1\mathcode`\-="2200}\else{\rm#1}\fi}

%\newcommand{\ckt}[1]{\ifmmode\mathord{\mathcode`-="702D\sc #1\mathcode`\-="2200}\else$\mathord{\mathcode`-="702D\sc #1\mathcode`\-="2200}$\fi}
\newcommand{\ckt}[1]{\ifmmode \sc #1\else$\sc #1$\fi}

%% Margin notes - use \notesfalse to turn off notes.
\setlength{\marginparwidth}{0.6in}
\reversemarginpar
\newif\ifnotes
\notestrue
\newcommand{\longnote}[1]{
  \ifnotes
    {\medskip\noindent Note: \marginpar[\hfill$\Longrightarrow$]
      {$\Longleftarrow$}{#1}\medskip}
  \fi}
\newcommand{\note}[1]{
  \ifnotes
    {\marginpar{\tiny \raggedright{#1}}}
  \fi}


\newcommand{\reals}{\mathbbm{R}}
\newcommand{\integers}{\mathbbm{Z}}
\newcommand{\naturals}{\mathbbm{N}}
\newcommand{\rationals}{\mathbbm{Q}}
\newcommand{\complex}{\mathbbm{C}}

\newcommand{\oldreals}{{\bf R}}
\newcommand{\oldintegers}{{\bf Z}}
\newcommand{\oldnaturals}{{\bf N}}
\newcommand{\oldrationals}{{\bf Q}}
\newcommand{\oldcomplex}{{\bf C}}

\newcommand{\w}{\omega}                 %% for fft chapter

\newenvironment{closeitemize}{\begin{list}
{$\bullet$}
{\setlength{\itemsep}{-0.2\baselineskip}
\setlength{\topsep}{0.2\baselineskip}
\setlength{\parskip}{0pt}}}
{\end{list}}

% These are necessary within a {problems} environment in order to restore
% the default separation between bullets and items.
\newenvironment{normalitemize}{\setlength{\labelsep}{0.5em}\begin{itemize}}
                              {\end{itemize}}
\newenvironment{normalenumerate}{\setlength{\labelsep}{0.5em}\begin{enumerate}}
                                {\end{enumerate}}

%\def\eqref#1{Equation~(\ref{eq:#1})}
%\newcommand{\eqref}[1]{Equation (\ref{eq:#1})}
\newcommand{\eqreftwo}[2]{Equations (\ref{eq:#1}) and~(\ref{eq:#2})}
\newcommand{\ineqref}[1]{Inequality~(\ref{ineq:#1})}
\newcommand{\ineqreftwo}[2]{Inequalities (\ref{ineq:#1}) and~(\ref{ineq:#2})}

\newcommand{\figref}[1]{Figure~\ref{fig:#1}}
\newcommand{\figreftwo}[2]{Figures \ref{fig:#1} and~\ref{fig:#2}}

\newcommand{\liref}[1]{line~\ref{li:#1}}
\newcommand{\Liref}[1]{Line~\ref{li:#1}}
\newcommand{\lirefs}[2]{lines \ref{li:#1}--\ref{li:#2}}
\newcommand{\Lirefs}[2]{Lines \ref{li:#1}--\ref{li:#2}}
\newcommand{\lireftwo}[2]{lines \ref{li:#1} and~\ref{li:#2}}
\newcommand{\lirefthree}[3]{lines \ref{li:#1}, \ref{li:#2}, and~\ref{li:#3}}

\newcommand{\lemlabel}[1]{\label{lem:#1}}
\newcommand{\lemref}[1]{Lemma~\ref{lem:#1}} 

\newcommand{\exref}[1]{Exercise~\ref{ex:#1}}

\newcommand{\handref}[1]{Handout~\ref{#1}}

\newcommand{\defref}[1]{Definition~\ref{def:#1}}

% (1997.8.16: Victor Luchangco)
% Modified \hlabel to only get date and to use handouts counter for number.
%   New \handout and \handoutwithouttitle commands in newmac.tex use this.
%   The date is referenced by <label>-date.
%   (Retained old definition as \hlabelold.)
%   Defined \hforcelabel to use an argument instead of the handouts counter.

\newcounter{handouts}
\setcounter{handouts}{0}

\newcommand{\hlabel}[2]{%
\stepcounter{handouts}
{\edef\next{\write\@auxout{\string\newlabel{#1}{{\arabic{handouts}}{0}}}}\next}
\write\@auxout{\string\newlabel{#1-date}{{#2}{0}}}
}

\newcommand{\hforcelabel}[3]{%          Does not step handouts counter.
\write\@auxout{\string\newlabel{#1}{{#2}{0}}}
\write\@auxout{\string\newlabel{#1-date}{{#3}{0}}}}


% less ugly underscore
% --juang, 2008 oct 05
\renewcommand{\_}{\vrule height 0 pt depth 0.4 pt width 0.5 em \,}

\usepackage{enumitem}
\newcommand{\theproblemsetnum}{6}
\newcommand{\releasedate}{Wednesday, May 1}
\newcommand{\partaduedate}{Friday, May 10}

\title{14.32 Problem Set \theproblemsetnum}

\begin{document}

\handout{Problem Set \theproblemsetnum}{\releasedate}
\textbf{All parts are due {\bf \partaduedate} at {\bf 11:59PM}}.

\setlength{\parindent}{0pt}
\medskip\hrulefill\medskip

{\bf Name:} Robert Durfee

\medskip

{\bf Collaborators:} None

\medskip\hrulefill

\begin{problems}

\problem  % Problem 1

\begin{problemparts}

\problempart  % Problem 1a

In the case were $\alpha_i$ is correlated with some of the regressors (yet
$\nu_{it}$ is still uncorrelated with all of the regressors), our estimates
from OLS would be biased and inconsistent. This comes from the violation of
the assumption,
$$ \cov\left(\varepsilon_{it}, x_{it}\right) = 0 $$
It is clear to see that this is violated as, $\varepsilon = \alpha_i +
\nu_{it}$ and we are told $\cov\left(\alpha_i, x_{it}\right) \neq 0$.
Therefore,
$$ \cov\left(\varepsilon_{it}, x_{it}\right) \neq 0 $$
Since this assumption is violated, the estimates from OLS will be
biased and inconsistent.

To fix this error, we have a few options. One option is to use the
first-differences model. This will cancel out all $\alpha_i$ terms and yield
a consistent estimator.

\problempart  % Problem 1b

If $\alpha_i$ is uncorrelated with the regressors (and $\nu_{it}$ is still
uncorrelated with all of the regressors), then $\cov\left(\varepsilon_{it},
x_{it}\right) = 0$ as $\varepsilon_{it} = \alpha_i + \nu_{it}$ and both of
which are uncorrelated with regressors $x_{it}$. This allows us to assume the
estimator is unbiased and consistent. However, this does not rule out serial
correlation (and heteroskedasticity, for that matter) in the error term which
would cause inconsistent hypothesis testing and confidence intervals.

\problempart  % Problem 1c

First we show the errors are homoskedastic after the transformation. 
$$ \var(\tilde{\varepsilon}_{it}) = \var(\varepsilon_{it} - \lambda
\bar{\varepsilon}_i) $$
First we split $\varepsilon_{it}$ into two uncorrelated components $\alpha_i$
and $\eta_{it}$.
$$ \var(\tilde{\varepsilon}_{it}) = \var((\alpha_i + \eta_{it}) - \lambda
(\alpha_i + \bar{\eta}_i) ) $$
Using the fact that $\alpha_i$ and $\eta_{it}$ are uncorrelated,
$$ \var(\tilde{\varepsilon}_{it}) = (1 - \lambda)^2 \var(\alpha_i) +
\var(\eta_{it} - \lambda \bar{\eta}_i) $$
Breaking up the empirical average of $\eta$,
\begin{align*}
    \var(\tilde{\varepsilon}_{it}) &= (1 - \lambda)^2 \var(\alpha_i) +
    \var\left(\eta_{it} - \frac{\lambda}{T} \sum_{r = 1}^T \eta_{ir}\right) \\
    &= (1 - \lambda)^2 \var(\alpha_i) + \var\left(\left(\frac{T -
    1}{T}\right) \eta_{it} - \frac{\lambda}{T} \sum_{r \neq t}
    \eta_{ir}\right)
\end{align*}
Now the $\eta_{it}$ is independent from the other $\eta_{ir}$ (shown below),
$$ \var(\tilde{\varepsilon}_{it}) = (1 - \lambda)^2 \var(\alpha_i) +
\left(\frac{T - 1}{T}\right)^2\var(\eta_{it}) -
\left(\frac{\lambda}{T}\right)^2\var\left(\sum_{r \neq t}\eta_{ir}\right) $$
Since $\var(\alpha_i) = \sigma_\alpha^2$ and $\var(\eta_{it}) =
\sigma_\eta^2$, this simplifies to,
\begin{align*}
    \var(\tilde{\varepsilon}_{it}) &= (1 - \lambda)^2 \sigma_\alpha^2 +
    \left(\frac{T - 1}{T}\right)^2\sigma_\eta^2 - \left(\frac{\lambda(T -
    1)}{T}\right)^2 \sigma_\eta^2 \\
    &= \boxed{(1 - \lambda)^2 \sigma_\alpha^2 + (1 - \lambda^2) \left(\frac{T
    - 1}{T}\right)^2 \sigma_\eta^2}
\end{align*}
Since this expression consists only of constants, it too is constant.
Therefore, we have homoskedasticity.

Now we show the errors are serially uncorrelated.
$$ \cov(\tilde{\varepsilon}_{it}, \tilde{\varepsilon}_{js}) =
\cov(\varepsilon_{it} - \lambda \bar{\varepsilon}_i, \varepsilon_{js} -
\lambda \bar{\varepsilon}_j) $$
If $i \neq j$, then this trivially reduces to zero as we can assume that no
individuals are correlated to one another. Therefore, we consider the case
where $t \neq s$. First we split $\varepsilon$ into two uncorrelated
components $\alpha_i$ and $\eta_{it}$.
$$ \cov(\tilde{\varepsilon}_{it}, \tilde{\varepsilon}_{js}) = \cov((\alpha_i
+ \eta_{it}) - \lambda (\alpha_i + \bar{\eta}_i), (\alpha_i + \eta_{is}) -
\lambda (\alpha_i + \bar{\eta}_i)) $$
Breaking this covariance up into components,
\begin{align*}
    \cov(\tilde{\varepsilon}_{it}, \tilde{\varepsilon}_{js}) &= \cov(\alpha_i
    + \eta_{it}, \alpha_i + \eta_{is}) - \lambda \cov(\alpha_i + \eta_{it},
    \alpha_i + \bar{\eta}_i) \\
    &\quad - \lambda \cov(\alpha_i + \bar{\eta}_i, \alpha_i + \eta_{is}) +
    \lambda^2 \var(\alpha_i + \bar{\eta}_i) \\
    &= \var(\alpha_i) + \cov(\alpha_i, \eta_{is}) + \cov(\eta_{it}, \alpha_i)
    + \cov(\eta_{it}, \eta_{is}) \\
    &\quad - \lambda\left(\var(\alpha_i) + \cov(\alpha_i, \bar{\eta}_i) +
    \cov(\eta_{it}, \alpha_i) + \cov(\eta_{it}, \bar{\eta}_i)\right) \\
    &\quad - \lambda\left(\var(\alpha_i) + \cov(\alpha_i, \eta_{is}) +
    \cov(\bar{\eta}_i, \alpha_i) + \cov(\bar{\eta}_i, \eta_{is})\right) \\
    &\quad + \lambda^2\left(\var(\alpha_i) + \var(\bar{\eta}_i)\right)
\end{align*}
By definition, we know that all $\alpha_i$ are uncorrelated with
$\eta_{i\cdot}$. Therefore, these covariances go to zero.
\begin{align*}
    \cov(\tilde{\varepsilon}_{it}, \tilde{\varepsilon}_{js}) &=
    \var(\alpha_i) + \cov(\eta_{it}, \eta_{is}) - \lambda\left(\var(\alpha_i)
    + \cov(\eta_{it}, \bar{\eta}_i)\right) \\
    &\quad - \lambda\left(\var(\alpha_i) + \cov(\bar{\eta}_i,
    \eta_{is})\right) + \lambda^2\left(\var(\alpha_i) +
    \var(\bar{\eta}_i)\right) \\
\end{align*}
Furthermore, the covariance between any $\eta_{it}$ and $\eta_{is}$ for $t
\neq s$ is taken to be zero. Therefore, we have,
$$ \cov(\tilde{\varepsilon}_{it}, \tilde{\varepsilon}_{js}) = \var(\alpha_i)
- 2\lambda\left(\var(\alpha_i) + \frac{1}{T}\var(\eta_{it})\right) +
\lambda^2\left(\var(\alpha_i) + \frac{1}{T^2}\sum_{r =
1}^T\var({\eta}_{ir})\right) $$
Lastly, substituting $\var(\alpha_i) = \sigma_\alpha^2$ and $\var(\eta_{it})
= \sigma_\eta^2$,
\begin{align*}
    \cov(\tilde{\varepsilon}_{it}, \tilde{\varepsilon}_{js}) &=
    \sigma_\alpha^2 - 2\lambda\left(\sigma_\alpha^2 +
    \frac{1}{T}\sigma_\eta^2\right) + \lambda^2\left(\sigma_\alpha^2 +
    \frac{1}{T^2}\sum_{r = 1}^T\sigma_\eta^2\right) \\
    &= \sigma_\alpha^2 - (2 - \lambda)\lambda\left(\sigma_\alpha^2 +
    \frac{1}{T}\sigma_\eta^2\right) \\
\end{align*}
Using the following value of $\lambda$,
$$ \lambda = 1 - \sqrt{\frac{\sigma_\eta^2}{\sigma_\eta^2 + T\sigma_\alpha^2}} $$
This expression reduces to zero,
$$ \cov(\tilde{\varepsilon}_{it}, \tilde{\varepsilon}_{js}) = \boxed{0} $$
Therefore, the error terms are serially uncorrelated.

\problempart  % Problem 1d

Now we guarantee that the errors are homoskedastic and not serially
correlated. This will ensure our variance estimates are consistent and will
lead to consistent hypothesis testing and confidence intervals. However, this
does not fix the correlation between $\alpha_i$ and the regressors.
Therefore, this method relies on the assumption $\cov\left(\alpha_i,
x_{it}\right) = 0$.

\problempart  % Problem 1e

\begin{enumerate}[label=\textbf{(\roman*)}]

    \item We can show that $\hat{\mathbb{E}}[\hat{u}_{it} \hat{u}_{is}]$ is a
    consistent estimator for $\sigma_\alpha^2$. To do this, we take the limit
    in probability,
    \begin{align*}
        \plim \hat{\mathbb{E}}[\hat{u}_{it} \hat{u}_{is}] &= \plim
        \hat{\mathbb{E}}[(\hat{\alpha}_i + \hat{\eta}_{it}) (\hat{\alpha}_i +
        \hat{\eta}_{is})] \\
        &= \plim \left(\hat{\mathbb{E}}[\hat{\alpha}_i^2 + \hat{\alpha}_i
        \hat{\eta}_{is} + \hat{\eta}_{it} \hat{\alpha}_i + \hat{\eta}_{it}
        \hat{\eta}_{is})]\right) \\
        &= \plim \left(\hat{\mathbb{E}}[\hat{\alpha}_i^2] +
        \hat{\mathbb{E}}[\hat{\alpha}_i \hat{\eta}_{is}] +
        \hat{\mathbb{E}}[\hat{\eta}_{it} \hat{\alpha}_i] +
        \hat{\mathbb{E}}[\hat{\eta}_{it} \hat{\eta}_{is}]\right) \\
    \end{align*}
    Since $\alpha_i$ and $\eta_{i\cdot}$ are uncorrelated by definition and
    so are $\eta_{it}$ and $\eta_{is}$, this reduces to,
    $$ \plim \hat{\mathbb{E}}[\hat{u}_{it} \hat{u}_{is}]= \plim
    \left(\hat{\mathbb{E}}[\hat{\alpha}_i^2] + \hat{\mathbb{E}}[\hat{\alpha}_i]
    \left(\hat{\mathbb{E}}[\hat{\eta}_{it}] +
    \hat{\mathbb{E}}[\hat{\eta}_{is}]\right) +
    \hat{\mathbb{E}}[\hat{\eta}_{it}] \hat{\mathbb{E}}[\hat{\eta}_{is}]\right) $$
    Taking the limit in probability,
    $$ \plim \hat{\mathbb{E}}[\hat{u}_{it} \hat{u}_{is}]=
    \mathbb{E}[\alpha_i^2] + \mathbb{E}[\alpha_i] \left(\mathbb{E}[\eta_{it}]
    + \mathbb{E}[\eta_{is}]\right) + \mathbb{E}[\eta_{it}]
    \mathbb{E}[\eta_{is}] $$
    Since there is a constant in the regression, we can assume all errors'
    average is zero. Therefore,
    $$ \plim \hat{\mathbb{E}}[\hat{u}_{it} \hat{u}_{is}]=
    \mathbb{E}[\alpha_i^2] $$
    The variance of $\alpha_i$ is equivalent to,
    $$ \var(\alpha_i) = \mathbb{E}[\alpha_i^2] - \mathbb{E}[\alpha_i]^2 $$
    But, once again, the average of the error is zero. Thus,
    $$ \var(\alpha_i) = \mathbb{E}[\alpha_i^2] $$
    Putting everything together,
    $$ \plim \hat{\mathbb{E}}[\hat{u}_{it} \hat{u}_{is}]= \boxed{\sigma_\alpha^2} $$

    \item We can also show that $\hat{\var}(\hat{u}_{it})$ is a consistent
    estimator for $\sigma_\alpha^2 + \sigma_\eta^2$. To do this, we take the
    limit in probability,
    $$ \plim \hat{\var}(\hat{u}_{it}) = \plim \hat{\var}(\hat{\alpha}_i +
    \hat{\eta}_{it}) $$
    Since $\alpha_i$ and $\eta_{it}$ are by definition uncorrelated,
    $$ \plim \hat{\var}(\hat{u}_{it}) = \plim
    \left(\hat{\var}(\hat{\alpha}_i) + \hat{\var}(\hat{\eta}_{it})\right) $$
    Taking the probability limit yields,
    $$ \plim \hat{\var}(\hat{u}_{it}) = \boxed{\sigma_\alpha^2 + \sigma_\eta^2} $$

    \item If we have have consistent estimators for $\sigma_\alpha^2$ and
    $\sigma_\alpha^2 + \sigma_\eta^2$, we can consistently estimate
    $\sigma_\alpha^2$, $\sigma_\eta^2$, and $\lambda$,
    $$ \hat{\sigma}_\alpha^2 = \boxed{\hat{\mathbb{E}}[\hat{u}_{it} \hat{u}_{is}]} $$
    $$ \hat{\sigma}_\eta^2 = \boxed{\hat{\var}(\hat{u}_{it}) -
    \hat{\mathbb{E}}[\hat{u}_{it} \hat{u}_{is}]} $$
    $$ \hat{\lambda} = \boxed{1 -
    \sqrt{\frac{\hat{\sigma}_\eta^2}{\hat{\sigma}_\eta^2 + T
    \hat{\sigma}_\alpha^2}}} $$

\end{enumerate}

\end{problemparts}

\newpage

\problem  % Problem 2

\begin{problemparts}

\problempart % Problem 2a

One factor could cause a bias in the estimator is the presence of a natural
trend. In other words, without the policy in place, obesity was already
trending upward or downward. If the trend were naturally upward, the policy
might've reduced the effect, but not enough to cause a net decrease in
obesity between the two time periods compared. Thus the policy's effect would
be understated (or even seem to increase obesity). The same could happen in
the reverse.

\problempart % Problem 2b

This is also a biased estimator as the two cities are likely very different.
The comparison would not take this into consideration. For example, a more
rural city could have less access to fast food restaurants and therefore
their obesity rate is just inherently lower than a more urban area. Thus, the
effect of the policy would be understated as the more rural city just
inherently has lower obesity. Different scenarios could lead to the opposite
conclusion as well.

\problempart % Problem 2c

The differences-in-differences estimator for the effect of the policy on
obesity is,
\begin{align*}
    \delta &= \bar{Y}_{\mathrm{NYC}}^{\mathrm{after}} -
    \left(\left(\bar{Y}_{\mathrm{WC}}^{\mathrm{after}} -
    \bar{Y}_{\mathrm{WC}}^{\mathrm{before}}\right) +
    \bar{Y}_{\mathrm{NYC}}^{\mathrm{before}}\right) \\
    &= \boxed{\left(\bar{Y}_{\mathrm{NYC}}^{\mathrm{after}}
    -\bar{Y}_{\mathrm{NYC}}^{\mathrm{before}}\right) -
    \left(\bar{Y}_{\mathrm{WC}}^{\mathrm{after}} -
    \bar{Y}_{\mathrm{WC}}^{\mathrm{before}}\right)}
\end{align*}

\problempart % Problem 2d

The differences-in-differences estimator takes both ideas from Part A and
Part B into consideration. Instead of comparing a singe individual over pre-
and post-treatment and neglecting natural trends (or other time-related
confounders) and instead of comparing two individuals over a single
post-treatment time period and neglecting individual differences, the two are
considered together.

\problempart % Problem 2e

You would still want the two individuals to share as much as possible to
allow the un-treated group be as close of an approximation of the treated
group before and after their treatment. If the two individuals are very
different, then the differences from the untreated group would not be a good
approximation of the expected changes in the treated group had it not been
treated.

\problempart % Problem 2f

We can define each of the quantities used in Part C in terms of regression
coefficients,
$$ \bar{Y}_{\mathrm{NYC}}^{\mathrm{after}} = \beta_0 + \beta_1 + \beta_2 +
\delta $$
$$ \bar{Y}_{\mathrm{NYC}}^{\mathrm{before}} = \beta_0 + \beta_1 $$
$$ \bar{Y}_{\mathrm{WC}}^{\mathrm{after}} = \beta_0 + \beta_2 $$
$$ \bar{Y}_{\mathrm{WC}}^{\mathrm{before}} = \beta_0 $$
Substituting into the differences-in-differences estimator from Part C,
\begin{align*}
    \delta &= \left(\bar{Y}_{\mathrm{NYC}}^{\mathrm{after}}
    -\bar{Y}_{\mathrm{NYC}}^{\mathrm{before}}\right) -
    \left(\bar{Y}_{\mathrm{WC}}^{\mathrm{after}} -
    \bar{Y}_{\mathrm{WC}}^{\mathrm{before}}\right) \\
    &= ((\beta_0 + \beta_1 + \beta_2 + \delta) - (\beta_0 + \beta_1)) -
    ((\beta_0 + \beta_2) - (\beta_0)) \\
    &= \delta
\end{align*}
Therefore, both specifications have the same result.

\end{problemparts}

\newpage

\problem  % Problem 3

\begin{problemparts}

\problempart % Problem 3a

First consider the latent dependent variable $y_i^*$. We use the following
model,
$$ y_i^* = X_i^T \beta - \varepsilon_i^* $$
We wish to derive the probability $\mathbb{P}(y_i = 1 \mid X_i)$. Using the
latent dependent variable,
\begin{align*}
    \mathbb{P}(y_i = 1 \mid X_i) &= \mathbb{P}(y_i^* \geq 0 \mid X_i) \\
    &= \mathbb{P}(X_i^T \beta - \varepsilon_i^* \geq 0 \mid X_i) \\
    &= \mathbb{P}(\varepsilon_i^* \leq X_i^T \beta \mid X_i)
\end{align*}
Using the CDF of the normal distribution given,
$$ \mathbb{P}(y_i = 1 \mid X_i) = \boxed{\Phi(X_i^T \beta)} $$

\problempart % Problem 3b

The likelihood of $\left\{y_i, X_i\right\}_{i = 1}^n$ is
given by the joint distribution and a candidate $\beta$.
$$ \mathcal{L}\left(\beta \mid \left\{y_i, X_i\right\}_{i = 1}^n\right) =
\mathbb{P} \left(\{y_i\}_{i = 1}^n \mid \{X_i\}_{i = 1}^n, \beta\right) $$
Since each $X_i$ is independent, the join distribution reduces to the product
of individual probabilities,
$$ \mathcal{L}\left(\beta \mid \left\{y_i, X_i\right\}_{i = 1}^n\right) =
\prod_{i = 1}^n \mathbb{P} \left(y_i \mid X_i, \beta\right) $$
To encode the choices between $y_i = 1$ and $y_i = 0$, we can write this as,
\begin{align*}
    \mathcal{L}\left(\beta \mid \left\{y_i, X_i\right\}_{i = 1}^n\right) &=
    \prod_{i = 1}^n \mathbb{P} \left(y_i = 1 \mid X_i, \beta\right)^{y_i}
    \mathbb{P} \left(y_i = 0 \mid X_i, \beta\right)^{(1 - y_i)} \\
    &= \prod_{i = 1}^n \mathbb{P} \left(y_i = 1 \mid X_i, \beta\right)^{y_i}
    \left(1 - \mathbb{P} \left(y_i = 1 \mid X_i, \beta\right)\right)^{(1 -
    y_i)} \\
    &= \boxed{\prod_{i = 1}^n \Phi \left(X_i^T \beta\right)^{y_i} \left(1 -
    \Phi \left(X_i^T \beta\right)\right)^{(1 - y_i)}} \\
\end{align*}
Taking the logarithm of this yields,
$$ \log \mathcal{L}\left(\beta \mid \left\{y_i, X_i\right\}_{i = 1}^n\right)
= \boxed{\sum_{i = 1}^n y_i \log \Phi \left(X_i^T \beta\right) + (1 - y_i)
\log \left(1 - \Phi \left(X_i^T \beta\right)\right)}$$

\problempart % Problem 3c

To find the estimate for $\beta$, we maximize the likelihood (or the
logarithm of the likelihood).
$$ \hat{\beta} = \arg\max_{\beta} \sum_{i = 1}^n y_i \log \Phi \left(X_i^T
\beta\right) + (1 - y_i) \log \left(1 - \Phi \left(X_i^T \beta\right)\right) $$
To find this maximum, we take the derivative with respect to $\beta$,
$$ \frac{\partial}{\partial \beta}(\cdot) = \sum_{i = 1}^n
\frac{y_i\Phi'\left(X_i^T \beta\right) X_i^T }{\Phi \left(X_i^T \beta\right)}
- \frac{(1 - y_i)\Phi'\left(X_i^T \beta\right) X_i^T}{\left(1 - \Phi
\left(X_i^T \beta\right)\right)}$$
Setting this equal to zero and solving for $\beta$ will yield $\hat{\beta}$
which maximizes the likelihood.
$$ 0 = \boxed{\sum_{i = 1}^n \frac{y_i\Phi'\left(X_i^T \hat{\beta}\right) X_i^T
}{\Phi \left(X_i^T \hat{\beta}\right)} - \frac{(1 - y_i)\Phi'\left(X_i^T
\hat{\beta}\right) X_i^T}{\left(1 - \Phi \left(X_i^T
\hat{\beta}\right)\right)}} $$
There is no closed form for this equation.

\problempart % Problem 3d

Starting with the definition of the probability from above,
$$ \mathbb{P}(y_i = 1 \mid X_i) = \boxed{\Phi(X_i^T \beta)} $$
We take the derivative with respect to $X_i$,
$$ \frac{\partial}{\partial X_i}(\cdot) = \boxed{\Phi'(X_i^T \beta) \beta} $$

\end{problemparts}

\end{problems}

\end{document}
