%
% 6.S077 problem set solutions template
%
\documentclass[12pt,twoside]{article}

\input{macros}
\usepackage{enumitem}
\newcommand{\theproblemsetnum}{6}
\newcommand{\releasedate}{Wednesday, May 1}
\newcommand{\partaduedate}{Friday, May 10}

\title{14.32 Problem Set \theproblemsetnum}

\begin{document}

\handout{Problem Set \theproblemsetnum}{\releasedate}
\textbf{All parts are due {\bf \partaduedate} at {\bf 9:00AM}}.

\setlength{\parindent}{0pt}
\medskip\hrulefill\medskip

{\bf Name:} Robert Durfee

\medskip

{\bf Collaborators:} None

\medskip\hrulefill

\begin{problems}

\problem  % Problem 1

\begin{problemparts}

\problempart  % Problem 1a

In the case were $\alpha_i$ is correlated with some of the regressors (yet
$\nu_{it}$ is still uncorrelated with all of the regressors), our estimates
from OLS would be biased and inconsistent. This comes from the violation of
the assumption,
$$ \cov\left(\varepsilon_{it}, x_{it}\right) = 0 $$
It is clear to see that this is violated as, $\varepsilon = \alpha_i +
\nu_{it}$ and we are told $\cov\left(\alpha_i, x_{it}\right) \neq 0$.
Therefore,
$$ \cov\left(\varepsilon_{it}, x_{it}\right) \neq 0 $$
Since this assumption is violated, the estimates from OLS will be
biased and inconsistent.

To fix this error, we have a few options. One option is to use the
first-differences model. This will cancel out all $\alpha_i$ terms and yield
a consistent estimator.

\problempart  % Problem 1b

If $\alpha_i$ is uncorrelated with the regressors (and $\nu_{it}$ is still
uncorrelated with all of the regressors), then $\cov\left(\varepsilon_{it},
x_{it}\right) = 0$ as $\varepsilon_{it} = \alpha_i + \nu_{it}$ and both of
which are uncorrelated with regressors $x_{it}$. This allows us to assume the
estimator is unbiased and consistent. However, this does not rule out serial
correlation (and heteroskedasticity, for that matter) in the error term which
would cause inconsistent hypothesis testing and confidence intervals.

\problempart  % Problem 1c

\problempart  % Problem 1d

Now we guarantee that the errors are homoskedastic and not serially
correlated. This will ensure our variance estimates are consistent and will
lead to consistent hypothesis testing and confidence intervals. However, this
does not fix the correlation between $\alpha_i$ and the regressors.
Therefore, this method relies on the assumption $\cov\left(\alpha_i,
x_{it}\right) = 0$.

\problempart  % Problem 1e

\begin{enumerate}[label=\textbf{(\roman*)}]
    \item
    \item
    \item
\end{enumerate}

\end{problemparts}

\newpage

\problem  % Problem 2

\begin{problemparts}

\problempart % Problem 2a

One factor could cause a bias in the estimator is the presence of a natural
trend. In other words, without the policy in place, obesity was already
trending upward or downward. If the trend were naturally upward, the policy
might've reduced the effect, but not enough to cause a net decrease in
obesity between the two time periods compared. Thus the policy's effect would
be understated (or even seem to increase obesity). The same could happen in
the reverse.

\problempart % Problem 2b

This is also a biased estimator as the two cities are likely very different.
The comparison would not take this into consideration. For example, a more
rural city could have less access to fast food restaurants and therefore
their obesity rate is just inherently lower than a more urban area. Thus, the
effect of the policy would be understated as the more rural city just
inherently has lower obesity. Different scenarios could lead to the opposite
conclusion as well.

\problempart % Problem 2c

\problempart % Problem 2d

The differences-in-differences estimator takes both ideas from Part A and
Part B into consideration. Instead of comparing a singe individual over pre-
and post-treatment and neglecting natural trends (or other time-related
confounders) and instead of comparing two individuals over a single
post-treatment time period and neglecting individual differences, the two are
considered together.

\problempart % Problem 2e

You would still want the two individuals to share as much as possible to
allow the un-treated group be as close of an approximation of the treated
group before and after their treatment. If the two individuals are very
different, then the differences from the untreated group would not be a good
approximation of the expected changes in the treated group had it not been
treated.

\problempart % Problem 2f

\end{problemparts}

\newpage

\problem  % Problem 3

\begin{problemparts}

\problempart % Problem 3a

\problempart % Problem 3b

\problempart % Problem 3c

\problempart % Problem 3d

\end{problemparts}

\end{problems}

\end{document}
