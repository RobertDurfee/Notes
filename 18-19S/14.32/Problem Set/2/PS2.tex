%
% 6.S077 problem set solutions template
%
\documentclass[12pt,twoside]{article}

\input{macros}
\newcommand{\theproblemsetnum}{2}
\newcommand{\releasedate}{Monday, February 25}
\newcommand{\partaduedate}{Monday, March 4}

\title{14.32 Problem Set \theproblemsetnum}

\begin{document}

\handout{Problem Set \theproblemsetnum}{\releasedate}
\textbf{All parts are due {\bf \partaduedate} at {\bf 9:00AM}}.

\setlength{\parindent}{0pt}
\medskip\hrulefill\medskip

{\bf Name:} Robert Durfee

\medskip

{\bf Collaborators:} None

\medskip\hrulefill

\begin{problems}

\problem  % Problem 1
%%%%%%%%%%%%%%%%%%%%%%%%%%%%%%%%%%%%%%%%%%%%%%%%%%%%%%%%%%%%%%%%%%%%%%%%%%%%%%%%
There are several reasons why multiple regression should be used instead of 
bivariate regression. Some principle reasons include:

\begin{enumerate}
    \item\ Control for other confounding variables that impact the coefficient
    you are trying to measure (e.g. including gender in a model designed to 
    measure the impact of education on earnings).
    \item\ Answer a question that simply involves multiple variables (e.g. 
    which has more impact on my GPA: an extra hour of sleep or an extra hour of 
    studying).
    \item\ Model nonlinear effects (e.g. if a response variable follows a
    quadratic trend with respect to a input variable, a model can be formed by
    adding an additional input variable as the square of the original input).
\end{enumerate}

\problem  % Problem 2

This is not a good econometric model and the resulting $R^2$ is not significant.

Mathematically, in 15-dimensional space, it is possible to fit a hyperplane
perfectly to exactly 15 data points. This is clearly understood in the 2-
dimensional and 3-dimensional spaces as a line is uniquely defined by 2 points 
and a plane is uniquely defined by 3 points. Extrapolating upwards, the same
applies to $n$-dimensional space and an $n$-dimensional hyperplane.

In a more statistical mindset, there are zero (or negative) degrees of freedom
as there are $15$ data points and $15$ indicator variables.
$$ df = n - k - 1 = 15 - 15 - 1 = -1 $$
Therefore, the model is completely rigid and depends significantly on which 15
observations were made. Furthermore, the model will not adequately represent
the population and not extrapolate well to new predictions.

\problem  % Problem 3

This $p$-value is associated with a single coefficient hypothesis test. Most
likely, the $t$-statistic and the Student's $t$-distribution were used. A $p$-
value indicates the probability that the null hypothesis is true. Assuming the
null hypothesis was that $x$ has no affect on the response variable, we can
say the probability that there is no effect from $x$ is 0.25. Therefore, the,
given a reasonable $\alpha$, we should not reject the null hypothesis as there
isn't significant evidence to suggest that $x$ has an effect on the response.

\newpage

\problem  % Problem 4

\begin{problemparts}

\problempart % Problem 4a

No, we do not know that $\mathbb{E}[\eta_i \mid chem_i] = 0$. 

From the problem statement, we know
$$ \mathbb{E}[\varepsilon_i \mid chem_i, dist_i] = 0 $$
Without the information about $dist_i$ in our model, we are asked
$$ \mathbb{E}[\eta_i \mid chem_i] = 0 $$
Taking the provided model as ground truth, this expectation can be rewritten
\begin{align*}
    \mathbb{E}[\eta_i \mid chem_i] &= \mathbb{E}[\beta_2 dist_i + \varepsilon_i
    \mid chem_i] \\
    &= \boxed{\beta_2 \mathbb{E}[dist_i \mid chem_i] + \mathbb{E}[\varepsilon_i 
    \mid chem_i]}
\end{align*}
In order for this to equal zero, both $\mathbb{E}[dist_i \mid chem_i]$ and
$\mathbb{E}[\varepsilon_i \mid chem_i]$ must equal zero. Yet, this information
is not specified and it is quite likely that it is not true as it makes sense
that the access to healthcare is correlated to chemotherapy treatments.

\problempart % Problem 4b
%%%%%%%%%%%%%%%%%%%%%%%%%%%%%%%%%%%%%%%%%%%%%%%%%%%%%%%%%%%%%%%%%%%%%%%%%%%%%%%%
By definition, the equation for the estimated coefficient $\hat{b}_1$ is
$$ \hat{b}_1 = \frac{\hat{\mathrm{cov}}(chem, y)}{\hat{\mathrm{var}}(chem)} $$
We can substitute the definition for the `ground truth' of $y$,
\begin{align*}
    \mathbb{E}[\hat{b}_1] &= \frac{\mathrm{cov}(chem, \beta_0 + \beta_1 chem + \beta_2 
    dist + \varepsilon)}{\mathrm{var}(chem)} \\
    &= \frac{\mathrm{cov}(chem, \beta_0) + \beta_1 \mathrm{cov}(chem, 
    chem) + \beta_2 \mathrm{cov}(chem, dist) + \mathrm{cov} (chem,
    \varepsilon)}{\mathrm{var}(chem)} \\
    &= \frac{\beta_1 \mathrm{var}(chem) + \beta_2\mathrm{cov}(chem, 
    dist)}{\mathrm{var}(chem)} \\
    &= \boxed{\beta_1 + \beta_2\frac{\mathrm{cov}(chem, dist)}{\mathrm{var}(chem)}}
\end{align*}

\problempart % Problem 4c

Given that a lower $dist$ causes a higher $chem$, then the covariance between
$dist$ and $chem$ must be less than zero and therefore the estimate is biased.
Assuming the $\beta_1$ is positive (i.e. that increasing chemotherapy increases 
survivability), then the effect of chemotherapy will be understated or less
than the actual. 

\problempart % Problem 4d

Intuitively, someone with smaller distance to a hospital should have a higher 
cancer survivability rate. Furthermore, the smaller the distance to a
hospital should increase the probability of getting chemotherapy treatment.

For simplicity, consider two groups: those close to a hospital and those far
from a hospital (with arbitrary, reasonable distinctions between close and
far). Those who receive chemotherapy in the close group should be reasonably 
expected to have higher survivability while those far away who receive
chemotherapy are likely to have lower survivability. It is reasonable to 
suspect the same for the groups when given no chemotherapy (if not even 
more reasonable). Thus, the measured relation of chemotherapy/no chemotherapy 
on survivability will be diminished.

\end{problemparts}

\problem  % Problem 5

\begin{problemparts}

\problempart % Problem 5a

Let the following be our hypothesis test,
$$ H_0 : \beta_1 = \beta_{1,0} $$
$$ H_1 : \beta_1 \neq \beta_{1,0} $$
Then, the $t$-statistic is defined as
$$ T_0 = \frac{\hat{\beta}_1 - \beta_{1,0}}{se(\hat{\beta}_1)} $$
Substituting the provided hypothesis test values,
$$ T_0 = \boxed{\frac{\hat{\beta}_1}{se(\hat{\beta}_1)}} $$

\problempart % Problem 5b

This test corresponds to the Student's $t$-distribution with $\boxed{n - (k + 1)}$
degrees of freedom.

\problempart % Problem 5c

Since the Student's $t$-distribution is approximately normal for degrees of 
freedom greater than $30$ and $49 > 30$, we can use the following values to 
conclude a statistically significant $\hat{\beta}_1$ with $\alpha = 0.05$
$$\vert T_0 \vert > 1.96$$
To be more specific, we can use the Student's $t$-distribution to get
$$\boxed{\vert T_0 \vert > 2.01}$$

\problempart % Problem 5d

Yes, we have sufficient evidence to reject the null hypothesis in favor of
the alternative. This arises from the fact that $\vert -2.30 \vert > 2.01$
and thus the probability that this $\hat{\beta}_1$ was measured if $\beta_1
= 0$ is less than 5\%.

\end{problemparts}

\newpage

\problem  % Problem 6

The coefficient for education in the multivariate regression is
$$ \hat{\beta}_{educ, m} = 0.0391199 $$
The coefficient for education in the bivariate regression is
$$ \hat{\beta}_{educ, s} = 0.0598392 $$
The difference between these two is
$$ \hat{\beta}_{educ, s} - \hat{\beta}_{educ, m} = 
\boxed{0.0207193} $$
These two values are different because there is a correlation between
education and IQ and IQ is important for earnings. Therefore, there is omitted
variable bias in the bivariate regression model and the actual effect of
education is overstated.

\problem  % Problem 7

\begin{problemparts}

\problempart % Problem 7a

From the multivariate regression, the coefficient for IQ is determined
$$ \beta_{IQ} = 0.0058631 $$
The regression coefficient for the regression of $IQ$ with respect to $educ$ will
give us the covariance vs the variance term,
$$ \frac{\hat{\mathrm{cov}}(educ, IQ)}{\hat{\mathrm{var}}(educ)} = 3.533829 $$
Therefore, the bias is 
$$ \hat{\beta}_{IQ} \frac{\hat{\mathrm{cov}}(educ, IQ)}{\hat{\mathrm{var}}(educ)}
= 0.0058631 \cdot 3.533829 = \boxed{0.0207192} $$ 

\problempart % Problem 7b

This value is almost exactly the same as the value from Part A. This makes sense
as the bias represents the difference between the coefficient with the omitted
variable and without. Therefore, the difference between coefficients with and
without the $IQ$ variable should reflect the bias.

\problempart % Problem 7c

The amount of bias corrected between including $IQ$ and leaving it out is
$\boxed{0.0207192}$.

\end{problemparts}

\problem  % Problem 8

Although we reduced the bias with respect to a single omitted variable, this does
not necessarily mean that there is not another omitted variable that is causing
bias in our estimator for the coefficients. As a result, it is unlikely that our 
estimator is \textit{completely} unbiased.

\end{problems}

\end{document}


