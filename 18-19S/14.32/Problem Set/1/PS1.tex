%
% 6.S077 problem set solutions template
%
\documentclass[12pt,twoside]{article}

\newcommand{\name}{}

\usepackage{amssymb}
\usepackage{amsmath}
\usepackage{graphicx}
\usepackage{latexsym}
\usepackage{times,url}
\usepackage{cprotect}
\usepackage{listings}
\usepackage{graphicx}
\usepackage[table]{xcolor}
\usepackage[letterpaper]{geometry}
\usepackage{tikz-qtree}
\usepackage{enumerate}

\newcommand{\profs}{Mauricio Karchmer, Aleksander Madry, Bruce Tidor}
\newcommand{\subj}{6.046}
\newcommand{\ttt}[1]{{\tt\small #1}}

\definecolor{dkgreen}{rgb}{0,0.6,0}
\definecolor{gray}{rgb}{0.5,0.5,0.5}
\definecolor{mauve}{rgb}{0.58,0,0.82}

\lstset{
  language=Python,
  aboveskip=1pc,
  belowskip=1pc,
  basicstyle={\footnotesize\ttfamily},
  numbers=left,
  showstringspaces=false,
  numberstyle=\tiny\color{gray},
  keywordstyle=\color{blue},
  commentstyle=\color{dkgreen},
  stringstyle=\color{mauve},
}

\tikzset{
  % every node/.style={minimum width=2em,draw,circle},
  % level 1/.style={sibling distance=2cm},
  level distance=1cm,
  edge from parent/.style=
  {draw,edge from parent path={(\tikzparentnode) -- (\tikzchildnode)}},
}

\newif\ifHideSolutions
\newcommand{\solution}[1]{\color{dkgreen}\textbf{Solution: }#1\color{black}}
\newcommand{\rubric}[1]{\color{dkgreen}{\bf Rubric:} #1\color{black}}

% \HideSolutionsfalse
% \ifHideSolutions
%   \renewcommand{\solution}[1]{}
%   \renewcommand{\rubric}[1]{}
% \fi

\newlength{\toppush}
\setlength{\toppush}{2\headheight}
\addtolength{\toppush}{\headsep}

\newcommand{\htitle}[2]{\noindent\vspace*{-\toppush}\newline\parbox{6.5in}
{\textit{Design and Analysis of Algorithms}\hfill\name\newline
Massachusetts Institute of Technology \hfill #2\newline
\profs\hfill #1 \vspace*{-.5ex}\newline
\mbox{}\hrulefill\mbox{}}\vspace*{1ex}\mbox{}\newline
\begin{center}{\Large\bf #1}\end{center}}

\newcommand{\handout}[2]{\thispagestyle{empty}
 \markboth{#1}{#1}
 \pagestyle{myheadings}\htitle{#1}{#2}}

\newcommand{\lecture}[3]{\thispagestyle{empty}
 \markboth{Lecture #1: #2}{Lecture #1: #2}
 \pagestyle{myheadings}\htitle{Lecture #1: #2}{#3}}

\newcommand{\htitlewithouttitle}[2]{\noindent\vspace*{-\toppush}\newline\parbox{6.5in}
{\textit{Design and Analysis of Algorithms}\hfill#2\newline
Massachusetts Institute of Technology \hfill 6.046\newline
\profs\hfill Handout #1\vspace*{-.5ex}\newline
\mbox{}\hrulefill\mbox{}}\vspace*{1ex}\mbox{}\newline}

\newcommand{\handoutwithouttitle}[2]{\thispagestyle{empty}
 \markboth{Handout \protect\ref{#1}}{Handout \protect\ref{#1}}
 \pagestyle{myheadings}\htitlewithouttitle{\protect\ref{#1}}{#2}}

\newcommand{\exam}[2]{% parameters: exam name, date
 \thispagestyle{empty}
 \markboth{\hspace{1cm}\subj\ #1\hspace{1in}Name\hrulefill\ \ }%
          {\subj\ #1\hspace{1in}Name\hrulefill\ \ }
 \pagestyle{myheadings}\examtitle{#1}{#2}
 \renewcommand{\theproblem}{Problem \arabic{problemnum}}
}
\newcommand{\examsolutions}[3]{% parameters: handout, exam name, date
 \thispagestyle{empty}
 \markboth{Handout \protect\ref{#1}: #2}{Handout \protect\ref{#1}: #2}
% \pagestyle{myheadings}\htitle{\protect\ref{#1}}{#2}{#3}
 \pagestyle{myheadings}\examsolutionstitle{\protect\ref{#1}} {#2}{#3}
 \renewcommand{\theproblem}{Problem \arabic{problemnum}}
}
\newcommand{\examsolutionstitle}[3]{\noindent\vspace*{-\toppush}\newline\parbox{6.5in}
{\textit{Design and Analysis of Algorithms}\hfill#3\newline
Massachusetts Institute of Technology \hfill 6.046\newline
%Singapore-MIT Alliance \hfill SMA5503\newline
\profs\hfill Handout #1\vspace*{-.5ex}\newline
\mbox{}\hrulefill\mbox{}}\vspace*{1ex}\mbox{}\newline
\begin{center}{\Large\bf #2}\end{center}}

\newcommand{\takehomeexam}[2]{% parameters: exam name, date
 \thispagestyle{empty}
 \markboth{\subj\ #1\hfill}{\subj\ #1\hfill}
 \pagestyle{myheadings}\examtitle{#1}{#2}
 \renewcommand{\theproblem}{Problem \arabic{problemnum}}
}

\makeatletter
\newcommand{\exambooklet}[2]{% parameters: exam name, date
 \thispagestyle{empty}
 \markboth{\subj\ #1}{\subj\ #1}
 \pagestyle{myheadings}\examtitle{#1}{#2}
 \renewcommand{\theproblem}{Problem \arabic{problemnum}}
 \renewcommand{\problem}{\newpage
 \item \let\@currentlabel=\theproblem
 \markboth{\subj\ #1, \theproblem}{\subj\ #1, \theproblem}}
}
\makeatother


\newcommand{\examtitle}[2]{\noindent\vspace*{-\toppush}\newline\parbox{6.5in}
{\textit{Design and Analysis of Algorithms}\hfill#2\newline
Massachusetts Institute of Technology \hfill 6.046 Spring 2019\newline
%Singapore-MIT Alliance \hfill SMA5503\newline
\profs\hfill #1\vspace*{-.5ex}\newline
\mbox{}\hrulefill\mbox{}}\vspace*{1ex}\mbox{}\newline
\begin{center}{\Large\bf #1}\end{center}}

\newcommand{\grader}[1]{\hspace{1cm}\textsf{\textbf{#1}}\hspace{1cm}}

\newcommand{\points}[1]{[#1 points]\ }
\newcommand{\parts}[1]
{
  \ifnum#1=1
  (1 part)
  \else
  (#1 parts)
  \fi
  \ 
}

\newcommand{\bparts}{\begin{problemparts}}
\newcommand{\eparts}{\end{problemparts}}
\newcommand{\ppart}{\problempart}

%\newcommand{\lg} {lg\ }

\setlength{\oddsidemargin}{0pt}
\setlength{\evensidemargin}{0pt}
\setlength{\textwidth}{6.5in}
\setlength{\topmargin}{0in}
\setlength{\textheight}{8.5in}


\newcommand{\Spawn}{{\bf spawn} }
\newcommand{\Sync}{{\bf sync}}

\newcommand{\cif}[1]{\mbox{if $#1$}}
\newcommand{\cwhen}[1]{\mbox{when $#1$}}

\newcounter{problemnum}
\newcommand{\theproblem}{Problem \theproblemsetnum-\arabic{problemnum}}
\newenvironment{problems}{
        \begin{list}{{\bf \theproblem. \hspace*{0.5em}}}
        {\setlength{\leftmargin}{0em}
         \setlength{\rightmargin}{0em}
         \setlength{\labelwidth}{0em}
         \setlength{\labelsep}{0em}
         \usecounter{problemnum}}}{\end{list}}
\makeatletter
\newcommand{\problem}[1][{}]{\item \let\@currentlabel=\theproblem \textbf{#1}}
\makeatother

\newcounter{problempartnum}[problemnum]
\newenvironment{problemparts}{
        \begin{list}{{\bf (\alph{problempartnum})}}
        {\setlength{\leftmargin}{2.5em}
         \setlength{\rightmargin}{2.5em}
         \setlength{\labelsep}{0.5em}}}{\end{list}}
\newcommand{\problempart}{\addtocounter{problempartnum}{1}\item}

\newenvironment{truefalseproblemparts}{
        \begin{list}{{\bf (\alph{problempartnum})\ \ \ T\ \ F\hfil}}
        {\setlength{\leftmargin}{4.5em}
         \setlength{\rightmargin}{2.5em}
         \setlength{\labelsep}{0.5em}
         \setlength{\labelwidth}{4.5em}}}{\end{list}}

\newcounter{exercisenum}
\newcommand{\theexercise}{Exercise \theproblemsetnum-\arabic{exercisenum}}
\newenvironment{exercises}{
        \begin{list}{{\bf \theexercise. \hspace*{0.5em}}}
        {\setlength{\leftmargin}{0em}
         \setlength{\rightmargin}{0em}
         \setlength{\labelwidth}{0em}
         \setlength{\labelsep}{0em}
        \usecounter{exercisenum}}}{\end{list}}
\makeatletter
\newcommand{\exercise}{\item \let\@currentlabel=\theexercise}
\makeatother

\newcounter{exercisepartnum}[exercisenum]
%\newcommand{\problem}[1]{\medskip\mbox{}\newline\noindent{\bf Problem #1.}\hspace*{1em}}
%\newcommand{\exercise}[1]{\medskip\mbox{}\newline\noindent{\bf Exercise #1.}\hspace*{1em}}

\newenvironment{exerciseparts}{
        \begin{list}{{\bf (\alph{exercisepartnum})}}
        {\setlength{\leftmargin}{2.5em}
         \setlength{\rightmargin}{2.5em}
         \setlength{\labelsep}{0.5em}}}{\end{list}}
\newcommand{\exercisepart}{\addtocounter{exercisepartnum}{1}\item}


% Macros to make captions print with small type and 'Figure xx' in bold.
\makeatletter
\def\fnum@figure{{\bf Figure \thefigure}}
\def\fnum@table{{\bf Table \thetable}}
\let\@mycaption\caption
%\long\def\@mycaption#1[#2]#3{\addcontentsline{\csname
%  ext@#1\endcsname}{#1}{\protect\numberline{\csname 
%  the#1\endcsname}{\ignorespaces #2}}\par
%  \begingroup
%    \@parboxrestore
%    \small
%    \@makecaption{\csname fnum@#1\endcsname}{\ignorespaces #3}\par
%  \endgroup}
%\def\mycaption{\refstepcounter\@captype \@dblarg{\@mycaption\@captype}}
%\makeatother
\let\mycaption\caption
%\newcommand{\figcaption}[1]{\mycaption[]{#1}}

\newcounter{totalcaptions}
\newcounter{totalart}

\newcommand{\figcaption}[1]{\addtocounter{totalcaptions}{1}\caption[]{#1}}

% \psfigures determines what to do for figures:
%       0 means just leave vertical space
%       1 means put a vertical rule and the figure name
%       2 means insert the PostScript version of the figure
%       3 means put the figure name flush left or right
\newcommand{\psfigures}{0}
\newcommand{\spacefigures}{\renewcommand{\psfigures}{0}}
\newcommand{\rulefigures}{\renewcommand{\psfigures}{1}}
\newcommand{\macfigures}{\renewcommand{\psfigures}{2}}
\newcommand{\namefigures}{\renewcommand{\psfigures}{3}}

\newcommand{\figpart}[1]{{\bf (#1)}\nolinebreak[2]\relax}
\newcommand{\figparts}[2]{{\bf (#1)--(#2)}\nolinebreak[2]\relax}


\macfigures     % STATE

% When calling \figspace, make sure to leave a blank line afterward!!
% \widefigspace is for figures that are more than 28pc wide.
\newlength{\halffigspace} \newlength{\wholefigspace}
\newlength{\figruleheight} \newlength{\figgap}
\newcommand{\setfiglengths}{\ifnum\psfigures=1\setlength{\figruleheight}{\hruleheight}\setlength{\figgap}{1em}\else\setlength{\figruleheight}{0pt}\setlength{\figgap}{0em}\fi}
\newcommand{\figspace}[2]{\ifnum\psfigures=0\leavefigspace{#1}\else%
\setfiglengths%
\setlength{\wholefigspace}{#1}\setlength{\halffigspace}{.5\wholefigspace}%
\rule[-\halffigspace]{\figruleheight}{\wholefigspace}\hspace{\figgap}#2\fi}
\newlength{\widefigspacewidth}
% Make \widefigspace put the figure flush right on the text page.
\newcommand{\widefigspace}[2]{
\ifnum\psfigures=0\leavefigspace{#1}\else%
\setfiglengths%
\setlength{\widefigspacewidth}{28pc}%
\addtolength{\widefigspacewidth}{-\figruleheight}%
\setlength{\wholefigspace}{#1}\setlength{\halffigspace}{.5\wholefigspace}%
\makebox[\widefigspacewidth][r]{#2\hspace{\figgap}}\rule[-\halffigspace]{\figruleheight}{\wholefigspace}\fi}
\newcommand{\leavefigspace}[1]{\setlength{\wholefigspace}{#1}\setlength{\halffigspace}{.5\wholefigspace}\rule[-\halffigspace]{0em}{\wholefigspace}}

% Commands for including figures with macpsfig.
% To use these commands, documentstyle ``macpsfig'' must be specified.
\newlength{\macfigfill}
\makeatother
\newlength{\bbx}
\newlength{\bby}
\newcommand{\macfigure}[5]{\addtocounter{totalart}{1}
\ifnum\psfigures=2%
\setlength{\bbx}{#2}\addtolength{\bbx}{#4}%
\setlength{\bby}{#3}\addtolength{\bby}{#5}%
\begin{flushleft}
\ifdim#4>28pc\setlength{\macfigfill}{#4}\addtolength{\macfigfill}{-28pc}\hspace*{-\macfigfill}\fi%
\mbox{\psfig{figure=./#1.ps,%
bbllx=#2,bblly=#3,bburx=\bbx,bbury=\bby}}
\end{flushleft}%
\else\ifdim#4>28pc\widefigspace{#5}{#1}\else\figspace{#5}{#1}\fi\fi}
\makeatletter

\newlength{\savearraycolsep}
\newcommand{\narrowarray}[1]{\setlength{\savearraycolsep}{\arraycolsep}\setlength{\arraycolsep}{#1\arraycolsep}}
\newcommand{\normalarray}{\setlength{\arraycolsep}{\savearraycolsep}}

\newcommand{\hint}{{\em Hint:\ }}

% Macros from /th/u/clr/mac.tex

\newcommand{\set}[1]{\left\{ #1 \right\}}
\newcommand{\abs}[1]{\left| #1\right|}
\newcommand{\card}[1]{\left| #1\right|}
\newcommand{\floor}[1]{\left\lfloor #1 \right\rfloor}
\newcommand{\ceil}[1]{\left\lceil #1 \right\rceil}
\newcommand{\ang}[1]{\ifmmode{\left\langle #1 \right\rangle}
   \else{$\left\langle${#1}$\right\rangle$}\fi}
        % the \if allows use outside mathmode,
        % but will swallow following space there!
\newcommand{\paren}[1]{\left( #1 \right)}
\newcommand{\bracket}[1]{\left[ #1 \right]}
\newcommand{\prob}[1]{\Pr\left\{ #1 \right\}}
\newcommand{\Var}{\mathop{\rm Var}\nolimits}
\newcommand{\expect}[1]{{\rm E}\left[ #1 \right]}
\newcommand{\expectsq}[1]{{\rm E}^2\left[ #1 \right]}
\newcommand{\variance}[1]{{\rm Var}\left[ #1 \right]}
\renewcommand{\choose}[2]{{{#1}\atopwithdelims(){#2}}}
\def\pmod#1{\allowbreak\mkern12mu({\rm mod}\,\,#1)}
\newcommand{\matx}[2]{\left(\begin{array}{*{#1}{c}}#2\end{array}\right)}
\newcommand{\Adj}{\mathop{\rm Adj}\nolimits}

\newtheorem{theorem}{Theorem}
\newtheorem{lemma}[theorem]{Lemma}
\newtheorem{corollary}[theorem]{Corollary}
\newtheorem{xample}{Example}
\newtheorem{definition}{Definition}
\newenvironment{example}{\begin{xample}\rm}{\end{xample}}
\newcommand{\proof}{\noindent{\em Proof.}\hspace{1em}}
\def\squarebox#1{\hbox to #1{\hfill\vbox to #1{\vfill}}}
\newcommand{\qedbox}{\vbox{\hrule\hbox{\vrule\squarebox{.667em}\vrule}\hrule}}
\newcommand{\qed}{\nopagebreak\mbox{}\hfill\qedbox\smallskip}
\newcommand{\eqnref}[1]{(\protect\ref{#1})}

%%\newcommand{\twodots}{\mathinner{\ldotp\ldotp}}
\newcommand{\transpose}{^{\mbox{\scriptsize \sf T}}}
\newcommand{\amortized}[1]{\widehat{#1}}

\newcommand{\punt}[1]{}

%%% command for putting definitions into boldface
% New style for defined terms, as of 2/23/88, redefined by THC.
\newcommand{\defn}[1]{{\boldmath\textit{\textbf{#1}}}}
\newcommand{\defi}[1]{{\textit{\textbf{#1\/}}}}

\newcommand{\red}{\leq_{\rm P}}
\newcommand{\lang}[1]{%
\ifmmode\mathord{\mathcode`-="702D\rm#1\mathcode`\-="2200}\else{\rm#1}\fi}

%\newcommand{\ckt}[1]{\ifmmode\mathord{\mathcode`-="702D\sc #1\mathcode`\-="2200}\else$\mathord{\mathcode`-="702D\sc #1\mathcode`\-="2200}$\fi}
\newcommand{\ckt}[1]{\ifmmode \sc #1\else$\sc #1$\fi}

%% Margin notes - use \notesfalse to turn off notes.
\setlength{\marginparwidth}{0.6in}
\reversemarginpar
\newif\ifnotes
\notestrue
\newcommand{\longnote}[1]{
  \ifnotes
    {\medskip\noindent Note: \marginpar[\hfill$\Longrightarrow$]
      {$\Longleftarrow$}{#1}\medskip}
  \fi}
\newcommand{\note}[1]{
  \ifnotes
    {\marginpar{\tiny \raggedright{#1}}}
  \fi}


\newcommand{\reals}{\mathbbm{R}}
\newcommand{\integers}{\mathbbm{Z}}
\newcommand{\naturals}{\mathbbm{N}}
\newcommand{\rationals}{\mathbbm{Q}}
\newcommand{\complex}{\mathbbm{C}}

\newcommand{\oldreals}{{\bf R}}
\newcommand{\oldintegers}{{\bf Z}}
\newcommand{\oldnaturals}{{\bf N}}
\newcommand{\oldrationals}{{\bf Q}}
\newcommand{\oldcomplex}{{\bf C}}

\newcommand{\w}{\omega}                 %% for fft chapter

\newenvironment{closeitemize}{\begin{list}
{$\bullet$}
{\setlength{\itemsep}{-0.2\baselineskip}
\setlength{\topsep}{0.2\baselineskip}
\setlength{\parskip}{0pt}}}
{\end{list}}

% These are necessary within a {problems} environment in order to restore
% the default separation between bullets and items.
\newenvironment{normalitemize}{\setlength{\labelsep}{0.5em}\begin{itemize}}
                              {\end{itemize}}
\newenvironment{normalenumerate}{\setlength{\labelsep}{0.5em}\begin{enumerate}}
                                {\end{enumerate}}

%\def\eqref#1{Equation~(\ref{eq:#1})}
%\newcommand{\eqref}[1]{Equation (\ref{eq:#1})}
\newcommand{\eqreftwo}[2]{Equations (\ref{eq:#1}) and~(\ref{eq:#2})}
\newcommand{\ineqref}[1]{Inequality~(\ref{ineq:#1})}
\newcommand{\ineqreftwo}[2]{Inequalities (\ref{ineq:#1}) and~(\ref{ineq:#2})}

\newcommand{\figref}[1]{Figure~\ref{fig:#1}}
\newcommand{\figreftwo}[2]{Figures \ref{fig:#1} and~\ref{fig:#2}}

\newcommand{\liref}[1]{line~\ref{li:#1}}
\newcommand{\Liref}[1]{Line~\ref{li:#1}}
\newcommand{\lirefs}[2]{lines \ref{li:#1}--\ref{li:#2}}
\newcommand{\Lirefs}[2]{Lines \ref{li:#1}--\ref{li:#2}}
\newcommand{\lireftwo}[2]{lines \ref{li:#1} and~\ref{li:#2}}
\newcommand{\lirefthree}[3]{lines \ref{li:#1}, \ref{li:#2}, and~\ref{li:#3}}

\newcommand{\lemlabel}[1]{\label{lem:#1}}
\newcommand{\lemref}[1]{Lemma~\ref{lem:#1}} 

\newcommand{\exref}[1]{Exercise~\ref{ex:#1}}

\newcommand{\handref}[1]{Handout~\ref{#1}}

\newcommand{\defref}[1]{Definition~\ref{def:#1}}

% (1997.8.16: Victor Luchangco)
% Modified \hlabel to only get date and to use handouts counter for number.
%   New \handout and \handoutwithouttitle commands in newmac.tex use this.
%   The date is referenced by <label>-date.
%   (Retained old definition as \hlabelold.)
%   Defined \hforcelabel to use an argument instead of the handouts counter.

\newcounter{handouts}
\setcounter{handouts}{0}

\newcommand{\hlabel}[2]{%
\stepcounter{handouts}
{\edef\next{\write\@auxout{\string\newlabel{#1}{{\arabic{handouts}}{0}}}}\next}
\write\@auxout{\string\newlabel{#1-date}{{#2}{0}}}
}

\newcommand{\hforcelabel}[3]{%          Does not step handouts counter.
\write\@auxout{\string\newlabel{#1}{{#2}{0}}}
\write\@auxout{\string\newlabel{#1-date}{{#3}{0}}}}


% less ugly underscore
% --juang, 2008 oct 05
\renewcommand{\_}{\vrule height 0 pt depth 0.4 pt width 0.5 em \,}

\newcommand{\theproblemsetnum}{1}
\newcommand{\releasedate}{Monday, February 11}
\newcommand{\partaduedate}{Tuesday, February 19}

\title{14.32 Problem Set \theproblemsetnum}

\begin{document}

\handout{Problem Set \theproblemsetnum}{\releasedate}
\textbf{All parts are due {\bf \partaduedate} at {\bf 9:00AM}}.

\setlength{\parindent}{0pt}
\medskip\hrulefill\medskip

{\bf Name:} Robert Durfee

\medskip

{\bf Collaborators:} None

\medskip\hrulefill

\begin{problems}

\problem  % Problem 1

\begin{problemparts}

\problempart % Problem 1a

The definition for the expected value of a continuous random variable is,
$$ \boxed{\mathbb{E}[Y] = \int_{-\infty}^{\infty} y f_Y(y) dy} $$

Intuitively, this takes every possible $y$ value and weights it according to the
probability of $Y = y$, which comes from the marginal distribution.

\problempart % Problem 1b

The definition for the expected value of a sample distribution is,
$$ \boxed{\hat{\mathbb{E}}\left[\left\{Y_i\right\}_{i = 1}^n\right] = \frac{1}{n} 
\sum_{i = 1}^n Y_i} $$

Intuitively, similar logic applies from the continuous distribution in the 
previous problem. However, instead of weighting each possible $y$ according to
its known probability, we use a discrete uniform probability distribution to 
represent the sample. 

\problempart % Problem 1c

The general definition of variance is,
$$ \mathrm{var}(X) = \mathbb{E}\left[\left(X - \mathbb{E}[X]\right)^2\right] $$
From this, it is easy to derive the definition of the variance of a continuous
random variable,
$$ \mathrm{var}(Y) = \int_{-\infty}^{\infty} \left(y - \mathbb{E}[Y]\right)^2
f_Y(y) dy $$
Or, equivalently,
$$ \boxed{\mathrm{var}(Y) = \int_{-\infty}^{\infty} y^2 f_Y(y) dy - \mathbb{E}[Y]^2} $$

Intuitively, the variance measures how much, on average, the values vary from 
the average value. Essentially, it determines how much spread there is in the
distribution.

\problempart % Problem 1d

The definition of the variance of a sample distribution is,
$$ \boxed{\hat{\mathrm{var}}(\left\{Y_i\right\}_{i = 1}^n) = \frac{1}{n - 1} \sum_{i =
1}^n \left(Y_i - \hat{\mathbb{E}}[\left\{Y_j\right\}_{j = 1}^n]\right)^2}$$

Intuitively, this is very similar to the continuous random variable version,
with the same modifications made with the sample expectation. Instead of weighting
each $y$ according to it's known distribution, we use a discrete uniform 
distribution. Furthermore, since this produces a slightly biased estimate of the
population, we subtract one from $n$ in our uniform distribution. This may be
omitted if the sample is large enough.

\problempart % Problem 1e

The general definition of covariance is,
$$ \mathrm{cov}(X, Y) = \mathbb{E}\left[\left(X - \mathbb{E}[X]\right)\left(Y - 
\mathbb{E}[Y]\right)\right] $$
From this, it is easy to derive the definition of the covariance between two
continuous random variables,
$$ \mathrm{cov}(X, Y) = \int_{-\infty}^{\infty} \int_{-\infty}^{\infty} \left(x
- \mathbb{E}[X]\right) \left(y - \mathbb{E}[Y]\right) f_{XY}(x, y) dx dy $$
Or, equivalently,
$$ \boxed{\mathrm{cov}(X, Y) = \int_{-\infty}^{\infty} \int_{-\infty}^{\infty} x y 
f_{XY}(x, y) dx dy - \mathbb{E}[X] \mathbb{E}[Y]} $$

Intuitively, this is very similar to variance. However, instead of measuring how
much a single variable varies in expectation from the expected value, we measure
how much two variables \textit{together} vary in expectation from the expected
value. It is clear to see this is the case because if the two tend to vary
together (positively or negatively) the higher the product is.

\problempart % Problem 1f

The definition of the covariance of a sample distribution is,
$$ \boxed{\hat{\mathrm{cov}}(\left\{X_i, Y_i\right\}_{i = 1}^n) = \frac{1}{n - 1} 
\sum_{i = 1}^{n} \left(X_i - \hat{\mathbb{E}}\left[\left\{X_j\right\}_{j =
1}^n\right]\right) \left(Y_i - \hat{\mathbb{E}}\left[\left\{Y_j\right\}_{j = 
1}^n\right]\right)} $$

Intuitively, this is very similar to the continuous random variable version,
with the same modifications made with the sample variance. Instead of weighting
each $y$ according to it's known distribution, we use a discrete uniform
distribution. Also, remember that we make the same adjustment to account for the
slight bias introduced in sampling. Ignore if sample size is large enough.

\problempart % Problem 1g

Using the definition reported above,
$$ \mathbb{E}[aY + bX] = \int_{-\infty}^{\infty} \int_{-\infty}^{\infty} \left(
ay + bx\right) f_{XY}(x, y) dx dy $$
Distributing the joint PDF and separating the integrals,
$$ \mathbb{E}[aY + bX] = a \int_{-\infty}^{\infty} \int_{-\infty}^{\infty} y
f_{XY}(x, y) dx dy + b \int_{-\infty}^{\infty} \int_{-\infty}^{\infty} x
f_{XY}(x, y) dx dy$$
Using the definition of a marginal distribution and how it relates to a joint
distribution,
$$ \mathbb{E}[aY + bX] = a \int_{-\infty}^{\infty} y f_{Y}(y) dy + b 
\int_{-\infty}^{\infty} x f_{X}(x) dx $$
The integrals each represent an expectation as defined above,
$$ \boxed{\mathbb{E}[aY + bX] = a \mathbb{E}[Y] + b \mathbb{E}[X]} $$
This property is also known as the Linearity of Expectation.

\problempart % Problem 1h

Using the definition of sample expectation established in Part B,
$$ \hat{\mathbb{E}}\left[\left\{a Y_i + b X_i \right\}_{i = 1}^n \right] = \frac{1}
{n} \sum_{i = 1}^n \left(a Y_i + b X_i\right) $$
By splitting apart the sum,
$$ \hat{\mathbb{E}}\left[\left\{a Y_i + b X_i \right\}_{i = 1}^n \right] =
\frac{a}{n} \sum_{i = 1}^n Y_i + \frac{b}{n} \sum_{i = 1}^n X_i $$
By using the definition presented above, this is the same as,
$$ \boxed{\hat{\mathbb{E}}\left[\left\{a Y_i + b X_i \right\}_{i = 1}^n \right] = a
\hat{\mathbb{E}}\left[\left\{Y_i\right\}_{i = 1}^n \right] + b
\hat{\mathbb{E}}\left[\left\{X_i\right\}_{i = 1}^n \right]} $$
Once again, the Linearity of Expectation holds.

\problempart % Problem 1i

From the definition for variance provided above,
$$ \mathrm{var}(a Y + b X) = \int_{-\infty}^{\infty} \int_{-\infty}^{\infty} (a y
+ b x) f_{XY}(x, y) dx dy - \mathbb{E}[aY + bX]^2 $$
Distributing and separating the integrals yields,
\begin{align*}
    \mathrm{var}(aY + bX) &= a^2 \int_{-\infty}^{\infty} \int_{-\infty}^{\infty} 
    y^2 f_{XY}(x, y) dx dy + 2ab \int_{-\infty}^{\infty}\int_{-\infty}^{\infty} yx
    f_{XY}(x, y) dx dy \\
    &+ b^2 \int_{-\infty}^{\infty}\int_{-\infty}^{\infty} x^2 f_{XY}(x, y) dx dy -
    \mathbb{E}[aY + bX]^2
\end{align*}
Then, using the definition of a marginal distribution from a joint distribution,
this can be simplified to,
\begin{align*}
    \mathrm{var}(aY + bX) &= a^2 \int_{-\infty}^{\infty} y^2 f_{Y}(y) dy + 2ab
    \int_{-\infty}^{\infty}\int_{-\infty}^{\infty} yx f_{XY}(x, y) dx dy \\
    &+ b^2 \int_{-\infty}^{\infty} x^2 f_{X}(x) dx - \mathbb{E}[aY + bX]^2
\end{align*}
We can substitute the definition for expectation derived in Part G,
\begin{align*}
    \mathrm{var}(aY + bX) &= a^2 \int_{-\infty}^{\infty} y^2 f_{Y}(y) dy + 2ab
    \int_{-\infty}^{\infty}\int_{-\infty}^{\infty} yx f_{XY}(x, y) dx dy \\
    &+ b^2 \int_{-\infty}^{\infty} x^2 f_{X}(x) dx - \left(a \mathbb{E}[Y] + b 
    \mathbb{E}[X]\right)^2
\end{align*}
Factoring out the exponent yields,
\begin{align*}
    \mathrm{var}(aY + bX) &= a^2 \int_{-\infty}^{\infty} y^2 f_{Y}(y) dy + 2ab
    \int_{-\infty}^{\infty}\int_{-\infty}^{\infty} yx f_{XY}(x, y) dx dy \\
    &+ b^2 \int_{-\infty}^{\infty} x^2 f_{X}(x) dx - a^2 \mathbb{E}[Y] - 2ab
    \mathbb{E}[X] \mathbb{E}[Y] - b^2 \mathbb{E}[X]
\end{align*}
Rearranging terms,
\begin{align*}
    \mathrm{var}(aY + bX) &= a^2 \left(\int_{-\infty}^{\infty} y^2 f_{Y}(y) dy - 
    \mathbb{E}[Y]^2\right) \\
    &+  2ab \left(\int_{-\infty}^{\infty}\int_{-\infty}^{\infty} yx f_{XY}(x, y) 
    dx dy - \mathbb{E}[X] \mathbb{E}[Y]\right) \\
    &+ b^2 \left(\int_{-\infty}^{\infty} x^2 f_{X}(x) dx - \mathbb{E}[X]^2\right)
\end{align*}
We can now substitute the definitions of variance and covariance provided above,
$$ \boxed{\mathrm{var}(aY + bX) = a^2 \mathrm{var}(Y) + b^2 \mathrm{var}(X) + 2ab
\mathrm{cov}(X, Y)} $$
If $X$ and $Y$ are independent, this simplifies to,
$$ \mathrm{var}(aY + bX) = a^2 \mathrm{var}(Y) + b^2 \mathrm{var}(X) $$

\problempart % Problem 1j

Using the definition of sample variance established in Part D,
$$ \hat{\mathrm{var}}\left(\left\{aY_i + bX_i\right\}_{i = 1}^n\right) = \frac{1}
{n - 1} \sum_{i = 1}^n \left( (aY_i + bX_i) - \hat{\mathbb{E}}\left[\left\{aY_j + 
bX_j\right\}_{j = 1}^n\right]\right)^2 $$

From Part H, we can break apart the expectation,
$$ \hat{\mathrm{var}}\left(\left\{aY_i + bX_i\right\}_{i = 1}^n\right) = \frac{1}
{n - 1} \sum_{i = 1}^n  \left(aY_i + bX_i - a \hat{\mathbb{E}}\left[\left\{Y_i
\right\}_{i = 1}^n \right] - b \hat{\mathbb{E}}\left[\left\{X_i\right\}_{i = 1}^n 
\right] \right)^2 $$
After expanding and grouping terms,
\begin{align*}
    \frac{1} {n - 1} \sum_{i = 1}^n  \Bigg( &a^2\left(Y_i^2 - 2 Y_i \hat{\mathbb{E}}
    \left[\left\{ Y_j \right\}_{j = 1}^n\right] + \hat{\mathbb{E}}\left[\left\{Y_j
    \right\}_{j = 1}^n \right]^2 \right) \\
    + &b^2 \left(X_i^2 - 2 X_i \hat{\mathbb{E}}\left[\left\{ X_j\right\}_{j = 1}^n\right] +
    \hat{\mathbb{E}}\left[\left\{X_j\right\}_{j = 1}^n  \right]^2\right) \\
    + &2ab \left( X_i Y_i - X_i \hat{\mathbb{E}}\left[\left\{Y_j\right\}_{j = 1}^n  \right] - Y_i
    \hat{\mathbb{E}}\left[\left\{X_j\right\}_{j = 1}^n \right] +
    \hat{\mathbb{E}}\left[\left\{X_j\right\}_{j = 1}^n 
    \right]\hat{\mathbb{E}}\left[\left\{Y_j\right\}_{j = 1}^n  \right] \right) \Bigg)
\end{align*}
Using we can refactor each individual grouping and split the summation,
\begin{align*}
    &\frac{a^2} {n - 1} \sum_{i = 1}^n  \left(Y_i - \hat{\mathbb{E}}
    \left[\left\{ Y_j \right\}_{j = 1}^n\right] \right)^2 
    + \frac{b^2} {n - 1} \sum_{i = 1}^n \left(X_i - \hat{\mathbb{E}}\left[\left\{ 
    X_j\right\}_{j = 1}^n\right] \right)^2 \\
    + &\frac{2ab} {n - 1} \sum_{i = 1}^n \left( Y_i - \hat{\mathbb{E}}\left[\left\{
    Y_j\right\}_{j = 1}^n  \right]\right)
    \left(X_i - \hat{\mathbb{E}}\left[\left\{X_j\right\}_{j = 1}^n \right] \right) 
\end{align*}
Using the definitions of sample variance and covariance defined above, this simplifies,
$$ \boxed{\hat{\mathrm{var}}\left(\left\{aY_i + bX_i\right\}_{i = 1}^n\right) = a^2
\hat{\mathrm{var}}(\left\{Y_i\right\}_{i = 1}^n) + b^2 \hat{\mathrm{var}}(\left\{X_i
\right\}_{i = 1}^n) + 2ab \hat{\mathrm{cov}}(\left\{X_i, Y_i\right\}_{i = 1}^n)} $$
Once again, if $X$ and $Y$ are independent,
$$ \hat{\mathrm{var}}\left(\left\{aY_i + bX_i\right\}_{i = 1}^n\right) = a^2
\hat{\mathrm{var}}(\left\{Y_i\right\}_{i = 1}^n) + b^2 \hat{\mathrm{var}}(\left\{X_i
\right\}_{i = 1}^n) $$

\problempart % Problem 1k

Using the definition of covariance established in Part E,
$$ \mathrm{cov}(X, aY + bX) = \int_{-\infty}^{\infty} \int_{-\infty}^{\infty} x (ay + bx) 
f_{XY}(x, y) dx dy - \mathbb{E}[X] \mathbb{E}[aY + bX] $$
Distributing terms and using Linearity of Expectation described in Part G,
\begin{align*}
    \mathrm{cov}(X, aY + bX) &= a \left(\int_{-\infty}^{\infty} \int_{-\infty}^{\infty} xy 
    f_{XY}(x, y) dx dy - \mathbb{E}[X]\mathbb{E}[Y]\right) \\
    &+ b \left(\int_{-\infty}^{\infty}
    \int_{-\infty}^{\infty} x^2 f_{XY}(x, y) dx dy - \mathbb{E}[X]^2 \right)
\end{align*}
Using the definitions for variance and covariance (and marginal distributions), this 
simplifies to,
$$ \boxed{\mathrm{cov}(X, aY + bX) = a \mathrm{cov}(X, Y) + b \mathrm{var}(X)} $$

\problempart % Problem 1l

Using the definition of sample covariance established in Part F,
\begin{align*}
    \hat{\mathrm{cov}}(\left\{X_i, aY_i + bX_i\right\}_{i = 1}^n) = \frac{1}{n - 1} 
    \sum_{i = 1}^{n} \Bigg( &\left(X_i - \hat{\mathbb{E}}\left[\left\{X_j\right\}_{j =
    1}^n\right]\right) \\
    &\left(\left(aY_i + b X_i\right) - \hat{\mathbb{E}}\left[\left\{aY_j + 
    bX_j\right\}_{j = 1}^n\right]\right)\Bigg)
\end{align*}
Using Linearity of Expectation derived in Part H, distributing, and regrouping yields,
\begin{align*}
    \frac{1}{n - 1} \sum_{i = 1}^{n} \Bigg( &a\left(X_i Y_i - X_i \hat{\mathbb{E}}\left[\left\{Y_j\right\}_{j = 1}^n  \right] - Y_i
    \hat{\mathbb{E}}\left[\left\{X_j\right\}_{j = 1}^n \right] +
    \hat{\mathbb{E}}\left[\left\{X_j\right\}_{j = 1}^n 
    \right]\hat{\mathbb{E}}\left[\left\{Y_j\right\}_{j = 1}^n  \right]\right) \\
    + &b\left(X_i^2 - 2 X_i \hat{\mathbb{E}}\left[\left\{ X_j\right\}_{j = 1}^n\right] +
    \hat{\mathbb{E}}\left[\left\{X_j\right\}_{j = 1}^n  \right]^2 \right)\Bigg)
\end{align*}
We can refactor each individual grouping and split the summation,
\begin{align*}
    &\frac{a} {n - 1} \sum_{i = 1}^n \left( Y_i - \hat{\mathbb{E}}\left[\left\{
    Y_j\right\}_{j = 1}^n  \right]\right)
    \left(X_i - \hat{\mathbb{E}}\left[\left\{X_j\right\}_{j = 1}^n \right] \right) \\ 
    + &\frac{b} {n - 1} \sum_{i = 1}^n \left(X_i - \hat{\mathbb{E}}\left[\left\{ 
    X_j\right\}_{j = 1}^n\right] \right)^2 \\
\end{align*}
Using the definitions of sample covariance and variance from above,
$$ \boxed{\hat{\mathrm{cov}}(\left\{X_i, aY_i + bX_i\right\}_{i = 1}^n) = a
\hat{\mathrm{cov}}( \left\{X_i, Y_i\right\}_{i = 1}^n) + b \hat{\mathrm{var}}
(\left\{X_i\right\}_{i = 1}^n)}  $$

\problempart % Problem 1m

The general definition of covariance is
$$ \mathrm{cov}(X, Y) = \mathbb{E}\left[\left(X - \mathbb{E}[X]\right)\left(Y - 
\mathbb{E}[Y]\right)\right] $$
Substituting the variables given,
$$ \mathrm{cov}(X, a) = \mathbb{E}\left[\left(X - \mathbb{E}[X]\right)\left(a - 
\mathbb{E}[a]\right)\right] $$
The expected value of a constant is that constant. Therefore,
$$ \mathrm{cov}(X, a) = \mathbb{E}\left[\left(X - \mathbb{E}[X]\right)\left(a - 
a\right)\right] $$
Clearly, this becomes zero.
$$ \boxed{\mathrm{cov}(X, a) = 0} $$

\end{problemparts}

\newpage

\problem  % Problem 2

The general equation for the inverse of a $2 \times 2$ matrix $D$ is,
$$ \boxed{\begin{pmatrix}
    a & b \\
    c & d
\end{pmatrix}^{-1} = \frac{1}{ad - bc} \begin{pmatrix}
    d & -b \\
    -c & a
\end{pmatrix}} $$
This can be verified using the Gauss-Jordan method,
$$ \begin{pmatrix}
    a & b & 1 & 0 \\
    c & d & 0 & 1
\end{pmatrix} \Longrightarrow \begin{pmatrix}
    1 & 0 & \frac{d}{ad-bc} & \frac{b}{bc-ad} \\
    0 & 1 & \frac{c}{bc-ad} & \frac{a}{ad-bc}
\end{pmatrix} $$

\problem  % Problem 3

If $X^TX$ were invertible, then the expression would clearly simplify to the
identity matrix $I$. The identity matrix is clearly both symmetric and 
idempotent. However, $X^T X$ is not guaranteed to be invertible. Thus we examine
each property individually.

First we take check the condition for symmetry,
$$ A^T = A $$
We take the transpose,
$$ \left(X \left(X^T X\right)^{-1} X^T\right)^T = X \left(\left(X^T
X\right)^{-1}\right)^T X^T $$
However, the transpose of an inverse is the same as the inverse of a transpose,
therefore this expression becomes,
$$ \boxed{X \left(X^T X\right)^{-1} X^T} $$
Thus, the matrix $P_X$ is symmetric.

Now we consider the condition for idempotent,
$$ A^2 = A $$
We take the power,
$$ \left(X \left(X^T X\right)^{-1} X^T\right) \left(X \left(X^T X\right)^{-1} 
X^T\right) $$
Rewriting with different parentheses,
$$ X \left(X^T X\right)^{-1} \left(X^T X\right) \left(X^T X\right)^{-1} 
X^T $$
We can cancel two of the inner terms to get,
$$ \boxed{X \left(X^T X\right)^{-1} X^T} $$
Thus, the matrix $P_X$ is idempotent.

\problem  % Problem 4

First we consider the condition for symmetry,
$$ A^T = A $$
We take the transpose,
$$ \left(I - P_X\right)^T $$
The transpose distributes, we are left with,
$$ I^T - P_X^T $$
But, both $I$ and $P$ are symmetric, thus this becomes,
$$ \boxed{I - P} $$
Therefore, $M_X$ is symmetric.

Now we consider idempotency,
$$ A^2 = A $$
Taking the product,
$$ (I - P_X)(I - P_X) $$
Distributing the terms,
$$ I^2 - 2P_X + P_X^2 $$
Since $I$ and $P_X$ are idempotent,
$$ I - 2P_X + P_X = \boxed{I - P_X} $$
Therefore, $M_X$ is idempotent.

\newpage

\problem  % Problem 5

The aggressive assumption
$$ \mathbb{E}[\varepsilon_i \mid X_i = x] = 0\quad \forall i $$
is essentially assuming that the dependence of $Y$ on variables other than
$X$ are not related to $X$. That is, given $X$, the mean of the distribution
of other factors is zero (assigned randomly). In abstract, this might not 
hold if it is reasonable to expect a higher $X_i$ will lead to a higher 
$\varepsilon_i$. More concretely, it is likely that an individual with more
education ($X_i$) has other factors ($\varepsilon_i$)---such as parental wealth, 
charisma, innate intelligence, etc.---that will contribute to higher earnings ($Y_i$).

\problem  % Problem 6

The definition of covariance is,
$$ \mathrm{cov}(X, Y) = \mathbb{E}[XY] - \mathbb{E}[X] \mathbb{E}[Y] $$
Substituting provided values,
$$ \mathrm{cov}(X_i, \varepsilon_i) = \mathbb{E}[X_i \varepsilon_i] - \mathbb{E}[X_i] 
\mathbb{E}[\varepsilon_i] $$
The Law of Iterated Expectation states,
$$ \mathbb{E}[X] = \mathbb{E}\left[\mathbb{E}[X \mid Z]\right] $$
Thus, we can condition on $X_i$,
$$ \mathrm{cov}(X_i, \varepsilon_i) = \mathbb{E}\left[\mathbb{E}[X_i \varepsilon_i 
\mid X_i]\right] - \mathbb{E}[X_i] \mathbb{E}\left[\mathbb{E}[\varepsilon_i \mid 
X_i]\right] $$
Given that $\mathbb{E}[\varepsilon_i \mid X_i] = 0$ and $\mathbb{E}[0] = 0$, this 
simplifies,
$$ \mathrm{cov}(X_i, \varepsilon_i) = \mathbb{E}\left[\mathbb{E}[X_i \varepsilon_i 
\mid X_i]\right] $$
Within the inner expectation, $X_i$ becomes a constant, so it can be pulled out,
$$ \mathrm{cov}(X_i, \varepsilon_i) = \mathbb{E}\left[X_i \mathbb{E}[\varepsilon_i 
\mid X_i]\right] $$
Once again, $\mathbb{E}[\varepsilon \mid X_i] = 0$ and $\mathbb{E}[0] = 0$, therefore,
$$ \boxed{\mathrm{cov}(X_i, \varepsilon_i) = 0} $$

\problem  % Problem 7

The definition of covariance is,
$$ \mathrm{cov}(X, Y) = \mathbb{E}[XY] - \mathbb{E}[X] \mathbb{E}[Y] $$
Substituting provided values,
$$ \mathrm{cov}(X_i, \varepsilon_i) = \mathbb{E}[X_i \varepsilon] - \mathbb{E}[X_i]
\mathbb{E}[\varepsilon_i] $$
Using the model, solving for $\varepsilon_i$,
$$ \varepsilon_i = Y_i - \beta_0 - \beta_1 X_i $$
Substituting this expression into the covariance expression and setting to zero,
$$ 0 = \mathbb{E}[X_i \left(Y_i - \beta_0 - \beta_1 X_i \right)] - \mathbb{E}[X_i]
\mathbb{E}[Y_i - \beta_0 - \beta_1 X_i] $$
Using Linearity of Expectation and distribution,
$$ 0 = \mathbb{E}[X_i Y_i] - \beta_0 \mathbb{E}[X_i] - \beta_1 \mathbb{E}[X_i^2] - 
\mathbb{E}[X_i] \mathbb{E}[Y_i] + \beta_0 \mathbb{E}[X_i] + \beta_1 \mathbb{E}[X_i]^2 $$
Combining like terms and using the definitions of sample covariance and variance,
$$ 0 = \hat{\mathrm{cov}}(X_i, Y_i) - \beta_1 \left(\hat{\mathrm{var}}(X_i)\right) $$
Solving for $\beta_1$,
$$ \boxed{\beta_1 = \frac{\hat{\mathrm{cov}}(X_i, Y_i)}{\hat{\mathrm{var}}(X_i)}} $$

\problem  % Problem 8

The solution $\beta_1$ in terms of expectations,
$$ \beta_1 = \frac{\mathbb{E}[X_i Y_i] - \mathbb{E}[X_i] \mathbb{E}[Y_i]}
{\mathbb{E}[X_i^2] - \mathbb{E}[X_i]^2} $$
Let $X_i$ be a Bernoulli random variable which is $1$ with probability $p$. First,
we derive $\mathbb{E}[X_i Y_i]$ using the Law of Total Expectation,
$$ \mathbb{E}[X_i Y_i] = (1 - p) \mathbb{E}[X_i Y_i \mid X_i = 0] + p \mathbb{E}[X_i Y_i
\mid X_i = 1] = p \bar{Y}_1 $$
By definition of a Bernoulli random variable, $\mathbb{E}[X_i]$ can be determined,
$$ \mathbb{E}[X_i] = p $$
We can also determine $\mathbb{E}[Y_i]$ using the Law of Total Expectation,
$$ \mathbb{E}[Y_i] = (1 - p) \mathbb{E}[Y_i \mid X_i = 0] + p \mathbb{E}[Y_i
\mid X_i = 1] = (1 - p) \bar{Y}_0 + p \bar{Y}_1 $$
Lastly, we need the expression for $\mathbb{E}[X_i^2]$. Using Total Expectation,
$$ \mathbb{E}[X_i^2] = (1 - p) \mathbb{E}[X_i^2 \mid X = 0] + p \mathbb{E}[X_i^2 \mid 
X_i = 1] = p $$
Putting this all together,
$$ \beta_1 = \frac{p \bar{Y}_1 - p (1 - p) \bar{Y}_0 - p^2 \bar{Y}_1}{p - p^2} $$
We can cancel a $p$ from every term,
$$ \beta_1 = \frac{\bar{Y}_1 - (1 - p) \bar{Y}_0 - p \bar{Y}_1}{1 - p} $$
Rearranging shows that we can also cancel a $(1 - p)$ from every term,
$$ \beta_1 = \frac{(1 - p) \bar{Y}_1 - (1 - p)}{1 - p} = \boxed{\bar{Y}_1 - \bar{Y}_0} $$

\newpage

\problem  % Problem 9

\begin{problemparts}

\problempart  % Problem 9a

The $\beta_1$ coefficient from the regression of log weekly wages with years of education
is,
$$ \boxed{\beta_1 = 0.0709} $$
Intuitively, this says that one more year of education results in (approximately) $7$\%
higher weekly wage.

\problempart  % Problem 9b

The $R^2$ of the regression of log weekly wages with years of education is,
$$ \boxed{R^2 = 0.1173} $$
Intuitively, this number tells us how much variation in $Y$ is explained by our regression.

\problempart  % Problem 9c

The mean squared error of the regression is,
$$ \boxed{\mathrm{MSE} = 0.6377^2 = 0.4067} $$

\end{problemparts}

\end{problems}

\end{document}


