%
% 6.S077 problem set solutions template
%
\documentclass[12pt,twoside]{article}

\newcommand{\name}{}

\usepackage{amssymb}
\usepackage{amsmath}
\usepackage{graphicx}
\usepackage{latexsym}
\usepackage{times,url}
\usepackage{cprotect}
\usepackage{listings}
\usepackage{graphicx}
\usepackage[table]{xcolor}
\usepackage[letterpaper]{geometry}
\usepackage{tikz-qtree}
\usepackage{enumerate}

\newcommand{\profs}{Mauricio Karchmer, Aleksander Madry, Bruce Tidor}
\newcommand{\subj}{6.046}
\newcommand{\ttt}[1]{{\tt\small #1}}

\definecolor{dkgreen}{rgb}{0,0.6,0}
\definecolor{gray}{rgb}{0.5,0.5,0.5}
\definecolor{mauve}{rgb}{0.58,0,0.82}

\lstset{
  language=Python,
  aboveskip=1pc,
  belowskip=1pc,
  basicstyle={\footnotesize\ttfamily},
  numbers=left,
  showstringspaces=false,
  numberstyle=\tiny\color{gray},
  keywordstyle=\color{blue},
  commentstyle=\color{dkgreen},
  stringstyle=\color{mauve},
}

\tikzset{
  % every node/.style={minimum width=2em,draw,circle},
  % level 1/.style={sibling distance=2cm},
  level distance=1cm,
  edge from parent/.style=
  {draw,edge from parent path={(\tikzparentnode) -- (\tikzchildnode)}},
}

\newif\ifHideSolutions
\newcommand{\solution}[1]{\color{dkgreen}\textbf{Solution: }#1\color{black}}
\newcommand{\rubric}[1]{\color{dkgreen}{\bf Rubric:} #1\color{black}}

% \HideSolutionsfalse
% \ifHideSolutions
%   \renewcommand{\solution}[1]{}
%   \renewcommand{\rubric}[1]{}
% \fi

\newlength{\toppush}
\setlength{\toppush}{2\headheight}
\addtolength{\toppush}{\headsep}

\newcommand{\htitle}[2]{\noindent\vspace*{-\toppush}\newline\parbox{6.5in}
{\textit{Design and Analysis of Algorithms}\hfill\name\newline
Massachusetts Institute of Technology \hfill #2\newline
\profs\hfill #1 \vspace*{-.5ex}\newline
\mbox{}\hrulefill\mbox{}}\vspace*{1ex}\mbox{}\newline
\begin{center}{\Large\bf #1}\end{center}}

\newcommand{\handout}[2]{\thispagestyle{empty}
 \markboth{#1}{#1}
 \pagestyle{myheadings}\htitle{#1}{#2}}

\newcommand{\lecture}[3]{\thispagestyle{empty}
 \markboth{Lecture #1: #2}{Lecture #1: #2}
 \pagestyle{myheadings}\htitle{Lecture #1: #2}{#3}}

\newcommand{\htitlewithouttitle}[2]{\noindent\vspace*{-\toppush}\newline\parbox{6.5in}
{\textit{Design and Analysis of Algorithms}\hfill#2\newline
Massachusetts Institute of Technology \hfill 6.046\newline
\profs\hfill Handout #1\vspace*{-.5ex}\newline
\mbox{}\hrulefill\mbox{}}\vspace*{1ex}\mbox{}\newline}

\newcommand{\handoutwithouttitle}[2]{\thispagestyle{empty}
 \markboth{Handout \protect\ref{#1}}{Handout \protect\ref{#1}}
 \pagestyle{myheadings}\htitlewithouttitle{\protect\ref{#1}}{#2}}

\newcommand{\exam}[2]{% parameters: exam name, date
 \thispagestyle{empty}
 \markboth{\hspace{1cm}\subj\ #1\hspace{1in}Name\hrulefill\ \ }%
          {\subj\ #1\hspace{1in}Name\hrulefill\ \ }
 \pagestyle{myheadings}\examtitle{#1}{#2}
 \renewcommand{\theproblem}{Problem \arabic{problemnum}}
}
\newcommand{\examsolutions}[3]{% parameters: handout, exam name, date
 \thispagestyle{empty}
 \markboth{Handout \protect\ref{#1}: #2}{Handout \protect\ref{#1}: #2}
% \pagestyle{myheadings}\htitle{\protect\ref{#1}}{#2}{#3}
 \pagestyle{myheadings}\examsolutionstitle{\protect\ref{#1}} {#2}{#3}
 \renewcommand{\theproblem}{Problem \arabic{problemnum}}
}
\newcommand{\examsolutionstitle}[3]{\noindent\vspace*{-\toppush}\newline\parbox{6.5in}
{\textit{Design and Analysis of Algorithms}\hfill#3\newline
Massachusetts Institute of Technology \hfill 6.046\newline
%Singapore-MIT Alliance \hfill SMA5503\newline
\profs\hfill Handout #1\vspace*{-.5ex}\newline
\mbox{}\hrulefill\mbox{}}\vspace*{1ex}\mbox{}\newline
\begin{center}{\Large\bf #2}\end{center}}

\newcommand{\takehomeexam}[2]{% parameters: exam name, date
 \thispagestyle{empty}
 \markboth{\subj\ #1\hfill}{\subj\ #1\hfill}
 \pagestyle{myheadings}\examtitle{#1}{#2}
 \renewcommand{\theproblem}{Problem \arabic{problemnum}}
}

\makeatletter
\newcommand{\exambooklet}[2]{% parameters: exam name, date
 \thispagestyle{empty}
 \markboth{\subj\ #1}{\subj\ #1}
 \pagestyle{myheadings}\examtitle{#1}{#2}
 \renewcommand{\theproblem}{Problem \arabic{problemnum}}
 \renewcommand{\problem}{\newpage
 \item \let\@currentlabel=\theproblem
 \markboth{\subj\ #1, \theproblem}{\subj\ #1, \theproblem}}
}
\makeatother


\newcommand{\examtitle}[2]{\noindent\vspace*{-\toppush}\newline\parbox{6.5in}
{\textit{Design and Analysis of Algorithms}\hfill#2\newline
Massachusetts Institute of Technology \hfill 6.046 Spring 2019\newline
%Singapore-MIT Alliance \hfill SMA5503\newline
\profs\hfill #1\vspace*{-.5ex}\newline
\mbox{}\hrulefill\mbox{}}\vspace*{1ex}\mbox{}\newline
\begin{center}{\Large\bf #1}\end{center}}

\newcommand{\grader}[1]{\hspace{1cm}\textsf{\textbf{#1}}\hspace{1cm}}

\newcommand{\points}[1]{[#1 points]\ }
\newcommand{\parts}[1]
{
  \ifnum#1=1
  (1 part)
  \else
  (#1 parts)
  \fi
  \ 
}

\newcommand{\bparts}{\begin{problemparts}}
\newcommand{\eparts}{\end{problemparts}}
\newcommand{\ppart}{\problempart}

%\newcommand{\lg} {lg\ }

\setlength{\oddsidemargin}{0pt}
\setlength{\evensidemargin}{0pt}
\setlength{\textwidth}{6.5in}
\setlength{\topmargin}{0in}
\setlength{\textheight}{8.5in}


\newcommand{\Spawn}{{\bf spawn} }
\newcommand{\Sync}{{\bf sync}}

\newcommand{\cif}[1]{\mbox{if $#1$}}
\newcommand{\cwhen}[1]{\mbox{when $#1$}}

\newcounter{problemnum}
\newcommand{\theproblem}{Problem \theproblemsetnum-\arabic{problemnum}}
\newenvironment{problems}{
        \begin{list}{{\bf \theproblem. \hspace*{0.5em}}}
        {\setlength{\leftmargin}{0em}
         \setlength{\rightmargin}{0em}
         \setlength{\labelwidth}{0em}
         \setlength{\labelsep}{0em}
         \usecounter{problemnum}}}{\end{list}}
\makeatletter
\newcommand{\problem}[1][{}]{\item \let\@currentlabel=\theproblem \textbf{#1}}
\makeatother

\newcounter{problempartnum}[problemnum]
\newenvironment{problemparts}{
        \begin{list}{{\bf (\alph{problempartnum})}}
        {\setlength{\leftmargin}{2.5em}
         \setlength{\rightmargin}{2.5em}
         \setlength{\labelsep}{0.5em}}}{\end{list}}
\newcommand{\problempart}{\addtocounter{problempartnum}{1}\item}

\newenvironment{truefalseproblemparts}{
        \begin{list}{{\bf (\alph{problempartnum})\ \ \ T\ \ F\hfil}}
        {\setlength{\leftmargin}{4.5em}
         \setlength{\rightmargin}{2.5em}
         \setlength{\labelsep}{0.5em}
         \setlength{\labelwidth}{4.5em}}}{\end{list}}

\newcounter{exercisenum}
\newcommand{\theexercise}{Exercise \theproblemsetnum-\arabic{exercisenum}}
\newenvironment{exercises}{
        \begin{list}{{\bf \theexercise. \hspace*{0.5em}}}
        {\setlength{\leftmargin}{0em}
         \setlength{\rightmargin}{0em}
         \setlength{\labelwidth}{0em}
         \setlength{\labelsep}{0em}
        \usecounter{exercisenum}}}{\end{list}}
\makeatletter
\newcommand{\exercise}{\item \let\@currentlabel=\theexercise}
\makeatother

\newcounter{exercisepartnum}[exercisenum]
%\newcommand{\problem}[1]{\medskip\mbox{}\newline\noindent{\bf Problem #1.}\hspace*{1em}}
%\newcommand{\exercise}[1]{\medskip\mbox{}\newline\noindent{\bf Exercise #1.}\hspace*{1em}}

\newenvironment{exerciseparts}{
        \begin{list}{{\bf (\alph{exercisepartnum})}}
        {\setlength{\leftmargin}{2.5em}
         \setlength{\rightmargin}{2.5em}
         \setlength{\labelsep}{0.5em}}}{\end{list}}
\newcommand{\exercisepart}{\addtocounter{exercisepartnum}{1}\item}


% Macros to make captions print with small type and 'Figure xx' in bold.
\makeatletter
\def\fnum@figure{{\bf Figure \thefigure}}
\def\fnum@table{{\bf Table \thetable}}
\let\@mycaption\caption
%\long\def\@mycaption#1[#2]#3{\addcontentsline{\csname
%  ext@#1\endcsname}{#1}{\protect\numberline{\csname 
%  the#1\endcsname}{\ignorespaces #2}}\par
%  \begingroup
%    \@parboxrestore
%    \small
%    \@makecaption{\csname fnum@#1\endcsname}{\ignorespaces #3}\par
%  \endgroup}
%\def\mycaption{\refstepcounter\@captype \@dblarg{\@mycaption\@captype}}
%\makeatother
\let\mycaption\caption
%\newcommand{\figcaption}[1]{\mycaption[]{#1}}

\newcounter{totalcaptions}
\newcounter{totalart}

\newcommand{\figcaption}[1]{\addtocounter{totalcaptions}{1}\caption[]{#1}}

% \psfigures determines what to do for figures:
%       0 means just leave vertical space
%       1 means put a vertical rule and the figure name
%       2 means insert the PostScript version of the figure
%       3 means put the figure name flush left or right
\newcommand{\psfigures}{0}
\newcommand{\spacefigures}{\renewcommand{\psfigures}{0}}
\newcommand{\rulefigures}{\renewcommand{\psfigures}{1}}
\newcommand{\macfigures}{\renewcommand{\psfigures}{2}}
\newcommand{\namefigures}{\renewcommand{\psfigures}{3}}

\newcommand{\figpart}[1]{{\bf (#1)}\nolinebreak[2]\relax}
\newcommand{\figparts}[2]{{\bf (#1)--(#2)}\nolinebreak[2]\relax}


\macfigures     % STATE

% When calling \figspace, make sure to leave a blank line afterward!!
% \widefigspace is for figures that are more than 28pc wide.
\newlength{\halffigspace} \newlength{\wholefigspace}
\newlength{\figruleheight} \newlength{\figgap}
\newcommand{\setfiglengths}{\ifnum\psfigures=1\setlength{\figruleheight}{\hruleheight}\setlength{\figgap}{1em}\else\setlength{\figruleheight}{0pt}\setlength{\figgap}{0em}\fi}
\newcommand{\figspace}[2]{\ifnum\psfigures=0\leavefigspace{#1}\else%
\setfiglengths%
\setlength{\wholefigspace}{#1}\setlength{\halffigspace}{.5\wholefigspace}%
\rule[-\halffigspace]{\figruleheight}{\wholefigspace}\hspace{\figgap}#2\fi}
\newlength{\widefigspacewidth}
% Make \widefigspace put the figure flush right on the text page.
\newcommand{\widefigspace}[2]{
\ifnum\psfigures=0\leavefigspace{#1}\else%
\setfiglengths%
\setlength{\widefigspacewidth}{28pc}%
\addtolength{\widefigspacewidth}{-\figruleheight}%
\setlength{\wholefigspace}{#1}\setlength{\halffigspace}{.5\wholefigspace}%
\makebox[\widefigspacewidth][r]{#2\hspace{\figgap}}\rule[-\halffigspace]{\figruleheight}{\wholefigspace}\fi}
\newcommand{\leavefigspace}[1]{\setlength{\wholefigspace}{#1}\setlength{\halffigspace}{.5\wholefigspace}\rule[-\halffigspace]{0em}{\wholefigspace}}

% Commands for including figures with macpsfig.
% To use these commands, documentstyle ``macpsfig'' must be specified.
\newlength{\macfigfill}
\makeatother
\newlength{\bbx}
\newlength{\bby}
\newcommand{\macfigure}[5]{\addtocounter{totalart}{1}
\ifnum\psfigures=2%
\setlength{\bbx}{#2}\addtolength{\bbx}{#4}%
\setlength{\bby}{#3}\addtolength{\bby}{#5}%
\begin{flushleft}
\ifdim#4>28pc\setlength{\macfigfill}{#4}\addtolength{\macfigfill}{-28pc}\hspace*{-\macfigfill}\fi%
\mbox{\psfig{figure=./#1.ps,%
bbllx=#2,bblly=#3,bburx=\bbx,bbury=\bby}}
\end{flushleft}%
\else\ifdim#4>28pc\widefigspace{#5}{#1}\else\figspace{#5}{#1}\fi\fi}
\makeatletter

\newlength{\savearraycolsep}
\newcommand{\narrowarray}[1]{\setlength{\savearraycolsep}{\arraycolsep}\setlength{\arraycolsep}{#1\arraycolsep}}
\newcommand{\normalarray}{\setlength{\arraycolsep}{\savearraycolsep}}

\newcommand{\hint}{{\em Hint:\ }}

% Macros from /th/u/clr/mac.tex

\newcommand{\set}[1]{\left\{ #1 \right\}}
\newcommand{\abs}[1]{\left| #1\right|}
\newcommand{\card}[1]{\left| #1\right|}
\newcommand{\floor}[1]{\left\lfloor #1 \right\rfloor}
\newcommand{\ceil}[1]{\left\lceil #1 \right\rceil}
\newcommand{\ang}[1]{\ifmmode{\left\langle #1 \right\rangle}
   \else{$\left\langle${#1}$\right\rangle$}\fi}
        % the \if allows use outside mathmode,
        % but will swallow following space there!
\newcommand{\paren}[1]{\left( #1 \right)}
\newcommand{\bracket}[1]{\left[ #1 \right]}
\newcommand{\prob}[1]{\Pr\left\{ #1 \right\}}
\newcommand{\Var}{\mathop{\rm Var}\nolimits}
\newcommand{\expect}[1]{{\rm E}\left[ #1 \right]}
\newcommand{\expectsq}[1]{{\rm E}^2\left[ #1 \right]}
\newcommand{\variance}[1]{{\rm Var}\left[ #1 \right]}
\renewcommand{\choose}[2]{{{#1}\atopwithdelims(){#2}}}
\def\pmod#1{\allowbreak\mkern12mu({\rm mod}\,\,#1)}
\newcommand{\matx}[2]{\left(\begin{array}{*{#1}{c}}#2\end{array}\right)}
\newcommand{\Adj}{\mathop{\rm Adj}\nolimits}

\newtheorem{theorem}{Theorem}
\newtheorem{lemma}[theorem]{Lemma}
\newtheorem{corollary}[theorem]{Corollary}
\newtheorem{xample}{Example}
\newtheorem{definition}{Definition}
\newenvironment{example}{\begin{xample}\rm}{\end{xample}}
\newcommand{\proof}{\noindent{\em Proof.}\hspace{1em}}
\def\squarebox#1{\hbox to #1{\hfill\vbox to #1{\vfill}}}
\newcommand{\qedbox}{\vbox{\hrule\hbox{\vrule\squarebox{.667em}\vrule}\hrule}}
\newcommand{\qed}{\nopagebreak\mbox{}\hfill\qedbox\smallskip}
\newcommand{\eqnref}[1]{(\protect\ref{#1})}

%%\newcommand{\twodots}{\mathinner{\ldotp\ldotp}}
\newcommand{\transpose}{^{\mbox{\scriptsize \sf T}}}
\newcommand{\amortized}[1]{\widehat{#1}}

\newcommand{\punt}[1]{}

%%% command for putting definitions into boldface
% New style for defined terms, as of 2/23/88, redefined by THC.
\newcommand{\defn}[1]{{\boldmath\textit{\textbf{#1}}}}
\newcommand{\defi}[1]{{\textit{\textbf{#1\/}}}}

\newcommand{\red}{\leq_{\rm P}}
\newcommand{\lang}[1]{%
\ifmmode\mathord{\mathcode`-="702D\rm#1\mathcode`\-="2200}\else{\rm#1}\fi}

%\newcommand{\ckt}[1]{\ifmmode\mathord{\mathcode`-="702D\sc #1\mathcode`\-="2200}\else$\mathord{\mathcode`-="702D\sc #1\mathcode`\-="2200}$\fi}
\newcommand{\ckt}[1]{\ifmmode \sc #1\else$\sc #1$\fi}

%% Margin notes - use \notesfalse to turn off notes.
\setlength{\marginparwidth}{0.6in}
\reversemarginpar
\newif\ifnotes
\notestrue
\newcommand{\longnote}[1]{
  \ifnotes
    {\medskip\noindent Note: \marginpar[\hfill$\Longrightarrow$]
      {$\Longleftarrow$}{#1}\medskip}
  \fi}
\newcommand{\note}[1]{
  \ifnotes
    {\marginpar{\tiny \raggedright{#1}}}
  \fi}


\newcommand{\reals}{\mathbbm{R}}
\newcommand{\integers}{\mathbbm{Z}}
\newcommand{\naturals}{\mathbbm{N}}
\newcommand{\rationals}{\mathbbm{Q}}
\newcommand{\complex}{\mathbbm{C}}

\newcommand{\oldreals}{{\bf R}}
\newcommand{\oldintegers}{{\bf Z}}
\newcommand{\oldnaturals}{{\bf N}}
\newcommand{\oldrationals}{{\bf Q}}
\newcommand{\oldcomplex}{{\bf C}}

\newcommand{\w}{\omega}                 %% for fft chapter

\newenvironment{closeitemize}{\begin{list}
{$\bullet$}
{\setlength{\itemsep}{-0.2\baselineskip}
\setlength{\topsep}{0.2\baselineskip}
\setlength{\parskip}{0pt}}}
{\end{list}}

% These are necessary within a {problems} environment in order to restore
% the default separation between bullets and items.
\newenvironment{normalitemize}{\setlength{\labelsep}{0.5em}\begin{itemize}}
                              {\end{itemize}}
\newenvironment{normalenumerate}{\setlength{\labelsep}{0.5em}\begin{enumerate}}
                                {\end{enumerate}}

%\def\eqref#1{Equation~(\ref{eq:#1})}
%\newcommand{\eqref}[1]{Equation (\ref{eq:#1})}
\newcommand{\eqreftwo}[2]{Equations (\ref{eq:#1}) and~(\ref{eq:#2})}
\newcommand{\ineqref}[1]{Inequality~(\ref{ineq:#1})}
\newcommand{\ineqreftwo}[2]{Inequalities (\ref{ineq:#1}) and~(\ref{ineq:#2})}

\newcommand{\figref}[1]{Figure~\ref{fig:#1}}
\newcommand{\figreftwo}[2]{Figures \ref{fig:#1} and~\ref{fig:#2}}

\newcommand{\liref}[1]{line~\ref{li:#1}}
\newcommand{\Liref}[1]{Line~\ref{li:#1}}
\newcommand{\lirefs}[2]{lines \ref{li:#1}--\ref{li:#2}}
\newcommand{\Lirefs}[2]{Lines \ref{li:#1}--\ref{li:#2}}
\newcommand{\lireftwo}[2]{lines \ref{li:#1} and~\ref{li:#2}}
\newcommand{\lirefthree}[3]{lines \ref{li:#1}, \ref{li:#2}, and~\ref{li:#3}}

\newcommand{\lemlabel}[1]{\label{lem:#1}}
\newcommand{\lemref}[1]{Lemma~\ref{lem:#1}} 

\newcommand{\exref}[1]{Exercise~\ref{ex:#1}}

\newcommand{\handref}[1]{Handout~\ref{#1}}

\newcommand{\defref}[1]{Definition~\ref{def:#1}}

% (1997.8.16: Victor Luchangco)
% Modified \hlabel to only get date and to use handouts counter for number.
%   New \handout and \handoutwithouttitle commands in newmac.tex use this.
%   The date is referenced by <label>-date.
%   (Retained old definition as \hlabelold.)
%   Defined \hforcelabel to use an argument instead of the handouts counter.

\newcounter{handouts}
\setcounter{handouts}{0}

\newcommand{\hlabel}[2]{%
\stepcounter{handouts}
{\edef\next{\write\@auxout{\string\newlabel{#1}{{\arabic{handouts}}{0}}}}\next}
\write\@auxout{\string\newlabel{#1-date}{{#2}{0}}}
}

\newcommand{\hforcelabel}[3]{%          Does not step handouts counter.
\write\@auxout{\string\newlabel{#1}{{#2}{0}}}
\write\@auxout{\string\newlabel{#1-date}{{#3}{0}}}}


% less ugly underscore
% --juang, 2008 oct 05
\renewcommand{\_}{\vrule height 0 pt depth 0.4 pt width 0.5 em \,}

\usepackage{enumitem}
\newcommand{\theproblemsetnum}{4}
\newcommand{\releasedate}{Monday, April 5}
\newcommand{\partaduedate}{Friday, April 12}

\title{14.32 Problem Set \theproblemsetnum}

\begin{document}

\handout{Problem Set \theproblemsetnum}{\releasedate}
\textbf{All parts are due {\bf \partaduedate} at {\bf 9:00AM}}.

\setlength{\parindent}{0pt}
\medskip\hrulefill\medskip

{\bf Name:} Robert Durfee

\medskip

{\bf Collaborators:} None

\medskip\hrulefill

\begin{problems}

\problem  % Problem 1

\begin{problemparts}

\problempart  % Problem 1a

We assume the population is governed by the following model,
$$ y_i^* = \beta_0 + \beta_1 x_i^* + \varepsilon_i $$
However, we have measurement error on $y_i$. Therefore, the model we have
access to is,
$$ y_i = \beta_0 + \beta_1 x_i^* + \varepsilon_i + \zeta_i $$
To show that the estimate $\hat{\beta}_1$ is still consistent, we take the
probability limit,
\begin{align*}
    \plim \hat{\beta}_1 &= \plim \left(\frac{\hat{\cov}(x_i^*,
    y_i)}{\hat{\var}(x_i^*)}\right) \\
    &= \plim \left(\frac{\hat{\cov} (x_i^*, \beta_0 + \beta_1 x_i^* +
    \varepsilon_i + \zeta_i)}{\hat{\var}(x_i^*)}\right) \\
    &= \plim \left(\frac{\beta_1 \hat{\var} (x_i^*) + \hat{\cov}(x_i^*,
    \varepsilon_i) + \hat{\cov}(x_i^*, \zeta_i)}{\hat{\var} (x_i^*)}\right) \\
    &= \plim \left(\beta_1 + \frac{\hat{\cov}(x_i^*, \varepsilon_i) +
    \hat{\cov}(x_i^*, \zeta_i)}{\hat{\var} (x_i^*)}\right) \\
    &= \beta_1 + \frac{\cov(x_i^*, \varepsilon_i) + \cov(x_i^*,
    \zeta_i)}{\var (x_i^*)} \\
\end{align*}
If we assume $\cov(x_i^*, \varepsilon_i) = 0$ and $\cov(x_i^*, \zeta_i) = 0$,
as stated in the problem, this reduces to,
$$ \plim \hat{\beta}_1 = \beta_1 $$
As a result, the estimate is consistent.

\problempart  % Problem 1b

Instead of considering consistency, we can also consider the variance of the
estimate.
\begin{align*}
    \var(\hat{\beta}_1) &= \var\left(\frac{\hat{\cov}(x_i^*,
    y_i)}{\hat{\var}(x_i^*)}\right) \\
    &= \var\left(\frac{\sum_i (x_i^* - \bar{x}^*) (y_i - \bar{y})}{\sum_i
    (x_i^* - \bar{x}^*)^2}\right) \\
    &= \var\left(\frac{\sum_i (x_i^* - \bar{x}^*) y_i}{\sum_i (x_i^* -
    \bar{x}^*)^2}\right) \\
    &= \var\left(\frac{\sum_i (x_i^* - \bar{x}^*) (\beta_0 + \beta_1 x_i^* +
    \varepsilon_i + \zeta_i)}{\sum_i (x_i^* - \bar{x}^*)^2}\right)
\end{align*}
Because only $\varepsilon_i$ and $\zeta_i$ are random, this reduces to,
\begin{align*}
    \var(\hat{\beta}_1) &= \var\left(\frac{\sum_i (x_i^* - \bar{x}^*)
    (\varepsilon_i + \zeta_i)}{\sum_i (x_i^* - \bar{x}^*)^2}\right) \\
    &= \frac{\sum_i (x_i^* - \bar{x}^*)(\var (\varepsilon_i) + \var
    (\zeta_i))}{\sum_i (x_i^* - \bar{x}^*)^2}
\end{align*}
Assuming homoskedasticity,
\begin{align*}
    \var(\hat{\beta}_1) &= \frac{\sum_i (x_i^* - \bar{x}^*)}{\sum_i (x_i^* -
    \bar{x}^*)^2}(\var (\varepsilon) + \var (\zeta))
\end{align*}
Comparing this to the case with no measurement error (i.e. no $\zeta_i$),
\begin{align*}
    \var(\hat{\beta}_1) &= \frac{\sum_i (x_i^* - \bar{x}^*)}{\sum_i (x_i^* -
    \bar{x}^*)^2}\var (\varepsilon)
\end{align*}
Therefore, so long as $\var(\zeta) \neq 0$, the variance of the estimator
$\hat{\beta}_1$ with measurement will be greater than the variance without
measurement error.

\end{problemparts}

\newpage

\problem  % Problem 2

\begin{problemparts}

\problempart % Problem 2a

We assume the population is governed by the following model,
$$ y_i^* = \beta_0 + \beta_1 x_i^* + \varepsilon_i $$
However, we have measurement error on $x_i = x_i^* + \nu_i$. To show that the
estimate $\hat{\beta}_1$ is now inconsistent, we take the probability limit,
\begin{align*}
    \plim \hat{\beta}_1 &= \plim \left(\frac{\hat{\cov}(x_i,
    y_i^*)}{\hat{\var}(x_i)}\right) \\
    &= \plim \left(\frac{\hat{\cov}(x_i^* + \nu_i, \beta_0 + \beta_1 x_i^* +
    \varepsilon_i)}{\hat{\var}(x_i^* + \nu_i)}\right) \\
    &= \plim \left(\frac{\beta_1 \hat{\var}(x_i^*) + \beta_1
    \hat{\cov}(\nu_i, x_i^*) + \hat{\cov}(x_i^*, \varepsilon_i) +
    \hat{\cov}(\nu_i, \varepsilon_i)}{\hat{\var}(x_i^*) +
    \hat{\var}(\nu_i)}\right) \\
    &= \frac{\beta_1 \var(x_i^*) + \beta_1 \cov(\nu_i, x_i^*) + \cov(x_i^*,
    \varepsilon_i) + \cov(\nu_i, \varepsilon_i)}{\var(x_i^*) +
    \var(\nu_i)}
\end{align*}
If we assume $\cov(\nu_i, x_i^*) = 0$, $\cov(x_i^*, \varepsilon_i) = 0$, and
$\cov(\nu_i, \varepsilon_i) = 0$, this simplifies to,
\begin{align*}
    \plim \hat{\beta}_1 &= \beta_1 \frac{\var(x_i^*)}{\var(x_i^*) +
    \var(\nu_i)}
\end{align*}
However, this cannot simplify any further. As a result, the estimate
$\hat{\beta}_1$ is inconsistent.

\problempart % Problem 2b

Using the same calculation as above, but substituting $x_i = x_i^* + \nu_i$
with $z_i = x_i^* + \eta_i$ yields,
\begin{align*}
    \plim \hat{\beta}_1 &= \beta_1 \frac{\var(x_i^*)}{\var(x_i^*) +
    \var(\eta_i)}
\end{align*}
This result is still inconsistent. Furthermore, since $\var(\eta_i)$ is
likely greater than $\var(\nu_i)$ (and thus further from zero), this result
is likely worse than the previous.

\problempart % Problem 2c

Suppose we are regressing using the following model,
$$ x_i = \beta_0 + \beta_1 z_i + \varepsilon_i $$
The estimate $\hat{\beta}_1$ will converge to the following,
\begin{align*}
    \plim \hat{\beta}_1 &= \plim \left(\frac{\hat{\cov}(x_i,
    z_i)}{\hat{\var}(z_i)}\right) \\
    &= \plim \left(\frac{\hat{\cov}(x_i^* + \nu_i, x_i^* +
    \eta_i)}{\hat{\var}(x_i^* + \eta_i)}\right) \\
    &= \plim \left(\frac{\hat{\var}(x_i^*) + \hat{\cov}(x_i^*, \eta_i) +
    \hat{\cov}(\nu_i, x_i^*) + \hat{\cov}(\nu_i, \eta_i)}{\hat{\var}(x_i^*) +
    \hat{\var}(\eta_i)}\right) \\
    &= \frac{\var(x_i^*) + \cov(x_i^*, \eta_i) +
    \cov(\nu_i, x_i^*) + \cov(\nu_i, \eta_i)}{\var(x_i^*) +
    \var(\eta_i)}
\end{align*}
If we assume $\cov(\nu_i, x_i^*) = 0$, $\cov(x_i^*, \eta_i) = 0$, and
$\cov(\nu_i, \eta_i) = 0$, this simplifies to,
\begin{align*}
    \plim \hat{\beta}_1 &= \frac{\var(x_i^*)}{\var(x_i^*) + \var(\eta_i)}
\end{align*}

\problempart % Problem 2d

If we take the estimate from Part B,
\begin{align*}
    \plim \hat{\beta}_1^{yz} &= \beta_1 \frac{\var(x_i^*)}{\var(x_i^*) +
    \var(\eta_i)}
\end{align*}
And divide it by the estimate from Part C,
\begin{align*}
    \plim \hat{\beta}_1^{xz} &= \frac{\var(x_i^*)}{\var(x_i^*) + \var(\eta_i)}
\end{align*}
We are left with the true $\beta_1$,
$$\plim \left(\frac{\hat{\beta}_1^{yz}}{\hat{\beta}_1^{xz}}\right) = \beta_1 $$

\problempart % Problem 2e

First, we check the condition of exogeneity (i.e. $\cov (\varepsilon_i, z_i)
= 0$).
\begin{align*}
    \cov (\varepsilon_i, z_i) &= \cov(\varepsilon_i, x_i^* + \eta_i) \\
    &= \cov(\varepsilon_i, x_i^*) + \cov(\varepsilon_i, \eta_i) \\
    &= 0
\end{align*}
Therefore, by the assumptions of the problem statement, the exogeneity
requirement is satisfied.

Next, we check the condition of relevance (i.e. $\cov(x_i^*, z_i) \neq 0$).
\begin{align*}
    \cov (x_i^*, z_i) &= \cov(x_i^*, x_i^* + \eta_i) \\
    &= \var(x_i^*) + \cov(x_i^*, \eta_i) \\
    &= \var(x_i^*)
\end{align*}
It is unlikely that the variance of $x_i^*$ will be zero. As a result, the
instrument is relevant.

\end{problemparts}

\newpage

\problem  % Problem 3

\begin{problemparts}

\problempart % Problem 3a

We consider the two simultaneous equations, 
$$ cri_i = \alpha_0 + \alpha_1 pol_i + \alpha_2 pop_i +
\varepsilon_i $$
$$ pol_i = \beta_0 + \beta_1 cri_i + \beta_2 pop_i + \eta_i $$
To show the estimate for $\alpha_1$ is biased, we prove $\cov(pol_i,
\varepsilon_i) \neq 0$,
\begin{align*}
    \cov(pol_i, \varepsilon_i) &= \cov(\beta_0 + \beta_1 cri_i + \beta_2
    pop_i + \eta_i, \varepsilon_i) \\
    &= \cov(\beta_0 + \beta_1 (\alpha_0 + \alpha_1 pol_i + \alpha_2
    pop_i + \varepsilon_i) + \beta_2 pop_i + \eta_i,
    \varepsilon_i) \\
    &= \cov(\beta_0 + \beta_1 (\alpha_0 + \alpha_1 pol_i + \alpha_2
    pop_i) + \beta_2 pop_i + \eta_i,
    \varepsilon_i) + \beta_1 \var(\varepsilon_i) 
\end{align*}
Since $\var (\varepsilon_i) \neq 0$, this shows that $\cov(pol_i,
\varepsilon_i) \neq 0$.

\problempart % Problem 3b

We start with the equation for $pol_i$,
\begin{align*}
    pol_i &= \beta_0 + \beta_1 cri_i + \beta_2 pop_i + \eta_i \\
    pol_i &= \beta_0 + \beta_1 (\alpha_0 + \alpha_1 pol_i +
    \alpha_2 pop_i + \varepsilon_i)+ \beta_2 pop_i + \eta_i \\
    (1 - \beta_1 \alpha_1) pol_i &= \beta_0 + \beta_1 \alpha_0 +
    (\beta_1 \alpha_2 + \beta_2) pop_i + \beta_1 \varepsilon_i +
    \eta_i \\
    pol_i &= \frac{\beta_0 + \beta_1 \alpha_0}{1 - \beta_1 \alpha_1}
    + \frac{\beta_1 \alpha_2 + \beta_2}{1 - \beta_1 \alpha_1} pop_i +
    \frac{\beta_1 \varepsilon_i + \eta_i}{1 - \beta_1 \alpha_1}
\end{align*}
Using the same method, the equation for $cri_i$ can be derived,
$$ cri_i = \frac{\alpha_0 + \alpha_1 \beta_0}{1 - \alpha_1 \beta_1} +
\frac{\alpha_1 \beta_2 - \alpha_2}{1 - \alpha_1 \beta_1} pop_i +
\frac{\alpha_1 \eta_i + \varepsilon_i}{1 - \alpha_1 \beta_1} $$

\problempart % Problem 3c

If we divide the coefficient of $pop_i$, let's call it
$\gamma^{cri}$, from the $cri_i$ equation by the coefficient, lets call
it $\gamma^{pol}$, from the $pol_i$ equation, our result doesn't
converge to $\alpha_1$.
\begin{align*}
    \frac{\gamma^{cri}}{\gamma^{pol}} &= \frac{\alpha_1 \beta_2 -
    \alpha_2}{1 - \alpha_1 \beta_1} \cdot \frac{1 - \beta_1 \alpha_1}{\beta_1
    \alpha_2 + \beta_2} \\
    &= \frac{\alpha_1 \beta_2 - \alpha_2}{\beta_1 \alpha_2 + \beta_2}
\end{align*}
This comes from the fact that $pop_i$ is accounted for twice in the
original model.

\problempart % Problem 3d

Instead of accounting for $pop_i$ in both equations, we can limit it to
a single equation, the $pol_i$ equation. The new model could be,
$$ cri_i = \alpha_0 + \alpha_1 pol_i + \varepsilon_i $$
$$ pol_i = \beta_0 + \beta_1 cri_i + \beta_2 pop_i + \eta_i $$
The reduced form of this model becomes,
$$ cri_i = \frac{\alpha_0 + \alpha_1 \beta_0}{1 - \alpha_1 \beta_1} +
\frac{\alpha_1 \beta_2}{1 - \alpha_1\beta_1} pop_i + \frac{\alpha_1 \eta_i +
\varepsilon_i}{1 - \alpha_1 \beta_1} $$
$$ pol_i = \frac{\beta_0 + \beta_1 \alpha_0}{1 - \beta_1 \alpha_1} +
\frac{\beta_2}{1 - \beta_1\alpha_1} pop_i + \frac{\beta_1 \varepsilon_i +
\eta_i}{1 - \beta_1 \alpha_1} $$
Now if we divide the coefficient of $pop_i$ from the $cri_i$ equation by the
coefficient from the $pol_i$ equation, our result converges to $\alpha_1$.
\begin{align*}
    \frac{\gamma^{cri}}{\gamma^{pol}} &= \frac{\alpha_1 \beta_2}{1 -
    \alpha_1\beta_1} \cdot \frac{1 - \beta_1\alpha_1}{\beta_2} \\
    &= \alpha_1
\end{align*}

\end{problemparts}

\newpage

\problem  % Problem 4

\begin{problemparts}

\problempart % Problem 4a

Refer to the attached Jupyter notebook printout.

\problempart % Problem 4b

There are several possibilities for bias in the estimate of the effect of
education on weekly earnings. Some examples could be
\begin{itemize}
    \item Parent's Income
    \item Intelligence
    \item Confidence
\end{itemize}

Consider the effect of ommitted variables mathematically,
\begin{align*}
    \beta^{\mathrm{obs}} &= \beta^{\mathrm{true}} + \beta^{\mathrm{omit}}
    \frac{\cov(x^{\mathrm{omit}},
    x^{\mathrm{true}})}{\var(x^{\mathrm{true}})} \\
    &= \beta^{\mathrm{true}} + \frac{\cov(y,
    x^{\mathrm{omit}})}{\var(x^{\mathrm{omit}})}\frac{\cov(x^{\mathrm{omit}},
    x^{\mathrm{true}})}{\var(x^{\mathrm{true}})} \\
\end{align*}
Therefore, to consider whether the coefficient is over- or under-stated for
each of these omitted variables, we need to consider two correlations:
$\cov(x^{\mathrm{omit}}, x^{\mathrm{true}})$ and $\cov(y,
x^{\mathrm{omit}})$. If both have the same sign, the observed estimate is
over-stated. If they have different signs, the observed estimate is
under-stated.

\begin{itemize}
    \item For parent's income ($parinc$), we should expect $\cov(parinc,
    educ)$ to be positive as the higher parent's income, the more likely
    someone is to attend college, graduate school, etc. But this could also
    be negative in some cases as rich children sometimes underperform and
    rely on their parent's money alone.

    On the other hand, for $\cov(lwklywge, parinc)$, we should expect
    children of wealthier parents to make more money themselves. Thus, this
    effect should also be positive. Nevertheless, trust fund babies do exist,
    so this could also be negative.

    In general, we should assume it's more likely that both are positive
    which would lead to an over-stated estimate.

    \item For intelligence ($intel$), we should expect $\cov(intel, educ)$ to
    be positive as a more intelligent person would find school easier and
    thus go further. However, even more intelligent individuals might find
    school boring (or realize their opportunity cost is too high) and drop
    out early. It's probably too difficult to say what the sign of this
    correlation is without some evidence.

    For $\cov(lwklywge, intel)$, it is very likely that more intelligent
    individuals make more money than less intelligent individuals. But, then
    again, there are professors who quite intelligent, but might not make as
    much as less intelligent individuals in non-academic settings. Though,
    overall, it's likely to be more positive.

    Since the first correlation is a toss up and the second is likely
    positive, it's hard to say whether this will lead to over- or
    under-stated estimation. If I had to guess, I would go with
    under-stated estimates.

    \item For confidence ($conf$), similar arguments apply. For $\cov(conf,
    educ)$, it's likely positive as more confident individuals are likely to
    do well in school and continue longer than those who are not. Yet, they
    might be so confident that they don't need school at all!

    For $\cov(lwklywge, conf)$, it is very reasonable to expect more
    confident individuals will make more money than less confident
    individuals. I can't really see why more confident people who make less,
    perhaps they take too many risks? All considered, this is likely
    positive.

    For the same reasons as intelligence, this could be over- or
    under-stated, but if I had to guess, I would go with under-stated
    estimation.

\end{itemize}

\problempart % Problem 4c

The two conditions the instrument $z$ needs to satisfy are relevance ($\cov(x,
z) \neq 0$) and exogeneity ($\cov(x, \varepsilon) = 0$).

First we consider relevance. Given the argument given by Angrist and Krueger,
it certainly seems reasonable that what time of the year would have an impact
on how much schooling you would receive. For some individuals, they are
planning to drop out as soon as they legally can. This will solely be
determined by the time of year they were born. However, in today's world, a
practice of "red-shirting" is common where parent's hold their kids back for
a year or so such that they are older than their classmates. This would mess
with the relevance. Nevertheless, I believe the practice was not nearly as
common back when Angrist and Krueger did their study. Therefore, I expect
this condition is reasonably satisfied.

For exogeneity, the analysis is expected to be a little simpler. Surely what
month you were born has a negligible effect on your earnings. After all, the
distribution of birthdays should be almost random (economist's dream). In
some cases or careers (like Malcolm Galdwell's analysis of birthdays of
professional hockey players) there might be some correlation to time of year
of birth and earnings, but on the average, this is likely negligible.
Therefore, I expect this condition is reasonably satisfied.

\problempart % Problem 4d

Refer to the attached Jupyter notebook printout.

\problempart % Problem 4e

Refer to the attached Jupyter notebook printout.

\problempart % Problem 4f

Refer to the attached Jupyter notebook printout.

\problempart % Problem 4g

Refer to the attached Jupyter notebook printout.

\problempart % Problem 4h

Refer to the attached Jupyter notebook printout.

\end{problemparts}

\end{problems}

\end{document}
