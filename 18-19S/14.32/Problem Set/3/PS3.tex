%
% 6.S077 problem set solutions template
%
\documentclass[12pt,twoside]{article}

\newcommand{\name}{}

\usepackage{amssymb}
\usepackage{amsmath}
\usepackage{graphicx}
\usepackage{latexsym}
\usepackage{times,url}
\usepackage{cprotect}
\usepackage{listings}
\usepackage{graphicx}
\usepackage[table]{xcolor}
\usepackage[letterpaper]{geometry}
\usepackage{tikz-qtree}
\usepackage{enumerate}

\newcommand{\profs}{Mauricio Karchmer, Aleksander Madry, Bruce Tidor}
\newcommand{\subj}{6.046}
\newcommand{\ttt}[1]{{\tt\small #1}}

\definecolor{dkgreen}{rgb}{0,0.6,0}
\definecolor{gray}{rgb}{0.5,0.5,0.5}
\definecolor{mauve}{rgb}{0.58,0,0.82}

\lstset{
  language=Python,
  aboveskip=1pc,
  belowskip=1pc,
  basicstyle={\footnotesize\ttfamily},
  numbers=left,
  showstringspaces=false,
  numberstyle=\tiny\color{gray},
  keywordstyle=\color{blue},
  commentstyle=\color{dkgreen},
  stringstyle=\color{mauve},
}

\tikzset{
  % every node/.style={minimum width=2em,draw,circle},
  % level 1/.style={sibling distance=2cm},
  level distance=1cm,
  edge from parent/.style=
  {draw,edge from parent path={(\tikzparentnode) -- (\tikzchildnode)}},
}

\newif\ifHideSolutions
\newcommand{\solution}[1]{\color{dkgreen}\textbf{Solution: }#1\color{black}}
\newcommand{\rubric}[1]{\color{dkgreen}{\bf Rubric:} #1\color{black}}

% \HideSolutionsfalse
% \ifHideSolutions
%   \renewcommand{\solution}[1]{}
%   \renewcommand{\rubric}[1]{}
% \fi

\newlength{\toppush}
\setlength{\toppush}{2\headheight}
\addtolength{\toppush}{\headsep}

\newcommand{\htitle}[2]{\noindent\vspace*{-\toppush}\newline\parbox{6.5in}
{\textit{Design and Analysis of Algorithms}\hfill\name\newline
Massachusetts Institute of Technology \hfill #2\newline
\profs\hfill #1 \vspace*{-.5ex}\newline
\mbox{}\hrulefill\mbox{}}\vspace*{1ex}\mbox{}\newline
\begin{center}{\Large\bf #1}\end{center}}

\newcommand{\handout}[2]{\thispagestyle{empty}
 \markboth{#1}{#1}
 \pagestyle{myheadings}\htitle{#1}{#2}}

\newcommand{\lecture}[3]{\thispagestyle{empty}
 \markboth{Lecture #1: #2}{Lecture #1: #2}
 \pagestyle{myheadings}\htitle{Lecture #1: #2}{#3}}

\newcommand{\htitlewithouttitle}[2]{\noindent\vspace*{-\toppush}\newline\parbox{6.5in}
{\textit{Design and Analysis of Algorithms}\hfill#2\newline
Massachusetts Institute of Technology \hfill 6.046\newline
\profs\hfill Handout #1\vspace*{-.5ex}\newline
\mbox{}\hrulefill\mbox{}}\vspace*{1ex}\mbox{}\newline}

\newcommand{\handoutwithouttitle}[2]{\thispagestyle{empty}
 \markboth{Handout \protect\ref{#1}}{Handout \protect\ref{#1}}
 \pagestyle{myheadings}\htitlewithouttitle{\protect\ref{#1}}{#2}}

\newcommand{\exam}[2]{% parameters: exam name, date
 \thispagestyle{empty}
 \markboth{\hspace{1cm}\subj\ #1\hspace{1in}Name\hrulefill\ \ }%
          {\subj\ #1\hspace{1in}Name\hrulefill\ \ }
 \pagestyle{myheadings}\examtitle{#1}{#2}
 \renewcommand{\theproblem}{Problem \arabic{problemnum}}
}
\newcommand{\examsolutions}[3]{% parameters: handout, exam name, date
 \thispagestyle{empty}
 \markboth{Handout \protect\ref{#1}: #2}{Handout \protect\ref{#1}: #2}
% \pagestyle{myheadings}\htitle{\protect\ref{#1}}{#2}{#3}
 \pagestyle{myheadings}\examsolutionstitle{\protect\ref{#1}} {#2}{#3}
 \renewcommand{\theproblem}{Problem \arabic{problemnum}}
}
\newcommand{\examsolutionstitle}[3]{\noindent\vspace*{-\toppush}\newline\parbox{6.5in}
{\textit{Design and Analysis of Algorithms}\hfill#3\newline
Massachusetts Institute of Technology \hfill 6.046\newline
%Singapore-MIT Alliance \hfill SMA5503\newline
\profs\hfill Handout #1\vspace*{-.5ex}\newline
\mbox{}\hrulefill\mbox{}}\vspace*{1ex}\mbox{}\newline
\begin{center}{\Large\bf #2}\end{center}}

\newcommand{\takehomeexam}[2]{% parameters: exam name, date
 \thispagestyle{empty}
 \markboth{\subj\ #1\hfill}{\subj\ #1\hfill}
 \pagestyle{myheadings}\examtitle{#1}{#2}
 \renewcommand{\theproblem}{Problem \arabic{problemnum}}
}

\makeatletter
\newcommand{\exambooklet}[2]{% parameters: exam name, date
 \thispagestyle{empty}
 \markboth{\subj\ #1}{\subj\ #1}
 \pagestyle{myheadings}\examtitle{#1}{#2}
 \renewcommand{\theproblem}{Problem \arabic{problemnum}}
 \renewcommand{\problem}{\newpage
 \item \let\@currentlabel=\theproblem
 \markboth{\subj\ #1, \theproblem}{\subj\ #1, \theproblem}}
}
\makeatother


\newcommand{\examtitle}[2]{\noindent\vspace*{-\toppush}\newline\parbox{6.5in}
{\textit{Design and Analysis of Algorithms}\hfill#2\newline
Massachusetts Institute of Technology \hfill 6.046 Spring 2019\newline
%Singapore-MIT Alliance \hfill SMA5503\newline
\profs\hfill #1\vspace*{-.5ex}\newline
\mbox{}\hrulefill\mbox{}}\vspace*{1ex}\mbox{}\newline
\begin{center}{\Large\bf #1}\end{center}}

\newcommand{\grader}[1]{\hspace{1cm}\textsf{\textbf{#1}}\hspace{1cm}}

\newcommand{\points}[1]{[#1 points]\ }
\newcommand{\parts}[1]
{
  \ifnum#1=1
  (1 part)
  \else
  (#1 parts)
  \fi
  \ 
}

\newcommand{\bparts}{\begin{problemparts}}
\newcommand{\eparts}{\end{problemparts}}
\newcommand{\ppart}{\problempart}

%\newcommand{\lg} {lg\ }

\setlength{\oddsidemargin}{0pt}
\setlength{\evensidemargin}{0pt}
\setlength{\textwidth}{6.5in}
\setlength{\topmargin}{0in}
\setlength{\textheight}{8.5in}


\newcommand{\Spawn}{{\bf spawn} }
\newcommand{\Sync}{{\bf sync}}

\newcommand{\cif}[1]{\mbox{if $#1$}}
\newcommand{\cwhen}[1]{\mbox{when $#1$}}

\newcounter{problemnum}
\newcommand{\theproblem}{Problem \theproblemsetnum-\arabic{problemnum}}
\newenvironment{problems}{
        \begin{list}{{\bf \theproblem. \hspace*{0.5em}}}
        {\setlength{\leftmargin}{0em}
         \setlength{\rightmargin}{0em}
         \setlength{\labelwidth}{0em}
         \setlength{\labelsep}{0em}
         \usecounter{problemnum}}}{\end{list}}
\makeatletter
\newcommand{\problem}[1][{}]{\item \let\@currentlabel=\theproblem \textbf{#1}}
\makeatother

\newcounter{problempartnum}[problemnum]
\newenvironment{problemparts}{
        \begin{list}{{\bf (\alph{problempartnum})}}
        {\setlength{\leftmargin}{2.5em}
         \setlength{\rightmargin}{2.5em}
         \setlength{\labelsep}{0.5em}}}{\end{list}}
\newcommand{\problempart}{\addtocounter{problempartnum}{1}\item}

\newenvironment{truefalseproblemparts}{
        \begin{list}{{\bf (\alph{problempartnum})\ \ \ T\ \ F\hfil}}
        {\setlength{\leftmargin}{4.5em}
         \setlength{\rightmargin}{2.5em}
         \setlength{\labelsep}{0.5em}
         \setlength{\labelwidth}{4.5em}}}{\end{list}}

\newcounter{exercisenum}
\newcommand{\theexercise}{Exercise \theproblemsetnum-\arabic{exercisenum}}
\newenvironment{exercises}{
        \begin{list}{{\bf \theexercise. \hspace*{0.5em}}}
        {\setlength{\leftmargin}{0em}
         \setlength{\rightmargin}{0em}
         \setlength{\labelwidth}{0em}
         \setlength{\labelsep}{0em}
        \usecounter{exercisenum}}}{\end{list}}
\makeatletter
\newcommand{\exercise}{\item \let\@currentlabel=\theexercise}
\makeatother

\newcounter{exercisepartnum}[exercisenum]
%\newcommand{\problem}[1]{\medskip\mbox{}\newline\noindent{\bf Problem #1.}\hspace*{1em}}
%\newcommand{\exercise}[1]{\medskip\mbox{}\newline\noindent{\bf Exercise #1.}\hspace*{1em}}

\newenvironment{exerciseparts}{
        \begin{list}{{\bf (\alph{exercisepartnum})}}
        {\setlength{\leftmargin}{2.5em}
         \setlength{\rightmargin}{2.5em}
         \setlength{\labelsep}{0.5em}}}{\end{list}}
\newcommand{\exercisepart}{\addtocounter{exercisepartnum}{1}\item}


% Macros to make captions print with small type and 'Figure xx' in bold.
\makeatletter
\def\fnum@figure{{\bf Figure \thefigure}}
\def\fnum@table{{\bf Table \thetable}}
\let\@mycaption\caption
%\long\def\@mycaption#1[#2]#3{\addcontentsline{\csname
%  ext@#1\endcsname}{#1}{\protect\numberline{\csname 
%  the#1\endcsname}{\ignorespaces #2}}\par
%  \begingroup
%    \@parboxrestore
%    \small
%    \@makecaption{\csname fnum@#1\endcsname}{\ignorespaces #3}\par
%  \endgroup}
%\def\mycaption{\refstepcounter\@captype \@dblarg{\@mycaption\@captype}}
%\makeatother
\let\mycaption\caption
%\newcommand{\figcaption}[1]{\mycaption[]{#1}}

\newcounter{totalcaptions}
\newcounter{totalart}

\newcommand{\figcaption}[1]{\addtocounter{totalcaptions}{1}\caption[]{#1}}

% \psfigures determines what to do for figures:
%       0 means just leave vertical space
%       1 means put a vertical rule and the figure name
%       2 means insert the PostScript version of the figure
%       3 means put the figure name flush left or right
\newcommand{\psfigures}{0}
\newcommand{\spacefigures}{\renewcommand{\psfigures}{0}}
\newcommand{\rulefigures}{\renewcommand{\psfigures}{1}}
\newcommand{\macfigures}{\renewcommand{\psfigures}{2}}
\newcommand{\namefigures}{\renewcommand{\psfigures}{3}}

\newcommand{\figpart}[1]{{\bf (#1)}\nolinebreak[2]\relax}
\newcommand{\figparts}[2]{{\bf (#1)--(#2)}\nolinebreak[2]\relax}


\macfigures     % STATE

% When calling \figspace, make sure to leave a blank line afterward!!
% \widefigspace is for figures that are more than 28pc wide.
\newlength{\halffigspace} \newlength{\wholefigspace}
\newlength{\figruleheight} \newlength{\figgap}
\newcommand{\setfiglengths}{\ifnum\psfigures=1\setlength{\figruleheight}{\hruleheight}\setlength{\figgap}{1em}\else\setlength{\figruleheight}{0pt}\setlength{\figgap}{0em}\fi}
\newcommand{\figspace}[2]{\ifnum\psfigures=0\leavefigspace{#1}\else%
\setfiglengths%
\setlength{\wholefigspace}{#1}\setlength{\halffigspace}{.5\wholefigspace}%
\rule[-\halffigspace]{\figruleheight}{\wholefigspace}\hspace{\figgap}#2\fi}
\newlength{\widefigspacewidth}
% Make \widefigspace put the figure flush right on the text page.
\newcommand{\widefigspace}[2]{
\ifnum\psfigures=0\leavefigspace{#1}\else%
\setfiglengths%
\setlength{\widefigspacewidth}{28pc}%
\addtolength{\widefigspacewidth}{-\figruleheight}%
\setlength{\wholefigspace}{#1}\setlength{\halffigspace}{.5\wholefigspace}%
\makebox[\widefigspacewidth][r]{#2\hspace{\figgap}}\rule[-\halffigspace]{\figruleheight}{\wholefigspace}\fi}
\newcommand{\leavefigspace}[1]{\setlength{\wholefigspace}{#1}\setlength{\halffigspace}{.5\wholefigspace}\rule[-\halffigspace]{0em}{\wholefigspace}}

% Commands for including figures with macpsfig.
% To use these commands, documentstyle ``macpsfig'' must be specified.
\newlength{\macfigfill}
\makeatother
\newlength{\bbx}
\newlength{\bby}
\newcommand{\macfigure}[5]{\addtocounter{totalart}{1}
\ifnum\psfigures=2%
\setlength{\bbx}{#2}\addtolength{\bbx}{#4}%
\setlength{\bby}{#3}\addtolength{\bby}{#5}%
\begin{flushleft}
\ifdim#4>28pc\setlength{\macfigfill}{#4}\addtolength{\macfigfill}{-28pc}\hspace*{-\macfigfill}\fi%
\mbox{\psfig{figure=./#1.ps,%
bbllx=#2,bblly=#3,bburx=\bbx,bbury=\bby}}
\end{flushleft}%
\else\ifdim#4>28pc\widefigspace{#5}{#1}\else\figspace{#5}{#1}\fi\fi}
\makeatletter

\newlength{\savearraycolsep}
\newcommand{\narrowarray}[1]{\setlength{\savearraycolsep}{\arraycolsep}\setlength{\arraycolsep}{#1\arraycolsep}}
\newcommand{\normalarray}{\setlength{\arraycolsep}{\savearraycolsep}}

\newcommand{\hint}{{\em Hint:\ }}

% Macros from /th/u/clr/mac.tex

\newcommand{\set}[1]{\left\{ #1 \right\}}
\newcommand{\abs}[1]{\left| #1\right|}
\newcommand{\card}[1]{\left| #1\right|}
\newcommand{\floor}[1]{\left\lfloor #1 \right\rfloor}
\newcommand{\ceil}[1]{\left\lceil #1 \right\rceil}
\newcommand{\ang}[1]{\ifmmode{\left\langle #1 \right\rangle}
   \else{$\left\langle${#1}$\right\rangle$}\fi}
        % the \if allows use outside mathmode,
        % but will swallow following space there!
\newcommand{\paren}[1]{\left( #1 \right)}
\newcommand{\bracket}[1]{\left[ #1 \right]}
\newcommand{\prob}[1]{\Pr\left\{ #1 \right\}}
\newcommand{\Var}{\mathop{\rm Var}\nolimits}
\newcommand{\expect}[1]{{\rm E}\left[ #1 \right]}
\newcommand{\expectsq}[1]{{\rm E}^2\left[ #1 \right]}
\newcommand{\variance}[1]{{\rm Var}\left[ #1 \right]}
\renewcommand{\choose}[2]{{{#1}\atopwithdelims(){#2}}}
\def\pmod#1{\allowbreak\mkern12mu({\rm mod}\,\,#1)}
\newcommand{\matx}[2]{\left(\begin{array}{*{#1}{c}}#2\end{array}\right)}
\newcommand{\Adj}{\mathop{\rm Adj}\nolimits}

\newtheorem{theorem}{Theorem}
\newtheorem{lemma}[theorem]{Lemma}
\newtheorem{corollary}[theorem]{Corollary}
\newtheorem{xample}{Example}
\newtheorem{definition}{Definition}
\newenvironment{example}{\begin{xample}\rm}{\end{xample}}
\newcommand{\proof}{\noindent{\em Proof.}\hspace{1em}}
\def\squarebox#1{\hbox to #1{\hfill\vbox to #1{\vfill}}}
\newcommand{\qedbox}{\vbox{\hrule\hbox{\vrule\squarebox{.667em}\vrule}\hrule}}
\newcommand{\qed}{\nopagebreak\mbox{}\hfill\qedbox\smallskip}
\newcommand{\eqnref}[1]{(\protect\ref{#1})}

%%\newcommand{\twodots}{\mathinner{\ldotp\ldotp}}
\newcommand{\transpose}{^{\mbox{\scriptsize \sf T}}}
\newcommand{\amortized}[1]{\widehat{#1}}

\newcommand{\punt}[1]{}

%%% command for putting definitions into boldface
% New style for defined terms, as of 2/23/88, redefined by THC.
\newcommand{\defn}[1]{{\boldmath\textit{\textbf{#1}}}}
\newcommand{\defi}[1]{{\textit{\textbf{#1\/}}}}

\newcommand{\red}{\leq_{\rm P}}
\newcommand{\lang}[1]{%
\ifmmode\mathord{\mathcode`-="702D\rm#1\mathcode`\-="2200}\else{\rm#1}\fi}

%\newcommand{\ckt}[1]{\ifmmode\mathord{\mathcode`-="702D\sc #1\mathcode`\-="2200}\else$\mathord{\mathcode`-="702D\sc #1\mathcode`\-="2200}$\fi}
\newcommand{\ckt}[1]{\ifmmode \sc #1\else$\sc #1$\fi}

%% Margin notes - use \notesfalse to turn off notes.
\setlength{\marginparwidth}{0.6in}
\reversemarginpar
\newif\ifnotes
\notestrue
\newcommand{\longnote}[1]{
  \ifnotes
    {\medskip\noindent Note: \marginpar[\hfill$\Longrightarrow$]
      {$\Longleftarrow$}{#1}\medskip}
  \fi}
\newcommand{\note}[1]{
  \ifnotes
    {\marginpar{\tiny \raggedright{#1}}}
  \fi}


\newcommand{\reals}{\mathbbm{R}}
\newcommand{\integers}{\mathbbm{Z}}
\newcommand{\naturals}{\mathbbm{N}}
\newcommand{\rationals}{\mathbbm{Q}}
\newcommand{\complex}{\mathbbm{C}}

\newcommand{\oldreals}{{\bf R}}
\newcommand{\oldintegers}{{\bf Z}}
\newcommand{\oldnaturals}{{\bf N}}
\newcommand{\oldrationals}{{\bf Q}}
\newcommand{\oldcomplex}{{\bf C}}

\newcommand{\w}{\omega}                 %% for fft chapter

\newenvironment{closeitemize}{\begin{list}
{$\bullet$}
{\setlength{\itemsep}{-0.2\baselineskip}
\setlength{\topsep}{0.2\baselineskip}
\setlength{\parskip}{0pt}}}
{\end{list}}

% These are necessary within a {problems} environment in order to restore
% the default separation between bullets and items.
\newenvironment{normalitemize}{\setlength{\labelsep}{0.5em}\begin{itemize}}
                              {\end{itemize}}
\newenvironment{normalenumerate}{\setlength{\labelsep}{0.5em}\begin{enumerate}}
                                {\end{enumerate}}

%\def\eqref#1{Equation~(\ref{eq:#1})}
%\newcommand{\eqref}[1]{Equation (\ref{eq:#1})}
\newcommand{\eqreftwo}[2]{Equations (\ref{eq:#1}) and~(\ref{eq:#2})}
\newcommand{\ineqref}[1]{Inequality~(\ref{ineq:#1})}
\newcommand{\ineqreftwo}[2]{Inequalities (\ref{ineq:#1}) and~(\ref{ineq:#2})}

\newcommand{\figref}[1]{Figure~\ref{fig:#1}}
\newcommand{\figreftwo}[2]{Figures \ref{fig:#1} and~\ref{fig:#2}}

\newcommand{\liref}[1]{line~\ref{li:#1}}
\newcommand{\Liref}[1]{Line~\ref{li:#1}}
\newcommand{\lirefs}[2]{lines \ref{li:#1}--\ref{li:#2}}
\newcommand{\Lirefs}[2]{Lines \ref{li:#1}--\ref{li:#2}}
\newcommand{\lireftwo}[2]{lines \ref{li:#1} and~\ref{li:#2}}
\newcommand{\lirefthree}[3]{lines \ref{li:#1}, \ref{li:#2}, and~\ref{li:#3}}

\newcommand{\lemlabel}[1]{\label{lem:#1}}
\newcommand{\lemref}[1]{Lemma~\ref{lem:#1}} 

\newcommand{\exref}[1]{Exercise~\ref{ex:#1}}

\newcommand{\handref}[1]{Handout~\ref{#1}}

\newcommand{\defref}[1]{Definition~\ref{def:#1}}

% (1997.8.16: Victor Luchangco)
% Modified \hlabel to only get date and to use handouts counter for number.
%   New \handout and \handoutwithouttitle commands in newmac.tex use this.
%   The date is referenced by <label>-date.
%   (Retained old definition as \hlabelold.)
%   Defined \hforcelabel to use an argument instead of the handouts counter.

\newcounter{handouts}
\setcounter{handouts}{0}

\newcommand{\hlabel}[2]{%
\stepcounter{handouts}
{\edef\next{\write\@auxout{\string\newlabel{#1}{{\arabic{handouts}}{0}}}}\next}
\write\@auxout{\string\newlabel{#1-date}{{#2}{0}}}
}

\newcommand{\hforcelabel}[3]{%          Does not step handouts counter.
\write\@auxout{\string\newlabel{#1}{{#2}{0}}}
\write\@auxout{\string\newlabel{#1-date}{{#3}{0}}}}


% less ugly underscore
% --juang, 2008 oct 05
\renewcommand{\_}{\vrule height 0 pt depth 0.4 pt width 0.5 em \,}

\usepackage{enumitem}
\newcommand{\theproblemsetnum}{3}
\newcommand{\releasedate}{Friday, March 8}
\newcommand{\partaduedate}{Friday, March 15}

\title{14.32 Problem Set \theproblemsetnum}

\begin{document}

\handout{Problem Set \theproblemsetnum}{\releasedate}
\textbf{All parts are due {\bf \partaduedate} at {\bf 9:00AM}}.

\setlength{\parindent}{0pt}
\medskip\hrulefill\medskip

{\bf Name:} Robert Durfee

\medskip

{\bf Collaborators:} None

\medskip\hrulefill

\begin{problems}

%%%%%%%%%%%%%%%%%%%%%%%%%%%%%%%%%%%%%%%%%%%%%%%%%%%%%%%%%%%%%%%%%%%%%%%%%%%%%%%%
\problem  % Problem 1

An $F$-test should be used when testing a join null hypothesis. That is, if the
test involves multiple variables in the regression. For example, if, using a
multiple regression on the effect of each color of jelly beans on acne, you
wanted to know if {\it any} colors had an effect on acne. An $F$-test does {\it 
not} need to be used by default in a multiple regression, however. In the same
example, if you only wanted to know if blue jelly beans had an impact on acne,
you could use a $t$-statistic.

%%%%%%%%%%%%%%%%%%%%%%%%%%%%%%%%%%%%%%%%%%%%%%%%%%%%%%%%%%%%%%%%%%%%%%%%%%%%%%%%
\problem  % Problem 2

\begin{problemparts}

\problempart % Problem 2a

For Gauss-Markov assumption A0:

\begin{enumerate}[label=\textbf{\roman*})]

    \item $\left(X^T X\right)^{-1}$ exists. Or, equivalently, $X$ has full 
        column rank.
        
    \item When we have more observations than features, this is stating that 
        all of our observations' features are linearly independent. This is a 
        totally reasonable assumption assumption prohibiting, for example, using
        both miles per gallon and kilometers per liter as features for fuel
        efficiency of cars.
        
    \item This can go wrong if we have multiple features that are related to
        other features linearly (or very nearly linearly). An example could be
        the above comparison of mpg vs kpl. Other examples as more subtle like
        using indicators for both female gender and male gender.
        
        This can also go wrong if we have more features than observations.
        
\end{enumerate}

\problempart % Problem 2b

For Gauss-Markov assumption A1:

\begin{enumerate}[label=\textbf{\roman*})]

    \item $\mathbb{E}\left[\mathcal{E} \mid X\right] = 0$.
    
    \item This is stating that, all other factors not included in our regressor 
        that could contribute to differences in our regressand, are not 
        systematically assigned. For example, if we are regressing income on 
        education, this assumes that other factors that have a substantial 
        impact on income (notably intelligence) doesn't increase systematically 
        as education increases (which is unlikely the case).
        
    \item This can go wrong if the treatment is self-selective. That is, if the
        individual chooses their treatment based off conditions they know about
        themselves. For example, if we are regressing skin cancer on sunscreen
        lotion usage, an individual would likely self-select to use lotion if 
        they know they are fair skinned.
        
        This can also go wrong if there is reverse-causality. That is, if the
        regressand causes the individual to have a certain treatment. Using the
        skin cancer example, if someone has a history of skin cancer, they are
        likely to use more sunscreen lotion.
\end{enumerate}

\problempart % Problem 2c

For Gauss-Markov assumption A2:

\begin{enumerate}[label=\textbf{\roman*})]

    \item $\mathrm{var}(\mathcal{E} \mid X) = \sigma^2 $.
    
    \item This is stating that the variability of the error term (the features
        not included as regressors in the model) is consistent regardless of 
        conditioning on a specific value of the regressor. For example, when
        regressing income on age, this assumes that income ranges are generally
        consistent as age increases (which is unlikely the case).
        
    \item This can go wrong if the recorded regressor observations are divided 
        into two groups where one is measured by a more precise tool and the
        other is less precise.
        
        This can also go wrong (almost surely) if the regressand is a binary
        variable.
        
\end{enumerate}

\problempart % Problem 2d

For Gauss-Markov assumption A3:

\begin{enumerate}[label=\textbf{\roman*})]

    \item $\mathrm{cov}(\varepsilon_i, \varepsilon_j \mid X) = 0\quad \forall i 
        \neq j$.
    
    \item This is stating that the error terms between different observations 
        are not related to each other. That is, if I know the error term for 
        observation $i$, that should provided no information about the error
        term for $j$.
    
    \item This typically occurs in time series data as the current observation
        is likely related to the previous few current observations in some
        functional form.
        
        This almost certainly occurs in panel data as we are making observations
        about the same individual many times (who likely maintains their same 
        error term over time).
        
        This can also occur in cross-sectional data when groups of individuals
        are sampled from shared environments (like households).
    
\end{enumerate}

\end{problemparts}

%%%%%%%%%%%%%%%%%%%%%%%%%%%%%%%%%%%%%%%%%%%%%%%%%%%%%%%%%%%%%%%%%%%%%%%%%%%%%%%%
\problem  % Problem 3

When assumptions A0--A3 hold, we know that our OLS estimator is BLUE: the {\bf 
b}est, {\bf l}inear, {\bf u}nbiased {\bf e}stimator. By `best', we mean the
estimator with the least variance.

%%%%%%%%%%%%%%%%%%%%%%%%%%%%%%%%%%%%%%%%%%%%%%%%%%%%%%%%%%%%%%%%%%%%%%%%%%%%%%%%
\problem  % Problem 4

\begin{problemparts}

\problempart % Problem 4a

By making this assumption, we are justified in using Student's $t$-distribution
as the model for our distribution of estimates if our null hypothesis is true
when conducting a single parameter hypothesis test.

\problempart % Problem 4b

The central limit theorem tells us that the sum of random variables will 
converge to a normal distribution. In this case, this is useful because it
suggests that our the distribution of our estimates will tend toward a normal
distribution for large enough samples. Thus, we can use Student's 
$t$-distribution for large samples without assuming Gaussian error.

\end{problemparts}

\newpage

%%%%%%%%%%%%%%%%%%%%%%%%%%%%%%%%%%%%%%%%%%%%%%%%%%%%%%%%%%%%%%%%%%%%%%%%%%%%%%%%
\problem  % Problem 5

\begin{problemparts}

\problempart % Problem 5a

The general form of the linear regression $y_i = a + b x_i + \varepsilon_i$ of
the form $Y = X \beta + \varepsilon$ written component-wise with the provided 
data is 
$$ \begin{pmatrix}
    1 \\
    3 \\
    1 \\
    2 \\
    4 \\
    1 \\
    2 \\
    1 \\
    3 \\
    2 \end{pmatrix} = \boxed{\begin{pmatrix}
    1 & 1 \\
    1 & 1 \\
    1 & 0 \\
    1 & 1 \\
    1 & 0 \\
    1 & 0 \\
    1 & 1 \\
    1 & 0 \\
    1 & 1 \\
    1 & 1 \end{pmatrix}} \cdot \boxed{\begin{pmatrix}
    a \\
    b \end{pmatrix}} + \begin{pmatrix}
    \varepsilon_1 \\
    \varepsilon_2 \\
    \varepsilon_3 \\
    \varepsilon_4 \\
    \varepsilon_5 \\
    \varepsilon_6 \\
    \varepsilon_7 \\
    \varepsilon_8 \\
    \varepsilon_9 \\
    \varepsilon_{10} \end{pmatrix} $$

\problempart % Problem 5b

The vector $\beta$ is provided in the previous part.

\problempart % Problem 5c

The equation for ordinary least squares estimator is
$$ \hat{\beta} = \left(X^T X\right)^{-1} X^T Y $$
Substituting the provided data and solving,
\begin{align*}
    \hat{\beta} &= \left(X^T X\right)^{-1} X^T Y \\
    &= \left(\begin{pmatrix}
    1 & 1 \\
    1 & 1 \\
    1 & 0 \\
    1 & 1 \\
    1 & 0 \\
    1 & 0 \\
    1 & 1 \\
    1 & 0 \\
    1 & 1 \\
    1 & 1 \end{pmatrix}^T \begin{pmatrix}
    1 & 1 \\
    1 & 1 \\
    1 & 0 \\
    1 & 1 \\
    1 & 0 \\
    1 & 0 \\
    1 & 1 \\
    1 & 0 \\
    1 & 1 \\
    1 & 1 \end{pmatrix}\right)^{-1} \begin{pmatrix}
    1 & 1 \\
    1 & 1 \\
    1 & 0 \\
    1 & 1 \\
    1 & 0 \\
    1 & 0 \\
    1 & 1 \\
    1 & 0 \\
    1 & 1 \\
    1 & 1 \end{pmatrix}^T \begin{pmatrix}
    1 \\
    3 \\
    1 \\
    2 \\
    4 \\
    1 \\
    2 \\
    1 \\
    3 \\
    2 \end{pmatrix} \\
    &= \begin{pmatrix}
    10 & 6 \\
    6 & 6 \end{pmatrix}^{-1} \begin{pmatrix}
    20 \\
    13 \end{pmatrix} \\
    &= \frac{1}{12}\begin{pmatrix}
    3 & -3 \\
    -3 & 5 \end{pmatrix} \begin{pmatrix}
    20 \\
    12 \end{pmatrix} \\
    &= \begin{pmatrix}
    7/4 \\
    5/12 \end{pmatrix}
\end{align*}
From this, $\boxed{\hat{a} = 7/4}$ and $\boxed{\hat{b} = 5/12}$.

\problempart % Problem 5d

To compute the standard errors, we first need the sum of squared residuals, 
$\mathrm{SSR}$.
\begin{align*}
    \mathrm{SSR} &= \sum_{i = 1}^n (y_i - \hat{a} - \hat{b} x_i)^2 \\
    &= \left(Y - X \hat{\beta}\right)^T \left(Y - \hat{\beta} X\right) \\
    &= \left(\begin{pmatrix}
    1 \\
    3 \\
    1 \\
    2 \\
    4 \\
    1 \\
    2 \\
    1 \\
    3 \\
    2 \end{pmatrix} - \begin{pmatrix}
    1 & 1 \\
    1 & 1 \\
    1 & 0 \\
    1 & 1 \\
    1 & 0 \\
    1 & 0 \\
    1 & 1 \\
    1 & 0 \\
    1 & 1 \\
    1 & 1 \end{pmatrix} \begin{pmatrix}
    7/4 \\
    5/12 \end{pmatrix}\right)^T \left(\begin{pmatrix}
    1 \\
    3 \\
    1 \\
    2 \\
    4 \\
    1 \\
    2 \\
    1 \\
    3 \\
    2 \end{pmatrix} - \begin{pmatrix}
    1 & 1 \\
    1 & 1 \\
    1 & 0 \\
    1 & 1 \\
    1 & 0 \\
    1 & 0 \\
    1 & 1 \\
    1 & 0 \\
    1 & 1 \\
    1 & 1 \end{pmatrix} \begin{pmatrix}
    7/4 \\
    5/12 \end{pmatrix}\right) \\
    &= \begin{pmatrix}
    -5/3 \\
    5/6 \\
    -3/4 \\
    -1/6 \\
    9/4 \\
    -3/4 \\
    -1/6 \\
    -3/4 \\
    5/6 \\
    -1/6 \end{pmatrix}^T \begin{pmatrix}
    -5/3 \\
    5/6 \\
    -3/4 \\
    -1/6 \\
    9/4 \\
    -3/4 \\
    -1/6 \\
    -3/4 \\
    5/6 \\
    -1/6 \end{pmatrix} \\
    &= 115 / 12 \\
    &\approx 9.583
\end{align*}
Next, using the $\mathrm{SSR}$, we can calculate the $\hat{\sigma}^2$,
\begin{align*}
    \hat{\sigma}^2 &= \frac{\mathrm{SSR}}{n - (k + 1)} \\
    &= \frac{115 / 12}{10 - (1 + 1)} \\
    &= 115 / 96 \\
    &\approx 1.198
\end{align*}
Now we can compute the standard error for $\hat{b}$,
\begin{align*}
    \hat{se}(\hat{b}) &= \sqrt{\hat{\sigma}^2 \left(X^T X\right)^{-1}_{1,1}} \\
    &= \sqrt{\frac{115}{96} \left(\begin{pmatrix}
    1 & 1 \\
    1 & 1 \\
    1 & 0 \\
    1 & 1 \\
    1 & 0 \\
    1 & 0 \\
    1 & 1 \\
    1 & 0 \\
    1 & 1 \\
    1 & 1 \end{pmatrix}^T \begin{pmatrix}
    1 & 1 \\
    1 & 1 \\
    1 & 0 \\
    1 & 1 \\
    1 & 0 \\
    1 & 0 \\
    1 & 1 \\
    1 & 0 \\
    1 & 1 \\
    1 & 1 \end{pmatrix}\right)^{-1}_{1, 1}} \\
    &= \sqrt{\frac{115}{96} \begin{pmatrix}
    10 & 6 \\
    6 & 6 \end{pmatrix}^{-1}_{1,1}} \\
    &= \sqrt{\frac{115}{96}\frac{1}{12}\begin{pmatrix}
    3 & -3 \\
    -3 & 5 \end{pmatrix}_{1, 1}} \\
    &= \sqrt{\frac{115}{96}\frac{5}{12}} \\
    &\approx 0.706
\end{align*}
From this, $\boxed{\hat{se}(\hat{b}) \approx 0.706}$. Using Student's
$t$-distribution with $n - (k + 1) = 8$ degrees of freedom, we know that the
critical value for an $\alpha = 0.05$ confidence interval is $2.306$. Using
this, we can construct a 95\% confidence interval for $\hat{b}$,
$$ \left[ 0.417 - 2.306 \cdot 0.706, 0.417 + 2.306 \cdot 0.706 \right] = \boxed{
\left[ -1.213, 2.046\right]} $$
Since this interval contains $b = 0$, we cannot reject the null hypothesis that
there is no gender discrimination in the military.

\problempart % Problem 5e

We can take the mean of $Y$ for just the males and just the females separately,
$$ \bar{Y}_m = 13/6,\quad \bar{Y}_f = 7/4 $$
Taking the difference between the two yields,
$$ \boxed{\bar{Y}_m - \bar{Y}_f = 5/12} $$
It makes sense that this is equal to $\hat{b}$ as we have a binary feature. Thus,
ordinary least squares will pass through the mean of the response from the first 
binary value and the mean of the response from the second binary value. Since
the two are separated by a single unit, the slope of ordinary least squares will
be equal to the difference of the two means.

\problempart % Problem 5f

$\bar{Y}_f$ is equal to the value of $\hat{a}$. The $\hat{a}$ value is the
intercept for ordinary least squares. The $\bar{Y}_f$ is the mean of the 
responses for a single value of the independent variable. Since this value of
the independent binary variable is $0$ for females, and ordinary least squares
is minimizing squared error, these two must coincide as the intercept is the
mean of the responses for when the independent binary variable is $0$.

\end{problemparts}

\end{problems}

\end{document}
