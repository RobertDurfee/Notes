%
% 6.046 problem set 1 solutions
%
\documentclass[12pt,twoside]{article}

\newcommand{\name}{}

\usepackage{amssymb}
\usepackage{amsmath}
\usepackage{graphicx}
\usepackage{latexsym}
\usepackage{times,url}
\usepackage{cprotect}
\usepackage{listings}
\usepackage{graphicx}
\usepackage[table]{xcolor}
\usepackage[letterpaper]{geometry}
\usepackage{tikz-qtree}
\usepackage{enumerate}

\newcommand{\profs}{Mauricio Karchmer, Aleksander Madry, Bruce Tidor}
\newcommand{\subj}{6.046}
\newcommand{\ttt}[1]{{\tt\small #1}}

\definecolor{dkgreen}{rgb}{0,0.6,0}
\definecolor{gray}{rgb}{0.5,0.5,0.5}
\definecolor{mauve}{rgb}{0.58,0,0.82}

\lstset{
  language=Python,
  aboveskip=1pc,
  belowskip=1pc,
  basicstyle={\footnotesize\ttfamily},
  numbers=left,
  showstringspaces=false,
  numberstyle=\tiny\color{gray},
  keywordstyle=\color{blue},
  commentstyle=\color{dkgreen},
  stringstyle=\color{mauve},
}

\tikzset{
  % every node/.style={minimum width=2em,draw,circle},
  % level 1/.style={sibling distance=2cm},
  level distance=1cm,
  edge from parent/.style=
  {draw,edge from parent path={(\tikzparentnode) -- (\tikzchildnode)}},
}

\newif\ifHideSolutions
\newcommand{\solution}[1]{\color{dkgreen}\textbf{Solution: }#1\color{black}}
\newcommand{\rubric}[1]{\color{dkgreen}{\bf Rubric:} #1\color{black}}

% \HideSolutionsfalse
% \ifHideSolutions
%   \renewcommand{\solution}[1]{}
%   \renewcommand{\rubric}[1]{}
% \fi

\newlength{\toppush}
\setlength{\toppush}{2\headheight}
\addtolength{\toppush}{\headsep}

\newcommand{\htitle}[2]{\noindent\vspace*{-\toppush}\newline\parbox{6.5in}
{\textit{Design and Analysis of Algorithms}\hfill\name\newline
Massachusetts Institute of Technology \hfill #2\newline
\profs\hfill #1 \vspace*{-.5ex}\newline
\mbox{}\hrulefill\mbox{}}\vspace*{1ex}\mbox{}\newline
\begin{center}{\Large\bf #1}\end{center}}

\newcommand{\handout}[2]{\thispagestyle{empty}
 \markboth{#1}{#1}
 \pagestyle{myheadings}\htitle{#1}{#2}}

\newcommand{\lecture}[3]{\thispagestyle{empty}
 \markboth{Lecture #1: #2}{Lecture #1: #2}
 \pagestyle{myheadings}\htitle{Lecture #1: #2}{#3}}

\newcommand{\htitlewithouttitle}[2]{\noindent\vspace*{-\toppush}\newline\parbox{6.5in}
{\textit{Design and Analysis of Algorithms}\hfill#2\newline
Massachusetts Institute of Technology \hfill 6.046\newline
\profs\hfill Handout #1\vspace*{-.5ex}\newline
\mbox{}\hrulefill\mbox{}}\vspace*{1ex}\mbox{}\newline}

\newcommand{\handoutwithouttitle}[2]{\thispagestyle{empty}
 \markboth{Handout \protect\ref{#1}}{Handout \protect\ref{#1}}
 \pagestyle{myheadings}\htitlewithouttitle{\protect\ref{#1}}{#2}}

\newcommand{\exam}[2]{% parameters: exam name, date
 \thispagestyle{empty}
 \markboth{\hspace{1cm}\subj\ #1\hspace{1in}Name\hrulefill\ \ }%
          {\subj\ #1\hspace{1in}Name\hrulefill\ \ }
 \pagestyle{myheadings}\examtitle{#1}{#2}
 \renewcommand{\theproblem}{Problem \arabic{problemnum}}
}
\newcommand{\examsolutions}[3]{% parameters: handout, exam name, date
 \thispagestyle{empty}
 \markboth{Handout \protect\ref{#1}: #2}{Handout \protect\ref{#1}: #2}
% \pagestyle{myheadings}\htitle{\protect\ref{#1}}{#2}{#3}
 \pagestyle{myheadings}\examsolutionstitle{\protect\ref{#1}} {#2}{#3}
 \renewcommand{\theproblem}{Problem \arabic{problemnum}}
}
\newcommand{\examsolutionstitle}[3]{\noindent\vspace*{-\toppush}\newline\parbox{6.5in}
{\textit{Design and Analysis of Algorithms}\hfill#3\newline
Massachusetts Institute of Technology \hfill 6.046\newline
%Singapore-MIT Alliance \hfill SMA5503\newline
\profs\hfill Handout #1\vspace*{-.5ex}\newline
\mbox{}\hrulefill\mbox{}}\vspace*{1ex}\mbox{}\newline
\begin{center}{\Large\bf #2}\end{center}}

\newcommand{\takehomeexam}[2]{% parameters: exam name, date
 \thispagestyle{empty}
 \markboth{\subj\ #1\hfill}{\subj\ #1\hfill}
 \pagestyle{myheadings}\examtitle{#1}{#2}
 \renewcommand{\theproblem}{Problem \arabic{problemnum}}
}

\makeatletter
\newcommand{\exambooklet}[2]{% parameters: exam name, date
 \thispagestyle{empty}
 \markboth{\subj\ #1}{\subj\ #1}
 \pagestyle{myheadings}\examtitle{#1}{#2}
 \renewcommand{\theproblem}{Problem \arabic{problemnum}}
 \renewcommand{\problem}{\newpage
 \item \let\@currentlabel=\theproblem
 \markboth{\subj\ #1, \theproblem}{\subj\ #1, \theproblem}}
}
\makeatother


\newcommand{\examtitle}[2]{\noindent\vspace*{-\toppush}\newline\parbox{6.5in}
{\textit{Design and Analysis of Algorithms}\hfill#2\newline
Massachusetts Institute of Technology \hfill 6.046 Spring 2019\newline
%Singapore-MIT Alliance \hfill SMA5503\newline
\profs\hfill #1\vspace*{-.5ex}\newline
\mbox{}\hrulefill\mbox{}}\vspace*{1ex}\mbox{}\newline
\begin{center}{\Large\bf #1}\end{center}}

\newcommand{\grader}[1]{\hspace{1cm}\textsf{\textbf{#1}}\hspace{1cm}}

\newcommand{\points}[1]{[#1 points]\ }
\newcommand{\parts}[1]
{
  \ifnum#1=1
  (1 part)
  \else
  (#1 parts)
  \fi
  \ 
}

\newcommand{\bparts}{\begin{problemparts}}
\newcommand{\eparts}{\end{problemparts}}
\newcommand{\ppart}{\problempart}

%\newcommand{\lg} {lg\ }

\setlength{\oddsidemargin}{0pt}
\setlength{\evensidemargin}{0pt}
\setlength{\textwidth}{6.5in}
\setlength{\topmargin}{0in}
\setlength{\textheight}{8.5in}


\newcommand{\Spawn}{{\bf spawn} }
\newcommand{\Sync}{{\bf sync}}

\newcommand{\cif}[1]{\mbox{if $#1$}}
\newcommand{\cwhen}[1]{\mbox{when $#1$}}

\newcounter{problemnum}
\newcommand{\theproblem}{Problem \theproblemsetnum-\arabic{problemnum}}
\newenvironment{problems}{
        \begin{list}{{\bf \theproblem. \hspace*{0.5em}}}
        {\setlength{\leftmargin}{0em}
         \setlength{\rightmargin}{0em}
         \setlength{\labelwidth}{0em}
         \setlength{\labelsep}{0em}
         \usecounter{problemnum}}}{\end{list}}
\makeatletter
\newcommand{\problem}[1][{}]{\item \let\@currentlabel=\theproblem \textbf{#1}}
\makeatother

\newcounter{problempartnum}[problemnum]
\newenvironment{problemparts}{
        \begin{list}{{\bf (\alph{problempartnum})}}
        {\setlength{\leftmargin}{2.5em}
         \setlength{\rightmargin}{2.5em}
         \setlength{\labelsep}{0.5em}}}{\end{list}}
\newcommand{\problempart}{\addtocounter{problempartnum}{1}\item}

\newenvironment{truefalseproblemparts}{
        \begin{list}{{\bf (\alph{problempartnum})\ \ \ T\ \ F\hfil}}
        {\setlength{\leftmargin}{4.5em}
         \setlength{\rightmargin}{2.5em}
         \setlength{\labelsep}{0.5em}
         \setlength{\labelwidth}{4.5em}}}{\end{list}}

\newcounter{exercisenum}
\newcommand{\theexercise}{Exercise \theproblemsetnum-\arabic{exercisenum}}
\newenvironment{exercises}{
        \begin{list}{{\bf \theexercise. \hspace*{0.5em}}}
        {\setlength{\leftmargin}{0em}
         \setlength{\rightmargin}{0em}
         \setlength{\labelwidth}{0em}
         \setlength{\labelsep}{0em}
        \usecounter{exercisenum}}}{\end{list}}
\makeatletter
\newcommand{\exercise}{\item \let\@currentlabel=\theexercise}
\makeatother

\newcounter{exercisepartnum}[exercisenum]
%\newcommand{\problem}[1]{\medskip\mbox{}\newline\noindent{\bf Problem #1.}\hspace*{1em}}
%\newcommand{\exercise}[1]{\medskip\mbox{}\newline\noindent{\bf Exercise #1.}\hspace*{1em}}

\newenvironment{exerciseparts}{
        \begin{list}{{\bf (\alph{exercisepartnum})}}
        {\setlength{\leftmargin}{2.5em}
         \setlength{\rightmargin}{2.5em}
         \setlength{\labelsep}{0.5em}}}{\end{list}}
\newcommand{\exercisepart}{\addtocounter{exercisepartnum}{1}\item}


% Macros to make captions print with small type and 'Figure xx' in bold.
\makeatletter
\def\fnum@figure{{\bf Figure \thefigure}}
\def\fnum@table{{\bf Table \thetable}}
\let\@mycaption\caption
%\long\def\@mycaption#1[#2]#3{\addcontentsline{\csname
%  ext@#1\endcsname}{#1}{\protect\numberline{\csname 
%  the#1\endcsname}{\ignorespaces #2}}\par
%  \begingroup
%    \@parboxrestore
%    \small
%    \@makecaption{\csname fnum@#1\endcsname}{\ignorespaces #3}\par
%  \endgroup}
%\def\mycaption{\refstepcounter\@captype \@dblarg{\@mycaption\@captype}}
%\makeatother
\let\mycaption\caption
%\newcommand{\figcaption}[1]{\mycaption[]{#1}}

\newcounter{totalcaptions}
\newcounter{totalart}

\newcommand{\figcaption}[1]{\addtocounter{totalcaptions}{1}\caption[]{#1}}

% \psfigures determines what to do for figures:
%       0 means just leave vertical space
%       1 means put a vertical rule and the figure name
%       2 means insert the PostScript version of the figure
%       3 means put the figure name flush left or right
\newcommand{\psfigures}{0}
\newcommand{\spacefigures}{\renewcommand{\psfigures}{0}}
\newcommand{\rulefigures}{\renewcommand{\psfigures}{1}}
\newcommand{\macfigures}{\renewcommand{\psfigures}{2}}
\newcommand{\namefigures}{\renewcommand{\psfigures}{3}}

\newcommand{\figpart}[1]{{\bf (#1)}\nolinebreak[2]\relax}
\newcommand{\figparts}[2]{{\bf (#1)--(#2)}\nolinebreak[2]\relax}


\macfigures     % STATE

% When calling \figspace, make sure to leave a blank line afterward!!
% \widefigspace is for figures that are more than 28pc wide.
\newlength{\halffigspace} \newlength{\wholefigspace}
\newlength{\figruleheight} \newlength{\figgap}
\newcommand{\setfiglengths}{\ifnum\psfigures=1\setlength{\figruleheight}{\hruleheight}\setlength{\figgap}{1em}\else\setlength{\figruleheight}{0pt}\setlength{\figgap}{0em}\fi}
\newcommand{\figspace}[2]{\ifnum\psfigures=0\leavefigspace{#1}\else%
\setfiglengths%
\setlength{\wholefigspace}{#1}\setlength{\halffigspace}{.5\wholefigspace}%
\rule[-\halffigspace]{\figruleheight}{\wholefigspace}\hspace{\figgap}#2\fi}
\newlength{\widefigspacewidth}
% Make \widefigspace put the figure flush right on the text page.
\newcommand{\widefigspace}[2]{
\ifnum\psfigures=0\leavefigspace{#1}\else%
\setfiglengths%
\setlength{\widefigspacewidth}{28pc}%
\addtolength{\widefigspacewidth}{-\figruleheight}%
\setlength{\wholefigspace}{#1}\setlength{\halffigspace}{.5\wholefigspace}%
\makebox[\widefigspacewidth][r]{#2\hspace{\figgap}}\rule[-\halffigspace]{\figruleheight}{\wholefigspace}\fi}
\newcommand{\leavefigspace}[1]{\setlength{\wholefigspace}{#1}\setlength{\halffigspace}{.5\wholefigspace}\rule[-\halffigspace]{0em}{\wholefigspace}}

% Commands for including figures with macpsfig.
% To use these commands, documentstyle ``macpsfig'' must be specified.
\newlength{\macfigfill}
\makeatother
\newlength{\bbx}
\newlength{\bby}
\newcommand{\macfigure}[5]{\addtocounter{totalart}{1}
\ifnum\psfigures=2%
\setlength{\bbx}{#2}\addtolength{\bbx}{#4}%
\setlength{\bby}{#3}\addtolength{\bby}{#5}%
\begin{flushleft}
\ifdim#4>28pc\setlength{\macfigfill}{#4}\addtolength{\macfigfill}{-28pc}\hspace*{-\macfigfill}\fi%
\mbox{\psfig{figure=./#1.ps,%
bbllx=#2,bblly=#3,bburx=\bbx,bbury=\bby}}
\end{flushleft}%
\else\ifdim#4>28pc\widefigspace{#5}{#1}\else\figspace{#5}{#1}\fi\fi}
\makeatletter

\newlength{\savearraycolsep}
\newcommand{\narrowarray}[1]{\setlength{\savearraycolsep}{\arraycolsep}\setlength{\arraycolsep}{#1\arraycolsep}}
\newcommand{\normalarray}{\setlength{\arraycolsep}{\savearraycolsep}}

\newcommand{\hint}{{\em Hint:\ }}

% Macros from /th/u/clr/mac.tex

\newcommand{\set}[1]{\left\{ #1 \right\}}
\newcommand{\abs}[1]{\left| #1\right|}
\newcommand{\card}[1]{\left| #1\right|}
\newcommand{\floor}[1]{\left\lfloor #1 \right\rfloor}
\newcommand{\ceil}[1]{\left\lceil #1 \right\rceil}
\newcommand{\ang}[1]{\ifmmode{\left\langle #1 \right\rangle}
   \else{$\left\langle${#1}$\right\rangle$}\fi}
        % the \if allows use outside mathmode,
        % but will swallow following space there!
\newcommand{\paren}[1]{\left( #1 \right)}
\newcommand{\bracket}[1]{\left[ #1 \right]}
\newcommand{\prob}[1]{\Pr\left\{ #1 \right\}}
\newcommand{\Var}{\mathop{\rm Var}\nolimits}
\newcommand{\expect}[1]{{\rm E}\left[ #1 \right]}
\newcommand{\expectsq}[1]{{\rm E}^2\left[ #1 \right]}
\newcommand{\variance}[1]{{\rm Var}\left[ #1 \right]}
\renewcommand{\choose}[2]{{{#1}\atopwithdelims(){#2}}}
\def\pmod#1{\allowbreak\mkern12mu({\rm mod}\,\,#1)}
\newcommand{\matx}[2]{\left(\begin{array}{*{#1}{c}}#2\end{array}\right)}
\newcommand{\Adj}{\mathop{\rm Adj}\nolimits}

\newtheorem{theorem}{Theorem}
\newtheorem{lemma}[theorem]{Lemma}
\newtheorem{corollary}[theorem]{Corollary}
\newtheorem{xample}{Example}
\newtheorem{definition}{Definition}
\newenvironment{example}{\begin{xample}\rm}{\end{xample}}
\newcommand{\proof}{\noindent{\em Proof.}\hspace{1em}}
\def\squarebox#1{\hbox to #1{\hfill\vbox to #1{\vfill}}}
\newcommand{\qedbox}{\vbox{\hrule\hbox{\vrule\squarebox{.667em}\vrule}\hrule}}
\newcommand{\qed}{\nopagebreak\mbox{}\hfill\qedbox\smallskip}
\newcommand{\eqnref}[1]{(\protect\ref{#1})}

%%\newcommand{\twodots}{\mathinner{\ldotp\ldotp}}
\newcommand{\transpose}{^{\mbox{\scriptsize \sf T}}}
\newcommand{\amortized}[1]{\widehat{#1}}

\newcommand{\punt}[1]{}

%%% command for putting definitions into boldface
% New style for defined terms, as of 2/23/88, redefined by THC.
\newcommand{\defn}[1]{{\boldmath\textit{\textbf{#1}}}}
\newcommand{\defi}[1]{{\textit{\textbf{#1\/}}}}

\newcommand{\red}{\leq_{\rm P}}
\newcommand{\lang}[1]{%
\ifmmode\mathord{\mathcode`-="702D\rm#1\mathcode`\-="2200}\else{\rm#1}\fi}

%\newcommand{\ckt}[1]{\ifmmode\mathord{\mathcode`-="702D\sc #1\mathcode`\-="2200}\else$\mathord{\mathcode`-="702D\sc #1\mathcode`\-="2200}$\fi}
\newcommand{\ckt}[1]{\ifmmode \sc #1\else$\sc #1$\fi}

%% Margin notes - use \notesfalse to turn off notes.
\setlength{\marginparwidth}{0.6in}
\reversemarginpar
\newif\ifnotes
\notestrue
\newcommand{\longnote}[1]{
  \ifnotes
    {\medskip\noindent Note: \marginpar[\hfill$\Longrightarrow$]
      {$\Longleftarrow$}{#1}\medskip}
  \fi}
\newcommand{\note}[1]{
  \ifnotes
    {\marginpar{\tiny \raggedright{#1}}}
  \fi}


\newcommand{\reals}{\mathbbm{R}}
\newcommand{\integers}{\mathbbm{Z}}
\newcommand{\naturals}{\mathbbm{N}}
\newcommand{\rationals}{\mathbbm{Q}}
\newcommand{\complex}{\mathbbm{C}}

\newcommand{\oldreals}{{\bf R}}
\newcommand{\oldintegers}{{\bf Z}}
\newcommand{\oldnaturals}{{\bf N}}
\newcommand{\oldrationals}{{\bf Q}}
\newcommand{\oldcomplex}{{\bf C}}

\newcommand{\w}{\omega}                 %% for fft chapter

\newenvironment{closeitemize}{\begin{list}
{$\bullet$}
{\setlength{\itemsep}{-0.2\baselineskip}
\setlength{\topsep}{0.2\baselineskip}
\setlength{\parskip}{0pt}}}
{\end{list}}

% These are necessary within a {problems} environment in order to restore
% the default separation between bullets and items.
\newenvironment{normalitemize}{\setlength{\labelsep}{0.5em}\begin{itemize}}
                              {\end{itemize}}
\newenvironment{normalenumerate}{\setlength{\labelsep}{0.5em}\begin{enumerate}}
                                {\end{enumerate}}

%\def\eqref#1{Equation~(\ref{eq:#1})}
%\newcommand{\eqref}[1]{Equation (\ref{eq:#1})}
\newcommand{\eqreftwo}[2]{Equations (\ref{eq:#1}) and~(\ref{eq:#2})}
\newcommand{\ineqref}[1]{Inequality~(\ref{ineq:#1})}
\newcommand{\ineqreftwo}[2]{Inequalities (\ref{ineq:#1}) and~(\ref{ineq:#2})}

\newcommand{\figref}[1]{Figure~\ref{fig:#1}}
\newcommand{\figreftwo}[2]{Figures \ref{fig:#1} and~\ref{fig:#2}}

\newcommand{\liref}[1]{line~\ref{li:#1}}
\newcommand{\Liref}[1]{Line~\ref{li:#1}}
\newcommand{\lirefs}[2]{lines \ref{li:#1}--\ref{li:#2}}
\newcommand{\Lirefs}[2]{Lines \ref{li:#1}--\ref{li:#2}}
\newcommand{\lireftwo}[2]{lines \ref{li:#1} and~\ref{li:#2}}
\newcommand{\lirefthree}[3]{lines \ref{li:#1}, \ref{li:#2}, and~\ref{li:#3}}

\newcommand{\lemlabel}[1]{\label{lem:#1}}
\newcommand{\lemref}[1]{Lemma~\ref{lem:#1}} 

\newcommand{\exref}[1]{Exercise~\ref{ex:#1}}

\newcommand{\handref}[1]{Handout~\ref{#1}}

\newcommand{\defref}[1]{Definition~\ref{def:#1}}

% (1997.8.16: Victor Luchangco)
% Modified \hlabel to only get date and to use handouts counter for number.
%   New \handout and \handoutwithouttitle commands in newmac.tex use this.
%   The date is referenced by <label>-date.
%   (Retained old definition as \hlabelold.)
%   Defined \hforcelabel to use an argument instead of the handouts counter.

\newcounter{handouts}
\setcounter{handouts}{0}

\newcommand{\hlabel}[2]{%
\stepcounter{handouts}
{\edef\next{\write\@auxout{\string\newlabel{#1}{{\arabic{handouts}}{0}}}}\next}
\write\@auxout{\string\newlabel{#1-date}{{#2}{0}}}
}

\newcommand{\hforcelabel}[3]{%          Does not step handouts counter.
\write\@auxout{\string\newlabel{#1}{{#2}{0}}}
\write\@auxout{\string\newlabel{#1-date}{{#3}{0}}}}


% less ugly underscore
% --juang, 2008 oct 05
\renewcommand{\_}{\vrule height 0 pt depth 0.4 pt width 0.5 em \,}

\newcommand{\theproblemsetnum}{1}
\newcommand{\releasedate}{Thursday, February 7}
\newcommand{\partaduedate}{Wednesday, February 13}

\title{6.046 Problem Set \theproblemsetnum}

\begin{document}

\handout{Problem Set \theproblemsetnum}{\releasedate}
\textbf{All parts are due {\bf \partaduedate} at {\bf 10PM}}.

\setlength{\parindent}{0pt}
\medskip\hrulefill\medskip

{\bf Name:} Robert Durfee

\medskip

{\bf Collaborators:} None

\medskip\hrulefill

\begin{problems}

\problem  % Problem 1

\begin{problemparts}

\problempart % Problem 1a

We examine the recurrence,
$$ T(n) = 5 T\left(\frac{n}{2}\right) + n^3 $$

With the standard notation used in the Master Theorem,  $a = 5$, $b = 2$, and 
$f(n) = n^3$. Using these values, we consider each case of the Master Theorem.

\textbf{Case I}: Is $n^3 \in O\left(n^{\log_2 5 - \varepsilon}\right)$ for some 
$\varepsilon > 0$? Since $\log_2 5 \approx 2.3$, there is no $\varepsilon$ that
can satisfy this equality. As a result, Case I of the Master Theorem does not
apply.

\textbf{Case II}: Is $n^3 \in \Theta\left(n^{\log_2 5}\right)$? Since we already
established that $\log_2 5 \approx 2.3 \neq 3$, this equality is not satisfied.

\textbf{Case III}: Is $n^3 \in \Omega\left(n^{\log_2 5 + \varepsilon}\right)$
for some $\varepsilon > 0$ and $5 f(n / 2) \leq c f(n)$ for some $c < 1$? Since
$\log_2 5 \approx 2.3$, the first condition is satisfied with $\varepsilon <
0.7$. Furthermore, evaluating the second condition yields $5 n^3 / 8 \leq c 
n^3$ which is true for $5/8 \leq c < 1$. Therefore, Case III of the Master 
Theorem applies.

By Case III of the Master Theorem,
$$ T(n) \in \Theta\left(n^3\right) $$
 
\problempart % Problem 1b

We examine the recurrence,
$$ T(n) = 3 T\left(\frac{n}{3}\right) + n $$

With the standard notation used in the Master Theorem, $a = b = 3$ and $f(n) 
= n$. Using these values, we consider, each case of the Master Theorem.

\textbf{Case I}: Is $n \in O\left(n^{\log_3 3 - \varepsilon}\right)$ for some
$\varepsilon > 0$? Since $\log_3 3 = 1$, there is no $\varepsilon > 0$ that can
satisfy this equation.

\textbf{Case II}: Is $n \in \Theta\left(n^{\log_3 3}\right)$? Since $\log_3 3 =
1$, this is true. As a result, Case II of the Master Theorem applies.

By Case II of the Master Theorem,
$$ T(n) \in \Theta\left(n \log n\right) $$

\problempart % Problem 1c

We examine the recurrence,
$$ T(n) = T(n - 3) + 2n $$

Since this does not follow the standard framework presented in the Master
Theorem, we construct a tree to come up with a guess, which we verify using the
substitution method.

There is no branching factor in this recurrence. As a result, we only have to
consider a chain of a certain length. Since the input decreases by $3$ after
each recursive call, the total number of calls is $n / 3$ (assuming $n$ is a
multiple of $3$ for simplicity). At the $k$th call, there is $2(n - 3k)$ work is
required. As a result, the total work can be written as
$$ T(n) = \sum_{k = 0}^{n / 3} 2(n - 3k) $$
Breaking apart the sum,
$$ T(n) = 2n \sum_{k = 0}^{n / 3} 1 - 6 \sum_{k = 0}^{n / 3} k $$
Which further simplifies to,
$$ T(n) = \frac{2n^2}{3} - n \left(\frac{n}{3} + 1\right) $$
From this, we can conclude that $T(n) \in \Theta\left(n^2\right)$.

Now we confirm using induction. The induction hypothesis to confirm the upper
bound is $P(n)$: $T(n) \leq c n^2$ for some $c > 0$ and $\forall n \geq n_0$.

\textbf{Base Case}: Consider when $n = 1$. From the problem definition, we assume
$T(n) \in O(1)$ for any $n < 3$. Therefore, by definition, $T(1) \leq c$ for any
constant $c$.

\textbf{Inductive Step}: Assume via strong induction that $P(m)$ holds for all
$m < n$. Now consider an arbitrary $n > n_0$. We can simplify the recurrence
using this assumption as
$$ T(n) = T(n - 3) + 2n \leq c (n - 3)^2 + 2n $$
Expanding the power,
$$ T(n) \leq c (n^2 - 6n + 9) + 2n $$
Now we show this satisfies $P(n)$,
\begin{align*}
    c (n^2 - 6n + 9) + 2n &\leq c n^2 \\
    c(-6n + 9) + 2n &\leq 0 \\
    n(-6c + 2) + 9c &\leq 0
\end{align*}
Therefore, if we choose $c > 1/3$, this condition will hold for sufficiently
large $n$.

By induction, this shows that $T(n) \in O(n^2)$. 

\problempart % Problem 1d

We examine the recurrence,
$$ T(n) = 5 T(\sqrt[2]{n}) + \log \log n $$

First, we set $n = 2^m$. The recurrence becomes,
$$ T(2^m) = 5 T(2^{m / 2}) + \log m $$
We assume that $T(2^m) = S(m)$. This new recurrence is
$$ S(m) = 5 S(m / 2) + \log m $$
Now we can use the Master Theorem. 

\textbf{Case 1}: Is $\log m \in O(n^{\log_2 5 - \varepsilon})$ for some 
$\varepsilon > 0$? It is well known that logarithms grow slower than polynomials.
That is,
$$ \forall \alpha, k > 0,\quad \left(\log n\right)^k \in O\left(n^\alpha\right) $$
Thus, for any $\varepsilon < \log_2 5$, this case will hold.  As a result, the
solution to the recurrence $S(m)$ is
$$ S(m) \in \Theta\left(m^{\log_2 5}\right) $$

Now we substitute $n = 2^m$ to get the solution to recurrence $T(n)$,
$$ T(n) \in \Theta\left((\log n)^{\log_2 5}\right) $$

\problempart % Problem 1e

We examine the recurrence,
$$ T(n) = 3 T(n / 5) + 2 T(n/7) + \Theta(n) $$

We construct a recurrence tree.
\begin{itemize}
    \item The work in the root node is
        $$c n $$
    \item The work in the first level is given by
        $$c \left(\frac{3n}{5} + \frac{2n}{7}\right) = \frac{31c n}{35} $$
    \item The work in the second level is given by
        $$c \left(\frac{9n}{25} + \frac{12n}{35} + \frac{4}{49}\right) = 
        \frac{961 c n}{1225} $$
\end{itemize}

Now it is easy to see that the work per level is given by
$$ c n \left(\frac{31}{35}\right)^\ell $$
Since the work per level is decreasing geometrically, the root node dictates
the total work. Therefore,
$$ T(n) \in \Theta(n) $$

\problempart % Problem 1f
  \begin{enumerate}
    \item The ranking for the provided functions goes as follows:

    $$ \log \log n^{10} \in o(\log n) \in \Theta(\log_7 n) \in \Theta(\log n^3) 
    \in o(\log^3 n) \in o(n^{1/3}) $$
    
    This results from the following limits:
    
    $$\lim_{n \to \infty} \frac{\log \log n^{10}}{\log n} = \lim_{n \to \infty} 
        \frac{10n}{n \log(n^{10})} = \lim_{n \to \infty} \frac{10}{\log (n^{10})} = 0 $$ 
    $$\lim_{n \to \infty} \frac{\log n}{\log_7 n} = \lim_{n \to \infty} \frac{n 
        \log 7}{n \log 2} = \log 7 $$
    $$\lim_{n \to \infty} \frac{\log_7 n}{\log n^3} = \lim_{n \to \infty} 
        \frac{n \log 2}{3 n \log 7} = \frac{1}{3} \log_7 2 $$
    $$\lim_{n \to \infty} \frac{\log n^3}{\log^3 n} = \lim_{n \to \infty} 
        \frac{3 n \log^3 2}{3 n \log 2 \log^2 n} = \lim_{n \to \infty} \frac{
        \log^2 2}{\log^2 n}= 0 $$ 
        
    And from the well-known theorem that shows polynomials always grow faster
    than logarithms,
    $$ \log^3 n \in o(n^{1/3}) $$
    
    \item The ranking goes as follows:
    
    $$ n \log n^2 \in \Theta(\log(n!)) \in o(n^{5/3}) \in o(n^2 \log n) < 2^n $$
    
    This results from the following limits:
    
    \begin{align*}
        \lim_{n \to \infty} \frac{n \log n^2}{\log(n!)} &\approx \lim_{n \to \infty}
        \frac{n \log n^2}{\log \left(\sqrt{2 \pi n}
        \left(\frac{n}{e}\right)^n\right)} = \lim_{n \to \infty} \frac{ 2n (\log
        n^2 + 2)}{2n \log n + 1} \\
        &= \lim_{n \to \infty} \frac{\log n^2 + 4}
        {\log n + 1} = \lim_{n \to \infty} \frac{2n}{n} = 2
    \end{align*}
    \begin{align*}
        \lim_{n \to \infty} \frac{\log(n!)}{n^{5/3}} &\approx \lim_{n \to \infty} 
        \frac{\log \left(\sqrt{2 \pi n} \left(\frac{n}{e}\right)^n\right)}{n^{5/3}}
        = \lim_{n \to \infty} \frac{3 (2n \log n + 1)}{10n^{5/3}} \\
        &= \lim_{n \to \infty}
        \frac{18(\log n + 1)}{50 n^{2/3}} = \lim_{n \to \infty} \frac{54 n^{1/3}}{100n}= 0 
    \end{align*}
    $$\lim_{n \to \infty} \frac{n^{5/3}}{n^2 \log n} = \lim_{n \to \infty} 
        \frac{5 n^{2/3}}{3(n + 2n \log n)} = \lim_{n \to \infty} \frac{10}{3 n^{1/3} 
        (6 \log n + 9)}= 0 $$
        
    Lastly, since the $2^n$ term will always exist in the denominator,
    $$\lim_{n \to \infty} \frac{n^2 \log n}{2^n} = 0 $$
    
    \item The ranking goes as follows:
    
    $$ n^{10 \log n} \in o(e^n) \in o\left((\log n)^{n - 5}\right) \in 
    o\left((\log n)^n\right) $$
    
    This results from the following limits:
    
    $$\lim_{n \to \infty} \frac{n^{10 \log n}}{e^n} = \lim_{n \to \infty} \frac{10 
        \log^2 n}{n} = 0 $$
    $$\lim_{n \to \infty} \frac{e^n}{\left(\log n\right)^{n}} = \lim_{n \to \infty}
        \frac{n}{\log\log^n n} = \lim_{n \to \infty} \frac{1}{\log\log n + 1 / \log n} = 0 $$
    
    Since the both are growing logarithmically, but one is growing just a little slower,
    $$\lim_{n \to \infty} \frac{\left(\log n\right)^{n - 5}}{\left(\log n \right)^n} = 0 $$
    
  \end{enumerate}

\end{problemparts}

\newpage
\problem  % Problem 2

\begin{problemparts}

\problempart % Problem 2a

\textbf{Description} First, guess that all $n$ board members will vote `yes'. 
Record the number of incorrect guesses (the total number of `no' votes) as $v_1$.
Next, only within the subset, $S$, you wish to determine the number of `no'
votes, switch your guess to `no'. Record the number of incorrect guesses as 
$v_2$. The number of `no' votes in the desired subset is given by
$$ \frac{k - (v_2 - v_1)}{2} $$

\textbf{Correctness} From the first query, where you guess all members will vote
`yes', you will get a count of all the `no' voters on the board. In the second
query, you only change your guesses for the partition of interest. Therefore,
the difference in errors $\delta = v_2 - v_1$ tells you how many more `yes' 
votes ($y$) than `no' votes ($n$) exist solely in the partition of interest. 
Since we also know that the total number of votes is $k$, we have a system of 
two equations:
$$ y = \delta + n $$
$$ k = y + n $$
Solving for $n$ yields the following,
$$ \frac{k - \delta}{2} = \frac{k - (v_2 - v_1)}{2} = n $$

\textbf{Running Time} Only two queries are executed. Therefore, this satisfies
the requirement for a constant number of queries.

\problempart % Problem 2b

\textbf{Description} This algorithm implements a modified binary search. Divide
the board members into two equal groups. Query each group to get a count of the
number of `no' votes using Part A. If neither group has any `no' votes, return 
\texttt{None}. Otherwise, choose the group with the most incorrect guesses 
(breaking ties arbitrarily) and recurse on this group discarding the other 
group. Continue until there is only one voter in the partition. If this voter
votes `yes', return \texttt{None}. If this voter votes `no', return the index
of that member.

\textbf{Correctness} By using the algorithm in Part A, you will get an accurate
count of all the `no' voters in a particular partition. By choosing the
partition with the greatest `no' votes (and ignoring partitions with zero `no'
votes), we assure that the section we recurse on will always have a `no' voter
to find. Once there is only one member in our partition, by induction, we know
that, if there is `no' voter to discover, it must be at this index. If not,
there are no `no' voters to discover and \texttt{None} can safely be returned.

\textbf{Running Time} The following recurrence describes the algorithm
$$ T(n) = T(n/2) + O(1) $$
We know that the amount of work to divide/combine is $O(1)$ from Part A. Each
time we recurse, we discard one half of the board members. Thus the problem
size decreases by a factor of $2$ with a branching factor of $1$. 

This recurrence can be solved using telescoping, assuming $T(1) = O(1) = 1$
\begin{align*}
    T(n) &= T(n / 2) + 1 \\
    T(n / 2) &= T(n / 4) + 1 \\
    &\vdots \\
    T(4) &= T(2) + 1 \\
    T(2) &= T(1) + 1
\end{align*}
Summing these equations,
$$ T(n) + T(n / 2) + \ldots + T(4) + T(2) = T(n / 2) + T(n / 4) + \ldots +
T(2) + T(1) + (1 + \ldots) $$
Given the branching factor of the recurrence tree, the number of leaves is
$\log n$. Thus, there are $\log n$ $1$'s. After cancelling terms,
$$ T(n) = 1 + \log n \in O(\log n)$$

In short, this is the same recurrence as Binary Search,
$$ T(n) \in O(\log n) $$

\problempart % Problem 2c

\textbf{Assumptions} From the wording of the problem, it appears that each of
the board members has a positive, integer weight given by $w_i$. These votes
satisfy the condition,
$$ \sum_{i = 0}^{n} w_i = W $$
This assumption will be used throughout the remainder of the algorithm. However,
this method should be robust enough to handle real-valued (including negative)
weights as long as the condition for a majority decision remains $W / 2$.
Nonetheless, we will maintain the assumption of positive, integral weights.

\textbf{Description} If there is only a single member, $x_1$, their ideal
salary for Melon, $s_1$, is the highest salary Melon can propose. If there are
only two members, $x_1$ and $x_2$, the ideal salary from member with more votes,
WLOG $s_1$, is the highest salary Melon can propose. 

If there are more than two members, use the \texttt{select} algorithm provided
in lecture to find the median salary, $s_m$, among $S = s_1, s_2, \ldots, s_n$.
Use the modified \texttt{partition} algorithm to partition the voters into
groups $L$ and $G$ such that $L = \{x_i \in X \mid s_i < s_m\}$ and $G = \{x_i
\in X \mid s_i >  s_m\}$. Calculate the weight of each group as $W_L =
\sum_{x_i \in L} w_i$ and $W_G = \sum_{x_i \in G} w_i$. If $W_L,\ W_G < W / 2$
$s_m$ is the highest salary Melon can propose. If $W_L > W / 2$, set the weight
$w_m = w_m + W_G$ and recurse on $L \cup \{x_m\}$. If $W_G > W / 2$, set the
weight $w_m = w_m + W_L$ and recurse on $\{x_m\} \cup G$.

\textbf{Correctness} The base cases are trivial. If there is only one voter,
Melon must propose to satisfy that one voter. If there are only two voters,
Melon must satisfy the one who has more votes.

When there are more than two voters, the median weighted ideal salary is given
by the ideal salary that has less than $W / 2$ weighted votes on either side.
If the scale is tipped in the direction of lower salaries, that is, more than
$W / 2$ is located $s < s_m$, then the weighted median must be less than $s_m$
as voters with more weight want a lower salary. If the scale is tipped in the
direction of higher, salaries, that is, more than $W / 2$ is located $s > s_m$,
then the weighted median must be greater than $s_m$ as voters with more weight
want a high salary. The adjustment of adding the weights of the ignored group
ensure weighting remains consistent when examining smaller partitions.

\textbf{Running Time} The recurrence that dictates this algorithm is given by
$$ T(n) = T(n / 2) + O(n) $$
The $T(n / 2)$ comes from the reduction of the problem by half upon each
recursion as we ignore the lower weighted half each time. At each recursion, we
first find the median of $n$ values. Using the algorithm \texttt{select}
proved in lecture, this must be $O(n)$. Then, we need to sum the weights of
each partition. This can be done trivially in $O(n)$ also.

Solving by Master Theorem. Case III of the Master theorem applies as $ O(n) 
\in \Omega(n^{\varepsilon}) $ for $\varepsilon \in (0, 1]$. Furthermore,
$n / 2 \leq c n$ for $c \geq 1/2$. Thus,
$$T(n) \in \Theta(n) $$

\end{problemparts}

\end{problems}

\end{document}


