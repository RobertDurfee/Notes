%
% 6.046 problem set 2 solutions
%
\documentclass[12pt,twoside]{article}
\usepackage{multirow}
\usepackage{algorithm, algpseudocode, float}

\input{macros}
\usepackage{enumitem}
\newcommand{\theproblemsetnum}{7}
\newcommand{\releasedate}{Thursday, April 4}
\newcommand{\partaduedate}{Wednesday, April 10}
\allowdisplaybreaks

\title{6.046 Problem Set \theproblemsetnum}

\begin{document}

\handout{Problem Set \theproblemsetnum}{\releasedate}
\textbf{All parts are due {\bf \partaduedate} at {\bf 10PM}}.

\makeatletter
\newenvironment{breakablealgorithm}
  {% \begin{breakablealgorithm}
   \begin{center}
     \refstepcounter{algorithm}% New algorithm
     \hrule height.8pt depth0pt \kern2pt% \@fs@pre for \@fs@ruled
     \renewcommand{\caption}[2][\relax]{% Make a new \caption
       {\raggedright\textbf{\ALG@name~\thealgorithm} ##2\par}%
       \ifx\relax##1\relax % #1 is \relax
         \addcontentsline{loa}{algorithm}{\protect\numberline{\thealgorithm}##2}%
       \else % #1 is not \relax
         \addcontentsline{loa}{algorithm}{\protect\numberline{\thealgorithm}##1}%
       \fi
       \kern2pt\hrule\kern2pt
     }
  }{% \end{breakablealgorithm}
     \kern2pt\hrule\relax% \@fs@post for \@fs@ruled
   \end{center}
  }
\makeatother

\setlength{\parindent}{0pt}
\medskip\hrulefill\medskip

{\bf Name:} Robert Durfee

\medskip

{\bf Collaborators:} None

\medskip\hrulefill

\begin{problems}

\problem  % Problem 1

\begin{problemparts}

\problempart % Problem 1a

\begin{enumerate}[label=\textbf{(\roman*)}]

  \item We first show that if $G$ has no cycles, then Ben can assign each of
  the $n$ items to the slot of one of its hashed values without collisions.

  Suppose Ben has an item $x$ and his first and second hash functions each
  yield a previously alotted slot $a_i$ and $a_j$, respectively. If $i = j$,
  this would yield a self-loop. Since cycles are not present in $G$, this
  case is not possible. If $i \neq j$, this implies either of the following:
  \begin{itemize}
    \item There exists a single edge between $a_i$ and $a_j$.
    \item There exist two edges, one with endpoint $a_i$ and another with
    endpoint $a_j$.
  \end{itemize}
  Since there are no cycles, the first case is not possible as this would
  lead to two parallel edges between $a_i$ and $a_j$.

  Now we examine the second case. Note that if we can successfully reassign
  the item in either $a_i$ or $a_j$ without collision, we can resolve the
  conflict of $x$. Therefore, without loss of generality, consider $a_i$
  only. As determined previously, for endpoint $a_i$, there must be an edge
  with another endpoint $a_k$ such that $k \neq j$. If there is no element in
  $a_k$, we can reassign the element in $a_i$ into $a_k$. If there is an
  element in $a_k$, repeat until there is some slot that has not been
  alotted. Since there are no cycles, there must eventually be a slot that
  hasn't been alotted along this chain as $a_j$ cannot be along the chain
  without a cycle forming. Therefore, if there are no cycles, a hashing can
  be determined without conflicts.

  \item We now demonstrate an algorithm to check for cycles and, if none are
  present, we assign each of the $n$ items to one of the slots without
  collisions.

  {\bf Description} Using the two provided hash functions $h_1$ and $h_2$,
  construct the vertices of the graph $G$ such that,
  $$V = \{ h(x) \mid \forall h \in \{h_1, h_2\}\ \mathrm{and}\ \forall x \in
  \{x_1, \ldots, x_n\}\}$$
  For all $u, v \in V$, connect vertex $u$ to $v$ with an undirected edge if
  the following is true,
  $$ \exists x . h_1(x) = u\ \mathrm{and}\ h_2(x) = v $$
  Let these edges make up the set $E$ for graph $G$.
  
  Now we will run a modified depth-first search on the graph $G$. If when
  visiting a node $v$, an adjacent node $u$ (which is not the parent $p$) has
  been visited before, return that a cycle has been detected. If a cycle
  isn't detected, continue visiting all nodes $u$ adjacent to $v$ in a
  depth-first manner as in 6.006. After visiting all nodes $u$ adjacent to
  $v$, use the edge between the parent $p$ and the current note $v$ to
  identify the corresponding $x_i$ such that $h_1(x_i) = p$ and $h_2(x_i) =
  v$ (or vice versa). Without loss of generality, assume $h_2(x_i) = v$, then
  assign $x_i$ to $v$ using hash function $h_2$. Continue this process
  recursively until the graph has been completely iterated over and return
  the mapping (if a cycle isn't detected before).

  {\bf Correctness} If there is a cycle, there must a vertice that when
  explored depth-first, one of its adjacent vertices is visited indirectly
  prior to being directly checked in the depth-first search procedure. As a
  result, this algorithm will correctly identify cycles. 
  
  The mapping will also be accurate for a graph with no cycles as depth-first
  search will construct a spanning forest. Since each spanning tree will have
  $\ell$ vertices and $\ell - 1$ edges, there will always be a way to assign
  the $\ell - 1$ items to $\ell$ slots. The simplest way to do this is to
  assign from the bottom of the tree's leaves up to the root which will
  ensure no conflicts. This is the order in which we assign the hash
  functions in the modified depth-first search procedure.

  {\bf Running Time} To construct the graph vertices $V$ of graph $G$, $O(n)$
  time is necessary as there are $n$ elements of which to compute hashes.
  Since there are only $n$ edges, it also takes $O(n)$ time to construct the
  edges of the graph.

  Running depth-first search on a graph with $n$ vertices and $n$ edges will
  take $O(n + n)$ time. As a result, the overall runtime is $O(n)$.

\end{enumerate}

\problempart % Problem 1b

Consider the probability of choosing a single edge $e$. Let this edge be
represented by an ordered tuple $(a, b)$. From the problem statement,
$h_1(x) = a$ and $h_2(x) = b$. Furthermore, there are $m$ possible
values for both $h_1(x)$ and $h_2(x)$. Thus, the number of different
$(a, b)$ pairs is $m^2$. However, an edge $(b, a)$ is equivalent to
$(a, b)$, so we divide the possible number of edges by $2$. Thus there
are $m^2 / 2$ possible edges. Also, an edge $(h_1(x), h_2(x)) = (a, b)$ can
be formed using any of $n$ different $x$ values. As a result, we further
divide the number of edges by $n$. Therefore, there are $m^2 / 2n$ total
edges to choose. Since each are equally likely, the probability of choosing
one is $2n / m^2$.

A $k$-length cycle is made up of $k$ {\it distinct} edges. Therefore, the
probability of choosing a single $k$-length cycle is given by,
$$ \left(\frac{2n}{m^2}\right) \left(\frac{2n}{m^2} - 1\right) \cdots
\left(\frac{2n}{m^2} - (k - 1)\right) $$
This is less than $\left(2n / m^2\right)^k$. Therefore, the probability of a
given $k$-length cycle appearing in $G$ is at most $\left(2n / m^2\right)^k$.

\problempart % Problem 1c

Consider how many $k$-length cycles are possible in a graph with $m$
vertices. Since a $k$-length cycle has $k - 1$ vertices, there are $k - 1$
distinct vertices to choose. There are $m$ choices for the first vertex, $m -
1$ choices for the second, and so on until $m - (k - 2)$ choices for the $k -
1$th vertex. Therefore, in total, the number of $k$-length cycles is,
$$ (m) (m - 1) \cdots (m - (k - 2)) $$
This is less than $m^k$. Therefore, the probability of any $k$-length cycles
appearing in $G$ is at most,
$$ m^k \cdot \left(\frac{2n}{m^2}\right)^k = \left(\frac{2n}{m}\right)^k $$

\problempart % Problem 1d

The probability of a cycle of length 1 or greater is given by the sum,
$$ \sum_{k = 1}^\infty \left(\frac{2n}{m}\right)^k $$
If $m = 10n$, then this reduces to,
$$ \sum_{k = 1}^\infty \left(\frac{2}{10}\right)^k $$
Using the geometric series formula,
$$ \sum_{k = 1}^\infty \left(\frac{2}{10}\right)^k = \frac{2/10}{1 - 2/10} =
\frac{1}{4} $$
This is less than $1/2$ therefore Ben can assign $n$ items to $10n$ slots
using two hash functions with probability greater than $1/2$.

\problempart % Problem 1e

{\bf Description} Generate two hash functions $h_1$ and $h_2$. Run the
algorithm in Part A with the provided $n$ values with $m = 10n$. If a cycle
is detected, run the algorithm again and continue until no cycles are
detected and a valid map is returned.

{\bf Correctness} The algorithm in Part A will successfully assign a mapping
of $n$ items into $10n$ slots using two hash functions or return a cycle was
detected. This is proved above. By repeating, we will eventually get a
correct mapping as the probability argued in Part D assures us there exists a
pair of hash functions that lead to a graph with no cycles.

{\bf Running Time} The running time of this algorithm is
$\mathbb{E}[\mathrm{iterations}] \cdot O(n)$. Since we showed that the
probability of getting a valid mapping on the first try is greater than
$1/2$, we have the following inequality.
$$ \mathbb{E}[\mathrm{iterations}] \leq \frac{1}{2} +
\frac{1}{2}\left(\mathbb{E}[\mathrm{iterations}] + 1\right) $$
Solving for $\mathbb{E}[\mathrm{iterations}] \leq 2 $. Therefore, the overall
runtime of the algorithm is $O(n)$.

\end{problemparts}

\newpage
\problem  % Problem 2

\begin{problemparts}

\problempart % Problem 2a

{\bf Description} Using a union-find data struture, for each vertices $v \in
V$, call a \textproc{Make-Set} operation. For each edge $e_k$ between vertices
$u$ and $v$, perform a \textproc{Find-Set} operation. If \textproc{Find-Set}
determines that both $u$ and $v$ belong to the same set, return that a cycle
has been detected. If $u$ and $v$ do not belong to the same set, perform a
\textproc{Union} operation.

{\bf Correctness} The standard union-find data structure allows us to keep
track of connected components. Whenever an edge has endpoints both within the
same connected component, a cycle must have been discovered as long as the
edges in the stream are all unique and there are no self-loops. In this case,
we are told the edges are all unique and it is safe to assume that someone is
not an enemy to themselves.

{\bf Space Complexity} The union-find data structure forms a forest of
elements internally. There are always $|V|$ elements in the forest. 

A forest with $n$ vertices and $m$ components will $n - m$ edges. This can be
seen since each tree $i$ has $\ell_i - 1$ edges. Therefore,
$$ \sum_{i = 1}^m (\ell_i - 1) = \sum_{i = 1}^m \ell_i - \sum_{i = 1}^m = n -
m $$

Therefore, in the worst case, there are $|V| - 1$ edges and $|V|$ vertices.
Therefore, the space complexity is $O(|V|)$.

{\bf Time Complexity} By using the union-by-rank or path compression
union-find data structure, all operations are amortized $O(\log |V|)$.
Therefore, in the worst case, if all $|E_k|$ edges are examined, the run time
is $O(|E_k| \log |V|)$.

\problempart % Problem 2b

{\bf Description} Using a modified union-find data structure. Initialize the
data structure by calling \textproc{Make-Set} for all vertices. For each
vertex, keep track of the pairity of the element to it's representative. For
\textproc{Make-Set}, initialize all pairities of the roots to $0$. For each
edge $e_k$ between vertices $u$ and $v$, perform a \textproc{Find-Set}
operation. Along the recursion, perform path compression. Update the
pairities during path compression using the xor operation. If $u$ and $v$
belong to different sets, perform the union of the two sets using
\textproc{Union}. Compute the pairity of $u$ to its representative, let it be
$x$. Compute the pairity of $v$ to it's representative, let it be $y$. Update
the representative's pairity with
$$ x \oplus y \oplus 1 $$
If $u$ and $v$ belong to the same set and have the same pairities to the
representative, there is an odd-length cycle.

{\bf Correctness} In each component of the union-find data structure, we wish
to maintain the pairity from an element to a common element. To initialize,
the pairity to the representative from the representative is zero. If two
elements are in the same component, if their pairities are both even, adding
a new edge would make the pairity odd. If their pairities are both odd,
adding a new edge would make the pairity odd.

If two elements are in different components, we need a way to update the old
representative. Since we are adding a single edge, the pairity needs to be
the pairity of the first element to its representative with the pairity of
the second element to its representative with one additional. This is
accomplished with the expression,
$$ x \oplus y \oplus 1 $$

Therefore, we can maintain proper pairities and whenever two pairites in the
same component are equal and connected, the result must be an odd cycle.

{\bf Space Complexity} The union-find data structure forms a forest of
elements internally. There are always $|V|$ elements in the forest. 

A forest with $n$ vertices and $m$ components will $n - m$ edges. This can be
seen since each tree $i$ has $\ell_i - 1$ edges. Therefore,
$$ \sum_{i = 1}^m (\ell_i - 1) = \sum_{i = 1}^m \ell_i - \sum_{i = 1}^m = n -
m $$

Therefore, in the worst case, there are $|V| - 1$ edges and $|V|$ vertices.
For each vertex, we only store constant extra space. Therefore, the space
complexity is $O(|V|)$.

{\bf Time Complexity} By using the path compression union-find data
structure, all operations are amortized $O(\log |V|)$. Therefore, in the
worst case, for all $|E|$ edges examined, the run time is $O(|E| \log |V|)$.

\problempart % Problem 2c

The running time requirement is already fulfilled in Part B.

\end{problemparts}

\end{problems}

\end{document}


