%
% 6.046 problem set 2 solutions
%
\documentclass[12pt,twoside]{article}
\usepackage{multirow}

\newcommand{\name}{}

\usepackage{amssymb}
\usepackage{amsmath}
\usepackage{graphicx}
\usepackage{latexsym}
\usepackage{times,url}
\usepackage{cprotect}
\usepackage{listings}
\usepackage{graphicx}
\usepackage[table]{xcolor}
\usepackage[letterpaper]{geometry}
\usepackage{tikz-qtree}
\usepackage{enumerate}

\newcommand{\profs}{Mauricio Karchmer, Aleksander Madry, Bruce Tidor}
\newcommand{\subj}{6.046}
\newcommand{\ttt}[1]{{\tt\small #1}}

\definecolor{dkgreen}{rgb}{0,0.6,0}
\definecolor{gray}{rgb}{0.5,0.5,0.5}
\definecolor{mauve}{rgb}{0.58,0,0.82}

\lstset{
  language=Python,
  aboveskip=1pc,
  belowskip=1pc,
  basicstyle={\footnotesize\ttfamily},
  numbers=left,
  showstringspaces=false,
  numberstyle=\tiny\color{gray},
  keywordstyle=\color{blue},
  commentstyle=\color{dkgreen},
  stringstyle=\color{mauve},
}

\tikzset{
  % every node/.style={minimum width=2em,draw,circle},
  % level 1/.style={sibling distance=2cm},
  level distance=1cm,
  edge from parent/.style=
  {draw,edge from parent path={(\tikzparentnode) -- (\tikzchildnode)}},
}

\newif\ifHideSolutions
\newcommand{\solution}[1]{\color{dkgreen}\textbf{Solution: }#1\color{black}}
\newcommand{\rubric}[1]{\color{dkgreen}{\bf Rubric:} #1\color{black}}

% \HideSolutionsfalse
% \ifHideSolutions
%   \renewcommand{\solution}[1]{}
%   \renewcommand{\rubric}[1]{}
% \fi

\newlength{\toppush}
\setlength{\toppush}{2\headheight}
\addtolength{\toppush}{\headsep}

\newcommand{\htitle}[2]{\noindent\vspace*{-\toppush}\newline\parbox{6.5in}
{\textit{Design and Analysis of Algorithms}\hfill\name\newline
Massachusetts Institute of Technology \hfill #2\newline
\profs\hfill #1 \vspace*{-.5ex}\newline
\mbox{}\hrulefill\mbox{}}\vspace*{1ex}\mbox{}\newline
\begin{center}{\Large\bf #1}\end{center}}

\newcommand{\handout}[2]{\thispagestyle{empty}
 \markboth{#1}{#1}
 \pagestyle{myheadings}\htitle{#1}{#2}}

\newcommand{\lecture}[3]{\thispagestyle{empty}
 \markboth{Lecture #1: #2}{Lecture #1: #2}
 \pagestyle{myheadings}\htitle{Lecture #1: #2}{#3}}

\newcommand{\htitlewithouttitle}[2]{\noindent\vspace*{-\toppush}\newline\parbox{6.5in}
{\textit{Design and Analysis of Algorithms}\hfill#2\newline
Massachusetts Institute of Technology \hfill 6.046\newline
\profs\hfill Handout #1\vspace*{-.5ex}\newline
\mbox{}\hrulefill\mbox{}}\vspace*{1ex}\mbox{}\newline}

\newcommand{\handoutwithouttitle}[2]{\thispagestyle{empty}
 \markboth{Handout \protect\ref{#1}}{Handout \protect\ref{#1}}
 \pagestyle{myheadings}\htitlewithouttitle{\protect\ref{#1}}{#2}}

\newcommand{\exam}[2]{% parameters: exam name, date
 \thispagestyle{empty}
 \markboth{\hspace{1cm}\subj\ #1\hspace{1in}Name\hrulefill\ \ }%
          {\subj\ #1\hspace{1in}Name\hrulefill\ \ }
 \pagestyle{myheadings}\examtitle{#1}{#2}
 \renewcommand{\theproblem}{Problem \arabic{problemnum}}
}
\newcommand{\examsolutions}[3]{% parameters: handout, exam name, date
 \thispagestyle{empty}
 \markboth{Handout \protect\ref{#1}: #2}{Handout \protect\ref{#1}: #2}
% \pagestyle{myheadings}\htitle{\protect\ref{#1}}{#2}{#3}
 \pagestyle{myheadings}\examsolutionstitle{\protect\ref{#1}} {#2}{#3}
 \renewcommand{\theproblem}{Problem \arabic{problemnum}}
}
\newcommand{\examsolutionstitle}[3]{\noindent\vspace*{-\toppush}\newline\parbox{6.5in}
{\textit{Design and Analysis of Algorithms}\hfill#3\newline
Massachusetts Institute of Technology \hfill 6.046\newline
%Singapore-MIT Alliance \hfill SMA5503\newline
\profs\hfill Handout #1\vspace*{-.5ex}\newline
\mbox{}\hrulefill\mbox{}}\vspace*{1ex}\mbox{}\newline
\begin{center}{\Large\bf #2}\end{center}}

\newcommand{\takehomeexam}[2]{% parameters: exam name, date
 \thispagestyle{empty}
 \markboth{\subj\ #1\hfill}{\subj\ #1\hfill}
 \pagestyle{myheadings}\examtitle{#1}{#2}
 \renewcommand{\theproblem}{Problem \arabic{problemnum}}
}

\makeatletter
\newcommand{\exambooklet}[2]{% parameters: exam name, date
 \thispagestyle{empty}
 \markboth{\subj\ #1}{\subj\ #1}
 \pagestyle{myheadings}\examtitle{#1}{#2}
 \renewcommand{\theproblem}{Problem \arabic{problemnum}}
 \renewcommand{\problem}{\newpage
 \item \let\@currentlabel=\theproblem
 \markboth{\subj\ #1, \theproblem}{\subj\ #1, \theproblem}}
}
\makeatother


\newcommand{\examtitle}[2]{\noindent\vspace*{-\toppush}\newline\parbox{6.5in}
{\textit{Design and Analysis of Algorithms}\hfill#2\newline
Massachusetts Institute of Technology \hfill 6.046 Spring 2019\newline
%Singapore-MIT Alliance \hfill SMA5503\newline
\profs\hfill #1\vspace*{-.5ex}\newline
\mbox{}\hrulefill\mbox{}}\vspace*{1ex}\mbox{}\newline
\begin{center}{\Large\bf #1}\end{center}}

\newcommand{\grader}[1]{\hspace{1cm}\textsf{\textbf{#1}}\hspace{1cm}}

\newcommand{\points}[1]{[#1 points]\ }
\newcommand{\parts}[1]
{
  \ifnum#1=1
  (1 part)
  \else
  (#1 parts)
  \fi
  \ 
}

\newcommand{\bparts}{\begin{problemparts}}
\newcommand{\eparts}{\end{problemparts}}
\newcommand{\ppart}{\problempart}

%\newcommand{\lg} {lg\ }

\setlength{\oddsidemargin}{0pt}
\setlength{\evensidemargin}{0pt}
\setlength{\textwidth}{6.5in}
\setlength{\topmargin}{0in}
\setlength{\textheight}{8.5in}


\newcommand{\Spawn}{{\bf spawn} }
\newcommand{\Sync}{{\bf sync}}

\newcommand{\cif}[1]{\mbox{if $#1$}}
\newcommand{\cwhen}[1]{\mbox{when $#1$}}

\newcounter{problemnum}
\newcommand{\theproblem}{Problem \theproblemsetnum-\arabic{problemnum}}
\newenvironment{problems}{
        \begin{list}{{\bf \theproblem. \hspace*{0.5em}}}
        {\setlength{\leftmargin}{0em}
         \setlength{\rightmargin}{0em}
         \setlength{\labelwidth}{0em}
         \setlength{\labelsep}{0em}
         \usecounter{problemnum}}}{\end{list}}
\makeatletter
\newcommand{\problem}[1][{}]{\item \let\@currentlabel=\theproblem \textbf{#1}}
\makeatother

\newcounter{problempartnum}[problemnum]
\newenvironment{problemparts}{
        \begin{list}{{\bf (\alph{problempartnum})}}
        {\setlength{\leftmargin}{2.5em}
         \setlength{\rightmargin}{2.5em}
         \setlength{\labelsep}{0.5em}}}{\end{list}}
\newcommand{\problempart}{\addtocounter{problempartnum}{1}\item}

\newenvironment{truefalseproblemparts}{
        \begin{list}{{\bf (\alph{problempartnum})\ \ \ T\ \ F\hfil}}
        {\setlength{\leftmargin}{4.5em}
         \setlength{\rightmargin}{2.5em}
         \setlength{\labelsep}{0.5em}
         \setlength{\labelwidth}{4.5em}}}{\end{list}}

\newcounter{exercisenum}
\newcommand{\theexercise}{Exercise \theproblemsetnum-\arabic{exercisenum}}
\newenvironment{exercises}{
        \begin{list}{{\bf \theexercise. \hspace*{0.5em}}}
        {\setlength{\leftmargin}{0em}
         \setlength{\rightmargin}{0em}
         \setlength{\labelwidth}{0em}
         \setlength{\labelsep}{0em}
        \usecounter{exercisenum}}}{\end{list}}
\makeatletter
\newcommand{\exercise}{\item \let\@currentlabel=\theexercise}
\makeatother

\newcounter{exercisepartnum}[exercisenum]
%\newcommand{\problem}[1]{\medskip\mbox{}\newline\noindent{\bf Problem #1.}\hspace*{1em}}
%\newcommand{\exercise}[1]{\medskip\mbox{}\newline\noindent{\bf Exercise #1.}\hspace*{1em}}

\newenvironment{exerciseparts}{
        \begin{list}{{\bf (\alph{exercisepartnum})}}
        {\setlength{\leftmargin}{2.5em}
         \setlength{\rightmargin}{2.5em}
         \setlength{\labelsep}{0.5em}}}{\end{list}}
\newcommand{\exercisepart}{\addtocounter{exercisepartnum}{1}\item}


% Macros to make captions print with small type and 'Figure xx' in bold.
\makeatletter
\def\fnum@figure{{\bf Figure \thefigure}}
\def\fnum@table{{\bf Table \thetable}}
\let\@mycaption\caption
%\long\def\@mycaption#1[#2]#3{\addcontentsline{\csname
%  ext@#1\endcsname}{#1}{\protect\numberline{\csname 
%  the#1\endcsname}{\ignorespaces #2}}\par
%  \begingroup
%    \@parboxrestore
%    \small
%    \@makecaption{\csname fnum@#1\endcsname}{\ignorespaces #3}\par
%  \endgroup}
%\def\mycaption{\refstepcounter\@captype \@dblarg{\@mycaption\@captype}}
%\makeatother
\let\mycaption\caption
%\newcommand{\figcaption}[1]{\mycaption[]{#1}}

\newcounter{totalcaptions}
\newcounter{totalart}

\newcommand{\figcaption}[1]{\addtocounter{totalcaptions}{1}\caption[]{#1}}

% \psfigures determines what to do for figures:
%       0 means just leave vertical space
%       1 means put a vertical rule and the figure name
%       2 means insert the PostScript version of the figure
%       3 means put the figure name flush left or right
\newcommand{\psfigures}{0}
\newcommand{\spacefigures}{\renewcommand{\psfigures}{0}}
\newcommand{\rulefigures}{\renewcommand{\psfigures}{1}}
\newcommand{\macfigures}{\renewcommand{\psfigures}{2}}
\newcommand{\namefigures}{\renewcommand{\psfigures}{3}}

\newcommand{\figpart}[1]{{\bf (#1)}\nolinebreak[2]\relax}
\newcommand{\figparts}[2]{{\bf (#1)--(#2)}\nolinebreak[2]\relax}


\macfigures     % STATE

% When calling \figspace, make sure to leave a blank line afterward!!
% \widefigspace is for figures that are more than 28pc wide.
\newlength{\halffigspace} \newlength{\wholefigspace}
\newlength{\figruleheight} \newlength{\figgap}
\newcommand{\setfiglengths}{\ifnum\psfigures=1\setlength{\figruleheight}{\hruleheight}\setlength{\figgap}{1em}\else\setlength{\figruleheight}{0pt}\setlength{\figgap}{0em}\fi}
\newcommand{\figspace}[2]{\ifnum\psfigures=0\leavefigspace{#1}\else%
\setfiglengths%
\setlength{\wholefigspace}{#1}\setlength{\halffigspace}{.5\wholefigspace}%
\rule[-\halffigspace]{\figruleheight}{\wholefigspace}\hspace{\figgap}#2\fi}
\newlength{\widefigspacewidth}
% Make \widefigspace put the figure flush right on the text page.
\newcommand{\widefigspace}[2]{
\ifnum\psfigures=0\leavefigspace{#1}\else%
\setfiglengths%
\setlength{\widefigspacewidth}{28pc}%
\addtolength{\widefigspacewidth}{-\figruleheight}%
\setlength{\wholefigspace}{#1}\setlength{\halffigspace}{.5\wholefigspace}%
\makebox[\widefigspacewidth][r]{#2\hspace{\figgap}}\rule[-\halffigspace]{\figruleheight}{\wholefigspace}\fi}
\newcommand{\leavefigspace}[1]{\setlength{\wholefigspace}{#1}\setlength{\halffigspace}{.5\wholefigspace}\rule[-\halffigspace]{0em}{\wholefigspace}}

% Commands for including figures with macpsfig.
% To use these commands, documentstyle ``macpsfig'' must be specified.
\newlength{\macfigfill}
\makeatother
\newlength{\bbx}
\newlength{\bby}
\newcommand{\macfigure}[5]{\addtocounter{totalart}{1}
\ifnum\psfigures=2%
\setlength{\bbx}{#2}\addtolength{\bbx}{#4}%
\setlength{\bby}{#3}\addtolength{\bby}{#5}%
\begin{flushleft}
\ifdim#4>28pc\setlength{\macfigfill}{#4}\addtolength{\macfigfill}{-28pc}\hspace*{-\macfigfill}\fi%
\mbox{\psfig{figure=./#1.ps,%
bbllx=#2,bblly=#3,bburx=\bbx,bbury=\bby}}
\end{flushleft}%
\else\ifdim#4>28pc\widefigspace{#5}{#1}\else\figspace{#5}{#1}\fi\fi}
\makeatletter

\newlength{\savearraycolsep}
\newcommand{\narrowarray}[1]{\setlength{\savearraycolsep}{\arraycolsep}\setlength{\arraycolsep}{#1\arraycolsep}}
\newcommand{\normalarray}{\setlength{\arraycolsep}{\savearraycolsep}}

\newcommand{\hint}{{\em Hint:\ }}

% Macros from /th/u/clr/mac.tex

\newcommand{\set}[1]{\left\{ #1 \right\}}
\newcommand{\abs}[1]{\left| #1\right|}
\newcommand{\card}[1]{\left| #1\right|}
\newcommand{\floor}[1]{\left\lfloor #1 \right\rfloor}
\newcommand{\ceil}[1]{\left\lceil #1 \right\rceil}
\newcommand{\ang}[1]{\ifmmode{\left\langle #1 \right\rangle}
   \else{$\left\langle${#1}$\right\rangle$}\fi}
        % the \if allows use outside mathmode,
        % but will swallow following space there!
\newcommand{\paren}[1]{\left( #1 \right)}
\newcommand{\bracket}[1]{\left[ #1 \right]}
\newcommand{\prob}[1]{\Pr\left\{ #1 \right\}}
\newcommand{\Var}{\mathop{\rm Var}\nolimits}
\newcommand{\expect}[1]{{\rm E}\left[ #1 \right]}
\newcommand{\expectsq}[1]{{\rm E}^2\left[ #1 \right]}
\newcommand{\variance}[1]{{\rm Var}\left[ #1 \right]}
\renewcommand{\choose}[2]{{{#1}\atopwithdelims(){#2}}}
\def\pmod#1{\allowbreak\mkern12mu({\rm mod}\,\,#1)}
\newcommand{\matx}[2]{\left(\begin{array}{*{#1}{c}}#2\end{array}\right)}
\newcommand{\Adj}{\mathop{\rm Adj}\nolimits}

\newtheorem{theorem}{Theorem}
\newtheorem{lemma}[theorem]{Lemma}
\newtheorem{corollary}[theorem]{Corollary}
\newtheorem{xample}{Example}
\newtheorem{definition}{Definition}
\newenvironment{example}{\begin{xample}\rm}{\end{xample}}
\newcommand{\proof}{\noindent{\em Proof.}\hspace{1em}}
\def\squarebox#1{\hbox to #1{\hfill\vbox to #1{\vfill}}}
\newcommand{\qedbox}{\vbox{\hrule\hbox{\vrule\squarebox{.667em}\vrule}\hrule}}
\newcommand{\qed}{\nopagebreak\mbox{}\hfill\qedbox\smallskip}
\newcommand{\eqnref}[1]{(\protect\ref{#1})}

%%\newcommand{\twodots}{\mathinner{\ldotp\ldotp}}
\newcommand{\transpose}{^{\mbox{\scriptsize \sf T}}}
\newcommand{\amortized}[1]{\widehat{#1}}

\newcommand{\punt}[1]{}

%%% command for putting definitions into boldface
% New style for defined terms, as of 2/23/88, redefined by THC.
\newcommand{\defn}[1]{{\boldmath\textit{\textbf{#1}}}}
\newcommand{\defi}[1]{{\textit{\textbf{#1\/}}}}

\newcommand{\red}{\leq_{\rm P}}
\newcommand{\lang}[1]{%
\ifmmode\mathord{\mathcode`-="702D\rm#1\mathcode`\-="2200}\else{\rm#1}\fi}

%\newcommand{\ckt}[1]{\ifmmode\mathord{\mathcode`-="702D\sc #1\mathcode`\-="2200}\else$\mathord{\mathcode`-="702D\sc #1\mathcode`\-="2200}$\fi}
\newcommand{\ckt}[1]{\ifmmode \sc #1\else$\sc #1$\fi}

%% Margin notes - use \notesfalse to turn off notes.
\setlength{\marginparwidth}{0.6in}
\reversemarginpar
\newif\ifnotes
\notestrue
\newcommand{\longnote}[1]{
  \ifnotes
    {\medskip\noindent Note: \marginpar[\hfill$\Longrightarrow$]
      {$\Longleftarrow$}{#1}\medskip}
  \fi}
\newcommand{\note}[1]{
  \ifnotes
    {\marginpar{\tiny \raggedright{#1}}}
  \fi}


\newcommand{\reals}{\mathbbm{R}}
\newcommand{\integers}{\mathbbm{Z}}
\newcommand{\naturals}{\mathbbm{N}}
\newcommand{\rationals}{\mathbbm{Q}}
\newcommand{\complex}{\mathbbm{C}}

\newcommand{\oldreals}{{\bf R}}
\newcommand{\oldintegers}{{\bf Z}}
\newcommand{\oldnaturals}{{\bf N}}
\newcommand{\oldrationals}{{\bf Q}}
\newcommand{\oldcomplex}{{\bf C}}

\newcommand{\w}{\omega}                 %% for fft chapter

\newenvironment{closeitemize}{\begin{list}
{$\bullet$}
{\setlength{\itemsep}{-0.2\baselineskip}
\setlength{\topsep}{0.2\baselineskip}
\setlength{\parskip}{0pt}}}
{\end{list}}

% These are necessary within a {problems} environment in order to restore
% the default separation between bullets and items.
\newenvironment{normalitemize}{\setlength{\labelsep}{0.5em}\begin{itemize}}
                              {\end{itemize}}
\newenvironment{normalenumerate}{\setlength{\labelsep}{0.5em}\begin{enumerate}}
                                {\end{enumerate}}

%\def\eqref#1{Equation~(\ref{eq:#1})}
%\newcommand{\eqref}[1]{Equation (\ref{eq:#1})}
\newcommand{\eqreftwo}[2]{Equations (\ref{eq:#1}) and~(\ref{eq:#2})}
\newcommand{\ineqref}[1]{Inequality~(\ref{ineq:#1})}
\newcommand{\ineqreftwo}[2]{Inequalities (\ref{ineq:#1}) and~(\ref{ineq:#2})}

\newcommand{\figref}[1]{Figure~\ref{fig:#1}}
\newcommand{\figreftwo}[2]{Figures \ref{fig:#1} and~\ref{fig:#2}}

\newcommand{\liref}[1]{line~\ref{li:#1}}
\newcommand{\Liref}[1]{Line~\ref{li:#1}}
\newcommand{\lirefs}[2]{lines \ref{li:#1}--\ref{li:#2}}
\newcommand{\Lirefs}[2]{Lines \ref{li:#1}--\ref{li:#2}}
\newcommand{\lireftwo}[2]{lines \ref{li:#1} and~\ref{li:#2}}
\newcommand{\lirefthree}[3]{lines \ref{li:#1}, \ref{li:#2}, and~\ref{li:#3}}

\newcommand{\lemlabel}[1]{\label{lem:#1}}
\newcommand{\lemref}[1]{Lemma~\ref{lem:#1}} 

\newcommand{\exref}[1]{Exercise~\ref{ex:#1}}

\newcommand{\handref}[1]{Handout~\ref{#1}}

\newcommand{\defref}[1]{Definition~\ref{def:#1}}

% (1997.8.16: Victor Luchangco)
% Modified \hlabel to only get date and to use handouts counter for number.
%   New \handout and \handoutwithouttitle commands in newmac.tex use this.
%   The date is referenced by <label>-date.
%   (Retained old definition as \hlabelold.)
%   Defined \hforcelabel to use an argument instead of the handouts counter.

\newcounter{handouts}
\setcounter{handouts}{0}

\newcommand{\hlabel}[2]{%
\stepcounter{handouts}
{\edef\next{\write\@auxout{\string\newlabel{#1}{{\arabic{handouts}}{0}}}}\next}
\write\@auxout{\string\newlabel{#1-date}{{#2}{0}}}
}

\newcommand{\hforcelabel}[3]{%          Does not step handouts counter.
\write\@auxout{\string\newlabel{#1}{{#2}{0}}}
\write\@auxout{\string\newlabel{#1-date}{{#3}{0}}}}


% less ugly underscore
% --juang, 2008 oct 05
\renewcommand{\_}{\vrule height 0 pt depth 0.4 pt width 0.5 em \,}

\newcommand{\theproblemsetnum}{2}
\newcommand{\releasedate}{Thursday, February 14}
\newcommand{\partaduedate}{Wednesday, February 20}

\title{6.046 Problem Set \theproblemsetnum}

\begin{document}

\handout{Problem Set \theproblemsetnum}{\releasedate}
\textbf{All parts are due {\bf \partaduedate} at {\bf 10PM}}.

\setlength{\parindent}{0pt}
\medskip\hrulefill\medskip

{\bf Name:} Robert Durfee

\medskip

{\bf Collaborators:} None

\medskip\hrulefill

\begin{problems}

\problem  % Problem 1

\begin{problemparts}

\problempart % Problem 1a

\textbf{Description} Let the $n$ people in the room be numbered $[1..n]$. 
Let the query $Q(i, j)$ be defined as,
$$ Q(i, j) = \begin{cases}
    1 & \mathrm{if\ persons}\ i\ \mathrm{and}\ j \ \mathrm{are\ from\ the\ same\
    universe} \\
    0 & \mathrm{otherwise}
\end{cases} $$
Let $d$ be the index of the last person such that $Q(i, j) = 0$. Then, ask the 
following set of queries,
$$ \mathbb{S} = \{ Q(1, j) \mid j \in [2..n] \} $$
While asking the queries, update $d$ whenever $s = 0$. After asking all the 
queries, if
$$ \sum_{s \in \mathbb{S}} s > \frac{\vert \mathbb{S} \vert}{2} $$
then return the first person. Otherwise, return the person whose index is stored
in $d$.

\textbf{Correctness} This algorithm relies chiefly on the fact that the number
of Xers is greater than the number of humans. We can compare a single person,
$p_1$, (whose universe is unknown) to every other person to split the group 
into Xers and Earthlings. Whichever group is larger must contain Xers. If the
sum of persons from the same universe as $p_1$ is larger than half, that person
must be an Xer. If not, the algorithm also keeps track of the last person from
a different universe than $p_1$. This person must be an Xer.

\textbf{Running Time} Only $n$ queries are made, as we only compare a single
person to all others. Since each of the queries is $O(1)$, the total run time is
$O(n)$.

\problempart % Problem 1b

Let there be exactly $n/2$ Xers and $n/2$ Earthlings. Let the Xers always report
an Earthling as an Xer and an Xer as an Earthling. Conversely, an Earthling will
always report another Earthling as an Earthling and an Xer as an Xer. This 
creates a symmetry between responses such that it becomes impossible to
differentiate between Xers and Earthlings.

\problempart % Problem 1c

To uncover the hidden states that yield each possible outcome, consider the
following table. The axes represent the true identity of persons $A$ and $B$.

\begin{center}
    \begin{tabular}{c c|c c}
        & & \multicolumn{2}{c}{B} \\
        & & X & E \\
        \hline 
        \multirow{6}{*}{A} & \multirow{4}{*}{X} & $(A, B) \in (E, E)$ & \\
        &  & $(A, B) \in (E, X)$ & $(A, B) \in (X, X)$ \\
        & & $(A, B) \in (X, E)$ & $(A, B) \in (X, E)$ \\
        & & $(A, B) \in (X, X)$ & \\
        & \multirow{2}{*}{E} & $(A, B) \in (X, X)$ & \multirow{2}{*}{$(A, B) \in (E, E)$} \\
        & & $(A, B) \in (E, X)$ & \\
    \end{tabular}
\end{center}

Examining each case in detail,
\begin{itemize}
    \item  In the upper left case $(A, B) \in (X, X)$, both would be motivated 
    to lie to keep each others' identities secret. However, they are not
    constrained to lie, they could also tell the truth. Thus, anything is 
    possible $(A, B) \in (E, E)$, $(A, B) \in (E, X)$, $(A, B) \in (X, E)$, 
    $(A, B) \in (X, X)$.
    \item In the upper right case $(A, B) \in (X, E)$, $B$ is required to tell 
    the truth and will identify $A$ as an Xer. However, $A$ is not bound to the
    truth and may report $B$ to be either an Xer or an Earthling. So there are
    two reported possibilities $(A, B) \in (X, E)$ or $(A, B) \in (X, X)$.
    \item In the lower left case $(A, B) \in (E, X)$, $A$ is required to tell
    the truth and will identify $B$ as an Xer. However, $B$ is not bound to the
    truth and may report $A$ to be either an Xer or an Earthling. So there are
    two reported possibilities $(A, B) \in (E, X)$ or $(A, B) \in (X, X)$.
    \item In the lower right case $(A, B) \in (E, E)$, both $A$ and $B$ would
    tell the truth and report $(A, B) \in (E, E)$.
\end{itemize}

From this analysis, the reported cases have the following possibilities,
\begin{itemize}
    \item $(X, X)$: There may be 1 or 2 Xers. 
    \item $(X, E)$: There may be 1 or 2 Xers.
    \item $(E, X)$: There may be 1 or 2 Xers.
    \item $(E, E)$: There may be 0 Xers (both Earthlings telling the truth) or
    there may be 2 Xers (colluding to protect each others' identity).
\end{itemize}

\problempart % Problem 1d

\textbf{Description} Pair up all the persons. For each pair, execute a query. If
the response is not that both are Earthlings, remove the pairs. For all the
responses that both are Earthlings, remove only one person from the pair. Repeat
until only one person remains. This person is an Earthling. With this person as
one of members of the pair, query all other people. Return this person 
concatenated with the list of any person they identify as an Earthling.

\textbf{Correctness} It is safe to remove all pairs that answer $(X, E)$, $(E, X)$,
or $(X, X)$ as there can only be either $1$ or $2$ Xers in those pairs. Thus, the 
invariant of a majority of Earthlings will remain correct. Among the responses of 
$(E, E)$, we know that there are either $0$ or $2$ Xers. In either case, they 
are from the same universe and one of the pair can be safely removed and the 
invariant of a majority of Earthlings will be preserved. Since there are a
majority of Earthlings in every subproblem, there is guaranteed to be an Earthling-
Earthling pairing every step. This pairing will always report $(E, E)$ and thus
our algorithm will always reduce to a terminating state of a single Earthling.

Once a single Earthling is identified, we can use them to identify all other 
Earthlings by pairing them will every other person. Their response must be
truthful so reported Earthlings must be Earthlings.

\textbf{Running Time} The recurrence that describes this process is,
$$ T(n) = T(n / 2) + n / 2 $$
At each subproblem, we (at worst) reduce the problem in half by discarding one
member from each pairing reporting $(E, E)$. Furthermore, we pair up all the
members so we only ask $n / 2$ queries.

We use the Master Theorem to solve the recurrence. Using the standard notation,
$a = 1$, $b = 2$, and $f(n) = n / 2$. If $\varepsilon > 1$, then $f(n) \in 
\Omega(n^{a \log_b a + \varepsilon})$. Furthermore, if $c > 1/4$, then $a f(n/2) 
\leq c f(n)$ for all $n$. Therefore, Case III of the Master Theorem applies and 
the solution of the recurrence is,
$$ T(n) \in \Theta(n) $$

Additionally, to use the identified Earthling to find all the other Earthlings 
also takes $\Theta(n)$ because every individual must be compared. From this, the
overall run time is $\Theta(n)$.

\end{problemparts}

\newpage
\problem  % Problem 2

\textbf{Description} Consider two binary strings $A$ and $B$ of lengths $2n$ 
and $n$, respectively. Construct a polynomial $A(x)$ such that the coefficient
$a_k = 1$ if $A[k] = 1$ and $a_k = -1$ if $A[k] = 0$. Construct a polynomial
$B(x)$ such that coefficient $b_k = 1$ if $B[n - k] = 1$ and $b_k = -1$ if
$B[n - k] = 0$. 

Use the FFT algorithm discussed in lecture to compute the polynomial product 
$A(x) \cdot B(x) = C(x)$ in coefficient representation. Consider the coefficients 
of $C$ of degree $k \in [n - 1, 2n - 1]$. If coefficient $c_k \geq n - 2 d$, 
add index $k - n + 1$ to the list of indexes with Hamming distance less than or 
equal to $d$. After every $c_k$ in the specified range has been visited, return 
the list of indices.

\textbf{Correctness} Multiplying two polynomials is the same process as
performing a discrete convolution of their coefficients. The convolution of 
discrete arrays $A$ and $B$ reverses $B$ and then slides it along $A$, computing
the dot product of the overlapping regions. 

In our problem, we want to slide $B$ along $A$ and select the locations with 
Hamming distance less than $d$. Because convolution reverses the list $B$---which
we do not want---we first counter this by reversing $B$ before convolving. 

Next we encode the Hamming distance into the dot product. Consider the dot product
between two vectors $v, w \in \{-1, 1\}^n$. A Hamming distance of 
$0$ is equivalent to the value $n$. This is clear to see as $1 \cdot 1 = 1$
and $-1 \cdot -1 = 1$ which will be the same for all $n$ elements.
Furthermore, a Hamming distance of $n$ is equivalent to the value $-n$. This
is clear to see as $-1 \cdot 1 = -1$ which will be the same for all $n$ elements.
From this, all Hamming distances can be related to the dot product result using the
formula,
$$ v \cdot w = n - 2d $$
In other words, a single mistake will have $2 \times$ the effect on the value.

Since the polynomial product will return the dot product of $B$ with $A$ along at
each index, we just need to find the indexes with dot products greater than or 
equal to $n - 2d$. However, another important note about convolution is that $A$
is actually padded with zeros on either end. Thus, we need to ignore the beginning
and end of the sequence and, hence, the lowermost and uppermost degree coefficients.
There will be $n - 1$ padding at the upper and lower, so we ignore those 
coefficients. This leaves coefficients with degree $k \in [n - 1, 2n - 1]$.

Lastly, since we ignore some coefficients, we need to adjust the degree to
correspond to an index. Since we ignore $n - 1$ elements, we subtract this from
the degree with coefficients $c_k \geq n - 2d$. Thus, the returned indexes are
$k - n + 1$.

\textbf{Running Time} To construct the polynomials requires $O(n)$ time. 
Multiplication of two polynomials in coefficient form requires $O(n \log n)$ 
using the FFT algorithm presented in lecture. To iterate over the coefficients
and transform them into into a list of indices requires $O(n)$ time. Overall,
this must take $O(n + n \log n) \in O(n \log n) $.

\end{problems}

\end{document}


