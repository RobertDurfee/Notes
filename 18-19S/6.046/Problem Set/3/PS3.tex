%
% 6.046 problem set 2 solutions
%
\documentclass[12pt,twoside]{article}
\usepackage{multirow}
\usepackage{algorithm, algpseudocode, float}

\newcommand{\name}{}

\usepackage{amssymb}
\usepackage{amsmath}
\usepackage{graphicx}
\usepackage{latexsym}
\usepackage{times,url}
\usepackage{cprotect}
\usepackage{listings}
\usepackage{graphicx}
\usepackage[table]{xcolor}
\usepackage[letterpaper]{geometry}
\usepackage{tikz-qtree}
\usepackage{enumerate}

\newcommand{\profs}{Mauricio Karchmer, Aleksander Madry, Bruce Tidor}
\newcommand{\subj}{6.046}
\newcommand{\ttt}[1]{{\tt\small #1}}

\definecolor{dkgreen}{rgb}{0,0.6,0}
\definecolor{gray}{rgb}{0.5,0.5,0.5}
\definecolor{mauve}{rgb}{0.58,0,0.82}

\lstset{
  language=Python,
  aboveskip=1pc,
  belowskip=1pc,
  basicstyle={\footnotesize\ttfamily},
  numbers=left,
  showstringspaces=false,
  numberstyle=\tiny\color{gray},
  keywordstyle=\color{blue},
  commentstyle=\color{dkgreen},
  stringstyle=\color{mauve},
}

\tikzset{
  % every node/.style={minimum width=2em,draw,circle},
  % level 1/.style={sibling distance=2cm},
  level distance=1cm,
  edge from parent/.style=
  {draw,edge from parent path={(\tikzparentnode) -- (\tikzchildnode)}},
}

\newif\ifHideSolutions
\newcommand{\solution}[1]{\color{dkgreen}\textbf{Solution: }#1\color{black}}
\newcommand{\rubric}[1]{\color{dkgreen}{\bf Rubric:} #1\color{black}}

% \HideSolutionsfalse
% \ifHideSolutions
%   \renewcommand{\solution}[1]{}
%   \renewcommand{\rubric}[1]{}
% \fi

\newlength{\toppush}
\setlength{\toppush}{2\headheight}
\addtolength{\toppush}{\headsep}

\newcommand{\htitle}[2]{\noindent\vspace*{-\toppush}\newline\parbox{6.5in}
{\textit{Design and Analysis of Algorithms}\hfill\name\newline
Massachusetts Institute of Technology \hfill #2\newline
\profs\hfill #1 \vspace*{-.5ex}\newline
\mbox{}\hrulefill\mbox{}}\vspace*{1ex}\mbox{}\newline
\begin{center}{\Large\bf #1}\end{center}}

\newcommand{\handout}[2]{\thispagestyle{empty}
 \markboth{#1}{#1}
 \pagestyle{myheadings}\htitle{#1}{#2}}

\newcommand{\lecture}[3]{\thispagestyle{empty}
 \markboth{Lecture #1: #2}{Lecture #1: #2}
 \pagestyle{myheadings}\htitle{Lecture #1: #2}{#3}}

\newcommand{\htitlewithouttitle}[2]{\noindent\vspace*{-\toppush}\newline\parbox{6.5in}
{\textit{Design and Analysis of Algorithms}\hfill#2\newline
Massachusetts Institute of Technology \hfill 6.046\newline
\profs\hfill Handout #1\vspace*{-.5ex}\newline
\mbox{}\hrulefill\mbox{}}\vspace*{1ex}\mbox{}\newline}

\newcommand{\handoutwithouttitle}[2]{\thispagestyle{empty}
 \markboth{Handout \protect\ref{#1}}{Handout \protect\ref{#1}}
 \pagestyle{myheadings}\htitlewithouttitle{\protect\ref{#1}}{#2}}

\newcommand{\exam}[2]{% parameters: exam name, date
 \thispagestyle{empty}
 \markboth{\hspace{1cm}\subj\ #1\hspace{1in}Name\hrulefill\ \ }%
          {\subj\ #1\hspace{1in}Name\hrulefill\ \ }
 \pagestyle{myheadings}\examtitle{#1}{#2}
 \renewcommand{\theproblem}{Problem \arabic{problemnum}}
}
\newcommand{\examsolutions}[3]{% parameters: handout, exam name, date
 \thispagestyle{empty}
 \markboth{Handout \protect\ref{#1}: #2}{Handout \protect\ref{#1}: #2}
% \pagestyle{myheadings}\htitle{\protect\ref{#1}}{#2}{#3}
 \pagestyle{myheadings}\examsolutionstitle{\protect\ref{#1}} {#2}{#3}
 \renewcommand{\theproblem}{Problem \arabic{problemnum}}
}
\newcommand{\examsolutionstitle}[3]{\noindent\vspace*{-\toppush}\newline\parbox{6.5in}
{\textit{Design and Analysis of Algorithms}\hfill#3\newline
Massachusetts Institute of Technology \hfill 6.046\newline
%Singapore-MIT Alliance \hfill SMA5503\newline
\profs\hfill Handout #1\vspace*{-.5ex}\newline
\mbox{}\hrulefill\mbox{}}\vspace*{1ex}\mbox{}\newline
\begin{center}{\Large\bf #2}\end{center}}

\newcommand{\takehomeexam}[2]{% parameters: exam name, date
 \thispagestyle{empty}
 \markboth{\subj\ #1\hfill}{\subj\ #1\hfill}
 \pagestyle{myheadings}\examtitle{#1}{#2}
 \renewcommand{\theproblem}{Problem \arabic{problemnum}}
}

\makeatletter
\newcommand{\exambooklet}[2]{% parameters: exam name, date
 \thispagestyle{empty}
 \markboth{\subj\ #1}{\subj\ #1}
 \pagestyle{myheadings}\examtitle{#1}{#2}
 \renewcommand{\theproblem}{Problem \arabic{problemnum}}
 \renewcommand{\problem}{\newpage
 \item \let\@currentlabel=\theproblem
 \markboth{\subj\ #1, \theproblem}{\subj\ #1, \theproblem}}
}
\makeatother


\newcommand{\examtitle}[2]{\noindent\vspace*{-\toppush}\newline\parbox{6.5in}
{\textit{Design and Analysis of Algorithms}\hfill#2\newline
Massachusetts Institute of Technology \hfill 6.046 Spring 2019\newline
%Singapore-MIT Alliance \hfill SMA5503\newline
\profs\hfill #1\vspace*{-.5ex}\newline
\mbox{}\hrulefill\mbox{}}\vspace*{1ex}\mbox{}\newline
\begin{center}{\Large\bf #1}\end{center}}

\newcommand{\grader}[1]{\hspace{1cm}\textsf{\textbf{#1}}\hspace{1cm}}

\newcommand{\points}[1]{[#1 points]\ }
\newcommand{\parts}[1]
{
  \ifnum#1=1
  (1 part)
  \else
  (#1 parts)
  \fi
  \ 
}

\newcommand{\bparts}{\begin{problemparts}}
\newcommand{\eparts}{\end{problemparts}}
\newcommand{\ppart}{\problempart}

%\newcommand{\lg} {lg\ }

\setlength{\oddsidemargin}{0pt}
\setlength{\evensidemargin}{0pt}
\setlength{\textwidth}{6.5in}
\setlength{\topmargin}{0in}
\setlength{\textheight}{8.5in}


\newcommand{\Spawn}{{\bf spawn} }
\newcommand{\Sync}{{\bf sync}}

\newcommand{\cif}[1]{\mbox{if $#1$}}
\newcommand{\cwhen}[1]{\mbox{when $#1$}}

\newcounter{problemnum}
\newcommand{\theproblem}{Problem \theproblemsetnum-\arabic{problemnum}}
\newenvironment{problems}{
        \begin{list}{{\bf \theproblem. \hspace*{0.5em}}}
        {\setlength{\leftmargin}{0em}
         \setlength{\rightmargin}{0em}
         \setlength{\labelwidth}{0em}
         \setlength{\labelsep}{0em}
         \usecounter{problemnum}}}{\end{list}}
\makeatletter
\newcommand{\problem}[1][{}]{\item \let\@currentlabel=\theproblem \textbf{#1}}
\makeatother

\newcounter{problempartnum}[problemnum]
\newenvironment{problemparts}{
        \begin{list}{{\bf (\alph{problempartnum})}}
        {\setlength{\leftmargin}{2.5em}
         \setlength{\rightmargin}{2.5em}
         \setlength{\labelsep}{0.5em}}}{\end{list}}
\newcommand{\problempart}{\addtocounter{problempartnum}{1}\item}

\newenvironment{truefalseproblemparts}{
        \begin{list}{{\bf (\alph{problempartnum})\ \ \ T\ \ F\hfil}}
        {\setlength{\leftmargin}{4.5em}
         \setlength{\rightmargin}{2.5em}
         \setlength{\labelsep}{0.5em}
         \setlength{\labelwidth}{4.5em}}}{\end{list}}

\newcounter{exercisenum}
\newcommand{\theexercise}{Exercise \theproblemsetnum-\arabic{exercisenum}}
\newenvironment{exercises}{
        \begin{list}{{\bf \theexercise. \hspace*{0.5em}}}
        {\setlength{\leftmargin}{0em}
         \setlength{\rightmargin}{0em}
         \setlength{\labelwidth}{0em}
         \setlength{\labelsep}{0em}
        \usecounter{exercisenum}}}{\end{list}}
\makeatletter
\newcommand{\exercise}{\item \let\@currentlabel=\theexercise}
\makeatother

\newcounter{exercisepartnum}[exercisenum]
%\newcommand{\problem}[1]{\medskip\mbox{}\newline\noindent{\bf Problem #1.}\hspace*{1em}}
%\newcommand{\exercise}[1]{\medskip\mbox{}\newline\noindent{\bf Exercise #1.}\hspace*{1em}}

\newenvironment{exerciseparts}{
        \begin{list}{{\bf (\alph{exercisepartnum})}}
        {\setlength{\leftmargin}{2.5em}
         \setlength{\rightmargin}{2.5em}
         \setlength{\labelsep}{0.5em}}}{\end{list}}
\newcommand{\exercisepart}{\addtocounter{exercisepartnum}{1}\item}


% Macros to make captions print with small type and 'Figure xx' in bold.
\makeatletter
\def\fnum@figure{{\bf Figure \thefigure}}
\def\fnum@table{{\bf Table \thetable}}
\let\@mycaption\caption
%\long\def\@mycaption#1[#2]#3{\addcontentsline{\csname
%  ext@#1\endcsname}{#1}{\protect\numberline{\csname 
%  the#1\endcsname}{\ignorespaces #2}}\par
%  \begingroup
%    \@parboxrestore
%    \small
%    \@makecaption{\csname fnum@#1\endcsname}{\ignorespaces #3}\par
%  \endgroup}
%\def\mycaption{\refstepcounter\@captype \@dblarg{\@mycaption\@captype}}
%\makeatother
\let\mycaption\caption
%\newcommand{\figcaption}[1]{\mycaption[]{#1}}

\newcounter{totalcaptions}
\newcounter{totalart}

\newcommand{\figcaption}[1]{\addtocounter{totalcaptions}{1}\caption[]{#1}}

% \psfigures determines what to do for figures:
%       0 means just leave vertical space
%       1 means put a vertical rule and the figure name
%       2 means insert the PostScript version of the figure
%       3 means put the figure name flush left or right
\newcommand{\psfigures}{0}
\newcommand{\spacefigures}{\renewcommand{\psfigures}{0}}
\newcommand{\rulefigures}{\renewcommand{\psfigures}{1}}
\newcommand{\macfigures}{\renewcommand{\psfigures}{2}}
\newcommand{\namefigures}{\renewcommand{\psfigures}{3}}

\newcommand{\figpart}[1]{{\bf (#1)}\nolinebreak[2]\relax}
\newcommand{\figparts}[2]{{\bf (#1)--(#2)}\nolinebreak[2]\relax}


\macfigures     % STATE

% When calling \figspace, make sure to leave a blank line afterward!!
% \widefigspace is for figures that are more than 28pc wide.
\newlength{\halffigspace} \newlength{\wholefigspace}
\newlength{\figruleheight} \newlength{\figgap}
\newcommand{\setfiglengths}{\ifnum\psfigures=1\setlength{\figruleheight}{\hruleheight}\setlength{\figgap}{1em}\else\setlength{\figruleheight}{0pt}\setlength{\figgap}{0em}\fi}
\newcommand{\figspace}[2]{\ifnum\psfigures=0\leavefigspace{#1}\else%
\setfiglengths%
\setlength{\wholefigspace}{#1}\setlength{\halffigspace}{.5\wholefigspace}%
\rule[-\halffigspace]{\figruleheight}{\wholefigspace}\hspace{\figgap}#2\fi}
\newlength{\widefigspacewidth}
% Make \widefigspace put the figure flush right on the text page.
\newcommand{\widefigspace}[2]{
\ifnum\psfigures=0\leavefigspace{#1}\else%
\setfiglengths%
\setlength{\widefigspacewidth}{28pc}%
\addtolength{\widefigspacewidth}{-\figruleheight}%
\setlength{\wholefigspace}{#1}\setlength{\halffigspace}{.5\wholefigspace}%
\makebox[\widefigspacewidth][r]{#2\hspace{\figgap}}\rule[-\halffigspace]{\figruleheight}{\wholefigspace}\fi}
\newcommand{\leavefigspace}[1]{\setlength{\wholefigspace}{#1}\setlength{\halffigspace}{.5\wholefigspace}\rule[-\halffigspace]{0em}{\wholefigspace}}

% Commands for including figures with macpsfig.
% To use these commands, documentstyle ``macpsfig'' must be specified.
\newlength{\macfigfill}
\makeatother
\newlength{\bbx}
\newlength{\bby}
\newcommand{\macfigure}[5]{\addtocounter{totalart}{1}
\ifnum\psfigures=2%
\setlength{\bbx}{#2}\addtolength{\bbx}{#4}%
\setlength{\bby}{#3}\addtolength{\bby}{#5}%
\begin{flushleft}
\ifdim#4>28pc\setlength{\macfigfill}{#4}\addtolength{\macfigfill}{-28pc}\hspace*{-\macfigfill}\fi%
\mbox{\psfig{figure=./#1.ps,%
bbllx=#2,bblly=#3,bburx=\bbx,bbury=\bby}}
\end{flushleft}%
\else\ifdim#4>28pc\widefigspace{#5}{#1}\else\figspace{#5}{#1}\fi\fi}
\makeatletter

\newlength{\savearraycolsep}
\newcommand{\narrowarray}[1]{\setlength{\savearraycolsep}{\arraycolsep}\setlength{\arraycolsep}{#1\arraycolsep}}
\newcommand{\normalarray}{\setlength{\arraycolsep}{\savearraycolsep}}

\newcommand{\hint}{{\em Hint:\ }}

% Macros from /th/u/clr/mac.tex

\newcommand{\set}[1]{\left\{ #1 \right\}}
\newcommand{\abs}[1]{\left| #1\right|}
\newcommand{\card}[1]{\left| #1\right|}
\newcommand{\floor}[1]{\left\lfloor #1 \right\rfloor}
\newcommand{\ceil}[1]{\left\lceil #1 \right\rceil}
\newcommand{\ang}[1]{\ifmmode{\left\langle #1 \right\rangle}
   \else{$\left\langle${#1}$\right\rangle$}\fi}
        % the \if allows use outside mathmode,
        % but will swallow following space there!
\newcommand{\paren}[1]{\left( #1 \right)}
\newcommand{\bracket}[1]{\left[ #1 \right]}
\newcommand{\prob}[1]{\Pr\left\{ #1 \right\}}
\newcommand{\Var}{\mathop{\rm Var}\nolimits}
\newcommand{\expect}[1]{{\rm E}\left[ #1 \right]}
\newcommand{\expectsq}[1]{{\rm E}^2\left[ #1 \right]}
\newcommand{\variance}[1]{{\rm Var}\left[ #1 \right]}
\renewcommand{\choose}[2]{{{#1}\atopwithdelims(){#2}}}
\def\pmod#1{\allowbreak\mkern12mu({\rm mod}\,\,#1)}
\newcommand{\matx}[2]{\left(\begin{array}{*{#1}{c}}#2\end{array}\right)}
\newcommand{\Adj}{\mathop{\rm Adj}\nolimits}

\newtheorem{theorem}{Theorem}
\newtheorem{lemma}[theorem]{Lemma}
\newtheorem{corollary}[theorem]{Corollary}
\newtheorem{xample}{Example}
\newtheorem{definition}{Definition}
\newenvironment{example}{\begin{xample}\rm}{\end{xample}}
\newcommand{\proof}{\noindent{\em Proof.}\hspace{1em}}
\def\squarebox#1{\hbox to #1{\hfill\vbox to #1{\vfill}}}
\newcommand{\qedbox}{\vbox{\hrule\hbox{\vrule\squarebox{.667em}\vrule}\hrule}}
\newcommand{\qed}{\nopagebreak\mbox{}\hfill\qedbox\smallskip}
\newcommand{\eqnref}[1]{(\protect\ref{#1})}

%%\newcommand{\twodots}{\mathinner{\ldotp\ldotp}}
\newcommand{\transpose}{^{\mbox{\scriptsize \sf T}}}
\newcommand{\amortized}[1]{\widehat{#1}}

\newcommand{\punt}[1]{}

%%% command for putting definitions into boldface
% New style for defined terms, as of 2/23/88, redefined by THC.
\newcommand{\defn}[1]{{\boldmath\textit{\textbf{#1}}}}
\newcommand{\defi}[1]{{\textit{\textbf{#1\/}}}}

\newcommand{\red}{\leq_{\rm P}}
\newcommand{\lang}[1]{%
\ifmmode\mathord{\mathcode`-="702D\rm#1\mathcode`\-="2200}\else{\rm#1}\fi}

%\newcommand{\ckt}[1]{\ifmmode\mathord{\mathcode`-="702D\sc #1\mathcode`\-="2200}\else$\mathord{\mathcode`-="702D\sc #1\mathcode`\-="2200}$\fi}
\newcommand{\ckt}[1]{\ifmmode \sc #1\else$\sc #1$\fi}

%% Margin notes - use \notesfalse to turn off notes.
\setlength{\marginparwidth}{0.6in}
\reversemarginpar
\newif\ifnotes
\notestrue
\newcommand{\longnote}[1]{
  \ifnotes
    {\medskip\noindent Note: \marginpar[\hfill$\Longrightarrow$]
      {$\Longleftarrow$}{#1}\medskip}
  \fi}
\newcommand{\note}[1]{
  \ifnotes
    {\marginpar{\tiny \raggedright{#1}}}
  \fi}


\newcommand{\reals}{\mathbbm{R}}
\newcommand{\integers}{\mathbbm{Z}}
\newcommand{\naturals}{\mathbbm{N}}
\newcommand{\rationals}{\mathbbm{Q}}
\newcommand{\complex}{\mathbbm{C}}

\newcommand{\oldreals}{{\bf R}}
\newcommand{\oldintegers}{{\bf Z}}
\newcommand{\oldnaturals}{{\bf N}}
\newcommand{\oldrationals}{{\bf Q}}
\newcommand{\oldcomplex}{{\bf C}}

\newcommand{\w}{\omega}                 %% for fft chapter

\newenvironment{closeitemize}{\begin{list}
{$\bullet$}
{\setlength{\itemsep}{-0.2\baselineskip}
\setlength{\topsep}{0.2\baselineskip}
\setlength{\parskip}{0pt}}}
{\end{list}}

% These are necessary within a {problems} environment in order to restore
% the default separation between bullets and items.
\newenvironment{normalitemize}{\setlength{\labelsep}{0.5em}\begin{itemize}}
                              {\end{itemize}}
\newenvironment{normalenumerate}{\setlength{\labelsep}{0.5em}\begin{enumerate}}
                                {\end{enumerate}}

%\def\eqref#1{Equation~(\ref{eq:#1})}
%\newcommand{\eqref}[1]{Equation (\ref{eq:#1})}
\newcommand{\eqreftwo}[2]{Equations (\ref{eq:#1}) and~(\ref{eq:#2})}
\newcommand{\ineqref}[1]{Inequality~(\ref{ineq:#1})}
\newcommand{\ineqreftwo}[2]{Inequalities (\ref{ineq:#1}) and~(\ref{ineq:#2})}

\newcommand{\figref}[1]{Figure~\ref{fig:#1}}
\newcommand{\figreftwo}[2]{Figures \ref{fig:#1} and~\ref{fig:#2}}

\newcommand{\liref}[1]{line~\ref{li:#1}}
\newcommand{\Liref}[1]{Line~\ref{li:#1}}
\newcommand{\lirefs}[2]{lines \ref{li:#1}--\ref{li:#2}}
\newcommand{\Lirefs}[2]{Lines \ref{li:#1}--\ref{li:#2}}
\newcommand{\lireftwo}[2]{lines \ref{li:#1} and~\ref{li:#2}}
\newcommand{\lirefthree}[3]{lines \ref{li:#1}, \ref{li:#2}, and~\ref{li:#3}}

\newcommand{\lemlabel}[1]{\label{lem:#1}}
\newcommand{\lemref}[1]{Lemma~\ref{lem:#1}} 

\newcommand{\exref}[1]{Exercise~\ref{ex:#1}}

\newcommand{\handref}[1]{Handout~\ref{#1}}

\newcommand{\defref}[1]{Definition~\ref{def:#1}}

% (1997.8.16: Victor Luchangco)
% Modified \hlabel to only get date and to use handouts counter for number.
%   New \handout and \handoutwithouttitle commands in newmac.tex use this.
%   The date is referenced by <label>-date.
%   (Retained old definition as \hlabelold.)
%   Defined \hforcelabel to use an argument instead of the handouts counter.

\newcounter{handouts}
\setcounter{handouts}{0}

\newcommand{\hlabel}[2]{%
\stepcounter{handouts}
{\edef\next{\write\@auxout{\string\newlabel{#1}{{\arabic{handouts}}{0}}}}\next}
\write\@auxout{\string\newlabel{#1-date}{{#2}{0}}}
}

\newcommand{\hforcelabel}[3]{%          Does not step handouts counter.
\write\@auxout{\string\newlabel{#1}{{#2}{0}}}
\write\@auxout{\string\newlabel{#1-date}{{#3}{0}}}}


% less ugly underscore
% --juang, 2008 oct 05
\renewcommand{\_}{\vrule height 0 pt depth 0.4 pt width 0.5 em \,}

\newcommand{\theproblemsetnum}{3}
\newcommand{\releasedate}{Thursday, February 21}
\newcommand{\partaduedate}{Wednesday, February 27}
\allowdisplaybreaks

\title{6.046 Problem Set \theproblemsetnum}

\begin{document}

\handout{Problem Set \theproblemsetnum}{\releasedate}
\textbf{All parts are due {\bf \partaduedate} at {\bf 10PM}}.

\makeatletter
\newenvironment{breakablealgorithm}
  {% \begin{breakablealgorithm}
   \begin{center}
     \refstepcounter{algorithm}% New algorithm
     \hrule height.8pt depth0pt \kern2pt% \@fs@pre for \@fs@ruled
     \renewcommand{\caption}[2][\relax]{% Make a new \caption
       {\raggedright\textbf{\ALG@name~\thealgorithm} ##2\par}%
       \ifx\relax##1\relax % #1 is \relax
         \addcontentsline{loa}{algorithm}{\protect\numberline{\thealgorithm}##2}%
       \else % #1 is not \relax
         \addcontentsline{loa}{algorithm}{\protect\numberline{\thealgorithm}##1}%
       \fi
       \kern2pt\hrule\kern2pt
     }
  }{% \end{breakablealgorithm}
     \kern2pt\hrule\relax% \@fs@post for \@fs@ruled
   \end{center}
  }
\makeatother

\setlength{\parindent}{0pt}
\medskip\hrulefill\medskip

{\bf Name:} Robert Durfee

\medskip

{\bf Collaborators:} None

\medskip\hrulefill

\begin{problems}

\problem  % Problem 1

\begin{problemparts}

\problempart % Problem 1a
%%%%%%%%%%%%%%%%%%%%%%%%%%%%%%%%%%%%%%%%%%%%%%%%%%%%%%%%%%%%%%%%%%%%%%%%%%%%%%%%
\textbf{Description} While a crumb has yet to be found, travel to new locations 
according to the following: Let the current location index be $i$ (i.e. this is 
the $i$th location that Andy will visit, zero indexed). The location will be of 
the form
$$ \ell_i = (-1)^i \cdot 2^i $$

\textbf{Correctness} The proof of correctness is trivial. Andy must travel over
continuous sections back and forth from the origin, therefore, as he expands 
his search, we will cover more of the log without missing any sections where a 
possible crumb may be. 

Note, however, that Andy might not find the nearest crumb. He might be on his 
way to $\ell_i$ on the right side of the log and find a crumb and go home. Yet,
there could be a crumb located before $-\ell_i$ on the left side of the log.

\textbf{Running Time} To analyze the running time of this algorithm, we will
compare it to the best off-line algorithm. Trivially, the best off-line
algorithm can find a crumb located $d$ inches away in exactly $d$ steps. We
denote this as the optimal cost, $C^*(d)$ given the crumb at $d$.

Let $L_d$ be a log with with a single crumb on it located $d$ inches from the
origin. From the on-line algorithm description, Andy will travel to locations
$\ell_i = (-1)^i \cdot 2^i$. Therefore, at worst, Andy will visit $\ceil{
\log_2 d} + 1$ locations before discovering the crumb at $d$. This can be
seen if the crumb is located at $d = \ell_i + 1$, then Andy must visit 
$\ell_{i + 1}$ (on the other side) before he finds the crumb at $d = \ell_i 
+ 1$. 

Since the $\ell_i$ requires $2^i$ steps to reach and $2^i$ steps to return
back to the origin to reset the invariant, each location $\ell_i$ takes $2^{i
+ 1}$ steps to visit. From this, the cost to visit the $\ceil{\log_2 d} + 1$
steps is
$$ \sum_{k = 1}^{\ceil{\log_2 d} + 1} 2^k $$
To remove the floor function and leave an upper bound we add one to the
number of previously visited locations,
$$ \sum_{k = 1}^{\ceil{\log_2 d} + 1} 2^k \leq \sum_{k = 1}^{\log_2 d + 2} 
2^k $$
Using the rules governing geometric series, this simplifies to
$$\sum_{k = 1}^{\log_2 d + 2} 2^k = 8d - 2 $$
This is the total, worst-case cost to visit all other locations before the 
crumb located at $d$ is found (and returning to the origin). Now, we just 
need to travel $d$ more spaces to find the crumb. Therefore, total cost is
$$ C(d) = 8d - 2 + d = 9d - 2 $$
By comparison, the optimal cost was $C^*(d) = d$. Using the definition for
an $\alpha$-competitive algorithm,
$$ C(d) \leq \alpha \cdot C^*(d) + k $$
It is clear that if $\alpha = 9$ and $k > -2$ this condition is satisfied.
As a result, the algorithm is $9$-competitive.

\problempart % Problem 1b

If the starting direction is non-deterministic, then we can expect to
visit $\ceil{\log_2 d}$ previous locations half the time and $\ceil{\log_2 
d} + 1$ locations the other half of the time. This is clear because we no
longer have the worst-case guarantee that we would have to visit one more
location past the crumb on the opposite side of the log. We could get 
lucky and not overshoot.

Substituting the new, smaller number of previously visited locations, our
cost to visit the previous locations (and return to the origin) is,
$$ \sum_{k = 1}^{\ceil{\log_2 d}} 2^k \leq \sum_{k = 1}^{\log_2 d + 1} 
2^k = 4d - 2 $$
And our total cost of discovering the crumb in this case is
$$ C(d) = 4d - 2 + d = 5d - 2 $$
From this, our expected cost of discovering the crumb at location $d$ is
$$ \mathbb{E}[C(d)] = 0.5 (9d - 2) + 0.5 (5d - 2) = 7d - 2 $$
By comparison, the optimal cost was $C^*(d) = d$. Using the definition for
an $\alpha$-competitive algorithm,
$$ C(d) \leq \alpha \cdot C^*(d) + k $$
It is clear that if $\alpha = 7$ and $k > -2$ this condition is satisfied.
As a result, the algorithm is $7$-competitive in expectation.

\end{problemparts}

\newpage
\problem  % Problem 2

\textbf{Note:} Parts A and B have been combined into a single section.

%%%%%%%%%%%%%%%%%%%%%%%%%%%%%%%%%%%%%%%%%%%%%%%%%%%%%%%%%%%%%%%%%%%%%%%%%%%%%%%%
\textbf{Description} Let the data structure maintain the following attributes:

\begin{itemize}
    \item\ $emps$: An array of employees. Let the index $i$ represent the
        unique identifier of the employee (for simplicity, imagine 1-indexed
        array such that employees are in the range $[1, n]$). Each element of 
        the array should contain two properties:
        \begin{itemize}
            \item\ $parent$: A pointer to the parent employee as maintained
                by the union-find data structure.
            \item\ $ratio$: The ratio of the employee's bonus to their 
                `parent'.
            \item\ $rank$: The height of the subtree rooted at this
                employee.
        \end{itemize}
    \item\ $U$: An augmented union-find data structure. Initialize by 
        adding all $n$ employees into the data structure by calling 
        \texttt{uf.make\_set(i)} for all $i \in [1, n]$.
    \item\ $n_{comp}$: An integer representing the number of connected
        components in the union-find data structure. Initialize to $n$.
\end{itemize}

Let the augmented union-find data structure support the following altered
methods. If the method or attribute is not described, it operates the same as
presented in lecture.

\begin{itemize}
    \item\ \textproc{MakeSet}$(i)$: Creates a new set from a single employee. The
        representative of the set is the employee.
        \begin{breakablealgorithm}
            \caption{Augmented Make Set with Ratio Initialization}
            \begin{algorithmic}[1]
                \Function{MakeSet}{$i$}
                    \State $emps[i].parent \leftarrow emps[i]$
                    \State $emps[i].ratio \leftarrow 1$
                    \State $emps[i].rank \leftarrow 1$
                \EndFunction
            \end{algorithmic}
        \end{breakablealgorithm}
    \item\ \textproc{FindSet}$(i)$: Returns the representative for an element in a
        set and also returns the ratio of that employee's bonus to the the
        bonus of their representative. 
        \begin{breakablealgorithm}
            \caption{Augmented Find Set with Path Compression}
            \begin{algorithmic}[1]
                \Function{FindSet}{$i$}
                    \State $S \leftarrow \emptyset$
                    \State $emp \leftarrow emps[i]$
                    \State \Call{Push}{$S$, $emp$}
                    \While{$emp \neq emp.parent$}
                        \State $emp \leftarrow emp.parent$
                        \State \Call{Push}{$S$, $emp$}
                    \EndWhile
                    \State $root \leftarrow$ \Call{Pop}{$S$}
                    \State $ratio \leftarrow 1$
                    \While{$S \neq \emptyset$}
                        \State $emp \leftarrow$ \Call{Pop}{$S$} 
                        \State $ratio \leftarrow ratio \cdot emp.ratio$
                        \State $emp.ratio \leftarrow ratio$
                        \State $emp.parent \leftarrow root$
                    \EndWhile
                    \State \Return $root, ratio$
                \EndFunction
            \end{algorithmic}
        \end{breakablealgorithm}
    \item\ \textproc{Union}$(i, j)$: Replaces the separate sets containing the 
        employees with the union of the two sets implementing both union by rank
        and path compression. Updates the ratios such that they remain consistent.
        \begin{breakablealgorithm}
            \caption{Augmented Union by Rank Algorithm}
            \begin{algorithmic}[1]
                \Function{Union}{$i$, $j$, $x$}
                    \State $rep_i, ratio_i \leftarrow \Call{FindSet}{i}$
                    \State $rep_j, ratio_j \leftarrow \Call{FindSet}{j}$
                    \If{$rep_i.rank = rep_j.ratio$}
                        \State $rep_j.parent \leftarrow rep_i$
                        \State $rep_i.rank \leftarrow rep_i.rank + 1$
                        \State $rep_j.ratio \leftarrow ratio_i \cdot ratio_j^{-1} \cdot x^{-1}$
                    \ElsIf{$rep_i.rank > rep_j.rank$}
                        \State $rep_j.parent \leftarrow rep_i$
                        \State $rep_j.ratio \leftarrow ratio_i \cdot ratio_j^{-1} \cdot x^{-1}$
                    \ElsIf{$rep_i.rank < rep_j.rank$}
                        \State $rep_i.parent \leftarrow rep_j$
                        \State $rep_i.ratio \leftarrow ratio_i^{-1} \cdot ratio_j \cdot x$
                    \EndIf
                \EndFunction
            \end{algorithmic}
        \end{breakablealgorithm}
\end{itemize}

Let the data structure have the following methods:
%%%%%%%%%%%%%%%%%%%%%%%%%%%%%%%%%%%%%%%%%%%%%%%%%%%%%%%%%%%%%%%%%%%%%%%%%%%%%%%%
\begin{itemize}
    \item\ \textproc{SetRatio}$(i, j, x)$: Performs the union between two 
        employees and sets the ratio of bonuses between them. This is only
        necessary if the employees has not yet been related implicitly or 
        explicitly.
        \begin{breakablealgorithm}
            \caption{Set the ratio between two employees}
            \begin{algorithmic}[1]
                \Function{SetRatio}{$i$, $j$, $x$}
                    \State $rep_i, _ \leftarrow \Call{FindSet}{i}$
                    \State $rep_j, _ \leftarrow \Call{FindSet}{j}$
                    \If{$rep_i \neq rep_j$}
                        \State $n_{comp} \leftarrow n_{comp} - 1$
                        \State \Call{Union}{$i$, $j$, $x$}
                    \EndIf
                \EndFunction
            \end{algorithmic}
        \end{breakablealgorithm}
    \item\ \textproc{GetRatio}$(i, j)$: Confirms that the two employees have
        been related either implicitly or explicitly and then returns the ratio
        of bonuses.
        \begin{breakablealgorithm}
            \caption{Get the ratio between two employees}
            \begin{algorithmic}[1]
                \Function{GetRatio}{$i$, $j$}
                    \State $rep_i, ratio_i \leftarrow \Call{FindSet}{i}$
                    \State $rep_j, ratio_j \leftarrow \Call{FindSet}{j}$
                    \If{$rep_i = rep_j$}
                        \State \Return $ratio_i \cdot ratio_j^{-1}$ 
                    \Else
                        \State \Return \textsc{none}
                    \EndIf
                \EndFunction
            \end{algorithmic}
        \end{breakablealgorithm}
    \item\ \textproc{PayDay}$()$: Checks if the number of connected components
        in the union find data structure is 1.
        \begin{breakablealgorithm}
            \caption{Indicates if all of the employees have been related}
            \begin{algorithmic}[1]
                \Function{PayDay}{}
                    \If{$n_{comp} = 1$}
                        \State \Return \textsc{true}
                    \Else
                        \State \Return \textsc{false}
                    \EndIf
                \EndFunction
            \end{algorithmic}
        \end{breakablealgorithm}
\end{itemize}

\textbf{Correctness} We wish to maintain a few invariants we depend on 
throughout execution:

\begin{itemize}
    \item\ Each employee's $ratio$ attribute represents the bonus ratio to 
    their parent.
    \item\ Each employee's $rank$ represents the size of their subtree.
    \item\ The total number of connected components is correct.
\end{itemize}

We can examine the correctness of each of the functions in turn confirming
correct results and maintenance of the invariants. Note the third invariant
is true at initialization as there are $n$ employees and therefore $n$ 
connected components (as none have been related yet).

\begin{itemize}
    \item\ \textproc{MakeSet}$(i)$: Each employee is currently unrelated.
        Trivially, each employee's $ratio$ to their parent (themselves) is
        $1$ and therefore correct. The size of their subtree is just
        themselves so their $rank$ is trivially $1$. The third invariant is 
        not used in this algorithm. Thus all invariants are upheld. 
    \item\ \textproc{FindSet}$(i)$: The second invariant is held by appealing to
        the proof given in lecture as the path compression functionality of the
        algorithm is unchanged. The third invariant is not used in this algorithm.
        
        The first invariant needs to be examined. In 
        the algorithm, we traverse parent pointers in reverse. Without loss of 
        generality, assume the ancestors of employee $i$ are employees $[1..i 
        - 1]$ such that employee $1$ is the root and employee $i - 1$ is employee
        $i$'s parent. As we traverse, we compute the following product
        $$ \frac{q(1)}{q(1)} \frac{q(2)}{q(1)} \ldots \frac{q(i - 1)}{q(i - 2)} 
        \frac{q(i)}{q(i - 1)} = \frac{q(i)}{q(1)} $$
        This is seen by assuming the invariant is true and that all employee's
        $ratio$ is between themselves and their parent. The product collapses
        and we are left with the ratio of the $i$th employee's bonus to the 
        root's bonus. Thus, when we compress the and set the $i$th employee's
        parent to the root, the employee's $ratio$ is maintained. Also, since
        all the children depend only on their parent, the subtree ratios are
        unchanged and thus correct.
        
        Furthermore, it is clear that the returned pointer is the representative
        as each employee points to their parent and the final ancestor points
        to themselves. When we reach this point, this is the representative as
        stated in lecture.
        
        The returned ratio is correct as shown by the rank product 
        calculated from top to bottom. It collapses to a ratio between the $i$th
        employee and their representative.
    \item\ \textproc{Union}$(i, j)$: The second invariant is held by appealing to
        the proof given in lecture as the union by rank functionality of the
        algorithm is unchanged. The third invariant is not used in this algorithm.
        
        The first invariant needs to be examined. It has been shown that the 
        \textproc{FindSet}$(i)$ algorithm returns the correct ratio and
        representative. Let $\bar{i}$ and $\bar{j}$ be the representatives for
        employees $i$ and $j$ respectively. The returned ratios must be
        $$ ratio_i = \frac{q(i)}{q(\bar{i})},\quad ratio_j = \frac{q(j)}{q(
        \bar{j})} $$
        Given that $x$ must be
        $$ x = \frac{q(i)}{q(j)} $$
        We can show that the update $rep_j.parent \leftarrow rep_i$ upholds the 
        correct ratios. The update step written in detail,
        \begin{align*}
            rep_j.ratio &\leftarrow ratio_i \cdot ratio_j^{-1} \cdot x^{-1} \\
            &\leftarrow \left(\frac{q(i)}{q(\bar{i})}\right) \left(
            \frac{q(j)}{q(\bar{j})}\right)^{-1} \left(\frac{q(i)}{q(j)}
            \right)^{-1} \\
            &\leftarrow \left(\frac{q(i)}{q(\bar{i})}\right) \left(
            \frac{q(\bar{j})}{q(j)}\right) \left(\frac{q(j)}{q(i)}
            \right) \\
            &\leftarrow \frac{q(\bar{j})}{q(\bar{i})}
        \end{align*}
        This is the correct ratio as the representative of $j$ should have a 
        ratio with respect their new representative, which is the representative
        of $i$. By inverting the result, we can also see this is correct with 
        the update step $rep_i.parent \leftarrow rep_j$. 
        
        Furthermore, since the other ratios in either group only depend on their
        parents, which are unchanged, ratios are maintained for all members. 
        Thus, the first invariant is upheld.
    \item\ \textproc{SetRatio}$(i, j, x)$: The first and second invariants are 
        upheld by appealing to the proofs above.
        
        For the third invariant, we recognize that we only decrease the number of
        connected components if we union two previously unconnected components.
        This action decreases the number of connected components by one. Thus, the
        invariant is maintained.
        
        The correctness of the action is apparent by appealing to the proof of 
        \textproc{Union}.
    \item\ \textproc{GetRatio}$(i, j)$: All three invariants are maintained as no
        alterations to the data structure result.
        
        Given that \textproc{FindSet} returns the correct representatives, if $i$ and
        $j$ have the same representative, then they must be implicitly related and 
        therefore we can return the ratio. The returned ratios from \textproc{FindSet}
        are
        $$ ratio_i = \frac{q(i)}{q(\bar{i})},\quad ratio_j = \frac{q(j)}{q(\bar{j})} $$
        The ratio between them is computed as
        \begin{align*}
            ratio &\leftarrow ratio_i \cdot ratio_j^{-1} \\
            &\leftarrow \left(\frac{q(i)}{q(\bar{i})}\right)
            \left(\frac{q(j)}{q(\bar{j})}\right)^{-1} \\
            &\leftarrow \left(\frac{q(i)}{q(\bar{i})}\right)
            \left(\frac{q(\bar{j})}{q(j)}\right)
        \end{align*}
        Since $\bar{i} = \bar{j}$, this becomes,
        $$ ratio \leftarrow \frac{q(i)}{q(j)} $$
        Therefore, the correct output is given.
    \item\ \textproc{PayDay}$()$: All invariants are maintained because no 
        modification of the data occurs.
        
        Given that the third invariant is maintained,  $n_{comp}$ must 
        represent the number of connected components. If there is only one 
        connected component, then all employees are related.
\end{itemize}

\textbf{Running Time} We can examine the running time of each method in the 
union find data structure to confirm no changes in run time.

\begin{itemize}
    \item\ \textproc{MakeSet}$(i)$: The functionality of this algorithm remains
        the same as that given in lecture, aside from constant time modifications
        used to initialize an additional parameter $ratio$.
    \item\ \textproc{FindSet}$(i)$: The functionality of this algorithm remains
        the same as that given in lecture, aside from constant time modifications
        used to uphold an additional parameter $ratio$.
    \item\ \textproc{Union}$(i, j)$: The functionality of this algorithm remains
        the same as that given in lecture, aside from constant time modifications
        used to uphold an additional parameter $ratio$.
\end{itemize}

Since none of the functions alter the run time beyond additional constants, all
operations remain amortized $O(\alpha(n))$ in the number of employees.

\begin{itemize}
    \item\ \textproc{SetRatio}$(i, j, x)$: This function calls \textproc{FindSet}
        two times and \textproc{Union} at most once. Therefore, the cost is
        amortized $O(2 \alpha(n) + \alpha(n)) \in O(\alpha(n))$.
    \item\ \textproc{GetRatio}$(i, j)$: This function calls \textproc{FindSet}.
        Therefore, the cost is amortized $O(2 \alpha(n)) \in O(\alpha(n))$
    \item\ \textproc{PayDay}$()$: This algorithm only has a constant amount of
        work as the $n_{comp}$ is maintained elsewhere. Thus it is trivially
        amortized $O(\alpha(n))$.
\end{itemize}

Therefore, the data structure is amortized $O(\alpha(n))$ for all operations.

\end{problems}

\end{document}


