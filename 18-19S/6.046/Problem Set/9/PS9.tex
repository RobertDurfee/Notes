%
% 6.046 problem set 2 solutions
%
\documentclass[12pt,twoside]{article}
\usepackage{multirow}
\usepackage{algorithm, algpseudocode, float}

\input{macros}
\usepackage{enumitem}
\newcommand{\theproblemsetnum}{9}
\newcommand{\releasedate}{Thursday, April 25}
\newcommand{\partaduedate}{Wednesday, May 1}
\allowdisplaybreaks

\title{6.046 Problem Set \theproblemsetnum}

\begin{document}

\handout{Problem Set \theproblemsetnum}{\releasedate}
\textbf{All parts are due {\bf \partaduedate} at {\bf 10PM}}.

\makeatletter
\newenvironment{breakablealgorithm}
  {% \begin{breakablealgorithm}
   \begin{center}
     \refstepcounter{algorithm}% New algorithm
     \hrule height.8pt depth0pt \kern2pt% \@fs@pre for \@fs@ruled
     \renewcommand{\caption}[2][\relax]{% Make a new \caption
       {\raggedright\textbf{\ALG@name~\thealgorithm} ##2\par}%
       \ifx\relax##1\relax % #1 is \relax
         \addcontentsline{loa}{algorithm}{\protect\numberline{\thealgorithm}##2}%
       \else % #1 is not \relax
         \addcontentsline{loa}{algorithm}{\protect\numberline{\thealgorithm}##1}%
       \fi
       \kern2pt\hrule\kern2pt
     }
  }{% \end{breakablealgorithm}
     \kern2pt\hrule\relax% \@fs@post for \@fs@ruled
   \end{center}
  }
\makeatother

\setlength{\parindent}{0pt}
\medskip\hrulefill\medskip

{\bf Name:} Robert Durfee

\medskip

{\bf Collaborators:} None

\medskip\hrulefill

\begin{problems}

\problem  % Problem 1

\begin{problemparts}

\problempart  % Problem 1a

To show that the scheduling problem is in NP, we show the size of the
verification certificiate, $|y|$, is polynomial in the size of the decision
problem input, $|x|$. That is, $|y| \leq |x|^c$ for some constant $c$.

\begin{itemize}

  \item {\bf Polynomial Certficate} In this problem, the input to the
  decision problem, $x$, is a list of students, $S$, and each students' list
  of classes (where each list if an element of $C$). To verify the scheduling
  problem decision, we require a certificate, $y$, that is a list of students
  $S$, a list of each students' classes $C$, and a list of exam slots
  assigned to each class $E$.

  This certificate is clearly polynomial in $|x|$ because we simply require
  the original $S$ and $C$ input to the decision problem, which is $x$, and
  an additional constant amount of information about each individual class
  (the exam assignment for each class which is one of three options).

  \item {\bf Polynomial Verification} The verifier of the certificate simply
  needs to iterate over each student's classes and see if any of those
  classes' exam assigments are equal. To check if any of the elements are
  equal, just keep a set of all exam slots represented. If for all the
  classes for each student do not have the same exam slot occupied more than
  once, then the certificate is valid. 

  Since the verifier only needs to look at each class for each student once,
  the runtime is clearly polynomial in $|x|$.

\end{itemize}

Since there is a polynomial-sized certificate and a polynomial-time
verification algorithm, the scheduling problem is in NP.

\problempart  % Problem 1b

To show the scheduling problem is equivalent to the 3-colorable problem,
we first reduce the scheduling problem to the 3-colorable problem. Then we
reduce the 3-colorable problem to the scheduling problem. In both cases, we
show that both yield equivalent results.

\begin{itemize}

  \item {\bf 3-Colorable to Scheduling Reduction} Assume we have a method to
  decide if a scheduling is possible for students $S$ with classes $C$ into
  one of three non-conflicting exam slots. If we are given a graph $G = (V,
  E)$, let $F$ be the algorithm which creates a vertex to represent each class.
  Then, if two vertices $u, v$ have an edge going between them, let $F$
  create a student and give them two classes corresponding to $u, v$. The
  constructed schedule is clearly an instance of the scheduling problem.

  The algorithm $F$ runs in polynomial time in the size of the input to the
  3-colorable problem $|x|$ as each vertex is visited once and each edge is
  also visited once.

  \begin{itemize}

    \item {\bf Scheduling $\implies$ 3-Coloring} If we are given a valid
    scheduling of the classes into the exam slots, we can easily turn this
    into a valid three coloring since there are only three exam slots. If a
    class is scheduled to exam slot one, color the corresponding vertice with
    the first color (respectively for the remaining two exam slots). Continue
    until all classes are examined. Since vertices are only connected when
    their exists a student taking both classes, if the scheduling is valid,
    these classes won't have exams at the same time and thus the coloring is
    valid.

    \item {\bf 3-Coloring $\implies$ Scheduling} If we are given a valid
    3-coloring of the graph, we can easily turn this into a valid scheduling
    as there are only three exam slots. If a vertice is colored the first
    color, schedule the class's exam into exam slot one (respectively for the
    remaining two colors). Continue until all vertices are visited. Since
    classes are taken by the same student only when the graph has an edge
    between two vertices, if the coloring is valid, these adjacent vertices
    won't have the same coloring and thus the scheduling is valid.

  \end{itemize}

  \item {\bf Scheduling to 3-Colorable Reduction} Assume we have a method to
  determine if a graph has a valid 3-coloring. If we are given a set of
  students $S$ and their corresponding classes $C$, let $F$ be the algorithm
  which constructs a graph $G = (V, E)$ where the vertices represent all the
  possible classes any student takes. Let $F$ connect any two vertices $u, v$
  if a student is taking both classes represented by the vertices $u, v$. The
  constructed graph $G$ is clearly an instance of the 3-colorable problem.

  The algorithm $F$ runs in polynomial time in the size of input to the
  scheduling problem $|x|$ as each student is visited only once and the edges
  between the classes forms a complete subgraph which has $O(n^2)$ edges
  (where $n$ is the number of classes the student is taking). Since $n$ is
  bounded by the size of the input $|x|$, this can all be computed in
  polynomial time.

  \begin{itemize}

    \item {\bf Scheduling $\implies$ 3-Coloring} Same proof as above.

    \item {\bf 3-Coloring $\implies$ Scheduling} Same proof as above.

  \end{itemize}

\end{itemize}

Since we have reduced the 3-colorable problem to the scheduling problem and
the scheduling problem to the 3-colorable problem, these two must be
equivalent.

\end{problemparts}

\newpage
\problem  % Problem 2

\begin{problemparts}

\problempart  % Problem 2a

To show there exists a polynomial-time algorithm that finds an optimal subset
of decides that there is no subset that satisfies the constraint, we
demonstrate such an algorithm (not necessarily the most optimal).

{\bf Description} Construct an undirected graph $G = (V, E)$ such that each
vertice represents an individual at the hackathon. Let an edge $\{u, v\}$
exist if hacker $u$ dislikes a hacker $v$. Furthermore, let there be a
weight function $w : V \longrightarrow \mathbb{Z}^+$ which represents the
leetness for each hacker. 

Identify all the connected components in $G$ by running graph traversal. For
each connected component, run the bipartite detection algorithm described in
Problem Set 7 Part 2b. If any of the connected components is not bipartite
(and contains more than one node), there is no subset that satisfies the
constraint. 

If all connected components are bipartite, start with a given node, color it
one of two colors, and recursively visit each adjacent node, coloring it the
opposite of parent. While traversing, keep track of the sum of weights $w$ of
the vertices for each respective color.

For each individual, connected component, choose the color with the highest
weight and take all the vertices of that color to put in the tent.

{\bf Correctness} The graph $G$ is bipartite if and only if the graph can be
separated into two groups such that every edge connects from one group to the
other. Equivalently, this is the same as all hackers who dislike each other
being separated from each other. Therefore, if a bipartite matching is
impossible, then the constraint cannot be met.

Since an unconnected graph has non-unique bipartite groupings, we most
consider each connected component individually. Each connected component does
have a unique bipartite grouping. Therefore, for each component, we can start
with a given node, color it one of two colors, and recursively visit each
adjacent node, coloring it the opposite of the color of the parent. This will
determine the unique 2-coloring.

Then, each component will have a color which has a greater sum of leetness.
Choose that color for each component and group them together to be in the
tent. They cannot dislike each other because the components are disconnected.

{\bf Running Time} Graph creation is clearly polynomial in the number of
hackers as each vertice is a hacker and there are at most $O(n^2)$ dislike
relationships. Graph traversal is also polynomial in the number of hackers.
The algorithm for identifying bipartiteness is also polynomial in the number
of hackers. Lastly, traversing each component is also polynomial if each
vertex is visited only once. Since all aspects of the algorithm are
polynomial, the complete algorithm is also polynomial.

Since this algorithm is correct and runs in polynomial time, it is possible
to solve the problem is polynomial time.

\problempart  % Problem 2b

To show the hackers problem is NP-complete, we first show that it is in NP.
Then we reduce a special case of the 3-SAT problem (where not all elements
can be equal) to the hackers problem.

\begin{itemize}

  \item {\bf Hackers is in NP} To show that the hackers problem is in NP, we
  must show the size of the verification certificiate, $|y|$, is polynomial
  in the size of the decision input, $|x|$. That is, $|y| \leq |x|^c$ for
  some constant $c$.

  \begin{itemize} 

    \item {\bf Polynomial Certficate} In this problem, the input to the
    decision problem $x$ is a set of groups of hackers $S_1, \ldots, S_m$
    (not necessarily disjoint) who, if put together, will cause trouble. To
    verify the hackers problem decision, we require a certificate $y$ that is
    the groups of hackers $S_1, \ldots, S_m$ who are troublesome if placed
    together and a list of each individual hacker's group placement $P$.

    This certificate is clearly polynomial in $|x|$ because we simply require
    the original $S_1, \ldots, S_m$, which is $x$, and an additional list of
    the placement of each hacker. Since the number of hackers is bounded by
    the number of hackers in each group, the overall size of the certificate
    is polynomially bounded by $|x|$.

    \item {\bf Polynomial Verification} The verifier of the certificate
    simple needs to iterate over each group of troublesome hackers keeping
    track of the hackers' corresponding assignments. If within each group,
    both $V_1$ and $V_2$ are represented by the hackers, then the certificate
    is valid.

    Since the verifier only needs to look at each member of each group once,
    the runtime is clearly polynomial in $|x|$.

  \end{itemize}

  \item {\bf Hackers is NP-Hard} To show that the hackers problem is NP-hard,
  we reduce a special case of the 3-SAT problem (where not all elements can
  be equal) to the hackers problem and show the two yield equivalent results.

  \begin{itemize}

    \item {\bf Special 3-SAT to Hackers Reduction} Assume we have a method to
    determine if the hackers problem is possible. If we are given a boolean
    expression for the special 3-SAT problem, let $F$ be the algorithm which
    constructs two troublesome hackers for each variable in the boolean
    expression. Let one hacker represent the variable and the other represent
    the not of the variable. Have $F$ put both of these hackers in the same
    troublesome group. Now for each clause, let $F$ place the hackers
    corresponding to each literal in a troublesome group (choosing the hacker
    representing the not literal as necessary). The constructed set of
    troublesome groups is clearly an instance of the hackers problem.

    The algorithm $F$ clearly runs in polynomial time in the size of the
    input as each literal is visited once and then the expression is scanned
    over once. (Note the number of literals is bounded by the size of the
    expression.) At each step, only a constant amount of work is necessary to
    create a troublesome group. Therefore, $F$ is polynomial in $|x|$.

    \item {\bf Hackers $\implies$ Special 3-SAT} If we are given a valid
    solution to the hackers problem, we can easily turn this into a valid
    solution to the special 3-SAT problem. Each troublesome group must have
    at least one member separated from the others. Then the special 3-SAT
    solution must have at least one $\mathrm{False}$ literal and at least one
    $\mathrm{True}$ literal corresponding to hackers in groups $V_1$ and
    $V_2$. Furthermore, negated literal relationships are enforced by putting
    both the hacker and the not-hacker in the same separate group such that
    both cannot be in the same group.
    
    \item {\bf Special 3-SAT $\implies$ Hackers} If we are given a valid
    solution to the special 3-SAT problem, we can easily turn this into a
    valid solution to the hackers problem. Each literal in a clause
    corresponds to a hacker in a troublesome group. Each clause must have at
    least one $\mathrm{True}$ and one $\mathrm{False}$. Therefore, each group
    of troublesome hackers must have at least one member in $V_1$ and at
    least on member in $V_2$ for each clause. Therefore a valid special 3-SAT
    implies a valid hacker partition.

  \end{itemize}

\end{itemize}

Since we have shown that special 3-SAT (known to be NP-complete from Part C)
can be correctly reduced to the hackers problem and that hackers problem is
itself in NP, the hackers problem must be NP-complete.

\problempart  % Problem 2c

To show the special 3-SAT problem is NP-complete, we first show that it is in
NP. Then we reduce regular 3-SAT problem to the special case of the 3-SAT
problem (where not all elements can be equal).

\begin{itemize}

  \item {\bf Special 3-SAT is in NP} To show that the special 3-SAT problem
  is in NP, we must show the size of the verification certificiate, $|y|$, is
  polynomial in the size of the decision input, $|x|$. That is, $|y| \leq
  |x|^c$ for some constant $c$.

  \begin{itemize} 

    \item {\bf Polynomial Certficate} In this problem, the input to the
    decision problem $x$ is a boolean expression. To verify the
    satisfiability of the expression, we require a certificate $y$ that is
    the boolean expression and each literal's truth assignment. 

    This certificate is clearly polynomial in $|x|$ becuase we simple require
    the original expression, which is $|x|$ and information about each
    literal's truth assignment. Since the number of literals is bounded by
    the size of the boolean expression (you can't have more literals than the
    size of the expression!), the certificate is bounded polynomially by
    $|x|$.

    \item {\bf Polynomial Verification} The verifier of the certificate
    simply needs to iterate over all CNF clauses of three literals to check
    if the clause equals to one and that at least one literal evaluates to
    false. If all clauses evaluate to one and each clause has at least one
    zero, the certificate is valid.

    Since the verifier just needs to scan over the boolean expression once,
    the runtime is clearly polynomial in $|x|$.

  \end{itemize}

  \item {\bf Special 3-SAT is NP-Hard} To show that the special 3-SAT problem
  is NP-hard, we reduce the regular 3-SAT problem to the special case 3-SAT
  problem (where not all elements can be equal) to the hackers problem and
  show the two yield equivalent results.

  \begin{itemize}

    \item {\bf 3-SAT to Special 3-SAT Reduction} Assume we have a method to
    determine if a boolean expression is satisfiable such that each CNF
    clause of three variables has at least one false literal. If we are given
    a CNF boolean expression with clauses of three literals as an input to
    the regular 3-SAT problem, let $F$ be the algorithm that, for each
    clause $(x_1 \lor x_2 \lor x_3)$, constructs the following two clauses:
    $$ (x_1 \lor x_2 \lor c) \land (x_3 \lor \overline{c} \lor
    \mathrm{False}) $$
    If $x_1 \lor x_2 = \mathrm{True}$, then let $c = \mathrm{False}$
    otherwise, let $c = \mathrm{True}$. Clearly this is an instance of the
    special 3-SAT problem as the boolean expression is still in CNF with
    three literals in each clause.

    The algorithm $F$ runs in polynomial time in the size of the input to the
    3-SAT problem $|x|$ as each clause is visited only once and the number of
    clauses is bounded by the size of the boolean expression. At each visit,
    only a constant amount of work is required to construct the corresponding
    two clauses.

    \item {\bf Special 3-SAT $\implies$ 3-SAT} If we are given a valid
    literal assignment for a special 3-SAT problem, we can easily turn this
    into a valid solution to regular 3-SAT. Consider the following truth
    table for all possible satisfiable clause pairs in the special 3-SAT
    problem,

    \begin{center}
      \begin{tabular}{c c c c c c c c c c c}
        $(x_1$ & $\lor$ & $x_2$ & $\lor$ & $c)$ & $\land$ & $(x_3$ & $\lor$ &
        $\overline{c}$ & $\lor$ & $\mathrm{F})$ \\
        \hline \hline
        1 && 1 && 0 && 1 && 1 && 0 \\
        1 && 1 && 0 && 0 && 1 && 0 \\
        0 && 1 && 0 && 1 && 1 && 0 \\
        0 && 1 && 0 && 0 && 1 && 0 \\
        1 && 0 && 0 && 1 && 1 && 0 \\
        1 && 0 && 0 && 0 && 1 && 0 \\
        0 && 0 && 1 && 1 && 0 && 0 \\
        1 && 0 && 1 && 1 && 0 && 0 \\
        0 && 1 && 1 && 1 && 0 && 0
      \end{tabular}
    \end{center}

    Each of these imply a corresponding solution to regular 3-SAT,

    \begin{center}
      \begin{tabular}{c c c c c}
        $(x_1$ & $\lor$ & $x_2$ & $\lor$ & $x_3)$ \\
        \hline \hline
        1 && 1 && 1 \\
        1 && 1 && 0 \\
        0 && 1 && 1 \\
        0 && 1 && 0 \\
        1 && 0 && 1 \\
        1 && 0 && 0 \\
        0 && 0 && 1 \\
        1 && 0 && 1 \\
        0 && 1 && 1 
      \end{tabular}
    \end{center}

    Therefore, any satisfiable clause in special 3-SAT implies a satisfiable
    clause in regular 3-SAT.

    \item {\bf 3-SAT $\implies$ Special 3-SAT} If we are given a valid
    literal assignment for a regular 3-SAT problem, we can easily turn this
    into a valid solution to the special 3-SAT problem. Consider the
    following truth table for all possible satisfiable clauses in the regular
    3-SAT problem,
    
    \begin{center}
      \begin{tabular}{c c c c c}
        $(x_1$ & $\lor$ & $x_2$ & $\lor$ & $x_3)$ \\
        \hline \hline
        1 && 1 && 1 \\
        1 && 1 && 0 \\
        1 && 0 && 1 \\
        0 && 1 && 1 \\
        0 && 0 && 1 \\
        0 && 1 && 0 \\
        1 && 0 && 0
      \end{tabular}
    \end{center}

    If we follow the rule that $c = \mathrm{False}$ if $x_1 \lor x_2 =
    \mathrm{True}$, otherwise let $c = \mathrm{False}$, each of these imply a
    solution to special 3-SAT,

    \begin{center}
      \begin{tabular}{c c c c c c c c c c c}
        $(x_1$ & $\lor$ & $x_2$ & $\lor$ & $c)$ & $\land$ & $(x_3$ & $\lor$ &
        $\overline{c}$ & $\lor$ & $\mathrm{F})$ \\
        \hline \hline
        1 && 1 && 0 && 1 && 1 && 0\\
        1 && 1 && 0 && 0 && 1 && 0\\
        1 && 0 && 0 && 1 && 1 && 0\\
        0 && 1 && 0 && 1 && 1 && 0\\
        0 && 0 && 1 && 1 && 0 && 0\\
        0 && 1 && 0 && 0 && 1 && 0\\
        1 && 0 && 0 && 0 && 1 && 0
      \end{tabular}
    \end{center}

    Therefore, any satisfiable clause in regular 3-SAT implies a satisfiable
    clause pair in special 3-SAT.
    
  \end{itemize}

\end{itemize}

Since we have shown that 3-SAT (known to be NP-complete) can be correctly
reduced to special 3-SAT and that special 3-SAT is itself in NP, special
3-SAT must be NP-complete.

\end{problemparts}

\end{problems}

\end{document}
