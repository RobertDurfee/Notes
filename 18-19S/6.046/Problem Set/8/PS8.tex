%
% 6.046 problem set 2 solutions
%
\documentclass[12pt,twoside]{article}
\usepackage{multirow}
\usepackage{algorithm, algpseudocode, float}

\newcommand{\name}{}

\usepackage{amssymb}
\usepackage{amsmath}
\usepackage{graphicx}
\usepackage{latexsym}
\usepackage{times,url}
\usepackage{cprotect}
\usepackage{listings}
\usepackage{graphicx}
\usepackage[table]{xcolor}
\usepackage[letterpaper]{geometry}
\usepackage{tikz-qtree}
\usepackage{enumerate}

\newcommand{\profs}{Mauricio Karchmer, Aleksander Madry, Bruce Tidor}
\newcommand{\subj}{6.046}
\newcommand{\ttt}[1]{{\tt\small #1}}

\definecolor{dkgreen}{rgb}{0,0.6,0}
\definecolor{gray}{rgb}{0.5,0.5,0.5}
\definecolor{mauve}{rgb}{0.58,0,0.82}

\lstset{
  language=Python,
  aboveskip=1pc,
  belowskip=1pc,
  basicstyle={\footnotesize\ttfamily},
  numbers=left,
  showstringspaces=false,
  numberstyle=\tiny\color{gray},
  keywordstyle=\color{blue},
  commentstyle=\color{dkgreen},
  stringstyle=\color{mauve},
}

\tikzset{
  % every node/.style={minimum width=2em,draw,circle},
  % level 1/.style={sibling distance=2cm},
  level distance=1cm,
  edge from parent/.style=
  {draw,edge from parent path={(\tikzparentnode) -- (\tikzchildnode)}},
}

\newif\ifHideSolutions
\newcommand{\solution}[1]{\color{dkgreen}\textbf{Solution: }#1\color{black}}
\newcommand{\rubric}[1]{\color{dkgreen}{\bf Rubric:} #1\color{black}}

% \HideSolutionsfalse
% \ifHideSolutions
%   \renewcommand{\solution}[1]{}
%   \renewcommand{\rubric}[1]{}
% \fi

\newlength{\toppush}
\setlength{\toppush}{2\headheight}
\addtolength{\toppush}{\headsep}

\newcommand{\htitle}[2]{\noindent\vspace*{-\toppush}\newline\parbox{6.5in}
{\textit{Design and Analysis of Algorithms}\hfill\name\newline
Massachusetts Institute of Technology \hfill #2\newline
\profs\hfill #1 \vspace*{-.5ex}\newline
\mbox{}\hrulefill\mbox{}}\vspace*{1ex}\mbox{}\newline
\begin{center}{\Large\bf #1}\end{center}}

\newcommand{\handout}[2]{\thispagestyle{empty}
 \markboth{#1}{#1}
 \pagestyle{myheadings}\htitle{#1}{#2}}

\newcommand{\lecture}[3]{\thispagestyle{empty}
 \markboth{Lecture #1: #2}{Lecture #1: #2}
 \pagestyle{myheadings}\htitle{Lecture #1: #2}{#3}}

\newcommand{\htitlewithouttitle}[2]{\noindent\vspace*{-\toppush}\newline\parbox{6.5in}
{\textit{Design and Analysis of Algorithms}\hfill#2\newline
Massachusetts Institute of Technology \hfill 6.046\newline
\profs\hfill Handout #1\vspace*{-.5ex}\newline
\mbox{}\hrulefill\mbox{}}\vspace*{1ex}\mbox{}\newline}

\newcommand{\handoutwithouttitle}[2]{\thispagestyle{empty}
 \markboth{Handout \protect\ref{#1}}{Handout \protect\ref{#1}}
 \pagestyle{myheadings}\htitlewithouttitle{\protect\ref{#1}}{#2}}

\newcommand{\exam}[2]{% parameters: exam name, date
 \thispagestyle{empty}
 \markboth{\hspace{1cm}\subj\ #1\hspace{1in}Name\hrulefill\ \ }%
          {\subj\ #1\hspace{1in}Name\hrulefill\ \ }
 \pagestyle{myheadings}\examtitle{#1}{#2}
 \renewcommand{\theproblem}{Problem \arabic{problemnum}}
}
\newcommand{\examsolutions}[3]{% parameters: handout, exam name, date
 \thispagestyle{empty}
 \markboth{Handout \protect\ref{#1}: #2}{Handout \protect\ref{#1}: #2}
% \pagestyle{myheadings}\htitle{\protect\ref{#1}}{#2}{#3}
 \pagestyle{myheadings}\examsolutionstitle{\protect\ref{#1}} {#2}{#3}
 \renewcommand{\theproblem}{Problem \arabic{problemnum}}
}
\newcommand{\examsolutionstitle}[3]{\noindent\vspace*{-\toppush}\newline\parbox{6.5in}
{\textit{Design and Analysis of Algorithms}\hfill#3\newline
Massachusetts Institute of Technology \hfill 6.046\newline
%Singapore-MIT Alliance \hfill SMA5503\newline
\profs\hfill Handout #1\vspace*{-.5ex}\newline
\mbox{}\hrulefill\mbox{}}\vspace*{1ex}\mbox{}\newline
\begin{center}{\Large\bf #2}\end{center}}

\newcommand{\takehomeexam}[2]{% parameters: exam name, date
 \thispagestyle{empty}
 \markboth{\subj\ #1\hfill}{\subj\ #1\hfill}
 \pagestyle{myheadings}\examtitle{#1}{#2}
 \renewcommand{\theproblem}{Problem \arabic{problemnum}}
}

\makeatletter
\newcommand{\exambooklet}[2]{% parameters: exam name, date
 \thispagestyle{empty}
 \markboth{\subj\ #1}{\subj\ #1}
 \pagestyle{myheadings}\examtitle{#1}{#2}
 \renewcommand{\theproblem}{Problem \arabic{problemnum}}
 \renewcommand{\problem}{\newpage
 \item \let\@currentlabel=\theproblem
 \markboth{\subj\ #1, \theproblem}{\subj\ #1, \theproblem}}
}
\makeatother


\newcommand{\examtitle}[2]{\noindent\vspace*{-\toppush}\newline\parbox{6.5in}
{\textit{Design and Analysis of Algorithms}\hfill#2\newline
Massachusetts Institute of Technology \hfill 6.046 Spring 2019\newline
%Singapore-MIT Alliance \hfill SMA5503\newline
\profs\hfill #1\vspace*{-.5ex}\newline
\mbox{}\hrulefill\mbox{}}\vspace*{1ex}\mbox{}\newline
\begin{center}{\Large\bf #1}\end{center}}

\newcommand{\grader}[1]{\hspace{1cm}\textsf{\textbf{#1}}\hspace{1cm}}

\newcommand{\points}[1]{[#1 points]\ }
\newcommand{\parts}[1]
{
  \ifnum#1=1
  (1 part)
  \else
  (#1 parts)
  \fi
  \ 
}

\newcommand{\bparts}{\begin{problemparts}}
\newcommand{\eparts}{\end{problemparts}}
\newcommand{\ppart}{\problempart}

%\newcommand{\lg} {lg\ }

\setlength{\oddsidemargin}{0pt}
\setlength{\evensidemargin}{0pt}
\setlength{\textwidth}{6.5in}
\setlength{\topmargin}{0in}
\setlength{\textheight}{8.5in}


\newcommand{\Spawn}{{\bf spawn} }
\newcommand{\Sync}{{\bf sync}}

\newcommand{\cif}[1]{\mbox{if $#1$}}
\newcommand{\cwhen}[1]{\mbox{when $#1$}}

\newcounter{problemnum}
\newcommand{\theproblem}{Problem \theproblemsetnum-\arabic{problemnum}}
\newenvironment{problems}{
        \begin{list}{{\bf \theproblem. \hspace*{0.5em}}}
        {\setlength{\leftmargin}{0em}
         \setlength{\rightmargin}{0em}
         \setlength{\labelwidth}{0em}
         \setlength{\labelsep}{0em}
         \usecounter{problemnum}}}{\end{list}}
\makeatletter
\newcommand{\problem}[1][{}]{\item \let\@currentlabel=\theproblem \textbf{#1}}
\makeatother

\newcounter{problempartnum}[problemnum]
\newenvironment{problemparts}{
        \begin{list}{{\bf (\alph{problempartnum})}}
        {\setlength{\leftmargin}{2.5em}
         \setlength{\rightmargin}{2.5em}
         \setlength{\labelsep}{0.5em}}}{\end{list}}
\newcommand{\problempart}{\addtocounter{problempartnum}{1}\item}

\newenvironment{truefalseproblemparts}{
        \begin{list}{{\bf (\alph{problempartnum})\ \ \ T\ \ F\hfil}}
        {\setlength{\leftmargin}{4.5em}
         \setlength{\rightmargin}{2.5em}
         \setlength{\labelsep}{0.5em}
         \setlength{\labelwidth}{4.5em}}}{\end{list}}

\newcounter{exercisenum}
\newcommand{\theexercise}{Exercise \theproblemsetnum-\arabic{exercisenum}}
\newenvironment{exercises}{
        \begin{list}{{\bf \theexercise. \hspace*{0.5em}}}
        {\setlength{\leftmargin}{0em}
         \setlength{\rightmargin}{0em}
         \setlength{\labelwidth}{0em}
         \setlength{\labelsep}{0em}
        \usecounter{exercisenum}}}{\end{list}}
\makeatletter
\newcommand{\exercise}{\item \let\@currentlabel=\theexercise}
\makeatother

\newcounter{exercisepartnum}[exercisenum]
%\newcommand{\problem}[1]{\medskip\mbox{}\newline\noindent{\bf Problem #1.}\hspace*{1em}}
%\newcommand{\exercise}[1]{\medskip\mbox{}\newline\noindent{\bf Exercise #1.}\hspace*{1em}}

\newenvironment{exerciseparts}{
        \begin{list}{{\bf (\alph{exercisepartnum})}}
        {\setlength{\leftmargin}{2.5em}
         \setlength{\rightmargin}{2.5em}
         \setlength{\labelsep}{0.5em}}}{\end{list}}
\newcommand{\exercisepart}{\addtocounter{exercisepartnum}{1}\item}


% Macros to make captions print with small type and 'Figure xx' in bold.
\makeatletter
\def\fnum@figure{{\bf Figure \thefigure}}
\def\fnum@table{{\bf Table \thetable}}
\let\@mycaption\caption
%\long\def\@mycaption#1[#2]#3{\addcontentsline{\csname
%  ext@#1\endcsname}{#1}{\protect\numberline{\csname 
%  the#1\endcsname}{\ignorespaces #2}}\par
%  \begingroup
%    \@parboxrestore
%    \small
%    \@makecaption{\csname fnum@#1\endcsname}{\ignorespaces #3}\par
%  \endgroup}
%\def\mycaption{\refstepcounter\@captype \@dblarg{\@mycaption\@captype}}
%\makeatother
\let\mycaption\caption
%\newcommand{\figcaption}[1]{\mycaption[]{#1}}

\newcounter{totalcaptions}
\newcounter{totalart}

\newcommand{\figcaption}[1]{\addtocounter{totalcaptions}{1}\caption[]{#1}}

% \psfigures determines what to do for figures:
%       0 means just leave vertical space
%       1 means put a vertical rule and the figure name
%       2 means insert the PostScript version of the figure
%       3 means put the figure name flush left or right
\newcommand{\psfigures}{0}
\newcommand{\spacefigures}{\renewcommand{\psfigures}{0}}
\newcommand{\rulefigures}{\renewcommand{\psfigures}{1}}
\newcommand{\macfigures}{\renewcommand{\psfigures}{2}}
\newcommand{\namefigures}{\renewcommand{\psfigures}{3}}

\newcommand{\figpart}[1]{{\bf (#1)}\nolinebreak[2]\relax}
\newcommand{\figparts}[2]{{\bf (#1)--(#2)}\nolinebreak[2]\relax}


\macfigures     % STATE

% When calling \figspace, make sure to leave a blank line afterward!!
% \widefigspace is for figures that are more than 28pc wide.
\newlength{\halffigspace} \newlength{\wholefigspace}
\newlength{\figruleheight} \newlength{\figgap}
\newcommand{\setfiglengths}{\ifnum\psfigures=1\setlength{\figruleheight}{\hruleheight}\setlength{\figgap}{1em}\else\setlength{\figruleheight}{0pt}\setlength{\figgap}{0em}\fi}
\newcommand{\figspace}[2]{\ifnum\psfigures=0\leavefigspace{#1}\else%
\setfiglengths%
\setlength{\wholefigspace}{#1}\setlength{\halffigspace}{.5\wholefigspace}%
\rule[-\halffigspace]{\figruleheight}{\wholefigspace}\hspace{\figgap}#2\fi}
\newlength{\widefigspacewidth}
% Make \widefigspace put the figure flush right on the text page.
\newcommand{\widefigspace}[2]{
\ifnum\psfigures=0\leavefigspace{#1}\else%
\setfiglengths%
\setlength{\widefigspacewidth}{28pc}%
\addtolength{\widefigspacewidth}{-\figruleheight}%
\setlength{\wholefigspace}{#1}\setlength{\halffigspace}{.5\wholefigspace}%
\makebox[\widefigspacewidth][r]{#2\hspace{\figgap}}\rule[-\halffigspace]{\figruleheight}{\wholefigspace}\fi}
\newcommand{\leavefigspace}[1]{\setlength{\wholefigspace}{#1}\setlength{\halffigspace}{.5\wholefigspace}\rule[-\halffigspace]{0em}{\wholefigspace}}

% Commands for including figures with macpsfig.
% To use these commands, documentstyle ``macpsfig'' must be specified.
\newlength{\macfigfill}
\makeatother
\newlength{\bbx}
\newlength{\bby}
\newcommand{\macfigure}[5]{\addtocounter{totalart}{1}
\ifnum\psfigures=2%
\setlength{\bbx}{#2}\addtolength{\bbx}{#4}%
\setlength{\bby}{#3}\addtolength{\bby}{#5}%
\begin{flushleft}
\ifdim#4>28pc\setlength{\macfigfill}{#4}\addtolength{\macfigfill}{-28pc}\hspace*{-\macfigfill}\fi%
\mbox{\psfig{figure=./#1.ps,%
bbllx=#2,bblly=#3,bburx=\bbx,bbury=\bby}}
\end{flushleft}%
\else\ifdim#4>28pc\widefigspace{#5}{#1}\else\figspace{#5}{#1}\fi\fi}
\makeatletter

\newlength{\savearraycolsep}
\newcommand{\narrowarray}[1]{\setlength{\savearraycolsep}{\arraycolsep}\setlength{\arraycolsep}{#1\arraycolsep}}
\newcommand{\normalarray}{\setlength{\arraycolsep}{\savearraycolsep}}

\newcommand{\hint}{{\em Hint:\ }}

% Macros from /th/u/clr/mac.tex

\newcommand{\set}[1]{\left\{ #1 \right\}}
\newcommand{\abs}[1]{\left| #1\right|}
\newcommand{\card}[1]{\left| #1\right|}
\newcommand{\floor}[1]{\left\lfloor #1 \right\rfloor}
\newcommand{\ceil}[1]{\left\lceil #1 \right\rceil}
\newcommand{\ang}[1]{\ifmmode{\left\langle #1 \right\rangle}
   \else{$\left\langle${#1}$\right\rangle$}\fi}
        % the \if allows use outside mathmode,
        % but will swallow following space there!
\newcommand{\paren}[1]{\left( #1 \right)}
\newcommand{\bracket}[1]{\left[ #1 \right]}
\newcommand{\prob}[1]{\Pr\left\{ #1 \right\}}
\newcommand{\Var}{\mathop{\rm Var}\nolimits}
\newcommand{\expect}[1]{{\rm E}\left[ #1 \right]}
\newcommand{\expectsq}[1]{{\rm E}^2\left[ #1 \right]}
\newcommand{\variance}[1]{{\rm Var}\left[ #1 \right]}
\renewcommand{\choose}[2]{{{#1}\atopwithdelims(){#2}}}
\def\pmod#1{\allowbreak\mkern12mu({\rm mod}\,\,#1)}
\newcommand{\matx}[2]{\left(\begin{array}{*{#1}{c}}#2\end{array}\right)}
\newcommand{\Adj}{\mathop{\rm Adj}\nolimits}

\newtheorem{theorem}{Theorem}
\newtheorem{lemma}[theorem]{Lemma}
\newtheorem{corollary}[theorem]{Corollary}
\newtheorem{xample}{Example}
\newtheorem{definition}{Definition}
\newenvironment{example}{\begin{xample}\rm}{\end{xample}}
\newcommand{\proof}{\noindent{\em Proof.}\hspace{1em}}
\def\squarebox#1{\hbox to #1{\hfill\vbox to #1{\vfill}}}
\newcommand{\qedbox}{\vbox{\hrule\hbox{\vrule\squarebox{.667em}\vrule}\hrule}}
\newcommand{\qed}{\nopagebreak\mbox{}\hfill\qedbox\smallskip}
\newcommand{\eqnref}[1]{(\protect\ref{#1})}

%%\newcommand{\twodots}{\mathinner{\ldotp\ldotp}}
\newcommand{\transpose}{^{\mbox{\scriptsize \sf T}}}
\newcommand{\amortized}[1]{\widehat{#1}}

\newcommand{\punt}[1]{}

%%% command for putting definitions into boldface
% New style for defined terms, as of 2/23/88, redefined by THC.
\newcommand{\defn}[1]{{\boldmath\textit{\textbf{#1}}}}
\newcommand{\defi}[1]{{\textit{\textbf{#1\/}}}}

\newcommand{\red}{\leq_{\rm P}}
\newcommand{\lang}[1]{%
\ifmmode\mathord{\mathcode`-="702D\rm#1\mathcode`\-="2200}\else{\rm#1}\fi}

%\newcommand{\ckt}[1]{\ifmmode\mathord{\mathcode`-="702D\sc #1\mathcode`\-="2200}\else$\mathord{\mathcode`-="702D\sc #1\mathcode`\-="2200}$\fi}
\newcommand{\ckt}[1]{\ifmmode \sc #1\else$\sc #1$\fi}

%% Margin notes - use \notesfalse to turn off notes.
\setlength{\marginparwidth}{0.6in}
\reversemarginpar
\newif\ifnotes
\notestrue
\newcommand{\longnote}[1]{
  \ifnotes
    {\medskip\noindent Note: \marginpar[\hfill$\Longrightarrow$]
      {$\Longleftarrow$}{#1}\medskip}
  \fi}
\newcommand{\note}[1]{
  \ifnotes
    {\marginpar{\tiny \raggedright{#1}}}
  \fi}


\newcommand{\reals}{\mathbbm{R}}
\newcommand{\integers}{\mathbbm{Z}}
\newcommand{\naturals}{\mathbbm{N}}
\newcommand{\rationals}{\mathbbm{Q}}
\newcommand{\complex}{\mathbbm{C}}

\newcommand{\oldreals}{{\bf R}}
\newcommand{\oldintegers}{{\bf Z}}
\newcommand{\oldnaturals}{{\bf N}}
\newcommand{\oldrationals}{{\bf Q}}
\newcommand{\oldcomplex}{{\bf C}}

\newcommand{\w}{\omega}                 %% for fft chapter

\newenvironment{closeitemize}{\begin{list}
{$\bullet$}
{\setlength{\itemsep}{-0.2\baselineskip}
\setlength{\topsep}{0.2\baselineskip}
\setlength{\parskip}{0pt}}}
{\end{list}}

% These are necessary within a {problems} environment in order to restore
% the default separation between bullets and items.
\newenvironment{normalitemize}{\setlength{\labelsep}{0.5em}\begin{itemize}}
                              {\end{itemize}}
\newenvironment{normalenumerate}{\setlength{\labelsep}{0.5em}\begin{enumerate}}
                                {\end{enumerate}}

%\def\eqref#1{Equation~(\ref{eq:#1})}
%\newcommand{\eqref}[1]{Equation (\ref{eq:#1})}
\newcommand{\eqreftwo}[2]{Equations (\ref{eq:#1}) and~(\ref{eq:#2})}
\newcommand{\ineqref}[1]{Inequality~(\ref{ineq:#1})}
\newcommand{\ineqreftwo}[2]{Inequalities (\ref{ineq:#1}) and~(\ref{ineq:#2})}

\newcommand{\figref}[1]{Figure~\ref{fig:#1}}
\newcommand{\figreftwo}[2]{Figures \ref{fig:#1} and~\ref{fig:#2}}

\newcommand{\liref}[1]{line~\ref{li:#1}}
\newcommand{\Liref}[1]{Line~\ref{li:#1}}
\newcommand{\lirefs}[2]{lines \ref{li:#1}--\ref{li:#2}}
\newcommand{\Lirefs}[2]{Lines \ref{li:#1}--\ref{li:#2}}
\newcommand{\lireftwo}[2]{lines \ref{li:#1} and~\ref{li:#2}}
\newcommand{\lirefthree}[3]{lines \ref{li:#1}, \ref{li:#2}, and~\ref{li:#3}}

\newcommand{\lemlabel}[1]{\label{lem:#1}}
\newcommand{\lemref}[1]{Lemma~\ref{lem:#1}} 

\newcommand{\exref}[1]{Exercise~\ref{ex:#1}}

\newcommand{\handref}[1]{Handout~\ref{#1}}

\newcommand{\defref}[1]{Definition~\ref{def:#1}}

% (1997.8.16: Victor Luchangco)
% Modified \hlabel to only get date and to use handouts counter for number.
%   New \handout and \handoutwithouttitle commands in newmac.tex use this.
%   The date is referenced by <label>-date.
%   (Retained old definition as \hlabelold.)
%   Defined \hforcelabel to use an argument instead of the handouts counter.

\newcounter{handouts}
\setcounter{handouts}{0}

\newcommand{\hlabel}[2]{%
\stepcounter{handouts}
{\edef\next{\write\@auxout{\string\newlabel{#1}{{\arabic{handouts}}{0}}}}\next}
\write\@auxout{\string\newlabel{#1-date}{{#2}{0}}}
}

\newcommand{\hforcelabel}[3]{%          Does not step handouts counter.
\write\@auxout{\string\newlabel{#1}{{#2}{0}}}
\write\@auxout{\string\newlabel{#1-date}{{#3}{0}}}}


% less ugly underscore
% --juang, 2008 oct 05
\renewcommand{\_}{\vrule height 0 pt depth 0.4 pt width 0.5 em \,}

\usepackage{enumitem}
\newcommand{\theproblemsetnum}{8}
\newcommand{\releasedate}{Thursday, April 11}
\newcommand{\partaduedate}{Wednesday, April 17}
\allowdisplaybreaks

\title{6.046 Problem Set \theproblemsetnum}

\begin{document}

\handout{Problem Set \theproblemsetnum}{\releasedate}
\textbf{All parts are due {\bf \partaduedate} at {\bf 10PM}}.

\makeatletter
\newenvironment{breakablealgorithm}
  {% \begin{breakablealgorithm}
   \begin{center}
     \refstepcounter{algorithm}% New algorithm
     \hrule height.8pt depth0pt \kern2pt% \@fs@pre for \@fs@ruled
     \renewcommand{\caption}[2][\relax]{% Make a new \caption
       {\raggedright\textbf{\ALG@name~\thealgorithm} ##2\par}%
       \ifx\relax##1\relax % #1 is \relax
         \addcontentsline{loa}{algorithm}{\protect\numberline{\thealgorithm}##2}%
       \else % #1 is not \relax
         \addcontentsline{loa}{algorithm}{\protect\numberline{\thealgorithm}##1}%
       \fi
       \kern2pt\hrule\kern2pt
     }
  }{% \end{breakablealgorithm}
     \kern2pt\hrule\relax% \@fs@post for \@fs@ruled
   \end{center}
  }
\makeatother

\setlength{\parindent}{0pt}
\medskip\hrulefill\medskip

{\bf Name:} Robert Durfee

\medskip

{\bf Collaborators:} None

\medskip\hrulefill

\begin{problems}

\problem  % Problem 1

{\bf Subproblems Definition} Let $A$ be a sequence (of arbitrary order) of
the $n$ boxes Kauricio has. Let the subproblem be $x(i, v)$, which represents
the minimum number of uses of the machine to alter boxes $\{a_i, \ldots,
a_n\}$ such that their total volume is exactly equal to $v$.

To show this is the optimal subproblem, consider the optimal number of uses
of Kauricio's machine for the boxes $\{a_1, \ldots, a_n\}$ so the total
volume is $V$. If this optimal solution passes through the boxes $\{a_i,
\ldots, a_n\}$ for a total volume of $v$, then the number of uses of the
machine is also optimal if we simply started with boxes $\{a_i, \ldots,
a_n\}$ and a volume of $v$. If there existed some smaller number of uses
that yielded the same volume, we could swap in the fewer number of uses.
However, this contradicts the statement that the original solution was
optimal.

{\bf Subproblems Relation} Let the function $dim(i, n)$ take a box $a_i$ with
dimensions $x_i, y_i, z_i$ and use Kauricio's machine to shrink or expand the
box $n$ times. Let any $n < 0$ represent shrinking the box $n$ times and let
$n > 0$ represent expanding the box $n$ times. Let this function return the
new dimensions of box $a_i'$: $x_i', y_i', z_i'$. Furthermore, let $vol(i,
n)$ perform the same operation as $dim(i, n)$, but return the volume of the
new box $a_i'$ instead of its dimensions.

Let the subproblem relation be the following: In English, for all possible
number, $n$, of shrinking, such that all the dimensions of the box $a_i$ are
greater than zero, or expanding, such that the total volume is less than $v$,
choose the minimum $|n|$ and recurse on the sequence of boxes without $a_i$
with the updated volume limit. Mathematically, this is expressed as,
$$ x(i, v) = \min_{n : 0 < dim(i, n)\ \mathrm{and}\ vol(i, n) \leq v} \{ |n|
+ x(i + 1, v - vol(i, n)) \} $$

This is a directed, acyclic graph because every $x(i, v)$ depends on a
smaller $x(i, v)$ as $i$ will always go to $i + 1$ and $v$ will always
decrease by at least $1$ given that all boxes have positive volume.

{\bf Base Cases} Suppose we have no more volume left, but we still have
remaining boxes to use. This refers to subproblem $x(i, 0)$. This will yield
an impossible solution as we must use all the boxes. To represent this as an
undesirable result, which we want to avoid if at all possible, we will return
$\infty$. Expressed mathematically,
$$ x(i, 0) = \infty\quad \forall i \in \{1, \ldots, n\} $$

Suppose instead we have no more boxes left, but we need to fill more volume.
This refers to subproblem $x(n, v)$. This also yields an impossible solution
as we must fill all of our volume. To represent this as an undesirable
result, which we want to avoid if at all possible, we will return $\infty$.
Expressed, mathematically,
$$ x(n, v) = \infty\quad \forall v > 0 $$

{\bf Solution from Subproblems} Since we only care about the minimum number
of uses of the machine, we only need to return the output of a particular
subproblem. In this case, we care about all values of $A$ so we start at
index $0$. We also care about a given total volume $V$. Therefore, we return
the value of the subproblem,
$$ x(0, V) $$
If the output of the subproblem is $\infty$, then we know we ran into some
impossiblity where we either ran out of boxes or we ran out of space. Given
the set up of the base cases, this will only be the result of the subproblem
if there is no other number of uses of the machine that yields a feasible
result. Therefore, if this is the case, we can return \textproc{Impossible}.

To compute this subproblem, one can either use bottom-up iteration or
top-down memoization. To use top-down memoization, set up a table for all
keys $(i, v)$ and initialize them to \textproc{None}. Before executing a call
for $x(i, v)$, check the table for a value in $(i, v)$. If there is a value,
return it. If there is not a value (\textproc{None}), recurse and call $x(i,
v)$ and save the value in the table. Continue until $(0, V)$ has a value in
the table and return it.

{\bf Running Time} The total number of subproblems is given by $n V$ as we
need to examine $n$ elements and consider, at worst, every integer value from
$0$ to $V$. Furthermore, each step in the subproblem requires at most
$\sqrt[3]{V}$ iterations. To see this, consider a box whose volume is exactly
$V$. We can shrink this box $n$ times before one of its dimensions equals
$0$. To make $n$ as large as possible, this box must be a cube, otherwise one
dimension will approach $0$ sooner than the others. Therefore, in the worst
case, $n = \sqrt[3]{V}$. Overall, this yields a running time of $O(n
V^{4/3})$.

\newpage
\problem  % Problem 2

{\bf Subproblems Definition} Let $S$ be a sequence of students that request
the shuttle such that $s_i$ is the $i$th person to request the shuttle. Let
each student $s_i \in S$ correspond to a position $p_i \in P$. Let the
subproblem be $x(i, \{p_{a-1}, p_{b-1}\})$, which represents the minimum
total distance for both drivers $A$ and $B$ starting with student $s_i$ and
given the previous stop for $A$ was $p_{a-1}$ and the previous stop for $B$
was $p_{b-1}$.

To show this is the optimal subproblem, consider the optimal schedules for
all students $\{s_1, \ldots, s_n\}$ both starting from Kresge $\{p_K, p_K\}$.
If the optimal solution passes through students $\{s_i, \ldots, s_n\}$
starting at some location $\{p_j, p_k\}$, then the schedule is also optimal
given that we started with student $s_i$ and buses were located at $p_j,
p_k$ to start with. If some other schedule is more optimal starting with
student $s_i$ at locations $p_j, p_k$, then this solution could be swapped in
to produce a more optimal result. However, this contradicts the fact that the
original schedule was optimal.

{\bf Subproblems Relation} Let the subproblem relation be the following: In
English, for both of the bus drivers' previous stops, compute the distance to
the new student. Take the minimum and recurse on the sequence without $s_i$
and update the previous stops to include $p_i$. Mathematically, this is
expressed as,
$$ x(i, \{p_{a-1}, p_{b-1}\}) = \min \begin{cases}
  d(p_{a-1}, p_i) + x(i + 1, \{p_i, p_{b-1}\}) \\
  d(p_{b-1}, p_i) + x(i + 1, \{p_{a-1}, p_i\})
\end{cases} \bigg\}$$

This is a directed, acyclic graph because each subproblem $x(i, \{p_{a-1},
p_{b-1}\})$ only references a smaller subproblem as $i$ always goes to $i +
1$ and the set $\{p_{a-1}, p_{b-1}\}$ is always updated such that one of the
two approaches the final student location.

{\bf Base Cases} Suppose there are no more students to pick up. Then, both
buses will need to return to Kresge. This can be represented by the
following,
$$ x(n, \{p_{a-1}, p_{b-1}\}) = d(p_{a-1}, p_K) + d(p_{b-1}, p_K) $$

{\bf Solution from Subproblems} We are interested in the optimal pickup
schedule for all students and starting from Kresge. This corresponds to the
following subproblem,
$$ x(0, \{p_K, p_K\}) $$

To compute this subproblem, one can either use bottom-up iteration or
top-down memoization. To use top-down memoization, set up a table for all
keys $(i, \{p_{a-1}, p_{b-1}\})$ and initialize them to \textproc{None}.
Before executing a call for $x(i, \{p_{a-1}, p_{b-1}\})$, check the table for
a value in $(i, \{p_{a-1}, p_{b-1}\})$. If there is a value, return it. If
there is not a value (\textproc{None}), recurse and call $x(i, \{p_{a-1},
p_{b-1}\})$ and save the value in the table. Continue until $(0, \{p_K,
p_K\})$ has a value in the table and return it.

However, in this problem, we do not care about the value of the shortest
distance combined schedule, we wish to know each bus's schedule with the
shortest distance. To recover this information, in addition to returning the
distance of the optimal schedule for each subproblem, denoted as $x$, we also
return a set of edges between locations, denoted as $y$. Consider a single
subproblem's set of edges (here subproblems are denoted as $y$ to make clear
the distinction that return values are sets of edges),
$$ y(i, \{p_{a-1}, p_{b-1}\}) = \begin{cases}
  \{(p_{a-1}, p_i)\} \cup y(i + 1, \{p_i, p_{b-1}\}) & \mathrm{if}\ d(p_{a-1}, p_i)
  + x(i + 1, \{p_i, p_{b-1}\}) \\
  & \ < d(p_{b-1}, p_i) + x(i + 1, \{p_{a-1}, p_i\}) \\
  \{(p_{b-1}, p_i)\} \cup y(i + 1, \{p_{a-1}, p_1\}) & \mathrm{otherwise}
\end{cases} $$
With base case,
$$ y(n, \{p_{a-1}, p_{b-1}\}) = \{(p_{a-1}, p_K), (p_{b-1}, p_K) \} $$

Therefore, the optimal schedules can be reconstructed from the set of edges
by forming a graph over all vertices $p \in P$ and running graph iteration to
identify two distinct loops through $p_K$.

{\bf Running Time} There are $n$ possible students to consider. For each
student $s_i$, there are also $i$ unordered pairs $\{p_{a-1}, p_{b-1}\}$ to
consider as all will have to contain $p_i$, but the other is only restricted
to $\{p_1, \ldots, p_{i - 1}\}$. Therefore, the number of subproblems is
given by,
$$ 1 + 1 + 2 + 3 + \ldots + n \in O(n^2) $$
Since each subproblem requires constant work, the runtime to compute
subproblems is $O(n^2)$. 

To reconstruct the schedules requires graph iteration. Assume for simplicity
that all places are unique (if not, make them unique by request time). The
constructed graph will have $n$ vertices. Since there are $n$ places to visit
and only one driver visits each place, the number of edges is $O(n)$.
Therefore, graph iteration will take $O(n)$ to reconstruct the schedules.

Overall, the running time must be $O(n^2)$.

\newpage
\problem  % Problem 3

{\bf Subproblems Definition} Let the subproblem be $x(v, t)$, which
represents the maximum sum of coefficients of fun for a tree rooted at $v$
where $v$ is included if $t = 1$ and $v$ is not included if $t = 0$.

To show this subproblem is optimal, consider a simpler subproblem $x(v)$
which represents the maximum sum of coefficients of fun for a tree rooted at
$v$. It is shown later how this can be transformed into the subproblem $x(v,
t)$. Now, consider the optimal sum of fun $x(r)$ for a hierarchy with Wusan
Sojcicki as the root $r$. If the optimal solution passes through some $x(u)$,
then the optimal sum of fun is the same from $u$ using all its descendants is
the same as if $u$ was the overall root to begin with. If there is some
other, more optimal sum of fun for the subtree rooted at $u$, then that
solution can be substituted into the original solution. However, this
contradicts the fact that the original solution was already optimal.

{\bf Subproblems Relation} Let the subproblem relation be the following: In
English, for a given vertex, you can either not take this vertex and recurse
on all the children, or you can take this vertex and recurse on all the
grandchildren. Mathematically, this is represented by,
$$ x(v) = \max \begin{cases}
  \sum_{u : (v, u) \in E} x(u) \\
  c_v + \sum_{u : (v, u) \in E} \sum_{w : (u, w) \in E} x(w)
\end{cases} \Bigg\} $$
However, there is some duplicate work being done in this recurrence. To
separate this work to maximize reuse, we introduce an indicator $t$. Now the
relation is,
$$ x(v, t) = \begin{cases}
  \sum_{u : (v, u) \in E} \max \{ x(u, 0), x(u, 1) \} & t = 0 \\
  c_v + \sum_{u : (v, u) \in E} x(u, 0) & t = 1
\end{cases} $$
This recurrence is equivalent to the previous, easier-to-understand
recurrence, but effectively reuses previous work.

This is a directed, acyclic graph because each subproblem $x(v, t)$ depends
only on their decendants (children or grandchildren, directly). As a result,
only smaller subproblems make up a given subproblem.

{\bf Base Cases} Suppose we have reached the summer interns (the tree rooted
at $v$ has no children, i.e. it is a leaf). If $t = 1$, then the intern is
included and their coefficient of fun is added. If $t = 0$, then the intern
is not included and their is no change to the sum of fun coefficients.
$$ x(v, 0) = 0 \quad \not\exists u.(v, u) \in E $$
$$ x(v, 1) = c_v \quad \not\exists u.(v, u) \in E $$

{\bf Solution from Subproblems} We are interested in the maximum possible sum
of the coefficients of fun of all the guests. Therefore, we construct a
directed tree with its root, $r$, as Wusan Sojcicki. Since we don't know
whether to include Wusan or not, we compute the following,
$$ \max\{x(r, 0), x(r, 1)\} $$

To compute this subproblem, one can use either a memoized top-down or
iterative bottom-up approach. Let's consider the iterative bottom-up
approach. Start by computing the $x(v, 0)$ and $x(v, 1)$ for all the interns.
Then, work the way up the tree by computing $x(v, 0)$ and $x(v, 1)$ for all
supervisors. Once the root $r$ is reached, return $\max \{ x(r, 0), x(r, 1)
\}$.

To reconstruct the invited guests (I am unclear as to if the problem asks for
this or not), have each subproblem return a list of invited guests, denote
this using $y$. Then, the additional return value is dictated by,
$$ y(v, t) = \begin{cases}
  \bigcup_{u : (v, u) \in E} \begin{cases}
    y(u, 0) & x(u, 0) > x(u, 1) \\
    y(u, 1) & \mathrm{otherwise}
  \end{cases} & t = 0 \\
  \{v\} \cup \bigcup_{u : (v, u) \in E} y(u, 0) & t = 1
\end{cases} $$
With base cases,
$$ y(v, 0) = \emptyset \quad \not\exists u.(v, u) \in E $$
$$ y(v, 1) = \{v\} \quad \not\exists u.(v, u) \in E $$
Then, the return set of guests from either $y(r, 0)$ or $y(r, 1)$ represents
the guest list.

{\bf Running Time} By examining the iterative, bottom-up approach, it is
clear that the vertices are being visited in DFS order. Therefore, each
vertex is being visited only a consant number of times (two, since we
calculate $x(v, 0)$ and $x(v, 1)$ only) so the runtime must be $O(n)$.

\end{problems}

\end{document}
