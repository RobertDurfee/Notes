%
% 6.046 problem set 2 solutions
%
\documentclass[12pt,twoside]{article}
\usepackage{multirow}
\usepackage{algorithm, algpseudocode, float}

\newcommand{\name}{}

\usepackage{amssymb}
\usepackage{amsmath}
\usepackage{graphicx}
\usepackage{latexsym}
\usepackage{times,url}
\usepackage{cprotect}
\usepackage{listings}
\usepackage{graphicx}
\usepackage[table]{xcolor}
\usepackage[letterpaper]{geometry}
\usepackage{tikz-qtree}
\usepackage{enumerate}

\newcommand{\profs}{Mauricio Karchmer, Aleksander Madry, Bruce Tidor}
\newcommand{\subj}{6.046}
\newcommand{\ttt}[1]{{\tt\small #1}}

\definecolor{dkgreen}{rgb}{0,0.6,0}
\definecolor{gray}{rgb}{0.5,0.5,0.5}
\definecolor{mauve}{rgb}{0.58,0,0.82}

\lstset{
  language=Python,
  aboveskip=1pc,
  belowskip=1pc,
  basicstyle={\footnotesize\ttfamily},
  numbers=left,
  showstringspaces=false,
  numberstyle=\tiny\color{gray},
  keywordstyle=\color{blue},
  commentstyle=\color{dkgreen},
  stringstyle=\color{mauve},
}

\tikzset{
  % every node/.style={minimum width=2em,draw,circle},
  % level 1/.style={sibling distance=2cm},
  level distance=1cm,
  edge from parent/.style=
  {draw,edge from parent path={(\tikzparentnode) -- (\tikzchildnode)}},
}

\newif\ifHideSolutions
\newcommand{\solution}[1]{\color{dkgreen}\textbf{Solution: }#1\color{black}}
\newcommand{\rubric}[1]{\color{dkgreen}{\bf Rubric:} #1\color{black}}

% \HideSolutionsfalse
% \ifHideSolutions
%   \renewcommand{\solution}[1]{}
%   \renewcommand{\rubric}[1]{}
% \fi

\newlength{\toppush}
\setlength{\toppush}{2\headheight}
\addtolength{\toppush}{\headsep}

\newcommand{\htitle}[2]{\noindent\vspace*{-\toppush}\newline\parbox{6.5in}
{\textit{Design and Analysis of Algorithms}\hfill\name\newline
Massachusetts Institute of Technology \hfill #2\newline
\profs\hfill #1 \vspace*{-.5ex}\newline
\mbox{}\hrulefill\mbox{}}\vspace*{1ex}\mbox{}\newline
\begin{center}{\Large\bf #1}\end{center}}

\newcommand{\handout}[2]{\thispagestyle{empty}
 \markboth{#1}{#1}
 \pagestyle{myheadings}\htitle{#1}{#2}}

\newcommand{\lecture}[3]{\thispagestyle{empty}
 \markboth{Lecture #1: #2}{Lecture #1: #2}
 \pagestyle{myheadings}\htitle{Lecture #1: #2}{#3}}

\newcommand{\htitlewithouttitle}[2]{\noindent\vspace*{-\toppush}\newline\parbox{6.5in}
{\textit{Design and Analysis of Algorithms}\hfill#2\newline
Massachusetts Institute of Technology \hfill 6.046\newline
\profs\hfill Handout #1\vspace*{-.5ex}\newline
\mbox{}\hrulefill\mbox{}}\vspace*{1ex}\mbox{}\newline}

\newcommand{\handoutwithouttitle}[2]{\thispagestyle{empty}
 \markboth{Handout \protect\ref{#1}}{Handout \protect\ref{#1}}
 \pagestyle{myheadings}\htitlewithouttitle{\protect\ref{#1}}{#2}}

\newcommand{\exam}[2]{% parameters: exam name, date
 \thispagestyle{empty}
 \markboth{\hspace{1cm}\subj\ #1\hspace{1in}Name\hrulefill\ \ }%
          {\subj\ #1\hspace{1in}Name\hrulefill\ \ }
 \pagestyle{myheadings}\examtitle{#1}{#2}
 \renewcommand{\theproblem}{Problem \arabic{problemnum}}
}
\newcommand{\examsolutions}[3]{% parameters: handout, exam name, date
 \thispagestyle{empty}
 \markboth{Handout \protect\ref{#1}: #2}{Handout \protect\ref{#1}: #2}
% \pagestyle{myheadings}\htitle{\protect\ref{#1}}{#2}{#3}
 \pagestyle{myheadings}\examsolutionstitle{\protect\ref{#1}} {#2}{#3}
 \renewcommand{\theproblem}{Problem \arabic{problemnum}}
}
\newcommand{\examsolutionstitle}[3]{\noindent\vspace*{-\toppush}\newline\parbox{6.5in}
{\textit{Design and Analysis of Algorithms}\hfill#3\newline
Massachusetts Institute of Technology \hfill 6.046\newline
%Singapore-MIT Alliance \hfill SMA5503\newline
\profs\hfill Handout #1\vspace*{-.5ex}\newline
\mbox{}\hrulefill\mbox{}}\vspace*{1ex}\mbox{}\newline
\begin{center}{\Large\bf #2}\end{center}}

\newcommand{\takehomeexam}[2]{% parameters: exam name, date
 \thispagestyle{empty}
 \markboth{\subj\ #1\hfill}{\subj\ #1\hfill}
 \pagestyle{myheadings}\examtitle{#1}{#2}
 \renewcommand{\theproblem}{Problem \arabic{problemnum}}
}

\makeatletter
\newcommand{\exambooklet}[2]{% parameters: exam name, date
 \thispagestyle{empty}
 \markboth{\subj\ #1}{\subj\ #1}
 \pagestyle{myheadings}\examtitle{#1}{#2}
 \renewcommand{\theproblem}{Problem \arabic{problemnum}}
 \renewcommand{\problem}{\newpage
 \item \let\@currentlabel=\theproblem
 \markboth{\subj\ #1, \theproblem}{\subj\ #1, \theproblem}}
}
\makeatother


\newcommand{\examtitle}[2]{\noindent\vspace*{-\toppush}\newline\parbox{6.5in}
{\textit{Design and Analysis of Algorithms}\hfill#2\newline
Massachusetts Institute of Technology \hfill 6.046 Spring 2019\newline
%Singapore-MIT Alliance \hfill SMA5503\newline
\profs\hfill #1\vspace*{-.5ex}\newline
\mbox{}\hrulefill\mbox{}}\vspace*{1ex}\mbox{}\newline
\begin{center}{\Large\bf #1}\end{center}}

\newcommand{\grader}[1]{\hspace{1cm}\textsf{\textbf{#1}}\hspace{1cm}}

\newcommand{\points}[1]{[#1 points]\ }
\newcommand{\parts}[1]
{
  \ifnum#1=1
  (1 part)
  \else
  (#1 parts)
  \fi
  \ 
}

\newcommand{\bparts}{\begin{problemparts}}
\newcommand{\eparts}{\end{problemparts}}
\newcommand{\ppart}{\problempart}

%\newcommand{\lg} {lg\ }

\setlength{\oddsidemargin}{0pt}
\setlength{\evensidemargin}{0pt}
\setlength{\textwidth}{6.5in}
\setlength{\topmargin}{0in}
\setlength{\textheight}{8.5in}


\newcommand{\Spawn}{{\bf spawn} }
\newcommand{\Sync}{{\bf sync}}

\newcommand{\cif}[1]{\mbox{if $#1$}}
\newcommand{\cwhen}[1]{\mbox{when $#1$}}

\newcounter{problemnum}
\newcommand{\theproblem}{Problem \theproblemsetnum-\arabic{problemnum}}
\newenvironment{problems}{
        \begin{list}{{\bf \theproblem. \hspace*{0.5em}}}
        {\setlength{\leftmargin}{0em}
         \setlength{\rightmargin}{0em}
         \setlength{\labelwidth}{0em}
         \setlength{\labelsep}{0em}
         \usecounter{problemnum}}}{\end{list}}
\makeatletter
\newcommand{\problem}[1][{}]{\item \let\@currentlabel=\theproblem \textbf{#1}}
\makeatother

\newcounter{problempartnum}[problemnum]
\newenvironment{problemparts}{
        \begin{list}{{\bf (\alph{problempartnum})}}
        {\setlength{\leftmargin}{2.5em}
         \setlength{\rightmargin}{2.5em}
         \setlength{\labelsep}{0.5em}}}{\end{list}}
\newcommand{\problempart}{\addtocounter{problempartnum}{1}\item}

\newenvironment{truefalseproblemparts}{
        \begin{list}{{\bf (\alph{problempartnum})\ \ \ T\ \ F\hfil}}
        {\setlength{\leftmargin}{4.5em}
         \setlength{\rightmargin}{2.5em}
         \setlength{\labelsep}{0.5em}
         \setlength{\labelwidth}{4.5em}}}{\end{list}}

\newcounter{exercisenum}
\newcommand{\theexercise}{Exercise \theproblemsetnum-\arabic{exercisenum}}
\newenvironment{exercises}{
        \begin{list}{{\bf \theexercise. \hspace*{0.5em}}}
        {\setlength{\leftmargin}{0em}
         \setlength{\rightmargin}{0em}
         \setlength{\labelwidth}{0em}
         \setlength{\labelsep}{0em}
        \usecounter{exercisenum}}}{\end{list}}
\makeatletter
\newcommand{\exercise}{\item \let\@currentlabel=\theexercise}
\makeatother

\newcounter{exercisepartnum}[exercisenum]
%\newcommand{\problem}[1]{\medskip\mbox{}\newline\noindent{\bf Problem #1.}\hspace*{1em}}
%\newcommand{\exercise}[1]{\medskip\mbox{}\newline\noindent{\bf Exercise #1.}\hspace*{1em}}

\newenvironment{exerciseparts}{
        \begin{list}{{\bf (\alph{exercisepartnum})}}
        {\setlength{\leftmargin}{2.5em}
         \setlength{\rightmargin}{2.5em}
         \setlength{\labelsep}{0.5em}}}{\end{list}}
\newcommand{\exercisepart}{\addtocounter{exercisepartnum}{1}\item}


% Macros to make captions print with small type and 'Figure xx' in bold.
\makeatletter
\def\fnum@figure{{\bf Figure \thefigure}}
\def\fnum@table{{\bf Table \thetable}}
\let\@mycaption\caption
%\long\def\@mycaption#1[#2]#3{\addcontentsline{\csname
%  ext@#1\endcsname}{#1}{\protect\numberline{\csname 
%  the#1\endcsname}{\ignorespaces #2}}\par
%  \begingroup
%    \@parboxrestore
%    \small
%    \@makecaption{\csname fnum@#1\endcsname}{\ignorespaces #3}\par
%  \endgroup}
%\def\mycaption{\refstepcounter\@captype \@dblarg{\@mycaption\@captype}}
%\makeatother
\let\mycaption\caption
%\newcommand{\figcaption}[1]{\mycaption[]{#1}}

\newcounter{totalcaptions}
\newcounter{totalart}

\newcommand{\figcaption}[1]{\addtocounter{totalcaptions}{1}\caption[]{#1}}

% \psfigures determines what to do for figures:
%       0 means just leave vertical space
%       1 means put a vertical rule and the figure name
%       2 means insert the PostScript version of the figure
%       3 means put the figure name flush left or right
\newcommand{\psfigures}{0}
\newcommand{\spacefigures}{\renewcommand{\psfigures}{0}}
\newcommand{\rulefigures}{\renewcommand{\psfigures}{1}}
\newcommand{\macfigures}{\renewcommand{\psfigures}{2}}
\newcommand{\namefigures}{\renewcommand{\psfigures}{3}}

\newcommand{\figpart}[1]{{\bf (#1)}\nolinebreak[2]\relax}
\newcommand{\figparts}[2]{{\bf (#1)--(#2)}\nolinebreak[2]\relax}


\macfigures     % STATE

% When calling \figspace, make sure to leave a blank line afterward!!
% \widefigspace is for figures that are more than 28pc wide.
\newlength{\halffigspace} \newlength{\wholefigspace}
\newlength{\figruleheight} \newlength{\figgap}
\newcommand{\setfiglengths}{\ifnum\psfigures=1\setlength{\figruleheight}{\hruleheight}\setlength{\figgap}{1em}\else\setlength{\figruleheight}{0pt}\setlength{\figgap}{0em}\fi}
\newcommand{\figspace}[2]{\ifnum\psfigures=0\leavefigspace{#1}\else%
\setfiglengths%
\setlength{\wholefigspace}{#1}\setlength{\halffigspace}{.5\wholefigspace}%
\rule[-\halffigspace]{\figruleheight}{\wholefigspace}\hspace{\figgap}#2\fi}
\newlength{\widefigspacewidth}
% Make \widefigspace put the figure flush right on the text page.
\newcommand{\widefigspace}[2]{
\ifnum\psfigures=0\leavefigspace{#1}\else%
\setfiglengths%
\setlength{\widefigspacewidth}{28pc}%
\addtolength{\widefigspacewidth}{-\figruleheight}%
\setlength{\wholefigspace}{#1}\setlength{\halffigspace}{.5\wholefigspace}%
\makebox[\widefigspacewidth][r]{#2\hspace{\figgap}}\rule[-\halffigspace]{\figruleheight}{\wholefigspace}\fi}
\newcommand{\leavefigspace}[1]{\setlength{\wholefigspace}{#1}\setlength{\halffigspace}{.5\wholefigspace}\rule[-\halffigspace]{0em}{\wholefigspace}}

% Commands for including figures with macpsfig.
% To use these commands, documentstyle ``macpsfig'' must be specified.
\newlength{\macfigfill}
\makeatother
\newlength{\bbx}
\newlength{\bby}
\newcommand{\macfigure}[5]{\addtocounter{totalart}{1}
\ifnum\psfigures=2%
\setlength{\bbx}{#2}\addtolength{\bbx}{#4}%
\setlength{\bby}{#3}\addtolength{\bby}{#5}%
\begin{flushleft}
\ifdim#4>28pc\setlength{\macfigfill}{#4}\addtolength{\macfigfill}{-28pc}\hspace*{-\macfigfill}\fi%
\mbox{\psfig{figure=./#1.ps,%
bbllx=#2,bblly=#3,bburx=\bbx,bbury=\bby}}
\end{flushleft}%
\else\ifdim#4>28pc\widefigspace{#5}{#1}\else\figspace{#5}{#1}\fi\fi}
\makeatletter

\newlength{\savearraycolsep}
\newcommand{\narrowarray}[1]{\setlength{\savearraycolsep}{\arraycolsep}\setlength{\arraycolsep}{#1\arraycolsep}}
\newcommand{\normalarray}{\setlength{\arraycolsep}{\savearraycolsep}}

\newcommand{\hint}{{\em Hint:\ }}

% Macros from /th/u/clr/mac.tex

\newcommand{\set}[1]{\left\{ #1 \right\}}
\newcommand{\abs}[1]{\left| #1\right|}
\newcommand{\card}[1]{\left| #1\right|}
\newcommand{\floor}[1]{\left\lfloor #1 \right\rfloor}
\newcommand{\ceil}[1]{\left\lceil #1 \right\rceil}
\newcommand{\ang}[1]{\ifmmode{\left\langle #1 \right\rangle}
   \else{$\left\langle${#1}$\right\rangle$}\fi}
        % the \if allows use outside mathmode,
        % but will swallow following space there!
\newcommand{\paren}[1]{\left( #1 \right)}
\newcommand{\bracket}[1]{\left[ #1 \right]}
\newcommand{\prob}[1]{\Pr\left\{ #1 \right\}}
\newcommand{\Var}{\mathop{\rm Var}\nolimits}
\newcommand{\expect}[1]{{\rm E}\left[ #1 \right]}
\newcommand{\expectsq}[1]{{\rm E}^2\left[ #1 \right]}
\newcommand{\variance}[1]{{\rm Var}\left[ #1 \right]}
\renewcommand{\choose}[2]{{{#1}\atopwithdelims(){#2}}}
\def\pmod#1{\allowbreak\mkern12mu({\rm mod}\,\,#1)}
\newcommand{\matx}[2]{\left(\begin{array}{*{#1}{c}}#2\end{array}\right)}
\newcommand{\Adj}{\mathop{\rm Adj}\nolimits}

\newtheorem{theorem}{Theorem}
\newtheorem{lemma}[theorem]{Lemma}
\newtheorem{corollary}[theorem]{Corollary}
\newtheorem{xample}{Example}
\newtheorem{definition}{Definition}
\newenvironment{example}{\begin{xample}\rm}{\end{xample}}
\newcommand{\proof}{\noindent{\em Proof.}\hspace{1em}}
\def\squarebox#1{\hbox to #1{\hfill\vbox to #1{\vfill}}}
\newcommand{\qedbox}{\vbox{\hrule\hbox{\vrule\squarebox{.667em}\vrule}\hrule}}
\newcommand{\qed}{\nopagebreak\mbox{}\hfill\qedbox\smallskip}
\newcommand{\eqnref}[1]{(\protect\ref{#1})}

%%\newcommand{\twodots}{\mathinner{\ldotp\ldotp}}
\newcommand{\transpose}{^{\mbox{\scriptsize \sf T}}}
\newcommand{\amortized}[1]{\widehat{#1}}

\newcommand{\punt}[1]{}

%%% command for putting definitions into boldface
% New style for defined terms, as of 2/23/88, redefined by THC.
\newcommand{\defn}[1]{{\boldmath\textit{\textbf{#1}}}}
\newcommand{\defi}[1]{{\textit{\textbf{#1\/}}}}

\newcommand{\red}{\leq_{\rm P}}
\newcommand{\lang}[1]{%
\ifmmode\mathord{\mathcode`-="702D\rm#1\mathcode`\-="2200}\else{\rm#1}\fi}

%\newcommand{\ckt}[1]{\ifmmode\mathord{\mathcode`-="702D\sc #1\mathcode`\-="2200}\else$\mathord{\mathcode`-="702D\sc #1\mathcode`\-="2200}$\fi}
\newcommand{\ckt}[1]{\ifmmode \sc #1\else$\sc #1$\fi}

%% Margin notes - use \notesfalse to turn off notes.
\setlength{\marginparwidth}{0.6in}
\reversemarginpar
\newif\ifnotes
\notestrue
\newcommand{\longnote}[1]{
  \ifnotes
    {\medskip\noindent Note: \marginpar[\hfill$\Longrightarrow$]
      {$\Longleftarrow$}{#1}\medskip}
  \fi}
\newcommand{\note}[1]{
  \ifnotes
    {\marginpar{\tiny \raggedright{#1}}}
  \fi}


\newcommand{\reals}{\mathbbm{R}}
\newcommand{\integers}{\mathbbm{Z}}
\newcommand{\naturals}{\mathbbm{N}}
\newcommand{\rationals}{\mathbbm{Q}}
\newcommand{\complex}{\mathbbm{C}}

\newcommand{\oldreals}{{\bf R}}
\newcommand{\oldintegers}{{\bf Z}}
\newcommand{\oldnaturals}{{\bf N}}
\newcommand{\oldrationals}{{\bf Q}}
\newcommand{\oldcomplex}{{\bf C}}

\newcommand{\w}{\omega}                 %% for fft chapter

\newenvironment{closeitemize}{\begin{list}
{$\bullet$}
{\setlength{\itemsep}{-0.2\baselineskip}
\setlength{\topsep}{0.2\baselineskip}
\setlength{\parskip}{0pt}}}
{\end{list}}

% These are necessary within a {problems} environment in order to restore
% the default separation between bullets and items.
\newenvironment{normalitemize}{\setlength{\labelsep}{0.5em}\begin{itemize}}
                              {\end{itemize}}
\newenvironment{normalenumerate}{\setlength{\labelsep}{0.5em}\begin{enumerate}}
                                {\end{enumerate}}

%\def\eqref#1{Equation~(\ref{eq:#1})}
%\newcommand{\eqref}[1]{Equation (\ref{eq:#1})}
\newcommand{\eqreftwo}[2]{Equations (\ref{eq:#1}) and~(\ref{eq:#2})}
\newcommand{\ineqref}[1]{Inequality~(\ref{ineq:#1})}
\newcommand{\ineqreftwo}[2]{Inequalities (\ref{ineq:#1}) and~(\ref{ineq:#2})}

\newcommand{\figref}[1]{Figure~\ref{fig:#1}}
\newcommand{\figreftwo}[2]{Figures \ref{fig:#1} and~\ref{fig:#2}}

\newcommand{\liref}[1]{line~\ref{li:#1}}
\newcommand{\Liref}[1]{Line~\ref{li:#1}}
\newcommand{\lirefs}[2]{lines \ref{li:#1}--\ref{li:#2}}
\newcommand{\Lirefs}[2]{Lines \ref{li:#1}--\ref{li:#2}}
\newcommand{\lireftwo}[2]{lines \ref{li:#1} and~\ref{li:#2}}
\newcommand{\lirefthree}[3]{lines \ref{li:#1}, \ref{li:#2}, and~\ref{li:#3}}

\newcommand{\lemlabel}[1]{\label{lem:#1}}
\newcommand{\lemref}[1]{Lemma~\ref{lem:#1}} 

\newcommand{\exref}[1]{Exercise~\ref{ex:#1}}

\newcommand{\handref}[1]{Handout~\ref{#1}}

\newcommand{\defref}[1]{Definition~\ref{def:#1}}

% (1997.8.16: Victor Luchangco)
% Modified \hlabel to only get date and to use handouts counter for number.
%   New \handout and \handoutwithouttitle commands in newmac.tex use this.
%   The date is referenced by <label>-date.
%   (Retained old definition as \hlabelold.)
%   Defined \hforcelabel to use an argument instead of the handouts counter.

\newcounter{handouts}
\setcounter{handouts}{0}

\newcommand{\hlabel}[2]{%
\stepcounter{handouts}
{\edef\next{\write\@auxout{\string\newlabel{#1}{{\arabic{handouts}}{0}}}}\next}
\write\@auxout{\string\newlabel{#1-date}{{#2}{0}}}
}

\newcommand{\hforcelabel}[3]{%          Does not step handouts counter.
\write\@auxout{\string\newlabel{#1}{{#2}{0}}}
\write\@auxout{\string\newlabel{#1-date}{{#3}{0}}}}


% less ugly underscore
% --juang, 2008 oct 05
\renewcommand{\_}{\vrule height 0 pt depth 0.4 pt width 0.5 em \,}

\newcommand{\theproblemsetnum}{5}
\newcommand{\releasedate}{Thursday, March 14}
\newcommand{\partaduedate}{Wednesday, March 20}
\allowdisplaybreaks

\title{6.046 Problem Set \theproblemsetnum}

\begin{document}

\handout{Problem Set \theproblemsetnum}{\releasedate}
\textbf{All parts are due {\bf \partaduedate} at {\bf 10PM}}.

\makeatletter
\newenvironment{breakablealgorithm}
  {% \begin{breakablealgorithm}
   \begin{center}
     \refstepcounter{algorithm}% New algorithm
     \hrule height.8pt depth0pt \kern2pt% \@fs@pre for \@fs@ruled
     \renewcommand{\caption}[2][\relax]{% Make a new \caption
       {\raggedright\textbf{\ALG@name~\thealgorithm} ##2\par}%
       \ifx\relax##1\relax % #1 is \relax
         \addcontentsline{loa}{algorithm}{\protect\numberline{\thealgorithm}##2}%
       \else % #1 is not \relax
         \addcontentsline{loa}{algorithm}{\protect\numberline{\thealgorithm}##1}%
       \fi
       \kern2pt\hrule\kern2pt
     }
  }{% \end{breakablealgorithm}
     \kern2pt\hrule\relax% \@fs@post for \@fs@ruled
   \end{center}
  }
\makeatother

\setlength{\parindent}{0pt}
\medskip\hrulefill\medskip

{\bf Name:} Robert Durfee

\medskip

{\bf Collaborators:} None

\medskip\hrulefill

\begin{problems}

\problem  % Problem 1

\begin{problemparts}

\problempart % Problem 1a

{\bf Description} The minimum disturbance linear program is:
$$ \min_{k_{ij} \forall (i, j) \in E} \sum_{(i, j) \in E} k_{ij} \cdot c_{ij} $$
Subject to:
\begin{align*}
  \sum_{i:(i, j) \in E} k_{ij} &= \sum_{\ell:(j, \ell) \in E} k_{j\ell}\quad
  \forall j \in V \setminus \{s, t\} \\
  \sum_{j:(s, j) \in E} k_{sj} - \sum_{i:(i, s) \in E} k_{is} &= p \\
  \sum_{i:(i, t) \in E} k_{it} - \sum_{j:(t, j) \in E} k_{tj} &= p \\
  k_{ij} &\leq u_{ij}\quad \forall (i, j) \in E \\
  k_{ij} &\geq 0\quad \forall (i, j) \in E
\end{align*}

{\bf Correctness} To prove this linear program is correct, we consider both
directions of the equality independently. That is, we show that a solution in
terms of the network of hallways implies the solution to the linear program
and we show that a solution of the linear program implies the solution in
terms of the network of hallways.

\begin{itemize}

  \item {\it Network of Hallways $\implies$ Linear Program}: Consider MIT's
  network of hallways. To represent this network, use a connected, directed
  graph $G = (V, E)$ with no duplicate edges. Let each $(i, j) \in E$
  represent a hallway and each $i,j \in V$ represent an intersection between
  hallways. Each hallway has a maximum rate of tourists it can support. Let
  this be $u_{ij}$ for hallway $(i, j) \in E$. Furthermore, each hallway $(i,
  j) \in E$ has a disturbance coefficient $c_{ij}$. Let the disturbance
  through hallway $(i, j) \in E$ be equal to $k_{ij} \cdot c_{ij}$. Along
  these hallways, we want to transfer some $p$ tourists from Kendall Square
  ($s$) to the student center ($t$) with as little disturbance as possible.

  Suppose we find a collection of rates of tourists $k_{ij}$ through each of
  the hallways $(i, j) \in E$ that minimizes the disturbance $k_{ij} \cdot
  c_{ij}$ for all hallways $(i, j) \in E$. Because the whole is the sum of
  its parts, this is implies the collection of rates of tourists $k_{ij}$
  satisfies
  $$ \min_{k_{ij} \forall (i, j) \in E} \sum_{(i, j) \in E} k_{ij} \cdot
  c_{ij} $$
  which is the optimization of the linear program. 
  
  This collection of rates of tourists $k_{ij}$ through each of the hallways
  $(i, j) \in E$ must satisfy flow conservation (i.e. no tourists can
  magically appear or disappear along the network, aside from at Kendall or
  the student center). This implies we uphold
  $$ \sum_{i:(i, j) \in E} k_{ij} = \sum_{\ell:(j, \ell) \in E}
  k_{j\ell}\quad \forall j \in V \setminus \{s, t\} $$
  which is the first constraint in the linear program. 
  
  We know that we want to push $p$ tourists from Kendall to the student
  center. Then, clearly, we want to send exactly $p$ tourists into the
  network of hallways at Kendall and have $p$ tourists come out at the
  student center. This implies we uphold
  $$ \sum_{j:(s, j) \in E} k_{sj} - \sum_{i:(i, s) \in E} k_{is} = p $$
  $$ \sum_{i:(i, t) \in E} k_{it} - \sum_{j:(t, j) \in E} k_{tj} = p $$
  which is the second (and third) contraint in the linear program.

  We also must respect the maximum rates $u_{ij}$ of tourist flow $k_{ij}$
  through any hallway $(i, j) \in E$. That is, we can't send more tourists
  through a hallway than that hallway's capacity. This implies we uphold
  $$ k_{ij} \leq u_{ij}\quad \forall (i, j) \in E $$
  which is the fourth constraint in the linear program.

  Lastly, and trivially, we cannot send a negative number of tourists
  $k_{ij}$ through any hallway $(i, j) \in E$. This implies we uphold
  $$ k_{ij} \geq 0\quad \forall (i, j) \in E $$
  which is the last constraint in the linear program.

  As a result, if we have a set of tourist flow rates $k_{ij}$ that minimizes
  the disturbance according to the problem specification, we have a solution
  to the linear program described.

  \item {\it Linear Program $\implies$ Network of Hallways}: The
  linear program is optimizing
  $$ \min_{k_{ij} \forall (i, j) \in E} \sum_{(i, j) \in E} k_{ij} \cdot
  c_{ij} $$
  If we let the $k_{ij}$ be the flow of tourists across some hallway $(i, j)
  \in E$ and $c_{ij}$ be the disturbance coefficient along that hallway (and
  because the whole is equal to the sum of its parts), this implies we are
  finding the minimum disturbance $k_{ij} \cdot c_{ij}$ of tourists flows
  through all hallways $(i, j) \in E$.

  The linear program has the first constraint
  $$ \sum_{i:(i, j) \in E} k_{ij} = \sum_{\ell:(j, \ell) \in E}
  k_{j\ell}\quad \forall j \in V \setminus \{s, t\} $$
  If we, once again, let $k_{ij}$ be the flow of tourists across some hallway
  $(i, j) \in E$ and $s$ be Kendall and $t$ be the student center, this
  implies we cannot have tourists disappear or reppear arbitrarily within our
  hallway network (except at Kendall and the student center).

  The linear program's second (and third) constraint is
  $$ \sum_{j:(s, j) \in E} k_{sj} - \sum_{i:(i, s) \in E} k_{is} = p $$
  $$ \sum_{i:(i, t) \in E} k_{it} - \sum_{j:(t, j) \in E} k_{tj} = p $$
  If we let $p$ be the number of tourists we want to send from Kendall to the
  student center and $s$ be Kendall and $t$ be the student center, this
  implies that the net flow of tourists from Kendall must be $p$ and the net
  flow of tourists into the student center must also be $p$.

  The linear program's fourth constraint is
  $$ k_{ij} \leq u_{ij}\quad \forall (i, j) \in E $$
  If we let $k_{ij}$ be the flow of tourists across hallway $(i, j) \in E$
  and $u_{ij}$ be the corresponding capacity of that hallway, this implies we
  cannot send more tourists through a hallway than that hallway's limit.

  Lastly, the linear program's final constraint is
  $$ k_{ij} \geq 0\quad \forall (i, j) \in E $$
  If we let $k_{ij}$ be the flow of tourists across hallway $(i, j) \in E$,
  this implies we cannot have a negative flow of tourists within our hallway
  network.

  As a result, if we have a set of $k_{ij}$ that minimizes the linear
  program's objective and is subject to its constraints, we have a set of
  flows in our hallway network $k_{ij}$ that minimizes the disturbance.

\end{itemize}

Since both implication directions are true, this linear program must be
equivalent to the minimum disturbance tourist flow as interpreted in the
solution presented above.

\problempart % Problem 1b

{\bf Description} First, we can write the linear program from the previous
part in standard primal form. Let the vertices $v \in V$ be uniquely
identified using $i \in {1, \ldots, |V|}$. Let the source $s$ have
idenfifier $1$ and the sink $t$ have identifier $|V|$.
$$ \max_{k_{ij} \forall (i, j) \in E} \sum_{(i, j) \in E} k_{ij} \cdot
\left(- c_{ij}\right) $$
Subject to:
\begin{align*}
  \sum_{i:(i, 1) \in E} k_{i1} - \sum_{\ell:(1, \ell) \in E} k_{1\ell} &\leq
    -p \\
  \sum_{i:(i, j) \in E} k_{ij} - \sum_{\ell:(j, \ell) \in E} k_{j\ell} &\leq
    0 \quad \forall j \in \{2, \ldots, |V| - 1\} \\
  \sum_{i:(i, |V|) \in E} k_{i|V|} - \sum_{\ell:(|V|, \ell) \in E}
    k_{|V|\ell} &\leq p \\
  -\left(\sum_{i:(i, 1) \in E} k_{i1} - \sum_{\ell:(1, \ell) \in E}
    k_{1\ell}\right) &\leq p \\
  -\left(\sum_{i:(i, j) \in E} k_{ij} - \sum_{\ell:(j, \ell) \in E}
    k_{j\ell}\right) &\leq 0 \quad \forall j \in \{2, \ldots, |V| - 1\} \\
  -\left(\sum_{i:(i, |V|) \in E} k_{i|V|} - \sum_{\ell:(|V|, \ell) \in E}
    k_{|V|\ell}\right) &\leq -p \\
  k_{ij} &\leq u_{ij}\quad \forall (i, j) \in E \\
  k_{ij} &\geq 0\quad \forall (i, j) \in E
\end{align*}
For non-negative scaling factors $w_1, \ldots, w_{|V|}$, $y_1, \ldots,
y_{|V|}$, and $z_1, \ldots, z_{|E|}$, we have the inequality
\begin{alignat*}{2}
  &w_1 \cdot \left(\sum_{i:(i, 1) \in E} k_{i1} - \sum_{\ell:(1, \ell) \in E}
    k_{1\ell}\right) &&+ \ldots +\\
  &w_{|V|} \cdot \left(\sum_{i:(i, |V|) \in E} k_{i|V|} - \sum_{\ell:(|V|,
    \ell) \in E} k_{|V|\ell}\right) &&+ \\
  &y_{1} \cdot -\left(\sum_{i:(i, 1) \in E} k_{i1} - \sum_{\ell:(1,
    \ell) \in E} k_{1\ell}\right) &&+ \ldots + \\
  &y_{|V|} \cdot -\left(\sum_{i:(i, |V|) \in E} k_{i|V|} - \sum_{\ell:(|V|,
    \ell) \in E} k_{|V|\ell}\right) &&+ \\
  &z_{1} \cdot k_{e_{1}} &&+ \ldots + \\
  &z_{|E|} \cdot k_{e_{|E|}} \\
  & \leq -w_1 p + w_{|V|} p + y_{1} p - y_{|V|} p + z_{1} u_{e_{1}} + \ldots
    + z_{|E|} u_{e_{|E|}}
\end{alignat*}
We can rewrite this sum with the $k_{ij}$ isolated (assume for simplicity that
the edges $k_{11}, k_{12}, k_{13}, \ldots, k_{21}, k_{22}, k_{23}$ exist).
\begin{alignat*}{2}
  &(z_{e_{11}})(k_{11}) &&+ \\
  &\left(-(w_1 - y_{1}) + (w_2 - y_{2}) + z_{e_{12}}\right)(k_{12}) &&+ \\
  &\left(-(w_1 - y_{1}) + (w_3 - y_{3}) + z_{e_{13}}\right)(k_{13}) &&+ \ldots + \\
  &\left(-(w_2 - y_{2}) + (w_1 - y_{1}) + z_{e_{21}}\right)(k_{21}) &&+ \\
  &(z_{e_{22}})(k_{22}) &&+ \\
  &\left(-(w_2 - y_{2}) + (w_3 - y_{3}) + z_{e_{23}}\right)(k_{23}) &&+ \ldots + \\
  & \leq -w_1 p + w_{|V|} p + y_{1} p - y_{|V|} p + z_{e_{1}}
    u_{e_{1}} &&+ \ldots + z_{e_{|E|}} u_{e_{|E|}}
\end{alignat*}
From this, we extract the dual:
$$ \min_{\{w\}_1^{|V|}, \{y\}_1^{|V|}, \{z\}_1^{|E|}} p \left(w_{|V|} - w_1 +
y_1 - y_{|V|}\right) + \sum_{(i, j) \in E} z_{ij} \cdot u_{ij} $$
Subject to:
$$ -(w_i - y_i) + (w_j - y_j) + z_{ij} \geq c_{ij} \quad \forall (i, j) \in E $$
$$ w_i \geq 0 \quad \forall i \in V $$
$$ y_i \geq 0 \quad \forall i \in V $$
$$ z_{ij} \geq 0 \quad \forall (i, j) \in E $$

\problempart % Problem 1c

{\bf Description} When $p = 1$ and $u_{ij} = \infty\ \forall (i, j) \in E$,
the linear program reduces to
$$ \min_{k_{ij} \forall (i, j) \in E} \sum_{(i, j) \in E} k_{ij} \cdot c_{ij} $$
Subject to:
\begin{align*}
  \sum_{i:(i, j) \in E} k_{ij} &= \sum_{\ell:(j, \ell) \in E} k_{j\ell}\quad
  \forall j \in V \setminus \{s, t\} \\
  \sum_{j:(s, j) \in E} k_{sj} - \sum_{i:(i, s) \in E} k_{is} &= 1 \\
  \sum_{i:(i, t) \in E} k_{it} - \sum_{j:(t, j) \in E} k_{tj} &= 1 \\
  k_{ij} &\geq 0\quad \forall (i, j) \in E
\end{align*}
This is equivalent to finding the shortest path from $s$ to $t$ where
`shortest' is measured as the minimum cost. That is, any edges $(i, j) \in E$
that have a positive flow $k_{ij}$ (or, actually, a weight equal to $1$),
these edges together yield the minimum cost path from $s$ to $t$.

{\bf Correctness} To show that flow $k_{ij}$ through each hallway $(i, j) \in
E$ yields the minimum cost path, we need to show the path goes from $s$ to
$t$, it is a valid flow, that all $k_{ij} \in \{0, 1\}$, and that the path
is, in fact, a minimum.

\begin{itemize}

  \item {\it Goes from $s$ to $t$}: From the second and third constraints of
  the reduced linear program, we know that exactly $1$ net unit of flow exits
  $s$. Furthermore, exactly $1$ net unit of flow arrives at $t$. Therefore,
  whatever flow(s) result from this linear program, it must travel from $s$
  to $t$.

  \item {\it Valid flow}: Given the first constraint of the linear program,
  we know that flow must be conserved within the network. We also know from
  the fourth constraint that this flow must be net non-negative. As a result,
  any flow that results from the linear program must be valid.

  \item {\it All $k_{ij}$ are $0$ or $1$}: To show this, we consider the flow
  decomposition lemma and show that fractional flows/cycles cannot be more
  optimal.
  
  Assume for contradiction that there exists a $k_{ij} \not\in \{0, 1\}$.
  Knowing that the total flow must be equal to $1$, we can decompose the flow
  into more than one path from $s$ to $t$ or cycle.

  Consider, WLOG, that there were two paths from $s$ to $t$ with flows. These
  paths must be fractional as their total must sum to $1$ (and can't be
  negative). Let the total disturbance along each path be $w_1$ and $w_2$. If
  $w_1 > w_2$ or $w_2 > w_1$, then it would be more optimal to push all the
  flow along the lesser disturbance path. Thus, this cannot occur without
  contradiction. If $w_1 = w_2$, it is clear that it would be equivalent to
  send all the flow along one or the other and thus it would not be more
  optimal to have fractional flow.

  Now suppose there exist cycles in the flow. If the weight of the cycle were
  positive, it is obvious that it would yield lower cost to not go along the
  cycle, but exit early. If the cycle is negative weight, then this will
  yield a negative infinity shortest path.

  Thus, all $k_{ij} \in \{0, 1\}$.

  \item {\it Is a minimum}: By the objective function, it is clear that we
  are minimizing the total disturbance. Thus, if the linear program is
  solved, the result must be a minimum.

\end{itemize}

\end{problemparts}

\newpage
\problem  % Problem 2

\begin{problemparts}

\problempart % Problem 2a

The proof is by induction on the number of tiers of advisors Melon has. Let
$P(n)$ be the induction hypothesis that if Beff uses a deterministic
strategy, there is a strategy that allows Melon to consult at most $2^n$
advisors (Beff's utility).

{\it Base Case}: Let $n = 1$. In this case, Melon has only a single tier of
advisors. Therefore, he only has $3$ possible advisors to consult. To
determine the majority vote, he must, at minimum, consult $2$ of these $3$
advisors. If Beff uses a deterministic strategy, there are four outcomes
possible:

\begin{itemize}
  \item Three yes': Melon can choose to consult any two advisors to get the
  majority decision.
  \item Two yes' and one no: Melon can choose to consult the two advisors
  Beff will set to yes to get the majority decision.
  \item One yes and two no's: Melon can choose to consult the two advisors
  Beff will set to no to get the majority decision.
  \item Three no's: Melon can choose to consult any two advisors to get the
  majority decision.
\end{itemize}

In all of the cases, Melon only needs to consult $2$ of his advisors. Since
$2 = 2^1$, the induction hypothesis is upheld for the base case.

{\it Inductive Step}: Assume by induction that Melon only has to consult
$2^{n - 1}$ advisors when he has $n - 1$ Tiers. Now, since Melon has $n$
tiers, his $2^{n - 1}$ advisors will refer him to each of their $3$ advisors.
Once again, as in the base case, Beff's deterministic strategy has the same
outcome: For every group of three advisors Melon needs to consult, he only
needs to consult $2$ of them. Since he must talk to $2$ for every $2^{n -
1}$, the total number of advisors he needs to consult is $2^n$. Therefore,
the induction hypothesis is upheld in the inductive step.

Since the induction hypothesis holds in the base case and the inductive step,
by induction, the induction hypothesis must be true for all $n$.

\problempart % Problem 2b

The proof is by induction on the number of tiers of advisors Melon has. Let
$P(n)$ be the induction hypothesis that if Melon has a deterministic
strategy, there is a strategy that allows Beff to force Melon to consult at
least $3^n$ of his advisors (Beff's utility).

{\it Base Case}: Let $n = 1$. In this case, Melon has only a single tier of
advisors. Therefore, he only has $3$ possible advisors to consult. Therefore,
there are $6$ possible strategies he can have:

\begin{itemize}
  \item Consult $1$, $2$, then $3$: Beff can choose to set $1$ and $2$
  opposite to force Melon to consult $3$.
  \item Consult $1$, $3$, then $2$: Beff can choose to set $1$ and $3$
  opposite to force Melon to consult $2$.
  \item Consult $2$, $1$, then $3$: Similarly.
  \item Consult $2$, $3$, then $1$: Similarly.
  \item Consult $3$, $1$, then $2$: Similarly.
  \item Consult $3$, $2$, then $1$: Similarly.
\end{itemize}

In all of the cases, Beff has a strategy that will force Melon to consult all
three of his advisors. Since $3 = 3^1$, the induction hypothesis is upheld in
the base case.

{\it Inductive Step}: Assume by induction that Melon is forced to consult
$3^{n - 1}$ of his advisors when he has $n - 1$ tiers. Now, since Melon has
$n$ tiers, his $3^{n - 1}$ advisors will refer him to each of their $3$
advisors. Once again, as in the base case, Melon's deterministic strategy has
the same outcome: For every group of three advisors, there is a strategy that
Beff can employ to force Melon to consult all $3$ of his advisors for each
group. Since there are $3^{n - 1}$ groups and Melon needs to consult $3$
advisors in each, he must consult at least $3^n$ advisors. Therefore, the
induction hypothesis is upheld in the inductive step.

Since the induction hypothesis is upheld in the base and inductive steps, by
induction it must hold for all $n$ tiers.

\problempart % Problem 2c

If both Melon and Beff want to maximize their expected utilities, they will
not want to use a deterministic strategy. 

Let's consider why Beff would not want to use a deterministic strategy. As
shown in Part A, no matter what deterministic strategy Beff chooses, there is
a strategy for Melon that can be employed such that Beff get's his lowest
possible utility.

Now consider why Melon would not want to use a deterministic strategy. As
shown in Part B, no matter what deterministic strategy Melon chooses, there
is a strategy for Beff that can be employed such that Melon get's his lowest
possible utility.

\problempart % Problem 2d

{\bf Description} Melon should choose to consult his advisors randomly from
the following possible orders uniformly:

\begin{itemize}
  \item $1,2,3$
  \item $1,3,2$
  \item $2,1,3$
  \item $2,3,1$
  \item $3,1,2$
  \item $3,2,1$
\end{itemize}

Beff should choose to set the opinions of Melon's advisors randomly from the
following possible orders uniformly:

\begin{itemize}
  \item $Y,Y,N$
  \item $Y,N,Y$
  \item $N,Y,Y$
  \item $Y,N,N$
  \item $N,Y,N$
  \item $N,N,Y$
\end{itemize}

This will yield both Melon and Beff an expected utility (magnitude) of
$\left(8 / 3\right)^n$ when Melon has $n$ tiers of advisors.

{\bf Correctness} First we must show that the expected utility is $\left(8 /
3\right)^n$. The proof is by induction on the number of tiers of advisors
Melon has. Let $P(n)$ be the induction hypothesis that if Beff and Melon both
use non-deterministic strategies, Melon will have to consult, in expectation,
$\left(8 / 3\right)^n$ of his advisors (Beff's utility).

{\it Base Case} Let $n = 1$. In this case, Melon has only a single tier of
advisors. Melon must consult either $2$ or $3$ of his advisors. Consider the
case where he consults $2$ advisors.

\begin{itemize}
  \item If his advisors are $Y, Y, N$, then if Melon chooses orders $1, 2,
  3$ or $2, 1, 3$, then he only needs to ask $2$ advisors.
  \item If his advisors are $Y, N, Y$, then if Melon chooses orders $1, 3, 2$
  or $3, 1, 2$, then he only needs to ask $2$ advisors.
  \item If his advisors are $N, Y, Y$, then if Melon chooses orders $2, 3, 1$
  or $3, 2, 1$, then he only needs to ask $2$ advisors.
  \item If his advisors are $Y, N, N$, then if Melon chooses orders $2, 3, 1$
  or $3, 2, 1$, then he only needs to ask $2$ advisors.
  \item If his advisors are $N, Y, N$, then if Melon chooses orders $1, 3, 2$
  or $3, 1, 2$, then he only needs to ask $2$ advisors.
  \item If his advisors are $N, N, Y$, then if Melon chooses orders $2, 3, 1$
  or $3, 2, 1$, then he only needs to ask $2$ advisors.
\end{itemize}

The probability of each of theses cases occuring is
$$ \frac{1}{6} \cdot \frac{2}{6} = \frac{1}{18} $$
Therefore, the total probability of Melon only needing to consult $2$
advisors is
$$ 6 \cdot \left(\frac{1}{6} \cdot \frac{2}{6}\right) = \frac{1}{3} $$
For each of the cases, we can calculate the probability of Melon needing to,
instead, consult $3$ of his advisors as
$$ \frac{1}{6} \cdot \frac{4}{6} = \frac{1}{9} $$
Therefore, the total probability of Melon needing to consult $3$ of his
advisors is
$$ 6 \cdot \left(\frac{1}{6} \cdot \frac{4}{6}\right) = \frac{2}{3} $$
From this, the expected number of advisors he needs to consult for the first
tier is
$$ 2 \cdot \frac{1}{3} + 3 \cdot \frac{2}{3} = \frac{8}{3} $$
Since $\left(8 / 3\right) = \left(8 / 3\right)^1$, the induction hypothesis
holds for the base case.

{\it Inductive Step} Assume by induction that Melon needs to consult $\left(8
/ 3\right)^{n - 1}$ of his advisors in expectation when he has $n - 1$ tiers.
Each of those $\left(8 / 3\right)^{n - 1}$ advisors now has another $3$
advisors that they will refer Melon to. For each of those groups of three, as
in the base case, we know Melon will need to consult, in expectation,
$\left(8 / 3\right)$ of them. As a result, the total number of advisors he
will need to consult is, in expectation $\left(8 / 3\right)^n$. Therefore,
the induction hypothesis is upheld for the inductive step.

Since the induction hypothesis is upheld in the base case and the inductive
step, it must hold for all $n$. Therefore, the expected utility is $\left(8 /
3\right)^n$.

Now we need to show that is a Nash equilibrium. In Nash equilibrium, no
player will want to switch strategies.

First, we consider if Beff will want to change strategies. If Beff were to
switch to any deterministic strategy, then it is possible for Melon to employ
a strategy such that his utility goes from $2.667^n$ down to $2^n$ as shown
in Part A.

Now, we consider if Melon will want to change strategies. If Melon were to
switch to any deterministic strategy, then it is possible for Beff to employ
a strategy such that his utility goes from $-2.667^n$ down to $-3^n$ as shown
in Part B.

Therefore, neither player would prefer to switch strategies and thus this is
a Nash equilibrium.

\end{problemparts}

\end{problems}

\end{document}


