%
% 6.046 problem set 2 solutions
%
\documentclass[12pt,twoside]{article}
\usepackage{multirow}
\usepackage{algorithm, algpseudocode, float}

\newcommand{\name}{}

\usepackage{amssymb}
\usepackage{amsmath}
\usepackage{graphicx}
\usepackage{latexsym}
\usepackage{times,url}
\usepackage{cprotect}
\usepackage{listings}
\usepackage{graphicx}
\usepackage[table]{xcolor}
\usepackage[letterpaper]{geometry}
\usepackage{tikz-qtree}
\usepackage{enumerate}

\newcommand{\profs}{Mauricio Karchmer, Aleksander Madry, Bruce Tidor}
\newcommand{\subj}{6.046}
\newcommand{\ttt}[1]{{\tt\small #1}}

\definecolor{dkgreen}{rgb}{0,0.6,0}
\definecolor{gray}{rgb}{0.5,0.5,0.5}
\definecolor{mauve}{rgb}{0.58,0,0.82}

\lstset{
  language=Python,
  aboveskip=1pc,
  belowskip=1pc,
  basicstyle={\footnotesize\ttfamily},
  numbers=left,
  showstringspaces=false,
  numberstyle=\tiny\color{gray},
  keywordstyle=\color{blue},
  commentstyle=\color{dkgreen},
  stringstyle=\color{mauve},
}

\tikzset{
  % every node/.style={minimum width=2em,draw,circle},
  % level 1/.style={sibling distance=2cm},
  level distance=1cm,
  edge from parent/.style=
  {draw,edge from parent path={(\tikzparentnode) -- (\tikzchildnode)}},
}

\newif\ifHideSolutions
\newcommand{\solution}[1]{\color{dkgreen}\textbf{Solution: }#1\color{black}}
\newcommand{\rubric}[1]{\color{dkgreen}{\bf Rubric:} #1\color{black}}

% \HideSolutionsfalse
% \ifHideSolutions
%   \renewcommand{\solution}[1]{}
%   \renewcommand{\rubric}[1]{}
% \fi

\newlength{\toppush}
\setlength{\toppush}{2\headheight}
\addtolength{\toppush}{\headsep}

\newcommand{\htitle}[2]{\noindent\vspace*{-\toppush}\newline\parbox{6.5in}
{\textit{Design and Analysis of Algorithms}\hfill\name\newline
Massachusetts Institute of Technology \hfill #2\newline
\profs\hfill #1 \vspace*{-.5ex}\newline
\mbox{}\hrulefill\mbox{}}\vspace*{1ex}\mbox{}\newline
\begin{center}{\Large\bf #1}\end{center}}

\newcommand{\handout}[2]{\thispagestyle{empty}
 \markboth{#1}{#1}
 \pagestyle{myheadings}\htitle{#1}{#2}}

\newcommand{\lecture}[3]{\thispagestyle{empty}
 \markboth{Lecture #1: #2}{Lecture #1: #2}
 \pagestyle{myheadings}\htitle{Lecture #1: #2}{#3}}

\newcommand{\htitlewithouttitle}[2]{\noindent\vspace*{-\toppush}\newline\parbox{6.5in}
{\textit{Design and Analysis of Algorithms}\hfill#2\newline
Massachusetts Institute of Technology \hfill 6.046\newline
\profs\hfill Handout #1\vspace*{-.5ex}\newline
\mbox{}\hrulefill\mbox{}}\vspace*{1ex}\mbox{}\newline}

\newcommand{\handoutwithouttitle}[2]{\thispagestyle{empty}
 \markboth{Handout \protect\ref{#1}}{Handout \protect\ref{#1}}
 \pagestyle{myheadings}\htitlewithouttitle{\protect\ref{#1}}{#2}}

\newcommand{\exam}[2]{% parameters: exam name, date
 \thispagestyle{empty}
 \markboth{\hspace{1cm}\subj\ #1\hspace{1in}Name\hrulefill\ \ }%
          {\subj\ #1\hspace{1in}Name\hrulefill\ \ }
 \pagestyle{myheadings}\examtitle{#1}{#2}
 \renewcommand{\theproblem}{Problem \arabic{problemnum}}
}
\newcommand{\examsolutions}[3]{% parameters: handout, exam name, date
 \thispagestyle{empty}
 \markboth{Handout \protect\ref{#1}: #2}{Handout \protect\ref{#1}: #2}
% \pagestyle{myheadings}\htitle{\protect\ref{#1}}{#2}{#3}
 \pagestyle{myheadings}\examsolutionstitle{\protect\ref{#1}} {#2}{#3}
 \renewcommand{\theproblem}{Problem \arabic{problemnum}}
}
\newcommand{\examsolutionstitle}[3]{\noindent\vspace*{-\toppush}\newline\parbox{6.5in}
{\textit{Design and Analysis of Algorithms}\hfill#3\newline
Massachusetts Institute of Technology \hfill 6.046\newline
%Singapore-MIT Alliance \hfill SMA5503\newline
\profs\hfill Handout #1\vspace*{-.5ex}\newline
\mbox{}\hrulefill\mbox{}}\vspace*{1ex}\mbox{}\newline
\begin{center}{\Large\bf #2}\end{center}}

\newcommand{\takehomeexam}[2]{% parameters: exam name, date
 \thispagestyle{empty}
 \markboth{\subj\ #1\hfill}{\subj\ #1\hfill}
 \pagestyle{myheadings}\examtitle{#1}{#2}
 \renewcommand{\theproblem}{Problem \arabic{problemnum}}
}

\makeatletter
\newcommand{\exambooklet}[2]{% parameters: exam name, date
 \thispagestyle{empty}
 \markboth{\subj\ #1}{\subj\ #1}
 \pagestyle{myheadings}\examtitle{#1}{#2}
 \renewcommand{\theproblem}{Problem \arabic{problemnum}}
 \renewcommand{\problem}{\newpage
 \item \let\@currentlabel=\theproblem
 \markboth{\subj\ #1, \theproblem}{\subj\ #1, \theproblem}}
}
\makeatother


\newcommand{\examtitle}[2]{\noindent\vspace*{-\toppush}\newline\parbox{6.5in}
{\textit{Design and Analysis of Algorithms}\hfill#2\newline
Massachusetts Institute of Technology \hfill 6.046 Spring 2019\newline
%Singapore-MIT Alliance \hfill SMA5503\newline
\profs\hfill #1\vspace*{-.5ex}\newline
\mbox{}\hrulefill\mbox{}}\vspace*{1ex}\mbox{}\newline
\begin{center}{\Large\bf #1}\end{center}}

\newcommand{\grader}[1]{\hspace{1cm}\textsf{\textbf{#1}}\hspace{1cm}}

\newcommand{\points}[1]{[#1 points]\ }
\newcommand{\parts}[1]
{
  \ifnum#1=1
  (1 part)
  \else
  (#1 parts)
  \fi
  \ 
}

\newcommand{\bparts}{\begin{problemparts}}
\newcommand{\eparts}{\end{problemparts}}
\newcommand{\ppart}{\problempart}

%\newcommand{\lg} {lg\ }

\setlength{\oddsidemargin}{0pt}
\setlength{\evensidemargin}{0pt}
\setlength{\textwidth}{6.5in}
\setlength{\topmargin}{0in}
\setlength{\textheight}{8.5in}


\newcommand{\Spawn}{{\bf spawn} }
\newcommand{\Sync}{{\bf sync}}

\newcommand{\cif}[1]{\mbox{if $#1$}}
\newcommand{\cwhen}[1]{\mbox{when $#1$}}

\newcounter{problemnum}
\newcommand{\theproblem}{Problem \theproblemsetnum-\arabic{problemnum}}
\newenvironment{problems}{
        \begin{list}{{\bf \theproblem. \hspace*{0.5em}}}
        {\setlength{\leftmargin}{0em}
         \setlength{\rightmargin}{0em}
         \setlength{\labelwidth}{0em}
         \setlength{\labelsep}{0em}
         \usecounter{problemnum}}}{\end{list}}
\makeatletter
\newcommand{\problem}[1][{}]{\item \let\@currentlabel=\theproblem \textbf{#1}}
\makeatother

\newcounter{problempartnum}[problemnum]
\newenvironment{problemparts}{
        \begin{list}{{\bf (\alph{problempartnum})}}
        {\setlength{\leftmargin}{2.5em}
         \setlength{\rightmargin}{2.5em}
         \setlength{\labelsep}{0.5em}}}{\end{list}}
\newcommand{\problempart}{\addtocounter{problempartnum}{1}\item}

\newenvironment{truefalseproblemparts}{
        \begin{list}{{\bf (\alph{problempartnum})\ \ \ T\ \ F\hfil}}
        {\setlength{\leftmargin}{4.5em}
         \setlength{\rightmargin}{2.5em}
         \setlength{\labelsep}{0.5em}
         \setlength{\labelwidth}{4.5em}}}{\end{list}}

\newcounter{exercisenum}
\newcommand{\theexercise}{Exercise \theproblemsetnum-\arabic{exercisenum}}
\newenvironment{exercises}{
        \begin{list}{{\bf \theexercise. \hspace*{0.5em}}}
        {\setlength{\leftmargin}{0em}
         \setlength{\rightmargin}{0em}
         \setlength{\labelwidth}{0em}
         \setlength{\labelsep}{0em}
        \usecounter{exercisenum}}}{\end{list}}
\makeatletter
\newcommand{\exercise}{\item \let\@currentlabel=\theexercise}
\makeatother

\newcounter{exercisepartnum}[exercisenum]
%\newcommand{\problem}[1]{\medskip\mbox{}\newline\noindent{\bf Problem #1.}\hspace*{1em}}
%\newcommand{\exercise}[1]{\medskip\mbox{}\newline\noindent{\bf Exercise #1.}\hspace*{1em}}

\newenvironment{exerciseparts}{
        \begin{list}{{\bf (\alph{exercisepartnum})}}
        {\setlength{\leftmargin}{2.5em}
         \setlength{\rightmargin}{2.5em}
         \setlength{\labelsep}{0.5em}}}{\end{list}}
\newcommand{\exercisepart}{\addtocounter{exercisepartnum}{1}\item}


% Macros to make captions print with small type and 'Figure xx' in bold.
\makeatletter
\def\fnum@figure{{\bf Figure \thefigure}}
\def\fnum@table{{\bf Table \thetable}}
\let\@mycaption\caption
%\long\def\@mycaption#1[#2]#3{\addcontentsline{\csname
%  ext@#1\endcsname}{#1}{\protect\numberline{\csname 
%  the#1\endcsname}{\ignorespaces #2}}\par
%  \begingroup
%    \@parboxrestore
%    \small
%    \@makecaption{\csname fnum@#1\endcsname}{\ignorespaces #3}\par
%  \endgroup}
%\def\mycaption{\refstepcounter\@captype \@dblarg{\@mycaption\@captype}}
%\makeatother
\let\mycaption\caption
%\newcommand{\figcaption}[1]{\mycaption[]{#1}}

\newcounter{totalcaptions}
\newcounter{totalart}

\newcommand{\figcaption}[1]{\addtocounter{totalcaptions}{1}\caption[]{#1}}

% \psfigures determines what to do for figures:
%       0 means just leave vertical space
%       1 means put a vertical rule and the figure name
%       2 means insert the PostScript version of the figure
%       3 means put the figure name flush left or right
\newcommand{\psfigures}{0}
\newcommand{\spacefigures}{\renewcommand{\psfigures}{0}}
\newcommand{\rulefigures}{\renewcommand{\psfigures}{1}}
\newcommand{\macfigures}{\renewcommand{\psfigures}{2}}
\newcommand{\namefigures}{\renewcommand{\psfigures}{3}}

\newcommand{\figpart}[1]{{\bf (#1)}\nolinebreak[2]\relax}
\newcommand{\figparts}[2]{{\bf (#1)--(#2)}\nolinebreak[2]\relax}


\macfigures     % STATE

% When calling \figspace, make sure to leave a blank line afterward!!
% \widefigspace is for figures that are more than 28pc wide.
\newlength{\halffigspace} \newlength{\wholefigspace}
\newlength{\figruleheight} \newlength{\figgap}
\newcommand{\setfiglengths}{\ifnum\psfigures=1\setlength{\figruleheight}{\hruleheight}\setlength{\figgap}{1em}\else\setlength{\figruleheight}{0pt}\setlength{\figgap}{0em}\fi}
\newcommand{\figspace}[2]{\ifnum\psfigures=0\leavefigspace{#1}\else%
\setfiglengths%
\setlength{\wholefigspace}{#1}\setlength{\halffigspace}{.5\wholefigspace}%
\rule[-\halffigspace]{\figruleheight}{\wholefigspace}\hspace{\figgap}#2\fi}
\newlength{\widefigspacewidth}
% Make \widefigspace put the figure flush right on the text page.
\newcommand{\widefigspace}[2]{
\ifnum\psfigures=0\leavefigspace{#1}\else%
\setfiglengths%
\setlength{\widefigspacewidth}{28pc}%
\addtolength{\widefigspacewidth}{-\figruleheight}%
\setlength{\wholefigspace}{#1}\setlength{\halffigspace}{.5\wholefigspace}%
\makebox[\widefigspacewidth][r]{#2\hspace{\figgap}}\rule[-\halffigspace]{\figruleheight}{\wholefigspace}\fi}
\newcommand{\leavefigspace}[1]{\setlength{\wholefigspace}{#1}\setlength{\halffigspace}{.5\wholefigspace}\rule[-\halffigspace]{0em}{\wholefigspace}}

% Commands for including figures with macpsfig.
% To use these commands, documentstyle ``macpsfig'' must be specified.
\newlength{\macfigfill}
\makeatother
\newlength{\bbx}
\newlength{\bby}
\newcommand{\macfigure}[5]{\addtocounter{totalart}{1}
\ifnum\psfigures=2%
\setlength{\bbx}{#2}\addtolength{\bbx}{#4}%
\setlength{\bby}{#3}\addtolength{\bby}{#5}%
\begin{flushleft}
\ifdim#4>28pc\setlength{\macfigfill}{#4}\addtolength{\macfigfill}{-28pc}\hspace*{-\macfigfill}\fi%
\mbox{\psfig{figure=./#1.ps,%
bbllx=#2,bblly=#3,bburx=\bbx,bbury=\bby}}
\end{flushleft}%
\else\ifdim#4>28pc\widefigspace{#5}{#1}\else\figspace{#5}{#1}\fi\fi}
\makeatletter

\newlength{\savearraycolsep}
\newcommand{\narrowarray}[1]{\setlength{\savearraycolsep}{\arraycolsep}\setlength{\arraycolsep}{#1\arraycolsep}}
\newcommand{\normalarray}{\setlength{\arraycolsep}{\savearraycolsep}}

\newcommand{\hint}{{\em Hint:\ }}

% Macros from /th/u/clr/mac.tex

\newcommand{\set}[1]{\left\{ #1 \right\}}
\newcommand{\abs}[1]{\left| #1\right|}
\newcommand{\card}[1]{\left| #1\right|}
\newcommand{\floor}[1]{\left\lfloor #1 \right\rfloor}
\newcommand{\ceil}[1]{\left\lceil #1 \right\rceil}
\newcommand{\ang}[1]{\ifmmode{\left\langle #1 \right\rangle}
   \else{$\left\langle${#1}$\right\rangle$}\fi}
        % the \if allows use outside mathmode,
        % but will swallow following space there!
\newcommand{\paren}[1]{\left( #1 \right)}
\newcommand{\bracket}[1]{\left[ #1 \right]}
\newcommand{\prob}[1]{\Pr\left\{ #1 \right\}}
\newcommand{\Var}{\mathop{\rm Var}\nolimits}
\newcommand{\expect}[1]{{\rm E}\left[ #1 \right]}
\newcommand{\expectsq}[1]{{\rm E}^2\left[ #1 \right]}
\newcommand{\variance}[1]{{\rm Var}\left[ #1 \right]}
\renewcommand{\choose}[2]{{{#1}\atopwithdelims(){#2}}}
\def\pmod#1{\allowbreak\mkern12mu({\rm mod}\,\,#1)}
\newcommand{\matx}[2]{\left(\begin{array}{*{#1}{c}}#2\end{array}\right)}
\newcommand{\Adj}{\mathop{\rm Adj}\nolimits}

\newtheorem{theorem}{Theorem}
\newtheorem{lemma}[theorem]{Lemma}
\newtheorem{corollary}[theorem]{Corollary}
\newtheorem{xample}{Example}
\newtheorem{definition}{Definition}
\newenvironment{example}{\begin{xample}\rm}{\end{xample}}
\newcommand{\proof}{\noindent{\em Proof.}\hspace{1em}}
\def\squarebox#1{\hbox to #1{\hfill\vbox to #1{\vfill}}}
\newcommand{\qedbox}{\vbox{\hrule\hbox{\vrule\squarebox{.667em}\vrule}\hrule}}
\newcommand{\qed}{\nopagebreak\mbox{}\hfill\qedbox\smallskip}
\newcommand{\eqnref}[1]{(\protect\ref{#1})}

%%\newcommand{\twodots}{\mathinner{\ldotp\ldotp}}
\newcommand{\transpose}{^{\mbox{\scriptsize \sf T}}}
\newcommand{\amortized}[1]{\widehat{#1}}

\newcommand{\punt}[1]{}

%%% command for putting definitions into boldface
% New style for defined terms, as of 2/23/88, redefined by THC.
\newcommand{\defn}[1]{{\boldmath\textit{\textbf{#1}}}}
\newcommand{\defi}[1]{{\textit{\textbf{#1\/}}}}

\newcommand{\red}{\leq_{\rm P}}
\newcommand{\lang}[1]{%
\ifmmode\mathord{\mathcode`-="702D\rm#1\mathcode`\-="2200}\else{\rm#1}\fi}

%\newcommand{\ckt}[1]{\ifmmode\mathord{\mathcode`-="702D\sc #1\mathcode`\-="2200}\else$\mathord{\mathcode`-="702D\sc #1\mathcode`\-="2200}$\fi}
\newcommand{\ckt}[1]{\ifmmode \sc #1\else$\sc #1$\fi}

%% Margin notes - use \notesfalse to turn off notes.
\setlength{\marginparwidth}{0.6in}
\reversemarginpar
\newif\ifnotes
\notestrue
\newcommand{\longnote}[1]{
  \ifnotes
    {\medskip\noindent Note: \marginpar[\hfill$\Longrightarrow$]
      {$\Longleftarrow$}{#1}\medskip}
  \fi}
\newcommand{\note}[1]{
  \ifnotes
    {\marginpar{\tiny \raggedright{#1}}}
  \fi}


\newcommand{\reals}{\mathbbm{R}}
\newcommand{\integers}{\mathbbm{Z}}
\newcommand{\naturals}{\mathbbm{N}}
\newcommand{\rationals}{\mathbbm{Q}}
\newcommand{\complex}{\mathbbm{C}}

\newcommand{\oldreals}{{\bf R}}
\newcommand{\oldintegers}{{\bf Z}}
\newcommand{\oldnaturals}{{\bf N}}
\newcommand{\oldrationals}{{\bf Q}}
\newcommand{\oldcomplex}{{\bf C}}

\newcommand{\w}{\omega}                 %% for fft chapter

\newenvironment{closeitemize}{\begin{list}
{$\bullet$}
{\setlength{\itemsep}{-0.2\baselineskip}
\setlength{\topsep}{0.2\baselineskip}
\setlength{\parskip}{0pt}}}
{\end{list}}

% These are necessary within a {problems} environment in order to restore
% the default separation between bullets and items.
\newenvironment{normalitemize}{\setlength{\labelsep}{0.5em}\begin{itemize}}
                              {\end{itemize}}
\newenvironment{normalenumerate}{\setlength{\labelsep}{0.5em}\begin{enumerate}}
                                {\end{enumerate}}

%\def\eqref#1{Equation~(\ref{eq:#1})}
%\newcommand{\eqref}[1]{Equation (\ref{eq:#1})}
\newcommand{\eqreftwo}[2]{Equations (\ref{eq:#1}) and~(\ref{eq:#2})}
\newcommand{\ineqref}[1]{Inequality~(\ref{ineq:#1})}
\newcommand{\ineqreftwo}[2]{Inequalities (\ref{ineq:#1}) and~(\ref{ineq:#2})}

\newcommand{\figref}[1]{Figure~\ref{fig:#1}}
\newcommand{\figreftwo}[2]{Figures \ref{fig:#1} and~\ref{fig:#2}}

\newcommand{\liref}[1]{line~\ref{li:#1}}
\newcommand{\Liref}[1]{Line~\ref{li:#1}}
\newcommand{\lirefs}[2]{lines \ref{li:#1}--\ref{li:#2}}
\newcommand{\Lirefs}[2]{Lines \ref{li:#1}--\ref{li:#2}}
\newcommand{\lireftwo}[2]{lines \ref{li:#1} and~\ref{li:#2}}
\newcommand{\lirefthree}[3]{lines \ref{li:#1}, \ref{li:#2}, and~\ref{li:#3}}

\newcommand{\lemlabel}[1]{\label{lem:#1}}
\newcommand{\lemref}[1]{Lemma~\ref{lem:#1}} 

\newcommand{\exref}[1]{Exercise~\ref{ex:#1}}

\newcommand{\handref}[1]{Handout~\ref{#1}}

\newcommand{\defref}[1]{Definition~\ref{def:#1}}

% (1997.8.16: Victor Luchangco)
% Modified \hlabel to only get date and to use handouts counter for number.
%   New \handout and \handoutwithouttitle commands in newmac.tex use this.
%   The date is referenced by <label>-date.
%   (Retained old definition as \hlabelold.)
%   Defined \hforcelabel to use an argument instead of the handouts counter.

\newcounter{handouts}
\setcounter{handouts}{0}

\newcommand{\hlabel}[2]{%
\stepcounter{handouts}
{\edef\next{\write\@auxout{\string\newlabel{#1}{{\arabic{handouts}}{0}}}}\next}
\write\@auxout{\string\newlabel{#1-date}{{#2}{0}}}
}

\newcommand{\hforcelabel}[3]{%          Does not step handouts counter.
\write\@auxout{\string\newlabel{#1}{{#2}{0}}}
\write\@auxout{\string\newlabel{#1-date}{{#3}{0}}}}


% less ugly underscore
% --juang, 2008 oct 05
\renewcommand{\_}{\vrule height 0 pt depth 0.4 pt width 0.5 em \,}

\newcommand{\theproblemsetnum}{6}
\newcommand{\releasedate}{Thursday, March 21}
\newcommand{\partaduedate}{Wednesday, April 3}
\allowdisplaybreaks

\title{6.046 Problem Set \theproblemsetnum}

\begin{document}

\handout{Problem Set \theproblemsetnum}{\releasedate}
\textbf{All parts are due {\bf \partaduedate} at {\bf 10PM}}.

\makeatletter
\newenvironment{breakablealgorithm}
  {% \begin{breakablealgorithm}
   \begin{center}
     \refstepcounter{algorithm}% New algorithm
     \hrule height.8pt depth0pt \kern2pt% \@fs@pre for \@fs@ruled
     \renewcommand{\caption}[2][\relax]{% Make a new \caption
       {\raggedright\textbf{\ALG@name~\thealgorithm} ##2\par}%
       \ifx\relax##1\relax % #1 is \relax
         \addcontentsline{loa}{algorithm}{\protect\numberline{\thealgorithm}##2}%
       \else % #1 is not \relax
         \addcontentsline{loa}{algorithm}{\protect\numberline{\thealgorithm}##1}%
       \fi
       \kern2pt\hrule\kern2pt
     }
  }{% \end{breakablealgorithm}
     \kern2pt\hrule\relax% \@fs@post for \@fs@ruled
   \end{center}
  }
\makeatother

\setlength{\parindent}{0pt}
\medskip\hrulefill\medskip

{\bf Name:} Robert Durfee

\medskip

{\bf Collaborators:} None

\medskip\hrulefill

\begin{problems}

\problem  % Problem 1

\begin{problemparts}

\problempart % Problem 1a

Assume for contradiction that there exists an algorithm that correctly
returns {\tt SORTED} for a completely sorted array and correctly returns {\tt
UNSORTED} for an array that is not mostly sorted which is not $\Omega(n)$.
This implies that the algorithm makes a determination about the input by
examining fewer than any fraction of $n$ (as any fraction of $n$ is
$\Omega(n)$). This includes the fraction $9n/10$. Since an array can be
switch from mostly sorted to not mostly sorted by changing these $9n/10$
elements, the adversary could change these elements such that the algorithm
is wrong. Thus, the algorithm is not always correct. This is a contradiction
and thus a correct algorithm must be $\Omega(n)$.

\problempart % Problem 1b

Trivially, the algorithm will always correctly identify a completely sorted
array (as all triples will be sorted). We just need to consider when a not
mostly sorted array is labeled as {\tt SORTED}.

Let the adversary provide an input that is initially completely sorted with
the first (more than) $n/10$ distinct elements removed from the front of the
sorted array and appended to the end of the array after the other (less than)
$9n/10$ sorted elements. This yields an array which is sorted in two
sections, the first with less than $9n/10$ and the second with more than
$n/10$ elements. Now, there exist only two triples that are unsorted (located
where the two sorted sections meet).

The probability that these two triples are picked is
$$ \frac{2}{n - 2} $$
Therefore the probability that these two are not picked is
$$ 1 - \frac{2}{n - 2} $$
The probability that this algorithm incorrectly identifies an unsorted array
as {\tt SORTED} after $k$ such random triple evaluations is,
$$ \left(1 - \frac{2}{n - 2}\right)^k $$
We want this probability to be less than $1/3$, therefore,
$$ \left(1 - \frac{2}{n - 2}\right)^k < \frac{1}{3} $$
Solving for $k$ yields
$$ k < \frac{\log \frac{1}{3}}{\log\left(1 - \frac{2}{n - 2}\right)} $$
Taking the limit of this over $n$,
$$ \lim_{n \to \infty} \frac{\log \frac{1}{3}}{n \log\left(1 - \frac{2}{n -
2}\right)} = \frac{\log 3}{2} $$
Therefore $k \in \Theta(n)$ which shows it is $\Omega(n)$.

\problempart % Problem 1c

At some point, \textproc{ Binary-Search}($A$, $x_1$, $0$, $n$) and \textproc{
Binary-Search}($A$, $x_2$, $0$, $n$) diverge. At this point, if {\tt NOT
FOUND} and {\tt OUT OF ORDER} are not returned the median evaluation
conditional evaluated differently (e.g. $x_1 < A[m]$ and $x_2 \geq A[m]$).
If $x_1 < A[m]$ and $x_2 \geq A[m]$, since \textproc{ Binary-Search} would
recurse left for $x_1$ and right for $x_2$, it must be true that \textproc{
Binary-Search}($A$, $x_1$, $0$, $n$) $<$ \textproc{ Binary-Search}($A$, $x_2$,
$0$, $n$). Furthermore, by transitivity, if $x_1 < A[m]$ and $x_2 \geq A[m]$,
and $x_1$ and $x_2$ are distinct, $x_1 < x_2$.

\problempart % Problem 1d

Suppose for contradiction that $A$ is not mostly sorted and more than $9n/10$
indices can pass the \textproc{ Binary-Search-Test}. From Part C, we know that
if the \textproc{ Binary-Search} succeeds, we can use the output to construct
the ordering of the more than $9n/10$ elements. Therefore, if more than
$9n/10$ elements are ordered, then the array is mostly sorted and this is a
contradiction.

\problempart % Problem 1e

{\bf Description} Randomly select an index and run \textproc{Binary-Search-Test}
for that index. Repeat 22 times (total). If any of the results fail,
terminate early and return {\tt UNSORTED}. If all of the results pass, return
{\tt SORTED}

{\bf Correctness} By definition, if the input is sorted, this algorithm will
always correctly return {\tt SORTED} as all \textproc{ Binary-Search-Test} will 
always pass. We just need to consider when the algorithm will take a not
mostly sorted array as input and return {\tt SORTED}. In order for this
algorithm to incorrectly return {\tt SORTED} for a not mostly sorted array.
For this to occur, all of the $k$ \textproc{ Binary-Search-Test} would pass. For
a not mostly sorted array, this will occur (from Part D) at most $9/10$ of
the time. Therefore, the probability of incorrectly returning {\tt SORTED}
for a not mostly sorted array is given by
$$ \left(\frac{9}{10}\right)^k $$
Since we want this to be less than $1/10$, we get the inequality,
$$ \left(\frac{9}{10}\right)^k < \frac{1}{10} $$
Solving for $k$ yields,
$$ k \approx 21.8543 $$
Therefore, by choosing $k = 22$, we can guarantee that the error is less than
$1/10$ of the time.

{\bf Running Time} Each \textproc{ Binary-Search-Test} takes $O(\log n)$. We
execute this only a constant number of times, therefore the overall runtime
is $O(\log n)$.

\end{problemparts}

\newpage
\problem  % Problem 2

\begin{problemparts}

\problempart % Problem 2a

{\bf Description} The Markov chain $H$ describing the provided procedure has
the following properties.

\begin{itemize}
  \item The set of all possible vertices $V_H$ represent all valid
  $q$-colorings $f$ of the graph $G$.
  \item The starting vertex $v_h$ is the initial valid $q$-coloring $f_0$ of
  the graph $G$ before the procedure begins.
  \item The edges $E_H$ represent transitions between one valid $q$-coloring
  $i$ to another valid $q$-coloring $i + 1$ where only a single vertex $v$
  changes color. Self-loops exist on all vertices.
  \item The weight $p_e$ for an edge $e \in E_H$ that changes the color of a
  vertex $v$ from $f_i(v)$ to $f_{i+1}(v)$ (such that $f_i(v) \neq f_{i +
  1}(v)$, assuming such an edge exists) is given by
  $$ \frac{1}{|V|} \cdot \frac{1}{|\{c \mid c \neq f_i(w)\ \forall w \in
  N(v)\}|} $$
  
  And the weight $p_e$ for an edge $e \in E_H$ that is a self-loop such that
  $f_i(v) = f_{i + 1}(v)\ \forall v \in V$ is given by
  $$ \frac{1}{|V|} \sum_{v \in V} \frac{1}{|\{c \mid c \neq f_i(w)\ \forall w
  \in N(v)\}|} $$
\end{itemize}

{\bf Correctness} We show that this procedure is correctly represented by the
provided Markov chain.

\begin{itemize}
  \item All possible $q$-colorings represent all the states of the procedure.
  As a result, they each have a vertex in the Markov chain. There is no other
  state needed to remember within the procedure.
  \item The first starting state of the procedure is the initial state. The
  initial starting state is the initial coloring of the graph.
  \item We can only transition between two states by changing a single
  vertex's color to another valid color. Since the current vertex color is
  also valid, there are self-loops.
  \item To transition between two states where a single vertex changes color,
  there are $|V|$ vertices to choose from and each vertex $v$ has $|\{c \mid
  c \neq f_i(w)\ \forall w \in N(v)\}|$ possible colors to choose from.
  Therefore, the probability of choosing one (since all are uniform) is 
  $$ \frac{1}{|V|} \cdot \frac{1}{|\{c \mid c \neq f_i(w)\ \forall w \in
  N(v)\}|} $$

  Furthermore, if transitioning in a self-loop, each vertex can change back
  to itself. Therefore, the same probability is summed over all vertices,
  $$ \frac{1}{|V|} \sum_{v \in V} \frac{1}{|\{c \mid c \neq f_i(w)\ \forall w
  \in N(v)\}|} $$
\end{itemize}

By definition, the the total weight going out from a vertex is summed to $1$.
Also, each of the state transitions are independent of previous transitions.
Therefore, this is a valid Markov chain.

\problempart % Problem 2b

To show that this Markov chain is strongly connected, we develop a process to
convert any valid $q$-coloring to another valid $q$-coloring using a walk in
the Markov chain (i.e. one vertex at a time such that each intermediate state
is also a valid $q$-coloring).

{\bf Description} Let $f_a$ be a valid, initial $q$-coloring for a graph $G$.
Let $f_b$ be the valid $q$-coloring we wish to transition to from $f_a$ using
the Markov chain defined in Part A. For every vertex $v \in V$, if $f_a(v) =
f_b(v)$, don't make a change to the graph coloring. If $f_a(v) \neq f_b(v)$
and $f_b(v) \not\in \{f_a(w) \mid\ \forall w \in N(v)\}$, change $f_a(v)$
such that $f_a'(v) = f_b(v)$. If, on the other hand, $f_b(v) \in \{f_a(w)
\mid\ \forall w \in N(v)\}$, find all vertices $w \in N(v)$ such that $f_a(w)
= f_b(v)$ (i.e. the conflicting colored neighbor vertex) and change $f_a(w)$
such that $f_a'(w) \neq f_b(v)$ and $f_a'(w) \neq f_a(u)\ \forall u \in
N(w)$. Now change $f_a'(v)$ such that $f_a''(v) = f_b(v)$.

{\bf Correctness} Considering case-by-case. If $f_a(v) = f_b(v)$, there is
nothing that needs to be done as the vertex is already correctly colored. If
$f_a(v) \neq f_b(v)$ and all the neighboring colors of $v$ in the current
state do not conflict with $f_b(v)$, then it is trivial to simply change that
color to the desired color. If $f_a(v) \neq f_b(v)$ and there are neighboring
colors of $v$ in the current state that conflict with $f_b(v)$, they must
first be changed before changing the color of $v$. Since $q$ is at least
$d_{\mathrm{max}} + 2$, every vertex has at least one color it can be changed
to without conflicting with any of its neighbors (the bound of the chromatic
number is $d_{\mathrm{max}} + 1$). We use this fact to change the color of
the conflicting neighbors to non-conflicting states. Then we are free to
change $v$ to the desired color $f_b(v)$. Note: This will never overwrite the
colors of previously corrected vertices as the resulting $q$-coloring $f_b$
must be valid and therefore any already-touched vertices cannot be
conflicting with $f_b(v)$.

Therefore, the Markov chain is strongly connected as any two colorings can be
transitioned to using the Markov chain.

\problempart % Problem 2c

Given that there are vertices in the Markov chain $H$ that have self-loops of
length 1 (as shown in Part A) and the graph is strongly connected (as shown
in Part B), the greatest common divisor of all cycle lengths in $H$ is
trivially 1.

\end{problemparts}

\newpage
\problem  % Problem 3

\begin{problemparts}

\problempart % Problem 3a

{\bf Description} The walk matrix for graph $G$ is
$$ W = \begin{pmatrix}
    0 & 1/2 & 1/3 &   0 & 1/2 \\
  1/3 &   0 & 1/3 &   0 &   0 \\
  1/3 & 1/2 &   0 & 1/2 &   0 \\
    0 &   0 & 1/3 &   0 & 1/2 \\
  1/3 &   0 &   0 & 1/2 &   0
\end{pmatrix} $$

{\bf Correctness} The element in row $i$ and column $j$ of the matrix $W$
represents the probability of transitioning from the $j$th vertex to the
$i$th vertex of graph $G$.

Vertex 1 in $G$ is connected to vertices 2, 3, and 5. Each have an equal
likelihood of being chosen so transition probabilities are each $1/3$. Hence
column 1 in matrix $W$ has elements 2, 3, and 5 equal to $1/3$ and the rest
are equal to zero as these vertices are unreachable directly from vertex 1. A
similar argument can be constructed for the remaining vertices of $G$ and
columns of $W$.

Furthermore, $W p_t = p_{t + 1}$ must be satisfied by definition. Given by
the setup where each column of $W$ represents transitions from a single
vertex of $G$ and the definition of matrix multiplication, this is trivially
true.

\problempart % Problem 3b

{\bf Description} The stationary distribution of graph $G$ is
$$ \pi = \begin{pmatrix}
  1/4 \\
  1/6 \\
  1/4 \\
  1/6 \\
  1/6
\end{pmatrix} $$

{\bf Correctness} The eigendecomposition of the matrix $W$ yields the
following eigenvalues
$$ \lambda_1 = 1, \lambda_2 = -5/6, \lambda_3 = -1/2, \lambda_4 = 0,
\lambda_5 = 1/3 $$
The eigenvector corresponding to the largest, positive eigenvalue yields the
stationary distribution. This eigenvector (normalized such that the sum is
equal to 1) is
$$ v_1 = \begin{pmatrix}
  1/4 \\
  1/6 \\
  1/4 \\
  1/6 \\
  1/6
\end{pmatrix} $$
Thus, this is the stationary distribution $\pi$.

Note: This can also be seen naively by computing $W^{100}$ which is
$$ W^{100} \approx \begin{pmatrix}
  1/4 & 1/4 & 1/4 & 1/4 & 1/4 \\
  1/6 & 1/6 & 1/6 & 1/6 & 1/6 \\
  1/4 & 1/4 & 1/4 & 1/4 & 1/4 \\
  1/6 & 1/6 & 1/6 & 1/6 & 1/6 \\
  1/6 & 1/6 & 1/6 & 1/6 & 1/6
\end{pmatrix} $$

\problempart % Problem 3c

{\bf Description} The minimal value of $t$ such that each entry of $p_t$ is
within an additive error of 0.01 of the corresponding entry of the stationary
distribution from Part B is $t = 17$.

{\bf Correctness} We can compute $W^{16} p_0 = p_{16}$
$$ p_{16} \approx \begin{pmatrix}
  0.261 \\
  0.167 \\
  0.239 \\
  0.177 \\
  0.156
\end{pmatrix} $$
The difference between $p_{16}$ and $p_{\infty}$ is
$$ |p_{16} - p_{\infty}| \approx \begin{pmatrix}
  0.011 \\
  0.000 \\
  0.011 \\
  0.011 \\
  0.011
\end{pmatrix} $$
Therefore, $p_{16}$ is not quite close enough to $p_{\infty}$. We can also
compute $W^{17} p_0 = p_{17}$
$$ p_{17} \approx \begin{pmatrix}
  0.241 \\
  0.167 \\
  0.259 \\
  0.158 \\
  0.176
\end{pmatrix} $$
The difference between $p_{17}$ and $p_{\infty}$ is
$$ |p_{17} - p_{\infty}| \approx \begin{pmatrix}
  0.009 \\
  0.000 \\
  0.009 \\
  0.009 \\
  0.009
\end{pmatrix} $$
Therefore, $p_{17}$ is the first $p_t$ close enough to $p_{\infty}$ within an
additive error of 0.01.

\end{problemparts}

\end{problems}

\end{document}


