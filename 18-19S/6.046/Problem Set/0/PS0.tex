%
% 6.046 problem set 0
%
\documentclass[12pt,twoside]{article}

\input{macros}
\newcommand{\theproblemsetnum}{0}
\newcommand{\releasedate}{Tuesday, February 5}
\newcommand{\partaduedate}{Friday, February 8}

\title{6.046 Problem Set \theproblemsetnum}

\begin{document}

\handout{Problem Set \theproblemsetnum}{\releasedate}
\textbf{All parts are due {\bf \partaduedate} at {\bf 10PM}}.

\setlength{\parindent}{0pt}
\medskip\hrulefill\medskip

{\bf Name:} Robert Durfee

\medskip

{\bf Collaborators:} None

\medskip\hrulefill

\begin{problems}

\problem

\begin{problemparts}

\problempart In this case, everyone should have started looking at the
problems for at least 40 minutes each before coming together to discuss.
However, they should still separate to write up their solutions individually.

\problempart You should have looked at the problem individually before and
then come together if things were unclear. You should not trade between work,
but understand the whole problem individually.

\problempart The solution should be independently written up without
consulting each other's work verbatim.

\problempart I don't see how this is wrong. The friend helped with a small
error, but didn't give away the entire solution, and pointed them into the
right direction.

\problempart This is probably not the best idea. You should work with them by
leading them to the right answer, but not by showing them the completed
problem.

\problempart Great idea! This gives constructive feedback and makes each
other's work better without giving away answers. Just be careful that
everyone did their best first.

\problempart You should probably review the solution that you wrote before,
otherwise you may not have complete understanding now.

\problempart Make sure you understand the solution such that you could
explain it orally without referencing material.

\problempart Better, but make sure others have attempted the problem first
and don't simply read the solution to the others.

\end{problemparts}

\end{problems}

\end{document}


