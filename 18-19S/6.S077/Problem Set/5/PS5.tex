%
% 6.S077 problem set solutions template
%
\documentclass[12pt,twoside]{article}
\usepackage{bbm}

\newcommand{\name}{}

\usepackage{amssymb}
\usepackage{amsmath}
\usepackage{graphicx}
\usepackage{latexsym}
\usepackage{times,url}
\usepackage{cprotect}
\usepackage{listings}
\usepackage{graphicx}
\usepackage[table]{xcolor}
\usepackage[letterpaper]{geometry}
\usepackage{tikz-qtree}
\usepackage{enumerate}

\newcommand{\profs}{Mauricio Karchmer, Aleksander Madry, Bruce Tidor}
\newcommand{\subj}{6.046}
\newcommand{\ttt}[1]{{\tt\small #1}}

\definecolor{dkgreen}{rgb}{0,0.6,0}
\definecolor{gray}{rgb}{0.5,0.5,0.5}
\definecolor{mauve}{rgb}{0.58,0,0.82}

\lstset{
  language=Python,
  aboveskip=1pc,
  belowskip=1pc,
  basicstyle={\footnotesize\ttfamily},
  numbers=left,
  showstringspaces=false,
  numberstyle=\tiny\color{gray},
  keywordstyle=\color{blue},
  commentstyle=\color{dkgreen},
  stringstyle=\color{mauve},
}

\tikzset{
  % every node/.style={minimum width=2em,draw,circle},
  % level 1/.style={sibling distance=2cm},
  level distance=1cm,
  edge from parent/.style=
  {draw,edge from parent path={(\tikzparentnode) -- (\tikzchildnode)}},
}

\newif\ifHideSolutions
\newcommand{\solution}[1]{\color{dkgreen}\textbf{Solution: }#1\color{black}}
\newcommand{\rubric}[1]{\color{dkgreen}{\bf Rubric:} #1\color{black}}

% \HideSolutionsfalse
% \ifHideSolutions
%   \renewcommand{\solution}[1]{}
%   \renewcommand{\rubric}[1]{}
% \fi

\newlength{\toppush}
\setlength{\toppush}{2\headheight}
\addtolength{\toppush}{\headsep}

\newcommand{\htitle}[2]{\noindent\vspace*{-\toppush}\newline\parbox{6.5in}
{\textit{Design and Analysis of Algorithms}\hfill\name\newline
Massachusetts Institute of Technology \hfill #2\newline
\profs\hfill #1 \vspace*{-.5ex}\newline
\mbox{}\hrulefill\mbox{}}\vspace*{1ex}\mbox{}\newline
\begin{center}{\Large\bf #1}\end{center}}

\newcommand{\handout}[2]{\thispagestyle{empty}
 \markboth{#1}{#1}
 \pagestyle{myheadings}\htitle{#1}{#2}}

\newcommand{\lecture}[3]{\thispagestyle{empty}
 \markboth{Lecture #1: #2}{Lecture #1: #2}
 \pagestyle{myheadings}\htitle{Lecture #1: #2}{#3}}

\newcommand{\htitlewithouttitle}[2]{\noindent\vspace*{-\toppush}\newline\parbox{6.5in}
{\textit{Design and Analysis of Algorithms}\hfill#2\newline
Massachusetts Institute of Technology \hfill 6.046\newline
\profs\hfill Handout #1\vspace*{-.5ex}\newline
\mbox{}\hrulefill\mbox{}}\vspace*{1ex}\mbox{}\newline}

\newcommand{\handoutwithouttitle}[2]{\thispagestyle{empty}
 \markboth{Handout \protect\ref{#1}}{Handout \protect\ref{#1}}
 \pagestyle{myheadings}\htitlewithouttitle{\protect\ref{#1}}{#2}}

\newcommand{\exam}[2]{% parameters: exam name, date
 \thispagestyle{empty}
 \markboth{\hspace{1cm}\subj\ #1\hspace{1in}Name\hrulefill\ \ }%
          {\subj\ #1\hspace{1in}Name\hrulefill\ \ }
 \pagestyle{myheadings}\examtitle{#1}{#2}
 \renewcommand{\theproblem}{Problem \arabic{problemnum}}
}
\newcommand{\examsolutions}[3]{% parameters: handout, exam name, date
 \thispagestyle{empty}
 \markboth{Handout \protect\ref{#1}: #2}{Handout \protect\ref{#1}: #2}
% \pagestyle{myheadings}\htitle{\protect\ref{#1}}{#2}{#3}
 \pagestyle{myheadings}\examsolutionstitle{\protect\ref{#1}} {#2}{#3}
 \renewcommand{\theproblem}{Problem \arabic{problemnum}}
}
\newcommand{\examsolutionstitle}[3]{\noindent\vspace*{-\toppush}\newline\parbox{6.5in}
{\textit{Design and Analysis of Algorithms}\hfill#3\newline
Massachusetts Institute of Technology \hfill 6.046\newline
%Singapore-MIT Alliance \hfill SMA5503\newline
\profs\hfill Handout #1\vspace*{-.5ex}\newline
\mbox{}\hrulefill\mbox{}}\vspace*{1ex}\mbox{}\newline
\begin{center}{\Large\bf #2}\end{center}}

\newcommand{\takehomeexam}[2]{% parameters: exam name, date
 \thispagestyle{empty}
 \markboth{\subj\ #1\hfill}{\subj\ #1\hfill}
 \pagestyle{myheadings}\examtitle{#1}{#2}
 \renewcommand{\theproblem}{Problem \arabic{problemnum}}
}

\makeatletter
\newcommand{\exambooklet}[2]{% parameters: exam name, date
 \thispagestyle{empty}
 \markboth{\subj\ #1}{\subj\ #1}
 \pagestyle{myheadings}\examtitle{#1}{#2}
 \renewcommand{\theproblem}{Problem \arabic{problemnum}}
 \renewcommand{\problem}{\newpage
 \item \let\@currentlabel=\theproblem
 \markboth{\subj\ #1, \theproblem}{\subj\ #1, \theproblem}}
}
\makeatother


\newcommand{\examtitle}[2]{\noindent\vspace*{-\toppush}\newline\parbox{6.5in}
{\textit{Design and Analysis of Algorithms}\hfill#2\newline
Massachusetts Institute of Technology \hfill 6.046 Spring 2019\newline
%Singapore-MIT Alliance \hfill SMA5503\newline
\profs\hfill #1\vspace*{-.5ex}\newline
\mbox{}\hrulefill\mbox{}}\vspace*{1ex}\mbox{}\newline
\begin{center}{\Large\bf #1}\end{center}}

\newcommand{\grader}[1]{\hspace{1cm}\textsf{\textbf{#1}}\hspace{1cm}}

\newcommand{\points}[1]{[#1 points]\ }
\newcommand{\parts}[1]
{
  \ifnum#1=1
  (1 part)
  \else
  (#1 parts)
  \fi
  \ 
}

\newcommand{\bparts}{\begin{problemparts}}
\newcommand{\eparts}{\end{problemparts}}
\newcommand{\ppart}{\problempart}

%\newcommand{\lg} {lg\ }

\setlength{\oddsidemargin}{0pt}
\setlength{\evensidemargin}{0pt}
\setlength{\textwidth}{6.5in}
\setlength{\topmargin}{0in}
\setlength{\textheight}{8.5in}


\newcommand{\Spawn}{{\bf spawn} }
\newcommand{\Sync}{{\bf sync}}

\newcommand{\cif}[1]{\mbox{if $#1$}}
\newcommand{\cwhen}[1]{\mbox{when $#1$}}

\newcounter{problemnum}
\newcommand{\theproblem}{Problem \theproblemsetnum-\arabic{problemnum}}
\newenvironment{problems}{
        \begin{list}{{\bf \theproblem. \hspace*{0.5em}}}
        {\setlength{\leftmargin}{0em}
         \setlength{\rightmargin}{0em}
         \setlength{\labelwidth}{0em}
         \setlength{\labelsep}{0em}
         \usecounter{problemnum}}}{\end{list}}
\makeatletter
\newcommand{\problem}[1][{}]{\item \let\@currentlabel=\theproblem \textbf{#1}}
\makeatother

\newcounter{problempartnum}[problemnum]
\newenvironment{problemparts}{
        \begin{list}{{\bf (\alph{problempartnum})}}
        {\setlength{\leftmargin}{2.5em}
         \setlength{\rightmargin}{2.5em}
         \setlength{\labelsep}{0.5em}}}{\end{list}}
\newcommand{\problempart}{\addtocounter{problempartnum}{1}\item}

\newenvironment{truefalseproblemparts}{
        \begin{list}{{\bf (\alph{problempartnum})\ \ \ T\ \ F\hfil}}
        {\setlength{\leftmargin}{4.5em}
         \setlength{\rightmargin}{2.5em}
         \setlength{\labelsep}{0.5em}
         \setlength{\labelwidth}{4.5em}}}{\end{list}}

\newcounter{exercisenum}
\newcommand{\theexercise}{Exercise \theproblemsetnum-\arabic{exercisenum}}
\newenvironment{exercises}{
        \begin{list}{{\bf \theexercise. \hspace*{0.5em}}}
        {\setlength{\leftmargin}{0em}
         \setlength{\rightmargin}{0em}
         \setlength{\labelwidth}{0em}
         \setlength{\labelsep}{0em}
        \usecounter{exercisenum}}}{\end{list}}
\makeatletter
\newcommand{\exercise}{\item \let\@currentlabel=\theexercise}
\makeatother

\newcounter{exercisepartnum}[exercisenum]
%\newcommand{\problem}[1]{\medskip\mbox{}\newline\noindent{\bf Problem #1.}\hspace*{1em}}
%\newcommand{\exercise}[1]{\medskip\mbox{}\newline\noindent{\bf Exercise #1.}\hspace*{1em}}

\newenvironment{exerciseparts}{
        \begin{list}{{\bf (\alph{exercisepartnum})}}
        {\setlength{\leftmargin}{2.5em}
         \setlength{\rightmargin}{2.5em}
         \setlength{\labelsep}{0.5em}}}{\end{list}}
\newcommand{\exercisepart}{\addtocounter{exercisepartnum}{1}\item}


% Macros to make captions print with small type and 'Figure xx' in bold.
\makeatletter
\def\fnum@figure{{\bf Figure \thefigure}}
\def\fnum@table{{\bf Table \thetable}}
\let\@mycaption\caption
%\long\def\@mycaption#1[#2]#3{\addcontentsline{\csname
%  ext@#1\endcsname}{#1}{\protect\numberline{\csname 
%  the#1\endcsname}{\ignorespaces #2}}\par
%  \begingroup
%    \@parboxrestore
%    \small
%    \@makecaption{\csname fnum@#1\endcsname}{\ignorespaces #3}\par
%  \endgroup}
%\def\mycaption{\refstepcounter\@captype \@dblarg{\@mycaption\@captype}}
%\makeatother
\let\mycaption\caption
%\newcommand{\figcaption}[1]{\mycaption[]{#1}}

\newcounter{totalcaptions}
\newcounter{totalart}

\newcommand{\figcaption}[1]{\addtocounter{totalcaptions}{1}\caption[]{#1}}

% \psfigures determines what to do for figures:
%       0 means just leave vertical space
%       1 means put a vertical rule and the figure name
%       2 means insert the PostScript version of the figure
%       3 means put the figure name flush left or right
\newcommand{\psfigures}{0}
\newcommand{\spacefigures}{\renewcommand{\psfigures}{0}}
\newcommand{\rulefigures}{\renewcommand{\psfigures}{1}}
\newcommand{\macfigures}{\renewcommand{\psfigures}{2}}
\newcommand{\namefigures}{\renewcommand{\psfigures}{3}}

\newcommand{\figpart}[1]{{\bf (#1)}\nolinebreak[2]\relax}
\newcommand{\figparts}[2]{{\bf (#1)--(#2)}\nolinebreak[2]\relax}


\macfigures     % STATE

% When calling \figspace, make sure to leave a blank line afterward!!
% \widefigspace is for figures that are more than 28pc wide.
\newlength{\halffigspace} \newlength{\wholefigspace}
\newlength{\figruleheight} \newlength{\figgap}
\newcommand{\setfiglengths}{\ifnum\psfigures=1\setlength{\figruleheight}{\hruleheight}\setlength{\figgap}{1em}\else\setlength{\figruleheight}{0pt}\setlength{\figgap}{0em}\fi}
\newcommand{\figspace}[2]{\ifnum\psfigures=0\leavefigspace{#1}\else%
\setfiglengths%
\setlength{\wholefigspace}{#1}\setlength{\halffigspace}{.5\wholefigspace}%
\rule[-\halffigspace]{\figruleheight}{\wholefigspace}\hspace{\figgap}#2\fi}
\newlength{\widefigspacewidth}
% Make \widefigspace put the figure flush right on the text page.
\newcommand{\widefigspace}[2]{
\ifnum\psfigures=0\leavefigspace{#1}\else%
\setfiglengths%
\setlength{\widefigspacewidth}{28pc}%
\addtolength{\widefigspacewidth}{-\figruleheight}%
\setlength{\wholefigspace}{#1}\setlength{\halffigspace}{.5\wholefigspace}%
\makebox[\widefigspacewidth][r]{#2\hspace{\figgap}}\rule[-\halffigspace]{\figruleheight}{\wholefigspace}\fi}
\newcommand{\leavefigspace}[1]{\setlength{\wholefigspace}{#1}\setlength{\halffigspace}{.5\wholefigspace}\rule[-\halffigspace]{0em}{\wholefigspace}}

% Commands for including figures with macpsfig.
% To use these commands, documentstyle ``macpsfig'' must be specified.
\newlength{\macfigfill}
\makeatother
\newlength{\bbx}
\newlength{\bby}
\newcommand{\macfigure}[5]{\addtocounter{totalart}{1}
\ifnum\psfigures=2%
\setlength{\bbx}{#2}\addtolength{\bbx}{#4}%
\setlength{\bby}{#3}\addtolength{\bby}{#5}%
\begin{flushleft}
\ifdim#4>28pc\setlength{\macfigfill}{#4}\addtolength{\macfigfill}{-28pc}\hspace*{-\macfigfill}\fi%
\mbox{\psfig{figure=./#1.ps,%
bbllx=#2,bblly=#3,bburx=\bbx,bbury=\bby}}
\end{flushleft}%
\else\ifdim#4>28pc\widefigspace{#5}{#1}\else\figspace{#5}{#1}\fi\fi}
\makeatletter

\newlength{\savearraycolsep}
\newcommand{\narrowarray}[1]{\setlength{\savearraycolsep}{\arraycolsep}\setlength{\arraycolsep}{#1\arraycolsep}}
\newcommand{\normalarray}{\setlength{\arraycolsep}{\savearraycolsep}}

\newcommand{\hint}{{\em Hint:\ }}

% Macros from /th/u/clr/mac.tex

\newcommand{\set}[1]{\left\{ #1 \right\}}
\newcommand{\abs}[1]{\left| #1\right|}
\newcommand{\card}[1]{\left| #1\right|}
\newcommand{\floor}[1]{\left\lfloor #1 \right\rfloor}
\newcommand{\ceil}[1]{\left\lceil #1 \right\rceil}
\newcommand{\ang}[1]{\ifmmode{\left\langle #1 \right\rangle}
   \else{$\left\langle${#1}$\right\rangle$}\fi}
        % the \if allows use outside mathmode,
        % but will swallow following space there!
\newcommand{\paren}[1]{\left( #1 \right)}
\newcommand{\bracket}[1]{\left[ #1 \right]}
\newcommand{\prob}[1]{\Pr\left\{ #1 \right\}}
\newcommand{\Var}{\mathop{\rm Var}\nolimits}
\newcommand{\expect}[1]{{\rm E}\left[ #1 \right]}
\newcommand{\expectsq}[1]{{\rm E}^2\left[ #1 \right]}
\newcommand{\variance}[1]{{\rm Var}\left[ #1 \right]}
\renewcommand{\choose}[2]{{{#1}\atopwithdelims(){#2}}}
\def\pmod#1{\allowbreak\mkern12mu({\rm mod}\,\,#1)}
\newcommand{\matx}[2]{\left(\begin{array}{*{#1}{c}}#2\end{array}\right)}
\newcommand{\Adj}{\mathop{\rm Adj}\nolimits}

\newtheorem{theorem}{Theorem}
\newtheorem{lemma}[theorem]{Lemma}
\newtheorem{corollary}[theorem]{Corollary}
\newtheorem{xample}{Example}
\newtheorem{definition}{Definition}
\newenvironment{example}{\begin{xample}\rm}{\end{xample}}
\newcommand{\proof}{\noindent{\em Proof.}\hspace{1em}}
\def\squarebox#1{\hbox to #1{\hfill\vbox to #1{\vfill}}}
\newcommand{\qedbox}{\vbox{\hrule\hbox{\vrule\squarebox{.667em}\vrule}\hrule}}
\newcommand{\qed}{\nopagebreak\mbox{}\hfill\qedbox\smallskip}
\newcommand{\eqnref}[1]{(\protect\ref{#1})}

%%\newcommand{\twodots}{\mathinner{\ldotp\ldotp}}
\newcommand{\transpose}{^{\mbox{\scriptsize \sf T}}}
\newcommand{\amortized}[1]{\widehat{#1}}

\newcommand{\punt}[1]{}

%%% command for putting definitions into boldface
% New style for defined terms, as of 2/23/88, redefined by THC.
\newcommand{\defn}[1]{{\boldmath\textit{\textbf{#1}}}}
\newcommand{\defi}[1]{{\textit{\textbf{#1\/}}}}

\newcommand{\red}{\leq_{\rm P}}
\newcommand{\lang}[1]{%
\ifmmode\mathord{\mathcode`-="702D\rm#1\mathcode`\-="2200}\else{\rm#1}\fi}

%\newcommand{\ckt}[1]{\ifmmode\mathord{\mathcode`-="702D\sc #1\mathcode`\-="2200}\else$\mathord{\mathcode`-="702D\sc #1\mathcode`\-="2200}$\fi}
\newcommand{\ckt}[1]{\ifmmode \sc #1\else$\sc #1$\fi}

%% Margin notes - use \notesfalse to turn off notes.
\setlength{\marginparwidth}{0.6in}
\reversemarginpar
\newif\ifnotes
\notestrue
\newcommand{\longnote}[1]{
  \ifnotes
    {\medskip\noindent Note: \marginpar[\hfill$\Longrightarrow$]
      {$\Longleftarrow$}{#1}\medskip}
  \fi}
\newcommand{\note}[1]{
  \ifnotes
    {\marginpar{\tiny \raggedright{#1}}}
  \fi}


\newcommand{\reals}{\mathbbm{R}}
\newcommand{\integers}{\mathbbm{Z}}
\newcommand{\naturals}{\mathbbm{N}}
\newcommand{\rationals}{\mathbbm{Q}}
\newcommand{\complex}{\mathbbm{C}}

\newcommand{\oldreals}{{\bf R}}
\newcommand{\oldintegers}{{\bf Z}}
\newcommand{\oldnaturals}{{\bf N}}
\newcommand{\oldrationals}{{\bf Q}}
\newcommand{\oldcomplex}{{\bf C}}

\newcommand{\w}{\omega}                 %% for fft chapter

\newenvironment{closeitemize}{\begin{list}
{$\bullet$}
{\setlength{\itemsep}{-0.2\baselineskip}
\setlength{\topsep}{0.2\baselineskip}
\setlength{\parskip}{0pt}}}
{\end{list}}

% These are necessary within a {problems} environment in order to restore
% the default separation between bullets and items.
\newenvironment{normalitemize}{\setlength{\labelsep}{0.5em}\begin{itemize}}
                              {\end{itemize}}
\newenvironment{normalenumerate}{\setlength{\labelsep}{0.5em}\begin{enumerate}}
                                {\end{enumerate}}

%\def\eqref#1{Equation~(\ref{eq:#1})}
%\newcommand{\eqref}[1]{Equation (\ref{eq:#1})}
\newcommand{\eqreftwo}[2]{Equations (\ref{eq:#1}) and~(\ref{eq:#2})}
\newcommand{\ineqref}[1]{Inequality~(\ref{ineq:#1})}
\newcommand{\ineqreftwo}[2]{Inequalities (\ref{ineq:#1}) and~(\ref{ineq:#2})}

\newcommand{\figref}[1]{Figure~\ref{fig:#1}}
\newcommand{\figreftwo}[2]{Figures \ref{fig:#1} and~\ref{fig:#2}}

\newcommand{\liref}[1]{line~\ref{li:#1}}
\newcommand{\Liref}[1]{Line~\ref{li:#1}}
\newcommand{\lirefs}[2]{lines \ref{li:#1}--\ref{li:#2}}
\newcommand{\Lirefs}[2]{Lines \ref{li:#1}--\ref{li:#2}}
\newcommand{\lireftwo}[2]{lines \ref{li:#1} and~\ref{li:#2}}
\newcommand{\lirefthree}[3]{lines \ref{li:#1}, \ref{li:#2}, and~\ref{li:#3}}

\newcommand{\lemlabel}[1]{\label{lem:#1}}
\newcommand{\lemref}[1]{Lemma~\ref{lem:#1}} 

\newcommand{\exref}[1]{Exercise~\ref{ex:#1}}

\newcommand{\handref}[1]{Handout~\ref{#1}}

\newcommand{\defref}[1]{Definition~\ref{def:#1}}

% (1997.8.16: Victor Luchangco)
% Modified \hlabel to only get date and to use handouts counter for number.
%   New \handout and \handoutwithouttitle commands in newmac.tex use this.
%   The date is referenced by <label>-date.
%   (Retained old definition as \hlabelold.)
%   Defined \hforcelabel to use an argument instead of the handouts counter.

\newcounter{handouts}
\setcounter{handouts}{0}

\newcommand{\hlabel}[2]{%
\stepcounter{handouts}
{\edef\next{\write\@auxout{\string\newlabel{#1}{{\arabic{handouts}}{0}}}}\next}
\write\@auxout{\string\newlabel{#1-date}{{#2}{0}}}
}

\newcommand{\hforcelabel}[3]{%          Does not step handouts counter.
\write\@auxout{\string\newlabel{#1}{{#2}{0}}}
\write\@auxout{\string\newlabel{#1-date}{{#3}{0}}}}


% less ugly underscore
% --juang, 2008 oct 05
\renewcommand{\_}{\vrule height 0 pt depth 0.4 pt width 0.5 em \,}

\newcommand{\theproblemsetnum}{5}
\newcommand{\releasedate}{Tuesday, March 12}
\newcommand{\partaduedate}{Tuesday, March 19}
\allowdisplaybreaks

\title{6.S077 Problem Set \theproblemsetnum}

\begin{document}

\handout{Problem Set \theproblemsetnum}{\releasedate}
\textbf{All parts are due {\bf \partaduedate} at {\bf 2:30PM}}.

\setlength{\parindent}{0pt}
\medskip\hrulefill\medskip

{\bf Name:} Robert Durfee

\medskip

{\bf Collaborators:} None

\medskip\hrulefill

\begin{problems}

\problem  % Problem 1

\begin{problemparts}

\problempart % Problem 1a
%%%%%%%%%%%%%%%%%%%%%%%%%%%%%%%%%%%%%%%%%%%%%%%%%%%%%%%%%%%%%%%%%%%%%%%%%%%%%%%%
The mean-squared error of this estimator is given by,
\begin{align*}
    \mathbb{E}\left[\left(\hat{\theta}(X) - \theta\right)^2\right] &= \mathbb{E}
        \left[(X - \theta)^2\right] \\
    &= \mathbb{E}\left[X^2 - 2X\theta + \theta^2\right] \\
    &= \mathbb{E}\left[X^2\right] - 2\theta\mathbb{E}\left[X\right] + 
        \theta^2
\end{align*}
Given that the variance of $X$ is $\sigma^2$ and the mean is $\theta$,
\begin{align*}
    \mathbb{E}\left[\left(\hat{\theta}(X) - \theta\right)^2\right] &= \sigma^2 + 
        \theta^2 - 2\theta^2 + \theta^2 \\
    &= \sigma^2
\end{align*}

\problempart % Problem 1b

The mean-squared error of this estimator is given by,
\begin{align*}
    \mathbb{E}\left[(\hat{\theta}_\alpha(X) - \theta)^2\right] &= \mathbb{E}\left[
        (\alpha X - \theta)^2\right] \\
    &= \alpha^2 \left(\sigma^2 - \theta^2\right) - 2 \alpha \theta^2 + \theta^2 \\
    &= \alpha^2 \sigma^2 + (1 - \alpha)^2 \theta^2
\end{align*}

\problempart % Problem 1c

We can minimize this expressiong in terms of $\alpha$,
$$ \alpha^* = \arg\min_{\alpha} \alpha^2 \sigma^2 + (1 - \alpha)^2 \theta^2 $$
Taking the derivative with respect to $\alpha$,
$$ \frac{\partial}{\partial \alpha}(\cdot) = 2 \alpha \sigma^2  - 2 (1 - \alpha) 
    \theta^2 $$
Setting equal to zero and solving for $\alpha^*$ yields,
$$ \alpha^* = \frac{\theta^2}{\sigma^2 + \theta^2} $$

\problempart % Problem 1d

Taking the limit of this expression as $\sigma \rightarrow \infty$,
$$ \lim_{\sigma \rightarrow \infty} \frac{\theta^2}{\sigma^2 + \theta^2} = 
    \frac{1}{\infty} = 0 $$
This suggests that as the variance of what we are estimating increases, we 
should decrease $\alpha$. By substituting $\alpha = 0$ into the expression for
mean square error, we get
$$ \mathrm{MSE} = \theta^2 $$
Therefore, we should just return $0$ as our estimator.

\problempart % Problem 1e

Taking the limit of this expression as $\sigma \rightarrow 0$,
$$ \lim_{\sigma \rightarrow 0} \frac{\theta^2}{\sigma^2 + \theta^2} = 
    \frac{\theta^2}{\theta^2} = 1 $$
This suggests that as the variance of what we are estimating decreases, we 
should increase $\alpha$. By substituting $\alpha = 1$ into the expression for
mean sqaured error, we get
$$ \mathrm{MSE} = \sigma^2 $$
Therefore, we should just return $X$ as our estimator.

\problempart % Problem 1f

First, we look at the mean of $\hat{X}_n$,
\begin{align*}
    \mathbb{E}\left[\hat{X}_n\right] &= \mathbb{E}\left[\frac{1}{n} 
        \sum_{i = 1}^n X_i \right] \\
    &= \frac{1}{n} \sum_{i = 1}^n \mathbb{E}[X_i] \\
    &= \frac{1}{n} \sum_{i = 1}^n \theta \\
    &= \theta
\end{align*}

Next, we consider the variance $\hat{X}_n$,
\begin{align*}
    \mathrm{var}(\hat{X}_n) &= \mathrm{var}(\frac{1}{n} \sum_{i = 1}^n X_i) \\
    &= \frac{1}{n^2} \sum_{i = 1}^n \mathrm{var}(X_i) \\
    &= \frac{1}{n^2} \sum_{i = 1}^n \sigma^2 \\
    &= \frac{\sigma}{n}
\end{align*}

Lastly, we know that the sum of Gaussian random variables must be Gaussian,
therefore,
$$ \hat{X}_n \sim  \mathcal{N}\left(\theta, \frac{\sigma}{n}\right) $$

From the previous sections, we know that the $\alpha^*$ must take the form,
$$ \alpha^* = \frac{\theta^2}{\frac{\sigma^2}{n} + \theta^2} $$
Taking the limit as $n \rightarrow \infty$,
$$ \lim_{n \rightarrow \infty} \frac{\theta^2}{\frac{\sigma^2}{n} + \theta^2}
    = \frac{\theta^2}{\theta^2} = 1 $$
This suggests that as the number of samples increases, we should increases
$\alpha$. That is, we should favor $\hat{X}_n$ as an estimator over $0$ as the
number of samples increases.

\end{problemparts}

\newpage

\problem  % Problem 2

\begin{problemparts}

\problempart % Problem 1a

Starting with the standard ridge regression optimization problem,
$$ \beta^* = \arg\min_{\beta} \lVert Y - X \beta \rVert_2^2 + \lambda \lVert 
\beta \rVert_2^2 $$
After substituting the provided values,
$$ \beta_1^*, \beta_2^* = \arg\min_{\beta_1, \beta_2} \left(y_1 - \beta_1
x_{11} - \beta_2 x_{12}\right)^2 + \left(y_2 - \beta_1 x_{21} - \beta_2
x_{22} \right)^2 + \lambda (\beta_1^2 + \beta_2^2) $$
Using the fact that $x_{11} = x_{12}$ and $x_{21} = x_{22}$, this simplifies
to
$$ \beta_1^*, \beta_2^* = \arg\min_{\beta_1, \beta_2} \left(y_1 - (\beta_1 +
\beta_2) x_{12}\right)^2 + \left(y_2 - (\beta_1 + \beta_2) x_{22} \right)^2 +
\lambda (\beta_1^2 + \beta_2^2) $$
Furthermore, since $x_{12} + x_{22} = 0$, this simplifies further to,
$$ \beta_1^*, \beta_2^* = \arg\min_{\beta_1, \beta_2} \left(y_1 - (\beta_1 +
\beta_2) x_{12}\right)^2 + \left(y_2 + (\beta_1 + \beta_2) x_{12} \right)^2 +
\lambda (\beta_1^2 + \beta_2^2) $$
Lastly, since $y_1 + y_2 = 0$, this simplifies to,
$$ \beta_1^*, \beta_2^* = \arg\min_{\beta_1, \beta_2} 2 \left(y_1 - (\beta_1
+ \beta_2) x_{12}\right)^2 + \lambda (\beta_1^2 + \beta_2^2) $$

\problempart % Problem 2b

Taking the derivative of this expression with respect to $\beta_1$ and
$\beta_2$ respectively,
$$ \frac{\partial}{\partial \beta_1}(\cdot) = - 4 x_{12} \left(y_1 - (\beta_1
+ \beta_2) x_{12}\right) + 2 \lambda \beta_1 $$
$$ \frac{\partial}{\partial \beta_2}(\cdot) = - 4 x_{12} \left(y_1 - (\beta_1
+ \beta_2) x_{12}\right) + 2 \lambda \beta_2 $$
Setting equal to zero and solving for $\beta_1^*$ and $\beta_2^*$ yields,
$$ \beta_1^* = \frac{2 x_{12} y_1}{\lambda + 4 x_{12}^2} $$
$$ \beta_2^* = \frac{2 x_{12} y_1}{\lambda + 4 x_{12}^2} $$
Therefore, $\beta_1^* = \beta_2^*$.

\problempart % Problem 2c

Using the same steps as in Part A, we are optimizing,
$$ \beta_1^*, \beta_2^* = \arg\min_{\beta_1, \beta_2} 2 \left(y_1 - (\beta_1
+ \beta_2) x_{12}\right)^2 + 2 \lambda (|\beta_1| + |\beta_2|)$$

\problempart % Problem 2d

Taking the derivative of the expression with respect to $\beta_1$ and $\beta_2$
respectively,
$$ \frac{\partial}{\partial \beta_1}(\cdot) = - 4 x_{12} \left(y_1 - (\beta_1
+ \beta_2) x_{12}\right) + 2 \lambda \mathrm{sgn}(\beta_1) $$
$$ \frac{\partial}{\partial \beta_2}(\cdot) = - 4 x_{12} \left(y_1 - (\beta_1
+ \beta_2) x_{12}\right) + 2 \lambda \mathrm{sgn}(\beta_2) $$
After some simplification, we are left with,
$$ (\beta_1 + \beta_2) = \frac{4 x_{12} y_1 - 2 \lambda
\mathrm{sgn}(\beta_1)}{4 x_{12}^2} $$
$$ (\beta_1 + \beta_2) = \frac{4 x_{12} y_1 - 2 \lambda
\mathrm{sgn}(\beta_2)}{4 x_{12}^2} $$
Thus, when $\beta_1, \beta_2 \leq 0$ or $\beta_1, \beta_2 > 0$, these equations
have infinite solutions and are thus not unique.

\end{problemparts}

\newpage

\problem  % Problem 3

\begin{problemparts}

\problempart % Problem 3a

Starting with the definition of Gaussian distribution,
\begin{align*}
    \mathcal{N}(\mu, 1) &= \frac{1}{\sqrt{2 \pi}}
        \mathrm{exp}\left\{-\frac{(y - \mu)^2}{2}\right\} \\
    &= \frac{1}{\sqrt{2 \pi}} \mathrm{exp}\left\{-\frac{y^2}{2}\right\}
        \mathrm{exp}\left\{y \mu - \frac{\mu^2}{2}\right\}
\end{align*}
Now, it is clear that the following are true,
$$ b(y) = \frac{1}{\sqrt{2 \pi}} \mathrm{exp}\left\{-\frac{y^2}{2}\right\} $$
$$ \eta = \mu $$
$$ a(\eta) = \frac{\eta^2}{2} $$
$$ T(y) = y $$

\problempart % Problem 3b

Starting with the definition of Bernoulli distribution,
\begin{align*}
    \mathrm{Bernoulli}(y; \mu) &= \mu^y (1 - \mu)^{1 - y} \\
    &= \mathrm{exp}\left\{y \log \mu + (1 - y) \log(1 - \mu) \right\} \\
    &= \mathrm{exp}\left\{y \log \left(\frac{\mu}{1 - \mu}\right) + \log(1 -
    \mu) \right\}
\end{align*}
Now, it is clear that the following are true,
$$ b(y) = 1 $$
$$ \eta = \log \left(\frac{\mu}{1 - \mu}\right) $$
$$ a(\eta) = \log \left(1 + e^\eta\right) $$
$$ T(y) = y $$

\problempart % Problem 3c

Given that
$$ \mathbb{E}\left[Y \mid X = x\right] = \mu $$
And that, for Gaussian distributions $\mathcal{N}(\mu, 1)$,
$$ \mu = \eta = \theta^T x $$
It is clear that
$$ \mathbb{E}\left[Y \mid X = x \right] = \theta^T x $$

\problempart % Problem 3d

Given that
$$ \mathbb{E}\left[Y \mid X = x\right] = \mu $$
And that, for Bernoulli distributions,
$$ \eta = \theta^T x $$
We just need to solve $\eta$ for $\mu$,
\begin{align*}
    \eta &= \log \left(\frac{\mu}{1 - \mu}\right) \\
    &\iff e^\eta = \frac{\mu}{1 - \mu} \\
    &\iff e^\eta - \mu e^\eta = \mu \\
    &\iff e^\eta = \mu + \mu e^\eta \\
    &\iff e^\eta = \mu \left(1 + e^\eta\right) \\
    &\iff \mu = \frac{e^\eta}{1 + e^\eta}
\end{align*}
Substituting $\eta = \theta^T x$ gives us,
$$ \mathbb{E}\left[Y \mid X = x\right] = \frac{e^{\theta^T x}}{1 +
e^{\theta^T x}} $$

\end{problemparts}

\end{problems}

\end{document}


