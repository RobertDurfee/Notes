
    




    
\documentclass[11pt]{article}

    
    \usepackage[breakable]{tcolorbox}
    \tcbset{nobeforeafter} % prevents tcolorboxes being placing in paragraphs
    \usepackage{float}
    \floatplacement{figure}{H} % forces figures to be placed at the correct location
    
    \usepackage[T1]{fontenc}
    % Nicer default font (+ math font) than Computer Modern for most use cases
    \usepackage{mathpazo}

    % Basic figure setup, for now with no caption control since it's done
    % automatically by Pandoc (which extracts ![](path) syntax from Markdown).
    \usepackage{graphicx}
    % We will generate all images so they have a width \maxwidth. This means
    % that they will get their normal width if they fit onto the page, but
    % are scaled down if they would overflow the margins.
    \makeatletter
    \def\maxwidth{\ifdim\Gin@nat@width>\linewidth\linewidth
    \else\Gin@nat@width\fi}
    \makeatother
    \let\Oldincludegraphics\includegraphics
    % Set max figure width to be 80% of text width, for now hardcoded.
    \renewcommand{\includegraphics}[1]{\Oldincludegraphics[width=.8\maxwidth]{#1}}
    % Ensure that by default, figures have no caption (until we provide a
    % proper Figure object with a Caption API and a way to capture that
    % in the conversion process - todo).
    \usepackage{caption}
    \DeclareCaptionLabelFormat{nolabel}{}
    \captionsetup{labelformat=nolabel}

    \usepackage{adjustbox} % Used to constrain images to a maximum size 
    \usepackage{xcolor} % Allow colors to be defined
    \usepackage{enumerate} % Needed for markdown enumerations to work
    \usepackage{geometry} % Used to adjust the document margins
    \usepackage{amsmath} % Equations
    \usepackage{amssymb} % Equations
    \usepackage{textcomp} % defines textquotesingle
    % Hack from http://tex.stackexchange.com/a/47451/13684:
    \AtBeginDocument{%
        \def\PYZsq{\textquotesingle}% Upright quotes in Pygmentized code
    }
    \usepackage{upquote} % Upright quotes for verbatim code
    \usepackage{eurosym} % defines \euro
    \usepackage[mathletters]{ucs} % Extended unicode (utf-8) support
    \usepackage[utf8x]{inputenc} % Allow utf-8 characters in the tex document
    \usepackage{fancyvrb} % verbatim replacement that allows latex
    \usepackage{grffile} % extends the file name processing of package graphics 
                         % to support a larger range 
    % The hyperref package gives us a pdf with properly built
    % internal navigation ('pdf bookmarks' for the table of contents,
    % internal cross-reference links, web links for URLs, etc.)
    \usepackage{hyperref}
    \usepackage{longtable} % longtable support required by pandoc >1.10
    \usepackage{booktabs}  % table support for pandoc > 1.12.2
    \usepackage[inline]{enumitem} % IRkernel/repr support (it uses the enumerate* environment)
    \usepackage[normalem]{ulem} % ulem is needed to support strikethroughs (\sout)
                                % normalem makes italics be italics, not underlines
    \usepackage{mathrsfs}
    

    
    % Colors for the hyperref package
    \definecolor{urlcolor}{rgb}{0,.145,.698}
    \definecolor{linkcolor}{rgb}{.71,0.21,0.01}
    \definecolor{citecolor}{rgb}{.12,.54,.11}

    % ANSI colors
    \definecolor{ansi-black}{HTML}{3E424D}
    \definecolor{ansi-black-intense}{HTML}{282C36}
    \definecolor{ansi-red}{HTML}{E75C58}
    \definecolor{ansi-red-intense}{HTML}{B22B31}
    \definecolor{ansi-green}{HTML}{00A250}
    \definecolor{ansi-green-intense}{HTML}{007427}
    \definecolor{ansi-yellow}{HTML}{DDB62B}
    \definecolor{ansi-yellow-intense}{HTML}{B27D12}
    \definecolor{ansi-blue}{HTML}{208FFB}
    \definecolor{ansi-blue-intense}{HTML}{0065CA}
    \definecolor{ansi-magenta}{HTML}{D160C4}
    \definecolor{ansi-magenta-intense}{HTML}{A03196}
    \definecolor{ansi-cyan}{HTML}{60C6C8}
    \definecolor{ansi-cyan-intense}{HTML}{258F8F}
    \definecolor{ansi-white}{HTML}{C5C1B4}
    \definecolor{ansi-white-intense}{HTML}{A1A6B2}
    \definecolor{ansi-default-inverse-fg}{HTML}{FFFFFF}
    \definecolor{ansi-default-inverse-bg}{HTML}{000000}

    % commands and environments needed by pandoc snippets
    % extracted from the output of `pandoc -s`
    \providecommand{\tightlist}{%
      \setlength{\itemsep}{0pt}\setlength{\parskip}{0pt}}
    \DefineVerbatimEnvironment{Highlighting}{Verbatim}{commandchars=\\\{\}}
    % Add ',fontsize=\small' for more characters per line
    \newenvironment{Shaded}{}{}
    \newcommand{\KeywordTok}[1]{\textcolor[rgb]{0.00,0.44,0.13}{\textbf{{#1}}}}
    \newcommand{\DataTypeTok}[1]{\textcolor[rgb]{0.56,0.13,0.00}{{#1}}}
    \newcommand{\DecValTok}[1]{\textcolor[rgb]{0.25,0.63,0.44}{{#1}}}
    \newcommand{\BaseNTok}[1]{\textcolor[rgb]{0.25,0.63,0.44}{{#1}}}
    \newcommand{\FloatTok}[1]{\textcolor[rgb]{0.25,0.63,0.44}{{#1}}}
    \newcommand{\CharTok}[1]{\textcolor[rgb]{0.25,0.44,0.63}{{#1}}}
    \newcommand{\StringTok}[1]{\textcolor[rgb]{0.25,0.44,0.63}{{#1}}}
    \newcommand{\CommentTok}[1]{\textcolor[rgb]{0.38,0.63,0.69}{\textit{{#1}}}}
    \newcommand{\OtherTok}[1]{\textcolor[rgb]{0.00,0.44,0.13}{{#1}}}
    \newcommand{\AlertTok}[1]{\textcolor[rgb]{1.00,0.00,0.00}{\textbf{{#1}}}}
    \newcommand{\FunctionTok}[1]{\textcolor[rgb]{0.02,0.16,0.49}{{#1}}}
    \newcommand{\RegionMarkerTok}[1]{{#1}}
    \newcommand{\ErrorTok}[1]{\textcolor[rgb]{1.00,0.00,0.00}{\textbf{{#1}}}}
    \newcommand{\NormalTok}[1]{{#1}}
    
    % Additional commands for more recent versions of Pandoc
    \newcommand{\ConstantTok}[1]{\textcolor[rgb]{0.53,0.00,0.00}{{#1}}}
    \newcommand{\SpecialCharTok}[1]{\textcolor[rgb]{0.25,0.44,0.63}{{#1}}}
    \newcommand{\VerbatimStringTok}[1]{\textcolor[rgb]{0.25,0.44,0.63}{{#1}}}
    \newcommand{\SpecialStringTok}[1]{\textcolor[rgb]{0.73,0.40,0.53}{{#1}}}
    \newcommand{\ImportTok}[1]{{#1}}
    \newcommand{\DocumentationTok}[1]{\textcolor[rgb]{0.73,0.13,0.13}{\textit{{#1}}}}
    \newcommand{\AnnotationTok}[1]{\textcolor[rgb]{0.38,0.63,0.69}{\textbf{\textit{{#1}}}}}
    \newcommand{\CommentVarTok}[1]{\textcolor[rgb]{0.38,0.63,0.69}{\textbf{\textit{{#1}}}}}
    \newcommand{\VariableTok}[1]{\textcolor[rgb]{0.10,0.09,0.49}{{#1}}}
    \newcommand{\ControlFlowTok}[1]{\textcolor[rgb]{0.00,0.44,0.13}{\textbf{{#1}}}}
    \newcommand{\OperatorTok}[1]{\textcolor[rgb]{0.40,0.40,0.40}{{#1}}}
    \newcommand{\BuiltInTok}[1]{{#1}}
    \newcommand{\ExtensionTok}[1]{{#1}}
    \newcommand{\PreprocessorTok}[1]{\textcolor[rgb]{0.74,0.48,0.00}{{#1}}}
    \newcommand{\AttributeTok}[1]{\textcolor[rgb]{0.49,0.56,0.16}{{#1}}}
    \newcommand{\InformationTok}[1]{\textcolor[rgb]{0.38,0.63,0.69}{\textbf{\textit{{#1}}}}}
    \newcommand{\WarningTok}[1]{\textcolor[rgb]{0.38,0.63,0.69}{\textbf{\textit{{#1}}}}}
    
    
    % Define a nice break command that doesn't care if a line doesn't already
    % exist.
    \def\br{\hspace*{\fill} \\* }
    % Math Jax compatibility definitions
    \def\gt{>}
    \def\lt{<}
    \let\Oldtex\TeX
    \let\Oldlatex\LaTeX
    \renewcommand{\TeX}{\textrm{\Oldtex}}
    \renewcommand{\LaTeX}{\textrm{\Oldlatex}}
    % Document parameters
    % Document title
    \title{NBAHypothesisTesting}
    
    
    
    
    
% Pygments definitions
\makeatletter
\def\PY@reset{\let\PY@it=\relax \let\PY@bf=\relax%
    \let\PY@ul=\relax \let\PY@tc=\relax%
    \let\PY@bc=\relax \let\PY@ff=\relax}
\def\PY@tok#1{\csname PY@tok@#1\endcsname}
\def\PY@toks#1+{\ifx\relax#1\empty\else%
    \PY@tok{#1}\expandafter\PY@toks\fi}
\def\PY@do#1{\PY@bc{\PY@tc{\PY@ul{%
    \PY@it{\PY@bf{\PY@ff{#1}}}}}}}
\def\PY#1#2{\PY@reset\PY@toks#1+\relax+\PY@do{#2}}

\expandafter\def\csname PY@tok@w\endcsname{\def\PY@tc##1{\textcolor[rgb]{0.73,0.73,0.73}{##1}}}
\expandafter\def\csname PY@tok@c\endcsname{\let\PY@it=\textit\def\PY@tc##1{\textcolor[rgb]{0.25,0.50,0.50}{##1}}}
\expandafter\def\csname PY@tok@cp\endcsname{\def\PY@tc##1{\textcolor[rgb]{0.74,0.48,0.00}{##1}}}
\expandafter\def\csname PY@tok@k\endcsname{\let\PY@bf=\textbf\def\PY@tc##1{\textcolor[rgb]{0.00,0.50,0.00}{##1}}}
\expandafter\def\csname PY@tok@kp\endcsname{\def\PY@tc##1{\textcolor[rgb]{0.00,0.50,0.00}{##1}}}
\expandafter\def\csname PY@tok@kt\endcsname{\def\PY@tc##1{\textcolor[rgb]{0.69,0.00,0.25}{##1}}}
\expandafter\def\csname PY@tok@o\endcsname{\def\PY@tc##1{\textcolor[rgb]{0.40,0.40,0.40}{##1}}}
\expandafter\def\csname PY@tok@ow\endcsname{\let\PY@bf=\textbf\def\PY@tc##1{\textcolor[rgb]{0.67,0.13,1.00}{##1}}}
\expandafter\def\csname PY@tok@nb\endcsname{\def\PY@tc##1{\textcolor[rgb]{0.00,0.50,0.00}{##1}}}
\expandafter\def\csname PY@tok@nf\endcsname{\def\PY@tc##1{\textcolor[rgb]{0.00,0.00,1.00}{##1}}}
\expandafter\def\csname PY@tok@nc\endcsname{\let\PY@bf=\textbf\def\PY@tc##1{\textcolor[rgb]{0.00,0.00,1.00}{##1}}}
\expandafter\def\csname PY@tok@nn\endcsname{\let\PY@bf=\textbf\def\PY@tc##1{\textcolor[rgb]{0.00,0.00,1.00}{##1}}}
\expandafter\def\csname PY@tok@ne\endcsname{\let\PY@bf=\textbf\def\PY@tc##1{\textcolor[rgb]{0.82,0.25,0.23}{##1}}}
\expandafter\def\csname PY@tok@nv\endcsname{\def\PY@tc##1{\textcolor[rgb]{0.10,0.09,0.49}{##1}}}
\expandafter\def\csname PY@tok@no\endcsname{\def\PY@tc##1{\textcolor[rgb]{0.53,0.00,0.00}{##1}}}
\expandafter\def\csname PY@tok@nl\endcsname{\def\PY@tc##1{\textcolor[rgb]{0.63,0.63,0.00}{##1}}}
\expandafter\def\csname PY@tok@ni\endcsname{\let\PY@bf=\textbf\def\PY@tc##1{\textcolor[rgb]{0.60,0.60,0.60}{##1}}}
\expandafter\def\csname PY@tok@na\endcsname{\def\PY@tc##1{\textcolor[rgb]{0.49,0.56,0.16}{##1}}}
\expandafter\def\csname PY@tok@nt\endcsname{\let\PY@bf=\textbf\def\PY@tc##1{\textcolor[rgb]{0.00,0.50,0.00}{##1}}}
\expandafter\def\csname PY@tok@nd\endcsname{\def\PY@tc##1{\textcolor[rgb]{0.67,0.13,1.00}{##1}}}
\expandafter\def\csname PY@tok@s\endcsname{\def\PY@tc##1{\textcolor[rgb]{0.73,0.13,0.13}{##1}}}
\expandafter\def\csname PY@tok@sd\endcsname{\let\PY@it=\textit\def\PY@tc##1{\textcolor[rgb]{0.73,0.13,0.13}{##1}}}
\expandafter\def\csname PY@tok@si\endcsname{\let\PY@bf=\textbf\def\PY@tc##1{\textcolor[rgb]{0.73,0.40,0.53}{##1}}}
\expandafter\def\csname PY@tok@se\endcsname{\let\PY@bf=\textbf\def\PY@tc##1{\textcolor[rgb]{0.73,0.40,0.13}{##1}}}
\expandafter\def\csname PY@tok@sr\endcsname{\def\PY@tc##1{\textcolor[rgb]{0.73,0.40,0.53}{##1}}}
\expandafter\def\csname PY@tok@ss\endcsname{\def\PY@tc##1{\textcolor[rgb]{0.10,0.09,0.49}{##1}}}
\expandafter\def\csname PY@tok@sx\endcsname{\def\PY@tc##1{\textcolor[rgb]{0.00,0.50,0.00}{##1}}}
\expandafter\def\csname PY@tok@m\endcsname{\def\PY@tc##1{\textcolor[rgb]{0.40,0.40,0.40}{##1}}}
\expandafter\def\csname PY@tok@gh\endcsname{\let\PY@bf=\textbf\def\PY@tc##1{\textcolor[rgb]{0.00,0.00,0.50}{##1}}}
\expandafter\def\csname PY@tok@gu\endcsname{\let\PY@bf=\textbf\def\PY@tc##1{\textcolor[rgb]{0.50,0.00,0.50}{##1}}}
\expandafter\def\csname PY@tok@gd\endcsname{\def\PY@tc##1{\textcolor[rgb]{0.63,0.00,0.00}{##1}}}
\expandafter\def\csname PY@tok@gi\endcsname{\def\PY@tc##1{\textcolor[rgb]{0.00,0.63,0.00}{##1}}}
\expandafter\def\csname PY@tok@gr\endcsname{\def\PY@tc##1{\textcolor[rgb]{1.00,0.00,0.00}{##1}}}
\expandafter\def\csname PY@tok@ge\endcsname{\let\PY@it=\textit}
\expandafter\def\csname PY@tok@gs\endcsname{\let\PY@bf=\textbf}
\expandafter\def\csname PY@tok@gp\endcsname{\let\PY@bf=\textbf\def\PY@tc##1{\textcolor[rgb]{0.00,0.00,0.50}{##1}}}
\expandafter\def\csname PY@tok@go\endcsname{\def\PY@tc##1{\textcolor[rgb]{0.53,0.53,0.53}{##1}}}
\expandafter\def\csname PY@tok@gt\endcsname{\def\PY@tc##1{\textcolor[rgb]{0.00,0.27,0.87}{##1}}}
\expandafter\def\csname PY@tok@err\endcsname{\def\PY@bc##1{\setlength{\fboxsep}{0pt}\fcolorbox[rgb]{1.00,0.00,0.00}{1,1,1}{\strut ##1}}}
\expandafter\def\csname PY@tok@kc\endcsname{\let\PY@bf=\textbf\def\PY@tc##1{\textcolor[rgb]{0.00,0.50,0.00}{##1}}}
\expandafter\def\csname PY@tok@kd\endcsname{\let\PY@bf=\textbf\def\PY@tc##1{\textcolor[rgb]{0.00,0.50,0.00}{##1}}}
\expandafter\def\csname PY@tok@kn\endcsname{\let\PY@bf=\textbf\def\PY@tc##1{\textcolor[rgb]{0.00,0.50,0.00}{##1}}}
\expandafter\def\csname PY@tok@kr\endcsname{\let\PY@bf=\textbf\def\PY@tc##1{\textcolor[rgb]{0.00,0.50,0.00}{##1}}}
\expandafter\def\csname PY@tok@bp\endcsname{\def\PY@tc##1{\textcolor[rgb]{0.00,0.50,0.00}{##1}}}
\expandafter\def\csname PY@tok@fm\endcsname{\def\PY@tc##1{\textcolor[rgb]{0.00,0.00,1.00}{##1}}}
\expandafter\def\csname PY@tok@vc\endcsname{\def\PY@tc##1{\textcolor[rgb]{0.10,0.09,0.49}{##1}}}
\expandafter\def\csname PY@tok@vg\endcsname{\def\PY@tc##1{\textcolor[rgb]{0.10,0.09,0.49}{##1}}}
\expandafter\def\csname PY@tok@vi\endcsname{\def\PY@tc##1{\textcolor[rgb]{0.10,0.09,0.49}{##1}}}
\expandafter\def\csname PY@tok@vm\endcsname{\def\PY@tc##1{\textcolor[rgb]{0.10,0.09,0.49}{##1}}}
\expandafter\def\csname PY@tok@sa\endcsname{\def\PY@tc##1{\textcolor[rgb]{0.73,0.13,0.13}{##1}}}
\expandafter\def\csname PY@tok@sb\endcsname{\def\PY@tc##1{\textcolor[rgb]{0.73,0.13,0.13}{##1}}}
\expandafter\def\csname PY@tok@sc\endcsname{\def\PY@tc##1{\textcolor[rgb]{0.73,0.13,0.13}{##1}}}
\expandafter\def\csname PY@tok@dl\endcsname{\def\PY@tc##1{\textcolor[rgb]{0.73,0.13,0.13}{##1}}}
\expandafter\def\csname PY@tok@s2\endcsname{\def\PY@tc##1{\textcolor[rgb]{0.73,0.13,0.13}{##1}}}
\expandafter\def\csname PY@tok@sh\endcsname{\def\PY@tc##1{\textcolor[rgb]{0.73,0.13,0.13}{##1}}}
\expandafter\def\csname PY@tok@s1\endcsname{\def\PY@tc##1{\textcolor[rgb]{0.73,0.13,0.13}{##1}}}
\expandafter\def\csname PY@tok@mb\endcsname{\def\PY@tc##1{\textcolor[rgb]{0.40,0.40,0.40}{##1}}}
\expandafter\def\csname PY@tok@mf\endcsname{\def\PY@tc##1{\textcolor[rgb]{0.40,0.40,0.40}{##1}}}
\expandafter\def\csname PY@tok@mh\endcsname{\def\PY@tc##1{\textcolor[rgb]{0.40,0.40,0.40}{##1}}}
\expandafter\def\csname PY@tok@mi\endcsname{\def\PY@tc##1{\textcolor[rgb]{0.40,0.40,0.40}{##1}}}
\expandafter\def\csname PY@tok@il\endcsname{\def\PY@tc##1{\textcolor[rgb]{0.40,0.40,0.40}{##1}}}
\expandafter\def\csname PY@tok@mo\endcsname{\def\PY@tc##1{\textcolor[rgb]{0.40,0.40,0.40}{##1}}}
\expandafter\def\csname PY@tok@ch\endcsname{\let\PY@it=\textit\def\PY@tc##1{\textcolor[rgb]{0.25,0.50,0.50}{##1}}}
\expandafter\def\csname PY@tok@cm\endcsname{\let\PY@it=\textit\def\PY@tc##1{\textcolor[rgb]{0.25,0.50,0.50}{##1}}}
\expandafter\def\csname PY@tok@cpf\endcsname{\let\PY@it=\textit\def\PY@tc##1{\textcolor[rgb]{0.25,0.50,0.50}{##1}}}
\expandafter\def\csname PY@tok@c1\endcsname{\let\PY@it=\textit\def\PY@tc##1{\textcolor[rgb]{0.25,0.50,0.50}{##1}}}
\expandafter\def\csname PY@tok@cs\endcsname{\let\PY@it=\textit\def\PY@tc##1{\textcolor[rgb]{0.25,0.50,0.50}{##1}}}

\def\PYZbs{\char`\\}
\def\PYZus{\char`\_}
\def\PYZob{\char`\{}
\def\PYZcb{\char`\}}
\def\PYZca{\char`\^}
\def\PYZam{\char`\&}
\def\PYZlt{\char`\<}
\def\PYZgt{\char`\>}
\def\PYZsh{\char`\#}
\def\PYZpc{\char`\%}
\def\PYZdl{\char`\$}
\def\PYZhy{\char`\-}
\def\PYZsq{\char`\'}
\def\PYZdq{\char`\"}
\def\PYZti{\char`\~}
% for compatibility with earlier versions
\def\PYZat{@}
\def\PYZlb{[}
\def\PYZrb{]}
\makeatother


    % For linebreaks inside Verbatim environment from package fancyvrb. 
    \makeatletter
        \newbox\Wrappedcontinuationbox 
        \newbox\Wrappedvisiblespacebox 
        \newcommand*\Wrappedvisiblespace {\textcolor{red}{\textvisiblespace}} 
        \newcommand*\Wrappedcontinuationsymbol {\textcolor{red}{\llap{\tiny$\m@th\hookrightarrow$}}} 
        \newcommand*\Wrappedcontinuationindent {3ex } 
        \newcommand*\Wrappedafterbreak {\kern\Wrappedcontinuationindent\copy\Wrappedcontinuationbox} 
        % Take advantage of the already applied Pygments mark-up to insert 
        % potential linebreaks for TeX processing. 
        %        {, <, #, %, $, ' and ": go to next line. 
        %        _, }, ^, &, >, - and ~: stay at end of broken line. 
        % Use of \textquotesingle for straight quote. 
        \newcommand*\Wrappedbreaksatspecials {% 
            \def\PYGZus{\discretionary{\char`\_}{\Wrappedafterbreak}{\char`\_}}% 
            \def\PYGZob{\discretionary{}{\Wrappedafterbreak\char`\{}{\char`\{}}% 
            \def\PYGZcb{\discretionary{\char`\}}{\Wrappedafterbreak}{\char`\}}}% 
            \def\PYGZca{\discretionary{\char`\^}{\Wrappedafterbreak}{\char`\^}}% 
            \def\PYGZam{\discretionary{\char`\&}{\Wrappedafterbreak}{\char`\&}}% 
            \def\PYGZlt{\discretionary{}{\Wrappedafterbreak\char`\<}{\char`\<}}% 
            \def\PYGZgt{\discretionary{\char`\>}{\Wrappedafterbreak}{\char`\>}}% 
            \def\PYGZsh{\discretionary{}{\Wrappedafterbreak\char`\#}{\char`\#}}% 
            \def\PYGZpc{\discretionary{}{\Wrappedafterbreak\char`\%}{\char`\%}}% 
            \def\PYGZdl{\discretionary{}{\Wrappedafterbreak\char`\$}{\char`\$}}% 
            \def\PYGZhy{\discretionary{\char`\-}{\Wrappedafterbreak}{\char`\-}}% 
            \def\PYGZsq{\discretionary{}{\Wrappedafterbreak\textquotesingle}{\textquotesingle}}% 
            \def\PYGZdq{\discretionary{}{\Wrappedafterbreak\char`\"}{\char`\"}}% 
            \def\PYGZti{\discretionary{\char`\~}{\Wrappedafterbreak}{\char`\~}}% 
        } 
        % Some characters . , ; ? ! / are not pygmentized. 
        % This macro makes them "active" and they will insert potential linebreaks 
        \newcommand*\Wrappedbreaksatpunct {% 
            \lccode`\~`\.\lowercase{\def~}{\discretionary{\hbox{\char`\.}}{\Wrappedafterbreak}{\hbox{\char`\.}}}% 
            \lccode`\~`\,\lowercase{\def~}{\discretionary{\hbox{\char`\,}}{\Wrappedafterbreak}{\hbox{\char`\,}}}% 
            \lccode`\~`\;\lowercase{\def~}{\discretionary{\hbox{\char`\;}}{\Wrappedafterbreak}{\hbox{\char`\;}}}% 
            \lccode`\~`\:\lowercase{\def~}{\discretionary{\hbox{\char`\:}}{\Wrappedafterbreak}{\hbox{\char`\:}}}% 
            \lccode`\~`\?\lowercase{\def~}{\discretionary{\hbox{\char`\?}}{\Wrappedafterbreak}{\hbox{\char`\?}}}% 
            \lccode`\~`\!\lowercase{\def~}{\discretionary{\hbox{\char`\!}}{\Wrappedafterbreak}{\hbox{\char`\!}}}% 
            \lccode`\~`\/\lowercase{\def~}{\discretionary{\hbox{\char`\/}}{\Wrappedafterbreak}{\hbox{\char`\/}}}% 
            \catcode`\.\active
            \catcode`\,\active 
            \catcode`\;\active
            \catcode`\:\active
            \catcode`\?\active
            \catcode`\!\active
            \catcode`\/\active 
            \lccode`\~`\~ 	
        }
    \makeatother

    \let\OriginalVerbatim=\Verbatim
    \makeatletter
    \renewcommand{\Verbatim}[1][1]{%
        %\parskip\z@skip
        \sbox\Wrappedcontinuationbox {\Wrappedcontinuationsymbol}%
        \sbox\Wrappedvisiblespacebox {\FV@SetupFont\Wrappedvisiblespace}%
        \def\FancyVerbFormatLine ##1{\hsize\linewidth
            \vtop{\raggedright\hyphenpenalty\z@\exhyphenpenalty\z@
                \doublehyphendemerits\z@\finalhyphendemerits\z@
                \strut ##1\strut}%
        }%
        % If the linebreak is at a space, the latter will be displayed as visible
        % space at end of first line, and a continuation symbol starts next line.
        % Stretch/shrink are however usually zero for typewriter font.
        \def\FV@Space {%
            \nobreak\hskip\z@ plus\fontdimen3\font minus\fontdimen4\font
            \discretionary{\copy\Wrappedvisiblespacebox}{\Wrappedafterbreak}
            {\kern\fontdimen2\font}%
        }%
        
        % Allow breaks at special characters using \PYG... macros.
        \Wrappedbreaksatspecials
        % Breaks at punctuation characters . , ; ? ! and / need catcode=\active 	
        \OriginalVerbatim[#1,codes*=\Wrappedbreaksatpunct]%
    }
    \makeatother

    % Exact colors from NB
    \definecolor{incolor}{HTML}{303F9F}
    \definecolor{outcolor}{HTML}{D84315}
    \definecolor{cellborder}{HTML}{CFCFCF}
    \definecolor{cellbackground}{HTML}{F7F7F7}
    
    % prompt
    \newcommand{\prompt}[4]{
        \llap{{\color{#2}[#3]: #4}}\vspace{-1.25em}
    }
    

    
    % Prevent overflowing lines due to hard-to-break entities
    \sloppy 
    % Setup hyperref package
    \hypersetup{
      breaklinks=true,  % so long urls are correctly broken across lines
      colorlinks=true,
      urlcolor=urlcolor,
      linkcolor=linkcolor,
      citecolor=citecolor,
      }
    % Slightly bigger margins than the latex defaults
    
    \geometry{verbose,tmargin=1in,bmargin=1in,lmargin=1in,rmargin=1in}
    
    

    \begin{document}
    
    
    
    

    
    \begin{tcolorbox}[breakable, size=fbox, boxrule=1pt, pad at break*=1mm,colback=cellbackground, colframe=cellborder]
\prompt{In}{incolor}{1}{\hspace{4pt}}
\begin{Verbatim}[commandchars=\\\{\}]
\PY{k+kn}{import} \PY{n+nn}{pandas} \PY{k}{as} \PY{n+nn}{pd}
\PY{k+kn}{import} \PY{n+nn}{matplotlib}\PY{n+nn}{.}\PY{n+nn}{pyplot} \PY{k}{as} \PY{n+nn}{plt}
\PY{k+kn}{import} \PY{n+nn}{scipy}\PY{n+nn}{.}\PY{n+nn}{stats} \PY{k}{as} \PY{n+nn}{stats}
\end{Verbatim}
\end{tcolorbox}

    \section*{NBA Hypothesis Testing}\label{nba-hypothesis-testing}

\subsection*{Problem 1}\label{problem-1}

First we load the NBA data set into a Pandas dataframe.

    \begin{tcolorbox}[breakable, size=fbox, boxrule=1pt, pad at break*=1mm,colback=cellbackground, colframe=cellborder]
\prompt{In}{incolor}{2}{\hspace{4pt}}
\begin{Verbatim}[commandchars=\\\{\}]
\PY{n}{df} \PY{o}{=} \PY{n}{pd}\PY{o}{.}\PY{n}{read\PYZus{}csv}\PY{p}{(}\PY{l+s+s1}{\PYZsq{}}\PY{l+s+s1}{nba\PYZus{}data.csv}\PY{l+s+s1}{\PYZsq{}}\PY{p}{)}
\end{Verbatim}
\end{tcolorbox}

    The \texttt{date} column is read as a string. We need to convert it to a
\texttt{datetime} data type.

    \begin{tcolorbox}[breakable, size=fbox, boxrule=1pt, pad at break*=1mm,colback=cellbackground, colframe=cellborder]
\prompt{In}{incolor}{3}{\hspace{4pt}}
\begin{Verbatim}[commandchars=\\\{\}]
\PY{n}{df}\PY{p}{[}\PY{l+s+s1}{\PYZsq{}}\PY{l+s+s1}{date}\PY{l+s+s1}{\PYZsq{}}\PY{p}{]} \PY{o}{=} \PY{n}{pd}\PY{o}{.}\PY{n}{to\PYZus{}datetime}\PY{p}{(}\PY{n}{df}\PY{p}{[}\PY{l+s+s1}{\PYZsq{}}\PY{l+s+s1}{date}\PY{l+s+s1}{\PYZsq{}}\PY{p}{]}\PY{p}{,} \PY{n+nb}{format}\PY{o}{=}\PY{l+s+s1}{\PYZsq{}}\PY{l+s+s1}{\PYZpc{}}\PY{l+s+s1}{A, }\PY{l+s+s1}{\PYZpc{}}\PY{l+s+s1}{B }\PY{l+s+si}{\PYZpc{}d}\PY{l+s+s1}{, }\PY{l+s+s1}{\PYZpc{}}\PY{l+s+s1}{Y}\PY{l+s+s1}{\PYZsq{}}\PY{p}{)}
\end{Verbatim}
\end{tcolorbox}

    \subsubsection*{Part A}\label{part-a}

We can count the number of games in the dataset.

    \begin{tcolorbox}[breakable, size=fbox, boxrule=1pt, pad at break*=1mm,colback=cellbackground, colframe=cellborder]
\prompt{In}{incolor}{4}{\hspace{4pt}}
\begin{Verbatim}[commandchars=\\\{\}]
\PY{n+nb}{print}\PY{p}{(}\PY{n}{f}\PY{l+s+s2}{\PYZdq{}}\PY{l+s+s2}{Number of games: }\PY{l+s+s2}{\PYZob{}}\PY{l+s+s2}{len(df)\PYZcb{}}\PY{l+s+s2}{\PYZdq{}}\PY{p}{)}
\end{Verbatim}
\end{tcolorbox}

    \begin{Verbatim}[commandchars=\\\{\}]
Number of games: 60688
\end{Verbatim}

    \subsubsection*{Part B}\label{part-b}

We can look at the columns in the dataset.

    \begin{tcolorbox}[breakable, size=fbox, boxrule=1pt, pad at break*=1mm,colback=cellbackground, colframe=cellborder]
\prompt{In}{incolor}{5}{\hspace{4pt}}
\begin{Verbatim}[commandchars=\\\{\}]
\PY{n+nb}{print}\PY{p}{(}\PY{l+s+s1}{\PYZsq{}}\PY{l+s+s1}{Columns:}\PY{l+s+s1}{\PYZsq{}}\PY{p}{)}
\PY{k}{for} \PY{n}{col} \PY{o+ow}{in} \PY{n}{df}\PY{o}{.}\PY{n}{columns}\PY{p}{:} 
    \PY{n+nb}{print}\PY{p}{(}\PY{n}{f}\PY{l+s+s2}{\PYZdq{}}\PY{l+s+s2}{  }\PY{l+s+si}{\PYZob{}col\PYZcb{}}\PY{l+s+s2}{\PYZdq{}}\PY{p}{)}
\end{Verbatim}
\end{tcolorbox}

    \begin{Verbatim}[commandchars=\\\{\}]
Columns:
  season
  date
  home\_pt
  home\_team
  away\_pt
  away\_team
  finished\_in
  winner\_pt
  margin
\end{Verbatim}

    \subsubsection*{Part C}\label{part-c}

We will create a column for the difference between home and away score
for each game.

    \begin{tcolorbox}[breakable, size=fbox, boxrule=1pt, pad at break*=1mm,colback=cellbackground, colframe=cellborder]
\prompt{In}{incolor}{6}{\hspace{4pt}}
\begin{Verbatim}[commandchars=\\\{\}]
\PY{n}{df}\PY{p}{[}\PY{l+s+s1}{\PYZsq{}}\PY{l+s+s1}{pt\PYZus{}diff}\PY{l+s+s1}{\PYZsq{}}\PY{p}{]} \PY{o}{=} \PY{n}{df}\PY{p}{[}\PY{l+s+s1}{\PYZsq{}}\PY{l+s+s1}{home\PYZus{}pt}\PY{l+s+s1}{\PYZsq{}}\PY{p}{]} \PY{o}{\PYZhy{}} \PY{n}{df}\PY{p}{[}\PY{l+s+s1}{\PYZsq{}}\PY{l+s+s1}{away\PYZus{}pt}\PY{l+s+s1}{\PYZsq{}}\PY{p}{]}
\end{Verbatim}
\end{tcolorbox}

    \subsubsection*{Part D}\label{part-d}

We will also create a column for the total number of points scored in
each game.

    \begin{tcolorbox}[breakable, size=fbox, boxrule=1pt, pad at break*=1mm,colback=cellbackground, colframe=cellborder]
\prompt{In}{incolor}{7}{\hspace{4pt}}
\begin{Verbatim}[commandchars=\\\{\}]
\PY{n}{df}\PY{p}{[}\PY{l+s+s1}{\PYZsq{}}\PY{l+s+s1}{total\PYZus{}pt}\PY{l+s+s1}{\PYZsq{}}\PY{p}{]} \PY{o}{=} \PY{n}{df}\PY{p}{[}\PY{l+s+s1}{\PYZsq{}}\PY{l+s+s1}{home\PYZus{}pt}\PY{l+s+s1}{\PYZsq{}}\PY{p}{]} \PY{o}{+} \PY{n}{df}\PY{p}{[}\PY{l+s+s1}{\PYZsq{}}\PY{l+s+s1}{away\PYZus{}pt}\PY{l+s+s1}{\PYZsq{}}\PY{p}{]}
\end{Verbatim}
\end{tcolorbox}

    \subsubsection*{Part E}\label{part-e}

We create a function \texttt{weekend\_new\_column()} to add two new
columns to the dataframe: - \texttt{day\_of\_week} for the day of the
week the game was held (\(\{0, \ldots, 6\}\)) - \texttt{weekend} for if
the game was on a weekend (\(\{0, 1\}\)).

    \begin{tcolorbox}[breakable, size=fbox, boxrule=1pt, pad at break*=1mm,colback=cellbackground, colframe=cellborder]
\prompt{In}{incolor}{8}{\hspace{4pt}}
\begin{Verbatim}[commandchars=\\\{\}]
\PY{k}{def} \PY{n+nf}{weekend\PYZus{}new\PYZus{}column}\PY{p}{(}\PY{n}{df}\PY{p}{)}\PY{p}{:}
    \PY{l+s+sd}{\PYZdq{}\PYZdq{}\PYZdq{}Adds two columns to the dataframe:}
\PY{l+s+sd}{        \PYZhy{} `day\PYZus{}of\PYZus{}week` for the day of the week the game was held.}
\PY{l+s+sd}{        \PYZhy{} `weekend` for if the game was on a weekend.}
\PY{l+s+sd}{        }
\PY{l+s+sd}{    Args:}
\PY{l+s+sd}{        df (DataFrame): A dataframe containing the existing NBA }
\PY{l+s+sd}{            historical data of games with a column `date`.}
\PY{l+s+sd}{    }
\PY{l+s+sd}{    Returns:}
\PY{l+s+sd}{        DataFrame: New dataframe with the additional two columns.}
\PY{l+s+sd}{        }
\PY{l+s+sd}{    \PYZdq{}\PYZdq{}\PYZdq{}}
    \PY{n}{df}\PY{p}{[}\PY{l+s+s1}{\PYZsq{}}\PY{l+s+s1}{day\PYZus{}of\PYZus{}week}\PY{l+s+s1}{\PYZsq{}}\PY{p}{]} \PY{o}{=} \PY{n}{df}\PY{p}{[}\PY{l+s+s1}{\PYZsq{}}\PY{l+s+s1}{date}\PY{l+s+s1}{\PYZsq{}}\PY{p}{]}\PY{o}{.}\PY{n}{apply}\PY{p}{(}\PY{k}{lambda} \PY{n}{date}\PY{p}{:} \PY{n}{date}\PY{o}{.}\PY{n}{weekday}\PY{p}{(}\PY{p}{)}\PY{p}{)}
    \PY{n}{df}\PY{p}{[}\PY{l+s+s1}{\PYZsq{}}\PY{l+s+s1}{weekend}\PY{l+s+s1}{\PYZsq{}}\PY{p}{]} \PY{o}{=} \PY{n}{df}\PY{p}{[}\PY{l+s+s1}{\PYZsq{}}\PY{l+s+s1}{day\PYZus{}of\PYZus{}week}\PY{l+s+s1}{\PYZsq{}}\PY{p}{]} \PY{o}{\PYZgt{}}\PY{o}{=} \PY{l+m+mi}{5}
    
    \PY{k}{return} \PY{n}{df}
\end{Verbatim}
\end{tcolorbox}

    \subsubsection*{Part F}\label{part-f}

We use the function defined in the previous part and create the new
columns specified above.

    \begin{tcolorbox}[breakable, size=fbox, boxrule=1pt, pad at break*=1mm,colback=cellbackground, colframe=cellborder]
\prompt{In}{incolor}{9}{\hspace{4pt}}
\begin{Verbatim}[commandchars=\\\{\}]
\PY{n}{df} \PY{o}{=} \PY{n}{weekend\PYZus{}new\PYZus{}column}\PY{p}{(}\PY{n}{df}\PY{p}{)}
\end{Verbatim}
\end{tcolorbox}

    \subsection*{Problem 2}\label{problem-2}

\subsubsection*{Part A}\label{part-a}

We can plot the average total score for each season.

    \begin{tcolorbox}[breakable, size=fbox, boxrule=1pt, pad at break*=1mm,colback=cellbackground, colframe=cellborder]
\prompt{In}{incolor}{10}{\hspace{4pt}}
\begin{Verbatim}[commandchars=\\\{\}]
\PY{n}{plt}\PY{o}{.}\PY{n}{plot}\PY{p}{(}\PY{n}{df}\PY{o}{.}\PY{n}{groupby}\PY{p}{(}\PY{l+s+s1}{\PYZsq{}}\PY{l+s+s1}{season}\PY{l+s+s1}{\PYZsq{}}\PY{p}{)}\PY{o}{.}\PY{n}{mean}\PY{p}{(}\PY{p}{)}\PY{p}{[}\PY{l+s+s1}{\PYZsq{}}\PY{l+s+s1}{total\PYZus{}pt}\PY{l+s+s1}{\PYZsq{}}\PY{p}{]}\PY{p}{)}

\PY{n}{plt}\PY{o}{.}\PY{n}{title}\PY{p}{(}\PY{l+s+s1}{\PYZsq{}}\PY{l+s+s1}{Average Total Points Over Seasons}\PY{l+s+s1}{\PYZsq{}}\PY{p}{)}
\PY{n}{plt}\PY{o}{.}\PY{n}{ylabel}\PY{p}{(}\PY{l+s+s1}{\PYZsq{}}\PY{l+s+s1}{Points}\PY{l+s+s1}{\PYZsq{}}\PY{p}{)}
\PY{n}{\PYZus{}} \PY{o}{=} \PY{n}{plt}\PY{o}{.}\PY{n}{xlabel}\PY{p}{(}\PY{l+s+s1}{\PYZsq{}}\PY{l+s+s1}{Season}\PY{l+s+s1}{\PYZsq{}}\PY{p}{)}
\end{Verbatim}
\end{tcolorbox}

    \begin{center}
    \adjustimage{max size={0.9\linewidth}{0.9\paperheight}}{output_18_0.png}
    \end{center}
    { \hspace*{\fill} \\}
    
    It seems that when this data was first recorded, around the 1950s, there
was significantly fewer total points scored than in the 1960s and
beyond. Once the 1960s arrived, there was a drastic increase in the
number of total points scored. After the 1960s, the total points begins
to generally decrease (particularly in the mid-1970s and mid-1990s)
until the mid 2000s. Now the total points are trending upwards.

    \subsubsection*{Part B}\label{part-b}

For every game, we can plot a histogram of the number of total points
scored.

    \begin{tcolorbox}[breakable, size=fbox, boxrule=1pt, pad at break*=1mm,colback=cellbackground, colframe=cellborder]
\prompt{In}{incolor}{11}{\hspace{4pt}}
\begin{Verbatim}[commandchars=\\\{\}]
\PY{n}{plt}\PY{o}{.}\PY{n}{hist}\PY{p}{(}\PY{n}{df}\PY{p}{[}\PY{l+s+s1}{\PYZsq{}}\PY{l+s+s1}{total\PYZus{}pt}\PY{l+s+s1}{\PYZsq{}}\PY{p}{]}\PY{p}{,} \PY{n}{bins}\PY{o}{=}\PY{l+m+mi}{120}\PY{p}{)}

\PY{n}{plt}\PY{o}{.}\PY{n}{title}\PY{p}{(}\PY{l+s+s1}{\PYZsq{}}\PY{l+s+s1}{Total Points Scored Per Game}\PY{l+s+s1}{\PYZsq{}}\PY{p}{)}
\PY{n}{plt}\PY{o}{.}\PY{n}{ylabel}\PY{p}{(}\PY{l+s+s1}{\PYZsq{}}\PY{l+s+s1}{Frequency}\PY{l+s+s1}{\PYZsq{}}\PY{p}{)}
\PY{n}{\PYZus{}} \PY{o}{=} \PY{n}{plt}\PY{o}{.}\PY{n}{xlabel}\PY{p}{(}\PY{l+s+s1}{\PYZsq{}}\PY{l+s+s1}{Points}\PY{l+s+s1}{\PYZsq{}}\PY{p}{)}
\end{Verbatim}
\end{tcolorbox}

    \begin{center}
    \adjustimage{max size={0.9\linewidth}{0.9\paperheight}}{output_21_0.png}
    \end{center}
    { \hspace*{\fill} \\}
    
    \subsubsection*{Part C}\label{part-c}

There appears to be strong evidence for normality. The distribution is
largely bell-shaped and symmetric. However, there appear to a few bins
that have significantly lower frequencies than their surrounding bins.

    \subsubsection*{Part D}\label{part-d}

We can plot a quantile-quantile plot to test our visual evidence for
normality.

    \begin{tcolorbox}[breakable, size=fbox, boxrule=1pt, pad at break*=1mm,colback=cellbackground, colframe=cellborder]
\prompt{In}{incolor}{12}{\hspace{4pt}}
\begin{Verbatim}[commandchars=\\\{\}]
\PY{n}{\PYZus{}}\PY{p}{,} \PY{n}{ax} \PY{o}{=} \PY{n}{plt}\PY{o}{.}\PY{n}{subplots}\PY{p}{(}\PY{p}{)}

\PY{n}{stats}\PY{o}{.}\PY{n}{probplot}\PY{p}{(}\PY{n}{df}\PY{p}{[}\PY{l+s+s1}{\PYZsq{}}\PY{l+s+s1}{total\PYZus{}pt}\PY{l+s+s1}{\PYZsq{}}\PY{p}{]}\PY{p}{,} \PY{n}{plot}\PY{o}{=}\PY{n}{ax}\PY{p}{)}

\PY{n}{ax}\PY{o}{.}\PY{n}{get\PYZus{}lines}\PY{p}{(}\PY{p}{)}\PY{p}{[}\PY{l+m+mi}{0}\PY{p}{]}\PY{o}{.}\PY{n}{set\PYZus{}color}\PY{p}{(}\PY{l+s+s1}{\PYZsq{}}\PY{l+s+s1}{\PYZsh{}1f77b4}\PY{l+s+s1}{\PYZsq{}}\PY{p}{)}
\PY{n}{ax}\PY{o}{.}\PY{n}{get\PYZus{}lines}\PY{p}{(}\PY{p}{)}\PY{p}{[}\PY{l+m+mi}{1}\PY{p}{]}\PY{o}{.}\PY{n}{set\PYZus{}color}\PY{p}{(}\PY{l+s+s1}{\PYZsq{}}\PY{l+s+s1}{\PYZsh{}ff7f0e}\PY{l+s+s1}{\PYZsq{}}\PY{p}{)}

\PY{n}{ax}\PY{o}{.}\PY{n}{get\PYZus{}lines}\PY{p}{(}\PY{p}{)}\PY{p}{[}\PY{l+m+mi}{0}\PY{p}{]}\PY{o}{.}\PY{n}{set\PYZus{}label}\PY{p}{(}\PY{l+s+s1}{\PYZsq{}}\PY{l+s+s1}{Observed}\PY{l+s+s1}{\PYZsq{}}\PY{p}{)}
\PY{n}{ax}\PY{o}{.}\PY{n}{get\PYZus{}lines}\PY{p}{(}\PY{p}{)}\PY{p}{[}\PY{l+m+mi}{1}\PY{p}{]}\PY{o}{.}\PY{n}{set\PYZus{}label}\PY{p}{(}\PY{l+s+s1}{\PYZsq{}}\PY{l+s+s1}{Normal}\PY{l+s+s1}{\PYZsq{}}\PY{p}{)}

\PY{n}{\PYZus{}} \PY{o}{=} \PY{n}{plt}\PY{o}{.}\PY{n}{legend}\PY{p}{(}\PY{p}{)}
\end{Verbatim}
\end{tcolorbox}

    \begin{center}
    \adjustimage{max size={0.9\linewidth}{0.9\paperheight}}{output_24_0.png}
    \end{center}
    { \hspace*{\fill} \\}
    
    It appears that the distribution closely resembles a normal
distribution. The evidence against is located at the right tail. It
appears the right tail might be a little more heavily-weighted than a
normal distribution.

    \subsection*{Problem 3}\label{problem-3}

In this problem we test whether or not scoring became more conservative
in the NBA during the 1990s.

\subsubsection*{Part A}\label{part-a}

We partition our dataset to just include seasons between \(1970\) and
\(2000\).

    \begin{tcolorbox}[breakable, size=fbox, boxrule=1pt, pad at break*=1mm,colback=cellbackground, colframe=cellborder]
\prompt{In}{incolor}{13}{\hspace{4pt}}
\begin{Verbatim}[commandchars=\\\{\}]
\PY{n}{df\PYZus{}subset} \PY{o}{=} \PY{n}{df}\PY{p}{[}\PY{p}{(}\PY{n}{df}\PY{p}{[}\PY{l+s+s1}{\PYZsq{}}\PY{l+s+s1}{season}\PY{l+s+s1}{\PYZsq{}}\PY{p}{]} \PY{o}{\PYZgt{}}\PY{o}{=} \PY{l+m+mi}{1970}\PY{p}{)} \PY{o}{\PYZam{}} \PY{p}{(}\PY{n}{df}\PY{p}{[}\PY{l+s+s1}{\PYZsq{}}\PY{l+s+s1}{season}\PY{l+s+s1}{\PYZsq{}}\PY{p}{]} \PY{o}{\PYZlt{}} \PY{l+m+mi}{2000}\PY{p}{)}\PY{p}{]}
\end{Verbatim}
\end{tcolorbox}

    \subsubsection*{Part B}\label{part-b}

Now we split the data subset into groups before and after \(1990\).

    \begin{tcolorbox}[breakable, size=fbox, boxrule=1pt, pad at break*=1mm,colback=cellbackground, colframe=cellborder]
\prompt{In}{incolor}{14}{\hspace{4pt}}
\begin{Verbatim}[commandchars=\\\{\}]
\PY{n}{df\PYZus{}subset\PYZus{}before} \PY{o}{=} \PY{n}{df\PYZus{}subset}\PY{p}{[}\PY{n}{df\PYZus{}subset}\PY{p}{[}\PY{l+s+s1}{\PYZsq{}}\PY{l+s+s1}{season}\PY{l+s+s1}{\PYZsq{}}\PY{p}{]} \PY{o}{\PYZlt{}} \PY{l+m+mi}{1990}\PY{p}{]}
\PY{n}{df\PYZus{}subset\PYZus{}after} \PY{o}{=} \PY{n}{df\PYZus{}subset}\PY{p}{[}\PY{n}{df\PYZus{}subset}\PY{p}{[}\PY{l+s+s1}{\PYZsq{}}\PY{l+s+s1}{season}\PY{l+s+s1}{\PYZsq{}}\PY{p}{]} \PY{o}{\PYZgt{}}\PY{o}{=} \PY{l+m+mi}{1990}\PY{p}{]}
\end{Verbatim}
\end{tcolorbox}

    \subsubsection*{Part C}\label{part-c}

The null hypothesis we wish to test is that average total points scored
in games between the two periods is equivalent. The alternative
hypothesis is that the two means are not equal. Mathematically,
\[ H_0 : \mu_{\mathrm{before}} = \mu_{\mathrm{after}} \]
\[ H_1 : \mu_{\mathrm{before}} \neq \mu_{\mathrm{after}} \]

    \subsubsection*{Part D}\label{part-d}

We can evaluate the mean total scores of each period.

    \begin{tcolorbox}[breakable, size=fbox, boxrule=1pt, pad at break*=1mm,colback=cellbackground, colframe=cellborder]
\prompt{In}{incolor}{15}{\hspace{4pt}}
\begin{Verbatim}[commandchars=\\\{\}]
\PY{n}{hat\PYZus{}mu\PYZus{}before} \PY{o}{=} \PY{n}{df\PYZus{}subset\PYZus{}before}\PY{p}{[}\PY{l+s+s1}{\PYZsq{}}\PY{l+s+s1}{total\PYZus{}pt}\PY{l+s+s1}{\PYZsq{}}\PY{p}{]}\PY{o}{.}\PY{n}{mean}\PY{p}{(}\PY{p}{)}
\PY{n}{hat\PYZus{}mu\PYZus{}after} \PY{o}{=} \PY{n}{df\PYZus{}subset\PYZus{}after}\PY{p}{[}\PY{l+s+s1}{\PYZsq{}}\PY{l+s+s1}{total\PYZus{}pt}\PY{l+s+s1}{\PYZsq{}}\PY{p}{]}\PY{o}{.}\PY{n}{mean}\PY{p}{(}\PY{p}{)}

\PY{n+nb}{print}\PY{p}{(}\PY{n}{f}\PY{l+s+s2}{\PYZdq{}}\PY{l+s+s2}{Mean Before: }\PY{l+s+si}{\PYZob{}hat\PYZus{}mu\PYZus{}before\PYZcb{}}\PY{l+s+s2}{\PYZdq{}}\PY{p}{)}
\PY{n+nb}{print}\PY{p}{(}\PY{n}{f}\PY{l+s+s2}{\PYZdq{}}\PY{l+s+s2}{Mean After: }\PY{l+s+si}{\PYZob{}hat\PYZus{}mu\PYZus{}after\PYZcb{}}\PY{l+s+s2}{\PYZdq{}}\PY{p}{)}
\end{Verbatim}
\end{tcolorbox}

    \begin{Verbatim}[commandchars=\\\{\}]
Mean Before: 217.3702106485321
Mean After: 202.07596566523605
\end{Verbatim}

    \subsubsection*{Part E}\label{part-e}

We run a two-sample \(t\)-test for the two means without assuming equal
variance.

    \begin{tcolorbox}[breakable, size=fbox, boxrule=1pt, pad at break*=1mm,colback=cellbackground, colframe=cellborder]
\prompt{In}{incolor}{16}{\hspace{4pt}}
\begin{Verbatim}[commandchars=\\\{\}]
\PY{n}{results} \PY{o}{=} \PY{n}{stats}\PY{o}{.}\PY{n}{ttest\PYZus{}ind}\PY{p}{(}\PY{n}{df\PYZus{}subset\PYZus{}before}\PY{p}{[}\PY{l+s+s1}{\PYZsq{}}\PY{l+s+s1}{total\PYZus{}pt}\PY{l+s+s1}{\PYZsq{}}\PY{p}{]}\PY{p}{,}
                          \PY{n}{df\PYZus{}subset\PYZus{}after}\PY{p}{[}\PY{l+s+s1}{\PYZsq{}}\PY{l+s+s1}{total\PYZus{}pt}\PY{l+s+s1}{\PYZsq{}}\PY{p}{]}\PY{p}{,} 
                          \PY{n}{equal\PYZus{}var}\PY{o}{=}\PY{k+kc}{False}\PY{p}{)}

\PY{n+nb}{print}\PY{p}{(}\PY{n}{f}\PY{l+s+s2}{\PYZdq{}}\PY{l+s+s2}{t: }\PY{l+s+si}{\PYZob{}results.statistic\PYZcb{}}\PY{l+s+s2}{\PYZdq{}}\PY{p}{)}
\PY{n+nb}{print}\PY{p}{(}\PY{n}{f}\PY{l+s+s2}{\PYZdq{}}\PY{l+s+s2}{p\PYZhy{}value: }\PY{l+s+si}{\PYZob{}results.pvalue\PYZcb{}}\PY{l+s+s2}{\PYZdq{}}\PY{p}{)}
\end{Verbatim}
\end{tcolorbox}

    \begin{Verbatim}[commandchars=\\\{\}]
t: 58.56557673842027
p-value: 0.0
\end{Verbatim}

    \subsubsection*{Part F}\label{part-f}

From the last part, we see that the \(p\)-value is zero and the
\(t\)-statistic is quite high. Therefore, we can reject the null that
the means are equivalent with low \(\alpha\)-level (\(\alpha\) less than
\(0.05\)).

    \subsection*{Problem 4}\label{problem-4}

In this problem we test whether there is a home-court advantage.

\subsubsection*{Part A}\label{part-a}

First we create a subset of data only including years in the inclusive
range \([2000, 2017]\).

    \begin{tcolorbox}[breakable, size=fbox, boxrule=1pt, pad at break*=1mm,colback=cellbackground, colframe=cellborder]
\prompt{In}{incolor}{17}{\hspace{4pt}}
\begin{Verbatim}[commandchars=\\\{\}]
\PY{n}{df\PYZus{}subset} \PY{o}{=} \PY{n}{df}\PY{p}{[}\PY{p}{(}\PY{n}{df}\PY{p}{[}\PY{l+s+s1}{\PYZsq{}}\PY{l+s+s1}{season}\PY{l+s+s1}{\PYZsq{}}\PY{p}{]} \PY{o}{\PYZgt{}}\PY{o}{=} \PY{l+m+mi}{2000}\PY{p}{)} \PY{o}{\PYZam{}} \PY{p}{(}\PY{n}{df}\PY{p}{[}\PY{l+s+s1}{\PYZsq{}}\PY{l+s+s1}{season}\PY{l+s+s1}{\PYZsq{}}\PY{p}{]} \PY{o}{\PYZlt{}}\PY{o}{=} \PY{l+m+mi}{2017}\PY{p}{)}\PY{p}{]}
\end{Verbatim}
\end{tcolorbox}

    \subsubsection*{Part B}\label{part-b}

We can check the mean difference between the points scored by the home
team and the away team in this subset.

    \begin{tcolorbox}[breakable, size=fbox, boxrule=1pt, pad at break*=1mm,colback=cellbackground, colframe=cellborder]
\prompt{In}{incolor}{18}{\hspace{4pt}}
\begin{Verbatim}[commandchars=\\\{\}]
\PY{n}{df\PYZus{}subset}\PY{p}{[}\PY{l+s+s1}{\PYZsq{}}\PY{l+s+s1}{pt\PYZus{}diff}\PY{l+s+s1}{\PYZsq{}}\PY{p}{]}\PY{o}{.}\PY{n}{mean}\PY{p}{(}\PY{p}{)}
\end{Verbatim}
\end{tcolorbox}

            \begin{tcolorbox}[breakable, boxrule=.5pt, size=fbox, pad at break*=1mm, opacityfill=0]
\prompt{Out}{outcolor}{18}{\hspace{3.5pt}}
\begin{Verbatim}[commandchars=\\\{\}]
3.204781839368176
\end{Verbatim}
\end{tcolorbox}
        
    \subsubsection*{Part C}\label{part-c}

The null hypothesis we wish to test is that average difference of points
scored by the home and away team is zero. The alternative hypothesis is
that the home team has an advantage, therefore the difference should be
positive. Mathematically,
\[ H_0 : \mu_{\mathrm{home}} - \mu_{\mathrm{away}} = 0 \]
\[ H_1 : \mu_{\mathrm{home}} - \mu_{\mathrm{away}} > 0 \]

    \subsubsection*{Part D}\label{part-d}

We run a one-sample \(t\)-test for the mean difference. Note we are
doing a one-sided \(t\)-test so we divide the \(p\)-value by \(2\).

    \begin{tcolorbox}[breakable, size=fbox, boxrule=1pt, pad at break*=1mm,colback=cellbackground, colframe=cellborder]
\prompt{In}{incolor}{19}{\hspace{4pt}}
\begin{Verbatim}[commandchars=\\\{\}]
\PY{n}{results} \PY{o}{=} \PY{n}{stats}\PY{o}{.}\PY{n}{ttest\PYZus{}1samp}\PY{p}{(}\PY{n}{df\PYZus{}subset}\PY{p}{[}\PY{l+s+s1}{\PYZsq{}}\PY{l+s+s1}{pt\PYZus{}diff}\PY{l+s+s1}{\PYZsq{}}\PY{p}{]}\PY{p}{,} \PY{l+m+mi}{0}\PY{p}{)}

\PY{n+nb}{print}\PY{p}{(}\PY{n}{f}\PY{l+s+s2}{\PYZdq{}}\PY{l+s+s2}{t: }\PY{l+s+si}{\PYZob{}results.statistic\PYZcb{}}\PY{l+s+s2}{\PYZdq{}}\PY{p}{)}
\PY{n+nb}{print}\PY{p}{(}\PY{n}{f}\PY{l+s+s2}{\PYZdq{}}\PY{l+s+s2}{p\PYZhy{}value: }\PY{l+s+s2}{\PYZob{}}\PY{l+s+s2}{results.pvalue / 2\PYZcb{}}\PY{l+s+s2}{\PYZdq{}}\PY{p}{)}
\end{Verbatim}
\end{tcolorbox}

    \begin{Verbatim}[commandchars=\\\{\}]
t: 37.492696922047486
p-value: 4.968793905787681e-299
\end{Verbatim}

    \subsubsection*{Part E}\label{part-e}

From the last part, we see that the \(p\)-value is nearly zero and the
\(t\)-statistic is high. Therefore, we can reject the null that the
difference is zero with low \(\alpha\)-level (\(\alpha\) less than
\(0.05\)).

    \subsection*{Problem 5}\label{problem-5}

In this problem we test if there is a difference between scoring on
weekday vs weekend games.

\subsubsection*{Part A}\label{part-a}

We again restrict the data to just include seasons \([2000, 2017]\).

    \begin{tcolorbox}[breakable, size=fbox, boxrule=1pt, pad at break*=1mm,colback=cellbackground, colframe=cellborder]
\prompt{In}{incolor}{20}{\hspace{4pt}}
\begin{Verbatim}[commandchars=\\\{\}]
\PY{n}{df\PYZus{}subset} \PY{o}{=} \PY{n}{df}\PY{p}{[}\PY{p}{(}\PY{n}{df}\PY{p}{[}\PY{l+s+s1}{\PYZsq{}}\PY{l+s+s1}{season}\PY{l+s+s1}{\PYZsq{}}\PY{p}{]} \PY{o}{\PYZgt{}}\PY{o}{=} \PY{l+m+mi}{2000}\PY{p}{)} \PY{o}{\PYZam{}} \PY{p}{(}\PY{n}{df}\PY{p}{[}\PY{l+s+s1}{\PYZsq{}}\PY{l+s+s1}{season}\PY{l+s+s1}{\PYZsq{}}\PY{p}{]} \PY{o}{\PYZlt{}}\PY{o}{=} \PY{l+m+mi}{2017}\PY{p}{)}\PY{p}{]}
\end{Verbatim}
\end{tcolorbox}

    \subsubsection*{Part B}\label{part-b}

We can compute two histograms for weekend and weekday game total scores.

    \begin{tcolorbox}[breakable, size=fbox, boxrule=1pt, pad at break*=1mm,colback=cellbackground, colframe=cellborder]
\prompt{In}{incolor}{21}{\hspace{4pt}}
\begin{Verbatim}[commandchars=\\\{\}]
\PY{n}{df\PYZus{}subset\PYZus{}weekday} \PY{o}{=} \PY{n}{df\PYZus{}subset}\PY{p}{[}\PY{n}{df\PYZus{}subset}\PY{p}{[}\PY{l+s+s1}{\PYZsq{}}\PY{l+s+s1}{weekend}\PY{l+s+s1}{\PYZsq{}}\PY{p}{]} \PY{o}{==} \PY{k+kc}{False}\PY{p}{]}
\PY{n}{df\PYZus{}subset\PYZus{}weekend} \PY{o}{=} \PY{n}{df\PYZus{}subset}\PY{p}{[}\PY{n}{df\PYZus{}subset}\PY{p}{[}\PY{l+s+s1}{\PYZsq{}}\PY{l+s+s1}{weekend}\PY{l+s+s1}{\PYZsq{}}\PY{p}{]} \PY{o}{==} \PY{k+kc}{True}\PY{p}{]}
\end{Verbatim}
\end{tcolorbox}

    \begin{tcolorbox}[breakable, size=fbox, boxrule=1pt, pad at break*=1mm,colback=cellbackground, colframe=cellborder]
\prompt{In}{incolor}{22}{\hspace{4pt}}
\begin{Verbatim}[commandchars=\\\{\}]
\PY{n}{plt}\PY{o}{.}\PY{n}{hist}\PY{p}{(}\PY{n}{df\PYZus{}subset\PYZus{}weekday}\PY{p}{[}\PY{l+s+s1}{\PYZsq{}}\PY{l+s+s1}{total\PYZus{}pt}\PY{l+s+s1}{\PYZsq{}}\PY{p}{]}\PY{p}{,} \PY{n}{bins}\PY{o}{=}\PY{l+m+mi}{100}\PY{p}{,} \PY{n}{label}\PY{o}{=}\PY{l+s+s1}{\PYZsq{}}\PY{l+s+s1}{Weekday}\PY{l+s+s1}{\PYZsq{}}\PY{p}{)}
\PY{n}{plt}\PY{o}{.}\PY{n}{hist}\PY{p}{(}\PY{n}{df\PYZus{}subset\PYZus{}weekend}\PY{p}{[}\PY{l+s+s1}{\PYZsq{}}\PY{l+s+s1}{total\PYZus{}pt}\PY{l+s+s1}{\PYZsq{}}\PY{p}{]}\PY{p}{,} \PY{n}{bins}\PY{o}{=}\PY{l+m+mi}{100}\PY{p}{,} \PY{n}{label}\PY{o}{=}\PY{l+s+s1}{\PYZsq{}}\PY{l+s+s1}{Weekend}\PY{l+s+s1}{\PYZsq{}}\PY{p}{)}

\PY{n}{plt}\PY{o}{.}\PY{n}{legend}\PY{p}{(}\PY{p}{)}
\PY{n}{plt}\PY{o}{.}\PY{n}{title}\PY{p}{(}\PY{l+s+s1}{\PYZsq{}}\PY{l+s+s1}{Weekday vs Weekend Total Points}\PY{l+s+s1}{\PYZsq{}}\PY{p}{)}
\PY{n}{plt}\PY{o}{.}\PY{n}{ylabel}\PY{p}{(}\PY{l+s+s1}{\PYZsq{}}\PY{l+s+s1}{Frequency}\PY{l+s+s1}{\PYZsq{}}\PY{p}{)}
\PY{n}{\PYZus{}} \PY{o}{=} \PY{n}{plt}\PY{o}{.}\PY{n}{xlabel}\PY{p}{(}\PY{l+s+s1}{\PYZsq{}}\PY{l+s+s1}{Points}\PY{l+s+s1}{\PYZsq{}}\PY{p}{)}
\end{Verbatim}
\end{tcolorbox}

    \begin{center}
    \adjustimage{max size={0.9\linewidth}{0.9\paperheight}}{output_48_0.png}
    \end{center}
    { \hspace*{\fill} \\}
    
    \subsubsection*{Part C}\label{part-c}

We can compute the average total points for weekday and weekend games.

    \begin{tcolorbox}[breakable, size=fbox, boxrule=1pt, pad at break*=1mm,colback=cellbackground, colframe=cellborder]
\prompt{In}{incolor}{23}{\hspace{4pt}}
\begin{Verbatim}[commandchars=\\\{\}]
\PY{n}{hat\PYZus{}mu\PYZus{}weekday} \PY{o}{=} \PY{n}{df\PYZus{}subset\PYZus{}weekday}\PY{p}{[}\PY{l+s+s1}{\PYZsq{}}\PY{l+s+s1}{total\PYZus{}pt}\PY{l+s+s1}{\PYZsq{}}\PY{p}{]}\PY{o}{.}\PY{n}{mean}\PY{p}{(}\PY{p}{)}
\PY{n}{hat\PYZus{}mu\PYZus{}weekend} \PY{o}{=} \PY{n}{df\PYZus{}subset\PYZus{}weekend}\PY{p}{[}\PY{l+s+s1}{\PYZsq{}}\PY{l+s+s1}{total\PYZus{}pt}\PY{l+s+s1}{\PYZsq{}}\PY{p}{]}\PY{o}{.}\PY{n}{mean}\PY{p}{(}\PY{p}{)}

\PY{n+nb}{print}\PY{p}{(}\PY{n}{f}\PY{l+s+s2}{\PYZdq{}}\PY{l+s+s2}{Mean Weekday: }\PY{l+s+si}{\PYZob{}hat\PYZus{}mu\PYZus{}weekday\PYZcb{}}\PY{l+s+s2}{\PYZdq{}}\PY{p}{)}
\PY{n+nb}{print}\PY{p}{(}\PY{n}{f}\PY{l+s+s2}{\PYZdq{}}\PY{l+s+s2}{Mean Weekend: }\PY{l+s+si}{\PYZob{}hat\PYZus{}mu\PYZus{}after\PYZcb{}}\PY{l+s+s2}{\PYZdq{}}\PY{p}{)}
\end{Verbatim}
\end{tcolorbox}

    \begin{Verbatim}[commandchars=\\\{\}]
Mean Weekday: 196.9075830797685
Mean Weekend: 202.07596566523605
\end{Verbatim}

    \subsubsection*{Part D}\label{part-d}

The null hypothesis we wish to test is that average total points scored
in weekday games is equal in weekend games. The alternative hypothesis
is that the two means are not equal. Mathematically,
\[ H_0 : \mu_{\mathrm{weekday}} = \mu_{\mathrm{weekend}} \]
\[ H_1 : \mu_{\mathrm{weekday}} \neq \mu_{\mathrm{weekend}} \]

    \subsubsection*{Part E}\label{part-e}

We run a two-sample \(t\)-test for the two means without assuming equal
variance.

    \begin{tcolorbox}[breakable, size=fbox, boxrule=1pt, pad at break*=1mm,colback=cellbackground, colframe=cellborder]
\prompt{In}{incolor}{24}{\hspace{4pt}}
\begin{Verbatim}[commandchars=\\\{\}]
\PY{n}{results} \PY{o}{=} \PY{n}{stats}\PY{o}{.}\PY{n}{ttest\PYZus{}ind}\PY{p}{(}\PY{n}{df\PYZus{}subset\PYZus{}weekday}\PY{p}{[}\PY{l+s+s1}{\PYZsq{}}\PY{l+s+s1}{total\PYZus{}pt}\PY{l+s+s1}{\PYZsq{}}\PY{p}{]}\PY{p}{,}
                          \PY{n}{df\PYZus{}subset\PYZus{}weekend}\PY{p}{[}\PY{l+s+s1}{\PYZsq{}}\PY{l+s+s1}{total\PYZus{}pt}\PY{l+s+s1}{\PYZsq{}}\PY{p}{]}\PY{p}{,} 
                          \PY{n}{equal\PYZus{}var}\PY{o}{=}\PY{k+kc}{False}\PY{p}{)}

\PY{n+nb}{print}\PY{p}{(}\PY{n}{f}\PY{l+s+s2}{\PYZdq{}}\PY{l+s+s2}{t: }\PY{l+s+si}{\PYZob{}results.statistic\PYZcb{}}\PY{l+s+s2}{\PYZdq{}}\PY{p}{)}
\PY{n+nb}{print}\PY{p}{(}\PY{n}{f}\PY{l+s+s2}{\PYZdq{}}\PY{l+s+s2}{p\PYZhy{}value: }\PY{l+s+si}{\PYZob{}results.pvalue\PYZcb{}}\PY{l+s+s2}{\PYZdq{}}\PY{p}{)}
\end{Verbatim}
\end{tcolorbox}

    \begin{Verbatim}[commandchars=\\\{\}]
t: 1.5318301686986437
p-value: 0.12559161393965187
\end{Verbatim}

    \subsubsection*{Part F}\label{part-f}

From the last part, we see that the \(p\)-value is roughly 13\% and the
\(t\)-statistic is relatively low. Therefore, we cannot reject the null
that the means are equal with low \(\alpha\)-level (\(\alpha\) less than
\(0.13\)).

    \subsection*{Problem 6}\label{problem-6}

In this problem, we consider if teams are more conservative (lower
scoring) in the playoffs than they are in the regular season.

\subsubsection*{Part A}\label{part-a}

We produce two datasets, one for games between April and June for
playoffs and one for games between September and February for regular
season.

    \begin{tcolorbox}[breakable, size=fbox, boxrule=1pt, pad at break*=1mm,colback=cellbackground, colframe=cellborder]
\prompt{In}{incolor}{25}{\hspace{4pt}}
\begin{Verbatim}[commandchars=\\\{\}]
\PY{n}{df\PYZus{}subset\PYZus{}playoffs} \PY{o}{=} \PY{n}{df}\PY{p}{[}\PY{n}{df}\PY{p}{[}\PY{l+s+s1}{\PYZsq{}}\PY{l+s+s1}{date}\PY{l+s+s1}{\PYZsq{}}\PY{p}{]}\PY{o}{.}\PY{n}{apply}\PY{p}{(}\PY{k}{lambda} \PY{n}{d}\PY{p}{:} \PY{p}{(}\PY{n}{d}\PY{o}{.}\PY{n}{month} \PY{o}{\PYZgt{}}\PY{o}{=} \PY{l+m+mi}{4}\PY{p}{)} 
                                         \PY{o}{\PYZam{}} \PY{p}{(}\PY{n}{d}\PY{o}{.}\PY{n}{month} \PY{o}{\PYZlt{}}\PY{o}{=} \PY{l+m+mi}{6}\PY{p}{)}\PY{p}{)}\PY{p}{]}
\PY{n}{df\PYZus{}subset\PYZus{}regular} \PY{o}{=} \PY{n}{df}\PY{p}{[}\PY{n}{df}\PY{p}{[}\PY{l+s+s1}{\PYZsq{}}\PY{l+s+s1}{date}\PY{l+s+s1}{\PYZsq{}}\PY{p}{]}\PY{o}{.}\PY{n}{apply}\PY{p}{(}\PY{k}{lambda} \PY{n}{d}\PY{p}{:} \PY{p}{(}\PY{n}{d}\PY{o}{.}\PY{n}{month} \PY{o}{\PYZgt{}}\PY{o}{=} \PY{l+m+mi}{9}\PY{p}{)} 
                                        \PY{o}{|} \PY{p}{(}\PY{n}{d}\PY{o}{.}\PY{n}{month} \PY{o}{\PYZlt{}}\PY{o}{=} \PY{l+m+mi}{2}\PY{p}{)}\PY{p}{)}\PY{p}{]}
\end{Verbatim}
\end{tcolorbox}

    \subsubsection*{Part B}\label{part-b}

We can check the number of playoff vs regular season games.

    \begin{tcolorbox}[breakable, size=fbox, boxrule=1pt, pad at break*=1mm,colback=cellbackground, colframe=cellborder]
\prompt{In}{incolor}{26}{\hspace{4pt}}
\begin{Verbatim}[commandchars=\\\{\}]
\PY{n+nb}{print}\PY{p}{(}\PY{n}{f}\PY{l+s+s2}{\PYZdq{}}\PY{l+s+s2}{Playoff Games: }\PY{l+s+s2}{\PYZob{}}\PY{l+s+s2}{len(df\PYZus{}subset\PYZus{}playoffs)\PYZcb{}}\PY{l+s+s2}{\PYZdq{}}\PY{p}{)}
\PY{n+nb}{print}\PY{p}{(}\PY{n}{f}\PY{l+s+s2}{\PYZdq{}}\PY{l+s+s2}{Regular Games: }\PY{l+s+s2}{\PYZob{}}\PY{l+s+s2}{len(df\PYZus{}subset\PYZus{}regular)\PYZcb{}}\PY{l+s+s2}{\PYZdq{}}\PY{p}{)}
\end{Verbatim}
\end{tcolorbox}

    \begin{Verbatim}[commandchars=\\\{\}]
Playoff Games: 8595
Regular Games: 41352
\end{Verbatim}

    \subsubsection*{Part C}\label{part-c}

We can calculate the mean total points between the subsets.

    \begin{tcolorbox}[breakable, size=fbox, boxrule=1pt, pad at break*=1mm,colback=cellbackground, colframe=cellborder]
\prompt{In}{incolor}{27}{\hspace{4pt}}
\begin{Verbatim}[commandchars=\\\{\}]
\PY{n}{hat\PYZus{}mu\PYZus{}playoffs} \PY{o}{=} \PY{n}{df\PYZus{}subset\PYZus{}playoffs}\PY{p}{[}\PY{l+s+s1}{\PYZsq{}}\PY{l+s+s1}{total\PYZus{}pt}\PY{l+s+s1}{\PYZsq{}}\PY{p}{]}\PY{o}{.}\PY{n}{mean}\PY{p}{(}\PY{p}{)}
\PY{n}{hat\PYZus{}mu\PYZus{}regular} \PY{o}{=} \PY{n}{df\PYZus{}subset\PYZus{}regular}\PY{p}{[}\PY{l+s+s1}{\PYZsq{}}\PY{l+s+s1}{total\PYZus{}pt}\PY{l+s+s1}{\PYZsq{}}\PY{p}{]}\PY{o}{.}\PY{n}{mean}\PY{p}{(}\PY{p}{)}

\PY{n+nb}{print}\PY{p}{(}\PY{n}{f}\PY{l+s+s2}{\PYZdq{}}\PY{l+s+s2}{Mean Playoffs: }\PY{l+s+si}{\PYZob{}hat\PYZus{}mu\PYZus{}playoffs\PYZcb{}}\PY{l+s+s2}{\PYZdq{}}\PY{p}{)}
\PY{n+nb}{print}\PY{p}{(}\PY{n}{f}\PY{l+s+s2}{\PYZdq{}}\PY{l+s+s2}{Mean Regular: }\PY{l+s+si}{\PYZob{}hat\PYZus{}mu\PYZus{}regular\PYZcb{}}\PY{l+s+s2}{\PYZdq{}}\PY{p}{)}
\end{Verbatim}
\end{tcolorbox}

    \begin{Verbatim}[commandchars=\\\{\}]
Mean Playoffs: 202.49680046538685
Mean Regular: 205.5076175275682
\end{Verbatim}

    \subsubsection*{Part D}\label{part-d}

We can plot the total scores throughout the seasons for both playoff and
regular season.

    \begin{tcolorbox}[breakable, size=fbox, boxrule=1pt, pad at break*=1mm,colback=cellbackground, colframe=cellborder]
\prompt{In}{incolor}{28}{\hspace{4pt}}
\begin{Verbatim}[commandchars=\\\{\}]
\PY{n}{plt}\PY{o}{.}\PY{n}{plot}\PY{p}{(}\PY{n}{df\PYZus{}subset\PYZus{}playoffs}\PY{o}{.}\PY{n}{groupby}\PY{p}{(}\PY{l+s+s1}{\PYZsq{}}\PY{l+s+s1}{season}\PY{l+s+s1}{\PYZsq{}}\PY{p}{)}\PY{o}{.}\PY{n}{mean}\PY{p}{(}\PY{p}{)}\PY{p}{[}\PY{l+s+s1}{\PYZsq{}}\PY{l+s+s1}{total\PYZus{}pt}\PY{l+s+s1}{\PYZsq{}}\PY{p}{]}\PY{p}{,} 
         \PY{n}{label}\PY{o}{=}\PY{l+s+s1}{\PYZsq{}}\PY{l+s+s1}{Playoffs}\PY{l+s+s1}{\PYZsq{}}\PY{p}{)}
\PY{n}{plt}\PY{o}{.}\PY{n}{plot}\PY{p}{(}\PY{n}{df\PYZus{}subset\PYZus{}regular}\PY{o}{.}\PY{n}{groupby}\PY{p}{(}\PY{l+s+s1}{\PYZsq{}}\PY{l+s+s1}{season}\PY{l+s+s1}{\PYZsq{}}\PY{p}{)}\PY{o}{.}\PY{n}{mean}\PY{p}{(}\PY{p}{)}\PY{p}{[}\PY{l+s+s1}{\PYZsq{}}\PY{l+s+s1}{total\PYZus{}pt}\PY{l+s+s1}{\PYZsq{}}\PY{p}{]}\PY{p}{,} 
         \PY{n}{label}\PY{o}{=}\PY{l+s+s1}{\PYZsq{}}\PY{l+s+s1}{Regular}\PY{l+s+s1}{\PYZsq{}}\PY{p}{)}

\PY{n}{plt}\PY{o}{.}\PY{n}{legend}\PY{p}{(}\PY{p}{)}
\PY{n}{plt}\PY{o}{.}\PY{n}{title}\PY{p}{(}\PY{l+s+s1}{\PYZsq{}}\PY{l+s+s1}{Total Points in Playoffs and Regular Seasons}\PY{l+s+s1}{\PYZsq{}}\PY{p}{)}
\PY{n}{plt}\PY{o}{.}\PY{n}{ylabel}\PY{p}{(}\PY{l+s+s1}{\PYZsq{}}\PY{l+s+s1}{Points}\PY{l+s+s1}{\PYZsq{}}\PY{p}{)}
\PY{n}{\PYZus{}} \PY{o}{=} \PY{n}{plt}\PY{o}{.}\PY{n}{xlabel}\PY{p}{(}\PY{l+s+s1}{\PYZsq{}}\PY{l+s+s1}{Season}\PY{l+s+s1}{\PYZsq{}}\PY{p}{)}
\end{Verbatim}
\end{tcolorbox}

    \begin{center}
    \adjustimage{max size={0.9\linewidth}{0.9\paperheight}}{output_62_0.png}
    \end{center}
    { \hspace*{\fill} \\}
    
    It seems that, for the most part, the total points in playoffs and
regular season are closely related. However, in the 1960s and 1970s (and
a little bit of the 1980s), it seems that the average scores in the
playoffs was often lower than the average scores in the regular season.

    \subsubsection*{Part E}\label{part-e}

The null hypothesis we wish to test is that average total points scored
in playoff games is equal in regular games. The alternative hypothesis
is that the two means are not equal. Mathematically,
\[ H_0 : \mu_{\mathrm{playoffs}} = \mu_{\mathrm{regular}} \]
\[ H_1 : \mu_{\mathrm{playoffs}} < \mu_{\mathrm{regular}} \]

    \subsubsection*{Part F}\label{part-f}

Given the differences over the complete time period, we consider two
time periods independently: \([1960, 1980]\) and \([2000, 2017]\). We
first consider the \([1960, 1980]\) time period.

    \begin{tcolorbox}[breakable, size=fbox, boxrule=1pt, pad at break*=1mm,colback=cellbackground, colframe=cellborder]
\prompt{In}{incolor}{29}{\hspace{4pt}}
\begin{Verbatim}[commandchars=\\\{\}]
\PY{n}{df\PYZus{}subset\PYZus{}playoffs\PYZus{}19} \PY{o}{=} \PY{n}{df\PYZus{}subset\PYZus{}playoffs}\PY{p}{[}\PY{p}{(}\PY{n}{df\PYZus{}subset\PYZus{}playoffs}\PY{p}{[}\PY{l+s+s1}{\PYZsq{}}\PY{l+s+s1}{season}\PY{l+s+s1}{\PYZsq{}}\PY{p}{]} \PY{o}{\PYZgt{}}\PY{o}{=} \PY{l+m+mi}{1960}\PY{p}{)} 
                                        \PY{o}{\PYZam{}} \PY{p}{(}\PY{n}{df\PYZus{}subset\PYZus{}playoffs}\PY{p}{[}\PY{l+s+s1}{\PYZsq{}}\PY{l+s+s1}{season}\PY{l+s+s1}{\PYZsq{}}\PY{p}{]} \PY{o}{\PYZlt{}}\PY{o}{=} \PY{l+m+mi}{1980}\PY{p}{)}\PY{p}{]}
\PY{n}{df\PYZus{}subset\PYZus{}regular\PYZus{}19} \PY{o}{=} \PY{n}{df\PYZus{}subset\PYZus{}regular}\PY{p}{[}\PY{p}{(}\PY{n}{df\PYZus{}subset\PYZus{}regular}\PY{p}{[}\PY{l+s+s1}{\PYZsq{}}\PY{l+s+s1}{season}\PY{l+s+s1}{\PYZsq{}}\PY{p}{]} \PY{o}{\PYZgt{}}\PY{o}{=} \PY{l+m+mi}{1960}\PY{p}{)} 
                                         \PY{o}{\PYZam{}} \PY{p}{(}\PY{n}{df\PYZus{}subset\PYZus{}regular}\PY{p}{[}\PY{l+s+s1}{\PYZsq{}}\PY{l+s+s1}{season}\PY{l+s+s1}{\PYZsq{}}\PY{p}{]} \PY{o}{\PYZlt{}}\PY{o}{=} \PY{l+m+mi}{1980}\PY{p}{)}\PY{p}{]}
\end{Verbatim}
\end{tcolorbox}

    \subsubsection*{Part G}\label{part-g}

We can compare the means of the total points scored between the playoffs
and regular season for the subset.

    \begin{tcolorbox}[breakable, size=fbox, boxrule=1pt, pad at break*=1mm,colback=cellbackground, colframe=cellborder]
\prompt{In}{incolor}{30}{\hspace{4pt}}
\begin{Verbatim}[commandchars=\\\{\}]
\PY{n}{hat\PYZus{}mu\PYZus{}playoffs\PYZus{}19} \PY{o}{=} \PY{n}{df\PYZus{}subset\PYZus{}playoffs\PYZus{}19}\PY{p}{[}\PY{l+s+s1}{\PYZsq{}}\PY{l+s+s1}{total\PYZus{}pt}\PY{l+s+s1}{\PYZsq{}}\PY{p}{]}\PY{o}{.}\PY{n}{mean}\PY{p}{(}\PY{p}{)}
\PY{n}{hat\PYZus{}mu\PYZus{}regular\PYZus{}19} \PY{o}{=} \PY{n}{df\PYZus{}subset\PYZus{}regular\PYZus{}19}\PY{p}{[}\PY{l+s+s1}{\PYZsq{}}\PY{l+s+s1}{total\PYZus{}pt}\PY{l+s+s1}{\PYZsq{}}\PY{p}{]}\PY{o}{.}\PY{n}{mean}\PY{p}{(}\PY{p}{)}

\PY{n+nb}{print}\PY{p}{(}\PY{n}{f}\PY{l+s+s2}{\PYZdq{}}\PY{l+s+s2}{Mean Playoffs: }\PY{l+s+si}{\PYZob{}hat\PYZus{}mu\PYZus{}playoffs\PYZus{}19\PYZcb{}}\PY{l+s+s2}{\PYZdq{}}\PY{p}{)}
\PY{n+nb}{print}\PY{p}{(}\PY{n}{f}\PY{l+s+s2}{\PYZdq{}}\PY{l+s+s2}{Mean Regular: }\PY{l+s+si}{\PYZob{}hat\PYZus{}mu\PYZus{}regular\PYZus{}19\PYZcb{}}\PY{l+s+s2}{\PYZdq{}}\PY{p}{)}
\end{Verbatim}
\end{tcolorbox}

    \begin{Verbatim}[commandchars=\\\{\}]
Mean Playoffs: 213.3367935409458
Mean Regular: 220.84210004952948
\end{Verbatim}

    \subsubsection*{Part H}\label{part-h}

We run a two-sample \(t\)-test for the two means without assuming equal
variance.

    \begin{tcolorbox}[breakable, size=fbox, boxrule=1pt, pad at break*=1mm,colback=cellbackground, colframe=cellborder]
\prompt{In}{incolor}{31}{\hspace{4pt}}
\begin{Verbatim}[commandchars=\\\{\}]
\PY{n}{results} \PY{o}{=} \PY{n}{stats}\PY{o}{.}\PY{n}{ttest\PYZus{}ind}\PY{p}{(}\PY{n}{df\PYZus{}subset\PYZus{}playoffs\PYZus{}19}\PY{p}{[}\PY{l+s+s1}{\PYZsq{}}\PY{l+s+s1}{total\PYZus{}pt}\PY{l+s+s1}{\PYZsq{}}\PY{p}{]}\PY{p}{,}
                          \PY{n}{df\PYZus{}subset\PYZus{}regular\PYZus{}19}\PY{p}{[}\PY{l+s+s1}{\PYZsq{}}\PY{l+s+s1}{total\PYZus{}pt}\PY{l+s+s1}{\PYZsq{}}\PY{p}{]}\PY{p}{,} 
                          \PY{n}{equal\PYZus{}var}\PY{o}{=}\PY{k+kc}{False}\PY{p}{)}

\PY{n+nb}{print}\PY{p}{(}\PY{n}{f}\PY{l+s+s2}{\PYZdq{}}\PY{l+s+s2}{t: }\PY{l+s+si}{\PYZob{}results.statistic\PYZcb{}}\PY{l+s+s2}{\PYZdq{}}\PY{p}{)}
\PY{n+nb}{print}\PY{p}{(}\PY{n}{f}\PY{l+s+s2}{\PYZdq{}}\PY{l+s+s2}{p\PYZhy{}value: }\PY{l+s+si}{\PYZob{}results.pvalue\PYZcb{}}\PY{l+s+s2}{\PYZdq{}}\PY{p}{)}
\end{Verbatim}
\end{tcolorbox}

    \begin{Verbatim}[commandchars=\\\{\}]
t: -10.064931585735033
p-value: 8.557608405175591e-23
\end{Verbatim}

    \subsubsection*{Part I}\label{part-i}

From the last part, we see that the \(p\)-value is nearly zero and the
\(t\)-statistic is high (in magnitude). Therefore, we can reject the
null that the difference is zero in favor of the alternative that the
average total score for playoffs is lower than regular season with low
\(\alpha\)-level (\(\alpha\) less than \(0.05\)).

    \subsubsection*{Part J}\label{part-j}

We now consider the \([2000, 2017]\) time period.

    \begin{tcolorbox}[breakable, size=fbox, boxrule=1pt, pad at break*=1mm,colback=cellbackground, colframe=cellborder]
\prompt{In}{incolor}{32}{\hspace{4pt}}
\begin{Verbatim}[commandchars=\\\{\}]
\PY{n}{df\PYZus{}subset\PYZus{}playoffs\PYZus{}20} \PY{o}{=} \PY{n}{df\PYZus{}subset\PYZus{}playoffs}\PY{p}{[}\PY{p}{(}\PY{n}{df\PYZus{}subset\PYZus{}playoffs}\PY{p}{[}\PY{l+s+s1}{\PYZsq{}}\PY{l+s+s1}{season}\PY{l+s+s1}{\PYZsq{}}\PY{p}{]} \PY{o}{\PYZgt{}}\PY{o}{=} \PY{l+m+mi}{2000}\PY{p}{)} 
                                        \PY{o}{\PYZam{}} \PY{p}{(}\PY{n}{df\PYZus{}subset\PYZus{}playoffs}\PY{p}{[}\PY{l+s+s1}{\PYZsq{}}\PY{l+s+s1}{season}\PY{l+s+s1}{\PYZsq{}}\PY{p}{]} \PY{o}{\PYZlt{}}\PY{o}{=} \PY{l+m+mi}{2017}\PY{p}{)}\PY{p}{]}
\PY{n}{df\PYZus{}subset\PYZus{}regular\PYZus{}20} \PY{o}{=} \PY{n}{df\PYZus{}subset\PYZus{}regular}\PY{p}{[}\PY{p}{(}\PY{n}{df\PYZus{}subset\PYZus{}regular}\PY{p}{[}\PY{l+s+s1}{\PYZsq{}}\PY{l+s+s1}{season}\PY{l+s+s1}{\PYZsq{}}\PY{p}{]} \PY{o}{\PYZgt{}}\PY{o}{=} \PY{l+m+mi}{2000}\PY{p}{)} 
                                         \PY{o}{\PYZam{}} \PY{p}{(}\PY{n}{df\PYZus{}subset\PYZus{}regular}\PY{p}{[}\PY{l+s+s1}{\PYZsq{}}\PY{l+s+s1}{season}\PY{l+s+s1}{\PYZsq{}}\PY{p}{]} \PY{o}{\PYZlt{}}\PY{o}{=} \PY{l+m+mi}{2017}\PY{p}{)}\PY{p}{]}
\end{Verbatim}
\end{tcolorbox}

    \subsubsection*{Part K}\label{part-k}

We can compare the means of the total points scored between the playoffs
and regular season for the subset.

    \begin{tcolorbox}[breakable, size=fbox, boxrule=1pt, pad at break*=1mm,colback=cellbackground, colframe=cellborder]
\prompt{In}{incolor}{33}{\hspace{4pt}}
\begin{Verbatim}[commandchars=\\\{\}]
\PY{n}{hat\PYZus{}mu\PYZus{}playoffs\PYZus{}20} \PY{o}{=} \PY{n}{df\PYZus{}subset\PYZus{}playoffs\PYZus{}20}\PY{p}{[}\PY{l+s+s1}{\PYZsq{}}\PY{l+s+s1}{total\PYZus{}pt}\PY{l+s+s1}{\PYZsq{}}\PY{p}{]}\PY{o}{.}\PY{n}{mean}\PY{p}{(}\PY{p}{)}
\PY{n}{hat\PYZus{}mu\PYZus{}regular\PYZus{}20} \PY{o}{=} \PY{n}{df\PYZus{}subset\PYZus{}regular\PYZus{}20}\PY{p}{[}\PY{l+s+s1}{\PYZsq{}}\PY{l+s+s1}{total\PYZus{}pt}\PY{l+s+s1}{\PYZsq{}}\PY{p}{]}\PY{o}{.}\PY{n}{mean}\PY{p}{(}\PY{p}{)}

\PY{n+nb}{print}\PY{p}{(}\PY{n}{f}\PY{l+s+s2}{\PYZdq{}}\PY{l+s+s2}{Mean Playoffs: }\PY{l+s+si}{\PYZob{}hat\PYZus{}mu\PYZus{}playoffs\PYZus{}20\PYZcb{}}\PY{l+s+s2}{\PYZdq{}}\PY{p}{)}
\PY{n+nb}{print}\PY{p}{(}\PY{n}{f}\PY{l+s+s2}{\PYZdq{}}\PY{l+s+s2}{Mean Regular: }\PY{l+s+si}{\PYZob{}hat\PYZus{}mu\PYZus{}regular\PYZus{}20\PYZcb{}}\PY{l+s+s2}{\PYZdq{}}\PY{p}{)}
\end{Verbatim}
\end{tcolorbox}

    \begin{Verbatim}[commandchars=\\\{\}]
Mean Playoffs: 196.16927490871151
Mean Regular: 196.49699610483924
\end{Verbatim}

    \subsubsection*{Part L}\label{part-l}

We run a two-sample \(t\)-test for the two means without assuming equal
variance.

    \begin{tcolorbox}[breakable, size=fbox, boxrule=1pt, pad at break*=1mm,colback=cellbackground, colframe=cellborder]
\prompt{In}{incolor}{34}{\hspace{4pt}}
\begin{Verbatim}[commandchars=\\\{\}]
\PY{n}{results} \PY{o}{=} \PY{n}{stats}\PY{o}{.}\PY{n}{ttest\PYZus{}ind}\PY{p}{(}\PY{n}{df\PYZus{}subset\PYZus{}playoffs\PYZus{}20}\PY{p}{[}\PY{l+s+s1}{\PYZsq{}}\PY{l+s+s1}{total\PYZus{}pt}\PY{l+s+s1}{\PYZsq{}}\PY{p}{]}\PY{p}{,}
                          \PY{n}{df\PYZus{}subset\PYZus{}regular\PYZus{}20}\PY{p}{[}\PY{l+s+s1}{\PYZsq{}}\PY{l+s+s1}{total\PYZus{}pt}\PY{l+s+s1}{\PYZsq{}}\PY{p}{]}\PY{p}{,} 
                          \PY{n}{equal\PYZus{}var}\PY{o}{=}\PY{k+kc}{False}\PY{p}{)}

\PY{n+nb}{print}\PY{p}{(}\PY{n}{f}\PY{l+s+s2}{\PYZdq{}}\PY{l+s+s2}{t: }\PY{l+s+si}{\PYZob{}results.statistic\PYZcb{}}\PY{l+s+s2}{\PYZdq{}}\PY{p}{)}
\PY{n+nb}{print}\PY{p}{(}\PY{n}{f}\PY{l+s+s2}{\PYZdq{}}\PY{l+s+s2}{p\PYZhy{}value: }\PY{l+s+si}{\PYZob{}results.pvalue\PYZcb{}}\PY{l+s+s2}{\PYZdq{}}\PY{p}{)}
\end{Verbatim}
\end{tcolorbox}

    \begin{Verbatim}[commandchars=\\\{\}]
t: -0.8505001514109403
p-value: 0.3950826083683683
\end{Verbatim}

    \subsubsection*{Part M}\label{part-m}

From the last part, we see that the \(p\)-value is roughly 40\% and the
\(t\)-statistic is low (in magnitude). Therefore, we cannot reject the
null that the difference is zero in favor of the alternative that the
average total score for playoffs is lower than regular season with low
\(\alpha\)-level (\(\alpha\) less than \(0.05\)).

I do not necessarily believe this is direct evidence that differences in
scoring are no long as different as they used to be. Although the
difference was statistically significant before and it is no longer, we
should probably run a hypothesis test on whether the difference between
the two periods is statistically significant.

    \subsection*{Problem 7}\label{problem-7}

In this problem we consider the significance of the Michael Jordan
effect.

\subsubsection*{Part A}\label{part-a}

We consider only the years \([1992, 1997]\) and games featuring the
Chicago Bulls.

    \begin{tcolorbox}[breakable, size=fbox, boxrule=1pt, pad at break*=1mm,colback=cellbackground, colframe=cellborder]
\prompt{In}{incolor}{35}{\hspace{4pt}}
\begin{Verbatim}[commandchars=\\\{\}]
\PY{n}{df\PYZus{}subset} \PY{o}{=} \PY{n}{df}\PY{p}{[}\PY{p}{(}\PY{n}{df}\PY{p}{[}\PY{l+s+s1}{\PYZsq{}}\PY{l+s+s1}{season}\PY{l+s+s1}{\PYZsq{}}\PY{p}{]} \PY{o}{\PYZgt{}}\PY{o}{=} \PY{l+m+mi}{1992}\PY{p}{)} \PY{o}{\PYZam{}} \PY{p}{(}\PY{n}{df}\PY{p}{[}\PY{l+s+s1}{\PYZsq{}}\PY{l+s+s1}{season}\PY{l+s+s1}{\PYZsq{}}\PY{p}{]} \PY{o}{\PYZlt{}}\PY{o}{=} \PY{l+m+mi}{1997}\PY{p}{)} 
               \PY{o}{\PYZam{}} \PY{p}{(}\PY{p}{(}\PY{n}{df}\PY{p}{[}\PY{l+s+s1}{\PYZsq{}}\PY{l+s+s1}{home\PYZus{}team}\PY{l+s+s1}{\PYZsq{}}\PY{p}{]} \PY{o}{==} \PY{l+s+s1}{\PYZsq{}}\PY{l+s+s1}{Chicago Bulls}\PY{l+s+s1}{\PYZsq{}}\PY{p}{)} 
                  \PY{o}{|} \PY{p}{(}\PY{n}{df}\PY{p}{[}\PY{l+s+s1}{\PYZsq{}}\PY{l+s+s1}{away\PYZus{}team}\PY{l+s+s1}{\PYZsq{}}\PY{p}{]} \PY{o}{==} \PY{l+s+s1}{\PYZsq{}}\PY{l+s+s1}{Chicago Bulls}\PY{l+s+s1}{\PYZsq{}}\PY{p}{)}\PY{p}{)}\PY{p}{]}\PY{o}{.}\PY{n}{copy}\PY{p}{(}\PY{p}{)}
\end{Verbatim}
\end{tcolorbox}

    \subsubsection*{Part B}\label{part-b}

We create three new columns: - \texttt{for\_score}: points scored by the
Chicago Bulls (home or away) in the seasons 1992-1997. -
\texttt{against\_score}: points scored against the Chicago Bulls (home
or away) in the seasons 1992-1997. - \texttt{net\_score}: net score by
the Chicago Bulls (home or away) in the seasons 1992-1997.

    \begin{tcolorbox}[breakable, size=fbox, boxrule=1pt, pad at break*=1mm,colback=cellbackground, colframe=cellborder]
\prompt{In}{incolor}{36}{\hspace{4pt}}
\begin{Verbatim}[commandchars=\\\{\}]
\PY{n}{df\PYZus{}subset}\PY{p}{[}\PY{l+s+s1}{\PYZsq{}}\PY{l+s+s1}{for\PYZus{}score}\PY{l+s+s1}{\PYZsq{}}\PY{p}{]} \PY{o}{=} \PY{n}{df\PYZus{}subset}\PY{p}{[}\PY{l+s+s1}{\PYZsq{}}\PY{l+s+s1}{home\PYZus{}pt}\PY{l+s+s1}{\PYZsq{}}\PY{p}{]}\PY{o}{.}\PY{n}{where}\PY{p}{(}
                                \PY{n}{df\PYZus{}subset}\PY{p}{[}\PY{l+s+s1}{\PYZsq{}}\PY{l+s+s1}{home\PYZus{}team}\PY{l+s+s1}{\PYZsq{}}\PY{p}{]} \PY{o}{==} \PY{l+s+s1}{\PYZsq{}}\PY{l+s+s1}{Chicago Bulls}\PY{l+s+s1}{\PYZsq{}}\PY{p}{,} 
                                \PY{n}{df\PYZus{}subset}\PY{p}{[}\PY{l+s+s1}{\PYZsq{}}\PY{l+s+s1}{away\PYZus{}pt}\PY{l+s+s1}{\PYZsq{}}\PY{p}{]}\PY{p}{)}
\PY{n}{df\PYZus{}subset}\PY{p}{[}\PY{l+s+s1}{\PYZsq{}}\PY{l+s+s1}{against\PYZus{}score}\PY{l+s+s1}{\PYZsq{}}\PY{p}{]} \PY{o}{=} \PY{n}{df\PYZus{}subset}\PY{p}{[}\PY{l+s+s1}{\PYZsq{}}\PY{l+s+s1}{away\PYZus{}pt}\PY{l+s+s1}{\PYZsq{}}\PY{p}{]}\PY{o}{.}\PY{n}{where}\PY{p}{(}
                                \PY{n}{df\PYZus{}subset}\PY{p}{[}\PY{l+s+s1}{\PYZsq{}}\PY{l+s+s1}{home\PYZus{}team}\PY{l+s+s1}{\PYZsq{}}\PY{p}{]} \PY{o}{==} \PY{l+s+s1}{\PYZsq{}}\PY{l+s+s1}{Chicago Bulls}\PY{l+s+s1}{\PYZsq{}}\PY{p}{,} 
                                \PY{n}{df\PYZus{}subset}\PY{p}{[}\PY{l+s+s1}{\PYZsq{}}\PY{l+s+s1}{home\PYZus{}pt}\PY{l+s+s1}{\PYZsq{}}\PY{p}{]}\PY{p}{)}
\PY{n}{df\PYZus{}subset}\PY{p}{[}\PY{l+s+s1}{\PYZsq{}}\PY{l+s+s1}{net\PYZus{}score}\PY{l+s+s1}{\PYZsq{}}\PY{p}{]} \PY{o}{=} \PY{n}{df\PYZus{}subset}\PY{p}{[}\PY{l+s+s1}{\PYZsq{}}\PY{l+s+s1}{for\PYZus{}score}\PY{l+s+s1}{\PYZsq{}}\PY{p}{]} \PY{o}{\PYZhy{}} \PY{n}{df\PYZus{}subset}\PY{p}{[}\PY{l+s+s1}{\PYZsq{}}\PY{l+s+s1}{against\PYZus{}score}\PY{l+s+s1}{\PYZsq{}}\PY{p}{]}
\end{Verbatim}
\end{tcolorbox}

    \subsubsection*{Part C}\label{part-c}

We now construct two disjoint set with and without Michael Jordan
playing.

    \begin{tcolorbox}[breakable, size=fbox, boxrule=1pt, pad at break*=1mm,colback=cellbackground, colframe=cellborder]
\prompt{In}{incolor}{37}{\hspace{4pt}}
\begin{Verbatim}[commandchars=\\\{\}]
\PY{n}{df\PYZus{}subset\PYZus{}jordan} \PY{o}{=} \PY{n}{df\PYZus{}subset}\PY{p}{[}\PY{p}{(}\PY{n}{df\PYZus{}subset}\PY{p}{[}\PY{l+s+s1}{\PYZsq{}}\PY{l+s+s1}{season}\PY{l+s+s1}{\PYZsq{}}\PY{p}{]} \PY{o}{\PYZlt{}}\PY{o}{=} \PY{l+m+mi}{1993}\PY{p}{)} 
                             \PY{o}{|} \PY{p}{(}\PY{n}{df\PYZus{}subset}\PY{p}{[}\PY{l+s+s1}{\PYZsq{}}\PY{l+s+s1}{season}\PY{l+s+s1}{\PYZsq{}}\PY{p}{]} \PY{o}{\PYZgt{}}\PY{o}{=} \PY{l+m+mi}{1996}\PY{p}{)}\PY{p}{]}
\PY{n}{df\PYZus{}subset\PYZus{}no\PYZus{}jordan} \PY{o}{=} \PY{n}{df\PYZus{}subset}\PY{p}{[}\PY{p}{(}\PY{n}{df\PYZus{}subset}\PY{p}{[}\PY{l+s+s1}{\PYZsq{}}\PY{l+s+s1}{season}\PY{l+s+s1}{\PYZsq{}}\PY{p}{]} \PY{o}{\PYZgt{}}\PY{o}{=} \PY{l+m+mi}{1994}\PY{p}{)} 
                                \PY{o}{\PYZam{}} \PY{p}{(}\PY{n}{df\PYZus{}subset}\PY{p}{[}\PY{l+s+s1}{\PYZsq{}}\PY{l+s+s1}{season}\PY{l+s+s1}{\PYZsq{}}\PY{p}{]} \PY{o}{\PYZlt{}}\PY{o}{=} \PY{l+m+mi}{1995}\PY{p}{)}\PY{p}{]}
\end{Verbatim}
\end{tcolorbox}

    \subsubsection*{Part D}\label{part-d}

The null hypothesis we wish to test is that average for points scored
with Michael Jordan games is equal in no Michael Jordan games. The
alternative hypothesis is that the average for points with Michael
Jordan is greater than without Michael Jordan. Mathematically,
\[ H_0 : \mu_{\mathrm{jordan}} = \mu_{\mathrm{no\ jordan}} \]
\[ H_1 : \mu_{\mathrm{jordan}} > \mu_{\mathrm{no\ jordan}} \]

Or, the null hypothesis we wish to test is that average against points
scored with Michael Jordan games is equal in no Michael Jordan games.
The alternative hypothesis is that the average against points with
Michael Jordan is less than without Michael Jordan. Mathematically,
\[ H_0 : \mu_{\mathrm{jordan}} = \mu_{\mathrm{no\ jordan}} \]
\[ H_1 : \mu_{\mathrm{jordan}} < \mu_{\mathrm{no\ jordan}} \]

Or, the null hypothesis we wish to test is that average net points
scored with Michael Jordan games is equal in no Michael Jordan games.
The alternative hypothesis is that the average net points with Michael
Jordan is greater than without Michael Jordan. Mathematically,
\[ H_0 : \mu_{\mathrm{jordan}} = \mu_{\mathrm{no\ jordan}} \]
\[ H_1 : \mu_{\mathrm{jordan}} > \mu_{\mathrm{no\ jordan}} \]

    \subsubsection*{Part E}\label{part-e}

We compute the mean for score with and without Michael Jordan.

    \begin{tcolorbox}[breakable, size=fbox, boxrule=1pt, pad at break*=1mm,colback=cellbackground, colframe=cellborder]
\prompt{In}{incolor}{38}{\hspace{4pt}}
\begin{Verbatim}[commandchars=\\\{\}]
\PY{n}{hat\PYZus{}mu\PYZus{}jordan} \PY{o}{=} \PY{n}{df\PYZus{}subset\PYZus{}jordan}\PY{p}{[}\PY{l+s+s1}{\PYZsq{}}\PY{l+s+s1}{for\PYZus{}score}\PY{l+s+s1}{\PYZsq{}}\PY{p}{]}\PY{o}{.}\PY{n}{mean}\PY{p}{(}\PY{p}{)}
\PY{n}{hat\PYZus{}mu\PYZus{}no\PYZus{}jordan} \PY{o}{=} \PY{n}{df\PYZus{}subset\PYZus{}no\PYZus{}jordan}\PY{p}{[}\PY{l+s+s1}{\PYZsq{}}\PY{l+s+s1}{for\PYZus{}score}\PY{l+s+s1}{\PYZsq{}}\PY{p}{]}\PY{o}{.}\PY{n}{mean}\PY{p}{(}\PY{p}{)}

\PY{n+nb}{print}\PY{p}{(}\PY{n}{f}\PY{l+s+s2}{\PYZdq{}}\PY{l+s+s2}{Mean Jordan: }\PY{l+s+si}{\PYZob{}hat\PYZus{}mu\PYZus{}jordan\PYZcb{}}\PY{l+s+s2}{\PYZdq{}}\PY{p}{)}
\PY{n+nb}{print}\PY{p}{(}\PY{n}{f}\PY{l+s+s2}{\PYZdq{}}\PY{l+s+s2}{Mean No Jordan: }\PY{l+s+si}{\PYZob{}hat\PYZus{}mu\PYZus{}no\PYZus{}jordan\PYZcb{}}\PY{l+s+s2}{\PYZdq{}}\PY{p}{)}
\end{Verbatim}
\end{tcolorbox}

    \begin{Verbatim}[commandchars=\\\{\}]
Mean Jordan: 104.39162561576354
Mean No Jordan: 99.33695652173913
\end{Verbatim}

    \subsubsection*{Part F}\label{part-f}

We run a two-sample \(t\)-test for the two means without assuming equal
variance.

    \begin{tcolorbox}[breakable, size=fbox, boxrule=1pt, pad at break*=1mm,colback=cellbackground, colframe=cellborder]
\prompt{In}{incolor}{39}{\hspace{4pt}}
\begin{Verbatim}[commandchars=\\\{\}]
\PY{n}{results} \PY{o}{=} \PY{n}{stats}\PY{o}{.}\PY{n}{ttest\PYZus{}ind}\PY{p}{(}\PY{n}{df\PYZus{}subset\PYZus{}jordan}\PY{p}{[}\PY{l+s+s1}{\PYZsq{}}\PY{l+s+s1}{for\PYZus{}score}\PY{l+s+s1}{\PYZsq{}}\PY{p}{]}\PY{p}{,}
                          \PY{n}{df\PYZus{}subset\PYZus{}no\PYZus{}jordan}\PY{p}{[}\PY{l+s+s1}{\PYZsq{}}\PY{l+s+s1}{for\PYZus{}score}\PY{l+s+s1}{\PYZsq{}}\PY{p}{]}\PY{p}{,} 
                          \PY{n}{equal\PYZus{}var}\PY{o}{=}\PY{k+kc}{False}\PY{p}{)}

\PY{n+nb}{print}\PY{p}{(}\PY{n}{f}\PY{l+s+s2}{\PYZdq{}}\PY{l+s+s2}{t: }\PY{l+s+si}{\PYZob{}results.statistic\PYZcb{}}\PY{l+s+s2}{\PYZdq{}}\PY{p}{)}
\PY{n+nb}{print}\PY{p}{(}\PY{n}{f}\PY{l+s+s2}{\PYZdq{}}\PY{l+s+s2}{p\PYZhy{}value: }\PY{l+s+s2}{\PYZob{}}\PY{l+s+s2}{results.pvalue / 2\PYZcb{}}\PY{l+s+s2}{\PYZdq{}}\PY{p}{)}
\end{Verbatim}
\end{tcolorbox}

    \begin{Verbatim}[commandchars=\\\{\}]
t: 4.572966583169566
p-value: 3.4316212038540005e-06
\end{Verbatim}

    \subsubsection*{Part G}\label{part-g}

We compute the mean against score with and without Michael Jordan.

    \begin{tcolorbox}[breakable, size=fbox, boxrule=1pt, pad at break*=1mm,colback=cellbackground, colframe=cellborder]
\prompt{In}{incolor}{40}{\hspace{4pt}}
\begin{Verbatim}[commandchars=\\\{\}]
\PY{n}{hat\PYZus{}mu\PYZus{}jordan} \PY{o}{=} \PY{n}{df\PYZus{}subset\PYZus{}jordan}\PY{p}{[}\PY{l+s+s1}{\PYZsq{}}\PY{l+s+s1}{against\PYZus{}score}\PY{l+s+s1}{\PYZsq{}}\PY{p}{]}\PY{o}{.}\PY{n}{mean}\PY{p}{(}\PY{p}{)}
\PY{n}{hat\PYZus{}mu\PYZus{}no\PYZus{}jordan} \PY{o}{=} \PY{n}{df\PYZus{}subset\PYZus{}no\PYZus{}jordan}\PY{p}{[}\PY{l+s+s1}{\PYZsq{}}\PY{l+s+s1}{against\PYZus{}score}\PY{l+s+s1}{\PYZsq{}}\PY{p}{]}\PY{o}{.}\PY{n}{mean}\PY{p}{(}\PY{p}{)}

\PY{n+nb}{print}\PY{p}{(}\PY{n}{f}\PY{l+s+s2}{\PYZdq{}}\PY{l+s+s2}{Mean Jordan: }\PY{l+s+si}{\PYZob{}hat\PYZus{}mu\PYZus{}jordan\PYZcb{}}\PY{l+s+s2}{\PYZdq{}}\PY{p}{)}
\PY{n+nb}{print}\PY{p}{(}\PY{n}{f}\PY{l+s+s2}{\PYZdq{}}\PY{l+s+s2}{Mean No Jordan: }\PY{l+s+si}{\PYZob{}hat\PYZus{}mu\PYZus{}no\PYZus{}jordan\PYZcb{}}\PY{l+s+s2}{\PYZdq{}}\PY{p}{)}
\end{Verbatim}
\end{tcolorbox}

    \begin{Verbatim}[commandchars=\\\{\}]
Mean Jordan: 95.02216748768473
Mean No Jordan: 95.60326086956522
\end{Verbatim}

    We run a two-sample \(t\)-test for the two means without assuming equal
variance.

    \begin{tcolorbox}[breakable, size=fbox, boxrule=1pt, pad at break*=1mm,colback=cellbackground, colframe=cellborder]
\prompt{In}{incolor}{41}{\hspace{4pt}}
\begin{Verbatim}[commandchars=\\\{\}]
\PY{n}{results} \PY{o}{=} \PY{n}{stats}\PY{o}{.}\PY{n}{ttest\PYZus{}ind}\PY{p}{(}\PY{n}{df\PYZus{}subset\PYZus{}jordan}\PY{p}{[}\PY{l+s+s1}{\PYZsq{}}\PY{l+s+s1}{against\PYZus{}score}\PY{l+s+s1}{\PYZsq{}}\PY{p}{]}\PY{p}{,}
                          \PY{n}{df\PYZus{}subset\PYZus{}no\PYZus{}jordan}\PY{p}{[}\PY{l+s+s1}{\PYZsq{}}\PY{l+s+s1}{against\PYZus{}score}\PY{l+s+s1}{\PYZsq{}}\PY{p}{]}\PY{p}{,} 
                          \PY{n}{equal\PYZus{}var}\PY{o}{=}\PY{k+kc}{False}\PY{p}{)}

\PY{n+nb}{print}\PY{p}{(}\PY{n}{f}\PY{l+s+s2}{\PYZdq{}}\PY{l+s+s2}{t: }\PY{l+s+si}{\PYZob{}results.statistic\PYZcb{}}\PY{l+s+s2}{\PYZdq{}}\PY{p}{)}
\PY{n+nb}{print}\PY{p}{(}\PY{n}{f}\PY{l+s+s2}{\PYZdq{}}\PY{l+s+s2}{p\PYZhy{}value: }\PY{l+s+s2}{\PYZob{}}\PY{l+s+s2}{results.pvalue / 2\PYZcb{}}\PY{l+s+s2}{\PYZdq{}}\PY{p}{)}
\end{Verbatim}
\end{tcolorbox}

    \begin{Verbatim}[commandchars=\\\{\}]
t: -0.5863766703116524
p-value: 0.2789993668005224
\end{Verbatim}

    \subsubsection*{Part H}\label{part-h}

From Part F, we see that the \(p\)-value is approximately zero and the
\(t\)-statistic is relatively high. Therefore, we can reject the null
that the difference is zero in favor of the alternative that the average
for score with Jordan is higher than without Jordan with low
\(\alpha\)-level (\(\alpha\) less than \(0.05\)).

From Part G, we see that the \(p\)-value is roughly 28\% and the
\(t\)-statistic is relatively low (in magnitude). Therefore, we cannot
reject the null that the difference is zero in favor of the alternative
that the average against score with Jordan is lows than without Jordan
with low \(\alpha\)-level (\(\alpha\) less than \(0.28\)).

From this, Jordan's presence statistically significantly increases the
Chicago Bulls' score, but does not statistically reduce the opponent's
score.

    \subsubsection*{Part I}\label{part-i}

We compute the mean net score with and without Michael Jordan.

    \begin{tcolorbox}[breakable, size=fbox, boxrule=1pt, pad at break*=1mm,colback=cellbackground, colframe=cellborder]
\prompt{In}{incolor}{42}{\hspace{4pt}}
\begin{Verbatim}[commandchars=\\\{\}]
\PY{n}{hat\PYZus{}mu\PYZus{}jordan} \PY{o}{=} \PY{n}{df\PYZus{}subset\PYZus{}jordan}\PY{p}{[}\PY{l+s+s1}{\PYZsq{}}\PY{l+s+s1}{net\PYZus{}score}\PY{l+s+s1}{\PYZsq{}}\PY{p}{]}\PY{o}{.}\PY{n}{mean}\PY{p}{(}\PY{p}{)}
\PY{n}{hat\PYZus{}mu\PYZus{}no\PYZus{}jordan} \PY{o}{=} \PY{n}{df\PYZus{}subset\PYZus{}no\PYZus{}jordan}\PY{p}{[}\PY{l+s+s1}{\PYZsq{}}\PY{l+s+s1}{net\PYZus{}score}\PY{l+s+s1}{\PYZsq{}}\PY{p}{]}\PY{o}{.}\PY{n}{mean}\PY{p}{(}\PY{p}{)}

\PY{n+nb}{print}\PY{p}{(}\PY{n}{f}\PY{l+s+s2}{\PYZdq{}}\PY{l+s+s2}{Mean Jordan: }\PY{l+s+si}{\PYZob{}hat\PYZus{}mu\PYZus{}jordan\PYZcb{}}\PY{l+s+s2}{\PYZdq{}}\PY{p}{)}
\PY{n+nb}{print}\PY{p}{(}\PY{n}{f}\PY{l+s+s2}{\PYZdq{}}\PY{l+s+s2}{Mean No Jordan: }\PY{l+s+si}{\PYZob{}hat\PYZus{}mu\PYZus{}no\PYZus{}jordan\PYZcb{}}\PY{l+s+s2}{\PYZdq{}}\PY{p}{)}
\end{Verbatim}
\end{tcolorbox}

    \begin{Verbatim}[commandchars=\\\{\}]
Mean Jordan: 9.369458128078819
Mean No Jordan: 3.733695652173913
\end{Verbatim}

    We run a two-sample \(t\)-test for the two means without assuming equal
variance.

    \begin{tcolorbox}[breakable, size=fbox, boxrule=1pt, pad at break*=1mm,colback=cellbackground, colframe=cellborder]
\prompt{In}{incolor}{43}{\hspace{4pt}}
\begin{Verbatim}[commandchars=\\\{\}]
\PY{n}{results} \PY{o}{=} \PY{n}{stats}\PY{o}{.}\PY{n}{ttest\PYZus{}ind}\PY{p}{(}\PY{n}{df\PYZus{}subset\PYZus{}jordan}\PY{p}{[}\PY{l+s+s1}{\PYZsq{}}\PY{l+s+s1}{net\PYZus{}score}\PY{l+s+s1}{\PYZsq{}}\PY{p}{]}\PY{p}{,}
                          \PY{n}{df\PYZus{}subset\PYZus{}no\PYZus{}jordan}\PY{p}{[}\PY{l+s+s1}{\PYZsq{}}\PY{l+s+s1}{net\PYZus{}score}\PY{l+s+s1}{\PYZsq{}}\PY{p}{]}\PY{p}{,} 
                          \PY{n}{equal\PYZus{}var}\PY{o}{=}\PY{k+kc}{False}\PY{p}{)}

\PY{n+nb}{print}\PY{p}{(}\PY{n}{f}\PY{l+s+s2}{\PYZdq{}}\PY{l+s+s2}{t: }\PY{l+s+si}{\PYZob{}results.statistic\PYZcb{}}\PY{l+s+s2}{\PYZdq{}}\PY{p}{)}
\PY{n+nb}{print}\PY{p}{(}\PY{n}{f}\PY{l+s+s2}{\PYZdq{}}\PY{l+s+s2}{p\PYZhy{}value: }\PY{l+s+s2}{\PYZob{}}\PY{l+s+s2}{results.pvalue / 2\PYZcb{}}\PY{l+s+s2}{\PYZdq{}}\PY{p}{)}
\end{Verbatim}
\end{tcolorbox}

    \begin{Verbatim}[commandchars=\\\{\}]
t: 4.808760080098417
p-value: 1.1737130782218827e-06
\end{Verbatim}

    \subsubsection*{Part J}\label{part-j}

From Part F, we see that the \(p\)-value is approximately zero and the
\(t\)-statistic is relatively high. Therefore, we can reject the null
that the difference is zero in favor of the alternative that the average
net score with Jordan is higher than without Jordan with low
\(\alpha\)-level (\(\alpha\) less than \(0.05\)).

This confirms the observations from the test in the previous parts as
Jordan's presence increases the Chicago Bulls' scores and doesn't
statistically significantly affect the opponent's score.

    \subsubsection*{Part K}\label{part-k}

Although Scottie Pipen and Phil Jackson were constant, we also need to
consider other players (and coaches) on the Chicago Bulls' team. For
example, other players likely retired during this period and new rookies
were added as well. These players likely also had an impact on the
scoring ability of the Chicago Bulls.

    \subsection*{Problem 8}\label{problem-8}

In this problem we test if back-to-back games has an effect on total
points scored.

\subsubsection*{Part A}\label{part-a}

First we create a function to compute the number of combined rest days
for the two teams.

    \begin{tcolorbox}[breakable, size=fbox, boxrule=1pt, pad at break*=1mm,colback=cellbackground, colframe=cellborder]
\prompt{In}{incolor}{44}{\hspace{4pt}}
\begin{Verbatim}[commandchars=\\\{\}]
\PY{k}{def} \PY{n+nf}{rest\PYZus{}days}\PY{p}{(}\PY{n}{df}\PY{p}{)}\PY{p}{:}
    \PY{l+s+sd}{\PYZdq{}\PYZdq{}\PYZdq{}Adds a column `test\PYZus{}days` which counts the number of days from }
\PY{l+s+sd}{        the last played game.}
\PY{l+s+sd}{        }
\PY{l+s+sd}{    Args:}
\PY{l+s+sd}{        df (DataFrame): A dataframe containing the existing NBA }
\PY{l+s+sd}{            historical data of games with a column `date`.}
\PY{l+s+sd}{    }
\PY{l+s+sd}{    Returns:}
\PY{l+s+sd}{        DataFrame: New dataframe with the additional column.}
\PY{l+s+sd}{        }
\PY{l+s+sd}{    \PYZdq{}\PYZdq{}\PYZdq{}}
    \PY{n}{last\PYZus{}game} \PY{o}{=} \PY{p}{\PYZob{}} \PY{n}{team}\PY{p}{:} \PY{n}{pd}\PY{o}{.}\PY{n}{to\PYZus{}datetime}\PY{p}{(}\PY{l+s+s1}{\PYZsq{}}\PY{l+s+s1}{1948\PYZhy{}01\PYZhy{}01}\PY{l+s+s1}{\PYZsq{}}\PY{p}{,} \PY{n+nb}{format}\PY{o}{=}\PY{l+s+s1}{\PYZsq{}}\PY{l+s+s1}{\PYZpc{}}\PY{l+s+s1}{Y\PYZhy{}}\PY{l+s+s1}{\PYZpc{}}\PY{l+s+s1}{m\PYZhy{}}\PY{l+s+si}{\PYZpc{}d}\PY{l+s+s1}{\PYZsq{}}\PY{p}{)} 
             \PY{k}{for} \PY{n}{team} \PY{o+ow}{in} \PY{n}{df}\PY{p}{[}\PY{l+s+s1}{\PYZsq{}}\PY{l+s+s1}{home\PYZus{}team}\PY{l+s+s1}{\PYZsq{}}\PY{p}{]}\PY{o}{.}\PY{n}{unique}\PY{p}{(}\PY{p}{)} \PY{p}{\PYZcb{}}

    \PY{k}{def} \PY{n+nf}{calc\PYZus{}rest}\PY{p}{(}\PY{n}{row}\PY{p}{)}\PY{p}{:}    

        \PY{n}{days} \PY{o}{=} \PY{n+nb}{min}\PY{p}{(}\PY{p}{(}\PY{n}{row}\PY{p}{[}\PY{l+s+s1}{\PYZsq{}}\PY{l+s+s1}{date}\PY{l+s+s1}{\PYZsq{}}\PY{p}{]} \PY{o}{\PYZhy{}} \PY{n}{last\PYZus{}game}\PY{p}{[}\PY{n}{row}\PY{p}{[}\PY{l+s+s1}{\PYZsq{}}\PY{l+s+s1}{home\PYZus{}team}\PY{l+s+s1}{\PYZsq{}}\PY{p}{]}\PY{p}{]}\PY{p}{)}\PY{o}{.}\PY{n}{days}\PY{p}{,} \PY{l+m+mi}{8}\PY{p}{)} \PYZbs{}
               \PY{o}{+} \PY{n+nb}{min}\PY{p}{(}\PY{p}{(}\PY{n}{row}\PY{p}{[}\PY{l+s+s1}{\PYZsq{}}\PY{l+s+s1}{date}\PY{l+s+s1}{\PYZsq{}}\PY{p}{]} \PY{o}{\PYZhy{}} \PY{n}{last\PYZus{}game}\PY{p}{[}\PY{n}{row}\PY{p}{[}\PY{l+s+s1}{\PYZsq{}}\PY{l+s+s1}{away\PYZus{}team}\PY{l+s+s1}{\PYZsq{}}\PY{p}{]}\PY{p}{]}\PY{p}{)}\PY{o}{.}\PY{n}{days}\PY{p}{,} \PY{l+m+mi}{8}\PY{p}{)}

        \PY{n}{last\PYZus{}game}\PY{p}{[}\PY{n}{row}\PY{p}{[}\PY{l+s+s1}{\PYZsq{}}\PY{l+s+s1}{home\PYZus{}team}\PY{l+s+s1}{\PYZsq{}}\PY{p}{]}\PY{p}{]} \PY{o}{=} \PY{n}{row}\PY{p}{[}\PY{l+s+s1}{\PYZsq{}}\PY{l+s+s1}{date}\PY{l+s+s1}{\PYZsq{}}\PY{p}{]}
        \PY{n}{last\PYZus{}game}\PY{p}{[}\PY{n}{row}\PY{p}{[}\PY{l+s+s1}{\PYZsq{}}\PY{l+s+s1}{away\PYZus{}team}\PY{l+s+s1}{\PYZsq{}}\PY{p}{]}\PY{p}{]} \PY{o}{=} \PY{n}{row}\PY{p}{[}\PY{l+s+s1}{\PYZsq{}}\PY{l+s+s1}{date}\PY{l+s+s1}{\PYZsq{}}\PY{p}{]}

        \PY{k}{return} \PY{n}{days}

    \PY{n}{df}\PY{p}{[}\PY{l+s+s1}{\PYZsq{}}\PY{l+s+s1}{rest\PYZus{}days}\PY{l+s+s1}{\PYZsq{}}\PY{p}{]} \PY{o}{=} \PY{n}{df}\PY{o}{.}\PY{n}{apply}\PY{p}{(}\PY{n}{calc\PYZus{}rest}\PY{p}{,} \PY{n}{axis}\PY{o}{=}\PY{l+m+mi}{1}\PY{p}{)}
    
    \PY{k}{return} \PY{n}{df}
\end{Verbatim}
\end{tcolorbox}

    \subsubsection*{Part B}\label{part-b}

We can then apply this function to the dataframe.

    \begin{tcolorbox}[breakable, size=fbox, boxrule=1pt, pad at break*=1mm,colback=cellbackground, colframe=cellborder]
\prompt{In}{incolor}{45}{\hspace{4pt}}
\begin{Verbatim}[commandchars=\\\{\}]
\PY{n}{df} \PY{o}{=} \PY{n}{rest\PYZus{}days}\PY{p}{(}\PY{n}{df}\PY{p}{)}
\end{Verbatim}
\end{tcolorbox}

    \subsubsection*{Part C}\label{part-c}

We plot a histogram of these rest days.

    \begin{tcolorbox}[breakable, size=fbox, boxrule=1pt, pad at break*=1mm,colback=cellbackground, colframe=cellborder]
\prompt{In}{incolor}{46}{\hspace{4pt}}
\begin{Verbatim}[commandchars=\\\{\}]
\PY{n}{plt}\PY{o}{.}\PY{n}{hist}\PY{p}{(}\PY{n}{df}\PY{p}{[}\PY{l+s+s1}{\PYZsq{}}\PY{l+s+s1}{rest\PYZus{}days}\PY{l+s+s1}{\PYZsq{}}\PY{p}{]}\PY{p}{,} \PY{n}{bins}\PY{o}{=}\PY{l+m+mi}{16}\PY{p}{)}

\PY{n}{plt}\PY{o}{.}\PY{n}{title}\PY{p}{(}\PY{l+s+s1}{\PYZsq{}}\PY{l+s+s1}{Rest Days}\PY{l+s+s1}{\PYZsq{}}\PY{p}{)}
\PY{n}{plt}\PY{o}{.}\PY{n}{ylabel}\PY{p}{(}\PY{l+s+s1}{\PYZsq{}}\PY{l+s+s1}{Frequency}\PY{l+s+s1}{\PYZsq{}}\PY{p}{)}
\PY{n}{\PYZus{}} \PY{o}{=} \PY{n}{plt}\PY{o}{.}\PY{n}{xlabel}\PY{p}{(}\PY{l+s+s1}{\PYZsq{}}\PY{l+s+s1}{Rest Days}\PY{l+s+s1}{\PYZsq{}}\PY{p}{)}
\end{Verbatim}
\end{tcolorbox}

    \begin{center}
    \adjustimage{max size={0.9\linewidth}{0.9\paperheight}}{output_106_0.png}
    \end{center}
    { \hspace*{\fill} \\}
    
    \subsubsection*{Part D}\label{part-d}

The histogram suggests most teams probably have more than one day of
rest each, but typically less six days of rest total.

    \subsubsection*{Part E}\label{part-e}

We create two subsets one for back-to-back games and longer than
back-to-back.

    \begin{tcolorbox}[breakable, size=fbox, boxrule=1pt, pad at break*=1mm,colback=cellbackground, colframe=cellborder]
\prompt{In}{incolor}{47}{\hspace{4pt}}
\begin{Verbatim}[commandchars=\\\{\}]
\PY{n}{df\PYZus{}subset} \PY{o}{=} \PY{n}{df}\PY{p}{[}\PY{p}{(}\PY{n}{df}\PY{p}{[}\PY{l+s+s1}{\PYZsq{}}\PY{l+s+s1}{season}\PY{l+s+s1}{\PYZsq{}}\PY{p}{]} \PY{o}{\PYZgt{}}\PY{o}{=} \PY{l+m+mi}{2000}\PY{p}{)} \PY{o}{\PYZam{}} \PY{p}{(}\PY{n}{df}\PY{p}{[}\PY{l+s+s1}{\PYZsq{}}\PY{l+s+s1}{season}\PY{l+s+s1}{\PYZsq{}}\PY{p}{]} \PY{o}{\PYZlt{}}\PY{o}{=} \PY{l+m+mi}{2017}\PY{p}{)}\PY{p}{]}

\PY{n}{df\PYZus{}subset\PYZus{}back\PYZus{}to\PYZus{}back} \PY{o}{=} \PY{n}{df\PYZus{}subset}\PY{p}{[}\PY{n}{df\PYZus{}subset}\PY{p}{[}\PY{l+s+s1}{\PYZsq{}}\PY{l+s+s1}{rest\PYZus{}days}\PY{l+s+s1}{\PYZsq{}}\PY{p}{]} \PY{o}{\PYZlt{}}\PY{o}{=} \PY{l+m+mi}{3}\PY{p}{]}
\PY{n}{df\PYZus{}subset\PYZus{}longer} \PY{o}{=} \PY{n}{df\PYZus{}subset}\PY{p}{[}\PY{p}{(}\PY{n}{df\PYZus{}subset}\PY{p}{[}\PY{l+s+s1}{\PYZsq{}}\PY{l+s+s1}{rest\PYZus{}days}\PY{l+s+s1}{\PYZsq{}}\PY{p}{]} \PY{o}{\PYZgt{}}\PY{o}{=} \PY{l+m+mi}{4}\PY{p}{)} 
                             \PY{o}{\PYZam{}} \PY{p}{(}\PY{n}{df\PYZus{}subset}\PY{p}{[}\PY{l+s+s1}{\PYZsq{}}\PY{l+s+s1}{rest\PYZus{}days}\PY{l+s+s1}{\PYZsq{}}\PY{p}{]} \PY{o}{\PYZlt{}}\PY{o}{=} \PY{l+m+mi}{5}\PY{p}{)}\PY{p}{]}
\end{Verbatim}
\end{tcolorbox}

    \subsubsection*{Part F}\label{part-f}

The null hypothesis test we would like to test is if the mean total
score for back-to-back games is equal to the mean total score for
non-back-to-back games. The alternative is that the means are different.
Mathematically,
\[ H_0 : \mu_{\mathrm{back\ to\ back}} = \mu_{\mathrm{longer}} \]
\[ H_1 : \mu_{\mathrm{back\ to\ back}} \neq \mu_{\mathrm{longer}} \]

    \subsubsection*{Part G}\label{part-g}

We compute the mean net score with and without rest days.

    \begin{tcolorbox}[breakable, size=fbox, boxrule=1pt, pad at break*=1mm,colback=cellbackground, colframe=cellborder]
\prompt{In}{incolor}{48}{\hspace{4pt}}
\begin{Verbatim}[commandchars=\\\{\}]
\PY{n}{hat\PYZus{}mu\PYZus{}back\PYZus{}to\PYZus{}back} \PY{o}{=} \PY{n}{df\PYZus{}subset\PYZus{}back\PYZus{}to\PYZus{}back}\PY{p}{[}\PY{l+s+s1}{\PYZsq{}}\PY{l+s+s1}{total\PYZus{}pt}\PY{l+s+s1}{\PYZsq{}}\PY{p}{]}\PY{o}{.}\PY{n}{mean}\PY{p}{(}\PY{p}{)}
\PY{n}{hat\PYZus{}mu\PYZus{}longer} \PY{o}{=} \PY{n}{df\PYZus{}subset\PYZus{}longer}\PY{p}{[}\PY{l+s+s1}{\PYZsq{}}\PY{l+s+s1}{total\PYZus{}pt}\PY{l+s+s1}{\PYZsq{}}\PY{p}{]}\PY{o}{.}\PY{n}{mean}\PY{p}{(}\PY{p}{)}

\PY{n+nb}{print}\PY{p}{(}\PY{n}{f}\PY{l+s+s2}{\PYZdq{}}\PY{l+s+s2}{Mean Back to Back: }\PY{l+s+si}{\PYZob{}hat\PYZus{}mu\PYZus{}back\PYZus{}to\PYZus{}back\PYZcb{}}\PY{l+s+s2}{\PYZdq{}}\PY{p}{)}
\PY{n+nb}{print}\PY{p}{(}\PY{n}{f}\PY{l+s+s2}{\PYZdq{}}\PY{l+s+s2}{Mean Longer: }\PY{l+s+si}{\PYZob{}hat\PYZus{}mu\PYZus{}longer\PYZcb{}}\PY{l+s+s2}{\PYZdq{}}\PY{p}{)}
\end{Verbatim}
\end{tcolorbox}

    \begin{Verbatim}[commandchars=\\\{\}]
Mean Back to Back: 196.54280821917808
Mean Longer: 197.18168168168168
\end{Verbatim}

    \subsubsection*{Part H}\label{part-h}

We run a two-sample \(t\)-test for the two means without assuming equal
variance.

    \begin{tcolorbox}[breakable, size=fbox, boxrule=1pt, pad at break*=1mm,colback=cellbackground, colframe=cellborder]
\prompt{In}{incolor}{49}{\hspace{4pt}}
\begin{Verbatim}[commandchars=\\\{\}]
\PY{n}{results} \PY{o}{=} \PY{n}{stats}\PY{o}{.}\PY{n}{ttest\PYZus{}ind}\PY{p}{(}\PY{n}{df\PYZus{}subset\PYZus{}back\PYZus{}to\PYZus{}back}\PY{p}{[}\PY{l+s+s1}{\PYZsq{}}\PY{l+s+s1}{total\PYZus{}pt}\PY{l+s+s1}{\PYZsq{}}\PY{p}{]}\PY{p}{,}
                          \PY{n}{df\PYZus{}subset\PYZus{}longer}\PY{p}{[}\PY{l+s+s1}{\PYZsq{}}\PY{l+s+s1}{total\PYZus{}pt}\PY{l+s+s1}{\PYZsq{}}\PY{p}{]}\PY{p}{,} 
                          \PY{n}{equal\PYZus{}var}\PY{o}{=}\PY{k+kc}{False}\PY{p}{)}

\PY{n+nb}{print}\PY{p}{(}\PY{n}{f}\PY{l+s+s2}{\PYZdq{}}\PY{l+s+s2}{t: }\PY{l+s+si}{\PYZob{}results.statistic\PYZcb{}}\PY{l+s+s2}{\PYZdq{}}\PY{p}{)}
\PY{n+nb}{print}\PY{p}{(}\PY{n}{f}\PY{l+s+s2}{\PYZdq{}}\PY{l+s+s2}{p\PYZhy{}value: }\PY{l+s+si}{\PYZob{}results.pvalue\PYZcb{}}\PY{l+s+s2}{\PYZdq{}}\PY{p}{)}
\end{Verbatim}
\end{tcolorbox}

    \begin{Verbatim}[commandchars=\\\{\}]
t: -2.0251625604004104
p-value: 0.042871485628645116
\end{Verbatim}

    \subsubsection*{Part I}\label{part-i}

From the previous part, we see that the \(p\)-value is relatively close
to zero and the \(t\)-statistic is moderately high (in magnitude).
Therefore, we can reject the null that the means are equal in favor of
the alternative that the average total score of back-to-back games is
different than longer games with relatively low \(\alpha\)-level
(\(\alpha\) less than \(0.05\)).

    \subsubsection*{Part J}\label{part-j}

A possible confounding variable in determining if the difference between
the back-to-back games and non-back-to-back games would be regular vs
playoff season. I would expect games would be played more frequently
during the playoffs than in the regular season. We already identified
that there was a difference between regular season and playoffs season
in terms of total points scored.


    % Add a bibliography block to the postdoc
    
    
    
    \end{document}
