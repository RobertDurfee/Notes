\documentclass{article}
\usepackage{tikz}
\usepackage{float}
\usepackage{caption}
\usepackage{subcaption}
\usepackage{enumerate}
\usepackage{amsmath}
\usepackage{bm}
\usepackage{indentfirst}
\usepackage{siunitx}
\usepackage[utf8]{inputenc}
\usepackage{graphicx}
\graphicspath{ {Images/} }
\usepackage{float}
\usepackage{mhchem}
\usepackage{chemfig}
\allowdisplaybreaks

\title{ 7.012 Problem Set 7 }
\author{ Robert Durfee }
\date{ December 6, 2017 }

\begin{document}

\maketitle

\section*{ Question 1 }

\begin{enumerate}[A.]
    \item It is evolutionarily good for a virus that humans can survive
        contraction of the virus because it has a chance to replicate and spread
        to other hosts (as is the overall goal of all viruses).
        
    \item Negative, Single-Stranded RNA: \textbf{Not affected.} The
        transcription and translation occurs in the cytoplasm.

        Double-Stranded RNA: \textbf{Not affected.} The transcription and
        translation occurs in the cytoplasm.
        
        Retrovirus: \textbf{Affected.} The RNA will be reverse transcribed into
        DNA which then infiltrates the chromosomal DNA in the nucleus.
       
        Single-Stranded DNA: \textbf{Affected.} The virus takes advantage of the
        host cell's DNA polymerase for replication within the nucleus.
       
        Double-Stranded DNA: \textbf{Affected.} The virus takes advantage of the
        host cell's DNA polymerase for replication within the nucleus.

    \item The best way to get a viral count in an infected organism is through a
        plaque forming unit assay because the production of viral particles is
        sloppy and often results in non-functioning viral particles. If you
        count the particles by hand using EM, then you would include these
        non-functioning particles in addition to the functioning ones where a
        plaque forming unit assay only counts the functioning viral particles.

    \item Positive, single-stranded RNA viruses make a double-stranded RNA
        intermediate in order to replicate its genome. The negative RNA can then
        be transcribed into the virus' positive, signle-stranded RNA genome.

\end{enumerate}

\section*{ Question 2 }

\begin{enumerate}[A.]
    \item Cancer takes a long time to develop because several mutations are
        necessary. A cell has several functions that keep uncontrolled
        proliferation in-check, such as tumor-suppressing genes, which must also
        acquire mutations to result in cancer. Multiple mutations take time to
        develop.

    \item 
        \begin{enumerate}[i.]
            \item There is a linear relationship because the varius toxins cause
                mutations in cells and both tumors and revertants are caused by
                mutations (mutagenicity is directly linked to carcinogenicity).
                The amount of toxin to cause revertants is thus directly
                correlated to the amount of toxin to cause tumors since it is a
                mutagenizing compound.

            \item \textbf{Toxin A} is the most effective among the mentioned
                toxins as it requires the least amount of toxin to create the
                same revertant and tumor effect.

            \item When the toxin is injested, it is metabolized by the liver in
                the organism. The byproducts can be more or less mutagenic than
                the original toxin. Either way, the compound will be just as
                carcinogenic before and after metabolism. However, the toxin
                will diverge from the curve. In this case, \textbf{Toxin B-2}
                diverged from the curve and its thus altered by the liver
                enzymes.
        
        \end{enumerate}

    \item Amino acid in position 12 in Ras protein is critical for GDP-GTP
        exchange and GTP hydrolysis. By changing the amino acid at this
        position, the binding to GDP/GTP is altered and can no longer function
        properly. The permanent binding of GTP causes a constant proliferation.
        RAS is a proto-oncogene as a mutation causes cancer where a
        tumor-supressor will prevent cancer unless it is mutated.

    \item Gene table:
        \begin{center}
            \begin{tabular}{ c c c c }
                \textbf{Gene} & \textbf{Type} & \textbf{Function} &
                \textbf{ Zygosity } \\
                Gene 1 & Proto-Oncogene & Gain-of-Function & Heterozygous \\
                Gene 2 & Tumor Suppressor & Loss-of-Function & Homozygous 
            \end{tabular}
        \end{center}
\end{enumerate}

\section*{ Question 3 }

\begin{enumerate}[A.]
    \item Axon: \textbf{C}

        Axon Hillock: \textbf{H}

        Cell Body: \textbf{B}

        Dendrite: \textbf{A}

        Nerve Terminal: \textbf{F}

    \item 
        \begin{enumerate}[i.]
            \item Excitatory Neuron: Action potential flows from \textbf{C to E}.

            \item Inhibitory Neuron: Action potential flows from \textbf{C to E}.
        \end{enumerate} 

    \item As the action potential travels down the axon, the section before the
        action potential will have closed sodium channels, the section with the
        action potential will have open sodium channels, and the section the
        action potential just left will have refractory sodium channels. Thus,
        from C to E, if the action potential is peaked at D, the order is:
        \textbf{Refractory, Open, Closed}.

    \item Voltage-Gated Calcium Channels: \textbf{Nerve Terminal}.

        Acetylcholine-Gated Channels: \textbf{Neuromuscular Junction}.

        Acetylcholine Vesicles: \textbf{Nerve Terminal}.

        Voltage-Gated Sodium Channels: \textbf{Axon}.

        Resting Potassium Channels: \textbf{Axon} and \textbf{Dendrites} of
        pre-synaptic neuron.

    \item Action Potentials (Total: Black, Neuron 1: Red, Neuron 2: Green, Neuron 3: Blue.)
        \begin{enumerate}[i.]
            \item Neuron 1: Inhibitory, Neuron 3: Excitory:
                \begin{figure}[H]
                    \centering
                    \includegraphics[scale=1.00]{"Part I"}
                    \caption{Part I}
                \end{figure}

            \item Neurons 1 and 2: Inhibitory, Neuron 3: Either:
                \begin{figure}[H]
                    \centering
                    \begin{subfigure}{.5\textwidth}
                        \centering
                        \includegraphics[width=.75\linewidth]{"Part IIa"}
                        \caption{Part IIa}
                    \end{subfigure}%
                    \begin{subfigure}{.5\textwidth}
                        \centering
                        \includegraphics[width=.75\linewidth]{"Part IIb"}
                        \caption{Part IIb}
                    \end{subfigure}
                    \caption{Part II}
                \end{figure}
          
            \item Neurons 1 and 2: Excitory, Neuron 3: Inhibitory:
                \begin{figure}[H]
                    \centering
                    \includegraphics[scale=1.00]{"Part III"}
                    \caption{Part III}
                \end{figure}
            
            \item Neurons 1, 2, and 3: Excitory:
                \begin{figure}[H]
                    \centering
                    \includegraphics[scale=1.00]{"Part IV"}
                    \caption{Part IV}
                \end{figure}

        \end{enumerate}

    \item Shown above in Figure 3.
\end{enumerate}

\section*{ Question 4 }

\begin{enumerate}[A.]
    \item The gates that control the flow of both negative and positive ions can
        be triggered by various molecules. A glutamate molecule can, for
        example, trigger a sodium channel which will cause sodium to flow into
        the cell.

    \item 
        \begin{enumerate}[i.]
            \item The acetylcholine will not release from its receptor site
                which will result in rigid paralysis as the muscle cannot relax.
                The action potentials will increase.

            \item Calcium channels free up neurotransmitters. Without their
                function, the muscle will not be signalled to contract and the
                action potential will stop at the nerve terminal. The action
                potentials will decrease.

            \item Chloride ions decrease the potential in the cell having an
                inhibitory effect. If the gates are always open, there will be
                less activation. The action potentials will decrease.
        \end{enumerate}
\end{enumerate}

\section*{ Question 5 }

\begin{enumerate}[A.]
    \item After cells produce an antibody to an antigen, they store them in the
        bone marrow. When the antigen appears again, these cells in the bone
        marrow begin proliferating. This is important for vaccines because
        exposure to a weakened form of a disease will cause our body to store
        antibodies for quick retrieval when exposed in the future.

    \item The adaptive immune system uses V(D)J recombination to alter the
        variable sections of an antibody by randomly combining different gene
        segments and by intentionally applying point mutations during this
        process.

    \item Each plasma cell makes its own specific antibody. The majority of
        antibodies found in the blood of a myeloma patient are all the same as
        they originate from a single, constantly proliferating plasma cell
        (which has its own specific antibody).
\end{enumerate}

\section*{ Bonus Question }

I am already familiar with the melody of "Hey Jude" by The Beatles.

\end{document}
