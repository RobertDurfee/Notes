\documentclass{article}
\usepackage{indentfirst}
\usepackage{enumerate}
\usepackage[utf8]{inputenc}

\title{7.012 Problem Set 5}
\author{Robert Durfee}
\date{November 8, 2017}

\begin{document}

\maketitle

\section*{Question 1}

\begin{enumerate}[A.]
    \item Protein Y acts as a repressor for the genes to produce enzymes A and B
        for the amino acid xynine. When xynine is added to the growth medium, it
        binds to the repressor protein Y which allows the protein to bind to the
        DNA and stop enzyme production.
    
    \item If there was a mutation in the gene for $P_{E}$ or in both genes for
        $A$ and $B$, enzymes A and B would be low activity always. Since it was
        specified there is only a single mutation, this must be in $P_{E}$. With
        a defective promoter for both enzymes, neither will be transcribed.
    
    \item If there was a mutation in the genes for $P_{Y}$, $Y$, or $O$, the
        repressor wouldn't be functional, therefore enzymes A and B would be
        high activity always. To create the repressor protein, $P_{Y}$ is needed
        to start transcription and $Y$ is needed for the sequence of the
        protein. It was also stated that $O$ is necessary for regulation by
        protein $Y$, most likely the binding site. Without this, the repressor
        couldn't stop enzyme production.
    
    \item If there was a mutation in the gene for only one enzyme, $A$ or $B$,
        that respective enzyme would have low activity always. In this case,
        that enzyme would be $B$.
    
    \item Now the mutations are limited to genes $P_{Y}$ and $Y$. If there was a
        mutation in either of these, the wild type plasmid could step in and
        create the repressor. 
    
    \item Now the mutation is limited to gene $O$. If there was a mutation in
        $O$, the binding site of the repressor, there would still be activity of
        both enzyme A and B with xynine from the mutated genomic region.
    
    \item Yes, the plasmid would supply the $P_{E}$ for mutant 2 allowing for
        the production of both enzyme A and B in the plasmid-supplemented mutant
        as shown in the second diagram. For mutant 4, the plasmid would supply
        the gene for $B$ (along with the promoter necessary for transcription)
        and allow the production of enzyme B in the plasmid-supplemented mutant
        as shown in the second diagram.
    
\end{enumerate}

\section*{Question 2}

\begin{enumerate}[A.]
    \item The A SNP allele is tightly associated with the disease locus in the
        family of individuals 1 and 2.
    
    \item List of genotypes:
    \begin{center}\begin{tabular}{c c}
        \textbf{Individuals} & \textbf{Genotype} \\
        1 & Dd \\
        2 & Dd \\
        3 & DD \\
        4 & Dd \\
        5 & dd \\
        6 & DD \\
        7 & dd \\
        8 & Dd \\
        9 & dd
    \end{tabular}\end{center}
    
    \item Probability that individual 10 is affected or carrier:
    \begin{center}
        \begin{tabular}{c c c}
            \textbf{Individual} & \textbf{Carrier} & \textbf{Affected} \\
            10 & 50\% & 0\%
        \end{tabular}
    \end{center}
    
    \item A/C or C/C.
    
    \item Genotypes of the expected twins:
    \begin{center}
        \begin{tabular}{c c}
            \textbf{Individual} & \textbf{Genotype} \\
            11 & dd \\
            12 & Dd \\
        \end{tabular}
    \end{center}
    
    \item The difference between the SNP genotypes and the disease phenotypes in
        individuals 3 and 5 can be described by the disease developing in the
        the family of individual 5 and has not spread to become associated with
        that SNP in all genomes.
    
\end{enumerate}

\section*{Question 3}

\begin{enumerate}
    \item Use the \textbf{embryonic stem cells} from a black mouse to knockout a
        gene.
    
    \item Perform a selection with \textbf{both a positive and a negative
        selection marker}.
    
    \item Inject the hemizygous cells into inner cell mass of an embryo of a
        mouse with coat color \textbf{white}. Implant the modified embryo into a
        surrogate and choose chimeric mouse with coat color \textbf{black and
        white}.
    
    \item Breed chimeric mice to white mice. The hemizygous knockout mice will
        have coat color that is \textbf{black}.
\end{enumerate}

\section*{Question 4}

\begin{center}
    \begin{tabular}{c c c}
        & \textbf{Target} & \textbf{Replacement} \\
        \textbf{Stem Cell} & DNA & Yes \\
        \textbf{RNAi} & RNA & No\\
        \textbf{CRISPR} & DNA & Yes
    \end{tabular}
\end{center}

\section*{Question 5}

\begin{enumerate}[A.]
    \item The CRISPR locus of the phage-resistant strains of bacteria contained
        sequences that matched the phage that they were resistant to. The
        ability to identify a previously encountered phage allowed the bacteria
        to target and attack the phage before it could do damage.
    
    \item CRISPR/Cas9 can be used to introduce mutations into diploid cells by
        targeting a certain sequence using single guide RNA. Once this sequence
        is found, Cas9 will cut the DNA there and continue to cut it after
        homologous recombination until it is repaired using NHEJ where there
        will very likely be a frame shift mutation caused by an insertion or
        deletion. Additionally, CRISPR will make an individual homozygous for a
        specific gene.
    
    \item CRISPR/Cas9 is much more efficient as it is easier to do in less time.
        Additionally, it will work more often with fewer complications.
    
    \item Both constructs should work the same. The Cas9 will cause a double
        stranded break at the site the single guide RNA matches to.
    
    \item Using CRISPR/Cas9, you can target any part of the genome: introns,
        exons, and promoters so long as you can match the sequence using single
        guide RNA.
    
    \item 
        \begin{enumerate}[i.]
            \item Exon 2.
            
            \item The amount of protein A found in the cell would be little as
                the the CRISPR cut occurred in an exon and the cell would repair
                the cut and cause a frame shift mutation.
            
            \item No, they would also need to provide an insert that the cell
                could use homologous recombination to insert into the genome at
                the break site created by the Cas9.
        \end{enumerate}
\end{enumerate}

\end{document}
