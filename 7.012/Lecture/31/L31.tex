\documentclass{article}
\usepackage{tikz}
\usepackage{float}
\usepackage{enumerate}
\usepackage{amsmath}
\usepackage{bm}
\usepackage{indentfirst}
\usepackage{siunitx}
\usepackage[utf8]{inputenc}
\usepackage{graphicx}
\graphicspath{ {Images/} }
\usepackage{float}
\usepackage{mhchem}
\usepackage{chemfig}
\allowdisplaybreaks

\title{ 7.012 Lecture 31 }
\author{ Robert Durfee }
\date{ December 4, 2017 }

\begin{document}

\maketitle

\section{ Monoclonal Antibodies }

A \textbf{monoclonal antibody} is a solution of antibody molecules that are all
identical to one another. How do we create this solution? There are different
antibodies in the blood stream and serum. They could be for the same virus or
not and could have varying affinity strengths.

We can \textbf{immortalize} a cell by giving it the power to proliferate
forever. This overcomes the fact that the b-cells will only proliferate for a
short period of time. We do this by taking a b-cell and fuse it to a
\textbf{myloma} cell (which proliferates forever). This is done by injecting the
nucleus into the myloma. In this case, the myloma cell does not create its own
antibody. The hybrid cell will have the oncogene from the myloma and the
antibody production from the b-cell. This is a \textbf{hybridoma}. Then all
these hybridomas are screened for antibody production because there will be
random chromosome loss. The hybridoma that strongly binds with the antigen and
only that antigen will then be isolated for massive proliferation.

A monoclonal antibody can let you know if a specific antigen is present. They
can also be used to highlight specific proteins in microscope images. They can
also be used to target cancer for therapy to reduce recurrence of cancer.

\section{ Variable Domain Creation }

There are gene segments in the cellular genome that encode the sequences found
in the antigen-combining varialbe domains of antibody molecules. Each variable
doam in composed of three subdomains: V, D, and J. These three domains are
present on the genome. A cell will then combine the three components to create a
wide array of different types of variable antibody sequences. Note that there is
a signal sequence prior to the V subdomain to allow the excretion of the
protein. 

Sometimes there are antibodies that are created that recognize our native
proteins. These are then deleted. In addition, the fusion is sloppy and thus
there can be out-of-reading-frame molecules that are incoherent. The immune
system has a method of deleting these as well. 

There is additional diversity from the recombination of different subdomains of
the variable region of the antibody through intentional point mutations. There
are also different constant zones as well. This allows them to have different
functions as well. One of these functions can cause the antibodies to localize
to different locations of the body.

\end{document}

