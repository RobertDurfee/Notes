\documentclass{article} 
\usepackage{tikz} 
\usepackage{float}
\usepackage{enumerate} 
\usepackage{amsmath} 
\usepackage{bm}
\usepackage{indentfirst} 
\usepackage{siunitx} 
\usepackage[utf8]{inputenc}
\usepackage{graphicx} 
\graphicspath{ {Images/} } 
\usepackage{float}
\usepackage{mhchem} 
\usepackage{chemfig} 
\allowdisplaybreaks

\title{ 7.012 Lecture 23 } 
\author{ Robert Durfee } 
\date{ November 8, 2017 }

\begin{document}

\maketitle

\section{ Tissues }

The process of \textbf{differentiation} represents the acquisition of
tissue-specific, specialized traits. As differentitation proceeds, the cell
becomes committed to an increasingly narrow range of developmental options. This
is accomplished by changes in \textbf{gene expression}. This is easy to see in a
microarray, which we have shown before.

The human \textbf{transcriptomes} change across different types of tissues.
\textbf{Transcriptional programs} drive phenotypic differentitaiton.

\section{ Stem Cells }

These cells undergo \textbf{asymetric division}. One daughter cell goes back to
become a stem cell again, and the other creates progeny that start
differentitation. As a result, stems cells are not depleated.  Tissues have
hierarchical organization where at the top there is a self-renewing stem cell.
At the bottom are fully differentitated tissue cells. The cells in between are
\textbf{transit-amplifying cells}. These make up the bulk of proliferation. This
organization allows a single stem cell division to spawn dozens of
differentiated progeny. 

\textbf{Niche-forming cells} are convincing one daughter cell to remain a stem
cell.

Stems cells can make more copies of themselves and can be committed to a
specific direction to a specific differentiation. This committment is
irreversible. The fully differentiated cells are called \textbf{post-mitotic}.

\textbf{Oligopotent stem cells} have multiple different types of differentiated
cells. Examples include cells in bone marrow and the gut. The cells in the gut
are produced at the bottom of a \textbf{crypt} and the progeny move up out of
the crypt. The number of stem cells is kept quite constant at the bottom of the
crypt.

Stem cell genomes are very important and need to be well protected. One way to
do this is to minimize the number of successive divisions that a stem cell
passes during the lifetime.

\subsection{ Colon Cancer Example }

With colon cancer, \textbf{APC protien} becomes mutangenized. This protein is
meant to move progeny out of the crypt. When it becomes mutated, it can't move
the cells. As a result, there becomes a build up of cells, or a \textbf{polyp}.

\subsection{ Hematopoietic System Eliminated in Mice Example }

If you irratiate a mouse, you can remove the entire blood cell creating system.
If you place syngeneic bone marrow, it will being to proliferate rapidly saving
the mouse from death. \textbf{Syngeneic} means that the bone marrow is from a
mouse of the same phenotype (essentially twins).

It was found that there were nodules found on the spleen. Their circular shape
suggests they came from one start cell. We can label the donor cells and implant
them in the mouse. What is found is all the cells carry the same marker from the
donor. This proves that single cells can create very different cell types.

\section{ Embryonic Stem Cells }

\textbf{Embryonic stem cells} can create all different types of cells while
adult stem cells can only do, at most, many differnt tyeps. You can take
embryonic stem cells from the \textbf{inner cell mass}. You can then grow these
cell \textbf{invitro} (in petri dish).

We can test if embryonic stem cells are \textbf{pluripotent} by taking ES cells
from a black mouse and inject them into the \textbf{blastocyst} of a white
mouse. Then, you can observe the effects in the \textbf{chimeric} mouse. There
will be black splotches all over the chimeric mouse. You can then breed a
chimeric mouse with a white mouse, you can end up with mouse that is all black.
This happens when the sexual organs of the chimeric mouse came from the ES cells
injected from the donor.

\subsection{ Application of Eternal Youth }

The goal of ES cell research is to provide cells for customized tissue repair in
damaged or degenerated tissues, including those weakened due to age. This also
includes trying to prevent degenerative diseases such as Alzheimers. If we can
replace these cells, then we can reverse the effects of aging. However, people
are \textbf{histoincompatible}. This means that the immune system will recognize
and eliminate the foreign cells. The only way to get around it is to take cells
from one's own body. Then they are \textbf{histocompatible}.

If you could take cells from other parts of the body and force them into ES
cells, you could create a clone. You can take a regular cell and make it
identical to the fertilized egg by extracting the nucleus from a cell and insert
it into an \textbf{enucleated} egg. Once the nucleus is injected, the signals in
the cytoplasm convince the nucleus to act as a newly fertilized egg. Every once
in awhile, this will work and you can completely clone the animal. This doesn't
always work and they often die early.

Instead of putting the ES cells and put them into another egg, rather, you can
take these cells and force them to differentiate. Now you can create genetically
identical cells of many different tissue types.  These cells would be
histocompatible and not be rejected by the host. If you could do this for all
cells, you can keep injecting young cells and, theoretically live forever. Or,
they can be used to recreate a damaged tissue and repair that organ.

\end{document}

