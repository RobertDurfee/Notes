\documentclass{article}
\usepackage{tikz}
\usepackage{float}
\usepackage{enumerate}
\usepackage{amsmath}
\usepackage{bm}
\usepackage{indentfirst}
\usepackage{siunitx}
\usepackage[utf8]{inputenc}
\usepackage{graphicx}
\graphicspath{ {Images/} }
\usepackage{float}
\usepackage{mhchem}
\usepackage{chemfig}
\allowdisplaybreaks

\title{ 7.012 Lecture 24 }
\author{ Robert Durfee }
\date{ November 13, 2017 }

\begin{document}

\maketitle

\section{ Protein Localization }

The plasma membrane is penetrated by a variety of proteins with transmembrane
domains and various cell-physiologic functions. Some anchor the cell to the
extracellular matrix and some are transporter proteins. These transporter
proteins solve the transportation problems involved in a cell. 

One other feature of the extracellular domains of these various membrane
proteins are \textbf{glycosylated} in that they have sugar side chains
covalently attached to them. This class of proteins are called
\textbf{glyoproteins}. 

\subsection{Post-Translational Import}

This process works for the nucleus, mitochondria, and peroxisomes. The nuclear
localization signal is necessary in a protein sequence. Without it, it will no
longer localize to the nucleus. When you attach it to the protein, it will go to
the nucleus. How do we get to other organelles? How are secreted and
transmembrane proteins made?

\subsection{Co-Translational Translocation}

Once you are tied to the ER, they you are destined to end up in the cytosol or
to be secreted. The rough ER is specialized to create the secreted proteins. In
order for a protein to bind to the ER, it needs a \textbf{signal recognition
particle}. 

\textbf{Translocation channel} allows a protein to enter the ER as it is being
made. This solves the transporation problem. The \textbf{signal sequence} is
then cleaved off and discarded.

These proteins can go into the ER. But what does this do for us? What is the
protein needs to be in the golgi apparatus, needing to be secreted? There is a
flow from the ER to the golgi apparatus. From the golgi, the protein can go to
the transmembrane. This happens by pinching off the membrane and become a
vesicle. 

Once you get to the plasma membrane, another fusion happens. Now this protein is
now secreted and outside the cell. 

Topologically, the inside of the ER is equivalent to the outside of the cell.
The inside of the cell is a reducing environment. However, inside the ER and
outside the cell, is not a reducing environment. 

If there is a signal sequence on the protein, then the protein will be secreted.
What about non-secreted proteins? Along the signal sequence, there is a
\textbf{hydrophobic patch} that prevents the proteins from continuing to be
transfered. Thus, the protein becomes a transmembrane protein. 

This process is required for ER, golgi, plasma membrane, secreted, and lysosomal
proteins to be created. This is \textbf{exocytosis}, but there is another
process of \textbf{endocytosis} where proteins come back into the cell.

When something is bonded transmembrane, which will then be transfered to the
plasma mebrane, the part that is inside the ER will be the part that is outside
the cell. Remember, inside the ER is equivalent to the out side the cell: an
oxidation environment.

\subsection{Endocytosis}

The process of \textbf{exocytosis} works the same in reverse. Whatever is being
taken into the cell is pinched off into a membrane as a vesicle. This is how we
carry cholesterol in our blood: by packaging the fatty things inside the cell. 

\end{document}

