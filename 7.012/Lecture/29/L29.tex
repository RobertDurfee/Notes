\documentclass{article}
\usepackage{tikz}
\usepackage{float}
\usepackage{enumerate}
\usepackage{amsmath}
\usepackage{bm}
\usepackage{indentfirst}
\usepackage{siunitx}
\usepackage[utf8]{inputenc}
\usepackage{graphicx}
\graphicspath{ {Images/} }
\usepackage{float}
\usepackage{mhchem}
\usepackage{chemfig}
\allowdisplaybreaks

\title{ 7.012 Lecture 29 }
\author{ Robert Durfee }
\date{ November 29, 2017 }

\begin{document}

\maketitle

\section{ Review }

The resting potential of the cell is $-70\ \si{ mV }$. There are high
concentrations of $\ce{ Na+ }$, $\ce{ Ca+ }$, and $\ce{ Cl- }$ on the outside of
the cell and high concentration of $\ce{ K+ }$ inside of the cell. There is a
gate that allows potassium to flow back and forth freely. Then there are
voltage-gated channels for sodium and potassium. The sodium channel is then
inactivated for a short period of time to allow forward travel of the action
potential. 

\section{ Patch Clamping }

All the above discussed channels and processes are inferred from differential
equations. However, \textbf{patch clamping} allowed experiemental justification
for the inferences. This method takes a small patch of membrane and attaches it
to a glass pipet. If the membrane patches are small enough, you can isolate a
single channel. There will be a change in voltage. If there are more than one
channels, this can be viewed as well.

\section{ Signaling Between Neurons }

In this case, we will examine a nerve connected to a muscle fiber. The nerve
terminal is thus connected to the muscle fiber. A signal will traverse the axon
to the nerve terminal. When the action potential reaches the nerve terminal,
there are little membrane vesicles called \textbf{synaptic vesicles} which
contain little \textbf{neurotransmitters}. Then synaptic vesicles, which were
tethered into place, come loose and merge with the membrane of the nerve
terminal and spill out. The protein that holds the vesicles in place is called
\textbf{synapsin}. 

A \textbf{voltage-gated calcium channel} opens as the action potential
reaches it. The calcium will now enter the cell due to concentration difference.
There will also be a \textbf{calcium-dependent protein kinase} which, once
activated, will put a phosphate onto the synapsin, releasing the vesicles
containing the neurotransmitters. 

\section{ Chemical Synapses }

The neurotransmitters travel across the \textbf{neuro-muscular junction} when
the calcium flows in. For the muscle-nerve system, the neurotransmitter is
\textbf{acetylcholine}. On the muscle, there is an \textbf{acetylcholine-gated
sodium channel}. When it lands on the post-synaptic cell, it binds with the
channel, causing sodium to rush in and the action potential mechanism will then
follow. 

Now, the neurotransmitters must be released from the channels causing muscle
constriction. This is accomplished through a \textbf{re-uptake mechanism}.
However, this also needs to happen quickly, thus the other posssibility is to
\textbf{degrade} the acetylcholine using an enzyme,
\textbf{acetycholinesterase}. 

\section{ Toxins }

This is a targettable process which, if goes wrong, can go really wrong. Thus,
there are some \textbf{toxins} that affect the neural system. For example,
\textbf{tetrototoxin}, found in Fugu fish, targets voltage gated sodium channel.
\textbf{Curare} is another toxin used in South American arrows which binds to
the \textbf{acetylcholine receptor}, preventing it from opening. This causes
\textbf{flaccid paralysis} where no muscles can flex. \textbf{Alphabungarow
toxin} irreversibly binds to AChR and prevents the binding of ACh. This would
also cause a flaccid paralysis. \textbf{Sarin gas} is a human-made neurotoxin
which inhibits the acetylcholinesterase enzyme. This causes \textbf{rigid
paralysis}. 

\section{ Nerve-Nerve Synapses }

Now we will take a look at the dendrite side of the neuron. The neurotransmitter
will reach the dendrite and activate a channel. The channel then allows an ion
(perhaps sodium) to enter. This causes the inside of the cell to be slightly
more positive at that spot. However, an action potential does not fire. 

Glutamate will activate sodium and potassium channels and glycine can open
chlorine channels. Chlorine, however, will lowed the charge in the cell instead
of raising it. But this doesn't cause an action potential. The \textbf{axon
hillock} is the point at which the potential needs to change. 

\end{document}

