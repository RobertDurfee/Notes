\documentclass{article}
\usepackage{tikz}
\usepackage{float}
\usepackage{enumerate}
\usepackage{amsmath}
\usepackage{bm}
\usepackage{indentfirst}
\usepackage{siunitx}
\usepackage[utf8]{inputenc}
\usepackage{graphicx}
\graphicspath{ {Images/} }
\usepackage{float}
\usepackage{mhchem}
\usepackage{chemfig}
\allowdisplaybreaks

\title{ 7.012 Lecture 35 }
\author{ Robert Durfee }
\date{ December 13, 2017 }

\begin{document}

\maketitle

\section{ Societal Implications }

\subsection{ Genome Sequencing }

Should it become regular practice to sequence everyone's genome? Right now,
babies are tested for a single genetic disorder that results in the inability to
break down phenylalanine. This leads to a build up of the amino acid in the
brain and causes mental retardation. This is common practice, along with a few
other genetic disorders. If we tested everyone, we could learn so much and
advance medicine.

We should not do genetic testing on babies because babies cannot give consent.
As a result, testing for research reasons should not be done. Also, there is a
privacy issue where we can use this data to predict certain things about an
individual. How can we protect this highly important information.

We can change the baby testing age to the age of majority: 18. Then, the people
can give adequate consent. However, we may not want to know about certain
disorders if they are not preventable. This could also lead to discrimination
through careers and insurance.

What if a company paid you to sequence your genome? We already get paid for
other medicine information, so why not this information? Even in all the
personal information is scrubbed, it doesn't take much additional data to
determine who you are.

\subsection{CRISPR}

Should we be allowed to edit the genome of babies using CRISPR? We could improve
and perfect the human genome by eliminating diseases from the germ line. This
could lead to a lack of genetic diversity. There could become a social norm of
the best types of traits that should be expressed. This could lead to
discrimination too as poor people would be unable to afford the procedure. This
would greatly dividing social classes.

Furthermore, reducing diversity could lead to health problems. Viruses targeting
certain genes could knock out the entire population. Additionally, by knocking
out certain genes, we might not know what side-effects result from eliminating
certain genes.

Once we edit the genomes, we don't know how these editions will work with each
other. This can lead to potentially disasterous results. But we already have
this in nature, so why is this any different? There is a question of
responsibility.

There is also the question of logistics. Will there be a cost for this
procedure? Most likely yes. This will lead to profit motives. There is already a
similar problem in the pharmaceutical industry right now.

\end{document}

