\documentclass{article}
\usepackage{tikz}
\usepackage{float}
\usepackage{enumerate}
\usepackage{amsmath}
\usepackage{bm}
\usepackage{indentfirst}
\usepackage{siunitx}
\usepackage[utf8]{inputenc}
\usepackage{graphicx}
\graphicspath{ {Images/} }
\usepackage{float}
\usepackage{mhchem}
\usepackage{chemfig}
\allowdisplaybreaks

\title{ 7.012 Lecture 32 }
\author{ Robert Durfee }
\date{ December 6, 2017 }

\begin{document}

\maketitle

\section{ Heart Disease }

Cardiovascular disease currently leads to 800,000 deaths. These are casued by
plaques forming in the blood stream, eventually blocking certain blood vessels.

\section{ Cholesterol }

The villian that causes these plaques to form. But not all cholesterol is bad.
It is simply a molecule. The structure of cholesterol is very hydrophobic. Or,
in other words, it is a waxy substance. In order to work with it, you turn it
into an ester so it is more soluble.

Cholesterol plays a structural role in plasma membranes. In a typically lipid
bilayer, half of the molecules are cholesterol. So it can't be evil. It is very
needed. It is an important precusor for pathways of steroids, vitamins, and bile
acids. Bile acids help to imulsify fats and helps take it up. This bile acid is
also recycled. 

Cholesterol comes from our diet. For example, eggs, milk, animal fat, etc. all
have animal fat. In addition, cholesterol is produced by the body. The first
component of the complex pathway for cholesterol synthesis is acetic acid. After
several steps, HMGCoA Reductase converts HMGCoA to mevalonate.

\section{ Lipoprotein Particles }

Cholesterol needs to be packaged in order to be soluble in the blood stream.
VLDL stands for very low denstity lipoprotein, LDL for low density, IDL for
intermediate density, and HDL for high density. These all have different
properties.

For LDL, there is a lipid monolayer with cholesterol ester inside with a ApoB100
protein attached within the lipid monolayer. The inside is highly hydrophobic. 

\section{ Connection to Heart Disease }

Cholesterol is at the location of the cause of a heart attack. But, this
correlation does not mean causation. As a result, an experiment was conducted:
high amounts of cholesterol were fed to rabbits. This induced athrosclerosis.
But, this doesn't necessarily carry over to humans. 

Epidemeology begins to look at humans. This involved looking at the same people
for a very long time. This showed that the level of LDL in blood was directly
correlated to heart attack. On the other hand, the level of HDL is indirectly
correlated to heart attack. This still doesn't mean LDL causes heart attack. 

\section{ Genetics of Cholesterol }

There were some patients isolated with extremely high levels of LDL. They seemed
to have a disease: familial hypercholesterolemia. Normal individuals had LDL 100
$\si{ mg\ dL^{-1} }$ and heterozygotes had LDL 250 $\si{ mg\ dL^{-1} }$ and
homozygotes had LDL 600 $\si{ mg\ dL^{-1} }$. The homozygotes got heart
disease in their teens. 

The found that the LDL receptor gene was mutated in the homozygote individuals.
When cholesterol was included in small amounts with cell growth, LDL and HMGCoA
production was increased. With large amounts, both went down.

\section{ Rational Therapy for Heart Disease }

What if we just consume less cholesterol? This will decrease LDL levels (about
10\%). This is not reduced nearly as much because there is a secondary method
for producing LDL.

If we prevent the bile recycling, we can also reduce LDL (about 20\%). This is
not reduced much because the body still will produce LDL.

If we inhibit the HMGCoA Reductase enzyme, there will be drastic reduction of
LDL particles (about 60\%). But this is a correlation still. There is still no
proof of causation. These statins were givien to people who had heart attacks
and those with the drug had a lower risk of heart attack. This truly works.

On the other hand, HDL has no effect on heart attack. This was merely as the
result of correlation.

\end{document}

