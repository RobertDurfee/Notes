\documentclass{article}
\usepackage{tikz}
\usepackage{float}
\usepackage{enumerate}
\usepackage{amsmath}
\usepackage{bm}
\usepackage{indentfirst}
\usepackage{siunitx}
\usepackage[utf8]{inputenc}
\usepackage{graphicx}
\graphicspath{ {Images/} }
\usepackage{float}
\usepackage{mhchem}
\usepackage{chemfig}
\allowdisplaybreaks

\title{ 7.012 Lecture 26 }
\author{ Robert Durfee }
\date{ November 17, 2017 }

\begin{document}

\maketitle

\section{ Transforming Viruses }

How do viral genomes get passed from a mother to daughter cell systematically
during cell mitosis? As a cell replicates, if the viral DNA is not associated
with the genome, it will be lost.

\subsection{Integrating Viral Genome}

Using a \textbf{sucrose gradient}, you can isolate the cell genome and the viral
DNA sequences. In the sucrose gradient, the double strands are broken up into
signle strands. Since the viral DNA did not seperate, it must be covalently
bonded to the cell's genome. 

Some viruses have genomes of ssRNA and still perpetuate. This is accomplished by
the use of reverse transcriptase. Thus the virus converts the ssRNA into DNA and
then is integrated. Then the virus genome can be transcribed into mRNA and
transcribed into the proteins necessary for the virus to develop. The mRNA can
be integrated into the virus as its genome.

\subsection{Creating Oncogenes}

But how can the viral genome transform the cell? The transforming virus has an
additional segment in its genome which can transform the transfected cell. This
was the result of a rare genelogical event.

It was expected that the SRC oncogene probe would only hybridize to those
chickens infected with the RSV. However, it was also hybridized to normal
chicken DNA.  This probe also hybridized to related birds such as the quail,
turkey, duck, and emu.

This shows that there is a \textbf{proto-oncogene} present in the DNA of
chickens and other birds and mammals (and even all animals). Thus, the RSV got
the oncogene from the genome of the chicken. This proto-oncogene is used in the
normal operation of the cell. However, it can be altered in such a way to make
the cell cancerous. The roots of cancer comes from perverting normal genes in an
organism, not by bringing in foreign information. 

\subsection{Different Oncogenes}

There are some cancers that rise from point mutations, such as the Ras protein.
These make sense beccause they alter the signal pathway for proliferation. But
there are also some cancers that rise from large deletions of genes that are
\textbf{cancer suppressing}.

For individuals where the cancer suppressing gene is heterozygously mutated,
then they have a higher risk of cancer in other cells because they might also
rely on the tumor suppressing functions of the retina protein. There are also
many other tumor suppressing genes.

Often, when looking at cancer, there are typically point mutations as well as
multiple tumor suppressing genes knocked out. Why are there many mutations
neccesary for cancer to develop? Since we go through so many divisions, we need
several successive mutations to overwhelm our defensive mechanisms.

\end{document}

