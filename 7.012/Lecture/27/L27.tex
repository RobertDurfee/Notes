\documentclass{article}
\usepackage{tikz}
\usepackage{float}
\usepackage{enumerate}
\usepackage{amsmath}
\usepackage{bm}
\usepackage{indentfirst}
\usepackage{siunitx}
\usepackage[utf8]{inputenc}
\usepackage{graphicx}
\graphicspath{ {Images/} }
\usepackage{float}
\usepackage{mhchem}
\usepackage{chemfig}
\allowdisplaybreaks

\title{ 7.012 Lecture 27 }
\author{ Robert Durfee }
\date{ November 22, 2017 }

\begin{document}

\maketitle

\section{ Virology }

The basic scheme: The virus first attaches to the cell. Then it penetrates the
cell membrane and injects nucleic acids. Then the nucleic acid replicates using
host cellular machinery. Lastly, new viral nucleic acids are packaged into viral
particles and released from the cell. The cell may or may not be destroyed in
this process.

Viruses are much smaller than the cell that they infect. Because they have
smaller genomes, they can be smaller and replicate faster. With the small
genome, how can they encapsulate their genome? They use simple genometry.

\subsection{Physical Properties}

The simplest form of geometry is \textbf{helical}. Another type of symmetry is
\textbf{icosahedral}, which is the symmetry used with human papilloma virus. The
proteins involved with these capsals can self-assemble.

The proteins that encapsulate the genome protect the viral genome, enable the
particle to attach to the surface, and inject the genome into the host.

Many viruses also have a membrane that surround the virus's genome. The inner
portion is called the nucleocapsid. This often includes reverse transcriptase in
the case where the virus's genome is ssRNA. On the outside of the membrane,
there are \textbf{glycoprotein spikes}. These spikes allow \textbf{adsorbing} to
specfic cell proteins.

\subsection{Biological Properties}

A \textbf{plaque} is a cell that has died as the result of virus infection. This
is done as a result of a \textbf{cytopathic effect}. This basically means that
the cell becomes very sick. Using the number of plaques created and give a good
idea of how many viruses are in a virus solution. You can't simply count them,
because that number is basically meaningless. This is because there may need to
be 100 viruses to cause just one plaque. Many of the viral particles could be
inactive because they are not correctly formed.

Viral growth first has an \textbf{eclipse period} where there is no increase in
number of particles. Then, there are exponentially increasing viral particles
within the cell. Shortly after, the number of viral particles outside the cell
will increase exponentially.

Viruses with plasma membranes push their way through the host cell's plasma
membrane. As it does so, it envelopes itself with the host cell's plasma mebrane
as its own. It acquires its glycoprotein spikes by hijacking the host cell's
golgi apparatus.

\subsection{Types}

\textbf{Class I}: This type has a dsDNA genome. It uses the host cell's DNA and
RNA polymerase to replicate and create mRNA. Thus, it limits the number of
enzymes needed to bring into the cell and carry with its own genome. In this
class, there is almost total parasitism on the host cell for DNA, RNA, and
protein synthesis. However, most host cells are not undergoing proliferation.
This is a problem for this type of virus, because its dsDNA genome cannot be
replicated because there is no DNA polymerase. Thus, it needs to persuade the
cell to undergo proliferation. As a result, the virus must include oncogenes to
cause the cell to proliferate.

\textbf{Class II}: This type has ssDNA genome. It's first step is to convert the
ssDNA into dsDNA. Once they do this, they can use the cell's DNA and RNA
polymerase for DNA, mRNA, and protein synthesis.

\textbf{Class III}: This type has dsRNA genome. It must include it's own RNA
polymerase enzyme to replicate it's own genome and create mRNA. Once the mRNA is
made, it can serve to make protein and serves as a template for the dsRNA which
must be included in the virus particle's prodgeny. 

\textbf{Class IV}: This type has (+)ssRNA genome. In this case, the virus must
encode it's own RNA polymerase. First the ssRNA is translated by the host cell.
This translation process then creates the RNA polymerase needed to replicate the
virus's genome. 

\textbf{Class V}: This type has (-)ssRNA genome. This strand is complementary
to the viral genome. Thus, they first need to create the (+)ssRNA for their
genome. They must also encode their own RNA polymerase. 

\end{document}
