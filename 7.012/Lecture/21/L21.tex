\documentclass{article}
\usepackage{indentfirst}
\usepackage[utf8]{inputenc}

\title{7.012 Lecture 21 - Genomics}
\author{Robert Durfee}
\date{November 2017}

\begin{document}

\maketitle

\section{Introduction}

Today we are going to complete the triangle. In the past, scientists new the codons, but they couldn't read the gene. Then came recombinant DNA. They could clone by complementation to find a gene. That allowed us to go from function to gene. For example, looking for hemoglobin, we could go looking for the amino acid sequence. We could use an antibody that matched that protein to go from protein to gene. This is done by protein expression. But we are unable to take a given gene and find a function.

You have to have a mutant in order to be sure you know the function of the gene. You can apply a mutantgenizing force, but how do you know that you only mutantgenize that specific gene. The way we go from gene to function is to knock out a gene in a directed fashion. This needs to be done in experimental subjects (not humans, unethical).

\section{Adding Genes}

\subsection{Plasmid}

You can put your gene of interest into a plasmid. Then you can put it back into your organism. Now your organism carries a plasmid with that gene. But it doesn't destroy the native gene. You can do this with bacteria or yeast.

\subsection{Fertilized Egg}

In mice, you can take a fertilized mice egg where the sperm has just combined with the egg. You can take a pipet and put DNA into the egg. Then you can put the egg into a pseudo-pregnant female.

\subsection{Embryonic Stem Cells}

There is an even cooler method. You can use embryonic stem (ES) cells. If you take an early stage embryo, and remove cells from the inner cell mass and grow them in a petri dish. If you take these cells and inject them later, they will contribute to the embryo. Now you can take cells out and manipulate them in the laboratory and put them back into the library.

But how do you know where your cells are in the new mouse? You can take ES cells from a black mouse and put them in a white mouse. Now there will be black splotches. These are called \textbf{chimera}.

\section{Subtracting Genes}

\subsection{Whole Organism Knockout}

Suppose you have a gene you want to knock out. Using recombinant DNA techniques, I can synthesis a piece of DNA with a mutation. Transfect this into the cell. But the cell will either destroy it or put it in a random spot. It won't go to the right spot. 

How can we select to ensure that piece of DNA was retained by the cell? We can interrupt the gene with an antibiotic resistant gene. Therefore, anything without the interrupted gene will die.

But how do we know it was put in the right spot? You can add a split it with a positive selector and a negative selector. So you look for the positive marker and not the negative market. Where this happened, almost always the correct recombination event occurred.

These knockout mice are usually dead. They die in the early embryonic stages. Then you have to determine the genotype to show that it was the homozygous knockout mice that died.

\subsection{Tissue Specific Knockout}

I am not going to mess with the gene itself. I will enter a sequence before and after the gene (the same sequence). These are \textbf{lox} sites. The \textbf{cre} protein will cause two consecutive lox sites to recombine and kick out the gene in between. You can now kick out by turning on the cre protein only in certain cell types. All you have to do is specify the cre gene in a cell-specific promoter. Now the gene will be knocked out in only that specific cell type.

\subsection{Localizing Protein}

You can find the protein by making the fluorescent by expressing the gene for it with the protein.

\section{RNA Interference}

If you want, you can knockout not just the DNA, but the RNA. IF you have a cell producing RNA from you gene, you can make an oligonucleotide matching sequence. If you have complementary sequences and inject them together, they will bind together creating dsRNA. Now, the cell will go crazy because it thinks it is being injected with a dsRNA virus. Thus, there was a strong knockout effect.

\section{CRISPR}

Restriction enzymes are another type of defense mechanism. They look for a sequence of 4-8 letters and cuts the DNA. There is another way.

CRISPR is a programmable restriction enzyme. It has a guide RNA of about 20 bases and it runs around looking for matching DNA sequence. Since it has so many bases, it is pretty specific.

\subsection{Mojica}

A scientist noticed bacteria had a specific set of sequence clusters of regularly interspaced palindromic repeats (CRISPR). This is present in many types of bacteria. The spacers matched a pathogen that infected the bacteria. These matches allowed it to be immune to that pathogen. This is a immune defensive system.

The small repeat sequences would be cut out and small guide RNA was generated. Then \textbf{Cas9} goes around looking for foreign DNA matching that sequence. By doing several tricks, you can do this with eukaryotic cells.

Now you can take some cells, transfect them with Cas9 and search the genome and find the sequence and cut. Then the cell will chew the ends and reconnect. You can also specify the sequence you want to be added there. You no longer need embryonic stem cells. Now you can edit genomes.

\subsection{CRISPR Library Screening}

You can grow a bunch of cells each with a different guide RNA. Then select for cells that are still alive. These are immune to what you are using to select with.

\subsection{Dead Cas9}

You can take dead Cas9 which can find but not cut DNA. This will allow delivery of proteins to specific genes.

\end{document}
