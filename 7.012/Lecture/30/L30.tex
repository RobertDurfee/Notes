\documentclass{article}
\usepackage{tikz}
\usepackage{float}
\usepackage{enumerate}
\usepackage{amsmath}
\usepackage{bm}
\usepackage{indentfirst}
\usepackage{siunitx}
\usepackage[utf8]{inputenc}
\usepackage{graphicx}
\graphicspath{ {Images/} }
\usepackage{float}
\usepackage{mhchem}
\usepackage{chemfig}
\allowdisplaybreaks

\title{ 7.012 Lecture 30 }
\author{ Robert Durfee }
\date{ December 1, 2017 }

\begin{document}

\maketitle

\section{ Immunology }

The immune system has many different "arms". We will focus on its
\textbf{humoral} and \textbf{cellular} arms. The humoral arms deal with soluble
substances and the cellular arm refers to the cellular response. 

The immune system is used for preventing and even irradicating diseases. We use
\textbf{vaccinations} to expose the immune system to a weakened disease so the
immune system can learn to combat it. Edward Jenner found the first vaccination
when he noticed that taking cow pox and injecting it into another person
prevented them from getting small pox.

\subsection{ Antibodies }

Our body produces \textbf{antibodies} which can recognize foreign protein
\textbf{oligopeptide} sequences. The virus-binding antibodies contain a serum,
\textbf{antiserum}, which neutralizes the virus by coating the virus.
\textbf{Macrophages} can also be eliminated by taking the viruses into the cell
through \textbf{endocytosis}. 

If we inject a mouse with a virus, we can note the amount of antigens that are
produced increases with time. The second time the mouse is injected with the
virus, the concentration of antigens increases faster than before.

\textbf{Antibody titer} is the effectiveness of the antibody to combat a virus.
This can be measured by the concentration of antibody serum needed to neutralize
a constant amount of viral particles. Using this method, it is clear to see that
the serum is dependent upon the virus itself. 

An antibody molecule is Y-shaped with two different binding sites for viral
particles. The top of the Y are the binding sites for the \textbf{antigens}. For
our class, we will refere to an antigen as a particle that provokes an immune
response. The two binding sites on the antibody can be different from each
other. 

\textbf{Epitopes} are sites that can function as antigens. Many
proteins in our bodies have these epitopes, but our body is familiar with them
so it will not attack the antigens. 

Since the antibody is made up of several different chains with some heavy and
some light. These chains have both constant and variable regions. It is actually
quite difficult to create an antigen binding site as two chains need to
correlate their variable sequences to the same antigen. Remember, the antibody
is \textbf{bivalent} meaning that it can have two functions. 

\subsection{ Antibody Production }

Looking at the results of cancer in plasma cells, which are monoclonal, we can
see that a single plasma cell only creates a single antibody.

In a normal immune system, there are \textbf{B cells} (blood cells in the bone
marrow) that remain quite and not proliferating. Each one has its own antibody
that it produces. Now, when exposed to an antigen, the cell will begin
proliferating. This is very similar to what happens in multiple myeloma (cancer
causing blood cell proliferation). However, instead of being triggered by an
oncogene, it is triggered by an antigen. 

There are two decsendents of the B cell proliferation. The first creates
antibodies, but the second kind don't create antibodies right away. Instead, the
hide out in the bone marrow to remember the virus for the future should there be
a secondary infection.

\end{document}

