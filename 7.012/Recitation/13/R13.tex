\documentclass{article}
\usepackage{tikz}
\usepackage{indentfirst}
\usepackage{graphicx}
\graphicspath{ {Images/} }
\usepackage{float}
\usepackage[utf8]{inputenc}

\title{7.012 Recitation 13}
\author{Robert Durfee}
\date{November 2, 2017}

\begin{document}

\maketitle

\section{Microarrays}

\textbf{Microarrays} are, in general, a high throughput method of detecting a
certain sequence within a sample. You can do many sequences at the same time in
an efficient manner. This can be used to look at mRNA expression and detect the
variants of the alleles in the genome. 

A microarray involves a chip with a bunch of different position where each
position has a bunch of complementary DNA sequences for what you are looking
for. These are called \textbf{probes}. Each probe is known at each given
location. You will wash your labeled sample over these probes and see where the
array lights up. Pieces of your sample will bind to a probe that they are
complementary. Because they are fluorescently labelled, you will be able to see
this light up. This will allow you to see which sequences are in your sample and
how much is present as well.

\subsection{Single Nucleotide Polymorphism Example}

The probes you will use are directly adjacent to your SNP, not your SNP
directly. This means that if you have a SNP at a specific location on the
genome, the probe is going to be complementary right next to it.

You will first fragment your genome. Then you will wash it over your chip. Then
you will add DNA polymerase and fluorescently labelled ddNTPs. This will extend
beyond the probe by one base and determine the base of a given SNP.

For a homozygote, you will see all the same bases light up. In a heterozygote,
you will see both different bases light up. 

\subsection{Cancerous Liver Cell RNA Expression Example}

You prepare probes for what you expect to be transcribed in a liver cell. Then
you will add cDNA for both cancerous and non-cancerous liver cells on the same
microarray. The cancerous and non-cancerous cDNAs will be labelled two different
colors. Therefore, in certain places, you will only get binding with cancerous,
others will only bind with the non-cancerous probe, others will get both, and
some will get none.

\section{Transposons}

A \textbf{transposon} is a genetic element that can move or copy itself into
other regions of the genome. This means that you can have a transposon on
Chromosome I and it may be able to cut itself and makes its way into Chromosome
II. Now there are two copies, one on each chromosome.

\section{Retrotransposon}

A \textbf{retrotransposon} copies itself via an RNA intermediate.

\subsection{Long Interspersed Nuclear Elements}

\textbf{Long interspersed nuclear elements} (LINEs) have the sequence for
reverse transcriptase. These typically favor reverse transcribing their own
sequence. They will produce cDNA which can then insert back into the genome. 

\subsection{Short Interspersed Nuclear Elements}

\textbf{Short interspersed nuclear elements} (SINEs) do not have the code for
reverse transcriptase. As a result, they need a LINE to convert them into cDNA
which can then be reinserted into the genome.

\section{Questions}

\begin{enumerate}

    \item \textbf{What do the bright spots on a microarray indicate?}
    
    The bright spots are places where there are a lot of sequences that match
    the probe sequences.
    
    \item \textbf{How might the pattern be different if the same microarray was
        probed with mRNA from a different tissue?}
    
    There will be some areas that still light up where mRNA is the same in the
    two different types of cells. There will also be more dark areas where the
    mRNA is different between the two types of cells.
    
\end{enumerate}

\end{document}
