\documentclass{article}
\usepackage{tikz}
\usepackage{float}
\usepackage{enumerate}
\usepackage{amsmath}
\usepackage{bm}
\usepackage{indentfirst}
\usepackage{siunitx}
\usepackage[utf8]{inputenc}
\usepackage{graphicx}
\graphicspath{ {Images/} }
\usepackage{float}
\usepackage{mhchem}
\usepackage{chemfig}
\allowdisplaybreaks

\title{ 7.012 Recitation 16 }
\author{ Robert Durfee }
\date{ November 14, 2017 }

\begin{document}

\maketitle

\section{ Protein Localization }

This is important because we need to localize proteins to appropriate
compartments in eukaryotic cells for proper function. We can accomplish this in
multiple different ways. The first, most basic way, is through signal sequences. 

A \textbf{signal peptide} is an N-terminal signal sequence that tells the cell
whether to make the protein in the cytoplasm or the ER. If this is present, the
translation occurs in the ER and it if is not present, translation occurs in the
cytoplasm.

A \textbf{signal recognition particle} (SRP) pauses translation. Then the
ribosome and mRNA are brought to the ER by the SRP. Once they reach the ER,
translation starts again and the protein is synthesized into the ER through a
pathway.

Once the protein is fully synthesized into the ER, the signal sequence will be
cleaved off and you will have a free-floating protein in the ER. This is for a
protein that is soluble in the lumen of the ER.

Some proteins have hydrophobic areas that don't want to be in the ER lumen. Then
the signal sequence is not cleaved off and you have a membrane-associated 
protein with transmembrane regions.

It is important to remember that the lumen of the ER is \textbf{topologically
equivalent} to the outside of the cell's plasma membrane. 

\section{Secretory Pathway}

This only applies to proteins synthesized into the ER. Once these proteins are
synthesized into the ER, the ER will send \textbf{vesicles} containing proteins
to the golgi apparatus. These vesicles are essentially small, transient ER
compartments. Once in the golgi apparatus, there are \textbf{covalent
modifications} made to the protein such as additions of sugars called
\textbf{glycosylation}.

After the vesicle leaves the golgi, it will reach the plasma membrane and fuse
expelled the protein outside the cell. Now that protein has been secreted. 

For a transmembrane protein, the same parts of the protein that are in the lumen
of the ER will be the parts that are outside the cell. This is because of the
topological equivalency of the lumen of the ER and the outside of the cell.

If a protein only has a signal peptide, that protein will be secreted (or end up
in the plasma membrane if it is a transmembrane protein). If proteins want to end
up in the ER, golgi, or lysosomes, you need a signal peptide and an
ER/golgi/lysosome retention signal to allow the protein to be localized to any
of those respective locations.

If a protein does not have a signal peptide, the protein will be synthesized in
the cytosol. In that case, the protein can localize to the cytoplasm or it can
localize to other organelles. If the protein has a \textbf{nulclear localization
signal}, that protein will localize to the nucleus. If ths protein has a
\textbf{mitochondrial localization signal}, it will localize to the
mitochondria.

\end{document}

