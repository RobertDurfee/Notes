\documentclass{article}
\usepackage{tikz}
\usepackage{float}
\usepackage{enumerate}
\usepackage{amsmath}
\usepackage{bm}
\usepackage{indentfirst}
\usepackage{siunitx}
\usepackage[utf8]{inputenc}
\usepackage{graphicx}
\graphicspath{ {Images/} }
\usepackage{float}
\usepackage{mhchem}
\usepackage{chemfig}
\allowdisplaybreaks

\title{ 7.012 Recitation 17 }
\author{ Robert Durfee }
\date{ November 16, 2017 }

\begin{document}

\maketitle

\section{ Reducing Gene Expression }

\subsection{CRISPR}

CRISPR allows you to knockout a gene by using the Cas9 protein and a single
guide RNA. The Cas9 cuts the DNA and the single guide RNA allows the target
sequence to be located. Or you can use a CRISPR RNA (which tells the Cas9 where
to cut) and the tracer RNA (which associates with the Cas9 protein).

CRISPR also allows you to "knock-in" a gene by replacing a gene and its
endogenous DNA locus (its normal copy in the genome). You need Cas9, a repair
template, and the single guide RNA (the the crRNA and trRNA described above).

\subsection{Embryonic Stem Cell}

You take ES cells (from a black mouse) and knockout a gene by inserting a
template with antibiotic resistance which is inbetween two homologous regions to
the genome around the gene you wish to alter. Outside the homology region, there
will be a negative marker. If improper template introduction, the negative
marker will also be incorporated. Therefore, we test by treating with antibiotic
for those who have the integration and negative test to make sure it was
integrated properly.

Then the KO ES cells are injected into the blastocyst of another (white) mouse.
Then the mouse grows up into a chimeric mouse with regions of both genomes
(black and white). Then this mouse is bred with a white mouse and we look for
the mice that are black because they are incorporated the ES cells in the germ
line. These are hemizygotes. (In this case, black is dominant to white.) Breed
these together to get homozygotes.

\subsection{Conditional Knock Out}

\textbf{Conditional KO} allow you to knockout a gene only in the type of cell
you want. This is accomplished using a \textbf{cre-lox} system. First there is
the cre gene that, when expressed, will cause recombination at loxP sites
flanking the gene you are looking for and expell the gene between the loxP sites
(which is then degraded). Thus, we can put the cre gene under expression in
certain types of cells (for example prepending a neural promoter). The same
steps as above in ES cells are used, but you reinsert the gene with loxP sites
flanking the gene (along with the antibiotic resistance used for selection).

\subsection{RNA Interference}

You introduce double stranded RNA which will be chopped up in the cell into
several small segments of dsRNA which will bind to the target strands of RNA
which will then be degraded or block translation. This process "knocks down" the
expression as expressiong is not completely eliminated.

\section{Cell Signalling}

Typically, with heterozygous mutation, normal function is dominate. This is
because there is some normal and some abnormal protein. Loss-of-function allele
tends to be recessive.

\section{Protein Localization}

The first decision made in localization is whether the protein is synthesized in
the cytosol or the ER by recognizing a signal peptide. The SRP will recognize
this signal peptide and for the protein synthesis to stop and resume when it
reaches the ER. Once the protein reaches the ER, the default is to have the
protein secreted. However, this protein can have an ER retention signal and stay
in the ER. This protein can also have a golgi retention signal or a lysosomal
retention signal. It is also important to not that all these proteins can be
transmembrane or soluble.

If the cell is translated in the cytosol, it can stay in the cytosol, go to
mitochondria, or the nucleus. These are controlled by localization signals.

\section{Operons}

Cis means that the regulation component must be on the same strand as the
regulated genes. Trans means the regulation component can be anywhere in the
genome. For example, a regulatory protein can be created anywhere in the genome,
but the operon sequence it binds to must be right before the gene.

\section{SNP}

A single nucleotide pair where there is variation within a population. This can
be used as a marker for genetic disease if there is very low recombination
between it an the gene for the disease. 

\section{Stem Cells}

Read stem cell pdf on Secret of Life.

\end{document}

