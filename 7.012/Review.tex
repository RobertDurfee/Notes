\documentclass{article}
\usepackage{tikz}
\usepackage{float}
\usepackage{enumerate}
\usepackage{amsmath}
\usepackage{bm}
\usepackage{indentfirst}
\usepackage{siunitx}
\usepackage[utf8]{inputenc}
\usepackage{graphicx}
\graphicspath{ {Images/} }
\usepackage{float}
\usepackage{mhchem}
\usepackage{chemfig}
\allowdisplaybreaks

\title{ 7.012 Review I }
\author{ Robert Durfee }
\date{ December 14, 2017 }

\begin{document}

\maketitle

\section{ Gene Linkage }

$$ F_{0}: \mathrm{AABB \times aabb} $$
$$ F_{1}: \mathrm{AaBb \times aabb} $$

\begin{center}
  \begin{tabular}{ c c }
    AaBb & 40 (Parental)\\
    aaBb & 10 (Recombinant)\\
    Aabb & 10 (Recombinant)\\
    aabb & 40 (Parental)\\
  \end{tabular}
\end{center}

The distance between the genes is equal to the number of recombinant offspring
over the total population size.

\section{Epistasis}

$$ \ce{A -> B -> C -> Lys} $$

Each step is controlled by a certain enzyme, $x$, $y$, or $z$ (in order from
left to right). First, grow the organism on a rich medium, then mutagenize and
place on a minimal medium (without lysine). If cells die, they are lysine
mutants.

\begin{center}
  \begin{tabular}{ c c c c c }
    Mutant & A & B & C & Lys \\
    x-mutant & - & + & + & + \\
    y-mutant & - & - & + & + \\
    z-mutant & - & - & - & +
  \end{tabular}
\end{center}

\section{Neural Action Potential}

The membrane at resting is $-70\ \si{ mV }$. To remain at this potential, there
is a \textbf{sodium-potassium pump} that exchanges sodium out and potassium in.
There is also an open, \textbf{passive potassium pump}. This allows the
potassium to reach an electrical and chemical gradient.

When the signal accumulates, once it reaches $-50\ \si{ mV }$, the
\textbf{voltage-gated sodium channel} opens. This causes
\textbf{depolarization}. At $+30\ \si{ mV }$, the \textbf{voltage-gated
potassium channel} opens. This causes \textbf{repolarization}.  These gates take
some time to close, this results in a dip below the normal resting potential
causing \textbf{hyperpolarization}. Eventually, this will reach equilibrium once
again after the voltage-gated potassium channel closes.

\section{Immunology}

A \textbf{antigen} is a foreign particle that is recognized by
\textbf{antibodies}. These antibodies are proteins with a Y structure. In this
structure, there are two heavy and light chains. These chains are joined
together by disulfide bonds. On both of these chains, there are variable and
constant regions. The base of the Y is constant and only the tips of the Y are
variable. Within the variable region on the heavy chain, there are VDJ regions
(the V region is on the very end of the Y). The variable region on the light
chain is made up of VJ regions. The different orders of VDJ are decided through
\textbf{V(D)J recombination} which randomly chooses V(D)J sections from the
genome.

There are \textbf{memory cells} that are stored in the bone marrow that remember
infections it has come across before. This allows the body to quickly react to
the antigen when reinfected.

The body is able to recognize the different between self and non-self cells. If
there is a problem with this system, you can develop an \textbf{autoimmune}
disease where your body targets your own proteins.

\section{Stem Cells}

The zygote is the only \textbf{totipotent} stem cell. They then differentiate
into emmbryo, \textbf{pluripotent} stem cells. These differentiate into
\textbf{multipotent} stem cells. These are more specialized. The
\textbf{hematopoetic} stem cells produce the \textbf{B cells} that can be
memory or plasma cells.

\section{Cholesterol}

This is a very prominent cause of heart attacks which arises when plaque builds
up in a blood vessel causing a blockage. In the case of a heart attack, this is
a blockage of an artery going to the heart. A main type of plaque is
cholesterol. Cholesterol is a very hydrophobic molecule. This is not very
soluble in the blood, so cholesterol is packaged into \textbf{lipoproteins}.
Lipoproteins are made up of a single layer of phospholipids, creating a
hydrophobic region inside. There are also proteins that stick out of this.
\textbf{HDL} is a small package and \textbf{LDL} is a large package.

Cholesterol is used in the plasma membrane to keep it flexible. It is also used
to produce bile acids and make steroids. Your body can get cholesterol through
your diet or it can synthesize it using \textbf{HMGCoA reductase enzyme}. Drugs
typically target this enzyme because, without it, you are unable to synthesis
cholesterol.

On the surface of cells, there are \textbf{LDL receptors} which regulate the
uptake of cholesterol in the cell. This remove the cholesterol from the
bloodstream.  If these fail, there can be excess LDL in the blood stream, which
will lead plaque formation and, eventually, a heart attack.

With \textbf{familiar hypocholesteroleimia}, if an individual is heterozygous,
there are fewer LDL receptors and these individuals will likely get heart
disease earlier. In homozygous individuals, there are no LDL receptors. This
individual will get heart disease in their teens.

We can treat this through diet, statins, or bile acid sequesterant. Diet may not
work because the body will produce more cholesterol to compensate. Statins are
most effective because they target the HMG-CoA reductase enzyme. Bile acid
sequesterant will lower LDL by preventing the recycling of bile acids.

\section{Traditional Gene Knock Out}

Support you have one gene with two regions: A and B, separated by an intron. The
antibiotic resistance will let you know whether or on the insert was
incorporated into cell. The negative selection marker will, when incorporated
into a different gene, will kill the cell.

\section{CRISPR}

This is a way that cell protect themselves from viruses. There are repeat
sequences in your genome with sequences in between that match the genome of
certain viruses. The \textbf{Cas9} protein will perform a double stranded cut of
the DNA. It uses a single guide RNA to determine where it will cut. Once the
stranded is cut, the cell will try to repair the damage through homologous
recombination (you can add an insert so the cell will fix the cut with your
provided DNA). Or, if no insert is provided, there will be non-homologous end
joining which will knock out the gene.

\section{RNA Interference}

You insert both the positive and negative strand of the RNA sequence you are
trying to knockout. The positive is the coding strand and the negative is its
complement. The cell will then destroy the double stranded RNA because it will
think it is a virus.

It is important to note that is is a \textbf{knock down}, not a knock out as
there is no permanent damage to the genome. Removing the double stranded insert
will resume the expression.

\section{Cre-Lox}

You take a gene and put lox sites around it. In certain cell types, you can
supply a promoter to express the Cre protein. This Cre protein will recombine
the lox sites with themselves, essentially expelling the gene in between the
sites.

\section{Ras Protein}

The Ras protein can be in two states: active and inactive. Ras is active when
bound to GTP and inactive when bound to GDP. To go from inactive to active,
\textbf{GEF} (quanine nucleotide exchange factor) is used. This turns the GDP
into a GTP. To go from active to inactive, \textbf{GAP} (GTPase activated
protein) is used. This hydrolyzes the GTP and takes away the phosphate.

\section{Signals}

We have two neurons: pre-synaptic and post-synaptic. \textbf{Excitatory} signals
release a ligand that results in a ligand-gated channel that brings in a
positive ion. \textbf{Inhibitory} signals release a ligand that results in a
ligand-gated channel that bring is a negative ion.

\section{PCR (Polymerase Chain Reaction)}

This is the process used to synthetically replicate DNA. For this procedure, you
need a DNA polymerase, a template, a reaction buffer, a forward primer, a
reverse primer, and dNTPs.

You start with a double stranded DNA. Then you heat it to separate the two
strands. DNA polymerase always synthesizes 5' to 3'. You need both the forward
and backward primers so that both strands are replicated. These primers will
stick after the temperature is lowered a little. Raising the temperature a
little will allow the DNA polymerase to complete the copy. This process is then
cycled to exponentially create copies of the DNA.

\section{Sequencing DNA}

This is the process used to determine the base pair order of DNA strands. You
need a DNA polymerase, a template, a reaction buffer, a forward \textbf{or}
reverse primer, normal dNTPs, and ddNTPs. The ddNTP prevents the DNA polymerase
from continuing the synthesis. These are in lower concentration than the dNTPs
to allow several lengths of DNA strands. Depending on the position on an
electro-gel, you can determine the sequence of the genome.

\section{Viruses}

There are different genomes: ssDNA, dsDNA, dsRNA, (+/-)ssRNA, and retroviruses.

\subsection{ssDNA}

This virus uses host RNA polymerase and ribosomes. This virus also needs to
replicate, so it uses host DNA polymerase. This will enter the nucleus.

\subsection{dsDNA}

This is very similar to the above one. It will also use host RNA and DNA
polymerase and ribosomes. This will also enter the nucleus.

\subsection{dsRNA}

This virus uses the host ribosomes. The virus needs to bring it's own RNA
dependent RNA polymerase. This will not enter the nucleus.

\subsection{(+)ssRNA}

This virus can be directly converted into protein. To replicated, you need to
create the (-)ssRNA. This will turn into a dsRNA intermediate. The (-)ssRNA will
be used as the template for replication. They also need to bring their own RNA
dependent RNA polymerase. This will not enter the nucleus.

\subsection{(-)ssRNA}

This virus cannot be directly converted into protein, the (+)ssRNA must first be
created. This will turn into a dsRNA intermediate. The (+)ssRNA will work as the
template for the RNA dependent RNA polymerase to replicated the (-)ssRNA genome.
This will not enter the nucleus.

\subsection{Retrovirus}

These viruses have an RNA genome that is reverse transcribed into DNA. It must
brings its own reverse transcriptase. Once it is turned into DNA, the host
processes can complete. This will enter the nucleus. Once in the nucleus, it
will embed into the host's genome. The other DNA viruses also have the ability
embed into the host's genome.

\section{Operons}

If we want to break down something, a regulator is made and it will stick to the
operator. This will prevent RNA polymerase from binding and transcribing the
gene needed to break down the material. When in the presence of the substance,
the substance will bind to the regulatory protein and it can no longer bind to
the operator. This is a \textbf{repressor}. With nothing left to bind to, the
repressor will be able to bind to the operator.

In the case of an \textbf{activator}, the regulatory protein will be floating
around until the presence of the substance. Then it will be able to bind to the
operator and block transcription.

\section{Cellular Respiration}

Start with glucose, then glycolysis results in 2 pyruvates. This can either go
\textbf{anaerobically}, without oxygen, through fermentation and produce a net
of 2 ATPs. This can also go \textbf{aerobically}, with oxygen, can enter the
citric acid cycle. Then the electron transport chain uses oxygen as an electron
receptor to make water. The electron transport chain will create an electrical
gradient and pumping in protons creates more ATP. This produces a net of 34-38
ATPs.

\end{document}

