\documentclass{article}
\usepackage{tikz}
\usepackage{float}
\usepackage{enumerate}
\usepackage{amsmath}
\usepackage{amsthm}
\usepackage{amsfonts}
\usepackage{bm}
\usepackage{indentfirst}
\usepackage{siunitx}
\usepackage[utf8]{inputenc}
\usepackage{graphicx}
\graphicspath{ {Images/} }
\usepackage{float}
\usepackage{mhchem}
\usepackage{chemfig}
\allowdisplaybreaks

\title{18.06 Problem Set 4}
\author{Robert Durfee}
\date{October 3, 2018}

\begin{document}

\maketitle

\section*{Problem 1}

\textit{Can a set of linearly independent vectors contain the $\vec{0}$
vector? Under what circumstances, if any?}

\bigbreak

No, from the definition of linear independence, $ \{ \vec{v}_1, \vec{v}_2,
\ldots, \vec{v}_n \} $ are independent if $ c_1 \vec{v}_1 + c_2 \vec{v}_2 +
\ldots + c_n \vec{v}_n = 0 $ only if $ c_1 = c_2 = \ldots = c_n = 0 $ But, if
the zero vector is the $ i $th vector in the set, then $c_i$ can equal any
real value while setting the all other $c$'s equal to zero. Therefore, any
set containing the zero vector must be linearly dependent.

\section*{Problem 2}

\textit{Give a basis and state the dimensionality for the following vector
spaces and subspaces (for multiplication by real scalars and the ordinary
addition/subtraction operations):}

\subsection*{Part A}

\textit{Functions $p(x)$ that are polynomials of degree $ \leq 3 $ (cubic or
less).}

\bigbreak

All cubic (or less) polynomials can be represented by form $ p(x) = ax^3 +
bx^2 + cx + d $ for all real values $ a, b, c, d $. A single cubic polynomial
can thus be expressed as the vector
$$ p(x) = \begin{pmatrix}
    a \\
    b \\
    c \\
    d
\end{pmatrix} $$
This can be written as a sum of vectors
$$ p(x) = a \begin{pmatrix}
    1 \\
    0 \\
    0 \\
    0
\end{pmatrix} + b \begin{pmatrix}
    0 \\
    1 \\
    0 \\
    0
\end{pmatrix} + c \begin{pmatrix}
    0 \\
    0 \\
    1 \\
    0
\end{pmatrix} + d \begin{pmatrix}
    0 \\
    0 \\
    0 \\
    1
\end{pmatrix} $$
Therefore, a basis takes the form
$$ \left\{\begin{pmatrix}
    1 \\
    0 \\
    0 \\
    0
\end{pmatrix},\, \begin{pmatrix}
    0 \\
    1 \\
    0 \\
    0
\end{pmatrix},\, \begin{pmatrix}
    0 \\
    0 \\
    1 \\
    0
\end{pmatrix},\, \begin{pmatrix}
    0 \\
    0 \\
    0 \\
    1
\end{pmatrix} \right\} $$
The vectors that make up this basis live in $ \mathbb{R}^4 $ and the
dimension of the space is also $ 4 $.

\subsection*{Part B}

\textit{3-component vectors $x \in \mathbb{R}^3$ whose components average to
zero.}

\subsection*{Part C}

\textit{$3\times 3$ matrices $A$ that are anti-symmetric: $A = -A^T$.}

\section*{Problem 3}

\textit{Come up with a matrix $A$ and a vector $b \ne 0$ such that the
solutions of $Ax=b$ form a line in $\mathbb{R}^3$, and all of the entries of
$A$ are nonzero. Find the complete solution (i.e., all solutions) $x$.}

\bigbreak

Let $ A $ be the $ 3 \times 3 $ matrix and $ b $ be the $3 \times 1$ vector
$$ A = \begin{pmatrix}
    1 & 1 & 2 \\
    1 & 2 & 4 \\
    1 & 3 & 6
\end{pmatrix},\, b = \begin{pmatrix}
    2 \\
    2 \\
    2
\end{pmatrix} $$
Then the nullspace of this matrix is given by the basis
$$ N(A) = \left\{ \begin{pmatrix}
    0 \\
    -2 \\
    1
\end{pmatrix} \right\} $$
This is a 1-dimensional nullspace in $ \mathbb{R}^3 $. Therefore, any
solution to $ Ax = b $ is given by
$$ x = x_s + c_1 \cdot \begin{pmatrix}
    0 \\
    -2 \\
    1
\end{pmatrix} $$
Which forms a line in $ \mathbb{R}^3 $. A possible special solution could be
$$ x_s = \begin{pmatrix}
    2 \\
    0 \\
    0
\end{pmatrix} $$
Thus, all solutions to $ Ax = b $ is explicitly
$$ x = \begin{pmatrix}
    2 \\
    0 \\
    0
\end{pmatrix} + c_1 \begin{pmatrix}
    0 \\
    -2 \\
    1
\end{pmatrix} $$

\section*{Problem 4}

\textit{The following matrix is from problem 5 of pset 3:}
$$ A = \begin{pmatrix}
    0 & 1 & 2 & 3 & 4 \\
    0 & 1 & 2 & 4 & 6 \\
    0 & 0 & 0 & 1 & 2
\end{pmatrix} $$

\subsection*{Part A}

\textit{Give a basis for $C(A)$ and the dimension of this subspace.}

\bigbreak

The column space of this matrix is given by the basis
$$ C(A) = \left\{ \begin{pmatrix}
    1 \\
    1 \\
    0
\end{pmatrix},\, \begin{pmatrix}
    3 \\
    4 \\
    1
\end{pmatrix} \right\} $$
This is a 2-dimensional column space whose components live in $ \mathbb{R}^3 $

\subsection*{Part B}

\textit{If $b = \begin{pmatrix} 3 \\ 6 \\ \beta \end{pmatrix}$, for what
values of the scalar $\beta$ will $Ax=b$ have a solution?}

\bigbreak

In order for $ Ax = b $ to have a solution, $ \beta = 3 $.

\subsection*{Part C}

\textit{For the $\beta$ from (b), find the complete solution to $Ax=b$.}

\bigbreak

The nullspace of $ A $ is given by the basis
$$ N(A) = \left\{ \begin{pmatrix}
    1 \\
    0 \\
    0 \\
    0 \\
    0
\end{pmatrix},\, \begin{pmatrix}
    0 \\
    -2 \\
    1 \\
    0 \\
    0
\end{pmatrix},\, \begin{pmatrix}
    0 \\
    2 \\
    0 \\
    -2 \\
    1
\end{pmatrix} \right\} $$
A special solution to $ Ax = b $ is
$$ x_s = \begin{pmatrix}
    0 \\
    -6 \\
    0 \\
    3 \\
    0
\end{pmatrix} $$
Therefore, the complete solution to $ Ax = b $ is
$$ x = \begin{pmatrix}
    0 \\
    -6 \\
    0 \\
    3 \\
    0
\end{pmatrix} + c_1 \begin{pmatrix}
    1 \\
    0 \\
    0 \\
    0 \\
    0
\end{pmatrix} + c_2 \begin{pmatrix}
    0 \\
    -2 \\
    1 \\
    0 \\
    0
\end{pmatrix} + c_3 \begin{pmatrix}
    0 \\
    2 \\
    0 \\
    -2 \\
    1
\end{pmatrix} $$

\end{document}