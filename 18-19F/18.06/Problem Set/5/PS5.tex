\documentclass{article}
\usepackage{tikz}
\usepackage{float}
\usepackage{enumerate}
\usepackage{amsmath}
\usepackage{amsthm}
\usepackage{amsfonts}
\usepackage{bm}
\usepackage{indentfirst}
\usepackage{siunitx}
\usepackage[utf8]{inputenc}
\usepackage{graphicx}
\graphicspath{ {Images/} }
\usepackage{float}
\usepackage{mhchem}
\usepackage{chemfig}
\allowdisplaybreaks

\title{18.06 Problem Set 5}
\author{Robert Durfee}
\date{October 10, 2018}

\begin{document}

\maketitle

\section*{Problem 1}

\textit{In pset 4, problem 4, you considered the matrix}
$$ A = \begin{pmatrix}
    0 & 1 & 2 & 3 & 4 \\
    0 & 1 & 2 & 4 & 6 \\
    0 & 0 & 0 & 1 & 2
\end{pmatrix} $$

\subsection*{Part A}

\textit{Find a basis for the row space $C(A^T)$.}

\subsection*{Part B}

\textit{Find a basis for the left nullspace $N(A^T)$. Use the fact that
$N(A^T) = C(A)^\perp$ to obtain a condition for $Ax=c$ to be solvable for a
vector $c = \begin{pmatrix} c_1 \\ c_2 \\ c_3 \end{pmatrix}$.}

\subsection*{Part C}

\textit{In pset 4, you considered the vector $b = \begin{pmatrix} 3 \\ 6 \\
\beta \end{pmatrix}$ and found that $Ax=b$ only had a solution if $\beta =
3$. Check that this condition also follows from your answer in (b).}

\section*{Problem 2}

\textit{The set of $2\times 2$ real matrices form a vector space
$\mathbb{R}^{2\times2}$. One possible basis for this vector space is the
following set of 4 matrices:}
$$ M_1 = \begin{pmatrix}
    1 & 0 \\
    0 & 0
\end{pmatrix}, \; M_2 = \begin{pmatrix}
    0 & 0 \\
    1 & 0
\end{pmatrix}, \; M_3 = \begin{pmatrix}
    0 & 1 \\
    0 & 0
\end{pmatrix}, \; M_4 = \begin{pmatrix}
    0 & 0 \\
    0 & 1
\end{pmatrix}. $$
\textit{That is, we can write any $A \in \mathbb{R}^{2\times2}$ as $A = a_1
M_1 + a_2 M_2 + a_3 M_3 + a_4 M_4$ for $a = \begin{pmatrix} a_1 \\ a_2 \\ a_3
\\ a_4 \end{pmatrix}$: representing matrices $A$ by vectors $a \in
\mathbb{R}^4$!}

\textit{Given any $2\times 2$ matrices}
$$ B = \begin{pmatrix}
    b_{11} & b_{12} \\
    b_{21} & b_{22}
\end{pmatrix}, \; C = \begin{pmatrix}
    c_{11} & c_{12} \\
    c_{21} & c_{22}
\end{pmatrix}, $$
\textit{we can define a linear transformation $T(A) = BAC$ that takes a
matrix $A \in \mathbb{R}^{2\times2}$ and gives you another matrix in
$\mathbb{R}^{2\times2}$.}

\subsection*{Part A}

\textit{Write this $T(A)$ using the basis $\{ M_1, M_2, M_3, M_4 \}$ as a
single matrix $D$ multipying the vector $a$ corresponding to $A$. Start by
expressing your $D$ as the product of two matrices (one representing
multipling on the left by $B$ and the other representing multiplying on the
right by $C$), and then multiply them to give a formula for $D$.}

\subsection*{Part B}

\textit{In Julia, you can get $a$ from $A$ by $a = vec(A)$. For the example
matrices given below, fill in $D$ and check your answer from (a): check that
$vec(B*A*C) = D*vec(A)$.}

\section*{Problem 4}

\textit{The following is an important property of the very important matrix
$A^T A$ (for real matrices) that will come up several times in 18.06:}

\subsection*{Part A}

\textit{If $A^TAx=0$ then $Ax=0$. Reason: If $A^TAx=0$, then $Ax$ is in the
nullspace of $A^T$ and also in the ?????? of $A$, and those spaces are
???????. Conclusion: $N(A^T A) = N(A)$.}

\subsection*{Part B}

\textit{Alternative proof: $A^TAx=0$, then $x^T A^T Ax = 0 = (Ax)^T (Ax)$.
Why does this imply that $Ax=0$? (Hint: if $y^Ty = 0$, can we have $y\ne
0$?)}

\section*{Problem 5}

\textit{Construct matrices with each of the following properties, or explain
why it is impossible:}

\subsection*{Part A}

\textit{Column space contains $\begin{pmatrix} 1\\1\\0 \end{pmatrix}$,
$\begin{pmatrix} 0\\0\\1 \end{pmatrix}$, and row space contains
$\begin{pmatrix} 1\\2 \end{pmatrix}$, $\begin{pmatrix} 2 \\5 \end{pmatrix}$}

\subsection*{Part B}

\textit{Column space has basis $\begin{pmatrix} 1\\1\\3
\end{pmatrix}$, nullspace has basis $\begin{pmatrix} 3\\1\\1
\end{pmatrix}$}

\subsection*{Part C}

\textit{Dimension of nullspace = 1 + dimension of left nullspace}

\subsection*{Part D}

\textit{Nullspace contains $\begin{pmatrix} 1\\3 \end{pmatrix}$, column space
contains $\begin{pmatrix} 3\\1 \end{pmatrix}$}

\subsection*{Part E}

\textit{Row space = column space, nullspace $\neq$ left nullspace.}

\section*{Problem 6}

\textit{Suppose S is spanned by (1,7,3) and (1,1,1). Then $S^\perp$ is the
nullspace of what matrix?}

\section*{Problem 7}

\textit{If a subspace $S$ is contained in a subspace $V$ ($S \subseteq V$),
then which of the following must be true? $S^\perp$ contains $V^\perp$ or
$V^\perp$ contains $S^\perp$? Why?}

\end{document}