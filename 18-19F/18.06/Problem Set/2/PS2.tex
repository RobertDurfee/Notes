\documentclass{article}
\usepackage{tikz}
\usepackage{float}
\usepackage{enumerate}
\usepackage{amsmath}
\usepackage{amsthm}
\usepackage{bm}
\usepackage{indentfirst}
\usepackage{siunitx}
\usepackage[utf8]{inputenc}
\usepackage{graphicx}
\graphicspath{ {Images/} }
\usepackage{float}
\usepackage{mhchem}
\usepackage{chemfig}
\allowdisplaybreaks

\title{18.06 Problem Set 2}
\author{Robert Durfee}
\date{September 19, 2018}

\begin{document}

\maketitle

\section*{Problem 1}

\subsection*{Part A}

Starting with the augmented matrix
\[
    \begin{pmatrix}
        1 & 2 & 1 & -1 & 1 & 0 \\
        1 & 0 & 1 & -1 & 0 & 0 \\
        1 & 2 & 1 & 0 & 0 & 0 \\
        1 & 1 & 0 & 0 & 0 & 0 \\
        -1 & 0 & 0 & 0 & 0 & 1
    \end{pmatrix}
\]
Applying the first elimination matrix
\[
    \begin{pmatrix}
        1 & 0 & 0 & 0 & 0 \\
        -1 & 1 & 0 & 0 & 0 \\
        -1 & 0 & 1 & 0 & 0 \\
        -1 & 0 & 0 & 1 & 0 \\
        1 & 0 & 0 & 0 & 1
    \end{pmatrix}
    \begin{pmatrix}
        1 & 2 & 1 & -1 & 1 & 0 \\
        1 & 0 & 1 & -1 & 0 & 0 \\
        1 & 2 & 1 & 0 & 0 & 0 \\
        1 & 1 & 0 & 0 & 0 & 0 \\
        -1 & 0 & 0 & 0 & 0 & 1
    \end{pmatrix}
    =
    \begin{pmatrix}
        1 & 2 & 1 & -1 & 1 & 0 \\
        0 & -2 & 0 & 0 & -1 & 0 \\
        0 & 0 & 0 & 1 & -1 & 0 \\
        0 & -1 & -1 & 1 & -1 & 0 \\
        0 & 2 & 1 & -1 & 1 & 1
    \end{pmatrix}
\]
Applying the second elimination matrix
\[
    \begin{pmatrix}
        1 & 0 & 0 & 0 & 0 \\
        0 & 1 & 0 & 0 & 0 \\
        0 & 0 & 1 & 0 & 0 \\
        0 & -1/2 & 0 & 1 & 0 \\
        0 & 1 & 0 & 0 & 1
    \end{pmatrix}
    \begin{pmatrix}
        1 & 2 & 1 & -1 & 1 & 0 \\
        0 & -2 & 0 & 0 & -1 & 0 \\
        0 & 0 & 0 & 1 & -1 & 0 \\
        0 & -1 & -1 & 1 & -1 & 0 \\
        0 & 2 & 1 & -1 & 1 & 1
    \end{pmatrix}
    =
    \begin{pmatrix}
        1 & 2 & 1 & -1 & 1 & 0 \\
        0 & -2 & 0 & 1 & -1 & 0 \\
        0 & 0 & 0 & 1 & -1 & 0 \\
        0 & 0 & -1 & 1 & -1/2 & 0 \\
        0 & 0 & 1 & -1 & 0 & 1
    \end{pmatrix}
\]
Swapping rows 3 and 4 and applying third elimination matrix
\[
    \begin{pmatrix}
        1 & 0 & 0 & 0 & 0 \\
        0 & 1 & 0 & 0 & 0 \\
        0 & 0 & 1 & 0 & 0 \\
        0 & 0 & 0 & 1 & 0 \\
        0 & 0 & 1 & 0 & 1
    \end{pmatrix}
    \begin{pmatrix}
        1 & 2 & 1 & -1 & 1 & 0 \\
        0 & -2 & 0 & 1 & -1 & 0 \\
        0 & 0 & -1 & 1 & -1/2 & 0 \\
        0 & 0 & 0 & 1 & -1 & 0 \\
        0 & 0 & 1 & -1 & 0 & 1
    \end{pmatrix}
    =
    \begin{pmatrix}
        1 & 2 & 1 & -1 & 1 & 0 \\
        0 & -2 & 0 & 0 & -1 & 0 \\
        0 & 0 & -1 & 1 & -1/2 & 0 \\
        0 & 0 & 0 & 1 & -1 & 0 \\
        0 & 0 & 0 & 0 & -1/2 & 1
    \end{pmatrix}
\]
Applying backsubstitution
$$ -1/2 x_5 = 1 \implies x_5 = -2 $$
$$ x_4 - x_5 = 0 \implies x_4 = -2 $$
$$ -x_3 + x_4 - 1/2 x_5 = 0 \implies x_3 = -1 $$
$$ -2 x_2 - x_5 = 0 \implies x_2 = 1 $$
$$ x_1 + 2 x_2 + x_3 - x_4 + x_5 = 0 \implies x_1 = -1 $$

\subsection*{Part B}

The vector $ \vec{b} $ simply selects the last column of any matrix multiplied
to it on the left. Therefore, since $ x = A^{-1} b $, the vector $ \vec{x} $
must be the last column of $ A^{-1} $.

\section*{Problem 2}

\subsection*{Part A}

\[
    \begin{pmatrix}
        1 & 4 & 1 \\
        1 & 2 & -1 \\
        3 & 14 & 6
    \end{pmatrix}
    \begin{pmatrix}
        1 & -4 & -1 \\
        0 & 1 & 0 \\
        0 & 0 & 1
    \end{pmatrix}
    =
    \begin{pmatrix}
        1 & 0 & 0 \\
        1 & -2 & -2 \\
        3 & 2 & 3
    \end{pmatrix}
\]
Therefore
\[
    E = \begin{pmatrix}
        1 & -4 & -1 \\
        0 & 1 & 0 \\
        0 & 0 & 1
    \end{pmatrix},\ 
    E^{-1} = \begin{pmatrix}
        1 & 4 & 1 \\
        0 & 1 & 0 \\
        0 & 0 & 1
    \end{pmatrix}
\]

\subsection*{Part B}

The resulting matrix should be the inverse of matrix $ A $.

\subsection*{Part C}

Picking up from previous step in Part A
\[
    \begin{pmatrix}
        1 & 0 & 0 \\
        1 & -2 & -2 \\
        3 & 2 & 3
    \end{pmatrix}
    \begin{pmatrix}
        1 & 0 & 0 \\
        1/2 & 1 & 0 \\
        0 & 0 & 1
    \end{pmatrix}
    =
    \begin{pmatrix}
        1 & 0 & 0 \\
        0 & -2 & 0 \\
        4 & 2 & 1
    \end{pmatrix}
\]
\[
    \begin{pmatrix}
        1 & 0 & 0 \\
        0 & -2 & 0 \\
        4 & 2 & 1
    \end{pmatrix}
    \begin{pmatrix}
        1 & 0 & 0 \\
        0 & -2 & 0 \\
        -4 & -2 & 1
    \end{pmatrix}
    =
    \begin{pmatrix}
        1 & 0 & 0 \\
        0 & -2 & 0 \\
        0 & 0 & 1
    \end{pmatrix}
\]
\[
    \begin{pmatrix}
        1 & 0 & 0 \\
        0 & -2 & 0 \\
        0 & 0 & 1
    \end{pmatrix}
    \begin{pmatrix}
        1 & 0 & 0 \\
        0 & -1/2 & 0 \\
        0 & 0 & 1
    \end{pmatrix}
    =
    \begin{pmatrix}
        1 & 0 & 0 \\
        0 & 1 & 0 \\
        0 & 0 & 1
    \end{pmatrix}
\]

Now applying in reverse
\[
    \begin{pmatrix}
        1 & 0 & 0 \\
        0 & 1 & 0 \\
        0 & 0 & 1
    \end{pmatrix}
    \begin{pmatrix}
        1 & -4 & 1 \\
        0 & 1 & 0 \\
        0 & 0 & 1
    \end{pmatrix}
    =
    \begin{pmatrix}
        1 & -4 & -1 \\
        0 & 1 & 0 \\
        0 & 0 & 1
    \end{pmatrix}
\]
\[
    \begin{pmatrix}
        1 & -4 & -1 \\
        0 & 1 & 0 \\
        0 & 0 & 1
    \end{pmatrix}
    \begin{pmatrix}
        1 & 0 & 0 \\
        0 & 1 & -1 \\
        0 & 0 & 1
    \end{pmatrix}
    =
    \begin{pmatrix}
        1 & -4 & 3 \\
        0 & 1 & -1 \\
        0 & 0 & 1
    \end{pmatrix}
\]
\[
    \begin{pmatrix}
        1 & -4 & 3 \\
        0 & 1 & -1 \\
        0 & 0 & 1
    \end{pmatrix}
    \begin{pmatrix}
        1 & 0 & 0 \\
        1/2 & 1 & 0 \\
        0 & 0 & 1
    \end{pmatrix}
    =
    \begin{pmatrix}
        -1 & -4 & 3 \\
        1/2 & 1 & -1 \\
        0 & 0 & 1
    \end{pmatrix}
\]
\[
    \begin{pmatrix}
        -1 & -4 & 3 \\
        1/2 & 1 & -1 \\
        0 & 0 & 1
    \end{pmatrix}
    \begin{pmatrix}
        1 & 0 & 0 \\
        0 & 1 & 0 \\
        -4 & -2 & 1
    \end{pmatrix}
    =
    \begin{pmatrix}
        -13 & -10 & 3 \\
        9/2 & 3 & -1 \\
        -4 & -2 & 1
    \end{pmatrix}
\]
\[
    \begin{pmatrix}
        -13 & -10 & 3 \\
        9/2 & 3 & -1 \\
        -4 & -2 & 1
    \end{pmatrix}
    \begin{pmatrix}
        1 & 0 & 0 \\
        0 & -1/2 & 0 \\
        0 & 0 & 1
    \end{pmatrix}
    =
    \begin{pmatrix}
        -13 & 5 & 3 \\
        9/2 & -3/2 & -1 \\
        -4 & 1 & 1
    \end{pmatrix}
\]

\section*{Problem 3}

Starting with the given matrix
\[
    \begin{pmatrix}
        a & a & a & a \\
        a & b & b & b \\
        a & b & c & c \\
        a & b & c & d
    \end{pmatrix}
\]
Applying $ R2 - R1 $, $R3 - R1 $, $ R4 - R1 $.
\[
    \begin{pmatrix}
        a & a & a & a \\
        0 & b - a & b - a & b - a \\
        0 & b - a & c - a & c - a \\
        0 & b - a & c - a & d - a
    \end{pmatrix}
\]
Applying $ R3 - R2 $ and $ R4 - R2 $.
\[
    \begin{pmatrix}
        a & a & a & a \\
        0 & b - a & b - a & b - a \\
        0 & 0 & c - b & c - b \\
        0 & 0 & c - b & d - b
    \end{pmatrix}
\]
Applying $ R4 - R3 $.
\[
    \begin{pmatrix}
        a & a & a & a \\
        0 & b - a & b - a & b - a \\
        0 & 0 & c - b & c - b \\
        0 & 0 & 0 & d - c
    \end{pmatrix}
\]
Therefore, in order for there to be four pivots,
$$ a \neq 0,\ b - a \neq 0,\ c - b \neq 0,\ d - c \neq 0 $$

\section*{Problem 4}

\subsection*{Part A}

Starting with the given matrix
\[
    \begin{pmatrix}
        1 & 0 & 0 \\
        a & 1 & 0 \\
        b & c & 1
    \end{pmatrix}
\]
Applying $ R2 - a R1 $ and $ R3 - b R1 $.
\[
    \begin{pmatrix}
        1 & 0 & 0 \\
        0 & 1 & 0 \\
        0 & c & 1
    \end{pmatrix}
\]
Applying $ R3 - c R2 $.
\[
    \begin{pmatrix}
        1 & 0 & 0 \\
        0 & 1 & 0 \\
        0 & 0 & 1
    \end{pmatrix}
\]

\subsection*{Part B}

Starting with the identity matrix
\[
    \begin{pmatrix}
        1 & 0 & 0 \\
        0 & 1 & 0 \\
        0 & 0 & 1
    \end{pmatrix}
\]
Applying $ R2 - a R1 $ and $ R3 - b R1 $.
\[
    \begin{pmatrix}
        1 & 0 & 0 \\
        -a & 1 & 0 \\
        -b & 0 & 1
    \end{pmatrix}
\]
Applying $ R3 - c R2 $.
\[
    \begin{pmatrix}
        1 & 0 & 0 \\
        -a & 1 & 0 \\
        ca - b & -c & 1
    \end{pmatrix}
\]

\subsection*{Part C}

\section*{Problem 5}

\subsection*{Part A}

The given expression $ A = U B^{-1} L $ can be rewritten using the inverse
product rule
$$ A^{-1} = L^{-1} B U^{-1} $$
Let $ C = L^{-1} B $. Using row reduction, $ C $ is determined
\[
    C = \begin{pmatrix}
        1 & 2 & 3 \\
        4 & 4 & 4 \\
        -1 & 0 & 3
    \end{pmatrix}
\]
Using the column wise definition of matrix multiplication,
$$ a_2^{-1} = C u_2^{-1} $$
The definition of identities gives
$$ U u_2^{-1} = e_2 $$
Using row reduction, $ u_2^{-1} $ can be determined.
\[
    u_2^{-1} = \begin{pmatrix}
        -1 \\
        1 \\
        0
    \end{pmatrix}
\]
Substituting back to solve for second column of $ A^{1} $
\[
    a_2^{-1} = \begin{pmatrix}
        1 & 2 & 3 \\
        4 & 4 & 4 \\
        -1 & 0 & 3
    \end{pmatrix}
    \begin{pmatrix}
        -1 \\
        1 \\
        0
    \end{pmatrix}
    = \begin{pmatrix}
        1 \\
        0 \\
        1
    \end{pmatrix}
\]

\section*{Problem 6}

\subsection*{Part A}

Each column in a tridiagonal matrix only needs to complete one row operation to
eliminate the element directly below the pivot point. Therefore, the $ L $
pattern should be
\[
    L = \begin{pmatrix}
        1 & 0 & 0 & 0 \\
        \star & 1 & 0 & 0 \\
        0 & \star & 1 & 0 \\
        0 & 0 & \star & 1
    \end{pmatrix}
\]

Furthermore, since each column only has one element directly above the pivot,
and all the above elements are zeros, they will be unaffected by row elimination
steps. Therefore, the $ U $ pattern should be
\[
    U \begin{pmatrix}
        \star & \star & 0 & 0 \\
        0 & \star & \star & 0 \\
        0 & 0 & \star & \star \\
        0 & 0 & 0 & \star
    \end{pmatrix}
\]

\subsection*{Part B}

Since each column only has one nonzero element below the pivot, it only takes
one row operation to make zero below each pivot. In this row operation, only
four elements are involved. Therefore, there are 2 scalar multiplications and
additions. Thus, each column requires constant time to perform row reduction.
Since there are $ m $ columns, the number of scalar operations is roughly
proportional to $ m $.

\end{document}

