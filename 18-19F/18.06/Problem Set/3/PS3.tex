\documentclass{article}
\usepackage{tikz}
\usepackage{float}
\usepackage{enumerate}
\usepackage{amsmath}
\usepackage{amsthm}
\usepackage{amsfonts}
\usepackage{bm}
\usepackage{indentfirst}
\usepackage{siunitx}
\usepackage[utf8]{inputenc}
\usepackage{graphicx}
\graphicspath{ {Images/} }
\usepackage{float}
\usepackage{mhchem}
\usepackage{chemfig}
\allowdisplaybreaks

\title{18.06 Problem Set 3}
\author{Robert Durfee}
\date{September 26, 2018}

\begin{document}

\maketitle

\section*{Problem 1}

\textit{Which of the following sets are vectors spaces (with the usual
definition of $\pm$ and multiplication by real scalars), and why? }

\begin{enumerate}
    \item \textit{The set of all functions $f(x)$ whose integral
    $\int_{-\infty}^\infty f(x) dx$ is zero.}

    \bigbreak
    
    {\bf Zero}: The function $ f(x) = 0 $ is in the set of all functions
    whose integral is $ \int_{-\infty}^{\infty} f(x) dx = 0 $ as $
    \int_{-\infty}^{\infty} 0 dx = 0 $.
    
    \bigbreak
    
    {\bf Addition}: Let $ f(x) $ and $ g(x) $ be in the set described above.
    Then, the following must be true
    $$ \int\limits_{-\infty}^{\infty} f(x) dx = 0,\,
    \int\limits_{-\infty}^{\infty} g(x) dx = 0 $$
    Using this information, checking if $ f(x) + g(x) $ is also in the set
    described above:
    $$ \int\limits_{-\infty}^{\infty} f(x) + g(x) dx =
    \int\limits_{-\infty}^{\infty} f(x) dx + \int\limits_{-\infty}^{\infty}
    g(x) dx = 0 + 0 = 0 $$
    Therefore, this set is closed under addition.

    \bigbreak

    {\bf Scalar Multiplication}: Let $ f(x) $ be in the set described above
    and $ k $ be some real-valued constant. The following must be true
    $$ \int\limits_{-\infty}^{\infty} f(x) dx = 0 $$
    Using this information, checking if $ k f(x) $ is also in the set
    described above:
    $$ \int\limits_{-\infty}^{\infty} k f(x) dx = k
    \int\limits_{-\infty}^{\infty} f(x) dx = k \cdot 0 = 0 $$
    Therefore, this set is closed under scalar multiplication.

    \bigbreak

    Since all the conditions for vector spaces are satisfied, this must be a
    vector space.

    \item \textit{Given a subspace $V$ of $ \mathbb{R}^n$ and an $m\times n$
    matrix $A$, the set of all vectors $Ax$ for all $x \in V$.}
    
    \bigbreak

    {\bf Zero}: The zero vector can be represented by the following, for example:
    $$ A = I_n,\, x = \vec{0}_n $$
    As the vector subspace $ V $ must contain the zero vector $ \vec{0}_n $.
    
    \bigbreak

    {\bf Addition}: Let $ A $, and $ B $ be two $ m \times n $ dimensional
    matrices and $ x $ and $ y $ be two $ n \times 1 $ dimensional vectors in
    $ V $. Then, both $ Ax $ and $ By $ are in the set defined above. Using
    this information, checking if $ Ax + By $ is also in the set defined
    above. That is, checking if
    $$ Ax + By = Cz $$
    for some $ m \times n $ matrix $ C $ and some $ n \times 1 $ vector $ z $.
    Expanding these matrices and vectors element-wise yields
    $$ \begin{pmatrix}
        a_{11} & a_{12} & \ldots & a_{1n} \\
        a_{21} & a_{22} & \ldots & a_{2n} \\
        \vdots & \vdots & \ddots & \vdots \\
        a_{m1} & a_{m2} & \ldots & a_{mn}
    \end{pmatrix}
    \begin{pmatrix}
        x_1 \\
        x_2 \\
        \vdots \\
        x_n 
    \end{pmatrix}
    +
    \begin{pmatrix}
        b_{11} & b_{12} & \ldots & b_{1n} \\
        b_{21} & b_{22} & \ldots & b_{2n} \\
        \vdots & \vdots & \ddots & \vdots \\
        b_{m1} & b_{m2} & \ldots & b_{mn}
    \end{pmatrix}
    \begin{pmatrix}
        y_1 \\
        y_2 \\
        \vdots \\
        y_n 
    \end{pmatrix} $$
    Applying multiplication yields
    $$ \begin{pmatrix}
        a_{11} x_1 + a_{12} x_2 + \ldots + a_{1n} x_n \\
        a_{21} x_1 + a_{22} x_2 + \ldots + a_{2n} x_n \\
        \vdots \\
        a_{m1} x_1 + a_{m2} x_2 + \ldots + a_{mn} x_n
    \end{pmatrix}
    +
    \begin{pmatrix}
        b_{11} y_1 + b_{12} y_2 + \ldots + b_{1n} y_n \\
        b_{21} y_1 + b_{22} y_2 + \ldots + b_{2n} y_n \\
        \vdots \\
        b_{m1} y_1 + b_{m2} y_2 + \ldots + b_{mn} y_n
    \end{pmatrix} $$
    Applying addition yields
    $$ \begin{pmatrix}
        a_{11} x_1 + b_{11} y_1 + a_{12} x_2 + b_{12} y_2 + \ldots + a_{1n} x_n + b_{1n} y_n \\
        a_{21} x_1 + b_{21} y_1 + a_{22} x_2 + b_{22} y_2 + \ldots + a_{2n} x_n + b_{2n} y_n \\
        \vdots \\
        a_{m1} x_1 + b_{m1} y_1 + a_{m2} x_2 + b_{m2} y_2 + \ldots + a_{mn} x_n + b_{mn} y_n
    \end{pmatrix} $$
    Now, this \textit{can} be separated into a matrix-vector product
    $$ \begin{pmatrix}
        a_{11} & b_{11} & a_{12} & b_{12} & \ldots & a_{1n} & b_{1n} \\
        a_{21} & b_{21} & a_{22} & b_{22} & \ldots & a_{2n} & b_{2n} \\
        \vdots & \vdots & \vdots & \vdots & \ddots & \vdots & \vdots \\
        a_{m1} & b_{m1} & a_{m2} & b_{m2} & \ldots & a_{mn} & b_{mn}
    \end{pmatrix}
    \begin{pmatrix}
        x_1 \\
        y_1 \\
        x_2 \\
        y_2 \\
        \vdots \\
        x_n \\
        y_n
    \end{pmatrix} $$
    However, the dimenions are \textit{not} $ m \times n $ and $ n \times 1
    $. Therefore, this set is not closed under addition.

    \bigbreak

    Since the set in not closed under addition, it cannot be a vector space.

    \item \textit{The set of 3-component vectors whose components sum to 1.}

    \bigbreak

    {\bf Zero}: Let $ v $ be a vector contained in the set described above.
    $ v $ can be written element-wise
    $$ v = \begin{pmatrix}
        v_1 \\
        v_2 \\
        v_3
    \end{pmatrix} $$
    Where $ v_1 + v_2 + v_3 = 1 $ There is no for the zero vector to be
    present in this set as $ 0 + 0 + 0 \neq 1 $.

    \bigbreak

    Since the zero vector is not present in the set, it cannot be a vector
    space.

    \item \textit{The set of 2-component vectors whose components have product 0.}

    \bigbreak

    {\bf Zero}: Let $ v $ be a vector contained in the set described above. $
    v $ can be written element-wise
    $$ v = \begin{pmatrix}
        v_1 \\
        v_2
    \end{pmatrix} $$
    Where $ v_1 \cdot v_2 = 0 $. The zero vector is clearly in this set as $
    0 \cdot 0 = 0 $.

    \bigbreak

    {\bf Addition}: Let $ v $ and $ w $ be in the set described above. Then,
    the following must be true
    $$ v_1 \cdot v_2 = 0,\, w_1 \cdot w_2 = 0 $$
    Using this information, checking if $ v + w $ is also in the set
    described above
    $$ v + w = \begin{pmatrix}
        v_1 + w_1 \\
        v_2 + w_2
    \end{pmatrix} $$ 
    Checking component conditions
    $$ (v_1 + w_1) \cdot (v_2 + w_2) = v_1 v_2 + v_1 w_2 + w_1 v_2 + w_1 w_2
    = v_1 w_2 + w_1 v_2 $$
    This is \textit{not} guaranteed to be zero as $ v_1 v_2 = 0 $ if only $
    v_1 = 0 $ and $ w_1 w_2 = 0 $ if only $ w_2 = 0 $. Therefore, the above
    expression would reduce further to $ w_1 v_2 $ which wouldn't necessarily
    be zero. Therefore, this set is not closed under addition.
    
    \bigbreak

    Since this set is not closed under addition, this is not a vector space.

    \item \textit{The union of two subspaces $S$ and $T$: all vectors that
    are in *either* $S$ or $T$.}

    \bigbreak

    {\bf Zero}: Since both $S$ and $T$ are vector subspaces, then the zero vector
    must be in each of them and therefore in there union.

    \bigbreak

    {\bf Addition}: Since $ S $ is not fully contained in $ T $, there is
    some $ s \in S $ yet $ s \notin T $. Similarly for $ t \in T $ yet $ t
    \notin S $.
    
    Proof by contradiction. Assume that $ S \cup T $ is a subspace. Then,
    there must exist two vectors $ s, t \in S \cup T $ such that $ s + t \in
    S \cup T $. Thus, there is some vector $ s + t = s' $ that lies in either
    $ S $ or $ T $.
    
    If $ s' \in S $, then $ s + t = s' \in S $ and since $ s, s' \in S $,
    then, because $ S $ is a subspace and closed under addition, the difference
    $ s' - s = t \in S $. But this contradicts the choice for $ t \in T $ but
    $ t \notin S $.

    From this, $ s + t \in T $. Then $ s + t = t' \in T $ and since $ t, t'
    \in T $, then, because $ T $ is also a subspace and closed under
    addition, the difference $ t' - t = s \in T $. But this also contradicts
    the choice for $ s \in S $ but $ s \notin T $.

    Therefore, the union is not closed under addition.

    \bigbreak

    Since the union is not closed under addition, it cannot be a vector space.

    \item \textit{The intersection of two subspaces $S$ and $T$: all vectors
    that are in *both* $S$ and $T$.}

    \bigbreak

    {\bf Zero}: Since both $ S $ and $ T $ are vector subspaces, they must
    both contain the zero vector. Thus, that zero vector must be in the
    intersection.

    \bigbreak

    {\bf Addition}: Let $ s,t \in S \cap T $. This also means that $ s \in S,
    T $ and $ t \in S, T $. Because $ S $ is a subspace and $ s $ and $ t $
    are both within it, then $ s + t \in S $. Furthermore, because $ T $ is a
    subspace and $ s $ and $ t $ are both within it, then $ s + t \in T $.
    Therefore the sum is in both $ S $ and $ T $ and therefore must be in $ S
    \cap T $.

    \bigbreak

    {\bf Scalar Multiplication}: Let $ k $ be a scalar constant. Let $ v \in
    S \cap T $. This also means that $ v \in S, T $. Because $ S $ is a
    subspace and $ v \in S $, then it is closed under scalar multiplication
    and therefore $ kv \in S $. The same goes for $ T $. Since $ kv \in S, T
    $, then it must also be in $ S \cap T $.

    \bigbreak

    Since all the conditions for vector spaces are satisfied, the intersection
    of two subspaces must also be a subspace.

\end{enumerate}

\section*{Problem 2}

\textit{If $C(B)$ is a subspace of $N(A)$ for $3\times 3$ matrices $A$ and
$B$, then $AB$ must be what?}

\bigbreak

The matrix product $ AB $ can be written in the column method of matrix
multiplication
$$ AB = A \begin{pmatrix}
    b_1 & b_2 & b_3
\end{pmatrix} 
= \begin{pmatrix}
    A b_1 & A b_2 & A b_3
\end{pmatrix} $$
Knowing that the column space of $ B $ is within the null space of $ A $, this
further reduces to
$$ \begin{pmatrix}
    A b_1 = 0 & A b_2 = 0 & A b_3 = 0
\end{pmatrix}
=
\begin{pmatrix}
    \vec{0} & \vec{0} & \vec{0}
\end{pmatrix} $$
Therefore, $ AB $ must be the $ 3 \times 3 $ zero matrix.

\section*{Problem 4}

\section*{Problem 5}

\section*{Problem 6}

\section*{Problem 7}

\end{document}