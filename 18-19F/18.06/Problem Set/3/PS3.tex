\documentclass{article}
\usepackage{tikz}
\usepackage{float}
\usepackage{enumerate}
\usepackage{amsmath}
\usepackage{amsthm}
\usepackage{amsfonts}
\usepackage{bm}
\usepackage{indentfirst}
\usepackage{siunitx}
\usepackage[utf8]{inputenc}
\usepackage{graphicx}
\graphicspath{ {Images/} }
\usepackage{float}
\usepackage{mhchem}
\usepackage{chemfig}
\allowdisplaybreaks

\title{18.06 Problem Set 3}
\author{Robert Durfee}
\date{September 26, 2018}

\begin{document}

\maketitle

\section*{Problem 1}

\textit{Which of the following sets are vectors spaces (with the usual
definition of $\pm$ and multiplication by real scalars), and why? }

\begin{enumerate}
    \item \textit{The set of all functions $f(x)$ whose integral
    $\int_{-\infty}^\infty f(x) dx$ is zero.}

    \bigbreak
    
    {\bf Zero}: The function $ f(x) = 0 $ is in the set of all functions
    whose integral is $ \int_{-\infty}^{\infty} f(x) dx = 0 $ as $
    \int_{-\infty}^{\infty} 0 dx = 0 $.
    
    \bigbreak
    
    {\bf Addition}: Let $ f(x) $ and $ g(x) $ be in the set described above.
    Then, the following must be true
    $$ \int\limits_{-\infty}^{\infty} f(x) dx = 0,\,
    \int\limits_{-\infty}^{\infty} g(x) dx = 0 $$
    Using this information, checking if $ f(x) + g(x) $ is also in the set
    described above:
    $$ \int\limits_{-\infty}^{\infty} f(x) + g(x) dx =
    \int\limits_{-\infty}^{\infty} f(x) dx + \int\limits_{-\infty}^{\infty}
    g(x) dx = 0 + 0 = 0 $$
    Therefore, this set is closed under addition.

    \bigbreak

    {\bf Scalar Multiplication}: Let $ f(x) $ be in the set described above
    and $ k $ be some real-valued constant. The following must be true
    $$ \int\limits_{-\infty}^{\infty} f(x) dx = 0 $$
    Using this information, checking if $ k f(x) $ is also in the set
    described above:
    $$ \int\limits_{-\infty}^{\infty} k f(x) dx = k
    \int\limits_{-\infty}^{\infty} f(x) dx = k \cdot 0 = 0 $$
    Therefore, this set is closed under scalar multiplication.

    \bigbreak

    Since all the conditions for vector spaces are satisfied, this must be a
    vector space.

    \item \textit{Given a subspace $V$ of $ \mathbb{R}^n$ and an $m\times n$
    matrix $A$, the set of all vectors $Ax$ for all $x \in V$.}
    
    \bigbreak

    {\bf Zero}: The zero vector can be represented by the following, for example:
    $$ A = I_n,\, x = \vec{0}_n $$
    As the vector subspace $ V $ must contain the zero vector $ \vec{0}_n $.
    
    \bigbreak

    {\bf Addition}: Let $ x, y \in V $. Checking if $ Ax + Ay $ is also in
    the set defined above
    $$ Ax + Ay = A (x + y) $$
    Given the fact that $ x, y \in V $ and $ V $ is a vector subspace, then $
    x + y \in V $ also. As a result, the set is closed under addition.

    \bigbreak

    {\bf Scalar Multiplication}: Let $ x \in V $ and $ k \in \mathbb{R} $.
    Checking if $ k (Ax) $ is also in the set defined above
    $$ k (Ax) = A (kx) $$
    Given the fact that $ x \in V $ and $ V $ is a vector subspace, then $ kx
    \in V $ as well. Therefore, the set is closed under multiplication.

    \bigbreak

    Since all the conditions for vector spaces are satisfied, this must be a
    vector space.

    \item \textit{The set of 3-component vectors whose components sum to 1.}

    \bigbreak

    {\bf Zero}: Let $ v $ be a vector contained in the set described above.
    $ v $ can be written element-wise
    $$ v = \begin{pmatrix}
        v_1 \\
        v_2 \\
        v_3
    \end{pmatrix} $$
    Where $ v_1 + v_2 + v_3 = 1 $ There is no for the zero vector to be
    present in this set as $ 0 + 0 + 0 \neq 1 $.

    \bigbreak

    Since the zero vector is not present in the set, it cannot be a vector
    space.

    \item \textit{The set of 2-component vectors whose components have product 0.}

    \bigbreak

    {\bf Zero}: Let $ v $ be a vector contained in the set described above. $
    v $ can be written element-wise
    $$ v = \begin{pmatrix}
        v_1 \\
        v_2
    \end{pmatrix} $$
    Where $ v_1 \cdot v_2 = 0 $. The zero vector is clearly in this set as $
    0 \cdot 0 = 0 $.

    \bigbreak

    {\bf Addition}: Let $ v $ and $ w $ be in the set described above. Then,
    the following must be true
    $$ v_1 \cdot v_2 = 0,\, w_1 \cdot w_2 = 0 $$
    Using this information, checking if $ v + w $ is also in the set
    described above
    $$ v + w = \begin{pmatrix}
        v_1 + w_1 \\
        v_2 + w_2
    \end{pmatrix} $$ 
    Checking component conditions
    $$ (v_1 + w_1) \cdot (v_2 + w_2) = v_1 v_2 + v_1 w_2 + w_1 v_2 + w_1 w_2
    = v_1 w_2 + w_1 v_2 $$
    This is \textit{not} guaranteed to be zero as $ v_1 v_2 = 0 $ if only $
    v_1 = 0 $ and $ w_1 w_2 = 0 $ if only $ w_2 = 0 $. Therefore, the above
    expression would reduce further to $ w_1 v_2 $ which wouldn't necessarily
    be zero. Therefore, this set is not closed under addition.
    
    \bigbreak

    Since this set is not closed under addition, this is not a vector space.

    \item \textit{The union of two subspaces $S$ and $T$: all vectors that
    are in *either* $S$ or $T$.}

    \bigbreak

    {\bf Zero}: Since both $S$ and $T$ are vector subspaces, then the zero vector
    must be in each of them and therefore in there union.

    \bigbreak

    {\bf Addition}: Since $ S $ is not fully contained in $ T $, there is
    some $ s \in S $ yet $ s \notin T $. Similarly for $ t \in T $ yet $ t
    \notin S $.
    
    Proof by contradiction. Assume that $ S \cup T $ is a subspace. Then,
    there must exist two vectors $ s, t \in S \cup T $ such that $ s + t \in
    S \cup T $. Thus, there is some vector $ s + t = s' $ that lies in either
    $ S $ or $ T $.
    
    If $ s' \in S $, then $ s + t = s' \in S $ and since $ s, s' \in S $,
    then, because $ S $ is a subspace and closed under addition, the difference
    $ s' - s = t \in S $. But this contradicts the choice for $ t \in T $ but
    $ t \notin S $.

    From this, $ s + t \in T $. Then $ s + t = t' \in T $ and since $ t, t'
    \in T $, then, because $ T $ is also a subspace and closed under
    addition, the difference $ t' - t = s \in T $. But this also contradicts
    the choice for $ s \in S $ but $ s \notin T $.

    Therefore, the union is not closed under addition.

    \bigbreak

    Since the union is not closed under addition, it cannot be a vector space.

    \item \textit{The intersection of two subspaces $S$ and $T$: all vectors
    that are in *both* $S$ and $T$.}

    \bigbreak

    {\bf Zero}: Since both $ S $ and $ T $ are vector subspaces, they must
    both contain the zero vector. Thus, that zero vector must be in the
    intersection.

    \bigbreak

    {\bf Addition}: Let $ s,t \in S \cap T $. This also means that $ s \in S,
    T $ and $ t \in S, T $. Because $ S $ is a subspace and $ s $ and $ t $
    are both within it, then $ s + t \in S $. Furthermore, because $ T $ is a
    subspace and $ s $ and $ t $ are both within it, then $ s + t \in T $.
    Therefore the sum is in both $ S $ and $ T $ and therefore must be in $ S
    \cap T $.

    \bigbreak

    {\bf Scalar Multiplication}: Let $ k $ be a scalar constant. Let $ v \in
    S \cap T $. This also means that $ v \in S, T $. Because $ S $ is a
    subspace and $ v \in S $, then it is closed under scalar multiplication
    and therefore $ kv \in S $. The same goes for $ T $. Since $ kv \in S, T
    $, then it must also be in $ S \cap T $.

    \bigbreak

    Since all the conditions for vector spaces are satisfied, the intersection
    of two subspaces must also be a subspace.

\end{enumerate}

\section*{Problem 2}

\textit{If $C(B)$ is a subspace of $N(A)$ for $3\times 3$ matrices $A$ and
$B$, then $AB$ must be what?}

\bigbreak

The matrix product $ AB $ can be written in the column method of matrix
multiplication
$$ AB = A \begin{pmatrix}
    b_1 & b_2 & b_3
\end{pmatrix} 
= \begin{pmatrix}
    A b_1 & A b_2 & A b_3
\end{pmatrix} $$
Knowing that the column space of $ B $ is within the null space of $ A $, this
further reduces to
$$ \begin{pmatrix}
    A b_1 = 0 & A b_2 = 0 & A b_3 = 0
\end{pmatrix}
=
\begin{pmatrix}
    \vec{0} & \vec{0} & \vec{0}
\end{pmatrix} $$
Therefore, $ AB $ must be the $ 3 \times 3 $ zero matrix.

\section*{Problem 4}

\textit{Consider the vector space $V$ of infinitely differentiable
real-valued functions $f(x)$ on the real line $x\in \mathbb{R}$ (this vector
space is commonly called "$C^\infty$"). The analogue of a "matrix" acting on
such a space is a *linear operator* $\hat{A}$, which satisfies
$\hat{A}(\alpha f + \beta g) = \alpha \hat{A} f + \beta \hat{A} g$ for any
functions $f$ and $g$ in $V$ and any scalars $\alpha$ and $\beta$.}

\subsection*{Part A}

\textit{What is the null space $N(\hat{A})$ for $\hat{A} = \frac{d^2}{dx^2}$,
defined the same way as we did for matrices?}

\bigbreak

The matrix definition for the nullspace of the linear operation of second
derivative is defined as
$$ N(\hat{A}) = \{ x \in \mathbb{R} \mid \hat{A}x = 0 \} $$
Converting this into terms using derivatives
$$ N\left(\frac{d^2}{dx^2}\right) = \left\{ x \in \mathbb{R} \mid \frac{d^2
x}{dx^2} = 0 \right\} $$
Which means all functions whose second derivative is $ 0 $. For example, any
polynomial of degree less than or equal to $ 1 $.

\subsection*{Part B}

\textit{If the differential equation $\hat{A} u = \frac{d^2 u}{dx^2} = b$ has
a particular solution $u(x)$ for a right-hand-side $b(x)$, what does your
answer from (a) tell you about the uniqueness (or lack thereof) of this
solution? What are the other solutions, if any?}

\bigbreak

Given the columns of the linear operation $ \hat{A} $ are linearly
independent, then the existence of nontrivial solutions in the nullspace
would conclude that the solutions to $ \hat{A} u = b $ is not unqiue.

Let $ v \in N(A) $. Then
$$ \hat{A}(u + v) = \hat{A} u + \hat{A} v = b + \hat{A} v = b + 0 = b $$
The solution $ u $ plus a scalar multiple of any element in the nullspace
will also be a solution.

\section*{Problem 5}

$$ A = \begin{pmatrix}
    0 & 1 & 2 & 3 & 4 \\
    0 & 1 & 2 & 4 & 6 \\
    0 & 0 & 0 & 1 & 2
\end{pmatrix} $$

\subsection*{Part A}

\textit{Find the rref form of A.}

\bigbreak

$$ \mathrm{rref} (A) = \begin{pmatrix}
    0 & 1 & 2 & 0 & -2 \\
    0 & 0 & 0 & 1 & 2 \\
    0 & 0 & 0 & 0 & 0
\end{pmatrix} $$

\subsection*{Part B}

\textit{Find the special solutions (a basis) for $N(A)$.}

\bigbreak

Solutions to the $ \mathrm{rref} (A) $ can be written in equations
\begin{align*}
    x_1 &= x_1 \\
    x_2 &= -2 x_3 + x_5 \\
    x_3 &= x_3 \\
    x_4 &= -2 x_5 \\
    x_5 &= x_5
\end{align*}
Which is the same as
$$ c_1 \begin{pmatrix}
    1 \\
    0 \\
    0 \\
    0 \\
    0
\end{pmatrix}
+ c_2 \begin{pmatrix}
    0 \\
    -2 \\
    1 \\
    0 \\
    0
\end{pmatrix}
+ c_3 \begin{pmatrix}
    0 \\
    2 \\
    0 \\
    -2 \\
    1
\end{pmatrix}
,\quad
\left\{
    \begin{pmatrix}
        1 \\
        0 \\
        0 \\
        0 \\
        0
    \end{pmatrix},
    \begin{pmatrix}
        0 \\
        -2 \\
        1 \\
        0 \\
        0
    \end{pmatrix},
    \begin{pmatrix}
        0 \\
        2 \\
        0 \\
        -2 \\
        1
    \end{pmatrix}
\right\} $$

\section*{Problem 6}

\textit{How is the nullspace $N(C)$ related to $N(A)$ and $N(B)$ if $C =
\begin{pmatrix} A \\ B \end{pmatrix}$? ($A$ is $m \times n$ and $B$ is $p
\times n$.)}

\bigbreak

The nullspace of $ C $ is defined as
$$ N(C) = N(A) \cap N(B) $$
That is, the nullspace of $ C $ contains all the basis vectors that are
present in both $ A $ and $ B $.

\section*{Problem 7}

\subsection*{Part A}

\textit{Explain why $N(B)$ must be a subspace of $N(AB)$ for any $m \times n$
matrix $A$ and any $n \times p$ matrix $B$.}

\bigbreak

The nullspace of $ AB $ is defined as
$$ N(AB) = \{ x \in \mathbb{R}^p \mid (AB)x = 0 \} $$
The nullspace of $ B $ is defined as
$$ N(B) = \{ x \in \mathbb{R}^p \mid Bx = 0 \} $$
Therefore, any $ x $ that satisfies the condition for the nullspace of $ B $
must also satisfy the condition for the nullspace of $ A $ because
$$ (AB) x = A (B x) = A \cdot \vec{0} = \vec{0} $$
From this, it is clear that all vectors in the nullspace of $ B $ are in the
nullspace of $ AB $. There is no need to prove the properties of a subspace
hold as, by definition, a nullspace contains the zero vector and is closed
under addition and scalar multiplication.

\subsection*{Part B}

\textit{What must be true of $A$ to have $N(AB) = N(B)$?}

\bigbreak

$ A $ must not create any additional vectors in the nullspace basis after
multiplying with $ B $. This will certainly not occur if $ A $ has all of
its pivots. That is, the rank of $ A $ is equal to the number of columns.

\subsection*{Part C}

\textit{We know from class that the dimension of the nullspace (the number of
special solutions) = \#columns - rank. What does your answer from (a) tell you
about the relationship between the rank of $AB$ and the rank of $B$? (For
example, is one bigger than the other?)}

\bigbreak

From Part A, the nullspace of $ B $ is at least as large as the nullspace of
$ AB $. Also, both $ AB $ and $ B $ must have the same number of columns.
Therefore, the rank of $ AB $ is less than or equal to the rank of $ B $.

\end{document}