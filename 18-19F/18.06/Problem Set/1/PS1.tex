\documentclass{article}
\usepackage{tikz}
\usepackage{float}
\usepackage{enumerate}
\usepackage{amsmath}
\usepackage{amsthm}
\usepackage{bm}
\usepackage{indentfirst}
\usepackage{siunitx}
\usepackage[utf8]{inputenc}
\usepackage{graphicx}
\graphicspath{ {Images/} }
\usepackage{float}
\usepackage{mhchem}
\usepackage{chemfig}
\allowdisplaybreaks

\title{18.06 Problem Set 1}
\author{Robert Durfee}
\date{September 12, 2018}

\begin{document}

\maketitle

\section*{Problem 1}

\subsection*{Part A}

\textit{If $ A $ is a $ 3 \times 4 $ matrix, $ B $ is $ 4 \times 5 $, $ x $ is $
4 \times 1 $, and $ r $ is $ 1 \times 3 $, which of the following make sense and
(for those that make sense) what is the shape of the result?}

\begin{enumerate}
    \item \textit{$ A^2 $, $ AB $, and/or $ BA $?}

        $ A^2 $ : $ (3 \times 4) (3 \times 4) $. This does not make sense.

        $ AB $ : $ (3 \times 4) (4 \times 5) $. Result dimensions: $ (3 \times
        5) $.

        $ BA $ : $ (4 \times 5) (3 \times 4) $. This does not make sense.

    \item \textit{$ 3x + A $ and/or $ 3x + x $?}

        $ 3x + A $ : $ (4 \times 1) + (3 \times 4) $. This does not make sense.

        $ 3x + x $ : $ (4 \times 1) + (4 \times 1) $. Result dimensions: $ (4
        \times 1) $.

    \item \textit{$ Ax $, $ Bx $, $ Ar $, $ Br $, $ xA $, $ xB $, $ rA $, and/or
        $ rB $?}

        $ Ax $ : $ (3 \times 4) (4 \times 1) $. Output dimensions: $ (3 \times
        1) $.

        $ Bx $ : $ (4 \times 5) (4 \times 1) $. This does not make sense.

        $ Ar $ : $ (3 \times 4) (1 \times 3) $. This does not make sense.

        $ Br $ : $ (4 \times 5) (1 \times 3) $. This does not make sense.

        $ xA $ : $ (4 \times 1) (3 \times 4) $. This does not make sense.

        $ xB $ : $ (4 \times 1) (4 \times 5) $. This does not make sense.

        $ rA $ : $ (1 \times 3) (3 \times 4) $. Output dimensions: $ (1 \times
        4) $.

        $ rB $ : $ (1 \times 3) (4 \times 5) $. This does not make sense.

    \item \textit{$ xx $, $ xr $, and/or $ rx $?}

        $ xx $ : $ (4 \times 1) (4 \times 1) $. This does not make sense.

        $ xr $ : $ (4 \times 1) (1 \times 3) $. Output dimensions: $ (4 \times
        3) $.

        $ rx $ : $ (1 \times 3) (4 \times 1) $. This does not make sense.
\end{enumerate}

\subsection*{Part B}

\textit{Given the matrices from before, which of the following make sense, and
(for those that make sense), what is the shape of the result?}

\begin{enumerate}
    \item \textit{$ A^T A $ and/or $ A A^T $?}

        $ A^T A $ : $ (4 \times 3) (3 \times 4) $. Output dimensions: $ (4
        \times 4) $.

        $ A A^T $ : $ (3 \times 4) (4 \times 3) $. Output dimensions: $ (3
        \times 3) $.

    \item \textit{$ x^T x $ and/or $ x x^T $?}

        $ x^T x $ : $ (1 \times 4) (4 \times 1) $. Output dimensions: $ (1
        \times 1) $.

        $ x x^T $ : $ (4 \times 1) (1 \times 4) $. Output dimensions: $ (4
        \times 4) $.
\end{enumerate}

\section*{Problem 2}

\subsection*{Part A}

\textit{Give an exact count (a formula in terms of $ m $, $ n $, $ p $) of the
number of scalar multiplications required to compute the matrix product $ AB $,
where $ A $ is an $ m \times n $ matrix and  $ B $ is an $ n \times p $ matrix.}

\bigbreak

Each element of the resulting matrix $ AB $, where $ A $ is an $ m \times n $
matrix and $ B $ is an $ n \times p $ matrix is given by $ n $. The number of
elements in the resulting matrix $ AB $ is given by the outer dimensions $ m
\times p $. Therefore, the total number of scalar multiplications is given by $
m n p $ as each of the $ mp $ elements has $ n $ scalar multiplications.

\subsection*{Part B}

\textit{Give an exact count (a formula in terms of $ m $) of the
number of scalar multiplications required to compute the matrix product $ Ax $,
where $ A $ is an $ m \times m $ matrix and  $ x $ is an $ m $-component vector.
Explain how this is equivalent to your answer from part (a) in a special case.}

\bigbreak

Each element of the resulting matrix $ Ax $, where $ A $ is an $ m \times m $
matrix and $ x $ is an $ m \times 1 $ vectors is given by $ m $. The number of
elements in the resulting matrix $ Ax $ is given by the outer dimensions $ m
\times 1 $. Therefore, the total number of scalar multiplications is given by $
m^2 $ as each of the $ m $ elements has $ m $ scalar multiplications.

\subsection*{Part C}

\textit{ Computing $ ABx $ can be done by $ (AB)x $ or by  $ A(Bx) $ because
matrix multiplication is associative. If $ A $ and $ B $ are $ 1000 \times 1000
$ matrices and $ x $ is a 1000-component vector, explain why your answers from
(a) and (b) imply that one of these ways of computing $ ABx $ is much faster
than the other way.}

\bigbreak

To calculate $ AB $, where $ A $ and $ B $ are $ 1000 \times 1000 $ matrices,
from above it must require $ 1000^3 $ scalar multiplications. Then, to finish
the computation of $ (AB)x $, it will require $ 1000^2 $ scalar multiplications.
In total, there will be $ 1000^3 + 1000^2 $ scalar multiplications.

To calculate $ Bx $, where $ B $ is a $ 1000 \times 1000 $ matrix and $ x $ is
a $ 1000 \times 1 $ vector, from above it must require $ 1000^2 $ scalar
multiplications. Then, to finish the computation of $ A(Bx) $, it will require $
1000^2 $ more scalar multiplications. In total, there will be $ 2 \cdot 1000^2 $
scalar multiplications. This is considerably fewer than the other method.

\section*{Problem 3}

\textit{Consider Gaussian elimination on the following system of equations:}
$$ 2x + 5y + z = 0 $$
$$ 4x + dy + z = 2 $$
$$ y - z = 3 $$

\subsection*{Part A}

\textit{What number $ d $ forces you to do a row exchange during elimination,
and what (non-singular) triangular system do you obtain for that $ d $?}

\bigbreak

When $ d = 10 $, there must be a row exchange during elimination. The original
matrix A:
\[
    A = \begin{pmatrix}
        2 & 5 & 1 \\
        4 & 10 & 1 \\
        0 & 1 & -1
    \end{pmatrix}
\]
The first elimination matrix:
\[
    E_1 = \begin{pmatrix}
        1 & 0 & 0 \\
        -2 & 1 & 0 \\
        0 & 0 & 1
    \end{pmatrix}
\]
Applying the first elimination matrix results in:
\[
    E_1 A = \begin{pmatrix}
        2 & 5 & 1 \\
        0 & 0 & -1 \\
        0 & 1 & -1
    \end{pmatrix}
\]
Swapping rows 2 and 3:
\[
    U = \begin{pmatrix}
        2 & 5 & 1 \\
        0 & 1 & -1 \\
        0 & 0 & -1
    \end{pmatrix}
\]
Full triangular system when $ d = 10 $:
\[
    \begin{pmatrix}
        2 & 5 & 1 \\
        4 & 10 & 1 \\
        0 & 1 & -1
    \end{pmatrix}
    =
    \begin{pmatrix}
        1 & 0 & 0 \\
        0 & 0 & 1 \\
        0 & 1 & 0
    \end{pmatrix}
    \begin{pmatrix}
        1 & 0 & 0 \\
        0 & 1 & 0 \\
        2 & 0 & 1
    \end{pmatrix}
    \begin{pmatrix}
        2 & 5 & 1 \\
        0 & 1 & -1 \\
        0 & 0 & -1
    \end{pmatrix}
\]
\subsection*{Part B}

\textit{What value of $ d $ would make this system singular (no third pivot,
i.e. no way to get a triangular system with 3 nonzero values on the diagonal)?}

\bigbreak

When $ d = 11 $, this system becomes singular. The original matrix $ A $:
\[
    A = \begin{pmatrix}
        2 & 5 & 1 \\
        4 & 11 & 1 \\
        0 & 1 & -1
    \end{pmatrix}
\]
The first elimination matrix:
\[
    E_1 = \begin{pmatrix}
        1 & 0 & 0 \\
        -2 & 1 & 0 \\
        0 & 0 & 1
    \end{pmatrix}
\]
Applying the first elimination matrix results in:
\[
    E_1 A = \begin{pmatrix}
        2 & 5 & 1 \\
        0 & 1 & -1 \\
        0 & 1 & -1
    \end{pmatrix}
\]
The second elimination matrix:
\[
    E_2 = \begin{pmatrix}
        1 & 0 & 0 \\
        0 & 1 & 0 \\
        0 & -1 & 1
    \end{pmatrix}
\]
Applying the second elimination matrix results in:
\[
    E_2 E_1 A = \begin{pmatrix}
        2 & 5 & 1 \\
        0 & 1 & -1 \\
        0 & 0 & 0
    \end{pmatrix}
\]
Which has no third pivot and is thus singular.

\section*{Problem 4}

\textit{A system of linear equations $ Ax = b $ cannot have exactly two
solutions. An easy way to see why: if two vectors $ x $ and $ y \neq x $ are two
solutions (i.e. $ Ax = b $ and $ Ay = b $), what is another solution? (Hint: $ x
+ y $ is almost right.)}

\bigbreak

Taking the provided equations $ Ax = b $ and $ Ay = b $ and adding them
together,
$$ Ax + Ay = b + b $$
Simplifying further,
$$ A \left( \frac{x + y}{2} \right) = b $$
Therefore, if $ x $ and $ y $ are a solution to the system of equations, then $
(x + y) / 2 $ must also be a solution. This can be carried further to show
infinite solutions, as a result.

\section*{Problem 5}

\textit{Suppose we want to solve $ Ax = b $ for more than one right-hand side $
b $. For example, suppose}
\[
    \begin{pmatrix}
        1 & 6 & -3 \\
        -2 & 3 & 4 \\
        1 & 0 & -2
    \end{pmatrix}
\]
\textit{and want to solve both $ Ax_1 = b_1 $ and $ Ax_2 = b_2 $ for the
right-hand sides:}
\[
    b_1 = \begin{pmatrix}
        7 \\
        3 \\
        0
    \end{pmatrix}
    b_2 = \begin{pmatrix}
        0 \\
        -2 \\
        1
    \end{pmatrix}
\]

\subsection*{Part A}

\textit{Show that solving both  $ Ax_1 = b_1 $  and $ Ax_2=b_2 $ is equivalent
to solving $ AX = B $ where $ X $ is an unknown matrix (of what shape?) and $ B
$ is a given matrix on the right-hand-side. Give $ B $ explicitly, and relate $
X $ to your desired solutions $ x_1 $ and $ x_2 $.}

\bigbreak

Thinking in terms of matrix $ \times $ columns matrix multiplication, $ AX = B $
can be rewritten as
\[
    AX = \begin{pmatrix}
        A x_1 & A x_2
    \end{pmatrix}
    = B = \begin{pmatrix}
        b_1 & b_2
    \end{pmatrix}
\]
Therefore, $ X $ is an $ n \times 2 $ matrix where $ n $ equals the number of
columns in $ A $. Explicitly,
\[
    X = \begin{pmatrix}
        x_1 & x_2
    \end{pmatrix}
    B = \begin{pmatrix}
        7 & 0 \\
        3 & -2 \\
        0 & 1
    \end{pmatrix}
\]
Where $ x_1 $ and $ x_2 $ are the column vectors that make up the columns of $ X
$.

\subsection*{Part B}

\textit{Solve your $ AX = B $ equation by forming the augmented matrix $ (AB) $,
reducing it to upper-triangular form (once), and doing back-substitution (twice)
to obtain $ X $ and hence  $ x1 $ and $ x2 $.}

\bigbreak

Original augmented matrix:
\[
    \begin{pmatrix}
        A & B
    \end{pmatrix}
    =
    \begin{pmatrix}
        1 & 6 & -3 & 7 & 0 \\
        -2 & 3 & 4 & 3 & -2 \\
        1 & 0 & -2 & 0 & 1
    \end{pmatrix}
\]
First elimination matrix:
\[
    E_1 = \begin{pmatrix}
        1 & 0 & 0 \\
        2 & 1 & 0 \\
        -1 & 0 & 1
    \end{pmatrix}
\]
Applying first elimination matrix yields:
\[
    E_1 \begin{pmatrix}
        A & B
    \end{pmatrix}
    =
    \begin{pmatrix}
        1 & 6 & -3 & 7 & 0 \\
        0 & 15 & -2 & 17 & -2 \\
        0 & -6 & 1 & -7 & 1
    \end{pmatrix}
\]
Second elimination matrix
\[
    E_2 = \begin{pmatrix}
        1 & 0 & 0 \\
        0 & 1 & 0 \\
        0 & 0.4 & 1
    \end{pmatrix}
\]
Applying second elimination matrix yields:
\[
    E_2 E_1 \begin{pmatrix}
        A & B
    \end{pmatrix}
    =
    \begin{pmatrix}
        1 & 6 & -3 & 7 & 0 \\
        0 & 15 & -2 & 17 & -2 \\
        0 & 0 & 0.2 & -0.2 & 0.2
    \end{pmatrix}
\]
Back substitution yields
\[
    X = \begin{pmatrix}
        -2 & 3 \\
        1 & 0 \\
        -1 & 1
    \end{pmatrix}
\]
Therefore,
\[
    x_1 = \begin{pmatrix}
        -2 \\
        1 \\
        -1
    \end{pmatrix}
    x_2 = \begin{pmatrix}
        3 \\
        0 \\
        1
    \end{pmatrix}
\]
\end{document}

