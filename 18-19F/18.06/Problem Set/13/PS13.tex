\documentclass{article}
\usepackage{tikz}
\usepackage{float}
\usepackage{enumerate}
\usepackage{amsmath}
\usepackage{amsthm}
\usepackage{amsfonts}
\usepackage{bm}
\usepackage{indentfirst}
\usepackage{siunitx}
\usepackage[utf8]{inputenc}
\usepackage{graphicx}
\graphicspath{ {Images/} }
\usepackage{float}
\usepackage{mhchem}
\usepackage{chemfig}
\allowdisplaybreaks

\title{18.06 Problem Set 13}
\author{Robert Durfee}
\date{December 5, 2018}

\begin{document}

\maketitle

\section*{Problem 1}

\subsection*{Part A}

Expanding the dot product,
\begin{align*}
  \vec{w_1}(x) \cdot \left(A \vec{w_2}(x)\right)&= \begin{pmatrix}
  u_1(x) \\
  v_1(x)
\end{pmatrix} \cdot \left(\begin{pmatrix}
  0 & \partial / \partial x \\
  \partial / \partial x & 0
\end{pmatrix} \begin{pmatrix}
  u_2(x) \\
  v_2(x)
\end{pmatrix}\right) \\
&= \begin{pmatrix}
  u_1(x) \\
  v_1(x)
\end{pmatrix} \cdot \begin{pmatrix}
  v_2'(x) \\
  u_2'(x)
\end{pmatrix}
\end{align*}
Using the given definition of a dot product,
\begin{align*}
  \begin{pmatrix}
    u_1(x) \\
    v_1(x)
  \end{pmatrix} \cdot \begin{pmatrix}
    v_2'(x) \\
    u_2'(x)
  \end{pmatrix} &= \int_0^L \begin{pmatrix}
    u_1(x) \\
    v_1(x)
  \end{pmatrix}^H \begin{pmatrix}
    v_2'(x) \\
    u_2'(x)
  \end{pmatrix} dx
\end{align*}
Applying integration by parts,
\begin{align*}
  \int_0^L \begin{pmatrix}
    u_1(x) \\
    v_1(x)
  \end{pmatrix}^H \begin{pmatrix}
    v_2'(x) \\
    u_2'(x)
  \end{pmatrix} dx &= \begin{pmatrix}
    u_1(x) \\
    v_1(x)
  \end{pmatrix}^H \begin{pmatrix}
    v_2(x) \\
    u_2(x)
  \end{pmatrix} \Bigg\vert_0^L - \int_0^L \begin{pmatrix}
    u_1'(x) \\
    v_1'(x)
  \end{pmatrix}^H \begin{pmatrix}
    v_2(x) \\
    u_2(x)
  \end{pmatrix} dx
\end{align*}
Assuming $u_k(0) = u_k(L) = 0$, the first term goes to zero,
$$ \int_0^L \begin{pmatrix}
    u_1(x) \\
    v_1(x)
  \end{pmatrix}^H \begin{pmatrix}
    v_2'(x) \\
    u_2'(x)
  \end{pmatrix} dx = - \int_0^L \begin{pmatrix}
    u_1'(x) \\
    v_1'(x)
  \end{pmatrix}^H \begin{pmatrix}
    v_2(x) \\
    u_2(x)
  \end{pmatrix} dx $$
Moving the negative sign into the integral,
$$ \int_0^L \begin{pmatrix}
    u_1(x) \\
    v_1(x)
  \end{pmatrix}^H \begin{pmatrix}
    v_2'(x) \\
    u_2'(x)
  \end{pmatrix} dx = \int_0^L \begin{pmatrix}
    - u_1'(x) \\
    - v_1'(x)
  \end{pmatrix}^H \begin{pmatrix}
    v_2(x) \\
    u_2(x)
  \end{pmatrix} dx $$
Therefore,
$$ \vec{w_1}(x) \cdot \left(A \vec{w_2}(x)\right) = \int_0^L \begin{pmatrix}
    - u_1'(x) \\
    - v_1'(x)
  \end{pmatrix}^H \begin{pmatrix}
    v_2(x) \\
    u_2(x)
  \end{pmatrix} dx $$
Expanding the integrand out yields,
$$ \vec{w_1}(x) \cdot \left(A \vec{w_2}(x)\right) = \int_0^L -\overline{u_1'(x)}
v_2(x) - \overline{v_1'(x)} u_2(x) dx $$
Doing the same for the integrand of $\left(-A \vec{w_1}(x)\right) \cdot
\vec{w_2}(x)$,
$$ \left(-A \vec{w_1}(x)\right) \cdot \vec{w_2}(x) = \int_0^L -\overline{v_1'(x)}
u_2(x) - \overline{u_1'(x)} v_2(x) dx $$
These are the same and from this it is clear,
$$ \vec{w_1}(x) \cdot \left(A \vec{w_2}(x)\right) = \left(-A
\vec{w_1}(x)\right) \cdot \vec{w_2}(x) $$

\subsection*{Part B}

Continuing the algebra given,
\begin{align*}
  \frac{\partial}{\partial t} \lVert w \rVert^2 &= \frac{\partial w}{\partial
  t} \cdot w + w \cdot \frac{\partial w}{\partial t} \\
  &= \left(A w\right) \cdot w + w \cdot \left(A w\right) \\
  &= (A w) \cdot w + (-A w) \cdot w \\
  &= (A w) \cdot w - (A w) \cdot w \\
  &= 0
\end{align*}

\subsection*{Part C}

Given the following for $w \neq 0$,
$$ A w = \lambda w $$
Taking the complex dot product on the left by $w$,
\begin{align*}
  w^H (A w) &= w^H (\lambda w) \\
  (-A w)^H w &= \lambda \lVert w \rVert^2 \\
  (-\lambda w)^H w &= \lambda \Vert w \rVert^2 \\
  -\overline{\lambda} \lVert w \rVert^2 &=  \lambda \lVert w \rVert^2 \\
  -\overline{\lambda} &= \lambda
\end{align*}
This implies that $\lambda$ is purely imaginary.

\section*{Problem 2}

\subsection*{Part A}

There are two ways to perform this operation,
$$ \left(\frac{x_1 x_1^T}{x_1^T x_1}\right)x \quad \mathrm{or} \quad x_1
\left(\frac{x_1^T x}{x_1^T x_1}\right) $$
In the first case, there are $n^2$ proportional operations as matrix
mutliplication is involved. In the second case, only dot products are
involved and thus there are only $n$ proportional operations. The second
choice is best.

\subsection*{Part B}

To compute $c_1$, dot products should be used. For example,
$$ \left(\frac{x_1^T x}{x_1^T x_1}\right) = \frac{1}{x_1^T x_1} \left(x_1^T
c_1 x_1 + x_1^T c_2 x_2 + \cdots + x_1^T c_n x_n\right)$$
Knowing that $x_1, x_2, \ldots, x_n$ are orthogonal, this simplifies to,
$$ \left(\frac{x_1^T x}{x_1^T x_1}\right) = c_1 $$

\subsection*{Part C}

From the definition of an inverse,
$$ X^{-1} X = I $$
This can be written as a column-wise operation,
$$ \begin{pmatrix}
  X^{-1} x_1 & X^{-1} x_2 & \cdots & X^{-1} x_n
\end{pmatrix} = I $$
We want each column to be $e_1, e_2, \ldots, e_n$. We know that $X$ is
orthogonal. Therefore, $X^T x_1$, where $x_1$ is a column of $X$ will be all
zeros except for the first entry.
$$ \begin{pmatrix}
  X^T x_1 & X^T x_2 & \cdots & X^T x_n
\end{pmatrix} = \mathrm{diagonal} $$
The entries are the dot products $x_1^T x_1, x_2^T x_2, \ldots, x_n^T x_n$.
To make these entries equal to $1$, divide by the norm.
$$ \begin{pmatrix}
  \frac{X^T x_1}{x_1^T x_1} & \frac{X^T x_2}{x_2^T x_2} & \cdots & \frac{X^T
  x_n}{x_n^T x_n}
\end{pmatrix} = I $$
Converting this to matrix operations,
$$ \begin{pmatrix}
 \frac{1}{x_1^T x_1} &   &   &   \\
   & \frac{1}{x_2^T x_2} & & \\
   & & \ddots & \\
   & & & \frac{1}{x_n^T x_n}
\end{pmatrix} X^T X = I $$
Therefore,
$$ X^{-1} = \begin{pmatrix}
  \frac{1}{x_1^T x_1} &   &   &   \\
   & \frac{1}{x_2^T x_2} & & \\
   & & \ddots & \\
   & & & \frac{1}{x_n^T x_n}
\end{pmatrix} X^T $$

\end{document}