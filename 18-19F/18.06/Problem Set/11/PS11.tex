\documentclass{article}
\usepackage{tikz}
\usepackage{float}
\usepackage{enumerate}
\usepackage{amsmath}
\usepackage{amsthm}
\usepackage{amsfonts}
\usepackage{bm}
\usepackage{indentfirst}
\usepackage{siunitx}
\usepackage[utf8]{inputenc}
\usepackage{graphicx}
\graphicspath{ {Images/} }
\usepackage{float}
\usepackage{mhchem}
\usepackage{chemfig}
\allowdisplaybreaks

\title{18.06 Problem Set 11}
\author{Robert Durfee}
\date{November 21, 2018}

\begin{document}

\maketitle

\section*{Problem 1}

\subsection*{Part A}

\begin{itemize}
  \item The solution approaches a constant because the real component of the
  eigenvalues is negative and therefore the solution decays.
  \item The angular frequency of oscillation is given by the imaginary
  component of the eigenvalues. In this case, the imaginary component is
  $\sqrt{227} / 2$. Converting to period,
  $$ T = \frac{2 \pi}{\omega} = \frac{4 \pi}{\sqrt{227}} $$
\end{itemize}

\subsection*{Part B}

Given that $v^T x(t)$ is constant, then its time-derivative is $0$. Therefore,
$$ \frac{d v^T x(t)}{dt} = v^T \frac{dx(t)}{dt} = 0 $$
Given the differential equation,
$$ v^T A x = 0 $$
However, this must be true for all $x$, thus
$$ v^T A = A^T v = 0 $$
Therefore, $v$ is in the left nullspace of $A$. The left nullspace in this
case is $1$-dimensional. $v$ is also an eigenvector of $A^T$.

\section*{Problem 2}

\subsection*{Part A}

The Taylor series is given by
$$ \sin(A) = A - \frac{A^3}{3!} + \frac{A^5}{5!} - \frac{A^7}{7!} + \cdots $$
Multiplying by $x$,
$$ \sin(A) x = Ax - \frac{A^3 x}{3!} + \frac{A^5 x}{5!} - \frac{A^7 x}{7!} +
\cdots $$
Given that $x$ is an eigenvector,
$$ \sin(A) x = \lambda x - \frac{\lambda^3 x}{3!} + \frac{\lambda^5 x}{5!} -
\frac{\lambda^7 x}{7!} + \cdots $$
Taking out the $x$ yields,
$$ \sin(A) x = \sin(\lambda) x $$
Therefore, if $\lambda$ is an eigenvalue of $A$ and $x$ an eigenvector, then
$\sin(\lambda)$ is an eigenvalue of $\sin(A)$ and $x$ and eigenvector.

\subsection*{Part B}

Taking the first derivative of $\sin(At)$ with respect to time,
$$ \frac{d}{dt} \sin(At) = A - \frac{A^3 t^2}{2!} + \frac{A^5 t^4}{4!} -
\frac{A^7 t^6}{6!} + \cdots $$
Pulling an $A$ to the front,
$$ \frac{d}{dt} \sin(At) = A \left(I - \frac{(At)^2}{2!} + \frac{(At)^4}{4!}
- \frac{(At)^6}{6!} + \cdots\right) $$
Taking the second derivative,
$$ \frac{d^2}{dt^2} \sin(At) = A \left(-A^2 t + \frac{A^4 t^3}{3!} -
\frac{A^6 t^5}{5!} + \cdots\right) $$
Pulling a $-A$ to the front,
$$ \frac{d^2}{dt^2} \sin(At) = -A^2 \left(At + \frac{(At)^3}{3!} -
\frac{(At)^5}{5!} + \cdots \right) $$
Replacing the Taylor Series for $\sin(At)$,
$$ \frac{d^2}{dt^2} \sin(At) = -A^2 \sin(At) $$

\subsection*{Part C}

\subsection*{Part D}

Computing the first eigenvalue $\lambda_1$,
$$ \frac{\pi}{2} \begin{pmatrix}
  1 & 1 \\
  1 & 1 
\end{pmatrix} \begin{pmatrix}
  1 \\
  1
\end{pmatrix} = \pi \begin{pmatrix}
  1 \\
  1
\end{pmatrix} $$
Thus, the first eigenvalue is $\lambda_1 = \pi$.

Computing the second eigenvalue $\lambda_2$,
$$ \frac{\pi}{2} \begin{pmatrix}
  1 & 1 \\
  1 & 1 
\end{pmatrix} \begin{pmatrix}
  1 \\
  -1
\end{pmatrix} = \begin{pmatrix}
  0 \\
  0
\end{pmatrix} $$
Thus, the second eigenvalue is $\lambda_2 = 0$

Using these eigenvalues and the rule computed in Part A, the corresponding
eigenvalues for $\sin A$ are,
$$ \lambda_1' = \sin \lambda_1 = \sin \pi = 0 $$
$$ \lambda_2' = \sin \lambda_2 = \sin 0 = 0 $$
Since both eigenvalues are zero and $\sin A$ is diagonalizable, then $\sin A$
must be the zero matrix.

\section*{Problem 3}

\subsection*{Part A}

To show that $U$ is orthogonal, we show that,
$$ \left(e^{A}\right)^T \left(e^A\right) = I $$
However, through the Taylor Series for the exponential,
$$ \left(e^A\right)^T = \left(e^{A^T}\right) $$
Furthermore, since $A$ is skew-symmetric, $A^T = -A$. Therefore, we just need
to show that
$$ \left(e^{-A}\right) \left(e^{A}\right) = I $$
Using the Taylor Series for exponential,
$$ \left(I - A + \frac{A^2}{2!} - \frac{A^3}{3!} + \cdots\right) \left(I + A
+ \frac{A^2}{2!} + \frac{A^3}{3!} + \cdots\right) $$
Writing out the first few terms,
$$ I + A + \frac{A^2}{2!} + \frac{A^3}{3!} - A - A^2 - \frac{A^3}{2!} +
\frac{A^2}{2!} + \frac{A^3}{2!} + \frac{A^5}{2!3!} - \frac{A^3}{3!} -
\frac{A^5}{2!3!} + \cdots $$
All terms cancel except $I$. Therefore, $U$ is orthogonal.

\subsection*{Part B}

\subsection*{Part C}

The matrix $iA$ is a symmetric matrix with eigenvalues that are purely real,
so $A$ must have eigenvalues that are purely imaginary and eigenvectors that
are orthogonal and imaginary.

\section*{Problem 4}

\subsection*{Part A}

Since we know two of the three eigenvalues, the third can be found using the
trace of $A$.
$$ \mathrm{trace}(A) = 7 - 2 + 7 = 12 $$
Since $\lambda_1 = -6$ and $\lambda_2 = 6$, $\lambda_3 = 12$. All eigenvalues
must be real because $A$ is symmetric.

\subsection*{Part B}

Using the definition of eigenvectors $(A - \lambda I) x = 0$,
$$ \left(\begin{pmatrix}
  7 & 4 & -5 \\
  4 & -2 & 4 \\
  -5 & 4 & 7
\end{pmatrix} - 12 I\right) x = \begin{pmatrix}
  -5 & 4 & -5 \\
  4 & -14 & 4 \\
  -5 & 4 & 5
\end{pmatrix} x = 0 $$
By looking at the matrix, $x$ must be
$$ x = \begin{pmatrix}
  1 \\
  0 \\
  -1
\end{pmatrix} $$
These vectors must be orthogonal to each because $A$ is symmetric.

\subsection*{Part C}

\subsection*{Part D}

\subsection*{Part E}

\end{document}