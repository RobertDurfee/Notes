\documentclass{article}
\usepackage{tikz}
\usepackage{float}
\usepackage{enumerate}
\usepackage{amsmath}
\usepackage{amsthm}
\usepackage{amsfonts}
\usepackage{bm}
\usepackage{indentfirst}
\usepackage{siunitx}
\usepackage[utf8]{inputenc}
\usepackage{graphicx}
\graphicspath{ {Images/} }
\usepackage{float}
\usepackage{mhchem}
\usepackage{chemfig}
\allowdisplaybreaks

\title{18.06 Problem Set 8}
\author{Robert Durfee}
\date{October 31, 2018}

\begin{document}

\maketitle

\section*{Problem 1}

\textit{In class, we showed that the pseudoinverse $A^+ = V \Sigma^{-1} U^T$
gives the minimum-norm least-squares solution $\hat{x} = A^+ b$.}

\textit{Suppose that $A$ is an $m\times n$ matrix with full column rank. In
this case, $V$ is a ... matrix. By plugging in the SVD $A = U \Sigma V^T$
(and keeping in mind the shapes of the matrices), show explicitly in this
case that}
$$ A^+ = (A^T A)^{-1} A^T $$

\bigbreak

In this case, $V$ is a square matrix such that $V V^T = I$ and $V^T V = I$.
Substituting the SVD for $A$ into the provided expression:
\begin{align*}
  A^{+} &= (A^T A)^{-1} A^T \\
  &= ((U \Sigma V^T)^T(U \Sigma V^T))^{-1} (U \Sigma V^T)^T \\
  &= ((V \Sigma^T U^T)(U \Sigma V^T))^{-1} (V \Sigma^T U^T)
\end{align*}
However, from the SVD, $U^T U = I$, therefore
\begin{align*}
  A^{+} &= (V \Sigma^T \Sigma V^T)^{-1} (V \Sigma^T U^T)
\end{align*}
Furthermore, $\Sigma$ is diagonal therefore $\Sigma^T = \Sigma$
\begin{align*}
  A^{+} &= (V \Sigma^2 V^T)^{-1} (V \Sigma U^T) \\
  &= V (\Sigma^2)^{-1} V^{-1} V \Sigma U^T \\
  &= V (\Sigma^2)^{-1} \Sigma U^T \\
  &= V \Sigma^{-1} U^T
\end{align*}

\section*{Problem 2}

\textit{Recall the polynomial fitting exercise from the least-square fitting
notebook in lecture 12. We saw that one of the things that can go terribly
wrong is overfitting. Here, we will use the SVD to help us understand this
phenomenon.}

\textit{In class, we fitted a polynomial of degree $n$ to 50 data points
generated from a degree-3 polynomial $1 + 2a + 3a^2 + 4a^3$ plus noise, using
the $m \times n$ Vandermonde matrix}
$$ A = \begin{pmatrix}
  1 & a_1 & a_1^2 & \cdots & a_1^{n-1} \\
  1 & a_2 & a_2^2 & \cdots & a_2^{n-1} \\
  1 & a_3 & a_3^2 & \cdots & a_3^{n-1} \\
  \vdots & \vdots & \vdots & \ddots & \vdots \\
  1 & a_m & a_m^2 & \cdots & a_m^{n-1} \\
\end{pmatrix} $$
\textit{and we found that the fit "went crazy" for $n=40$.}

\subsection*{Part A}

\textit{Look at the singular values $\sigma_k$ of $A$. What is the condition
number of the matrix? The matrix $A$ is very ill-conditioned, which means
that the columns are nearly linearly dependent. As you add more and more
columns to $A$ (that is, as you increase $n$), why are they becoming nearly
linearly dependent? What vector are the rightmost columns becoming nearly
parallel to?}

\bigbreak

The condition number of the matrix is $2.778 \cdot 10^{21}$. The columns are
becoming nearly dependent because the rightmost columns are becoming nearly
parallel to the previous columns.

\subsection*{Part B}

\textit{Another symptom of the fact that the matrix $A$ is ill-conditioned is
that our solution $\hat{x}$ from above had huge components (because $\hat{x}
= A^+ b$ divides by $\sigma$ values that are nearly zero). In consequence, a
common technique to cope with overfitting is ridge regression, also called a
Tikhonov regularization and many other names. Instead of minimizing
$\Vert b - Ax \Vert$, we minimize:}
$$ \Vert b - Ay \Vert^2 + \lambda \Vert y \Vert^2 $$
\textit{where $\lambda > 0$ is some constant parameter (whose magnitude
depends on the amount of noise etcetera), corresponding to a penalty term
$\lambda \Vert y \Vert^2$ that tries to make $\Vert y \Vert$ small. Using the
solution to problem 8 of problem set 6, give an equation for the minimum
$\hat{y}$ of the ridge-regularized fitting problem. It should look very
similar to the $A^T A \hat{x} = A^T b$ "normal equations" for ordinary
fitting, and should be identical for $\lambda = 0$.}

\bigbreak

Starting with the provided expression for ridge-regression
$$ \lVert b - Ay \rVert^2 + \lambda \lVert y \rVert^2 $$
This expression can be written as the following to closely match that in
problem set 6.
$$ \lVert b - Ay \rVert^2 + \lVert 0 - \sqrt{\lambda} I y \rVert^2 $$
Using the equation derived from problem set 6, the normal equations are given
by
$$ \begin{pmatrix}
  A^T & -\sqrt{\lambda} I
\end{pmatrix} \begin{pmatrix}
  A \\
  -\sqrt{\lambda} I
\end{pmatrix} \hat{y} = \begin{pmatrix}
  A^T & -\sqrt{\lambda} I
\end{pmatrix} \begin{pmatrix}
  b \\
  \vec{0}
\end{pmatrix} $$
Thus, the solution of $\hat{y}$ is
$$ \hat{y} = (A^T A + \lambda I)^{-1} A^T b $$

\subsection*{Part D}

\textit{For a full column-rank $m\times n$ matrix with SVD $A = U \Sigma
V^T$, write the ridge-regularized solution in a form resembling the
pseudoinverse:}
$$ \hat{y} = V \, (???) \, U^T b $$
\textit{where $(???)$ is some $n\times n$ matrix in terms of the singular
values $\sigma_k$ and the regularization parameter $\lambda$. Your result
should be equivalent to the pseudoinverse $A^+$ but with each singular
value $\sigma_k$ replaced by ..., which effectively makes the
singular values ... and the condition number ... ?}

\bigbreak

Substituting the SVD into the derived equation above,
\begin{align*}
  \hat{y} &= (A^T A + \lambda I)^{-1} A^T b \\
  &= ((U \Sigma V^T)^T(U \Sigma V^T) + \lambda I)^{-1} (U \Sigma V^T)^T b \\
  &= (V \Sigma^T U^T U \Sigma V^T) + \lambda I)^{-1} (V \Sigma^T U^T) b
\end{align*}
Once again, $U^T U = I$ and $\Sigma^T = \Sigma$, thus
\begin{align*}
  \hat{y} = (V \Sigma^2 V^T + \lambda I)^{-1} (V \Sigma U^T) b
\end{align*}
Given that $\Sigma$ is diagonal,
\begin{align*}
  \hat{y} &= (V (\Sigma^2 + \lambda I) V^T)^{-1} (V \Sigma U^T) b \\
  &= V (\Sigma^2 + \lambda I) V^{-1} V \Sigma U^T b \\
  &= V (\Sigma^2 + \lambda I) \Sigma U^T b
\end{align*}
Therefore $(???) = (\Sigma^2 + \lambda I) \Sigma$. Which, I think, is in
simplest terms.

From this, every $\sigma_k$ is replace with
$$ \sigma_k = \frac{\sigma_k}{\sigma_k^2 + \lambda} $$
Which makes the singular values smaller and the condition number smaller.

\end{document}