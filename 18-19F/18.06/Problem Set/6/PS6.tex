\documentclass{article}
\usepackage{tikz}
\usepackage{float}
\usepackage{enumerate}
\usepackage{amsmath}
\usepackage{amsthm}
\usepackage{amsfonts}
\usepackage{bm}
\usepackage{indentfirst}
\usepackage{siunitx}
\usepackage[utf8]{inputenc}
\usepackage{graphicx}
\graphicspath{ {Images/} }
\usepackage{float}
\usepackage{mhchem}
\usepackage{chemfig}
\allowdisplaybreaks

\title{18.06 Problem Set 6}
\author{Robert Durfee}
\date{October 17, 2018}

\begin{document}

\maketitle

\section*{Problem 1}

\textit{Recall that, if $x \in \mathbb{R}^n$, then $\nabla_x f(x)$ (for a
scalar-valued function $f$) is a column vector}
$$ \nabla_x f = \begin{pmatrix}
    \frac{\partial f}{\partial x_1} \\
    \frac{\partial f}{\partial x_2} \\
    \vdots \\
    \frac{\partial f}{\partial x_n}
\end{pmatrix} $$

\subsection*{Part A}

\textit{If $f(x) = \frac{x^T A x}{x^T x}$ for some $n \times n$ matrix $A$
and $x \neq 0$, write $\nabla_x f$ as a matrix expression involving $A$ and
$x$.}

\bigbreak

The quotient rule applies, therefore
$$ \nabla_x \frac{g(x)}{h(x)} = \frac{(\nabla_x g) h(x) - g(x) (\nabla_x
h)}{(h(x))^2} $$
Where, in this case,
$$ g(x) = x^T A x,\, h(x) = x^T x $$
The product rule also applies, therefore
$$ \nabla_x g'(x) h'(x) = (\nabla_x g') h'(x) + g'(x) (\nabla_x h') $$
Where, in this case,
$$ g'(x) = x^T,\, h'(x) = Ax $$
Therefore,
$$ \nabla_x g = Ax + A^T x,\, \nabla_x h = 2x $$
Putting this all together,
$$ \nabla_x f = \frac{(Ax + A^T x) (x^T x) - (x^T A x)(2x)}{(x^T x)^2} $$

\subsection*{Part B}

\textit{For the $f(x)$ from Part A, $f(\alpha x)$ has what relationship to
$f(x)$ for any real $\alpha \neq 0$? It follows that $\nabla_x f$ must be}
orthogonal \textit{to what vector? Check that this is true of your answer
from Part A.}

\bigbreak

Evaluating $f(\alpha x)$:
$$ f(\alpha x) = \frac{(\alpha x)^T A (\alpha x)}{(\alpha x)^T (\alpha x)} $$
Since $\alpha$ is a constant,
\begin{align*}
    f(\alpha x) &= \frac{\alpha^2 x^T A x}{\alpha^2 x^T x}
    &= \frac{x^T A x}{x^T x}
\end{align*}
Therefore $f(\alpha x) = f(x)$. From this, $f(x)$ along $x$, is unchanging
and therefore is represents a level curve. Thus, $\nabla_x f$ is orthogonal
to $x$.

Checking the dot product of $x$ and the answer to Part A
\begin{align*}
    \frac{(Ax + A^T x) (x^T x) - (x^T A x)(2x)}{(x^T x)^2} \cdot x &=
    \frac{(Ax \cdot x + A^T x \cdot x)(x^T x) - (x^T A x)(2x \cdot x)}{(x^T x)^2} \\
    &= \frac{(x^T A^T x + x^T A x)(x^T x)}{(x^T x)^2} - \frac{(x^T A x)(2x^T x)}{(x^T x)^2} \\
    &= \frac{x^T A^T x + x^T A x - 2(x^T A x)}{x^T x} \\
    &= \frac{x^T A x + x^T A x - 2 (x^T A x)}{x^T x} \\
    &= 0
\end{align*}
Therefore, the answer from Part A is orthogonal to $x$.

\section*{Problem 2}

\textit{If $f(A)$ is a scalar function of an $m \times n$ matrix $A$, then it
is useful to define the gradient with respect to the} matrix \textit{as
another $m \times n$ matrix:}
$$ \nabla_A f = \begin{pmatrix}
    \frac{\partial f}{\partial a_{11}} & \frac{\partial f}{\partial a_{12}} & \cdots \\
    \frac{\partial f}{\partial a_{21}} & \frac{\partial f}{\partial a_{22}} & \cdots \\
    \vdots & \vdots & \ddots
\end{pmatrix} $$
\textit{Given this definition, give a matrix expression for $\nabla_A f$ with
$f(A) = x^T A y$ where $x \in \mathbb{R}^m$ and $y \in \mathbb{R}^n$ are
constant vectors.}

\bigbreak

The partial derivative with respect to a single component $a_{ij}$ is given by
$$ \frac{\partial f}{\partial a_{ij}} = x_i y_j $$
Therefore, the entire gradient $\nabla_A f$ is given by the outer product
$$ \nabla_A f = x y^T $$

\section*{Problem 3}

\textit{Suppose that we minimize the length of a vector along a line:}
$$ \min_{a \in \mathbb{R}} || u + \alpha v || $$
\textit{for some nonzero vectors $u, v \in \mathbb{R}^n$, finding the
minimizer $\hat{\alpha}$.}

\subsection*{Part A}

\textit{If we write this in the form of a standard least-square problem $
\min_x || b - Ax ||$, what are $A$, $b$, and $x$ in terms of the above?}

\bigbreak

$$ b = u,\, A = -v,\, x = \alpha $$

\subsection*{Part B}

\textit{Solve the normal equations to find an explicit solution
$\hat{\alpha}$.}

\bigbreak

The normal equation is given by
$$ A^T A \hat{x} = A^T b $$
Substituting the values from above,
$$ (-v)^T (-v) \hat{\alpha} = (-v)^T u $$
Since these are just vectors,
$$ \hat{\alpha} = -\frac{v^T u}{v^T v} $$

\subsection*{Part C}

\textit{At this minimum, $u + \hat{\alpha} v$ is orthogonal to what vector?}

\bigbreak

Substituting the value found for $\hat{\alpha}$
$$ u - \frac{v^T u}{v^T v} v $$
This is simply the difference between $u$ and its projection onto $v$.
Therefore, it is clear that it must be orthogonal to $v$.

\section*{Problem 4}

\textit{Support that we have $m$ data points $\{(a_1, b_1), (a_2, b_2),
\ldots, (a_m, b_m)\}$ that we want to perform a least-square fit to a
function of the following form:}
$$ f(a) = x_1 + x_2 a + x_3 a^2 + x_4 (a - 1)^2 $$
\textit{That is, we want to minimize $\sum_{i = 1}^m [b_i - f(a_i)]^2$ over
all possible $x \in \mathbb{R}^4$.}

\subsection*{Part A}

\textit{Formulate this in matrix form as in class: we are minimizing $|| b -
Ax ||$ for what matrix $A$ and vector $b$?}

\bigbreak

$$ A = \begin{pmatrix}
    1 & a_1 & a_1^2 & (a_1 - 1)^2 \\
    1 & a_2 & a_2^2 & (a_2 - 1)^2 \\
    \vdots & \vdots & \vdots & \vdots \\
    1 & a_m & a_m^2 & (a_m - 1)^2
\end{pmatrix} $$

\subsection*{Part B}

\textit{Give the rank of $A$ and $A^T A$ and a basis for $N(A) = N(A^T A)$.
What does this tell you about the solutions to the normal equations $A^T A
\hat{x} = A^T b$ for the fit coefficients $\hat{x}$?}

\bigbreak

It is clear to see that the 4th column of $A$ is a combination of the other
columns (represented here using the variables $c$) taking the form
$$ c_4 = c_3 - 2 c_2 + c_1 $$
Therefore, the rank of $A$ (and therefore $A^T A$) is 3 and thus the null
space must have dimension 1. To find the basis, consider
$$ A = \begin{pmatrix}
    1 & a_1 & a_1^2 & (a_1 - 1)^2 \\
    1 & a_2 & a_2^2 & (a_2 - 1)^2 \\
    \vdots & \vdots & \vdots & \vdots \\
    1 & a_m & a_m^2 & (a_m - 1)^2
\end{pmatrix} \begin{pmatrix}
    x_1 \\
    x_2 \\
    x_3 \\
    x_4
\end{pmatrix} = \begin{pmatrix}
    0 \\
    0 \\
    \vdots \\
    0
\end{pmatrix} $$
Given the relationship between the columns,
$$ x = \begin{pmatrix}
    -1 \\
    2 \\
    -1 \\
    1
\end{pmatrix} $$
Therefore, the null space is given by the basis
$$ N(A) = N(A^T A) = \left\{ \begin{pmatrix}
    -1 \\
    2 \\
    -1 \\
    1
\end{pmatrix} \right\} $$
This tells us that the solutions to the normal equations is not unique.

\subsection*{Part D}

\textit{If the least-square solution is not unique, Julia find the $\hat{x}$
with minimum length. In this problem, that means that Julia's $\hat{x}$ must
be orthogonal to what vectors?}

\bigbreak

The solution must be orthogonal to all the vectors in the null space.

\section*{Problem 5}

\textit{Project $a_1 = \begin{pmatrix} 1 0 \end{pmatrix}^T$ onto the line
spanned by $a_2 = \begin{pmatrix} 1 2 \end{pmatrix}^T$. Then project the
result back onto the line spanned by $a_1$. Multiply these projection
matrices $P_1P_2$: is this a projection?}

\bigbreak

The projection matrix for $a_2$ is
\begin{align*}
    P_1 &= \frac{a_2 a_2^T}{a_2^T a_2} \\
    &= \frac{\begin{pmatrix} 1 \\ 2 \end{pmatrix} \begin{pmatrix} 1 & 2 \end{pmatrix}}{\begin{pmatrix} 1 & 2 \end{pmatrix} \begin{pmatrix} 1 \\ 2 \end{pmatrix}} \\
    &= \begin{pmatrix} 1 & 2 \\ 2 & 4 \end{pmatrix} / 5 \\
    &= \begin{pmatrix}
        1/5 & 2/5 \\
        2/5 & 4/5
    \end{pmatrix}
\end{align*}
Applying this to $a_1$
$$ \begin{pmatrix}
    1/5 & 2/5 \\
    2/5 & 4/5
\end{pmatrix} \begin{pmatrix}
    1 \\
    0
\end{pmatrix} = \begin{pmatrix}
    1/5 \\
    2/5
\end{pmatrix} $$

The projection matrix for $a_1$ is 
\begin{align*}
    P_2 &= \frac{a_1 a_1^T}{a_1^T a_1} \\
    &= \frac{\begin{pmatrix} 1 \\ 0 \end{pmatrix} \begin{pmatrix} 1 & 0 \end{pmatrix}}{\begin{pmatrix} 1 & 0 \end{pmatrix} \begin{pmatrix} 1 \\ 0 \end{pmatrix}} \\
    &= \begin{pmatrix} 1 & 0 \\ 0 & 0 \end{pmatrix} \\
    &= \begin{pmatrix}
        1 & 0 \\
        0 & 0
    \end{pmatrix}
\end{align*}
Applying this to the result from the previous projection
$$ \begin{pmatrix}
    1 & 0 \\
    0 & 0
\end{pmatrix} \begin{pmatrix}
    1/5 \\
    2/5
\end{pmatrix} = \begin{pmatrix}
    1/5 \\
    0
\end{pmatrix} $$

Multiplying $P_1 P_2$
$$ \begin{pmatrix}
    1/5 & 2/5 \\
    2/5 & 4/5
\end{pmatrix} \begin{pmatrix}
    1 & 0 \\
    0 & 0
\end{pmatrix} = \begin{pmatrix}
    1/5 & 0 \\
    2/5 & 0
\end{pmatrix} $$
Testing if this satisfies the property of a projection $P^2 = P$,
$$ \begin{pmatrix}
    1/5 & 0 \\
    2/5 & 0
\end{pmatrix}^2 = \begin{pmatrix}
    1/25 & 0 \\
    2/25 & 0
\end{pmatrix} \neq \begin{pmatrix}
    1/5 & 0 \\
    2/5 & 0
\end{pmatrix} $$
Therefore, this cannot be a projection.

\section*{Problem 6}

\textit{To find the projection matrix onto the plane $x - y - 2z = 0$, choose
two vectors in that plane (the null space of what matrix?) and make them
columns of $A$ so that the plane is $C(A)$. Then compute the projection of
the point $\begin{pmatrix} 0 & 6 & 12 \end{pmatrix}^T$ onto this plane, and
check your result in Julia.}

\bigbreak

The two following vectors lie in the plane $x - y - 2z = 0$ and form a matrix
$A$ whose column space spans the plane
$$ A = \begin{pmatrix}
    3 & 1 \\
    1 & 1 \\
    1 & 0
\end{pmatrix} $$
The normal equations allow a solution for $\hat{x}$
$$ A^T A \hat{x} = A^T b $$
Substituting $A$
$$ \begin{pmatrix}
    3 & 1 & 1 \\
    1 & 1 & 0
\end{pmatrix} \begin{pmatrix}
    3 & 1 \\
    1 & 1 \\
    1 & 0
\end{pmatrix} \hat{x} = \begin{pmatrix}
    3 & 1 & 1 \\
    1 & 1 & 0
\end{pmatrix} \begin{pmatrix}
    0 \\
    6 \\
    12
\end{pmatrix} $$
$$ \begin{pmatrix}
    11 & 4 \\
    4 & 2
\end{pmatrix} \hat{x} = \begin{pmatrix}
    18 \\
    6
\end{pmatrix} $$
Solving for $\hat{x}$
$$ \hat{x} = \begin{pmatrix}
    2 \\
    -1
\end{pmatrix} $$
Mutliplying by $A$ to get the projection
$$ A\hat{x} = \begin{pmatrix}
    3 & 1 \\
    1 & 1 \\
    1 & 0
\end{pmatrix} \begin{pmatrix}
    2 \\
    -1
\end{pmatrix} = \begin{pmatrix}
    5 \\
    1 \\
    2
\end{pmatrix} $$ 

\section*{Problem 7}

\subsection*{Part A}

\textit{Find the projection matrix $P_C$ onto the column space $C(A)$ for}
$$ A = \begin{pmatrix}
    3 & 6 & 6 \\
    4 & 8 & 8
\end{pmatrix} $$

\bigbreak

The column space is given by the basis
$$ C(A) = \left\{ \begin{pmatrix}
    3 \\
    4
\end{pmatrix} \right\} $$
Therefore, the projection can be defined only by the vector described in the
basis. Thus, the projection matrix is
$$ P_C = \frac{\begin{pmatrix} 3 \\ 4 \end{pmatrix} \begin{pmatrix} 3 & 4
\end{pmatrix}}{\begin{pmatrix} 3 & 4 \end{pmatrix} \begin{pmatrix} 3 \\ 4
\end{pmatrix}} = \begin{pmatrix}
    9 & 12 \\
    12 & 16
\end{pmatrix} / 25 = \begin{pmatrix}
    9/25 & 12/25 \\
    12/25 & 16/25
\end{pmatrix} $$

\subsection*{Part B}

\textit{Find the projection matrix $P_R$ onto the row space $A$. Multiply $B
= P_C A P_R$. Your answer $B$ may be a little surprising at first -- can you
explain it?}

\bigbreak

The row space is given by the basis
$$ R(A) = \left\{ \begin{pmatrix}
    3 \\
    6 \\
    6
\end{pmatrix} \right\} $$
Therefore, the projection can be defined only by the vector described in the
basis. Thus the projection matrix is
$$ P_C = \frac{\begin{pmatrix} 3 \\ 6 \\ 6 \end{pmatrix} \begin{pmatrix} 3 &
6 & 6 \end{pmatrix}}{\begin{pmatrix} 3 & 6 & 6 \end{pmatrix} \begin{pmatrix} 3 \\
6 \\ 6 \end{pmatrix}} = \begin{pmatrix}
    9 & 18 & 18 \\
    18 & 36 & 36 \\
    18 & 36 & 36
\end{pmatrix} / 81 = \begin{pmatrix}
    1/9 & 2/9 & 2/9 \\
    2/9 & 4/9 & 4/9 \\
    2/9 & 4/9 & 4/9
\end{pmatrix} $$

Multiplying these together,
$$ \begin{pmatrix}
    9/25 & 12/25 \\
    12/25 & 16/25
\end{pmatrix} \begin{pmatrix}
    3 & 6 & 6 \\
    4 & 8 & 8
\end{pmatrix} \begin{pmatrix}
    1/9 & 2/9 & 2/9 \\
    2/9 & 4/9 & 4/9 \\
    2/9 & 4/9 & 4/9
\end{pmatrix} = \begin{pmatrix}
    3 & 6 & 6 \\
    4 & 8 & 8
\end{pmatrix} $$
This result makes sense as the first product projects $A$ onto the column
space of $A$ which will leave $A$ unchanged. And the second product projects
anything on the right onto the row space, which, evidently, projects anything
on the right onto the column space, leaving $A$ unchanged as well.

\section*{Problem 8}

\textit{Given two $m \times n$ matrices $A$ and $B$ and two right-hand sides
$b, c \in \mathbb{R}^m$, suppose that we want to minimize:}
$$ f(x) = || b - Ax ||^2 + || c - Bx ||^2 $$
\textit{over $x \in \mathbb{R}^n$. That is, we are minimizing the} sum
\textit{of two least-square fitting problems.}

\subsection*{Part A}

\textit{$|| b ||^2 + || c ||^2 = || w ||^2$ for a vector $ w \in
\mathbb{R}^{2m}$. Give such a $w$.}

\bigbreak

Expanding the expression
$$ b_1^2 + b_2^2 + \ldots + b_m^2 + c_1^2 + c_2^2 + \ldots + c_m^2 $$
This is the same as
$$ || w ||^2 $$
For a $w$ made from $b$ and $c$ stacked
$$ w = \begin{pmatrix}
    b \\
    c
\end{pmatrix} $$

\subsection*{Part B}

\textit{Write down a matrix equation $C\hat{x} = d$ whose solution $\hat{x}$
gives the minimum of $f(x)$. Give explicit formulas for $C$ and $d$ in terms
of $A$, $B$, $b$, and $c$.}

\bigbreak

Using the result from Part A
$$ f(x) = \left|\left| \begin{pmatrix}
    b - Ax \\
    c - Bx
\end{pmatrix} \right| \right|^2 $$
Separating this to mirror the standard least-squares
$$ f(x) = \left|\left| \begin{pmatrix}
    b \\
    c
\end{pmatrix} - \begin{pmatrix}
    A \\
    B
\end{pmatrix} x \right|\right|^2 $$
Since minimizing the length squared is the smae as minimizing the length, the
normal equations can be used.
$$ \begin{pmatrix}
    A^T & B^T
\end{pmatrix} \begin{pmatrix}
    A \\
    B
\end{pmatrix} \hat{x} = \begin{pmatrix}
    A^T & B^T
\end{pmatrix} \begin{pmatrix}
    b \\
    c
\end{pmatrix} $$
Therefore, the expressions for $C$ and $d$ are
$$ C = A^T A + B^T B,\, d = A^T b + B^T c $$

\end{document}
