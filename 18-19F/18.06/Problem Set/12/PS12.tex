\documentclass{article}
\usepackage{tikz}
\usepackage{float}
\usepackage{enumerate}
\usepackage{amsmath}
\usepackage{amsthm}
\usepackage{amsfonts}
\usepackage{bm}
\usepackage{indentfirst}
\usepackage{siunitx}
\usepackage[utf8]{inputenc}
\usepackage{graphicx}
\graphicspath{ {Images/} }
\usepackage{float}
\usepackage{mhchem}
\usepackage{chemfig}
\allowdisplaybreaks

\title{18.06 Problem Set 12}
\author{Robert Durfee}
\date{November 28, 2018}

\begin{document}

\maketitle

\section*{Problem 1}

For the following parts, let $B = -A^T A$. As a result, $B$ is real-symmetric
and negative semi-definite. Therefore, the possible eigenvalues of $B$ are
$\lambda \leq 0$.

\subsection*{Part A}

\begin{itemize}

  \item {\bf $x(t)$ is a nonzero constant vector} is possible. Let the
  eigenvalues of $B$ be all zero and the initial condition $x(0)$ be nonzero.
  Given $B$ is negative semi-definite, negative eigenvalues are possible.

  \item {\bf $x(t)$ monotonically approaches nonzero constant vector} is
  possible. Let there exist a negative eigenvalue for $B$. Let the other
  eigenvalue be zero. This is possible because $B$ is negative semi-definite.
  
  \item {\bf $x(t)$ monotonically approaches zero vector} is possible. Let
  eigenvalues for $B$ be negative. This is possible because $B$ is negative
  semi-definite.

  \item {\bf $\lVert x(t) \rVert$ diverges} is not possible. Given that all
  eigenvalues of $B$ are less than or equal to zero, the solutions must
  decay.

  \item {\bf $x(t)$ oscillates} is not possible. Given that $A$ is real, then
  $B$ is real-symmetric and must have real eigenvalues. Thus there cannot be
  oscillating solutions.

  \item {\bf $x(t)$ has decaying oscillations} is not possible. Given that
  $A$ is real, then $B$ is real-symmetric and must have real eigenvalues.
  Thus there cannot be oscillating solutions.

  \item {\bf $x(t)$ jumps around discontinuously} is not possible. This
  behavior is undefined and not possible for a real-symmetric, negative
  semi-definite matrix $B$.

\end{itemize}

\subsection*{Part B}

\begin{itemize}

  \item {\bf $x(t)$ is a nonzero constant vector}. According to Part A, all
  the eigenvalues must be zero for $B$. An easy way to ensure this is to set
  $A$ to be the $3 \times 2$ zero matrix. 
  $$ A = \begin{pmatrix}
    0 & 0 \\
    0 & 0 \\
    0 & 0
  \end{pmatrix} $$
  Then $B = -A^T A$ must equal,
  $$ B = \begin{pmatrix}
    0 & 0 \\
    0 & 0
  \end{pmatrix} $$
  With eigenvalues and eigenvectors,
  $$ \lambda_1 = 0,\quad \lambda_2 = 0 $$
  $$ x_1 = \begin{pmatrix}
    1 \\
    0
  \end{pmatrix},\quad x_2 = \begin{pmatrix}
    0 \\
    1
  \end{pmatrix} $$
  Solving for coefficients using the initial condition $x(0) =
  \begin{pmatrix} 1 & 2 \end{pmatrix}^T$ yields
  $$ x(t) = \begin{pmatrix}
    1 \\
    0
  \end{pmatrix} + 2 \begin{pmatrix}
    0 \\
    1
  \end{pmatrix} = \begin{pmatrix}
    1 \\
    2
  \end{pmatrix} $$
  Clearly, the solution $x(t)$ is a nonzero constant for all $t$.

  \item {\bf $x(t)$ monotonically approaches nonzero constant vector}.
  According to Part A, there must be one negative eigenvalue and one zero
  eigenvalue. If $A$ equals the following,
  $$ A = \begin{pmatrix}
    1 & 2 \\
    1 & 2 \\
    1 & 2
  \end{pmatrix} $$
  Then $B = -A^T A$ must equal,
  $$ B = \begin{pmatrix}
    -3 & -6 \\
    -6 & -12
  \end{pmatrix} $$
  With eigenvalues and eigenvectors
  $$ \lambda_1 = -15,\quad \lambda_2 = 0 $$
  $$ x_1 = \begin{pmatrix}
    1 \\
    2
  \end{pmatrix},\quad x_2 = \begin{pmatrix}
    -2 \\
    1
  \end{pmatrix} $$
  Solving for coefficients using the initial condition $x(0) =
  \begin{pmatrix} 3 & 1 \end{pmatrix}^T$ yields
  $$ x(t) = e^{-15 t} \begin{pmatrix}
    1 \\
    2
  \end{pmatrix} - \begin{pmatrix}
    -2 \\
    1
  \end{pmatrix} $$
  Clearly, as $t \longrightarrow \infty$, the solution approaches
  $\begin{pmatrix} 2 & -1 \end{pmatrix}^T $

  \item {\bf $x(t)$ monotonically approaches zero vector}. According to Part
  A, there must be two negative eigenvalues. If $A$ equals the following,
  $$ A = \begin{pmatrix}
    1 & 1 \\
    1 & 1 \\
    1 & -1
  \end{pmatrix} $$
  Then $B = -A^T A$ must equal,
  $$ B = \begin{pmatrix}
    -3 & -1 \\
    -1 & -3
  \end{pmatrix} $$
  With eigenvalues and eigenvectors,
  $$ \lambda_1 = -4,\quad \lambda_2 = -2 $$
  $$ x_1 = \begin{pmatrix}
    1 \\
    1
  \end{pmatrix},\quad x_2 = \begin{pmatrix}
    -1 \\
    1
  \end{pmatrix} $$
  Solving for coefficients using the initial condition $x(0) =
  \begin{pmatrix} 1 & 1 \end{pmatrix}^T$ yields,
  $$ x(t) = e^{-4 t} \begin{pmatrix}
    1 \\
    1
  \end{pmatrix} $$
  Clearly, as $t \longrightarrow \infty$, the solution approaches
  $\begin{pmatrix} 0 & 0 \end{pmatrix}^T$ monotonically.

\end{itemize}

\subsection*{Part C}

\begin{itemize}
  
  \item {\bf $x(t)$ is a nonzero constant vector}. The matrix $A A^T$ is
  $$ A A^T = \begin{pmatrix}
    0 & 0 & 0 \\
    0 & 0 & 0 \\
    0 & 0 & 0
  \end{pmatrix} $$
  This matrix has eigenvectors and eigenvalues,
  $$ \lambda_1 = 0,\quad \lambda_2 = 0,\quad \lambda_3 = 0 $$
  $$ x_1 = \begin{pmatrix}
    1 \\
    0 \\
    0
  \end{pmatrix},\quad x_2 = \begin{pmatrix}
    0 \\
    1 \\
    0
  \end{pmatrix},\quad x_3 = \begin{pmatrix}
    0 \\
    0 \\
    1
  \end{pmatrix} $$
  The eigenvectors give the left singular vectors and the eigenvalues give
  the singular values. For the right singular vectors, the matrix $A^T A$ is
  $$ A^T A = \begin{pmatrix}
    0 & 0 \\
    0 & 0
  \end{pmatrix} $$
  This matrix has eigenvectors,
  $$ x_1 = \begin{pmatrix}
    1 \\
    0
  \end{pmatrix},\quad x_2 = \begin{pmatrix}
    0 \\
    1
  \end{pmatrix} $$
  These give the right singular vectors.

  \item {\bf $x(t)$ monotonically approaches nonzero constant vector}. The
  matrix $A A^T$ is
  $$ A A^T = \begin{pmatrix}
    5 & 5 & 5 \\
    5 & 5 & 5 \\
    5 & 5 & 5
  \end{pmatrix} $$
  This matrix has eigenvectors,
  $$ x_1 = \begin{pmatrix}
    1 \\
    -1 \\
    0
  \end{pmatrix},\quad x_2 = \begin{pmatrix}
    -1 \\
    -1 \\
    2
  \end{pmatrix},\quad x_3 = \begin{pmatrix}
    1 \\
    1 \\
    1
  \end{pmatrix} $$
  Normalizing these eigenvectors give the left singular vectors. 
  $$ u_1 = \frac{1}{\sqrt{2}}\begin{pmatrix}
    1 \\
    -1 \\
    0
  \end{pmatrix},\quad u_2 = \frac{1}{\sqrt{6}}\begin{pmatrix}
    -1 \\
    -1 \\
    2
  \end{pmatrix},\quad u_3 = \frac{1}{\sqrt{3}}\begin{pmatrix}
    1 \\
    1 \\
    1
  \end{pmatrix} $$
  For the right singular vectors and singular values, the matrix $A^T A$ is
  $$ A^T A = \begin{pmatrix}
    3 & 6 \\
    6 & 12
  \end{pmatrix} $$
  This matrix has eigenvectors and eigenvalues,
  $$ \lambda_1 = 0,\quad \lambda_2 = 15 $$
  $$ x_1 = \begin{pmatrix}
    -2 \\
    1
  \end{pmatrix},\quad x_2 = \begin{pmatrix}
    1 \\
    2
  \end{pmatrix} $$
  Normalizing these eigenvectors and taking the square root of the
  eigenvalues give the right singular vectors and singular values.
  $$ \sigma_1 = 0,\quad \sigma_2 = \sqrt{15} $$
  $$ v_1 = \frac{1}{\sqrt{5}}\begin{pmatrix}
    -2 \\
    1
  \end{pmatrix},\quad v_2 = \frac{1}{\sqrt{5}}\begin{pmatrix}
    1 \\
    2
  \end{pmatrix} $$

  \item {\bf $x(t)$ monotonically approaches zero vector}. The matrix $A A^T$
  is
  $$ A A^T = \begin{pmatrix}
    2 & 2 & 0 \\
    2 & 2 & 0 \\
    0 & 0 & 2
  \end{pmatrix} $$
  This matrix has eigenvectors,
  $$ x_1 = \begin{pmatrix}
    1 \\
    -1 \\
    0
  \end{pmatrix},\quad x_2 = \begin{pmatrix}
    0 \\
    0 \\
    1
  \end{pmatrix},\quad x_3 = \begin{pmatrix}
    1 \\
    1 \\
    0
  \end{pmatrix} $$
  Normalizing these eigenvectors give the left singular vectors. 
  $$ u_1 = \frac{1}{\sqrt{2}}\begin{pmatrix}
    1 \\
    -1 \\
    0
  \end{pmatrix},\quad u_2 = \begin{pmatrix}
    0 \\
    0 \\
    1
  \end{pmatrix},\quad u_3 = \frac{1}{\sqrt{2}}\begin{pmatrix}
    1 \\
    1 \\
    0
  \end{pmatrix} $$
  For the right singular vectors and singular values, the matrix $A^T A$ is
  $$ A^T A = \begin{pmatrix}
    3 & 1 \\
    1 & 3
  \end{pmatrix} $$
  This matrix has eigenvectors and eigenvalues,
  $$ \lambda_1 = 2,\quad \lambda_2 = 4 $$
  $$ x_1 = \begin{pmatrix}
    -1 \\
    1
  \end{pmatrix},\quad x_2 = \begin{pmatrix}
    1 \\
    1
  \end{pmatrix} $$
  Normalizing these eigenvectors and taking the square root of the
  eigenvalues give the right singular vectors and singular values.
  $$ \sigma_1 = \sqrt{2},\quad \sigma_2 = 2 $$
  $$ v_1 = \frac{1}{\sqrt{2}}\begin{pmatrix}
    -1 \\
    1
  \end{pmatrix},\quad v_2 = \frac{1}{\sqrt{2}}\begin{pmatrix}
    1 \\
    1
  \end{pmatrix} $$

\end{itemize}

\section*{Problem 2}

If $A + A^H$ is positive definite, then
$$ x^H \left(A + A^H\right) x > 0\quad \forall x > 0 $$
Using this fact, let $x$ be an eigenvector and $A$ and $\lambda$ be the
corresponding eigenvalue. Simplifying,
\begin{align*}
  x^H \left(A + A^H\right) x &= x^H A x + x^H A^H x \\
  &= x^H \left(A x\right) + \left(x^H A^H\right) x \\
  &= x^H \left(\lambda x\right) + \left(A x\right)^H x \\
  &= \lambda x^H x + \left(\lambda x\right)^H x \\
  &= \lambda x^H x + \bar{\lambda} x^H x \\
  &= \left(\lambda + \bar{\lambda}\right) \left(x^H x\right) > 0
\end{align*}
Since the sum of a complex number plus its conjugate maintain sign in the
real part and the complex part cancels, the real part must be positive as the
complex dot product $x^H x$ is also positive.

\end{document}