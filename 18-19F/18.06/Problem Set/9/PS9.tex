\documentclass{article}
\usepackage{tikz}
\usepackage{float}
\usepackage{enumerate}
\usepackage{amsmath}
\usepackage{amsthm}
\usepackage{amsfonts}
\usepackage{bm}
\usepackage{indentfirst}
\usepackage{siunitx}
\usepackage[utf8]{inputenc}
\usepackage{graphicx}
\graphicspath{ {Images/} }
\usepackage{float}
\usepackage{mhchem}
\usepackage{chemfig}
\allowdisplaybreaks

\title{18.06 Problem Set 9}
\author{Robert Durfee}
\date{November 7, 2018}

\begin{document}

\maketitle

\section*{Problem 1}

\subsection*{Part A}

Given that $Q^T = Q^{-1}$, the following must be true
$$ Q^{T} Q = I $$
From this, the determinant of the right must be equal to the determinant of
the left as follows
$$ \mathrm{det}(Q^T Q) = \mathrm{det}(I) $$
The determinant of the identity matrix is $1$, thus
$$ \mathrm{det}(Q^T Q) = 1 $$
The left side can be broken apart
$$ \mathrm{det}(Q^T) \mathrm{det}(Q) = 1 $$
From the properties of determinants
$$ \mathrm{det}(Q)^2 = 1 $$
Therefore the determinant must be $+1$ or $-1$.

\subsection*{Part B}

This matrix must be rank deficient. Therefore, the determinant is $0$.

\subsection*{Part C}

Using the following properties,
$$ \mathrm{det}(A) = \mathrm{det}(A^T) $$
$$ \mathrm{det}(c A) = c^n \mathrm{det}(A) $$
THe following is true for a skew-symmetric matrix,
$$ \mathrm{det}(A) = -\mathrm{det}(A) $$
As long as $n$ is odd (which it is in this case). Therefore the determinant
is $0$.

\section*{Problem 2}

Multiplying $B$ by $x_1$
$$ (2A + 3I) x_1 = 2 \lambda_1 x_1 + 3 x_1 = 13 x_1 $$
Therefore $\lambda_1' = 13$ and $x_1' = \begin{pmatrix} 1 & 1 \end{pmatrix}^T$

Multiplying $B$ by $x_2$
$$ (2A + 3I) x_2 = 2 \lambda_2 x_2 + 3 x_2 = x_2 $$
Therefore $\lambda_2' = 1$ and $x_2' = \begin{pmatrix} -2 & 1 \end{pmatrix}^T$

\section*{Problem 3}

\subsection*{Part A}

For a simple $3 \times 3$ matrix, this can be written as
$$ A \begin{pmatrix}
  1 & 0 & 0 \\
  0 & 1 & 0 \\
  0 & 0 & 1
\end{pmatrix} = \begin{pmatrix}
  \lambda_1 & 0 & 0 \\ 
  0 & \lambda_2 & 0 \\
  0 & 0 & \lambda_3
\end{pmatrix} \begin{pmatrix}
  1 & 0 & 0 \\ 
  0 & 1 & 0 \\
  0 & 0 & 1
\end{pmatrix} $$
From this, it is clear to see that $A$ must be diagonal.

\subsection*{Part B}

For a simple $3 \times 3$ matrix, this can be written as
$$ A \begin{pmatrix}
  \star & \star & \star \\
  0 & \star & \star \\
  0 & 0 & \star
\end{pmatrix} = \begin{pmatrix}
  \lambda_1 & 0 & 0 \\ 
  0 & \lambda_2 & 0 \\
  0 & 0 & \lambda_3
\end{pmatrix} \begin{pmatrix}
  \star & \star & \star \\
  0 & \star & \star \\
  0 & 0 & \star
\end{pmatrix} $$
Since an upper triangular matrix's inverse is also upper triangular and the
product of two upper triangular matrices is upper triangular,
$$ A = \begin{pmatrix}
  \lambda_1 & 0 & 0 \\ 
  0 & \lambda_2 & 0 \\
  0 & 0 & \lambda_3
\end{pmatrix} \begin{pmatrix}
  \star & \star & \star \\
  0 & \star & \star \\
  0 & 0 & \star
\end{pmatrix} $$
Thus, it is clear to see that $A$ is also upper triangular.

\section*{Problem 4}

\subsection*{Part A}

Expanding out the product $A^T A$, the trace is simply
$$ \mathrm{trace}(A^T A) = \sum_{i, j} a_{ij}^2 $$
Thus, the Frobenius norm must be
$$ \lVert A \rVert = \sqrt{\sum_{i, j} a_{ij}^2} $$

\subsection*{Part B}

Using the SVD to write $A$,
\begin{align*}
  \mathrm{trace}(A^T A) &= \mathrm{trace}(V \Sigma^T U^T U \Sigma V^T) \\
  &= \mathrm{trace}(V \Sigma^T \Sigma V^T)
\end{align*}
Using properties of the trace,
\begin{align*}
  \mathrm{trace}(A^T A) &= \mathrm{trace}(V^T V \Sigma^T \Sigma) \\
  &= \mathrm{trace}(\Sigma^T \Sigma)
\end{align*}
From Part A, we know that the trace of $\Sigma^T \Sigma$ is just
$$ \sum_{i} \sigma_i^2 $$
Therefore, the relation to the Frobenius norm is
$$ \rVert A \lVert = \sqrt{\sum_i \sigma_t^2} $$

\section*{Problem 5}

\subsection*{Part A}

From the definition provided,
$$ \begin{pmatrix}
  g_{k+2} \\
  g_{k+1}
\end{pmatrix} = \begin{pmatrix}
  1 - w & w \\
  1 & 0
\end{pmatrix} \begin{pmatrix}
  g_{k+1} \\
  g_k
\end{pmatrix} $$
Therefore,
$$ A =  \begin{pmatrix}
  1 - w & w \\
  1 & 0
\end{pmatrix} $$

\subsection*{Part B}

Using the equation for $2 \times 2$ characteristic polynomial,
$$ \lambda^2 - \mathrm{trace}(A) \lambda + \mathrm{det}(A) $$
The eigenvalues are given by
$$ \lambda^2 - (1 - w) \lambda - w $$
The roots are $1$ and $-w$, thus
$$ \lambda_1 = 1,\, \lambda_2 = -w $$
From this, the eigenvectors are given by
$$ (A - \lambda I) = 0 $$
Yielding
$$ x_1 = \begin{pmatrix}
  1 \\
  1
\end{pmatrix},\, x_2 = \begin{pmatrix}
  1 \\
  -1/w
\end{pmatrix} $$

\subsection*{Part C}

As $w$ approaches $-1$, the eigenvalues become $ \lambda_1 = \lambda_2 = 1 $.
And the eigenvectors become $ x_1 = x_2 = \begin{pmatrix} 1 & 1
\end{pmatrix}^T $. Therefore, this will no longer be diagonalizable.

\subsection*{Part D}

As $n$ approaches $\infty$, the sequence converges to a non-zero constant
which does depend on the initial $x$.

\subsection*{Part E}

From the eigenvalues and eigenvectors from before,
$$ A^n = \begin{pmatrix}
  1 & 1 \\
  1 & -1/w
\end{pmatrix} \begin{pmatrix}
  1^n & 0 \\
  0 & (-w)^n
\end{pmatrix} \begin{pmatrix}
  1/(w + 1) & w / (w + 1) \\
  1/(w + 1) & -w / (w + 1)
\end{pmatrix} $$
As $n$ approaches $\infty$, with $0 < w < 1$ this becomes
$$ A^n = \begin{pmatrix}
  1 & 1 \\
  1 & -1/w
\end{pmatrix} \begin{pmatrix}
  1 & 0 \\
  0 & 0
\end{pmatrix} \begin{pmatrix}
  1/(w + 1) & w / (w + 1) \\
  1/(w + 1) & -w / (w + 1)
\end{pmatrix} $$
As a result, 
$$ A^\infty = \begin{pmatrix}
  1/(w + 1) & w / (w + 1) \\
  1/(w + 1) & w / (w + 1)
\end{pmatrix} $$

\subsection*{Part F}

Substituting $w = 0.5$ and multiplying by $\begin{pmatrix} 1 & 0 \end{pmatrix}^T$ yields
$$ A^\infty \begin{pmatrix}
  1 \\
  0
\end{pmatrix} = \begin{pmatrix}
  2/3 \\
  2/3
\end{pmatrix} $$

\subsection*{Part G}

\end{document}