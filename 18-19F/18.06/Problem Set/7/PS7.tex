\documentclass{article}
\usepackage{tikz}
\usepackage{float}
\usepackage{enumerate}
\usepackage{amsmath}
\usepackage{amsthm}
\usepackage{amsfonts}
\usepackage{bm}
\usepackage{indentfirst}
\usepackage{siunitx}
\usepackage[utf8]{inputenc}
\usepackage{graphicx}
\graphicspath{ {Images/} }
\usepackage{float}
\usepackage{mhchem}
\usepackage{chemfig}
\allowdisplaybreaks

\title{18.06 Problem Set 7}
\author{Robert Durfee}
\date{October 24, 2018}

\begin{document}

\maketitle

\section*{Problem 1}

Finding the null space for $A$. Since the matrix $A$ is already in
row-echelon form, the null space can be directly read from the matrix.
$$ N(A) = \left\{ \begin{pmatrix}
    -1 \\
    1 \\
    0 \\
    0
\end{pmatrix},\, \begin{pmatrix}
    1 \\
    0 \\
    1 \\
    0
\end{pmatrix},\, \begin{pmatrix}
    -1 \\
    0 \\
    0 \\
    1
\end{pmatrix} \right\} $$

The first vector of the orthogonal basis $V$ is given by
$$ v_1 = a_1 = \begin{pmatrix}
    -1 \\
    1 \\
    0 \\
    0
\end{pmatrix} $$
The second vector is given by the Gram-Schmidt algorithm
\begin{align*}
    v_2 &= a_2 - \frac{v_1 v_1^T}{v_1^T v_1} a_2 \\
    &= \begin{pmatrix} 1 \\ 0 \\ 1 \\ 0 \end{pmatrix} - \frac{\begin{pmatrix}
    -1 \\ 1 \\ 0 \\ 0 \end{pmatrix} \begin{pmatrix} -1 & 1 & 0 & 0
    \end{pmatrix}}{\begin{pmatrix} -1 & 1 & 0 & 0 \end{pmatrix}
    \begin{pmatrix} -1 \\ 1 \\ 0 \\ 0 \end{pmatrix}} \begin{pmatrix} 1 \\ 0
    \\ 1 \\ 0 \end{pmatrix} \\
    &= \begin{pmatrix} 1/2 \\ 1/2 \\ 1 \\ 0 \end{pmatrix}
\end{align*}
The final vector is given by the Gram-Schmidt algorithm
\begin{align*}
    v_3 &= a_3 - \frac{v_1 v_1^T}{v_1^T v_1} a_3 - \frac{v_2 v_2^T}{v_2^T v_2} a_3 \\
    &= \begin{pmatrix} -1 \\ 0 \\ 0 \\ 1 \end{pmatrix} -
    \frac{\begin{pmatrix} -1 \\ 1 \\ 0 \\ 0 \end{pmatrix} \begin{pmatrix} -1
    & 1 & 0 & 0 \end{pmatrix}}{\begin{pmatrix} -1 & 1 & 0 & 0 \end{pmatrix}
    \begin{pmatrix} -1 \\ 1 \\ 0 \\ 0 \end{pmatrix}} \begin{pmatrix} -1 \\ 0
    \\ 0 \\ 1 \end{pmatrix} - \frac{\begin{pmatrix} 1/2 \\ 1/2 \\ 1 \\ 0
    \end{pmatrix} \begin{pmatrix} 1/2 & 1/2 & 1 & 0
    \end{pmatrix}}{\begin{pmatrix} 1/2 & 1/2 & 1 & 0 \end{pmatrix}
    \begin{pmatrix} 1/2 \\ 1/2 \\ 1 \\ 0 \end{pmatrix}} \begin{pmatrix} -1 \\ 0
    \\ 0 \\ 1 \end{pmatrix}\\
    &= \begin{pmatrix} -1/3 \\ -1/3 \\ 1/3 \\ 1 \end{pmatrix}
\end{align*}
Thus, the orthogonal basis is
$$ \left\{ \begin{pmatrix}
    -1 \\
    1 \\
    0 \\
    0
\end{pmatrix},\, \begin{pmatrix}
    1/2 \\
    1/2 \\
    1 \\
    0
\end{pmatrix},\, \begin{pmatrix}
    -1/3 \\
    -1/3 \\
    1/3 \\
    1
\end{pmatrix} \right\} $$
The length of each vector is given as
$$ || v_1 || = \sqrt{2},\, || v_2 || = \sqrt{\frac{3}{2}},\, || v_3 || =
\sqrt{\frac{4}{3}} $$
Therefore, the orthonormal basis is
$$ \left\{ \begin{pmatrix}
    -\sqrt{1/2} \\
    \sqrt{1/2} \\
    0 \\
    0
\end{pmatrix},\, \begin{pmatrix}
    \sqrt{1/6} \\
    \sqrt{1/6} \\
    \sqrt{2/3} \\
    0
\end{pmatrix},\, \begin{pmatrix}
    -\sqrt{1/12} \\
    -\sqrt{1/12} \\
    \sqrt{1/12} \\
    \sqrt{3/4}
\end{pmatrix} \right\} $$

\section*{Problem 2}

\subsection*{Part A}

Take, for example, the following matrix $A$
$$ A = \begin{pmatrix}
    1 & 2 \\
    1 & 2
\end{pmatrix} $$
These columns are not independent and therefore not full column rank. When
trying to do Gram-Schmidt, the first orthogonal vector is
$$ v_1 = a_1 = \begin{pmatrix} 1 \\ 2 \end{pmatrix} $$
And the second orthogonal vector is
\begin{align*}
    v_2 &= a_2 - \frac{v_1 v_1^2}{v_1^2 v_1} a_2 \\
    &= \begin{pmatrix} 2 \\ 2 \end{pmatrix} - \frac{\begin{pmatrix} 1 \\ 1
    \end{pmatrix} \begin{pmatrix} 1 & 1 \end{pmatrix}}{\begin{pmatrix} 1 & 1
    \end{pmatrix} \begin{pmatrix} 1 \\ 1 \end{pmatrix}} \begin{pmatrix} 2 \\
    2 \end{pmatrix} \\
    &= \begin{pmatrix} 2 \\ 2 \end{pmatrix} - \begin{pmatrix} 2 \\ 2 \end{pmatrix} \\
    &= \begin{pmatrix} 0 \\ 0 \end{pmatrix}
\end{align*}
But the second vector has zero length, therefore a division by zero will
occur when creating the orthonormal basis.

\subsection*{Part B}

This can be avoided if the pivot columns of $A$ are taken instead.

\section*{Problem 3}

\subsection*{Part A}

The $Q$ can be formed by concatenating the matrix $V$ which is the
orthonormal basis for the left null space of $A$.

\subsection*{Part B}

Given this new $Q$, the matrix $R$ takes the following form
$$ R = \begin{pmatrix}
    \hat{R} \\
    0
\end{pmatrix} $$
Where $0$ is a $(m - n) \times n $ zero matrix. Therefore, the dimensions of
$R$ must be $m \times n$.

\section*{Problem 4}

\subsection*{Part A}

The matrix $M$ is an upper triangular matrix, however, it is "mirrored" from
the traditional sense. For example,
$$ M = \begin{pmatrix}
    \star & \star & \star & \star \\
    \star & \star & \star & 0 \\
    \star & \star & 0 & 0 \\
    \star & 0 & 0 & 0
\end{pmatrix} $$

\section*{Problem 5}

\subsection*{Part A}

The first orthogonal polynomial is given by
$$ v_1(x) = a_1 = 1 $$
The second orthogonal polynomial is given through the Gram-Schmidt algorithm
\begin{align*}
    v_2(x) &= a_2 - v_1 \frac{v_1 \cdot a_2}{v_1 \cdot v_1} \\
    &= x - \frac{(1) [(1) \cdot (x)]}{(1) \cdot (1)} \\
    &= x - \frac{\int_0^{\infty} x e^{-1} dx}{\int_0^{\infty} e^{-x} dx} \\
    &= x - \frac{1!}{0!} \\
    &= x - 1
\end{align*}
Lastly, the final orthogonal polynomial is given through the Gram-Schmidt
algorithm
\begin{align*}
    v_3(x) &= a_3 - v_1 \frac{v_1 \cdot a_3}{v_1 \cdot v_1} - v_2 \frac{v_2 \cdot a_3}{v_2 \cdot v_2} \\
    &= x^2 - \frac{(1) [(1) \cdot (x^2)]}{(1) \cdot (1)} - \frac{(x - 1) [(x - 1) \cdot (x^2)]}{(x - 1) \cdot (x - 1)} \\
    &= x^2 - \frac{\int_0^\infty x^2 e^{-x} dx}{\int_0^\infty e^{-x} dx} - \frac{(x - 1) \int_0^\infty (x^3 - x^2) e^{-x} dx}{\int_0^\infty (x^2 -2x + 1) e^{-x} dx} \\
    &= x^2 - \frac{2!}{0!} - \frac{(x - 1) (3! - 2!)}{2! - 2 \cdot 1! + 0!} \\
    &= x^2 - 4x + 2
\end{align*}
The length of each of these polynomials is 
$$ || v_1(x) || = 1,\, || v_2(x) || = 1,\, || v_3(x) || = 2 $$
Therefore, the orthonormal basis for degree three polynomials (using this
definition of the dot product) is given by
$$ q_1(x) = 1,\, q_2(x) = x - 1,\, q_3(x) = \frac{1}{2} x^2 - 2x + 1 $$

\subsection*{Part B}

This is equivalent to projecting the function $f(x)$ onto the span of $x$.
Therefore, the coefficient $\alpha$ is given by
$$ \alpha = \frac{x \cdot f(x)}{x \cdot x} $$
Using the dot product definition provided,
$$ \alpha = \frac{\int_0^\infty x f(x) e^{-x} dx}{\int_0^\infty x^2 e^{-x}
dx} $$
Evaluating the integral piecewise
\begin{align*}
    \alpha &= \frac{\int_0^1 x^2 e^{-x} dx + \int_1^\infty x \cdot 0 \cdot
    e^{-x} dx}{\int_0^\infty x^2 e^{-x} dx} \\
    &= 1 - \frac{5}{2 e}
\end{align*}

\end{document}
