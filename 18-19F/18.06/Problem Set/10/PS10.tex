\documentclass{article}
\usepackage{tikz}
\usepackage{float}
\usepackage{enumerate}
\usepackage{amsmath}
\usepackage{amsthm}
\usepackage{amsfonts}
\usepackage{bm}
\usepackage{indentfirst}
\usepackage{siunitx}
\usepackage[utf8]{inputenc}
\usepackage{graphicx}
\graphicspath{ {Images/} }
\usepackage{float}
\usepackage{mhchem}
\usepackage{chemfig}
\allowdisplaybreaks

\title{18.06 Problem Set 9}
\author{Robert Durfee}
\date{November 7, 2018}

\begin{document}

\maketitle

\section*{Problem 1}

\subsection*{Part A}

From the $QR$ factorization,
$$ A = QR $$
Multiplying each side by $Q^T$,
$$ Q^T A = Q^T Q R $$
But it is known that $Q^T Q = I$, therefore
$$ Q^T A = R $$
Substituting into the expression for $B$,
$$ B = Q^T A Q $$
Once again, $Q^T = Q^{-1} $, therefore
$$ B = Q^{-1} A Q $$
Therefore, matrices $A$ and $B$ are similar to each other and have the same
eigenvalues.

\subsection*{Part B}

The result is converging to a diagonal matrix with the components
corresponding to the eigenvalues of $A$.

\section*{Problem 2}

\subsection*{Part A}

\subsection*{Part B}

\section*{Problem 3}

\subsection*{Part A}

The provided solution to the second-order differential equations
$$ x(t) = c e^{\kappa t} + d e^{-\kappa t} $$
Rearranging,
\begin{align*}
  x(t) &= \frac{1}{2} \left(c e^{\kappa t} + c e^{- \kappa t} + d e^{\kappa
  t} + d e^{-\kappa t} \right) + \frac{1}{2} \left(c e^{\kappa t} - c e^{-
  \kappa t} - d e^{\kappa t} + d e^{-\kappa t} \right) \\
  &= (c + d) \frac{e^{\kappa t} + e^{-\kappa t}}{2} + (c - d) \frac{e^{\kappa
  t} - e^{-\kappa t}}{2} \\
  &= (c + d) \cosh(\kappa t) + (c - d) \sinh(\kappa t) \\
  &= \alpha \cosh(\kappa t) + \beta \sinh(\kappa t)
\end{align*}
Where $\alpha$ and $\beta$ are given by $x(0)$ and $\dot{x}(0)$.

\subsection*{Part B}

Substituting $\kappa = i \omega$,
\begin{align*}
  x(t) &= x(0) \cosh(\kappa t) + \dot{x}(0) \sinh(\kappa t) \\
  &= x(0) \cosh(i \omega t) + \dot{x}(0) \sinh(i \omega t) \\
  &= x(0) \cos(\omega t) + \dot{x}(0) i \sin(\omega t)
\end{align*}

\subsection*{Part C}

\subsection*{Part D}

\subsection*{Part E}

\section*{Problem 4}

\subsection*{Part A}

\subsection*{Part B}

\end{document}