%
% 6.006 problem set 9 solutions template
%
\documentclass[12pt,twoside]{article}

\newcommand{\name}{}

\usepackage{amssymb}
\usepackage{amsmath}
\usepackage{graphicx}
\usepackage{latexsym}
\usepackage{times,url}
\usepackage{cprotect}
\usepackage{listings}
\usepackage{graphicx}
\usepackage[table]{xcolor}
\usepackage[letterpaper]{geometry}
\usepackage{tikz-qtree}
\usepackage{enumerate}

\newcommand{\profs}{Instructors: Zachary Abel, Erik Demaine, Jason Ku}
\newcommand{\subj}{6.006}
\newcommand{\ttt}[1]{{\tt\small #1}}

\definecolor{dkgreen}{rgb}{0,0.6,0}
\definecolor{gray}{rgb}{0.5,0.5,0.5}
\definecolor{mauve}{rgb}{0.58,0,0.82}

\lstset{
  language=Python,
  aboveskip=1pc,
  belowskip=1pc,
  basicstyle={\footnotesize\ttfamily},
  numbers=left,
  showstringspaces=false,
  numberstyle=\tiny\color{gray},
  keywordstyle=\color{blue},
  commentstyle=\color{dkgreen},
  stringstyle=\color{mauve},
}

\tikzset{
  % every node/.style={minimum width=2em,draw,circle},
  % level 1/.style={sibling distance=2cm},
  level distance=1cm,
  edge from parent/.style=
  {draw,edge from parent path={(\tikzparentnode) -- (\tikzchildnode)}},
}

\newif\ifHideSolutions
\newcommand{\solution}[1]{\color{dkgreen}\textbf{Solution: }#1\color{black}}
\newcommand{\rubric}[1]{\color{dkgreen}{\bf Rubric:} #1\color{black}}

% \HideSolutionsfalse
% \ifHideSolutions
%   \renewcommand{\solution}[1]{}
%   \renewcommand{\rubric}[1]{}
% \fi

\newlength{\toppush}
\setlength{\toppush}{2\headheight}
\addtolength{\toppush}{\headsep}

\newcommand{\htitle}[2]{\noindent\vspace*{-\toppush}\newline\parbox{6.5in}
{\textit{Introduction to Algorithms: 6.006}\hfill\name\newline
Massachusetts Institute of Technology \hfill #2\newline
\profs\hfill #1 \vspace*{-.5ex}\newline
\mbox{}\hrulefill\mbox{}}\vspace*{1ex}\mbox{}\newline
\begin{center}{\Large\bf #1}\end{center}}

\newcommand{\handout}[2]{\thispagestyle{empty}
 \markboth{#1}{#1}
 \pagestyle{myheadings}\htitle{#1}{#2}}

\newcommand{\lecture}[3]{\thispagestyle{empty}
 \markboth{Lecture #1: #2}{Lecture #1: #2}
 \pagestyle{myheadings}\htitle{Lecture #1: #2}{#3}}

\newcommand{\htitlewithouttitle}[2]{\noindent\vspace*{-\toppush}\newline\parbox{6.5in}
{\textit{Introduction to Algorithms}\hfill#2\newline
Massachusetts Institute of Technology \hfill 6.006\newline
\profs\hfill Handout #1\vspace*{-.5ex}\newline
\mbox{}\hrulefill\mbox{}}\vspace*{1ex}\mbox{}\newline}

\newcommand{\handoutwithouttitle}[2]{\thispagestyle{empty}
 \markboth{Handout \protect\ref{#1}}{Handout \protect\ref{#1}}
 \pagestyle{myheadings}\htitlewithouttitle{\protect\ref{#1}}{#2}}

\newcommand{\exam}[2]{% parameters: exam name, date
 \thispagestyle{empty}
 \markboth{\hspace{1cm}\subj\ #1\hspace{1in}Name\hrulefill\ \ }%
          {\subj\ #1\hspace{1in}Name\hrulefill\ \ }
 \pagestyle{myheadings}\examtitle{#1}{#2}
 \renewcommand{\theproblem}{Problem \arabic{problemnum}}
}
\newcommand{\examsolutions}[3]{% parameters: handout, exam name, date
 \thispagestyle{empty}
 \markboth{Handout \protect\ref{#1}: #2}{Handout \protect\ref{#1}: #2}
% \pagestyle{myheadings}\htitle{\protect\ref{#1}}{#2}{#3}
 \pagestyle{myheadings}\examsolutionstitle{\protect\ref{#1}} {#2}{#3}
 \renewcommand{\theproblem}{Problem \arabic{problemnum}}
}
\newcommand{\examsolutionstitle}[3]{\noindent\vspace*{-\toppush}\newline\parbox{6.5in}
{\textit{Introduction to Algorithms}\hfill#3\newline
Massachusetts Institute of Technology \hfill 6.006\newline
%Singapore-MIT Alliance \hfill SMA5503\newline
\profs\hfill Handout #1\vspace*{-.5ex}\newline
\mbox{}\hrulefill\mbox{}}\vspace*{1ex}\mbox{}\newline
\begin{center}{\Large\bf #2}\end{center}}

\newcommand{\takehomeexam}[2]{% parameters: exam name, date
 \thispagestyle{empty}
 \markboth{\subj\ #1\hfill}{\subj\ #1\hfill}
 \pagestyle{myheadings}\examtitle{#1}{#2}
 \renewcommand{\theproblem}{Problem \arabic{problemnum}}
}

\makeatletter
\newcommand{\exambooklet}[2]{% parameters: exam name, date
 \thispagestyle{empty}
 \markboth{\subj\ #1}{\subj\ #1}
 \pagestyle{myheadings}\examtitle{#1}{#2}
 \renewcommand{\theproblem}{Problem \arabic{problemnum}}
 \renewcommand{\problem}{\newpage
 \item \let\@currentlabel=\theproblem
 \markboth{\subj\ #1, \theproblem}{\subj\ #1, \theproblem}}
}
\makeatother


\newcommand{\examtitle}[2]{\noindent\vspace*{-\toppush}\newline\parbox{6.5in}
{\textit{Introduction to Algorithms}\hfill#2\newline
Massachusetts Institute of Technology \hfill 6.006 Fall 2018\newline
%Singapore-MIT Alliance \hfill SMA5503\newline
\profs\hfill #1\vspace*{-.5ex}\newline
\mbox{}\hrulefill\mbox{}}\vspace*{1ex}\mbox{}\newline
\begin{center}{\Large\bf #1}\end{center}}

\newcommand{\grader}[1]{\hspace{1cm}\textsf{\textbf{#1}}\hspace{1cm}}

\newcommand{\points}[1]{[#1 points]\ }
\newcommand{\parts}[1]
{
  \ifnum#1=1
  (1 part)
  \else
  (#1 parts)
  \fi
  \ 
}

\newcommand{\bparts}{\begin{problemparts}}
\newcommand{\eparts}{\end{problemparts}}
\newcommand{\ppart}{\problempart}

%\newcommand{\lg} {lg\ }

\setlength{\oddsidemargin}{0pt}
\setlength{\evensidemargin}{0pt}
\setlength{\textwidth}{6.5in}
\setlength{\topmargin}{0in}
\setlength{\textheight}{8.5in}


\newcommand{\Spawn}{{\bf spawn} }
\newcommand{\Sync}{{\bf sync}}

\renewcommand{\cases}[1]{\left\{ \begin{array}{ll}#1\end{array}\right.}
\newcommand{\cif}[1]{\mbox{if $#1$}}
\newcommand{\cwhen}[1]{\mbox{when $#1$}}

\newcounter{problemnum}
\newcommand{\theproblem}{Problem \theproblemsetnum-\arabic{problemnum}}
\newenvironment{problems}{
        \begin{list}{{\bf \theproblem. \hspace*{0.5em}}}
        {\setlength{\leftmargin}{0em}
         \setlength{\rightmargin}{0em}
         \setlength{\labelwidth}{0em}
         \setlength{\labelsep}{0em}
         \usecounter{problemnum}}}{\end{list}}
\makeatletter
\newcommand{\problem}[1][{}]{\item \let\@currentlabel=\theproblem \textbf{#1}}
\makeatother

\newcounter{problempartnum}[problemnum]
\newenvironment{problemparts}{
        \begin{list}{{\bf (\alph{problempartnum})}}
        {\setlength{\leftmargin}{2.5em}
         \setlength{\rightmargin}{2.5em}
         \setlength{\labelsep}{0.5em}}}{\end{list}}
\newcommand{\problempart}{\addtocounter{problempartnum}{1}\item}

\newenvironment{truefalseproblemparts}{
        \begin{list}{{\bf (\alph{problempartnum})\ \ \ T\ \ F\hfil}}
        {\setlength{\leftmargin}{4.5em}
         \setlength{\rightmargin}{2.5em}
         \setlength{\labelsep}{0.5em}
         \setlength{\labelwidth}{4.5em}}}{\end{list}}

\newcounter{exercisenum}
\newcommand{\theexercise}{Exercise \theproblemsetnum-\arabic{exercisenum}}
\newenvironment{exercises}{
        \begin{list}{{\bf \theexercise. \hspace*{0.5em}}}
        {\setlength{\leftmargin}{0em}
         \setlength{\rightmargin}{0em}
         \setlength{\labelwidth}{0em}
         \setlength{\labelsep}{0em}
        \usecounter{exercisenum}}}{\end{list}}
\makeatletter
\newcommand{\exercise}{\item \let\@currentlabel=\theexercise}
\makeatother

\newcounter{exercisepartnum}[exercisenum]
%\newcommand{\problem}[1]{\medskip\mbox{}\newline\noindent{\bf Problem #1.}\hspace*{1em}}
%\newcommand{\exercise}[1]{\medskip\mbox{}\newline\noindent{\bf Exercise #1.}\hspace*{1em}}

\newenvironment{exerciseparts}{
        \begin{list}{{\bf (\alph{exercisepartnum})}}
        {\setlength{\leftmargin}{2.5em}
         \setlength{\rightmargin}{2.5em}
         \setlength{\labelsep}{0.5em}}}{\end{list}}
\newcommand{\exercisepart}{\addtocounter{exercisepartnum}{1}\item}


% Macros to make captions print with small type and 'Figure xx' in bold.
\makeatletter
\def\fnum@figure{{\bf Figure \thefigure}}
\def\fnum@table{{\bf Table \thetable}}
\let\@mycaption\caption
%\long\def\@mycaption#1[#2]#3{\addcontentsline{\csname
%  ext@#1\endcsname}{#1}{\protect\numberline{\csname 
%  the#1\endcsname}{\ignorespaces #2}}\par
%  \begingroup
%    \@parboxrestore
%    \small
%    \@makecaption{\csname fnum@#1\endcsname}{\ignorespaces #3}\par
%  \endgroup}
%\def\mycaption{\refstepcounter\@captype \@dblarg{\@mycaption\@captype}}
%\makeatother
\let\mycaption\caption
%\newcommand{\figcaption}[1]{\mycaption[]{#1}}

\newcounter{totalcaptions}
\newcounter{totalart}

\newcommand{\figcaption}[1]{\addtocounter{totalcaptions}{1}\caption[]{#1}}

% \psfigures determines what to do for figures:
%       0 means just leave vertical space
%       1 means put a vertical rule and the figure name
%       2 means insert the PostScript version of the figure
%       3 means put the figure name flush left or right
\newcommand{\psfigures}{0}
\newcommand{\spacefigures}{\renewcommand{\psfigures}{0}}
\newcommand{\rulefigures}{\renewcommand{\psfigures}{1}}
\newcommand{\macfigures}{\renewcommand{\psfigures}{2}}
\newcommand{\namefigures}{\renewcommand{\psfigures}{3}}

\newcommand{\figpart}[1]{{\bf (#1)}\nolinebreak[2]\relax}
\newcommand{\figparts}[2]{{\bf (#1)--(#2)}\nolinebreak[2]\relax}


\macfigures     % STATE

% When calling \figspace, make sure to leave a blank line afterward!!
% \widefigspace is for figures that are more than 28pc wide.
\newlength{\halffigspace} \newlength{\wholefigspace}
\newlength{\figruleheight} \newlength{\figgap}
\newcommand{\setfiglengths}{\ifnum\psfigures=1\setlength{\figruleheight}{\hruleheight}\setlength{\figgap}{1em}\else\setlength{\figruleheight}{0pt}\setlength{\figgap}{0em}\fi}
\newcommand{\figspace}[2]{\ifnum\psfigures=0\leavefigspace{#1}\else%
\setfiglengths%
\setlength{\wholefigspace}{#1}\setlength{\halffigspace}{.5\wholefigspace}%
\rule[-\halffigspace]{\figruleheight}{\wholefigspace}\hspace{\figgap}#2\fi}
\newlength{\widefigspacewidth}
% Make \widefigspace put the figure flush right on the text page.
\newcommand{\widefigspace}[2]{
\ifnum\psfigures=0\leavefigspace{#1}\else%
\setfiglengths%
\setlength{\widefigspacewidth}{28pc}%
\addtolength{\widefigspacewidth}{-\figruleheight}%
\setlength{\wholefigspace}{#1}\setlength{\halffigspace}{.5\wholefigspace}%
\makebox[\widefigspacewidth][r]{#2\hspace{\figgap}}\rule[-\halffigspace]{\figruleheight}{\wholefigspace}\fi}
\newcommand{\leavefigspace}[1]{\setlength{\wholefigspace}{#1}\setlength{\halffigspace}{.5\wholefigspace}\rule[-\halffigspace]{0em}{\wholefigspace}}

% Commands for including figures with macpsfig.
% To use these commands, documentstyle ``macpsfig'' must be specified.
\newlength{\macfigfill}
\makeatother
\newlength{\bbx}
\newlength{\bby}
\newcommand{\macfigure}[5]{\addtocounter{totalart}{1}
\ifnum\psfigures=2%
\setlength{\bbx}{#2}\addtolength{\bbx}{#4}%
\setlength{\bby}{#3}\addtolength{\bby}{#5}%
\begin{flushleft}
\ifdim#4>28pc\setlength{\macfigfill}{#4}\addtolength{\macfigfill}{-28pc}\hspace*{-\macfigfill}\fi%
\mbox{\psfig{figure=./#1.ps,%
bbllx=#2,bblly=#3,bburx=\bbx,bbury=\bby}}
\end{flushleft}%
\else\ifdim#4>28pc\widefigspace{#5}{#1}\else\figspace{#5}{#1}\fi\fi}
\makeatletter

\newlength{\savearraycolsep}
\newcommand{\narrowarray}[1]{\setlength{\savearraycolsep}{\arraycolsep}\setlength{\arraycolsep}{#1\arraycolsep}}
\newcommand{\normalarray}{\setlength{\arraycolsep}{\savearraycolsep}}

\newcommand{\hint}{{\em Hint:\ }}

% Macros from /th/u/clr/mac.tex

\newcommand{\set}[1]{\left\{ #1 \right\}}
\newcommand{\abs}[1]{\left| #1\right|}
\newcommand{\card}[1]{\left| #1\right|}
\newcommand{\floor}[1]{\left\lfloor #1 \right\rfloor}
\newcommand{\ceil}[1]{\left\lceil #1 \right\rceil}
\newcommand{\ang}[1]{\ifmmode{\left\langle #1 \right\rangle}
   \else{$\left\langle${#1}$\right\rangle$}\fi}
        % the \if allows use outside mathmode,
        % but will swallow following space there!
\newcommand{\paren}[1]{\left( #1 \right)}
\newcommand{\bracket}[1]{\left[ #1 \right]}
\newcommand{\prob}[1]{\Pr\left\{ #1 \right\}}
\newcommand{\Var}{\mathop{\rm Var}\nolimits}
\newcommand{\expect}[1]{{\rm E}\left[ #1 \right]}
\newcommand{\expectsq}[1]{{\rm E}^2\left[ #1 \right]}
\newcommand{\variance}[1]{{\rm Var}\left[ #1 \right]}
\renewcommand{\choose}[2]{{{#1}\atopwithdelims(){#2}}}
\def\pmod#1{\allowbreak\mkern12mu({\rm mod}\,\,#1)}
\newcommand{\matx}[2]{\left(\begin{array}{*{#1}{c}}#2\end{array}\right)}
\newcommand{\Adj}{\mathop{\rm Adj}\nolimits}

\newtheorem{theorem}{Theorem}
\newtheorem{lemma}[theorem]{Lemma}
\newtheorem{corollary}[theorem]{Corollary}
\newtheorem{xample}{Example}
\newtheorem{definition}{Definition}
\newenvironment{example}{\begin{xample}\rm}{\end{xample}}
\newcommand{\proof}{\noindent{\em Proof.}\hspace{1em}}
\def\squarebox#1{\hbox to #1{\hfill\vbox to #1{\vfill}}}
\newcommand{\qedbox}{\vbox{\hrule\hbox{\vrule\squarebox{.667em}\vrule}\hrule}}
\newcommand{\qed}{\nopagebreak\mbox{}\hfill\qedbox\smallskip}
\newcommand{\eqnref}[1]{(\protect\ref{#1})}

%%\newcommand{\twodots}{\mathinner{\ldotp\ldotp}}
\newcommand{\transpose}{^{\mbox{\scriptsize \sf T}}}
\newcommand{\amortized}[1]{\widehat{#1}}

\newcommand{\punt}[1]{}

%%% command for putting definitions into boldface
% New style for defined terms, as of 2/23/88, redefined by THC.
\newcommand{\defn}[1]{{\boldmath\textit{\textbf{#1}}}}
\newcommand{\defi}[1]{{\textit{\textbf{#1\/}}}}

\newcommand{\red}{\leq_{\rm P}}
\newcommand{\lang}[1]{%
\ifmmode\mathord{\mathcode`-="702D\rm#1\mathcode`\-="2200}\else{\rm#1}\fi}

%\newcommand{\ckt}[1]{\ifmmode\mathord{\mathcode`-="702D\sc #1\mathcode`\-="2200}\else$\mathord{\mathcode`-="702D\sc #1\mathcode`\-="2200}$\fi}
\newcommand{\ckt}[1]{\ifmmode \sc #1\else$\sc #1$\fi}

%% Margin notes - use \notesfalse to turn off notes.
\setlength{\marginparwidth}{0.6in}
\reversemarginpar
\newif\ifnotes
\notestrue
\newcommand{\longnote}[1]{
  \ifnotes
    {\medskip\noindent Note: \marginpar[\hfill$\Longrightarrow$]
      {$\Longleftarrow$}{#1}\medskip}
  \fi}
\newcommand{\note}[1]{
  \ifnotes
    {\marginpar{\tiny \raggedright{#1}}}
  \fi}


\newcommand{\reals}{\mathbbm{R}}
\newcommand{\integers}{\mathbbm{Z}}
\newcommand{\naturals}{\mathbbm{N}}
\newcommand{\rationals}{\mathbbm{Q}}
\newcommand{\complex}{\mathbbm{C}}

\newcommand{\oldreals}{{\bf R}}
\newcommand{\oldintegers}{{\bf Z}}
\newcommand{\oldnaturals}{{\bf N}}
\newcommand{\oldrationals}{{\bf Q}}
\newcommand{\oldcomplex}{{\bf C}}

\newcommand{\w}{\omega}                 %% for fft chapter

\newenvironment{closeitemize}{\begin{list}
{$\bullet$}
{\setlength{\itemsep}{-0.2\baselineskip}
\setlength{\topsep}{0.2\baselineskip}
\setlength{\parskip}{0pt}}}
{\end{list}}

% These are necessary within a {problems} environment in order to restore
% the default separation between bullets and items.
\newenvironment{normalitemize}{\setlength{\labelsep}{0.5em}\begin{itemize}}
                              {\end{itemize}}
\newenvironment{normalenumerate}{\setlength{\labelsep}{0.5em}\begin{enumerate}}
                                {\end{enumerate}}

%\def\eqref#1{Equation~(\ref{eq:#1})}
%\newcommand{\eqref}[1]{Equation (\ref{eq:#1})}
\newcommand{\eqreftwo}[2]{Equations (\ref{eq:#1}) and~(\ref{eq:#2})}
\newcommand{\ineqref}[1]{Inequality~(\ref{ineq:#1})}
\newcommand{\ineqreftwo}[2]{Inequalities (\ref{ineq:#1}) and~(\ref{ineq:#2})}

\newcommand{\figref}[1]{Figure~\ref{fig:#1}}
\newcommand{\figreftwo}[2]{Figures \ref{fig:#1} and~\ref{fig:#2}}

\newcommand{\liref}[1]{line~\ref{li:#1}}
\newcommand{\Liref}[1]{Line~\ref{li:#1}}
\newcommand{\lirefs}[2]{lines \ref{li:#1}--\ref{li:#2}}
\newcommand{\Lirefs}[2]{Lines \ref{li:#1}--\ref{li:#2}}
\newcommand{\lireftwo}[2]{lines \ref{li:#1} and~\ref{li:#2}}
\newcommand{\lirefthree}[3]{lines \ref{li:#1}, \ref{li:#2}, and~\ref{li:#3}}

\newcommand{\lemlabel}[1]{\label{lem:#1}}
\newcommand{\lemref}[1]{Lemma~\ref{lem:#1}} 

\newcommand{\exref}[1]{Exercise~\ref{ex:#1}}

\newcommand{\handref}[1]{Handout~\ref{#1}}

\newcommand{\defref}[1]{Definition~\ref{def:#1}}

% (1997.8.16: Victor Luchangco)
% Modified \hlabel to only get date and to use handouts counter for number.
%   New \handout and \handoutwithouttitle commands in newmac.tex use this.
%   The date is referenced by <label>-date.
%   (Retained old definition as \hlabelold.)
%   Defined \hforcelabel to use an argument instead of the handouts counter.

\newcounter{handouts}
\setcounter{handouts}{0}

\newcommand{\hlabel}[2]{%
\stepcounter{handouts}
{\edef\next{\write\@auxout{\string\newlabel{#1}{{\arabic{handouts}}{0}}}}\next}
\write\@auxout{\string\newlabel{#1-date}{{#2}{0}}}
}

\newcommand{\hforcelabel}[3]{%          Does not step handouts counter.
\write\@auxout{\string\newlabel{#1}{{#2}{0}}}
\write\@auxout{\string\newlabel{#1-date}{{#3}{0}}}}


% less ugly underscore
% --juang, 2008 oct 05
\renewcommand{\_}{\vrule height 0 pt depth 0.4 pt width 0.5 em \,}

\newcommand{\theproblemsetnum}{9}
\newcommand{\releasedate}{Thursday, November 15}
\newcommand{\partaduedate}{Thursday, November 29}

\title{6.006 Problem Set 9}

\begin{document}

\handout{Problem Set \theproblemsetnum}{\releasedate}
\textbf{All parts are due {\bf \partaduedate} at {\bf 11PM}}.

\setlength{\parindent}{0pt}
\medskip\hrulefill\medskip

{\bf Name:} Robert Durfee

\medskip

{\bf Collaborators:} Joanna Cohen

\medskip\hrulefill

\begin{problems}

\problem % Problem 1

\begin{problemparts}
\problempart % Problem 1a
{\bf Subproblems} Let $x(n)$ be the number of BSTs that can be formed from
$n$ distinct nodes.

{\bf Relation} Guess a root for the tree. Recurse on the right and left
subtrees.
$$ x(n) = \sum_{i = 0}^{n - 1} x(i) \cdot x\left(n - i - 1\right) $$

The recurrence is acyclic because it depends only on strictly smaller $n$.

{\bf Base Cases} Zero or one nodes form exactly one tree.
$$ x(0) = x(1) = 1 $$

{\bf Solution} Given $n$ distinct nodes, the solution is
$$ x(n) $$

Compute by memoized bottom up or iterative top down.

{\bf Running Time} There are $O(n)$ total subproblems and each requires
$O(n)$ arithmetic operations. Therefore, total work is $O(n^2)$ arithmetic
operations.

\problempart % Problem 1b
{\bf Subproblems} Let $x(n, h)$ be the number of AVL trees that can be formed
from $n$ distinct nodes and of height $h$.

{\bf Relation} Guess a root for the tree. Recurse of the right and left
subtrees. This time, however, the heights can only differ by one.
\begin{align*}
  x(n, h) = \sum_{i = 0}^{n - 1} &(x(i, h - 1) \cdot x(n - i - 1, h - 1) \\
  &+ x(i, h - 1) \cdot x(n - i - 1, h - 2) \\
  &+ x(i, h - 2) \cdot x(n - i - 1, h - 1))
\end{align*}

The recurrence is acyclic because it depends only on strictly smaller $n$ and
$h$.

{\bf Base Cases} If there are negative heights, no tree can be formed. Zero
or one nodes form exactly one tree so long as their respective heights are
also zero and one.
$$ x(n, h) = 0\quad \forall h < 0 $$
$$ x(0, 0) = x(1, 1) = 1 $$

{\bf Solution} To choose the height for the AVL tree, consider that the
maximum height is $1.44 \log_2(n)$ and the height only differs by plus or
minus one. Therefore, the solution is given by,
$$ \sum_{i \in S} x(n, i) $$
Where $S$ is the set containing possible heights, $\{1.44 \log_2(n) \pm 1\}$.

Compute each dynamic program iteration by memoized bottom up or iterative top
down.

{\bf Running Time} There are $O(n \log_2(n))$ subproblems because $h \in
O(\log_2(n))$. Each subproblem requires $O(n)$ arithmetic operations.
Therefore, the dynamic program is $O(n^2 \log_2(n))$ arithmetic operations.
Since it is only run at most $O(1)$ times, the overall algorithm is also
$O(n^2 \log_2(n))$ arithmetic operations.

\problempart % Problem 1c
{\bf Subproblems} Let $x(n)$ be the number of binary heaps that can be formed
from $n$ distinct nodes.

{\bf Relation} Given $n$ distinct nodes, the least must be the root. The left
and right tree can be arbitrarily divided as they all must be less than the
root. There are $n - 1$ choose $\ell$ ways to do this.
$$ x(n) = \binom{n - 1}{\ell} x(\ell) \cdot x(r) $$
Where $\ell$ and $r$ are determined by the shape of the tree. Since the tree
must be complete, each subtree is also complete. The left subtree will either
be full (if the number of nodes in the final level is greater than half the
max possible) or just complete (if the number of nodes in the final level is
less than half the max possible). The right is made up of whatever nodes are
not in the left.
$$ \ell = \begin{cases}
  2^{\lfloor \log_2(n)\rfloor} - 1 & n - (2^{\lfloor \log_2(n)\rfloor} - 1)
  \geq 2^{\lfloor \log_2(n)\rfloor - 1} \\
  n - 2^{\lfloor \log_2(n)\rfloor - 1} & n - (2^{\lfloor \log_2(n)\rfloor} -
  1) < 2^{\lfloor \log_2(n)\rfloor - 1}
\end{cases} $$
$$ r = n - \ell - 1 $$

The recurrence is acyclic because it only depends on strictly smaller $n$.

{\bf Base Cases} Zero or one nodes form exactly one heap.
$$ x(0) = x(1) = 1 $$

{\bf Solution} Given $n$ distince nodes, the solution is
$$ x(n) $$

Compute by memoized bottom up or iterative top down.

{\bf Running Time} There are $O(n)$ subproblems. Each subproblem requires
$O(1)$ arithmetic operations (aside from the choose operation). For the
choose operation, compute all $n$ choose $n$ ahead of time using dynamic
programming which takes $O(n)$ arithmetic operations. Therefore, the overall
solution, when using precomputed choose, is $O(n)$ arithmetic operations.

\end{problemparts}

\newpage
\problem  % Problem 2
{\bf Subproblems} Split the board into two arrays of length $n$. Call them
$A$ and $B$. The two rows $A$ and $B$ vertically stacked represent the $2
\times n$ board. Then, let $x(i, j)$ be the maximum placement of ships on a
board consisting of the upper row from $i$ to the end and the lower row from
$j$ to the end.

{\bf Relation} If $i = j$, then the two sequence are even.
\begin{align*}
  x(i, j) = \max(&A[i] + A[i + 1] + x(i + 2, j + 1), \\
    &B[j] + B[j + 1] + x(i + 1, j + 2), \\
    &A[j] + B[j] + x(i + 1, j + 1), \\
    &x(i + 1, j + 1))
\end{align*}
This corresponds to the following,
\begin{itemize}
  \item\ Place a ship horizontally on the top row.
  \item\ Place a ship horizontally on the bottom row.
  \item\ Place a ship vertically.
  \item\ Ignore both leftmost spots.
\end{itemize}

If $i > j$, then the top sequence is ahead of the bottom.
\begin{align*}
  x(i, j) = \max(&B[j] + B[j + 1] + x(i, j + 2),\\
    &x(i, j + 1))
\end{align*}
This corresponds to the following,
\begin{itemize}
  \item\ Place a ship horizontally on the bottom row.
  \item\ Ignore the lower leftmost spot.
\end{itemize}

If $i < j$, then the bottom sequence is ahead of the top. This case is
analogous to the previous by interchanging $i$ and $j$ and $A$ and $B$.

The recurrence is acyclic as it only depends on strictly shorter subsequences
of the top and bottom rows.

{\bf Base Cases} If both $i$ and $j$ are at the end of the board, either
takeboth or leave both.
$$ x(n - 1, n - 1) = \max(0, A[n - 1] + B[n - 1]) $$

If $i$ is beyond the edge of the board and $j$ is at the end of the board (or
vice versa), only a single block remains and must be ignored.
$$ x(n, n - 1) = x(n - 1, n) = 0 $$

{\bf Solution} Initiate the dynamic program by calling
$$ x(0, 0) $$

Compute using memoized bottom up or iterative top down.

When computing the individual subproblems, store the index of the ship move
that led to the result in the form, for example, $(i, 0)$, $(i + 1, 0)$ if
the placement was on the top row horizontally from $i$ to $i + 1$.
Reconstruct the ship placements by looking at the parent pointers stored with
the memo from the returned result.

{\bf Running Time} There are only $O(n)$ subproblems. Throughout the
recurrence, each sequence is either lined up or off by one. Therefore, there
are {\bf not} $O(n^2)$ subproblems. Furthermore, only $O(1)$ work is done per
subproblem, therefore the overall work is $O(n)$.

\newpage
\problem  % Problem 3
{\bf Subproblems} Let $x(i, j, d)$ be the maximum number of districts won
along the sequence from $i$ to $j$ using $d$ districts.

{\bf Relation} Guess a possible district from the beginning of the sequence
to $k$. The rest of the sequence can be divided into $d - 1$ districts now.
$$ x(i, j, d) = \max(x(i, k, 1) + x(k + 1, j, d - 1)\quad k \in \{i, \ldots,
j\}) $$

This recurrence is acyclic becuase it only depends on strictly shorter
subsequences.

{\bf Base Cases} If $d = 1$ and the length of the sequence is odd,
$$ x(i, j, 1) = v(i, j) $$
Where $v(i, j)$ is $1$ if the sequence from $i$ to $j$ has more Maryonettes
than Itnizks, $0$ otherwise.

If $d = 1$ and the length of the sequence is even,
$$ x(i, j, 1) = -\infty $$

{\bf Solution} Duplicate Circleworld $n$ times. Cut each circle at a
different index such that the $n$ created arrays represent $n$ different ways
to guess the first division of the nation. Then, feed each array into the
dynamic program as
$$ x(0, n, d) $$

Compute each dynamic program result using memoized bottom up or iterative top
down. Then choose and return the maximum from each possible ``flattened''
Circleworld.

Within each subproblem evaluation, save the choice of district sequences
along with the score. Reconstruct the result using the memo and parent
pointers returned from the result.

{\bf Note}: If it is not possible to divide Circleworld into $d$ groups such
that each has an odd number of residents, then the result will be $-\infty$.

{\bf Running Time} The dynamic program has $O(n^2d)$ subproblems. Each
subproblem requires $O(n)$ work. Therefore, the dynamic program needs $O(n^3
d)$ work to complete. Running this algorithm $n$ times leads to $O(n^4 d)$.

\newpage
\problem  % Problem 4
{\bf Subproblems} Let $x(i, j)$ be the maximum score possible in the sequence
from $i$ to $j$ of $A$.

{\bf Relation} Either the bottles at the ends are to be broken, or they are
not. If they are not, guess a place between them to recurse upon.
$$ x(i, j) = \max\begin{cases}
  x(i, k) + x(k + 1, j) & k \in \{i, \ldots, j\} \\
  x(i + 1, j - 1) + A[i] * A[j] 
\end{cases} $$

The recurrence is acyclic because it only depends on strictly smaller
subsequences.

{\bf Base Cases} If the sequence includes one or fewer elements,
$$ x(i, j) = 0 $$

{\bf Solution} Given an array of lenght $n$, start recursion on the whole
sequence,
$$ x(0, n) $$

Compute using memoized bottom up or iterative top down.

{\bf Running Time} There are $O(n^2)$ subproblems. Each subproblem requires
$O(n)$ work. Therefore, the overall runtime is $O(n^3)$.

\newpage
\problem  % Problem 5

\begin{problemparts}
\problempart % Problem 5a
{\bf Subproblems} Let $x(i, j)$ be the length of the longest palindrome in
the sequences from $i$ to $j$. {\bf Note}: This will be twice the length of
the encoded message.

{\bf Relation} If $A[i] \neq A[j]$ then the first and the last characters
can't both be a part of the longest palindrome. Guess one to keep and
recurse on the subsequence.
$$ x(i, j) = \max(x(i + 1, j), x(i, j - 1)) $$
If $A[i] = A[j]$, then the two characters will be a part of the longest
palindrome. Take them and recurse on the inner sequence.
$$ x(i, j) = 2 + x(i + 1, j - 1) $$

This recurrences is acyclic because it only depends on strictly smaller
subsequences.

{\bf Base Cases} If the length of the sequence is $0$ characters,
$$ x(i, j) = 0 $$
If the length of the sequence is $1$ character, 
$$ x(i, j) = 1 $$

{\bf Solution} Initiate the dynamic program on an array of length $n$ by
calling,
$$ x(0, n) $$

For each of the subproblems, store the character saved that led to the max
score along with the memo. Reconstruct the message by following parent
pointers and using the memo.

{\bf Running Time} There are $O(n^2)$ subproblems. Each subproblem requires
$O(1)$ work. Therefore, the total runtime is $O(n^2)$.

\problempart Submit your implementation to {\small\url{alg.mit.edu/PS9}}
\end{problemparts}

\end{problems}

\end{document}
