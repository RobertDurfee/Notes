%
% 6.006 problem set 8 solutions template
%
\documentclass[12pt,twoside]{article}
\usepackage{amsmath}

\newcommand{\name}{}

\usepackage{amssymb}
\usepackage{amsmath}
\usepackage{graphicx}
\usepackage{latexsym}
\usepackage{times,url}
\usepackage{cprotect}
\usepackage{listings}
\usepackage{graphicx}
\usepackage[table]{xcolor}
\usepackage[letterpaper]{geometry}
\usepackage{tikz-qtree}
\usepackage{enumerate}

\newcommand{\profs}{Instructors: Zachary Abel, Erik Demaine, Jason Ku}
\newcommand{\subj}{6.006}
\newcommand{\ttt}[1]{{\tt\small #1}}

\definecolor{dkgreen}{rgb}{0,0.6,0}
\definecolor{gray}{rgb}{0.5,0.5,0.5}
\definecolor{mauve}{rgb}{0.58,0,0.82}

\lstset{
  language=Python,
  aboveskip=1pc,
  belowskip=1pc,
  basicstyle={\footnotesize\ttfamily},
  numbers=left,
  showstringspaces=false,
  numberstyle=\tiny\color{gray},
  keywordstyle=\color{blue},
  commentstyle=\color{dkgreen},
  stringstyle=\color{mauve},
}

\tikzset{
  % every node/.style={minimum width=2em,draw,circle},
  % level 1/.style={sibling distance=2cm},
  level distance=1cm,
  edge from parent/.style=
  {draw,edge from parent path={(\tikzparentnode) -- (\tikzchildnode)}},
}

\newif\ifHideSolutions
\newcommand{\solution}[1]{\color{dkgreen}\textbf{Solution: }#1\color{black}}
\newcommand{\rubric}[1]{\color{dkgreen}{\bf Rubric:} #1\color{black}}

% \HideSolutionsfalse
% \ifHideSolutions
%   \renewcommand{\solution}[1]{}
%   \renewcommand{\rubric}[1]{}
% \fi

\newlength{\toppush}
\setlength{\toppush}{2\headheight}
\addtolength{\toppush}{\headsep}

\newcommand{\htitle}[2]{\noindent\vspace*{-\toppush}\newline\parbox{6.5in}
{\textit{Introduction to Algorithms: 6.006}\hfill\name\newline
Massachusetts Institute of Technology \hfill #2\newline
\profs\hfill #1 \vspace*{-.5ex}\newline
\mbox{}\hrulefill\mbox{}}\vspace*{1ex}\mbox{}\newline
\begin{center}{\Large\bf #1}\end{center}}

\newcommand{\handout}[2]{\thispagestyle{empty}
 \markboth{#1}{#1}
 \pagestyle{myheadings}\htitle{#1}{#2}}

\newcommand{\lecture}[3]{\thispagestyle{empty}
 \markboth{Lecture #1: #2}{Lecture #1: #2}
 \pagestyle{myheadings}\htitle{Lecture #1: #2}{#3}}

\newcommand{\htitlewithouttitle}[2]{\noindent\vspace*{-\toppush}\newline\parbox{6.5in}
{\textit{Introduction to Algorithms}\hfill#2\newline
Massachusetts Institute of Technology \hfill 6.006\newline
\profs\hfill Handout #1\vspace*{-.5ex}\newline
\mbox{}\hrulefill\mbox{}}\vspace*{1ex}\mbox{}\newline}

\newcommand{\handoutwithouttitle}[2]{\thispagestyle{empty}
 \markboth{Handout \protect\ref{#1}}{Handout \protect\ref{#1}}
 \pagestyle{myheadings}\htitlewithouttitle{\protect\ref{#1}}{#2}}

\newcommand{\exam}[2]{% parameters: exam name, date
 \thispagestyle{empty}
 \markboth{\hspace{1cm}\subj\ #1\hspace{1in}Name\hrulefill\ \ }%
          {\subj\ #1\hspace{1in}Name\hrulefill\ \ }
 \pagestyle{myheadings}\examtitle{#1}{#2}
 \renewcommand{\theproblem}{Problem \arabic{problemnum}}
}
\newcommand{\examsolutions}[3]{% parameters: handout, exam name, date
 \thispagestyle{empty}
 \markboth{Handout \protect\ref{#1}: #2}{Handout \protect\ref{#1}: #2}
% \pagestyle{myheadings}\htitle{\protect\ref{#1}}{#2}{#3}
 \pagestyle{myheadings}\examsolutionstitle{\protect\ref{#1}} {#2}{#3}
 \renewcommand{\theproblem}{Problem \arabic{problemnum}}
}
\newcommand{\examsolutionstitle}[3]{\noindent\vspace*{-\toppush}\newline\parbox{6.5in}
{\textit{Introduction to Algorithms}\hfill#3\newline
Massachusetts Institute of Technology \hfill 6.006\newline
%Singapore-MIT Alliance \hfill SMA5503\newline
\profs\hfill Handout #1\vspace*{-.5ex}\newline
\mbox{}\hrulefill\mbox{}}\vspace*{1ex}\mbox{}\newline
\begin{center}{\Large\bf #2}\end{center}}

\newcommand{\takehomeexam}[2]{% parameters: exam name, date
 \thispagestyle{empty}
 \markboth{\subj\ #1\hfill}{\subj\ #1\hfill}
 \pagestyle{myheadings}\examtitle{#1}{#2}
 \renewcommand{\theproblem}{Problem \arabic{problemnum}}
}

\makeatletter
\newcommand{\exambooklet}[2]{% parameters: exam name, date
 \thispagestyle{empty}
 \markboth{\subj\ #1}{\subj\ #1}
 \pagestyle{myheadings}\examtitle{#1}{#2}
 \renewcommand{\theproblem}{Problem \arabic{problemnum}}
 \renewcommand{\problem}{\newpage
 \item \let\@currentlabel=\theproblem
 \markboth{\subj\ #1, \theproblem}{\subj\ #1, \theproblem}}
}
\makeatother


\newcommand{\examtitle}[2]{\noindent\vspace*{-\toppush}\newline\parbox{6.5in}
{\textit{Introduction to Algorithms}\hfill#2\newline
Massachusetts Institute of Technology \hfill 6.006 Fall 2018\newline
%Singapore-MIT Alliance \hfill SMA5503\newline
\profs\hfill #1\vspace*{-.5ex}\newline
\mbox{}\hrulefill\mbox{}}\vspace*{1ex}\mbox{}\newline
\begin{center}{\Large\bf #1}\end{center}}

\newcommand{\grader}[1]{\hspace{1cm}\textsf{\textbf{#1}}\hspace{1cm}}

\newcommand{\points}[1]{[#1 points]\ }
\newcommand{\parts}[1]
{
  \ifnum#1=1
  (1 part)
  \else
  (#1 parts)
  \fi
  \ 
}

\newcommand{\bparts}{\begin{problemparts}}
\newcommand{\eparts}{\end{problemparts}}
\newcommand{\ppart}{\problempart}

%\newcommand{\lg} {lg\ }

\setlength{\oddsidemargin}{0pt}
\setlength{\evensidemargin}{0pt}
\setlength{\textwidth}{6.5in}
\setlength{\topmargin}{0in}
\setlength{\textheight}{8.5in}


\newcommand{\Spawn}{{\bf spawn} }
\newcommand{\Sync}{{\bf sync}}

\renewcommand{\cases}[1]{\left\{ \begin{array}{ll}#1\end{array}\right.}
\newcommand{\cif}[1]{\mbox{if $#1$}}
\newcommand{\cwhen}[1]{\mbox{when $#1$}}

\newcounter{problemnum}
\newcommand{\theproblem}{Problem \theproblemsetnum-\arabic{problemnum}}
\newenvironment{problems}{
        \begin{list}{{\bf \theproblem. \hspace*{0.5em}}}
        {\setlength{\leftmargin}{0em}
         \setlength{\rightmargin}{0em}
         \setlength{\labelwidth}{0em}
         \setlength{\labelsep}{0em}
         \usecounter{problemnum}}}{\end{list}}
\makeatletter
\newcommand{\problem}[1][{}]{\item \let\@currentlabel=\theproblem \textbf{#1}}
\makeatother

\newcounter{problempartnum}[problemnum]
\newenvironment{problemparts}{
        \begin{list}{{\bf (\alph{problempartnum})}}
        {\setlength{\leftmargin}{2.5em}
         \setlength{\rightmargin}{2.5em}
         \setlength{\labelsep}{0.5em}}}{\end{list}}
\newcommand{\problempart}{\addtocounter{problempartnum}{1}\item}

\newenvironment{truefalseproblemparts}{
        \begin{list}{{\bf (\alph{problempartnum})\ \ \ T\ \ F\hfil}}
        {\setlength{\leftmargin}{4.5em}
         \setlength{\rightmargin}{2.5em}
         \setlength{\labelsep}{0.5em}
         \setlength{\labelwidth}{4.5em}}}{\end{list}}

\newcounter{exercisenum}
\newcommand{\theexercise}{Exercise \theproblemsetnum-\arabic{exercisenum}}
\newenvironment{exercises}{
        \begin{list}{{\bf \theexercise. \hspace*{0.5em}}}
        {\setlength{\leftmargin}{0em}
         \setlength{\rightmargin}{0em}
         \setlength{\labelwidth}{0em}
         \setlength{\labelsep}{0em}
        \usecounter{exercisenum}}}{\end{list}}
\makeatletter
\newcommand{\exercise}{\item \let\@currentlabel=\theexercise}
\makeatother

\newcounter{exercisepartnum}[exercisenum]
%\newcommand{\problem}[1]{\medskip\mbox{}\newline\noindent{\bf Problem #1.}\hspace*{1em}}
%\newcommand{\exercise}[1]{\medskip\mbox{}\newline\noindent{\bf Exercise #1.}\hspace*{1em}}

\newenvironment{exerciseparts}{
        \begin{list}{{\bf (\alph{exercisepartnum})}}
        {\setlength{\leftmargin}{2.5em}
         \setlength{\rightmargin}{2.5em}
         \setlength{\labelsep}{0.5em}}}{\end{list}}
\newcommand{\exercisepart}{\addtocounter{exercisepartnum}{1}\item}


% Macros to make captions print with small type and 'Figure xx' in bold.
\makeatletter
\def\fnum@figure{{\bf Figure \thefigure}}
\def\fnum@table{{\bf Table \thetable}}
\let\@mycaption\caption
%\long\def\@mycaption#1[#2]#3{\addcontentsline{\csname
%  ext@#1\endcsname}{#1}{\protect\numberline{\csname 
%  the#1\endcsname}{\ignorespaces #2}}\par
%  \begingroup
%    \@parboxrestore
%    \small
%    \@makecaption{\csname fnum@#1\endcsname}{\ignorespaces #3}\par
%  \endgroup}
%\def\mycaption{\refstepcounter\@captype \@dblarg{\@mycaption\@captype}}
%\makeatother
\let\mycaption\caption
%\newcommand{\figcaption}[1]{\mycaption[]{#1}}

\newcounter{totalcaptions}
\newcounter{totalart}

\newcommand{\figcaption}[1]{\addtocounter{totalcaptions}{1}\caption[]{#1}}

% \psfigures determines what to do for figures:
%       0 means just leave vertical space
%       1 means put a vertical rule and the figure name
%       2 means insert the PostScript version of the figure
%       3 means put the figure name flush left or right
\newcommand{\psfigures}{0}
\newcommand{\spacefigures}{\renewcommand{\psfigures}{0}}
\newcommand{\rulefigures}{\renewcommand{\psfigures}{1}}
\newcommand{\macfigures}{\renewcommand{\psfigures}{2}}
\newcommand{\namefigures}{\renewcommand{\psfigures}{3}}

\newcommand{\figpart}[1]{{\bf (#1)}\nolinebreak[2]\relax}
\newcommand{\figparts}[2]{{\bf (#1)--(#2)}\nolinebreak[2]\relax}


\macfigures     % STATE

% When calling \figspace, make sure to leave a blank line afterward!!
% \widefigspace is for figures that are more than 28pc wide.
\newlength{\halffigspace} \newlength{\wholefigspace}
\newlength{\figruleheight} \newlength{\figgap}
\newcommand{\setfiglengths}{\ifnum\psfigures=1\setlength{\figruleheight}{\hruleheight}\setlength{\figgap}{1em}\else\setlength{\figruleheight}{0pt}\setlength{\figgap}{0em}\fi}
\newcommand{\figspace}[2]{\ifnum\psfigures=0\leavefigspace{#1}\else%
\setfiglengths%
\setlength{\wholefigspace}{#1}\setlength{\halffigspace}{.5\wholefigspace}%
\rule[-\halffigspace]{\figruleheight}{\wholefigspace}\hspace{\figgap}#2\fi}
\newlength{\widefigspacewidth}
% Make \widefigspace put the figure flush right on the text page.
\newcommand{\widefigspace}[2]{
\ifnum\psfigures=0\leavefigspace{#1}\else%
\setfiglengths%
\setlength{\widefigspacewidth}{28pc}%
\addtolength{\widefigspacewidth}{-\figruleheight}%
\setlength{\wholefigspace}{#1}\setlength{\halffigspace}{.5\wholefigspace}%
\makebox[\widefigspacewidth][r]{#2\hspace{\figgap}}\rule[-\halffigspace]{\figruleheight}{\wholefigspace}\fi}
\newcommand{\leavefigspace}[1]{\setlength{\wholefigspace}{#1}\setlength{\halffigspace}{.5\wholefigspace}\rule[-\halffigspace]{0em}{\wholefigspace}}

% Commands for including figures with macpsfig.
% To use these commands, documentstyle ``macpsfig'' must be specified.
\newlength{\macfigfill}
\makeatother
\newlength{\bbx}
\newlength{\bby}
\newcommand{\macfigure}[5]{\addtocounter{totalart}{1}
\ifnum\psfigures=2%
\setlength{\bbx}{#2}\addtolength{\bbx}{#4}%
\setlength{\bby}{#3}\addtolength{\bby}{#5}%
\begin{flushleft}
\ifdim#4>28pc\setlength{\macfigfill}{#4}\addtolength{\macfigfill}{-28pc}\hspace*{-\macfigfill}\fi%
\mbox{\psfig{figure=./#1.ps,%
bbllx=#2,bblly=#3,bburx=\bbx,bbury=\bby}}
\end{flushleft}%
\else\ifdim#4>28pc\widefigspace{#5}{#1}\else\figspace{#5}{#1}\fi\fi}
\makeatletter

\newlength{\savearraycolsep}
\newcommand{\narrowarray}[1]{\setlength{\savearraycolsep}{\arraycolsep}\setlength{\arraycolsep}{#1\arraycolsep}}
\newcommand{\normalarray}{\setlength{\arraycolsep}{\savearraycolsep}}

\newcommand{\hint}{{\em Hint:\ }}

% Macros from /th/u/clr/mac.tex

\newcommand{\set}[1]{\left\{ #1 \right\}}
\newcommand{\abs}[1]{\left| #1\right|}
\newcommand{\card}[1]{\left| #1\right|}
\newcommand{\floor}[1]{\left\lfloor #1 \right\rfloor}
\newcommand{\ceil}[1]{\left\lceil #1 \right\rceil}
\newcommand{\ang}[1]{\ifmmode{\left\langle #1 \right\rangle}
   \else{$\left\langle${#1}$\right\rangle$}\fi}
        % the \if allows use outside mathmode,
        % but will swallow following space there!
\newcommand{\paren}[1]{\left( #1 \right)}
\newcommand{\bracket}[1]{\left[ #1 \right]}
\newcommand{\prob}[1]{\Pr\left\{ #1 \right\}}
\newcommand{\Var}{\mathop{\rm Var}\nolimits}
\newcommand{\expect}[1]{{\rm E}\left[ #1 \right]}
\newcommand{\expectsq}[1]{{\rm E}^2\left[ #1 \right]}
\newcommand{\variance}[1]{{\rm Var}\left[ #1 \right]}
\renewcommand{\choose}[2]{{{#1}\atopwithdelims(){#2}}}
\def\pmod#1{\allowbreak\mkern12mu({\rm mod}\,\,#1)}
\newcommand{\matx}[2]{\left(\begin{array}{*{#1}{c}}#2\end{array}\right)}
\newcommand{\Adj}{\mathop{\rm Adj}\nolimits}

\newtheorem{theorem}{Theorem}
\newtheorem{lemma}[theorem]{Lemma}
\newtheorem{corollary}[theorem]{Corollary}
\newtheorem{xample}{Example}
\newtheorem{definition}{Definition}
\newenvironment{example}{\begin{xample}\rm}{\end{xample}}
\newcommand{\proof}{\noindent{\em Proof.}\hspace{1em}}
\def\squarebox#1{\hbox to #1{\hfill\vbox to #1{\vfill}}}
\newcommand{\qedbox}{\vbox{\hrule\hbox{\vrule\squarebox{.667em}\vrule}\hrule}}
\newcommand{\qed}{\nopagebreak\mbox{}\hfill\qedbox\smallskip}
\newcommand{\eqnref}[1]{(\protect\ref{#1})}

%%\newcommand{\twodots}{\mathinner{\ldotp\ldotp}}
\newcommand{\transpose}{^{\mbox{\scriptsize \sf T}}}
\newcommand{\amortized}[1]{\widehat{#1}}

\newcommand{\punt}[1]{}

%%% command for putting definitions into boldface
% New style for defined terms, as of 2/23/88, redefined by THC.
\newcommand{\defn}[1]{{\boldmath\textit{\textbf{#1}}}}
\newcommand{\defi}[1]{{\textit{\textbf{#1\/}}}}

\newcommand{\red}{\leq_{\rm P}}
\newcommand{\lang}[1]{%
\ifmmode\mathord{\mathcode`-="702D\rm#1\mathcode`\-="2200}\else{\rm#1}\fi}

%\newcommand{\ckt}[1]{\ifmmode\mathord{\mathcode`-="702D\sc #1\mathcode`\-="2200}\else$\mathord{\mathcode`-="702D\sc #1\mathcode`\-="2200}$\fi}
\newcommand{\ckt}[1]{\ifmmode \sc #1\else$\sc #1$\fi}

%% Margin notes - use \notesfalse to turn off notes.
\setlength{\marginparwidth}{0.6in}
\reversemarginpar
\newif\ifnotes
\notestrue
\newcommand{\longnote}[1]{
  \ifnotes
    {\medskip\noindent Note: \marginpar[\hfill$\Longrightarrow$]
      {$\Longleftarrow$}{#1}\medskip}
  \fi}
\newcommand{\note}[1]{
  \ifnotes
    {\marginpar{\tiny \raggedright{#1}}}
  \fi}


\newcommand{\reals}{\mathbbm{R}}
\newcommand{\integers}{\mathbbm{Z}}
\newcommand{\naturals}{\mathbbm{N}}
\newcommand{\rationals}{\mathbbm{Q}}
\newcommand{\complex}{\mathbbm{C}}

\newcommand{\oldreals}{{\bf R}}
\newcommand{\oldintegers}{{\bf Z}}
\newcommand{\oldnaturals}{{\bf N}}
\newcommand{\oldrationals}{{\bf Q}}
\newcommand{\oldcomplex}{{\bf C}}

\newcommand{\w}{\omega}                 %% for fft chapter

\newenvironment{closeitemize}{\begin{list}
{$\bullet$}
{\setlength{\itemsep}{-0.2\baselineskip}
\setlength{\topsep}{0.2\baselineskip}
\setlength{\parskip}{0pt}}}
{\end{list}}

% These are necessary within a {problems} environment in order to restore
% the default separation between bullets and items.
\newenvironment{normalitemize}{\setlength{\labelsep}{0.5em}\begin{itemize}}
                              {\end{itemize}}
\newenvironment{normalenumerate}{\setlength{\labelsep}{0.5em}\begin{enumerate}}
                                {\end{enumerate}}

%\def\eqref#1{Equation~(\ref{eq:#1})}
%\newcommand{\eqref}[1]{Equation (\ref{eq:#1})}
\newcommand{\eqreftwo}[2]{Equations (\ref{eq:#1}) and~(\ref{eq:#2})}
\newcommand{\ineqref}[1]{Inequality~(\ref{ineq:#1})}
\newcommand{\ineqreftwo}[2]{Inequalities (\ref{ineq:#1}) and~(\ref{ineq:#2})}

\newcommand{\figref}[1]{Figure~\ref{fig:#1}}
\newcommand{\figreftwo}[2]{Figures \ref{fig:#1} and~\ref{fig:#2}}

\newcommand{\liref}[1]{line~\ref{li:#1}}
\newcommand{\Liref}[1]{Line~\ref{li:#1}}
\newcommand{\lirefs}[2]{lines \ref{li:#1}--\ref{li:#2}}
\newcommand{\Lirefs}[2]{Lines \ref{li:#1}--\ref{li:#2}}
\newcommand{\lireftwo}[2]{lines \ref{li:#1} and~\ref{li:#2}}
\newcommand{\lirefthree}[3]{lines \ref{li:#1}, \ref{li:#2}, and~\ref{li:#3}}

\newcommand{\lemlabel}[1]{\label{lem:#1}}
\newcommand{\lemref}[1]{Lemma~\ref{lem:#1}} 

\newcommand{\exref}[1]{Exercise~\ref{ex:#1}}

\newcommand{\handref}[1]{Handout~\ref{#1}}

\newcommand{\defref}[1]{Definition~\ref{def:#1}}

% (1997.8.16: Victor Luchangco)
% Modified \hlabel to only get date and to use handouts counter for number.
%   New \handout and \handoutwithouttitle commands in newmac.tex use this.
%   The date is referenced by <label>-date.
%   (Retained old definition as \hlabelold.)
%   Defined \hforcelabel to use an argument instead of the handouts counter.

\newcounter{handouts}
\setcounter{handouts}{0}

\newcommand{\hlabel}[2]{%
\stepcounter{handouts}
{\edef\next{\write\@auxout{\string\newlabel{#1}{{\arabic{handouts}}{0}}}}\next}
\write\@auxout{\string\newlabel{#1-date}{{#2}{0}}}
}

\newcommand{\hforcelabel}[3]{%          Does not step handouts counter.
\write\@auxout{\string\newlabel{#1}{{#2}{0}}}
\write\@auxout{\string\newlabel{#1-date}{{#3}{0}}}}


% less ugly underscore
% --juang, 2008 oct 05
\renewcommand{\_}{\vrule height 0 pt depth 0.4 pt width 0.5 em \,}

\newcommand{\theproblemsetnum}{8}
\newcommand{\releasedate}{Thursday, November 1}
\newcommand{\partaduedate}{Sunday, November 11}

\title{6.006 Problem Set 8}

\begin{document}

\handout{Problem Set \theproblemsetnum}{\releasedate}
\textbf{All parts are due {\bf \partaduedate} at {\bf 11PM}}.

\setlength{\parindent}{0pt}
\medskip\hrulefill\medskip

{\bf Name:} Robert Durfee

\medskip

{\bf Collaborators:} Johanna Cohen

\medskip\hrulefill

\begin{problems}

\problem

\begin{problemparts}
\problempart Visited order:
{\tt (s, b, a, d, e, f, c)}

Shortest paths:
{\tt (0, 6, 7, 10, 10, 10, 15)}

\problempart {\bf Description} Run graph exploration from $s$ to identity all
the negative edges. For each adjacent vertice to a negative edge, run graph
exploration again to determine all reachable vertices from the negative
edges. These vertices all have negative infinite shortest path lengths. Now,
remove the negative edges. Convert the undirected graph into an directed
graph by creating parallel edges in opposing directions with the same weights
as the undirected edges. Run Dijkstra on the new graph.

{\bf Correctness} Graph traversal will reveal all reachable negative weight
edges. All vertices reachable from these edges must have negative infinity
shortest paths as the negative undirected edges form negative weight cycles
in an analogous directed graph. Graph traversal will find all these reachable
nodes. Once these edges are removed, the remaining graph must have either
positive infinite paths (unreachable) or positive. Dijkstra can handle both
cases.

{\bf Running Time} Graph traversal is $O(|V| + |E|)$. Dijkstra is $O(|E| +
|V| \log |V|)$. Total must be $O(|E| + |V| \log |V|)$.

\problempart {\bf Description} Run Bellman-Ford, from $s$, $|V|$ times. On
the last iteration, keep track of adjacent vertices of relaxed edges. For
each, traverse the parent pointers until a vertice is hit twice. These will
represent the negative-weight cycles. For each cycle, run graph traversal to
see if both $t$ and $s$ are reachable from the same cycle. Return the path
from $s$ to a vertice on the cycle, the path describing the cycle, and the
path from a vertice on the cycle to $t$.

{\bf Correctness} Bellman-Ford will identify the edges that are part of a
negative weight cycle. The parent pointer traversal will yield the cycles.
Graph traversal will find the cycle reachable from both $s$ and $t$ (which
there must be one given that $\delta(s, t) = - \infty$).

{\bf Running Time} Bellman-Ford is $O(|V||E|)$. Graph traversal is $O(|V| +
|E|)$. Therefore, overall run time is $O(|V||E|)$.

\end{problemparts}

\newpage
\problem Let $G$ be defined by a set of $n$ vertices $V$ representing the
spaceports and a set of $f$ edges $E$ representing the flights between each
spaceport. These edges have weights given by the flying time.

\begin{problemparts}
\problempart {\bf Description} Let $s$ be the vertice representing Nighton,
Ohio. Let $t$ be the vertice representing Moon Base Beta. Run Dijkstra on $G$
starting at $s$. Return $\delta(s, t)$ as determined by Dijkstra.

{\bf Correctness} Dijkstra will return the shortest path from $s$ to all $v
\in V$ including $t$.

{\bf Running Time} Dijkstra runs in $O(|E| + |V| \log |V|)$.

\problempart {\bf Description} Let $s$ be the vertice representing Moon Base
Beta. Let $t$ be the vertice representing Deep Space Ten. Run Bellman-Ford on
$G$ starting at $s$. Return $\delta(s, t)$ as determined by Bellman-Ford.

{\bf Correctness} Given the Consistent Timeline Hypothesis, there cannot be
any negative weight cycles. Therefore, Bellman-ford will return the shortest
path from $s$ to any $v \in V$, including $t$.

{\bf Running Time} Bellman-Ford runs in $O(|V||E|)$.

\problempart {\bf Description} Add a new vertice to $G$ that is connected to
all vertices with weight $0$. Run Bellman-Ford from this vertice and augment
the graph as described in the Johnson algorithm in lecture. For each
potential new base, run Dijkstra. Iterate over all computed distances and sum
them together. Return the potential new base with the smallest distance sum.

{\bf Correctness} This is just the Johnson algorithm which computes the all
sources shortest paths problem. Thus the sums of the distances will be the
sums of the minimum flying times from that source to all spaceports.

{\bf Running Time} Johnson's algorithm runs in $O(|V||E| + |V|^2 \log |V|)$.
The graph traversal to sum all the weights is $O(|V||E| + |V|^2)$. Therefore,
the overall runtime is $O(|V||E| + |V|^2 \log |V|)$.

\end{problemparts}

\newpage
\problem {\bf Description} 
  \begin{enumerate}
    \item {\bf Subproblems} Let the subproblem $x(i, j)$ be the maximum score
    along the contiguous subsequence from $i$ to $j$ in range $[0, n - 1]$.

    \item {\bf Relate Subproblems} Let the subproblem be determined by the
    rule
    $$ x(i, j) = \max\left\{x(i, k - 1), x(k, j)\right\}\quad \forall k \in
    [i + 1, j] $$
    These cases are each dependent upon strictly smaller subsequences and
    therefore form a directed acyclic graph where smallest subsequences
    should be computed prior to larger ones.

    \item {\bf Base Cases} The base cases are subsequences of length $1$:
    $$ x(i, i) = 0\quad \forall i \in [0, n - 1] $$
    And subsequences of length $2$ (let $a$ be the given sequence):
    $$ x(i, i + 1) = a_i \cdot a_{i + 1} \quad \forall i \in [0, n - 2] $$

    \item {\bf Solution from Subproblems} The solution to the problem is
    given by $x(0, n - 1)$.

  \end{enumerate}

{\bf Correctness} By induction, it is clear to see that if the base cases are
solved, the inductive cases follow.

{\bf Running Time} There are $n (n + 1) / 2$ subproblems each require $O(1)$
work. Therefore, the overall runtime is $O(n)$.

\newpage

\problem {\bf Description}
  \begin{enumerate}
    \item {\bf Subproblems} Let the subproblem $x(i, v)$ be the maximum score
    along the contiguous subsequence of ramps from $i$ to $n - 1$ with
    starting velocity $v$.

    \item {\bf Relate Subproblems} Let the subproblem be determined by the rule
    $$ x(i, v) = \max\left\{ -1 + x(i + 1, v - 1), -1 + x(i + 1, v +
    1), f(i, v) \right\}\quad \forall i \in [0, n - 1] $$
    Where $f$ is defined as follows,
    $$ f(i, v) = \begin{cases}
      0 & i + v > n - 1 \\
      3v + x(i + v, v), & \mathrm{black} \\
      v + x(i + v, v), & \mathrm{white}
    \end{cases} $$
    These all depend upon strictly smaller subsequences. Also, the $v$ is
    constrained between $1$ and $i$ for the subproblem defined by starting
    point $i$, therefore there is a constant number of subsequences to be
    computed. Therefore, this is a directed acyclic graph where smaller
    subsequences should be computed before larger ones.

    {\bf Note}: $v$ is constrained between $1$ and $40$ so there should be a
    $\min$ and a $\max$ around the $v + 1$ and $v - 1$, respectively, to
    confirm that. For cleanliness, I have removed this from the notation.

    \item {\bf Base Cases}: The base case is the empty subsequence from $n -
    1$ to $n - 1$ defined as
    $$ x(n - 1, v) = 0 \quad \forall v \in [1, n] $$

    \item {\bf Solution from Subproblems} The solution to the problem is
    given by $x(0, 1)$.

  \end{enumerate}

{\bf Correctness} By induction, it is clear to see that if the base cases
are solved, the inductive cases follow.

{\bf Running Time} There are $n$ different contiguous subsequences ending
with $n$ and each $i$th subproblem has a possible starting velocity between
$1$ and $i$. Therefore, there are $n (n + 1) / 2$ total subproblems each
requiring $O(1)$ work so the overall runtime is $O(n)$.

\newpage
\problem 

\begin{problemparts}
\problempart The length $D$ corresponds to a path $p$ from $s$ to $t$ through
some vertice $v$. $\delta(s, t)$ corresponds to the shortest path from $s$ to
$t$. Given by the safety lemma, our algorithm cannot yield a path length that
is less than the shortest path.

\problempart Given the stopping condition defined, $v_s$ will be reached by
the Dijkstra starting from $s$ and $v_t$ will be reached by the Dijkstra
starting from $t$. Dijkstra is guaranteed to return the shortest distance to
all vertices. Therefore, $d(s, v_s) = \delta(s, v_s)$ and $d(t, v_t) =
\delta(t, v_t)$.

\problempart There are two cases to consider: an path with an odd number of
edges and a path with an even number of edges.

If the path $p$ has an odd number of edges, the $v_s$ and $v_t$ are two
distinct vertices separated by a single weight $w(v_s, v_t)$. From this,
$d(s, v_t) = d(s, v_s) + w(v_s, v_t)$. Given from Part B, this must be
$\delta(s, v_t) = \delta(s, v_s) + w(v_s, v_t)$. Furthermore, $\delta(s, t) =
\delta(s, v_t) + \delta(t, v_t)$.

If the path $p$ has an even number of edges, the $v_s$ and $v_t$ are the same
vertice. Therefore, $\delta(s, v_s) = \delta(s, v_t)$. Therefore, $\delta(s,
t) = \delta(s, v_t) + \delta(t, v_t)$.

\problempart Given the stopping condition, $D$ is the minimum $d(s, v) + d(s,
t)$. From Part C, any $v$ along the path $p$ will yield a $D$ equal to the
shortest path. If the $v$ is not on the path, it may be lower, becuase of the
minimum, but not greater. Therefore $D \leq \delta(s, t)$.

\problempart Given that each vertice has a degree of at most $k$, then $E \in
O(kV)$. The runtime for Dijkstra becomes $O(kV + V \log V)$. Now, given that
the number of vertices is $\Theta(r^k)$, the runtime for Dijkstra becomes
$O(k r^k + r^k \log r^k)$. Therefore, each individual Dijkstra will only have
to cover $r / 2$. Therefore the bidirectional Dijkstra runtime is
$$ O\left(2\left(k \left(\frac{r}{2}\right)^k + \left(\frac{r}{2}\right)^k
\log \left(\frac{r}{2}\right)^k \right)\right) $$
Therefore, pulling the $2^k$ to the front,
$$ O\left(\frac{1}{2^{(k - 1)}} \left(k r^k + r^k \log
\left(\frac{r}{2}\right)^k \right)\right) $$
This shows the bidirectional Dijkstra is faster by a factor of $\Theta(2^k)$.
\problempart Submit your implementation to {\small\url{alg.mit.edu/PS8}}
\end{problemparts}

\end{problems}

\end{document}

