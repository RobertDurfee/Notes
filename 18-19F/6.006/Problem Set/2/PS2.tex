\documentclass[12pt,twoside]{article}

\newcommand{\name}{}

\usepackage{amssymb}
\usepackage{amsmath}
\usepackage{graphicx}
\usepackage{latexsym}
\usepackage{times,url}
\usepackage{cprotect}
\usepackage{listings}
\usepackage{graphicx}
\usepackage[table]{xcolor}
\usepackage[letterpaper]{geometry}
\usepackage{tikz-qtree}
\usepackage{enumerate}

\newcommand{\profs}{Instructors: Zachary Abel, Erik Demaine, Jason Ku}
\newcommand{\subj}{6.006}
\newcommand{\ttt}[1]{{\tt\small #1}}

\definecolor{dkgreen}{rgb}{0,0.6,0}
\definecolor{gray}{rgb}{0.5,0.5,0.5}
\definecolor{mauve}{rgb}{0.58,0,0.82}

\lstset{
  language=Python,
  aboveskip=1pc,
  belowskip=1pc,
  basicstyle={\footnotesize\ttfamily},
  numbers=left,
  showstringspaces=false,
  numberstyle=\tiny\color{gray},
  keywordstyle=\color{blue},
  commentstyle=\color{dkgreen},
  stringstyle=\color{mauve},
}

\tikzset{
  % every node/.style={minimum width=2em,draw,circle},
  % level 1/.style={sibling distance=2cm},
  level distance=1cm,
  edge from parent/.style=
  {draw,edge from parent path={(\tikzparentnode) -- (\tikzchildnode)}},
}

\newif\ifHideSolutions
\newcommand{\solution}[1]{\color{dkgreen}\textbf{Solution: }#1\color{black}}
\newcommand{\rubric}[1]{\color{dkgreen}{\bf Rubric:} #1\color{black}}

% \HideSolutionsfalse
% \ifHideSolutions
%   \renewcommand{\solution}[1]{}
%   \renewcommand{\rubric}[1]{}
% \fi

\newlength{\toppush}
\setlength{\toppush}{2\headheight}
\addtolength{\toppush}{\headsep}

\newcommand{\htitle}[2]{\noindent\vspace*{-\toppush}\newline\parbox{6.5in}
{\textit{Introduction to Algorithms: 6.006}\hfill\name\newline
Massachusetts Institute of Technology \hfill #2\newline
\profs\hfill #1 \vspace*{-.5ex}\newline
\mbox{}\hrulefill\mbox{}}\vspace*{1ex}\mbox{}\newline
\begin{center}{\Large\bf #1}\end{center}}

\newcommand{\handout}[2]{\thispagestyle{empty}
 \markboth{#1}{#1}
 \pagestyle{myheadings}\htitle{#1}{#2}}

\newcommand{\lecture}[3]{\thispagestyle{empty}
 \markboth{Lecture #1: #2}{Lecture #1: #2}
 \pagestyle{myheadings}\htitle{Lecture #1: #2}{#3}}

\newcommand{\htitlewithouttitle}[2]{\noindent\vspace*{-\toppush}\newline\parbox{6.5in}
{\textit{Introduction to Algorithms}\hfill#2\newline
Massachusetts Institute of Technology \hfill 6.006\newline
\profs\hfill Handout #1\vspace*{-.5ex}\newline
\mbox{}\hrulefill\mbox{}}\vspace*{1ex}\mbox{}\newline}

\newcommand{\handoutwithouttitle}[2]{\thispagestyle{empty}
 \markboth{Handout \protect\ref{#1}}{Handout \protect\ref{#1}}
 \pagestyle{myheadings}\htitlewithouttitle{\protect\ref{#1}}{#2}}

\newcommand{\exam}[2]{% parameters: exam name, date
 \thispagestyle{empty}
 \markboth{\hspace{1cm}\subj\ #1\hspace{1in}Name\hrulefill\ \ }%
          {\subj\ #1\hspace{1in}Name\hrulefill\ \ }
 \pagestyle{myheadings}\examtitle{#1}{#2}
 \renewcommand{\theproblem}{Problem \arabic{problemnum}}
}
\newcommand{\examsolutions}[3]{% parameters: handout, exam name, date
 \thispagestyle{empty}
 \markboth{Handout \protect\ref{#1}: #2}{Handout \protect\ref{#1}: #2}
% \pagestyle{myheadings}\htitle{\protect\ref{#1}}{#2}{#3}
 \pagestyle{myheadings}\examsolutionstitle{\protect\ref{#1}} {#2}{#3}
 \renewcommand{\theproblem}{Problem \arabic{problemnum}}
}
\newcommand{\examsolutionstitle}[3]{\noindent\vspace*{-\toppush}\newline\parbox{6.5in}
{\textit{Introduction to Algorithms}\hfill#3\newline
Massachusetts Institute of Technology \hfill 6.006\newline
%Singapore-MIT Alliance \hfill SMA5503\newline
\profs\hfill Handout #1\vspace*{-.5ex}\newline
\mbox{}\hrulefill\mbox{}}\vspace*{1ex}\mbox{}\newline
\begin{center}{\Large\bf #2}\end{center}}

\newcommand{\takehomeexam}[2]{% parameters: exam name, date
 \thispagestyle{empty}
 \markboth{\subj\ #1\hfill}{\subj\ #1\hfill}
 \pagestyle{myheadings}\examtitle{#1}{#2}
 \renewcommand{\theproblem}{Problem \arabic{problemnum}}
}

\makeatletter
\newcommand{\exambooklet}[2]{% parameters: exam name, date
 \thispagestyle{empty}
 \markboth{\subj\ #1}{\subj\ #1}
 \pagestyle{myheadings}\examtitle{#1}{#2}
 \renewcommand{\theproblem}{Problem \arabic{problemnum}}
 \renewcommand{\problem}{\newpage
 \item \let\@currentlabel=\theproblem
 \markboth{\subj\ #1, \theproblem}{\subj\ #1, \theproblem}}
}
\makeatother


\newcommand{\examtitle}[2]{\noindent\vspace*{-\toppush}\newline\parbox{6.5in}
{\textit{Introduction to Algorithms}\hfill#2\newline
Massachusetts Institute of Technology \hfill 6.006 Fall 2018\newline
%Singapore-MIT Alliance \hfill SMA5503\newline
\profs\hfill #1\vspace*{-.5ex}\newline
\mbox{}\hrulefill\mbox{}}\vspace*{1ex}\mbox{}\newline
\begin{center}{\Large\bf #1}\end{center}}

\newcommand{\grader}[1]{\hspace{1cm}\textsf{\textbf{#1}}\hspace{1cm}}

\newcommand{\points}[1]{[#1 points]\ }
\newcommand{\parts}[1]
{
  \ifnum#1=1
  (1 part)
  \else
  (#1 parts)
  \fi
  \ 
}

\newcommand{\bparts}{\begin{problemparts}}
\newcommand{\eparts}{\end{problemparts}}
\newcommand{\ppart}{\problempart}

%\newcommand{\lg} {lg\ }

\setlength{\oddsidemargin}{0pt}
\setlength{\evensidemargin}{0pt}
\setlength{\textwidth}{6.5in}
\setlength{\topmargin}{0in}
\setlength{\textheight}{8.5in}


\newcommand{\Spawn}{{\bf spawn} }
\newcommand{\Sync}{{\bf sync}}

\renewcommand{\cases}[1]{\left\{ \begin{array}{ll}#1\end{array}\right.}
\newcommand{\cif}[1]{\mbox{if $#1$}}
\newcommand{\cwhen}[1]{\mbox{when $#1$}}

\newcounter{problemnum}
\newcommand{\theproblem}{Problem \theproblemsetnum-\arabic{problemnum}}
\newenvironment{problems}{
        \begin{list}{{\bf \theproblem. \hspace*{0.5em}}}
        {\setlength{\leftmargin}{0em}
         \setlength{\rightmargin}{0em}
         \setlength{\labelwidth}{0em}
         \setlength{\labelsep}{0em}
         \usecounter{problemnum}}}{\end{list}}
\makeatletter
\newcommand{\problem}[1][{}]{\item \let\@currentlabel=\theproblem \textbf{#1}}
\makeatother

\newcounter{problempartnum}[problemnum]
\newenvironment{problemparts}{
        \begin{list}{{\bf (\alph{problempartnum})}}
        {\setlength{\leftmargin}{2.5em}
         \setlength{\rightmargin}{2.5em}
         \setlength{\labelsep}{0.5em}}}{\end{list}}
\newcommand{\problempart}{\addtocounter{problempartnum}{1}\item}

\newenvironment{truefalseproblemparts}{
        \begin{list}{{\bf (\alph{problempartnum})\ \ \ T\ \ F\hfil}}
        {\setlength{\leftmargin}{4.5em}
         \setlength{\rightmargin}{2.5em}
         \setlength{\labelsep}{0.5em}
         \setlength{\labelwidth}{4.5em}}}{\end{list}}

\newcounter{exercisenum}
\newcommand{\theexercise}{Exercise \theproblemsetnum-\arabic{exercisenum}}
\newenvironment{exercises}{
        \begin{list}{{\bf \theexercise. \hspace*{0.5em}}}
        {\setlength{\leftmargin}{0em}
         \setlength{\rightmargin}{0em}
         \setlength{\labelwidth}{0em}
         \setlength{\labelsep}{0em}
        \usecounter{exercisenum}}}{\end{list}}
\makeatletter
\newcommand{\exercise}{\item \let\@currentlabel=\theexercise}
\makeatother

\newcounter{exercisepartnum}[exercisenum]
%\newcommand{\problem}[1]{\medskip\mbox{}\newline\noindent{\bf Problem #1.}\hspace*{1em}}
%\newcommand{\exercise}[1]{\medskip\mbox{}\newline\noindent{\bf Exercise #1.}\hspace*{1em}}

\newenvironment{exerciseparts}{
        \begin{list}{{\bf (\alph{exercisepartnum})}}
        {\setlength{\leftmargin}{2.5em}
         \setlength{\rightmargin}{2.5em}
         \setlength{\labelsep}{0.5em}}}{\end{list}}
\newcommand{\exercisepart}{\addtocounter{exercisepartnum}{1}\item}


% Macros to make captions print with small type and 'Figure xx' in bold.
\makeatletter
\def\fnum@figure{{\bf Figure \thefigure}}
\def\fnum@table{{\bf Table \thetable}}
\let\@mycaption\caption
%\long\def\@mycaption#1[#2]#3{\addcontentsline{\csname
%  ext@#1\endcsname}{#1}{\protect\numberline{\csname 
%  the#1\endcsname}{\ignorespaces #2}}\par
%  \begingroup
%    \@parboxrestore
%    \small
%    \@makecaption{\csname fnum@#1\endcsname}{\ignorespaces #3}\par
%  \endgroup}
%\def\mycaption{\refstepcounter\@captype \@dblarg{\@mycaption\@captype}}
%\makeatother
\let\mycaption\caption
%\newcommand{\figcaption}[1]{\mycaption[]{#1}}

\newcounter{totalcaptions}
\newcounter{totalart}

\newcommand{\figcaption}[1]{\addtocounter{totalcaptions}{1}\caption[]{#1}}

% \psfigures determines what to do for figures:
%       0 means just leave vertical space
%       1 means put a vertical rule and the figure name
%       2 means insert the PostScript version of the figure
%       3 means put the figure name flush left or right
\newcommand{\psfigures}{0}
\newcommand{\spacefigures}{\renewcommand{\psfigures}{0}}
\newcommand{\rulefigures}{\renewcommand{\psfigures}{1}}
\newcommand{\macfigures}{\renewcommand{\psfigures}{2}}
\newcommand{\namefigures}{\renewcommand{\psfigures}{3}}

\newcommand{\figpart}[1]{{\bf (#1)}\nolinebreak[2]\relax}
\newcommand{\figparts}[2]{{\bf (#1)--(#2)}\nolinebreak[2]\relax}


\macfigures     % STATE

% When calling \figspace, make sure to leave a blank line afterward!!
% \widefigspace is for figures that are more than 28pc wide.
\newlength{\halffigspace} \newlength{\wholefigspace}
\newlength{\figruleheight} \newlength{\figgap}
\newcommand{\setfiglengths}{\ifnum\psfigures=1\setlength{\figruleheight}{\hruleheight}\setlength{\figgap}{1em}\else\setlength{\figruleheight}{0pt}\setlength{\figgap}{0em}\fi}
\newcommand{\figspace}[2]{\ifnum\psfigures=0\leavefigspace{#1}\else%
\setfiglengths%
\setlength{\wholefigspace}{#1}\setlength{\halffigspace}{.5\wholefigspace}%
\rule[-\halffigspace]{\figruleheight}{\wholefigspace}\hspace{\figgap}#2\fi}
\newlength{\widefigspacewidth}
% Make \widefigspace put the figure flush right on the text page.
\newcommand{\widefigspace}[2]{
\ifnum\psfigures=0\leavefigspace{#1}\else%
\setfiglengths%
\setlength{\widefigspacewidth}{28pc}%
\addtolength{\widefigspacewidth}{-\figruleheight}%
\setlength{\wholefigspace}{#1}\setlength{\halffigspace}{.5\wholefigspace}%
\makebox[\widefigspacewidth][r]{#2\hspace{\figgap}}\rule[-\halffigspace]{\figruleheight}{\wholefigspace}\fi}
\newcommand{\leavefigspace}[1]{\setlength{\wholefigspace}{#1}\setlength{\halffigspace}{.5\wholefigspace}\rule[-\halffigspace]{0em}{\wholefigspace}}

% Commands for including figures with macpsfig.
% To use these commands, documentstyle ``macpsfig'' must be specified.
\newlength{\macfigfill}
\makeatother
\newlength{\bbx}
\newlength{\bby}
\newcommand{\macfigure}[5]{\addtocounter{totalart}{1}
\ifnum\psfigures=2%
\setlength{\bbx}{#2}\addtolength{\bbx}{#4}%
\setlength{\bby}{#3}\addtolength{\bby}{#5}%
\begin{flushleft}
\ifdim#4>28pc\setlength{\macfigfill}{#4}\addtolength{\macfigfill}{-28pc}\hspace*{-\macfigfill}\fi%
\mbox{\psfig{figure=./#1.ps,%
bbllx=#2,bblly=#3,bburx=\bbx,bbury=\bby}}
\end{flushleft}%
\else\ifdim#4>28pc\widefigspace{#5}{#1}\else\figspace{#5}{#1}\fi\fi}
\makeatletter

\newlength{\savearraycolsep}
\newcommand{\narrowarray}[1]{\setlength{\savearraycolsep}{\arraycolsep}\setlength{\arraycolsep}{#1\arraycolsep}}
\newcommand{\normalarray}{\setlength{\arraycolsep}{\savearraycolsep}}

\newcommand{\hint}{{\em Hint:\ }}

% Macros from /th/u/clr/mac.tex

\newcommand{\set}[1]{\left\{ #1 \right\}}
\newcommand{\abs}[1]{\left| #1\right|}
\newcommand{\card}[1]{\left| #1\right|}
\newcommand{\floor}[1]{\left\lfloor #1 \right\rfloor}
\newcommand{\ceil}[1]{\left\lceil #1 \right\rceil}
\newcommand{\ang}[1]{\ifmmode{\left\langle #1 \right\rangle}
   \else{$\left\langle${#1}$\right\rangle$}\fi}
        % the \if allows use outside mathmode,
        % but will swallow following space there!
\newcommand{\paren}[1]{\left( #1 \right)}
\newcommand{\bracket}[1]{\left[ #1 \right]}
\newcommand{\prob}[1]{\Pr\left\{ #1 \right\}}
\newcommand{\Var}{\mathop{\rm Var}\nolimits}
\newcommand{\expect}[1]{{\rm E}\left[ #1 \right]}
\newcommand{\expectsq}[1]{{\rm E}^2\left[ #1 \right]}
\newcommand{\variance}[1]{{\rm Var}\left[ #1 \right]}
\renewcommand{\choose}[2]{{{#1}\atopwithdelims(){#2}}}
\def\pmod#1{\allowbreak\mkern12mu({\rm mod}\,\,#1)}
\newcommand{\matx}[2]{\left(\begin{array}{*{#1}{c}}#2\end{array}\right)}
\newcommand{\Adj}{\mathop{\rm Adj}\nolimits}

\newtheorem{theorem}{Theorem}
\newtheorem{lemma}[theorem]{Lemma}
\newtheorem{corollary}[theorem]{Corollary}
\newtheorem{xample}{Example}
\newtheorem{definition}{Definition}
\newenvironment{example}{\begin{xample}\rm}{\end{xample}}
\newcommand{\proof}{\noindent{\em Proof.}\hspace{1em}}
\def\squarebox#1{\hbox to #1{\hfill\vbox to #1{\vfill}}}
\newcommand{\qedbox}{\vbox{\hrule\hbox{\vrule\squarebox{.667em}\vrule}\hrule}}
\newcommand{\qed}{\nopagebreak\mbox{}\hfill\qedbox\smallskip}
\newcommand{\eqnref}[1]{(\protect\ref{#1})}

%%\newcommand{\twodots}{\mathinner{\ldotp\ldotp}}
\newcommand{\transpose}{^{\mbox{\scriptsize \sf T}}}
\newcommand{\amortized}[1]{\widehat{#1}}

\newcommand{\punt}[1]{}

%%% command for putting definitions into boldface
% New style for defined terms, as of 2/23/88, redefined by THC.
\newcommand{\defn}[1]{{\boldmath\textit{\textbf{#1}}}}
\newcommand{\defi}[1]{{\textit{\textbf{#1\/}}}}

\newcommand{\red}{\leq_{\rm P}}
\newcommand{\lang}[1]{%
\ifmmode\mathord{\mathcode`-="702D\rm#1\mathcode`\-="2200}\else{\rm#1}\fi}

%\newcommand{\ckt}[1]{\ifmmode\mathord{\mathcode`-="702D\sc #1\mathcode`\-="2200}\else$\mathord{\mathcode`-="702D\sc #1\mathcode`\-="2200}$\fi}
\newcommand{\ckt}[1]{\ifmmode \sc #1\else$\sc #1$\fi}

%% Margin notes - use \notesfalse to turn off notes.
\setlength{\marginparwidth}{0.6in}
\reversemarginpar
\newif\ifnotes
\notestrue
\newcommand{\longnote}[1]{
  \ifnotes
    {\medskip\noindent Note: \marginpar[\hfill$\Longrightarrow$]
      {$\Longleftarrow$}{#1}\medskip}
  \fi}
\newcommand{\note}[1]{
  \ifnotes
    {\marginpar{\tiny \raggedright{#1}}}
  \fi}


\newcommand{\reals}{\mathbbm{R}}
\newcommand{\integers}{\mathbbm{Z}}
\newcommand{\naturals}{\mathbbm{N}}
\newcommand{\rationals}{\mathbbm{Q}}
\newcommand{\complex}{\mathbbm{C}}

\newcommand{\oldreals}{{\bf R}}
\newcommand{\oldintegers}{{\bf Z}}
\newcommand{\oldnaturals}{{\bf N}}
\newcommand{\oldrationals}{{\bf Q}}
\newcommand{\oldcomplex}{{\bf C}}

\newcommand{\w}{\omega}                 %% for fft chapter

\newenvironment{closeitemize}{\begin{list}
{$\bullet$}
{\setlength{\itemsep}{-0.2\baselineskip}
\setlength{\topsep}{0.2\baselineskip}
\setlength{\parskip}{0pt}}}
{\end{list}}

% These are necessary within a {problems} environment in order to restore
% the default separation between bullets and items.
\newenvironment{normalitemize}{\setlength{\labelsep}{0.5em}\begin{itemize}}
                              {\end{itemize}}
\newenvironment{normalenumerate}{\setlength{\labelsep}{0.5em}\begin{enumerate}}
                                {\end{enumerate}}

%\def\eqref#1{Equation~(\ref{eq:#1})}
%\newcommand{\eqref}[1]{Equation (\ref{eq:#1})}
\newcommand{\eqreftwo}[2]{Equations (\ref{eq:#1}) and~(\ref{eq:#2})}
\newcommand{\ineqref}[1]{Inequality~(\ref{ineq:#1})}
\newcommand{\ineqreftwo}[2]{Inequalities (\ref{ineq:#1}) and~(\ref{ineq:#2})}

\newcommand{\figref}[1]{Figure~\ref{fig:#1}}
\newcommand{\figreftwo}[2]{Figures \ref{fig:#1} and~\ref{fig:#2}}

\newcommand{\liref}[1]{line~\ref{li:#1}}
\newcommand{\Liref}[1]{Line~\ref{li:#1}}
\newcommand{\lirefs}[2]{lines \ref{li:#1}--\ref{li:#2}}
\newcommand{\Lirefs}[2]{Lines \ref{li:#1}--\ref{li:#2}}
\newcommand{\lireftwo}[2]{lines \ref{li:#1} and~\ref{li:#2}}
\newcommand{\lirefthree}[3]{lines \ref{li:#1}, \ref{li:#2}, and~\ref{li:#3}}

\newcommand{\lemlabel}[1]{\label{lem:#1}}
\newcommand{\lemref}[1]{Lemma~\ref{lem:#1}} 

\newcommand{\exref}[1]{Exercise~\ref{ex:#1}}

\newcommand{\handref}[1]{Handout~\ref{#1}}

\newcommand{\defref}[1]{Definition~\ref{def:#1}}

% (1997.8.16: Victor Luchangco)
% Modified \hlabel to only get date and to use handouts counter for number.
%   New \handout and \handoutwithouttitle commands in newmac.tex use this.
%   The date is referenced by <label>-date.
%   (Retained old definition as \hlabelold.)
%   Defined \hforcelabel to use an argument instead of the handouts counter.

\newcounter{handouts}
\setcounter{handouts}{0}

\newcommand{\hlabel}[2]{%
\stepcounter{handouts}
{\edef\next{\write\@auxout{\string\newlabel{#1}{{\arabic{handouts}}{0}}}}\next}
\write\@auxout{\string\newlabel{#1-date}{{#2}{0}}}
}

\newcommand{\hforcelabel}[3]{%          Does not step handouts counter.
\write\@auxout{\string\newlabel{#1}{{#2}{0}}}
\write\@auxout{\string\newlabel{#1-date}{{#3}{0}}}}


% less ugly underscore
% --juang, 2008 oct 05
\renewcommand{\_}{\vrule height 0 pt depth 0.4 pt width 0.5 em \,}

\newcommand{\theproblemsetnum}{2}
\newcommand{\releasedate}{Thursday, September 13}
\newcommand{\partaduedate}{Thursday, September 20}

\title{6.006 Problem Set 2}

\begin{document}

\handout{Problem Set \theproblemsetnum}{\releasedate}
\textbf{All parts are due {\bf \partaduedate} at {\bf 11PM}}.

\setlength{\parindent}{0pt}
\medskip\hrulefill\medskip

{\bf Name:} Robert Durfee

\medskip

{\bf Collaborators:} None

\medskip\hrulefill

\begin{problems}

\problem 

\begin{problemparts}

    \problempart From the fact that there is a zipline path between $ e_0 $
        and $ e_1 $, it is known that all elements between are less than both $
        e_0 $ and $ e_1 $. Therefore, in the left half (with $ e_0 $), all the
        elements to the right are less than $ e_0 $. Similarly, for the right
        half (with $ e_1 $), all the elements to the left are less than $ e_1 $.

        Furthermore, from the fact that $ e_0 $ to $ e_1 $ is the longest
        zipline path, if there was an element to the left of $ e_0 $,
        assuming WLOG that $ e_0 < e_1 $, there would exist an longer zipline
        path from this new element to $ e_1 $. Therefore, since there is not,
        $ e_0 $ must be the maximum in the left half as it must be greater
        than all elements to the left and right.

    \problempart {\bf Description} First, split the array into two equal
        sized halves, $ A_L $ and $ A_R $. Now, the longest zipline path is
        either fully contained in $ A_L $ or $ A_R $, or it has one endpoint in $
        A_L $ one endpoint in $ A_R$.

        To handle the first two cases, recurse over the individual halves $
        A_L $ and $ A_R $ until the length of each half is $ 2 $, then return
        the elements as a tuple representing a zipline path of length $ 2 $
        as there is no smaller division to be made.
        
        For the second case, compute the maxima $ e_L $ and $ e_R $ from each
        half $ A_L $ and $ A_R $, respectively. Select the lesser of the two
        maxima. If the lesser maxima is $ e_L $, then the longest zipline
        path between the halves $ A_L $ and $ A_R $ must start at $ e_L $.
        Scan to the right until the first element greater than $ e_L $ is
        found within $ A_R $. If the lesser maxima is $ e_R $, the the
        longest zipline path between halves $ A_L $ and $ A_R $ must end at $
        e_R $. Scan to the left until the first element greater than $ e_R $
        is found within $ A_L$.

        \smallbreak

        {\bf Correctness} For an array $ A $, let $P(A)$ be the statement
        that my algorithm correctly returns the longest zipline path (as
        defined in the problem) in an array $A$. I will prove $ P(A) $ is
        true for any array $ A $ by strong induction on $|A|$.

        {\it Base Case}: Consider when $|A| = 2$. My algorithm returns the
        two elements in $ A $ and states it is the longest zipline path. This
        is trivially true and is explicitly stated in the problem.

        {\it Inductive Case}: For the induction hypothesis, suppose that
        $P(A)$ is true for all arrays of length $< n$; that is, suppose that
        for any array $ A $ of length $< n$ my algorithm correctly returns
        the longest zipline path.

        Now consider an array $ A $ of length $n$. My algorithm divides $ A $
        into two halves $ A_0 $ and $A_1$ of size $ < n $; therefore, the
        returned longest zipline paths $ p_0 $ and $ p_1 $ are correct by the
        induction hypothesis.

        Now, the maximum within each $A_0$ and $ A_1 $ is determined. Given
        the work in Part A, the lesser of the two maxima, $ e_i $ must be an
        endpoint of any path that straddles $ A_0 $ and $ A_1 $. Therefore,
        to find the other endpoint, scan towards the opposite section until
        the first element greater than $ e_i $ is located. Since by
        definition, every element between the endpoints has to be less than
        both the endpoints, this must be the other endpoint.

        The length of all three prospective zipline paths is now known and
        the largest is then returned. Therefore I have shown by induction
        that my algorithm correctly returns the largest zipline path in any
        array $ A $ with distinct elements.

        \smallbreak

        {\bf Running Time} To find the two maxima in each half requires a
        total of $ O(n) $ work. To find he the matching endpoint of the
        lesser maxima also requires, at worst, $ O(n) $ work. Therefore, the
        combining step of the divide and conquer algorithm described is $
        O(n) $. Each division reduces the amount of work by $ 2 $ and splits
        into $ 2 $ recursions. The recurrence that matches this description is
        $$ T(n) = 2 T(n/2) + O(n) $$
        Therefore, by the Master Theorem Case II, the running time is $ O(n
        \log n ) $.

\end{problemparts}

\newpage
\problem {\bf Description} (This is a little much to explain in words...
    Python "pseudo" code provided on next page.)
    
    First, separate the panel lower and upper heights into two arrays. Sort
    them in increasing order. Then, iterate across the lower and upper height
    arrays concurrently. If the next lower height is less than the next upper
    height, an interval has been found. After the interval is output,
    increase the count for number of overlapping panels. If the next lower
    height is greater than the next upper height, an interval has been found
    as well. After the interval is output, decrease the count for number of
    overlapping panels. If the next lower and upper heights are the same, an
    interval has not been found, just increase both indexes and output
    nothing. Continue similarly through whichever array has remaining
    elements after the concurrent loop terminates.

    \smallbreak

    {\bf Correctness} The intervals must appear between consectutive
    lower/lower, upper/upper, lower/upper, or upper/lower height. By sorting
    the arrays of upper and lower heights individually, and iterating over
    each, each interval will be encountered. Additionally, any time an upper
    edge is encountered, the number of panels must decrease and any time a
    lower edge is encountered, the number of panels must increase, unless
    their interval is zero. Therefore, since my algorithm will cover every
    interval and correctly increment or decrement the number of panels
    overlapping, the output must be correct.

    \smallbreak

    {\bf Running Time} The sorting of the arrays each takes $ O(n \log n ) $,
    assuming merge/heap sort. Then, the concurrent loop will require $ O(2n)
    $ work to iterate over completely. Therefore, total work is $ O(2 n \log
    n + 2n) \in O(n \log n) $.

\newpage

{\tt
def shadow\_art\_divide(A):

\quad l = sorted(map(lambda p: p[0], A))

\quad     u = sorted(map(lambda p: p[1], A))

\quad     n = 1, li = 1, ui = 0, v = l[0], o = []

\quad     while li < len(l) and ui < len(l):

\quad \quad         if l[li] < u[ui]:

\quad \quad \quad             if v != l[li]: o.append(((v, l[li]), n))

\quad \quad \quad             n += 1, v = l[li], li += 1

\quad \quad         elif l[li] > u[ui]:

\quad \quad \quad             if v != u[ui]: o.append(((v, u[ui]), n))

\quad \quad \quad             n -= 1, v = u[ui], ui += 1

\quad \quad         else: ui += 1, li += 1

\quad     while li < len(l):

\quad \quad         if v != l[li]: o.append(((v, l[li]), n))

\quad \quad         n += 1, v = l[li], li += 1

\quad     while ui < len(u):

\quad \quad         if v != u[ui]: o.append(((v, u[ui]), n))

\quad \quad         n -= 1, v = u[ui], ui += 1

\quad     return o
}

\newpage

\problem {\bf Description} The double sided dynamic array has five properties:

    \begin{itemize}
        \item \ {\tt next\_left}: An index pointing to slot for {\tt
            insert\_left()}. Initialized to $ -1 $.
        \item \ {\tt next\_right}: An index pointing to slot for {\tt
            insert\_right()}. Initialized to $ 0 $.
        \item \ {\tt size}: Total number of slots in {\tt array}. Initialized
            to $ 0 $
        \item \ {\tt array}: Data contained in the dynamic array. Initialized
            to {\tt []}.
        \item \ {\tt len}: A computed property representing {\tt next\_right -
            next\_left + 1}, which is the number of nonempty elements in the
            array.
    \end{itemize}

    The double sided dynamic array has seven methods:

    \begin{itemize}
        \item \ {\tt at(i)}: Returns the element located at the $ i $th index
            offset base $ 0 $ from the first nonempty element in the dynamic
            array. This is calculated by {\tt array[next\_left + i + 1]}.
        \item \ {\tt resize\_left(new\_size)}: Sets the {\tt size} of the
            dynamic array to {\tt new\_size} and places all newly allocated
            empty slots to the left of the data contained in the array.
            Resets the {\tt next\_left} and {\tt next\_right} values
            accordingly.
        \item \ {\tt resize\_right(new\_size)}: Sets the {\tt size} of the
            dynamic array to {\tt new\_size} and places all newly allocated
            empty slots to the right of the data contained in the array. {\tt
            next\_left} and {\tt next\_right} values should remain the same.
        \item \ {\tt insert\_left(x)}: Inserts the element {\tt x} at {\tt
            next\_left}, which is the next open slot in the array, and
            decrements {\tt next\_left}. If {\tt next\_left == -1}, then {\tt
            resize\_left()} is called with the {\tt new\_size = size + len}
            which doubles the number of empty elements to the left of the
            data in the array.
        \item \ {\tt insert\_right(x)}: Inserts the element {\tt x} at {\tt
            next\_right}, which is the next open slot in the array, and
            increments {\tt next\_right}. If {\tt next\_right == size}, then
            {\tt resize\_right()} is called with the {\tt new\_size = size +
            len} which doubles the number of empty elements to the right of
            the data in the array.
        \item \ {\tt delete\_left()}: Reduces the index {\tt next\_left}. If
            the number of empty elements to the left has become {\tt 3 *
            len}, then {\tt resize\_left()} is called with {\tt new\_size =
            size - 2 * len} which ensures that {\tt len} empty elements still
            exist to the left to prevent immediate resizing.
        \item \ {\tt delete\_right()}: Reduces the index {\tt next\_right}. If
            the number of empty elements to the right has become {\tt 3 *
            len}, then {\tt resize\_right()} is called with {\tt new\_size =
            size - 2 * len} which ensures that {\tt len} empty elements still
            exist to the right to prevent immediate resizing.
    \end{itemize}

    \smallbreak

    {\bf Correctness}: 

    \begin{itemize}
        \item \ {\tt at(i)}: Since {\tt next\_left} holds the location of the
            first empty element to the left of all the data in the array,
            {\tt next\_left + 1} must be the first nonempty element in the
            array. Any base index $ i $ plus that will correspond the $i$th
            nonempty element in the array.
        \item \ {\tt insert\_left(new\_size)}: Since {\tt next\_left} is
            initialized to one less than the current element and is always
            decremented after insert, it will always point to the next empty
            slot to the left. Also, if {\tt next\_left} is outside the bounds
            of the array, it is resized with additional elements to the left,
            thus there will always be space for the new element.
        \item \ {\tt insert\_right(x)}: Since {\tt next\_right} is
            initialized to one more than the current element and is always
            incremented after insert, it will always point to the next empty
            slot to the right. Also, if {\tt next\_right} is outside the bounds
            of the array, it is resized with additional elements to the
            right, thus there will always be space for the new element.
        \item \ {\tt delete\_left()}: Since {\tt next\_left} represents the
            information regarding the left bound of the data contained in the
            array, if it is incremented, the data recognized by the data
            structure will shrink by one element on the left, there by
            removing the leftmost element. Furthermore, shrinking will only
            reduce the number of elements in such a way that prevents
            accidental removal of data contained in the array.
        \item \ {\tt delete\_right()}: Since {\tt next\_right} represents the
            information regarding the right bound of the data contained in the
            array, if it is decremented, the data recognized by the data
            structure will shrink by one element on the right, there by
            removing the rightmost element. Furthermore, shrinking will only
            reduce the number of elements in such a way that prevents
            accidental removal of data contained in the array.
    \end{itemize}

    \smallbreak

    {\bf Running Time}: 

    \begin{itemize}
        \item \ {\tt at(i)}: The calculation performed is constant therefore
            running time is $ O(1) $.
        \item \ {\tt insert\_left(new\_size)}: On a typical insert, only one
            element is set which happens in $ O(1) $ work. However, every $ 2^k
            $ inserts will result in a resize. Therefore, $ n $ inserts will
            will cost $ \Theta(1 + 2 + 4 + \ldots + n) = \Theta(n) $ work.
            Thus, insert is amortized $ O(1) $.
        \item \ {\tt insert\_right(x)}: Same as above.
        \item \ {\tt delete\_left()}: On a typical delete, only the index
            {\tt next\_left} is altered which happens in $ O(1) $ work.
            However, every $ 4^k $ deletes, a resize occurs to reduce storage
            which takes $ \Theta(1 + 4 + 16 + \ldots + n) = \Theta(n) $ work.
            Thus, delete is amortized $ O(1) $.
        \item \ {\tt delete\_right()}: Same as above.
    \end{itemize}

\newpage
\problem

\begin{problemparts}
\problempart Given that $ \delta = \min(\delta_L, \delta_R) $, then every
    point in the left and right half (individually) must be at least $ \delta
    $ away from each other. It is clear the see that for each point $ q_i $
    in $ V $, we only need to look at all point $ q_j $ such that $ | y_i -
    y_j | < \delta $. Given that all points are at least $ \delta $ apart,
    the worst case situation is there are 8 points together forming a
    rectangle $ 2 \delta \times \delta $, one in each corner and two sets of
    two coindicental points along the edge of the rectangle and along the
    median. Since one of these points must be the point $ q_i $, there are
    only at most 7 other points that must be checked.
\problempart If we sort the points by $ x $- and then $ y $- components ahead
    of time, and assure that our algorithm is stable, then each division
    reduces the amount of work by $ 2 $ and splits into $ 2 $ recursions and
    each step requires $ O(n) $ work (given that we will never have to make
    the full $n ^ 2$ comparisons within $ V $ as proven above). Thus, the
    recurrence that matches this algorithm is
    $$ T(n) = 2 T(n/2) + O(n) $$
    Therefore, by the Master Theorem Case II, the running time is $ O(n \log
    n) $.
\problempart Submit your implementation to {\small\url{alg.mit.edu/PS2}}
\end{problemparts}

\end{problems}

\end{document}

