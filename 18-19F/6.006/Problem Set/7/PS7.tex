\documentclass[12pt,twoside]{article}

\input{macros-fa18}
\newcommand{\theproblemsetnum}{7}
\newcommand{\releasedate}{Thursday, October 25}
\newcommand{\partaduedate}{Thursday, November 1}

\title{6.006 Problem Set 7}

\begin{document}

\handout{Problem Set \theproblemsetnum}{\releasedate}
\textbf{All parts are due {\bf \partaduedate} at {\bf 11PM}}.

\setlength{\parindent}{0pt}
\medskip\hrulefill\medskip

{\bf Name:} Robert Durfee

\medskip

{\bf Collaborators:} Joanna Cohen 

\medskip\hrulefill

\begin{problems}

\problem

\begin{problemparts}

\problempart There are cycles in the graph between vertices $(4-7)$ and $(7-9)$.

\problempart After running DFS visiting each node in order 0-12, here is the
parent forest (shown in red)

\begin{center}
    \includegraphics[scale=0.55]{Images/P1B.PNG}
\end{center}

The {\it reversed} finishing time order is

{\tt 12, 11, 8, 3, 4, 7, 9, 6, 10, 5, 2, 1, 0}

\problempart The reversed order would be a topological sort if the edges $(7
\rightarrow 4)$ and $(9 \rightarrow 7)$ were removed.

\end{problemparts}

\newpage

\problem {\bf Description} Run Bellman-Ford on the unmodified graph. However,
only run the outer loop (which typically runs $|V| - 1$ times) $k$ times.

{\bf Correctness} When running Bellman-Ford, we typically run the relaxation
process $|V| - 1$ times. This is to ensure that if there is a path connecting all
vertices, all its edges have been relaxed to the shortest possible distance.
But, in this case, there is a guarantee that the longest path only involves
$k$ edges. Therefore, we can safely terminate Bellman-Ford after $k$
iterations of the outer loop. The rest of the proof is analogous to the
standard Bellman-Ford given in lecture.

{\bf Running Time} As in the standard Bellman-Ford, initialization takes
$O(|V|)$. Then for $k$ times (instead of $|V| - 1$ times), relax $|E|$ edges.
Therefore, the running time is $O(|V| + k|E|)$.

\problem {\bf Description} Create a graph where the different states of
tranfiguration are vertices and the altered mass ratios $w = -r + 1$ as the
edges. Run Bellman-Ford on this graph with source salt where the shortest
path is the one that provides the most of any material starting with salt. If
there is a negative-weight cycle along the path from salt to a material, then
an infinite amount of that material can be produced.

To determine the maximum gold, assuming the path from salt to gold does not
contain a negative-weight cycle for now, follow parent pointers from gold to
salt multiplying the original amount of $s$ by the ratio $r = -(w - 1)$
(where $w$ is the weight of the edge in the constructed graph).

If there is a negative-weight cycle detected along this path, then infinite
gold can be produced.

{\bf Correctness} Bellman-Ford will give the shortest path as shown in
lecture. The shortest path between states with weights $w = -r + 1$ must
yield the maximum gold because maximum paths now become minimum paths as all
$r$ were originally greater than $0$. Furthermore, negative-weight cycles
will exist only for cycles of weight $w < 0$ or, equivalently, $r > 1$, which
yields infinite material reachable by that cycle.

{\bf Running Time} Analogous to Bellman-Ford runtime analysis with graph
construction done linearly. Therefore, $O(|V||E|)$.

\newpage

\problem

\problem {\bf Description} Construct a graph with vertices as road
intersections and roads are represented by two directed, weighted edges. One
edge has weight $a - h_R$ and the other $a - h_L$ to represent traveling
along the road one direction or the other. Run Bellman-Ford to find the SSSP
from Lee's location. If the shortest path from Lee to Whike is negative (in
terms of the weights constructed), assuming no negative-weight cycles for
now, then Lee can travel to Whike with strictly more candy. If Bellman-Ford
detects a cycle reachable to both Whike and Lee, then an infinite amount of
candy can be accumulated.

{\bf Correctness} Bellman-Ford will give the shortest path as shown in
lecture. The shortest path between Whike and Lee using the weights $a - h_R$
and $a - h_L$ must be the one that maximizes candy accumulation as maximum
paths become minimum paths when weights are negated. If a negative-weight
cycle is reached, then there must exist a cycle where more candy is gained
than lost, thus infinite candy can be gained.

{\bf Running Time} Analogous to Bellman-Ford runtime analysis with graph
construction done linearly. Therefore $O(|V||E|)$.

\newpage

\problem

\begin{problemparts}

\problempart {\bf Description} Run topological sort on the graph. Visit each
vertex in topological sort order. For each vertex, {\it contract} each
adjacent edge. Contract is the opposite of relax in that a distance
estimation is increased if {\tt d[v] < d[u] + w(u, v)} and the parent is
updated. The distances stored will be the maximum weighted paths.

{\bf Correctness} Since this is a DAG, a topological sort exists.

{\bf Running Time} 

\problempart {\bf Description} 

{\bf Correctness}

{\bf Running Time}

\problempart Submit your implementation to {\small\url{alg.mit.edu/PS7}}

\end{problemparts}

\end{problems}

\end{document}

