\documentclass[12pt,twoside]{article}

\input{macros-fa18}
\newcommand{\theproblemsetnum}{7}
\newcommand{\releasedate}{Thursday, October 25}
\newcommand{\partaduedate}{Thursday, November 1}

\title{6.006 Problem Set 7}

\begin{document}

\handout{Problem Set \theproblemsetnum}{\releasedate}
\textbf{All parts are due {\bf \partaduedate} at {\bf 11PM}}.

\setlength{\parindent}{0pt}
\medskip\hrulefill\medskip

{\bf Name:} Robert Durfee

\medskip

{\bf Collaborators:} Joanna Cohen 

\medskip\hrulefill

\begin{problems}

\problem

\begin{problemparts}

\problempart There are cycles in the graph between vertices $(4-7)$ and $(7-9)$.

\problempart After running DFS visiting each node in order 0-12, here is the
parent forest (shown in red)

\begin{center}
    \includegraphics[scale=0.55]{Images/P1B.PNG}
\end{center}

The {\it reversed} finishing time order is

{\tt 12, 11, 8, 3, 4, 7, 9, 6, 10, 5, 2, 1, 0}

\problempart The reversed order would be a topological sort if the edges $(7
\rightarrow 4)$ and $(9 \rightarrow 7)$ were removed.

\end{problemparts}

\newpage

\problem {\bf Description} Run Bellman-Ford on the unmodified graph. However,
only run the outer loop (which typically runs $|V| - 1$ times) $k$ times.

{\bf Correctness} When running Bellman-Ford, we typically run the relaxation
process $|V| - 1$ times. This is to ensure that if there is a path connecting all
vertices, all its edges have been relaxed to the shortest possible distance.
But, in this case, there is a guarantee that the longest path only involves
$k$ edges. Therefore, we can safely terminate Bellman-Ford after $k$
iterations of the outer loop. The rest of the proof is analogous to the
standard Bellman-Ford given in lecture.

{\bf Running Time} As in the standard Bellman-Ford, initialization takes
$O(|V|)$. Then for $k$ times (instead of $|V| - 1$ times), relax $|E|$ edges.
Therefore, the running time is $O(|V| + k|E|)$.

\problem {\bf Description} Create a graph where the different states of
tranfiguration are vertices and the altered mass ratios $w = -\log(r)$ as the
edges. Run Bellman-Ford on this graph with source salt where the shortest
path is the one that provides the most of any material starting with salt. If
there is a negative-weight cycle along the path from salt to a material, then
an infinite amount of that material can be produced.

To determine the maximum gold, assuming the path from salt to gold does not
contain a negative-weight cycle for now, follow parent pointers from gold to
salt multiplying the original amount of $s$ by the ratio $r = e^{-w}$
(where $w$ is the weight of the edge in the constructed graph).

If there is a negative-weight cycle detected along this path, then infinite
gold can be produced.

{\bf Correctness} Bellman-Ford will give the shortest path as shown in
lecture. The shortest path between states with weights $w = -\log(r)$ must
yield the maximum gold because maximum multiplicative paths now become
minimum additive paths given the monotonically increasing nature of
logarithms and ability to convert products into sums. Furthermore,
negative-weight cycles will exist only for cycles of weight $w < 0$ or,
equivalently, $r > 1$, which yields infinite material reachable by that
cycle.

{\bf Running Time} Analogous to Bellman-Ford runtime analysis with graph
construction done linearly. Therefore, $O(|V||E|)$.

\newpage

\problem {\bf Description} Construct a graph where each of the edges are
weighted according to the health lost at the destination cave if and only if
the destination cave is of strictly higher elevation. If the cave is of equal
height, delete the edge. Run topological sort and DAG SSSP which will return
the shortest weighted path from their current location to all destinations.
If a destination is an exit to the tunnel system and has weight less thant
$k$, it is possible for them to escape without seriously harming Zink's
sword.

{\bf Correctness} By construction, edges are directed in the direction of
strictly increasing height. This creates a DAG as once you go up a level, you
must continue to go up, you cannot go down or stay on the same level. From
this, a topological sort must exist and DAG SSSP will return all shortest
paths. If the shortest path to a system exit is less than $k$, then that path
provides the way for Zink and Lelda to leave the system.

{\bf Running Time} The traversal to correct the edges is linear $O(|V| +
|E|)$. Then, DAG SSSP will take $O(|V| + |E|)$. Therefore, overall runtime is
$O(|V| + |E|)$.

\problem {\bf Description} Note: This assumes that the source and destination
houses are located at intersections. If they are not, Lee should start/stop
at the closest intersection and proceed as follows. 

Construct a graph with vertices as road intersections and roads are
represented by two directed, weighted edges. One edge has weight $a - h_R$
and the other $a - h_L$ to represent traveling along the road one direction
or the other. Run Bellman-Ford to find the SSSP from Lee's location. If the
shortest path from Lee to Whike is negative (in terms of the weights
constructed), assuming no negative-weight cycles for now, then Lee can travel
to Whike with strictly more candy. If Bellman-Ford detects a cycle reachable
to both Whike and Lee, then an infinite amount of candy can be accumulated.

{\bf Correctness} Bellman-Ford will give the shortest path as shown in
lecture. The shortest path between Whike and Lee using the weights $a - h_R$
and $a - h_L$ must be the one that maximizes candy accumulation as maximum
paths become minimum paths when weights are negated. If a negative-weight
cycle is reached, then there must exist a cycle where more candy is gained
than lost, thus infinite candy can be gained.

{\bf Running Time} Analogous to Bellman-Ford runtime analysis with graph
construction done linearly. Therefore $O(|V||E|)$.

\newpage

\problem

\begin{problemparts}

\problempart {\bf Description} Run topological sort on the graph. Negate all
edge weights. Visit each vertex in topological sort order performing DAG SSSP
described in lecture. The shortest paths returned are actually the longest.

{\bf Correctness} Since this is a DAG, a topological sort exists. It was
shown in lecture that topological order is the optimal way to perform
relaxation. The shortest negative path is, intuitively, the longest positive
path. Note that it was also shown that DAG SSSP works for negative weights.

{\bf Running Time} This is essentially the same algorithm as the DAG SSSP
presented in lecture. Therefore, it has the same running time $O(|V| + |E|)$
as each vertice is hit once as well as each edge.

\problempart {\bf Description} Take the transformations and construct a
reversed directed graph where edges point from the target of each
transformation to one of the sources for each source. The weights of each
edge is the time it takes to construct the transformation. Run the algorithm
described above on the target which will give the maximum path to each
possible source. For each source, follow the parent pointers back to the
target and record the length along the way. Return the greatest length.

{\bf Correctness} It wouldn't make sense for the file conversions to result
in a cycle, therefore I will assume that there are no cycles. The reverse of a
directed graph will turn SSSP into single-target shortest-path, as discussed
in lecture. As a result, since the algorithm described above performs
efficient single-source longest path on the reverse DAG, this will convert to
the single-target longest path for each possible source. The longest of each
of the provided sources is the limiting factor on the overall speed of the
parallelization of the DAG. Therefore, the maximum longest path with respect
to the given sources returns the minimum build time.

{\bf Running Time} To construct the graph takes linear time $O(|V| + |E|)$.
The algorithm described above also takes linear time $O(|V| + |E|)$. Then,
each iteration over sources $|S|$ could, at worst, hit all edges, thus
$O(|E|)$. Therefore, the overall runtime is $O(|V| + |S||E|)$. If $|S|
\longrightarrow |V|$, this approaches $O(|V||E|)$.

\problempart Submit your implementation to {\small\url{alg.mit.edu/PS7}}

\end{problemparts}

\end{problems}

\end{document}
