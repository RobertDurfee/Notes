\documentclass{article}
\usepackage{tikz}
\usepackage{float}
\usepackage{enumerate}
\usepackage{amsmath}
\usepackage{amsthm}
\usepackage{bm}
\usepackage{indentfirst}
\usepackage{siunitx}
\usepackage[utf8]{inputenc}
\usepackage{graphicx}
\graphicspath{ {Images/} }
\usepackage{float}
\usepackage{mhchem}
\usepackage{chemfig}
\allowdisplaybreaks

\title{6.041 Problem Set 2}
\author{Robert Durfee}
\date{September 18, 2018}

\begin{document}

\maketitle

\section*{Problem 1}

\textit{Consider the events $ A $, $ B $, and $ C $. Are the following
statements always true? Justify your answer. (Assume that all conditioning
events have positive probability.)}

\subsection*{Part A}

\textit{$ P \left( A \cap B \cap \overline{C} \right) = P \left( A \cap B
\right) P \left( \overline{C} \mid A \cap B \right) $}

\subsection*{Part B}

\textit{$ P \left( A \cap B \cap \overline{C} \right) = P \left( A \right) P
\left( \overline{C} \mid A \right) P \left( B \mid A \cap \overline{C} \right)
$}

\subsection*{Part C}

\textit{$ P \left( A \cap B \cap \overline{C} \right) = P \left( A \right) P
\left( \overline{C} \cap A \mid A \right) P \left( B \mid A \cap \overline{C}
\right) $}

\subsection*{Part D}

\textit{$ P \left( A \cap B \mid C \right) = P \left( A \mid C \right) P \left(
B \mid A \cap C \right) $}

\section*{Problem 2}

\textit{Suppose that events $ A $ and $ B $ are conditionally independent given
event $ C $. Suppose that $ P \left( C \right) > 0 $ and $ P \left( \overline{C}
\right) > 0 $.}

\subsection*{Part A}

\textit{Are $ A $ and $ \overline{B} $ guaranteed to be conditionally
independent given $ C $? Justify your answer.}

\subsection*{Part B}

\textit{Are $ A $ and $ B $ guaranteed to be conditionally independent given $
\overline{C} $? Justify your answer.}

\section*{Problem 3}

\textit{You roll two five-sided dice. The sides of each die are numbered from $
1 $ to $ 5 $. The dice are “fair” (all sides are equally likely), and the two
die rolls are independent.}

\subsection*{Part A}

\textit{Event $ A $ is “the total is $ 10 $” (i.e., the sum of the results of
the two die rolls is $ 10 $).}

\subsubsection*{Part I}

\textit{Is event $ A $ independent of the event “at least one of the die rolls
resulted in a $ 5 $”?}

\subsubsection*{Part II}

\textit{Is event $ A $ independent of the event “at least one of the die rolls
resulted in a $ 1 $”?}

\subsection*{Part B}

\textit{Event $ B $ is “the total is $ 8 $.”}

\subsubsection*{Part I}

\textit{Is event $ B $ independent of getting "doubles" (i.e., both dice
resulting in the same number)?}

\subsubsection*{Part II}

\textit{Given that the total is $ 8 $, what is the probability that at least one
of the die rolls resulted in a $ 3 $?}

\section*{Problem 4}

\textit{Consider the communication network shown in the figure below and suppose
that each link can fail with probability $ p $. Assume that failures of
different links are independent.}

\begin{figure}[H]
    \centering
    \includegraphics[scale=1]{"P4"}
    \caption{ Communication Network }
\end{figure}

\subsection*{Part A}

\textit{Assume that $ p = 1/3 $. Find the probability that there exists a path
from $ A $ to $ B $ along which no link has failed.}

\subsection*{Part B}

\textit{Given that exactly one link in the network has failed, find the
probability that there exists a path from $ A $ to $ B $ along which no link has
failed.}

\section*{Problem 5}

\textit{Oscar has lost his dog in either forest $ A $ (with probability 0.4) or
in forest $ B $ (with probability 0.6).}

\textit{If the dog is in forest $ A $ and Oscar spends a day searching for it in
forest $ A $, the conditional probability that he will find the dog that day is
0.25.  Similarly, if the dog is in forest $ B $ and Oscar spends a day looking
for it there, he will find the dog that day with probability 0.15.}

\textit{The dog cannot go from one forest to the other. Oscar can search only in
the daytime, and he can travel from one forest to the other only overnight. The
dog is alive during day 0, when Oscar loses it, and during day 1, when Oscar
starts searching. It is alive during day 2 with probability $ 2/3 $. In general,
for $ n \leq 1 $, if the dog is alive during day $ n - 1 $, then the probability
it is alive during day $ n $ is $ 2/(n + 1) $. The dog can only die overnight.
Oscar stops searching as soon as he finds his dog, either alive or dead.}

\subsection*{Part A}

\textit{ In which forest should Oscar look on the first day of the search to
maximize the probability he finds his dog that day?}

\subsection*{Part B}

\textit{Oscar looked in forest $ A $ on the first day but didn’t find his dog.
What is the probability that the dog is in forest $ A $?}

\subsection*{Part C}

\textit{Oscar flips a fair coin to determine where to look on the first day and
finds the dog on the first day. What is the probability that he looked in forest
$ A $?}

\subsection*{Part D}

\textit{Oscar decides to look in forest $ A $ for the first two days. What is
the probability that he finds his dog alive for the first time on the second
day?}

\subsection*{Part E}

\textit{Oscar decides to look in forest $ A $ for the first two days. Given that
he did not find his dog on the first day, find the probability that he does not
find his dog dead on the second day.}

\subsection*{Part F}

\textit{Oscar finally finds his dog on the fourth day of the search. He looked
in forest $ A $ for the first 3 days and in forest $ B $ on the fourth day.
Given this information, what is the probability that he found his dog alive?}

\section*{Problem 6}

\textit{Before leaving for work, Victor checks the weather report in order to
decide whether to carry an umbrella. On any given day, with probability 0.2 the
forecast is “rain” and with probability 0.8 the forecast is “no rain”. If the
forecast is “rain”, the probability of actually having rain on that day is 0.8.
On the other hand, if the forecast is “no rain”, the probability of actually
raining is 0.1.}

\subsection*{Part A}

\textit{One day, Victor missed the forecast and it rained. What is the
probability that the forecast was “rain”?}

\subsection*{Part B}

\textit{Victor misses the morning forecast with probability 0.2 on any day in
the year. If he misses the forecast, Victor will flip a fair coin to decide
whether to carry an umbrella. (We assume that the result of the coin flip is
independent from the forecast and the weather.) On any day he sees the forecast,
if it says “rain” he will always carry an umbrella, and if it says “no rain” he
will not carry an umbrella. Let $ U $ be the event that “Victor is carrying an
umbrella”, and let $ N $ be the event that the forecast is “no rain”. Are events
$ U $ and $ N $ independent?}

\subsection*{Part D}

\textit{Victor is carrying an umbrella and it is not raining. What is the
probability that he saw the forecast?}

\end{document}

