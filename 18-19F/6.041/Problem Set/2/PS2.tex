\documentclass{article}
\usepackage{tikz}
\usepackage{float}
\usepackage{enumerate}
\usepackage{amsmath}
\usepackage{amsthm}
\usepackage{bm}
\usepackage{indentfirst}
\usepackage{siunitx}
\usepackage[utf8]{inputenc}
\usepackage{graphicx}
\graphicspath{ {Images/} }
\usepackage{float}
\usepackage{mhchem}
\usepackage{chemfig}
\allowdisplaybreaks

\title{6.041 Problem Set 2}
\author{Robert Durfee}
\date{September 18, 2018}

\begin{document}

\maketitle

\section*{Problem 1}

\textit{Consider the events $ A $, $ B $, and $ C $. Are the following
statements always true? Justify your answer. (Assume that all conditioning
events have positive probability.)}

\subsection*{Part A}

\textit{$ P \left( A \cap B \cap \overline{C} \right) = P \left( A \cap B
\right) P \left( \overline{C} \mid A \cap B \right) $}

\bigbreak

From the definition of conditional probability
$$ P ( \overline{C} \mid A \cap B ) = \frac{ P ( A \cap B \cap \overline{C} ) }{
    P ( A \cap B ) } $$
Substituting into the provided expression
$$ P ( A \cap B \cap \overline{C} ) = P ( A \cap B ) \frac{ P ( A \cap B \cap
\overline{C} ) }{ P ( A \cap B ) } $$
Canceling terms
$$ P ( A \cap B \cap \overline{C} ) =  P ( A \cap B \cap \overline{C} ) $$

\subsection*{Part B}

\textit{$ P \left( A \cap B \cap \overline{C} \right) = P \left( A \right) P
\left( \overline{C} \mid A \right) P \left( B \mid A \cap \overline{C} \right)
$}

\bigbreak

From the definition of conditional probability
$$ P ( \overline{C} \mid A ) = \frac{ P ( \overline{C} \cap A ) }{ P ( A ) },\ P
( B \mid A \cap \overline{C} ) = \frac{ P ( A \cap B \cap \overline{C} ) }{ P (
A \cap \overline{C} ) } $$
Substituting into the provided expression
$$ P ( A \cap B \cap \overline{C} ) = P ( A ) \frac{ P ( \overline{C} \cap A )
}{ P ( A ) } \frac{ P ( A \cap B \cap \overline{C} ) }{ P ( A \cap \overline{C}
) } $$
Canceling terms
$$ P ( A \cap B \cap \overline{C} ) =  P ( A \cap B \cap \overline{C} ) $$

\subsection*{Part C}

\textit{$ P \left( A \cap B \cap \overline{C} \right) = P \left( A \right) P
\left( \overline{C} \cap A \mid A \right) P \left( B \mid A \cap \overline{C}
\right) $}

\bigbreak

From the definition of conditional probability
$$ P ( \overline{C} \cap A \mid A ) = \frac{ P ( \overline{C} \cap A \cap A ) }{
    P ( A ) },\ P ( B \mid A \cap \overline{C} ) = \frac{ P ( B \cap A \cap
\overline{C} ) }{ P ( A \cap \overline{C} ) } $$
Substituting into the provided expression
$$ P ( A \cap B \cap \overline{C} ) = P ( A ) \frac{ P ( \overline{C} \cap A )
}{ P ( A ) } \frac{ P ( B \cap A \cap \overline{C} ) }{ P ( A \cap \overline{C}
) } $$
Canceling terms
$$ P ( A \cap B \cap \overline{C} ) =  P ( A \cap B \cap \overline{C} ) $$

\subsection*{Part D}

\textit{$ P \left( A \cap B \mid C \right) = P \left( A \mid C \right) P \left(
B \mid A \cap C \right) $}

\bigbreak

From the definition of conditional probability
$$ P ( A \cap B \mid C ) = \frac{ P ( A \cap B \cap C ) }{ P ( C ) },\ P ( A
\mid C ) = \frac{ P ( A \cap C ) }{ P ( C ) } $$
$$ P ( B \mid A \cap C ) = \frac{ P ( A \cap B \cap C ) }{ P ( A \cap C ) } $$
Substituting into provided expression
$$ \frac{ P ( A \cap B \cap C ) }{ P ( C ) } = \frac{ P ( A \cap C ) }{ P ( C )
} \frac{ P ( A \cap B \cap C ) }{ P ( A \cap C ) } $$
Canceling terms
$$ P ( A \cap B \cap C ) =  P ( A \cap B \cap C ) $$

\section*{Problem 2}

\textit{Suppose that events $ A $ and $ B $ are conditionally independent given
event $ C $. Suppose that $ P \left( C \right) > 0 $ and $ P \left( \overline{C}
\right) > 0 $.}

\subsection*{Part A}

\textit{Are $ A $ and $ \overline{B} $ guaranteed to be conditionally
independent given $ C $? Justify your answer.}

\bigbreak

If $ P ( B \cap C ) > 0 $, then $ P (A \cap B \mid C) = P (A \mid C) P(B \mid C)
$ is the same as $ P(A \mid B \cap C) = P(A \mid C) $. Furthermore, since
conditioning on $ B $ has no effect, the conditioning on $ \overline{B} $ must
also have no effect. Therefore, $ P(A \mid \overline{B} \cap C) = P(A \mid C) $
which is the same as $ P(\overline{B} \mid A \cap C) = P(\overline{B} \mid C) $.
Therefore $ A $ and $ \overline{B} $ are conditionally independent given $ C $.

\subsection*{Part B}

\textit{Are $ A $ and $ B $ guaranteed to be conditionally independent given $
\overline{C} $? Justify your answer.}

\bigbreak

A counterexample exists when $ A $ and $ B $ only intersect within $ C $.
Therefore, $ P(A \cap B \mid \overline{C}) = 0 $. However, if $ A $ and $ B $
still exist in $ \overline{C} $, $ P(A \mid \overline{C}) $ and $ P(B \mid
\overline{C}) $ are both positive. Thus, $ P(A \cap B \mid \overline{C}) \neq
P(A \mid \overline{C}) P(B \mid \overline{C}) $ and thus are not conditionally
independent given $ \overline{C} $.

\section*{Problem 3}

\textit{You roll two five-sided dice. The sides of each die are numbered from $
1 $ to $ 5 $. The dice are “fair” (all sides are equally likely), and the two
die rolls are independent.}

\subsection*{Part A}

\textit{Event $ A $ is “the total is $ 10 $” (i.e., the sum of the results of
the two die rolls is $ 10 $).}

\subsubsection*{Part I}

\textit{Is event $ A $ independent of the event “at least one of the die rolls
resulted in a $ 5 $”?}

\bigbreak

Let $ I $ be the event "at least one of the die rolls resulted in a $ 5 $."
Given probabilities
$$ P(A \cap I) = 1/25,\ P(A) = 1/25,\ P(I) = 9/25 $$
Checking independence
$$ P(A \cap I) \neq P(A) P(I) $$
Therefore they are not independent.

\subsubsection*{Part II}

\textit{Is event $ A $ independent of the event “at least one of the die rolls
resulted in a $ 1 $”?}

\bigbreak

Let $ II $ be the event “at least one of the die rolls resulted in a $ 1 $”.
Given probabilities
$$ P(A \cap II) = 0,\ P(A) = 1/25,\ P(II) = 9/25 $$
Checking independence
$$ P(A \cap II) \neq P(A) P(II) $$
Therefore they are not independent.

\subsection*{Part B}

\textit{Event $ B $ is “the total is $ 8 $.”}

\subsubsection*{Part I}

\textit{Is event $ B $ independent of getting "doubles" (i.e., both dice
resulting in the same number)?}

\bigbreak

Let $ I $ be the event of getting doubles. Given probabilities
$$ P(B \cap I) = 1/25,\ P(B) = 3/25,\ P(I) = 5/25 $$
Checking independence
$$ P(B \cap I) \neq P(B) P(I) $$
Therefore they are not independent.

\subsubsection*{Part II}

\textit{Given that the total is $ 8 $, what is the probability that at least one
of the die rolls resulted in a $ 3 $?}

\bigbreak

Let $ II $ be the event that at least one of the die rolls resulted in a $ 3 $.
Given probabilities
$$ P(II \cap B) = 2/25,\ P(B) = 3/25 $$
Using the definition of conditional probability
$$ P(II \mid B) = \frac{P(II \cap B)}{P(B)} = 2/3 $$

\section*{Problem 4}

\textit{Consider the communication network shown in the figure below and suppose
that each link can fail with probability $ p $. Assume that failures of
different links are independent.}

\begin{figure}[H]
    \centering
    \includegraphics[scale=1]{"P4"}
    \caption{ Communication Network }
\end{figure}

\subsection*{Part A}

\textit{Assume that $ p = 1/3 $. Find the probability that there exists a path
from $ A $ to $ B $ along which no link has failed.}

\bigbreak

Let $ 1 $, $ 2 $, ... be the events that Link 1, Link 2, ... fail respectively.
A path exists from $ A $ to $ B $ if $ (\overline{1} \cap \overline{2} \cap
\overline{5}) \cup (\overline{3} \cap \overline{4} \cap \overline{5}) $. Using
the inclusion-exclusion principle, the probability of this event is
$$ P((\overline{1} \cap \overline{2} \cap \overline{5}) \cup (\overline{3} \cap
\overline{4} \cap \overline{5})) = P(\overline{1} \cap \overline{2} \cap
\overline{5}) + P(\overline{3} \cap \overline{4} \cap \overline{5}) -
P(\overline{1} \cap \overline{2} \cap \overline{3} \cap \overline{4} \cap
\overline{5}) $$
Since each link failing is independent of all the others, this equivalent
$$ P(\overline{1})P(\overline{2})P(\overline{5}) +
P(\overline{3})P(\overline{4})P(\overline{5}) -
P(\overline{1})P(\overline{2})P(\overline{3})P(\overline{4})P(\overline{5}) $$
Therefore
$$ P((\overline{1} \cap \overline{2} \cap \overline{5}) \cup (\overline{3} \cap
\overline{4} \cap \overline{5})) = 2(1 - p)^3 - (1 - p)^5 \approx 0.46 $$

\subsection*{Part B}

\textit{Given that exactly one link in the network has failed, find the
probability that there exists a path from $ A $ to $ B $ along which no link has
failed.}

\bigbreak

The event that exactly one link fails is given by
$$ (1 \cap \overline{2} \cap \overline{3} \cap \overline{4} \cap \overline{5})
\cup (\overline{1} \cap 2 \cap \overline{3} \cap \overline{4} \cap \overline{5})
\cup \ldots $$
Given that the link failures are independent and the outcomes of the event are
mutually exclusive, the probability becomes
$$ P(1)P(\overline{2})P(\overline{3})P(\overline{4})P(\overline{5}) +
P(\overline{1})P(2)P(\overline{3})P(\overline{4})P(\overline{5}) + \ldots = 5
p(1 - p)^4 $$
The only outcome in this event that would be leave no path from $ A $ to $ B $
is the failure of Link 5, which has a probability of $ p(1 - p)^4 $. Therefore,
the probability that there is a path from $ A $ to $ B $ given that exactly one
link failed is
$$ \frac{ 4 p (1 - p)^4 }{ 5 p (1 - p)^4 } = 4/5 $$

\section*{Problem 5}

\textit{Oscar has lost his dog in either forest $ A $ (with probability 0.4) or
in forest $ B $ (with probability 0.6).}

\textit{If the dog is in forest $ A $ and Oscar spends a day searching for it in
forest $ A $, the conditional probability that he will find the dog that day is
0.25.  Similarly, if the dog is in forest $ B $ and Oscar spends a day looking
for it there, he will find the dog that day with probability 0.15.}

\textit{The dog cannot go from one forest to the other. Oscar can search only in
the daytime, and he can travel from one forest to the other only overnight. The
dog is alive during day 0, when Oscar loses it, and during day 1, when Oscar
starts searching. It is alive during day 2 with probability $ 2/3 $. In general,
for $ n \leq 1 $, if the dog is alive during day $ n - 1 $, then the probability
it is alive during day $ n $ is $ 2/(n + 1) $. The dog can only die overnight.
Oscar stops searching as soon as he finds his dog, either alive or dead.}

\bigbreak

Let $ L $ be the random variable representing the location of the dog. Let $ F_i
$ be the event that the dog is found on the $ i $th day. Let $ S_i $ be the
random variable representing the forest Oscar searches on the $ i $th day. Let $
H_i $ be the event that the dog is healthy on the $ i $th day. Then the given
probabilities are
$$ P(L = A) = 0.4, P(L = B) = 0.6 $$
$$ P(F_i \mid L = A \cap S_i = A) = 0.25,\ P(F_i \mid L = B \cap S_i = B) = 0.15 $$
$$ P(F_i \mid L = A \cap S_i = B) = 0,\ P(F_i \mid L = B \cap S_i = A) = 0 $$
$$ P(H_i \mid H_{i - 1} \cap H_{i - 2} \cap \ldots \cap H_{0}) = 2 / (i + 1) $$

\subsection*{Part A}

\textit{ In which forest should Oscar look on the first day of the search to
maximize the probability he finds his dog that day?}

\bigbreak

To figure out the best strategy, compare the following
$$ P(F_1 \mid S_1 = A),\ P(F_1 \mid S_1 = B) $$

First for $ P(F_1 \mid S_1 = A) $. Using the law of total probability, this is
equivalent
$$ P(L = A) P(F_1 \mid S_1 = A \cap L = A) + P(L = B) P(F_1 \mid S_1 = A \cap L
= B) $$
Using the probabilities described above,
$$ P(F_1 \mid S_1 = A) = (0.4)(0.25) + (0.6)(0.0) = 0.10 $$

Now for $ P(F_1 \mid S_1 = B) $. Using the law of total probability, this is
equivalent
$$ P(L = A) P(F_1 \mid S_1 = B \cap L = A) + P(L = B) P(F_1 \mid S_1 = B \cap L
= B) $$
Using the probabilities described above,
$$ P(F_1 \mid S_1 = A) = (0.4)(0.0) + (0.6)(0.15) = 0.09 $$
Therefore Oscar should search forest $ A $.

\subsection*{Part B}

\textit{Oscar looked in forest $ A $ on the first day but didn’t find his dog.
What is the probability that the dog is in forest $ A $?}

\bigbreak

The probability asked for is $ P(L = A \mid S_1 = A \cap \overline{F_1}) $.
Using Bayes' Rule, this can be written as
$$ \frac{ P(\overline{F_1} \mid L = A \cap S_1 = A) P(L = A \cap S_1 = A) }{
    P(S_1 = A \cap \overline{F_1}) } $$
Since it is reasonable to assume that the forest Oscar searches is independent
of where the dog is (otherwise he would know where the dog was to begin with),
This can be simplified
$$ \frac{ (1 - P(F_1 \mid L = A \cap S_1 = A)) P(L = A) }{ 1 - P(F_1 \mid S_1 =
A) } $$
Everything is know except $ P(\overline{F_1} \mid S_1 = A) $. Using the law of
total probability, this is the same as
$$ P(L = A) (1 - P(F_1 \mid S_1 = A \cap L = A)) + P(L = B) (1 - P(F_1 \mid S_1
= A \cap L = B)) $$
Using the probabilities defined above leads to the simplification
$$ P(\overline{F_1} \mid S_1 = A) = P(L = A) (1 - P(F_1 \mid S_1 = A \cap L =
A)) + P(L = B) $$
Substituting back into the equation determined above
$$ P(L = A \mid S_1 = A \cap \overline{F_1}) = \frac{ (1 - P(F_1 \mid L =
A \cap S_1 = A)) P(L = A) }{ P(L = A) (1 - P(F_1 \mid S_1 = A \cap L = A)) + P(L
= B) } $$
Using the probabilities above,
$$ P(L = A \mid S_1 = A \cap \overline{F_1}) = \frac{ (1 - 0.25) (0.40) }{
    (0.40) (1 - 0.25) + 0.60 } = 1/3 $$

\subsection*{Part C}

\textit{Oscar flips a fair coin to determine where to look on the first day and
finds the dog on the first day. What is the probability that he looked in forest
$ A $?}

\bigbreak

The probability being asked for is $ P(S_1 = A \mid F_1) $. Using Bayes' Rule,
this can be rewritten as
$$ \frac{ P(F_1 \mid S_1 = A) P(S_1 = A) }{ P(F_1) } $$
The probability $ P(F_1 \mid S_1 = A) $ can be determined using the law of total
probability as follows
$$ P(F_1 \mid S_1 = A) = P(L = A) P(F_1 \mid S_1 = A \cap L = A) + P(L = B)
P(F_1 \mid S_1 = A \cap L = B) $$
Using the probabilities above, this simplifies
$$ P(F_1 \mid S_1 = A) = P(L = A) P(F_1 \mid S_1 = A \cap L = A) $$
Furthermore, the probability $ P(F_1) $ can be found also using the law of total
probability,
$$ P(F_1) = P(S_1 = A) P(F_1 \mid S_1 = A) + P(S_1 = B) P(F_1 | S_1 = B) $$
Using the above derived expression reduces this to known probabilities
\begin{align*}
    P(F_1) = &P(S_1 = A) P(L = A) P(F_1 \mid S_1 = A \cap L = A) \\
    &+ P(S_1 = B) P(L = B) P(F_1 \mid S_1 = B \cap L = B)
\end{align*}
Substituting this back into the desired expression and using probabilities above
$$ P(S_1 = A \mid F_1) = \frac{ (0.4) (0.25) (0.5) }{ (0.5)(0.4)(0.25) +
(0.5)(0.6)(0.15) } \approx 0.53 $$

\subsection*{Part D}

\textit{Oscar decides to look in forest $ A $ for the first two days. What is
the probability that he finds his dog alive for the first time on the second
day?}

\bigbreak

The probability being asked for is $ P ( \overline{F_1} \cap F_2 \cap H_2 \mid
S_1 = S_2 = A) $. If knowing $ S_1 = S_2 = A $ occurs, also knowing $ F_1 $ will
have no impact on $ F_2 $ and vice versa. The same can be said for the health of
the dog on each day and relative to $ F_i $. Therefore, there is conditional
independence and the probability can be rewritten as
\begin{align*}
    P(\overline{F_1} \mid S_1 = S_2 = A) P(F_2 \mid S_1 = S_2 = A) P(H_1 \mid
    S_1 = S_2 = A) \\
    P(H_2 \mid S_1 = S_2 = A)
\end{align*}
It is also reasonable to assume that the dog's health is independent of where
Oscars is looking. Thus
$$ P(\overline{F_1} \mid S_1 = S_2 = A) P(F_2 \mid S_1 = S_2 = A) P(H_1) P(H_2)
$$
Furthermore, where Oscar looks on day 1 will not impact whether he will find the
dog on day 2 and vice versa. Therefore
$$ P(\overline{F_1} \mid S_1 = A) P(F_2 \mid S_2 = A) P(H_1) P(H_2) $$
Using the expression for $ P(F_1 \mid S_1 = A) $ from above yields
$$ (1 - P(L = A)P(F_1 \mid S_1 = A \cap L = A)) P(L = A) P(F_2 \mid S_2 = A \cap
L = A) P(H_1) P(H_2) $$
Substituting the probabilities given above
$$ P ( \overline{F_1} \cap F_2 \cap H_2 \mid S_1 = S_2 = A) = (1 -
(0.4)(0.25))(0.4)(0.25)(1)(0.66) = 0.06 $$

\subsection*{Part E}

\textit{Oscar decides to look in forest $ A $ for the first two days. Given that
he did not find his dog on the first day, find the probability that he does not
find his dog dead on the second day.}

\subsection*{Part F}

\textit{Oscar finally finds his dog on the fourth day of the search. He looked
in forest $ A $ for the first 3 days and in forest $ B $ on the fourth day.
Given this information, what is the probability that he found his dog alive?}

\section*{Problem 6}

\textit{Before leaving for work, Victor checks the weather report in order to
decide whether to carry an umbrella. On any given day, with probability 0.2 the
forecast is “rain” and with probability 0.8 the forecast is “no rain”. If the
forecast is “rain”, the probability of actually having rain on that day is 0.8.
On the other hand, if the forecast is “no rain”, the probability of actually
raining is 0.1.}

\bigbreak

Let $ F $ be the event that the forecast predicts rain. Let $ R $ be the event
that it actually rains. Let $ U $ be the event that Victor is carrying an
umbrella. The given probabilities are
$$ P(F) = 0.2,\ P(\overline{F}) = 0.8 $$
$$ P(R \mid F) = 0.8,\ P(R \mid \overline{F}) = 0.1 $$

\subsection*{Part A}

\textit{One day, Victor missed the forecast and it rained. What is the
probability that the forecast was “rain”?}

\bigbreak

The probability being asked for is $ P(F \mid R) $. Using Bayes' Rule, this is
the same as
$$ \frac{P(R \mid F) P(F)}{P(F) P(R \mid F) + P(\overline{F}) P(R \mid
\overline{F}) } $$
Using the given probabilities above
$$ P(F \mid R) = \frac{(0.8) (0.2)}{ (0.2) (0.8) + (0.8) (0.1) } = 2/3 $$

\subsection*{Part B}

\textit{Victor misses the morning forecast with probability 0.2 on any day in
the year. If he misses the forecast, Victor will flip a fair coin to decide
whether to carry an umbrella. (We assume that the result of the coin flip is
independent from the forecast and the weather.) On any day he sees the forecast,
if it says “rain” he will always carry an umbrella, and if it says “no rain” he
will not carry an umbrella. Let $ U $ be the event that “Victor is carrying an
umbrella”, and let $ N $ be the event that the forecast is “no rain”. Are events
$ U $ and $ N $ independent?}

\bigbreak

Let $ S $ be the event that Victor sees the forecast. The given probabilities
$$ P(S) = 0.8,\ P(\overline{S}) = 0.2 $$
$$ P(U \mid \overline{S}) = 0.5 $$
$$ P(U \mid S \cap F) = 1,\ P(U \mid S \cap \overline{F}) = 0 $$

To assess whether $ \overline{F} $ and $ U $ are independent, the following must
be shown
$$ P(\overline{F} \cap U) = P(\overline{F}) P(U) $$

Starting with $ P(U) $. Using the total law of probability, this is the same as
$$ P(U) = P(S) P(U \mid S) + P(\overline{S}) P(U \mid \overline{S}) $$
Everything is known except $ P(U \mid S) $. This can be rewritten using the law
of total probability
$$ P(U \mid S) = P(F) P(U \mid S \cap F) + P(\overline{F}) P(U \mid S \cap
\overline{F}) $$
Using the probabilities given, this can be simplified
$$ P(U \mid S) = P(F) $$
Therefore, putting it all together,
$$ P(U) = P(S) P(F) + P(\overline{S}) P(U \mid \overline{S}) $$
Using the probabilities above,
$$ P(U) = (0.8) (0.2) + (0.2) (0.5) = 0.26 $$

Next for $ P(\overline{F}) $. This is given as $ P(\overline{F}) = 0.8 $.

Lastly, solving for $ P(\overline{F} \cap U) $. Using the multiplication rule
$$ P(\overline{F} \cap U) = P(\overline{F}) P(U \mid \overline{F}) $$
The only unknown is $ P(U \mid \overline{F}) $. Using the law of total
probability,
$$ P(U \mid \overline{F}) = P(S) P(U \mid \overline{F} \cap S) + P(\overline{S})
P(U \mid \overline{F} \cap \overline{S}) $$
Substituting into the above desired expression,
$$ P(\overline{F} \cap U) = P(\overline{F}) (P(S) P(U \mid \overline{F} \cap S)
+ P(\overline{S}) P(U \mid \overline{F} \cap \overline{S})) $$
Using the probabilities above,
$$ P(\overline{F} \cap U) = (0.8) ( (0.8) (0.0) + (0.2) (0.5) ) = 0.08 $$

Since $ (0.8) (0.26) \neq 0.08 $, $ U $ and $ N $ are not independent.

\subsection*{Part D}

\textit{Victor is carrying an umbrella and it is not raining. What is the
probability that he saw the forecast?}

\end{document}

