\documentclass{article}
\usepackage{tikz}
\usepackage{float}
\usepackage{enumerate}
\usepackage{amsmath}
\usepackage{amsthm}
\usepackage{bm}
\usepackage{indentfirst}
\usepackage{siunitx}
\usepackage[utf8]{inputenc}
\usepackage{graphicx}
\graphicspath{ {Images/} }
\usepackage{float}
\usepackage{mhchem}
\usepackage{chemfig}
\allowdisplaybreaks

\title{6.041 Problem Set 12}
\author{Robert Durfee - R02}
\date{November 29, 2018}

\begin{document}

\maketitle

\section*{Problem 1}

\subsection*{Part A}

Consider two disjoint intervals $(0, t]$ and $(t, t + s]$. Let $N$ be the
number of arrivals in the time interval $(0, t]$ and $L$ be the number
of arrivals in the interval $(t, t + s]$. Thus, the number of arrivals in $(0,
t + s]$ is $M = N + L$. The PMF of $L$ is given by the Poisson,
$$ p_L(\ell) = P(k = \ell, \tau = s; \lambda) = \frac{(\lambda
s)^\ell}{\ell!} e^{-\lambda s},\quad \ell \in \{0, 1, \ldots\}$$
Now, the conditional PMF of $M$ given $N$ is just the PMF of $L$ offset by
the amount $N$ as $M = N + L$. Thefore, $\ell = m - n$ and the conditional
PMF is,
$$ p_{M|N}(m \mid n) = \frac{(\lambda s)^{m - n}}{(m - n)!} e^{-\lambda
s},\quad n \leq m \in \{0, 1, \ldots\} $$

\subsection*{Part B}

Using the multiplication rule,
$$ P_{MN}(m, n) = p_N(n) p_{M|N}(m \mid n) $$
The PMF for $N$ is determined from the Poisson,
$$ p_N(n) = P(k = n, \tau = t; \lambda) = \frac{(\lambda t)^n}{n!}
e^{-\lambda t},\quad n \in \{0, 1, \ldots\} $$
The PMF for $M$ given $N$ was computed in Part A.
$$ p_{M|N}(m \mid n) = \frac{(\lambda s)^{m - n}}{(m - n)!} e^{-\lambda
s},\quad n \leq m \in \{0, 1, \ldots\} $$
Multiplying these two together yields the joint PMF,
$$ p_{MN}(m, n) = \left(\frac{(\lambda t)^n}{n!} e^{-\lambda t}\right)
\left(\frac{(\lambda s)^{m - n}}{(m - n)!} e^{-\lambda s}\right),\quad n \leq
m \in \{0, 1, \ldots\} $$
Simplified,
$$ p_{MN}(m, n) = \frac{\lambda^m t^n s^{m - n}}{n!(m - n)!} e^{-\lambda(t +
s)},\quad n \leq m \in \{0, 1, \ldots\} $$

\subsection*{Part C}

Using the definition of conditional probability,
$$ p_{N|M}(n \mid m) = \frac{p_{MN}(m, n)}{p_M(m)} $$
The joint PMF was calculated in Part B,
$$ p_{MN}(m, n) = \frac{\lambda^m t^n s^{m - n}}{n!(m - n)!} e^{-\lambda(t +
s)},\quad n \leq m \in \{0, 1, \ldots\} $$
The PMF for $M$ is given by the Poisson,
$$ p_M(m) = P(k = m, \tau = t + s; \lambda) = \frac{\left(\lambda (s +
t)\right)^m}{m!} e^{-\lambda (t + s)},\quad m \in \{0, 1, \ldots\} $$
Putting these two together using the definition above,
$$ p_{N|M}(n \mid m) = \left(\frac{\lambda^m t^n s^{m - n}}{n!(m - n)!}
e^{-\lambda(t + s)}\right) / \left(\frac{\left(\lambda (t + s)\right)^m}{m!}
e^{-\lambda (s + t)}\right) $$
Simplified,
$$ p_{N|M}(n \mid m) = \frac{m!}{n! (m - n)!} \frac{t^m s^{m - n}}{(t +
s)^m},\quad n \leq m \in \{0, 1, \ldots\} $$

\subsection*{Part D}

Consider again the random variable $L = M - N$ from Part A. Therefore, the
expectation can be rewritten as,
$$ E[MN] = E[N (N + L)] $$
Since $L$ and $N$ are independent, this can be split apart.
\begin{align*}
  E[MN] &= E[N (N + L)] \\
  &= E[N^2 + NL] \\
  &= E[N^2] + E[N]E[L] \\
  &= \mathrm{var}(N) + E[N]^2 + E[N]E[L] \\
  &= (\lambda t) + (\lambda t)^2 + (\lambda t)(\lambda s) \\
  &= \lambda t + \lambda^2 t^2 + \lambda^2 s t
\end{align*}

\section*{Problem 2}

\subsection*{Part A}

A bus arrives deterministically every hour. Therefore, the time interval of
interest in the passenger arrival process is one hour. The expected number of
passengers arriving in one hour is given by the Poisson,
$$ E[N_\tau] = \lambda \tau $$
$$ E[N_1] = \lambda $$

\subsection*{Part B}

Yes, the buses depart according to a Poisson process. Interdeparture times
are exponentially distributed by a common parameter $\mu$ and each departure
is independent from the others.

\subsection*{Part C}

An event defined by either a passenger or a bus arriving is given by the
merged passenger-bus Poisson process. The parameter for this process is the
sum of the individual processes' $\lambda' = \lambda + \mu$. The expected
number of event in a single hour is given by this Poisson,
$$ E[N_1] = \lambda' = \lambda + \mu $$

\subsection*{Part D}

Due to the memorylessness property of the Poisson process, the arrivals prior
do not affect the arrivals after. Therefore, the expected time until the next
bus remains the same and this is given by the expected value of an
exponential distribution.
$$ E[T] = \frac{1}{\mu} $$

\subsection*{Part E}

Considering the merged passenger-bus process. An event is caused by a
passenger with probability,
$$ P(\mathrm{Passenger}) = \frac{\lambda}{\mu + \lambda} $$
An event is caused by a bus with probability,
$$ P(\mathrm{Bus}) = \frac{\mu}{\mu + \lambda} $$
Thus, we can consider every passenger arrival as a failure and the bus
arrival as a success. To find the PMT of $N$, the number of people on a bus,
there must be $n$ failures and $1$ success. This is given by,
$$ p_N(n) = \left(\frac{\lambda}{\mu + \lambda}\right)^n \left(\frac{\mu}{\mu
+ \lambda}\right),\quad n \in \{0, 1, \ldots\} $$
Substituting $\lambda = 20$ and $\mu = 2$,
$$ p_N(n) = \frac{2 \cdot 20^n}{22^{n+1}},\quad n \in \{0, 1, \ldots\} $$

\section*{Problem 3}

\subsection*{Part A}

Consider the merged Poisson process of eastbound and westbound ships with
parameter $\lambda_E + \lambda_W$. The past is independent from the future
and therefore the probability that the next ship will be westbound is given
by,
$$ P(\mathrm{Westbound}) = \frac{\lambda_W}{\lambda_E + \lambda_W} $$

\subsection*{Part B}

The change in direction occurs when the next eastbound ship arrives. This is
given by the exponential distribution.
$$ f_X(x) = \lambda_E e^{-\lambda_E x},\quad x \geq 0 $$

\subsection*{Part C}

In order for an eastbound ship not to pass a westbound ship, no westbound
ship can enter $t$ before the eastbound or $t$ after the eastbound, as it
takes $t$ time to cross the canal. As a result, the probability is given by
the Poisson,
$$ P(\mathrm{No\ Westbound}) = P(k = 0, \tau = 2t; \lambda = \lambda_W) =
e^{-2 \lambda_W t} $$

\subsection*{Part D}

The distribution of $V$ is given by the $k$th arrival time of the eastbound
process where $k = 7$,
$$ f_{Y_7}(v) = f_V(v) = \frac{\lambda_E^7 v^6 e^{-\lambda_E v}}{6!},\quad v
\geq 0 $$

\subsection*{Part E}

The change in direction of the pointer occurs in two cases
\begin{itemize}
  \item The previous ship was eastbound and the next ship is westbound.
  \item The previous ship was westbound and the next ship is eastbound.
\end{itemize}
Using the Law of Total Probability,
$$ P(\mathrm{Change}) = P(\mathrm{East})P(\mathrm{West} \mid \mathrm{East}) +
P(\mathrm{West})P(\mathrm{East} \mid \mathrm{West}) $$
However, since this is a Poisson process, the next arrival is independent of
the previous. Therefore this becomes,
$$ P(\mathrm{Change}) = 2 P(\mathrm{East})P(\mathrm{West}) $$
Substituting the respective probabilities from the merged Poisson,
$$ P(\mathrm{Change}) = \frac{2 \lambda_E \lambda_W}{(\lambda_E +
\lambda_W)^2} $$

\subsection*{Part F}

The number of ships seen up until the seventh eastbound is given by the
binomial distribution where eastbounds are successes and westbounds are
failures. We want $7$ successes and $k - 7$ failures. However, the last
ship's arrival time is fixed. So, we want the number of ways to choose
size-$6$ subgroups of a total of $k - 1$ ships. From this, the PMF is
$$ p_K(k) = \binom{k - 1}{6} \left(\frac{\lambda_E}{\lambda_E +
\lambda_W}\right)^7 \left(\frac{\lambda_W}{\lambda_E + \lambda_W}\right)^{k -
7},\quad k \in \{7, 8, \ldots\} $$

\section*{Problem 4}

\subsection*{Part A}

The length $R$ is given by the sum of two exponentials. This is equivalent to
the Erlang distribution of order-$2$. The expectation of an order-$k$ Erlang is
$$ E[Y_k] = \frac{k}{\lambda} $$
Substituting for order-$2$,
$$ E[R] = \frac{2}{\lambda} $$

\subsection*{Part B}

The length $R$ is now given by the sum of three exponentials as $f_X(x)$ is
an order-$3$ Erlang. The middle segment's expected length is given by the
solution to Part A, $2 / \lambda$. Then, add two more exponentials on either
side, each with expected length $1 / \lambda$ as determined from order-$1$
Erlangs. Therefore, the total expected length is,
$$ E[R] = \frac{4}{\lambda} $$

\end{document}