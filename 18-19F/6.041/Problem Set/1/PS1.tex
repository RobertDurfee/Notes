\documentclass{article}
\usepackage{tikz}
\usepackage{float}
\usepackage{enumerate}
\usepackage{amsmath}
\usepackage{amsthm}
\usepackage{bm}
\usepackage{indentfirst}
\usepackage{siunitx}
\usepackage[utf8]{inputenc}
\usepackage{graphicx}
\graphicspath{ {Images/} }
\usepackage{float}
\usepackage{mhchem}
\usepackage{chemfig}
\allowdisplaybreaks

\title{6.041 Problem Set 1}
\author{Robert Durfee}
\date{September 11, 2018}

\begin{document}

\maketitle

\section*{Problem 1}

\textit{Let $ A $, $ B $, and $ C $ be some events. For each one of the events
below, give a set-theoretic expression and also show the event pictorially on a
Venn diagram.}

\subsection*{Part A}

\textit{At least two of the events $ A $, $ B $, $ C $ occur.}

\begin{figure}[H]
    \centering
    \includegraphics[scale=0.5]{"P1A"}
    \caption{ $ (A \cap B) \cup (B \cap C) \cup (C \cap A) $ }
\end{figure}

\subsection*{Part B}

\textit{At most two of the events $ A $, $ B $, $ C $ occur.}

\begin{figure}[H]
    \centering
    \includegraphics[scale=0.5]{"P1B"}
    \caption{ $ \overline{(A \cap B \cap C)} $ }
\end{figure}

\subsection*{Part C}

\textit{None of the events $ A $, $ B $, $ C $ occurs.}

\begin{figure}[H]
    \centering
    \includegraphics[scale=0.5]{"P1C"}
    \caption{ $ \overline{(A \cup B \cup C)} $ }
\end{figure}

\subsection*{Part D}

\textit{Either event $ B $ occurs or, if not, then $ C $ also does not occur.}

\begin{figure}[H]
    \centering
    \includegraphics[scale=0.5]{"P1D"}
    \caption{ $ \overline{B} \implies \overline{C} $ }
\end{figure}

\section*{Problem 2}

\textit{Find the value of $ P \left( A \cup \overline{ \left( \overline{B} \cup
\overline{C} \right) } \right) $ for each of the following cases.}

\subsection*{Part A}

\textit{The events $ A $, $ B $, $ C $ are disjoint events and $ P(A) = 2/5 $.}

\bigbreak

Since both $ B $ and $ C $ are disjoint, the union of their complements span the
entire sample space $ \Omega $. The complement of the entire sample space $
\Omega $ is the empty set $ \emptyset $. Lastly, any set's union with the empty
set is that same set. Therefore, using $ P(A) = 2/5 $,
\begin{align*}
    P \left( A \cup \overline{ \left( \overline{B} \cup \overline{C} \right) }
    \right) &= P \left( A \cup \overline{ \Omega } \right) \\
    &= P \left( A \cup \emptyset \right) \\
    &= P(A) \\
    &= 2/5
\end{align*}

\subsection*{Part B}

\textit{The events $ A $ and $ C $ are disjoint, and $ P(A) = 1/2 $, and $ P(B
\cap C) = 1/4 $.}

\bigbreak

From DeMorgan's Law, $ \overline{ \overline{B} \cup \overline{C} } = B \cap C $.
Additionally, given that $ A $ and $ C $ are disjoint, and that the intersect of
$ B $ and $ C $ must be a subset of $ C $, then $ A $ and $ B \cap C $ must also
be disjoint. Thus, the additional rule can be used. Therefore, using $ P(A) =
1/2 $ and $ P(B \cap C) = 1/4 $,
\begin{align*}
    P \left( A \cup \overline{ \left( \overline{B} \cup \overline{C} \right) }
    \right) &= P \left( A \cup \left( B \cap C \right) \right) \\
    &= P(A) + P(B \cap C) \\
    &= 1/2 + 1/4 \\
    &= 3/4
\end{align*}

\subsection*{Part C}

\textit{$ P \left( \overline{A} \cap \left( \overline{B} \cup \overline{C}
\right) \right) = 0.7 $}

\bigbreak

Using DeMorgan's Law, $ \overline{\overline{A} \cap \left( \overline{B} \cup
\overline{C} \right)} = A \cup \overline{\left( \overline{B}
\cup \overline{C} \right)} $. Therefore, given $ P \left( \overline{A} \cap
\left( \overline{B} \cup \overline{C} \right) \right) = 0.7 $,
\begin{align*}
    P \left( A \cup \overline{\left( \overline{B} \cup \overline{C} \right) }
    \right) &= 1 - P \left( \overline{A \cup \overline{\left( \overline{B} \cup
    \overline{C} \right)} } \right) \\
    &= 1 - P \left( \overline{A} \cap \left( \overline{B} \cup \overline{C}
    \right) \right) \\
    &= 1 - 0.7 \\
    &= 0.3
\end{align*}

\section*{Problem 3}

\textit{Mary and Tom park their cars in an empty parking lot with $ n \geq 2 $
consecutive parking spaces (i.e, n spaces in a row, where only one car fits in
each space). Mary and Tom pick parking spaces at random. (All pairs of distinct
parking spaces are equally likely.) What is the probability that there is
exactly one empty parking space between them?}

\bigbreak

With $ n $ parking spaces all in a row, there are $ n - 2 $ distinct ways such
that there is exactly one parking space between the two cars. Furthermore, the
number of distinct ways two cars can be placed in $ n $ parking spaces all in a
row is $ \binom{n}{2} $. Therefore, the probability of the event $ E $ that two
cars have exactly one empty space between them, given every outcome is uniform,
follows,
$$ P(E) = (n - 2) / \binom{n}{2} $$

\section*{Problem 4}

\textit{Alice and Bob each choose at random a real number between zero and one.
We assume that the pair of numbers is chosen according to the uniform
probability law on the unit square, so that the probability of an event is equal
to its area.}

\textit{We define the following events:}

\begin{itemize}
    \item[] $ A $ = \{ \textit{The magnitude of the difference of the two
        numbers is $ > $ 1/3} \}
    \item[] $ B $ = \{ \textit{At least one of the numbers is $ > $ 1/4} \}
    \item[] $ C $ = \{ \textit{The sum of the two numbers is 1} \}
    \item[] $ D $ = \{ \textit{Alice’s number is $ > $ 1/4} \}
\end{itemize}

\textit{Find the following probabilities:}

\subsection*{Part A}

\textit{$ P(A) $}

\bigbreak

The set notation for event $ A $ is,
$$ A = \{ (x, y) \mid \vert x - y \vert > 1/3 \} $$
This can be broken down into
$$ A = \{ (x, y) \mid x - y > 1/3 \lor x - y < -1/3 \} $$
Over the unit square, this creates two equal triangles whose areas are given by
$$ A_T = (1/2) (1 - 1/3)^2 $$
Therefore, the total probability, given by the total area is
$$ P(A) = (1 - 1/3)^2 = 4/9 $$

\subsection*{Part B}

\textit{$ P(B) $}

\bigbreak

The set notation for event $ B $ is,
$$ B = \{ (x, y) \mid \max{(x, y)} > 1/4 \} $$
The area of this region over the unit square is given by
$$ A_R = 1 - (1/4)^2 $$
Therefore, the total probability, given by the area is
$$ P(B) = 1 - (1/4)^2 = 15/16 $$

\subsection*{Part C}

\textit{$ P(A \cap B) $}

\bigbreak

The set notation for event $ A \cap B $ is,
$$ A \cap B = \{ (x, y) \mid \vert x - y \vert > 1/3 \land \max{(x, y)} > 1/4 \}
$$
These two different regions over the unit square overlap exactly where region $
A $ did as calculated above. Therefore, the event $ A \cap B  = A $. So the
total probability, given by the area of region $ A $,
$$ P(A \cap B) = P(A) = 4/9 $$

\subsection*{Part D}

\textit{$ P(C) $}

\bigbreak

The set notation for event $ C $ is,
$$ C = \{ (x, y) \mid x + y = 1 \} $$
Since this is only a line with no area, and the probability is defined by area,
$$ P(C) = 0 $$

\subsection*{Part E}

\textit{$ P(D) $}

\bigbreak

The set notation for event $ D $, taking $ x $ as Alice's number, is,
$$ D = \{ (x, y) \mid x > 1/4 \} $$
This region over the unit square takes up $ 3/4 $ of its area. Therefore, the
total probability is,
$$ P(D) = 3/4 $$

\subsection*{Part F}

\textit{$ P(A \cap D) $}

\bigbreak

The set notation for event $ A \cap D $, taking $ x $ as Alice's number, is,
$$ A \cap D = \{ (x, y) \mid \vert x - y \vert > 1/3 \land x > 1/4 \} $$
These two regions intersect making one triangle of area equal to half of $ P(A)
A $,
$$ A_{T1} = P(A) / 2 $$
And the other given by,
$$ A_{T2} = (1/2)(2/3 - 1/4)^2 $$
Therefore, the total probability, given by area,
$$ P(A \cap D) = (1/2)(4/9) + (1/2)(2/3 - 1/4)^2 \approx 0.309 $$

\end{document}

