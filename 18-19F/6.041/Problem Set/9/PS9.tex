\documentclass{article}
\usepackage{tikz}
\usepackage{float}
\usepackage{enumerate}
\usepackage{amsmath}
\usepackage{amsthm}
\usepackage{bm}
\usepackage{indentfirst}
\usepackage{siunitx}
\usepackage[utf8]{inputenc}
\usepackage{graphicx}
\graphicspath{ {Images/} }
\usepackage{float}
\usepackage{mhchem}
\usepackage{chemfig}
\allowdisplaybreaks

\title{6.041 Problem Set 9}
\author{Robert Durfee - R02}
\date{November 7, 2018}

\begin{document}

\maketitle

\section*{Problem 1}

From the Bernoulli distribution with $k$ trials,
$$ E[X \mid P = p] = k p $$
Treating $p$ as a random variable,
$$ E[X \mid P] = k P $$
Using the law of iterated expectations,
$$ E[X] = E[E[X \mid P]] = k E[P] $$
From the uniform distribution in the range $[0, 1]$,
$$ E[X] = k / 2 $$

From the Bernoulli distribution with $k$ trials,
$$ \mathrm{var}(X \mid P = p) = k p (1 - p) $$
Treating $p$ as a random variable,
$$ \mathrm{var}(X \mid P) = k P (1 - P) $$
Using the law of total variance
\begin{align*}
  \mathrm{var}(X) &= E[\mathrm{var}(X \mid P)] + \mathrm{var}(E[X \mid P]) \\
  &= E[k P(1 - P)] + \mathrm{var}(k P) \\
  &= k(E[P] - E[P^2]) + k^2 \mathrm{var}(P)
\end{align*}
Using the definition of variance,
\begin{align*}
  \mathrm{var}(X) &= k (E[P] - \mathrm{var}(P) - (E[P])^2) + k^2 \mathrm{var}(P) \\
  &= k / 6 - k^2 / 12
\end{align*}

\section*{Problem 2}

Let $X = X_1 + X_2 + \ldots + X_k$ where $k$ is given by the random variable
$K$ with a geometric distribution as defined by the probability of success $p
= 1/2$. Then, the mean and variance are given by
$$ E[K] = (1 - p) / p = 1 $$
$$ \mathrm{var}(K) = (1 - p) / p^2 = 2 $$
Each individual $X_i$ is given by a uniform distribution on the interval $[0,
3]$. Therefore, the mean and variance are given by
$$ E[X_i] = 3 / 2 $$
$$ \mathrm{var}(X_i) = 3 / 4 $$

Using the law of iterated expectation,
\begin{align*}
  E[X] &= E[E[X \mid K]] \\
  &= E[K E[X_i]] \\
  &= (3 / 2) E[K] \\
  &= 3 / 2
\end{align*}

Using the law of total variance,
\begin{align*}
  \mathrm{var}(X) &= E[\mathrm{var}(X \mid K)] + \mathrm{var}(E[X \mid K]) \\
  &= E[K \mathrm{var}(X_i)] + \mathrm{var}(K E[X_i]) \\
  &= (3 / 4) E[K] + (3 / 2)^2 \mathrm{var}(K) \\
  &= (3 / 4) + (9 / 4)(2) \\
  &= 21 / 4
\end{align*}

\section*{Problem 3}

\subsection*{Part A}

Let $C = 1$ correspond to choosing coin 1 and $C = 2$ to choosing coin 2. Let
$K$ be the random variable corresponding to the number of heads flipped in
$3$ trials. Using Bayes' rule,
$$ P(C = 1 \mid K = k) = \frac{P(C = 1) P(K = k \mid C = 1)}{P(K = k)} $$
The probability of coin 1 being selected is $p$, therefore,
$$ P(C = 1) = p $$
$K$ for coin 1 is governed by a Bernoulli distribution with trials $3$ and
probability of success $1/3$, therefore
$$ P(K = k \mid C = 1) = \binom{3}{k} \left(\frac{1}{3}\right)^k
\left(\frac{2}{3}\right)^{3 - k} $$
Using the law of total probability,
$$ P(K = k) = P(C = 1) P(K = k \mid C = 1) + P(C = 2) P(K = k \mid C = 2) $$
$K$ for coin 2 is also governed by a Bernoulli distribution with trials $3$
and probability of success $2/3$, therefore
$$ P(K = k) = p \binom{3}{k} \left(\frac{1}{3}\right)^k
\left(\frac{2}{3}\right)^{3 - k} + (1 - p) \binom{3}{k}
\left(\frac{2}{3}\right)^k \left(\frac{1}{3}\right)^{3 - k} $$
Putting this all together,
\begin{align*}
  P(C = 1 \mid K = k) &= \frac{p \binom{3}{k} \left(\frac{1}{3}\right)^k
  \left(\frac{2}{3}\right)^{3 - k}}{p \binom{3}{k} \left(\frac{1}{3}\right)^k
  \left(\frac{2}{3}\right)^{3 - k} + (1 - p) \binom{3}{k}
  \left(\frac{2}{3}\right)^k \left(\frac{1}{3}\right)^{3 - k}} \\
  &= \frac{2^{3 - k} p }{2^{3 - k} p + 2^k(1 - p)}
\end{align*}

\subsection*{Part B}

Using the MAP rule, Bob should use the following,
$$ P(C = 1 \mid K = k) \geq P(C = 2 | K = k) $$
Using the Bayes' rule,
\begin{align*}
  P(K = k \mid C = 1) &\geq P(K = k \mid C = 2) \\
  \frac{2^{3 - k} p }{2^{3 - k} p + 2^k(1 - p)} &\geq \frac{2^{k} (1 - p)
  }{2^{3 - k} p + 2^k(1 - p)} \\
  2^{3 - k} p &\geq 2^{k} (1 - p) \\
  \frac{1}{2} \log_2\left(\frac{p}{1 - p}\right) + \frac{3}{2} &\geq k
\end{align*}

\subsection*{Part C}

\subsubsection*{Part I}

The probability that Bob guesses correctly is given by the law of total
probability,
$$ P(\mathrm{Correct}) = P(C = 1) P(\mathrm{Correct} \mid C = 1) + P(C = 2)
P(\mathrm{Correct} \mid C = 2) $$
Using the decision rule from above with $p = 2/3$, Bob guesses coin 1 for $k
\leq 2$ and coin 2 for $k = 3$. Therefore,
\begin{align*}
  P(\mathrm{Correct}) &= p \cdot P(K \leq 2 \mid C = 1) + (1 - p) \cdot P(K =
  3 \mid C = 2) \\
  &= (2/3) (1 - (1/3)^3) + (1/3) (2/3)^3 \\
  &= 20 / 27
\end{align*}

\subsubsection*{Part II}

If Bob does not gain any additional evidence, aside from the fact that $p =
2/3$, he will minimize the error by always choosing coin 1 as it is more
likely to be picked than coin 2. Calculating correctness as before,
\begin{align*}
  P(\mathrm{Correct}) &= P(C = 1) P(\mathrm{Correct} \mid C = 1) + P(C = 2)
  P(\mathrm{Correct} \mid C = 2) \\
  &= 1 \cdot (2/3) + 0 \cdot (1/3) \\
  &= 2/3
\end{align*}

\subsection*{Part D}

If the value from Part B is less than zero, Bob will never choose the first
coin. Therefore, this is true for $p$ given by
\begin{align*}
  \frac{1}{2} \log_2\left(\frac{p}{1 - p}\right) + \frac{3}{2} &< 0 \\
  \log_2\left(\frac{p}{1 - p}\right) &< -3 \\
  \frac{p}{1 - p} &< \frac{1}{8} \\
  p &< \frac{1}{9}
\end{align*}

\section*{Problem 4}

\subsection*{Part A}

Under the MAP rule, hypothesis $\Theta = 1$ will be chosen if
$$ P(\Theta = 1 \mid X = x) \geq P(\Theta = 0 \mid X = x) $$
Using Bayes' Rule,
\begin{align*}
  P(\Theta = 1 \mid X = x) &\geq P(\Theta = 0 \mid X = x) \\
  \frac{P(\Theta = 1) P(X = x \mid \Theta = 1)}{P(X = x)} &\geq
  \frac{P(\Theta = 0) P(X = x \mid \Theta = 0)}{P(X = x)} \\
  P(\Theta = 1) P(X = x \mid \Theta = 1) &\geq P(\Theta = 0) P(X = x \mid
  \Theta = 0)
\end{align*}
The hypotheses $\Theta = 0$ and $\Theta = 1$ have probabilities $1 - p$ and
$p$, respectively. The following distributions determine the conditional PDFs
for $X$,
$$ f_{X|\Theta}(x \mid 0) = \begin{cases}
  1 & x \in [0, 1] \\
  0 & \mathrm{otherwise}
\end{cases},\, f_{X|\Theta}(x \mid 1) = \begin{cases}
  2x & x \in [0, 1] \\
  0 & \mathrm{otherwise}
\end{cases} $$
From this, the MAP rule becomes,
\begin{align*}
  p (2x) &\geq (1 - p) (1) \\
  x &\geq \frac{1 - p}{2p}
\end{align*}
If $p = 3/5$, then this becomes $1/3$.

\subsection*{Part B}

If $x = 1$, then the MAP rule always chooses hypothesis $\Theta = 0$. When $x
= 1$, $p < 1/3$

\subsection*{Part C}

The probability of error given the hypothesis is $\Theta = 0$ is given by 
$$ P(\mathrm{Wrong} \mid \Theta = 0) $$
Using the decision rule from above, the wrong answer is given when
$$ P(\mathrm{Wrong} \mid \Theta = 0) = P\left(x > \frac{1 - p}{p} \mid \Theta
= 0\right) $$
This is the same as
$$ P(\mathrm{Wrong} \mid \Theta = 0) = 1 - P\left(x \leq \frac{1 - p}{p} \mid
\Theta = 0\right) $$
By using the conditional CDF for $X$,
\begin{align*}
  P(\mathrm{Wrong} \mid \Theta = 0) &= 1 - F_{X|\Theta}\left(\frac{1 - p}{p}
  \mid 0 \right) \\
  &= 1 - \frac{1 - p}{p}
\end{align*}
Substituting $p = 3/5$, the probability of error is $2/3$

\subsection*{Part D}

The overall error is given by the law of total probability,
$$ P(\mathrm{Wrong}) = P(\mathrm{Wrong} \mid \Theta = 0) P(\Theta = 0) +
P(\mathrm{Wrong} \mid \Theta = 1) P(\Theta = 1) $$
Using the decision rule from above, the wrong guess is given when,
$$ P(\mathrm{Wrong}) = P\left(x > \frac{1 - p}{2p} \mid \Theta = 0\right)(1 -
p) + P\left(x \leq \frac{1 - p}{2p} \mid \Theta = 1\right) p $$
Using conditional CDFs for $X$,
\begin{align*}
  P(\mathrm{Wrong}) &= (1 - p) \cdot \left(1 - F_{X|\Theta}\left(\frac{1 -
  p}{2p} \mid 0\right)\right) + p \cdot F_{X|\Theta}\left(\frac{1 - p}{2p}
  \mid 1\right) \\
  &= (1 - p) \cdot \left(1 - \frac{1 - p}{2p}\right) + p \cdot \left(\frac{1
  - p}{2p}\right)^2
\end{align*}

\section*{Problem 6}

Let $\Sigma = 1$ correspond to $\sigma = 1$ and $\Sigma = 4$ correspond to
$\sigma = 4$. Using the MAP rule, hypothesis $\Sigma = 1$ is chosen if,
$$ P(\Sigma = 1 \mid t_1, t_2, \ldots, t_n) \geq P(\Sigma = 4 \mid t_1, t_2,
\ldots, t_n) $$
Using Bayes' Rule,
\begin{align*}
  \frac{P(\Sigma = 1) P(t_1, t_2, \ldots, t_n \mid \Sigma = 1)}{P(t_1, t_2,
  \ldots, t_n)} &\geq \frac{P(\Sigma = 4) P(t_1, t_2, \ldots, t_n \mid \Sigma
  = 4)}{P(t_1, t_2, \ldots, t_n)} \\
  P(\Sigma = 1) P(t_1, t_2, \ldots, t_n \mid \Sigma = 1) &\geq P(\Sigma = 4)
  P(t_1, t_2, \ldots, t_n \mid \Sigma = 4)
\end{align*}
Given that each hypothesis is initially equally likely, this reduces to
$$ P(t_1, t_2, \ldots, t_n \mid \Sigma = 1) \geq P(t_1, t_2, \ldots, t_n \mid
\Sigma = 4) $$
Since each observation is independent and normally distributed, this becomes,
\begin{align*}
  \prod_{i=1}^n \frac{1}{\sqrt{2 \pi}} \exp\left(-\frac{t_i^2}{2}\right)
  &\geq \prod_{i=1}^n \frac{1}{2 \sqrt{2 \pi}}
  \exp\left(-\frac{t_i^2}{8}\right) \\
  \sum_{i=1}^n \left(\ln\left(\frac{1}{\sqrt{2 \pi}}\right) -
  \frac{t_i^2}{2}\right) &\geq \sum_{i=1}^n \left(\ln\left(\frac{1}{2 \sqrt{2
  \pi}}\right) - \frac{t_i^2}{8}\right) \\
  n \ln \left(\frac{1}{\sqrt{2 \pi}}\right) - \frac{1}{2}\sum_{i=1}^n t_i^2
  &\geq n \ln \left(\frac{1}{2\sqrt{2\pi}}\right) - \frac{1}{8}\sum_{i=1}^n t_i^2 \\
  n \ln 2 &\geq \frac{3}{8} \sum_{i=1}^n t_i^2 \\
  1 &\geq \frac{3}{8 n \ln 2} \sum_{i=1}^n t_i^2
\end{align*}
Therefore, the coefficients are
$$ c_1 = \frac{3}{8 n \ln 2},\, c_2 = 0 $$


\end{document}