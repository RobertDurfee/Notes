\documentclass{article}
\usepackage{tikz}
\usepackage{float}
\usepackage{enumerate}
\usepackage{amsmath}
\usepackage{amsthm}
\usepackage{bm}
\usepackage{indentfirst}
\usepackage{siunitx}
\usepackage[utf8]{inputenc}
\usepackage{graphicx}
\graphicspath{ {Images/} }
\usepackage{float}
\usepackage{mhchem}
\usepackage{chemfig}
\allowdisplaybreaks

\title{6.041 Problem Set 4}
\author{Robert Durfee - R02}
\date{October 2, 2018}

\begin{document}

\maketitle

\section*{Problem 1}

\textit{Let X, Y and Z be discrete random variables. For each of the
statements that follow, determine whether it is true (meaning, always true)
or false (meaning, not always true), and provide a very brief explanation
(not a complete proof). Assume that all conditioning events have non-zero
probability.}

\subsection*{Part A}

\textit{$ p_{X,Y,Z}(x, y, z) = p_Y(y) p_{Z|Y}(z \mid y) p_{X|Y,Z}(x \mid y, z) $}

\bigbreak

This is simply the multiplication rule written in different notation. For
example,
$$ P(A \cap B \cap C) = P(B) P(C \mid B) P(A \mid B \cap C) $$
Therefore, this is always true.

\subsection*{Part B}

\textit{$ p_{X,Y|Z}(x, y \mid z) = p_X(x) p_{Y|Z} (y \mid z) $}

\bigbreak

Using probability of events notation and the definition of conditional
probability, the left side can be rewritten
$$ P(A \cap B \mid C) = \frac{P(A \cap B \cap C)}{P(C)} $$
And the right hand side
$$ P(A) P(B \mid C) = P(A) \frac{P(B \cap C)}{P(C)} $$
Equating both sides yields
$$ P(A \cap B \cap C) = P(A) P(B \cap C) $$
As a result, this is not necessarily always true.

\subsection*{Part C}

\textit{$ p_{X,Y|Z}(x, y \mid z) = p_{X|Z} (x \mid z) p_{Y|X,Z}(y \mid x, z) $}

\bigbreak

Using the probability of events notation and the definition of conditional
probability, the left side can be rewritten
$$ P(A \cap B \mid C) = \frac{P(A \cap B \cap C)}{P(C)} $$
And the right side
$$ P(A \mid C) P(B \mid A \cap C) = \frac{P(A \cap C)}{P(C)} \frac{P(B \cap A
\cap C)}{P(A \cap C)} $$
Equating both sides
$$ P(A \cap B \cap C) = P(B \cap A \cap C) $$
Therefore, this is always true.

\subsection*{Part D}

\textit{$ \sum\limits_x p_{X,Y|Z}(x, y \mid z) = 1 $}

\bigbreak

This is not always true summing over a single $ y $ may not represent the
entire universe $ z $. Take, for example, if $ y $ never occurred, then
the probability would be $ 0 $.

\subsection*{Part E}

\textit{$ \sum\limits_x\sum\limits_y p_{X,Y|Z}(x, y \mid z) = 1 $}

\bigbreak

Yes, assuming that all $ x $ and $ y $ are disjoint, as is the definition
of a meaningful random variable, and all $ x $ and $ y $ are summed over,
this will cover the entire universe $ z $ and therefore must equal 1.

\subsection*{Part F}

\textit{$ p_{X,Y|Z}(x, y \mid z) = \frac{p_{X,Y,Z}(x, y, z)}{p_Z(z)} $}

\bigbreak

Using the probability of events notation and the definition of conditional
probability, the left side can be rewritten
$$ P(A \cap B \mid C) = \frac{P(A \cap B \cap C)}{P(C)} $$
Therefore this is always true, it is simply different notation.

\subsection*{Part G}

\textit{$ p_{X|Y,Z}(x \mid y, z) = \frac{p_{X,Y,Z}(x, y, z)}{p_{Y,Z}(y, z)} $}

\bigbreak

Using the probability of events notation and the definition of conditional
probability, the left side can be rewritten
$$ P(A \mid B \cap C) = \frac{P(A \cap B \cap C)}{P(B \cap C)} $$
Therefore this is always true, it is simply different notation.

\section*{Problem 2}

\textit{The random variables X and Y have the joint PMF}
$$ p_{X,Y}(x, y) = \begin{cases}
    c \cdot (x + y)^2 & \mathrm{if}\, x \in \{ 1, 2, 4 \}\, \mathrm{and}\, y
    \in \{ 1, 3 \} \\
    0 & \mathrm{otherwise}
\end{cases} $$

\subsection*{Part A}

\textit{Find the value of the constant c.}

\bigbreak

The following sum spans the entire possible values for $ X $ and $ Y $
$$ \sum\limits_{x \in \{ 1, 2, 4 \}} \sum\limits_{y \in \{ 1, 3 \}} c \cdot
(x + y)^2 = c \cdot \sum\limits_{x \in \{ 1, 2, 4 \}} \sum\limits_{y \in \{
1, 3 \}} (x + y)^2  = c \cdot 125 = 1$$
Therefore, $ c = 1/125 $.

\subsection*{Part B}

\textit{Find $ P(Y < X) $.}

\bigbreak

This is true for $ (x, y) = (2, 1),\, (4, 1),\, (4, 3) $. Summing over the
PMF for these values yields
$$ P(Y < X) = 83/125 \approx 0.664 $$

\subsection*{Part C}

\textit{Find $ P(Y = X) $.}

\bigbreak

This is true for $ (x, y) = (1, 1) $. Summing over the PMF for this value yields
$$ P(Y = X) = 1/125 \approx 0.008 $$

\subsection*{Part D}

\textit{Find the PMF of X.}

$$ p_X(x) = \begin{cases}
    17/125 & x = 1 \\
    34/125 & x = 2 \\
    74/125 & x = 3
\end{cases} $$

\subsection*{Part E}

\textit{Find the expectations $ E[X] $ and $ E[XY] $.}

$$ E[X] = 17/125 \cdot 1 + 34/125 \cdot 2 + 74/125 \cdot 4 = 381/125 \approx
3.048 $$
\begin{align*}
    E[XY] &= 1/125 \cdot 1 + 9/125 \cdot 2 + 25/125 \cdot 4 + 16/125 \cdot 3
    + 25/125 \cdot 6 + 49/125 \cdot 12 \\
    &= 181/25 \approx 7.240
\end{align*}

\subsection*{Part F}

\textit{Find the variance of X.}

\bigbreak

Using the definition of variance
$$ \sigma_X^2 = E[X^2] - E[X]^2 = E[X^2] - (381/125)^2 $$
Calculating $ E[X^2] $
$$ E[X^2] = 17/125 \cdot 1^2 + 34/125 \cdot 2^2 + 74/125 \cdot 4^2 = 1337/125 $$
Plugging into variance
$$ \sigma_X^2 = 1337/125 - (381/125)^2 = 21964/15625 \approx 1.406 $$

\section*{Problem 3}

\textit{Consider a sequence of independent tosses of a biased coin at times $
k = 0, 1, 2, \ldots , n $. On each toss, the probability of Heads is p, and
the probability of Tails is $ 1 - p $.}

\textit{A reward of one unit is given at time k, for $ k \in \{ 1, 2, \ldots
, n \} $, if the toss at time k resulted in Tails and the toss at time $ k -
1 $ resulted in Heads. Otherwise, no reward is given at time k.}

\textit{Let R be the sum of the rewards collected at times $ 1, 2, \ldots, n
$.}

\subsection*{Part A}

\textit{Find E[R]. Express your answer in terms of p and/or n.}

\bigbreak

Let $ R_i $ be the reward at time $ i $ for $ i \in \{ 1, 2, \ldots, n \} $.
Then,
$$ p_{R_i}(r_i) = \begin{cases}
    p(1 - p) & r_i = 1 \\
    1 - p(1 - p) & r_i = 0
\end{cases} $$
Which, trivially, has an expected value of $ E[R_i] = p(1 - p) $.

The expected value of $ R $ is given by
$$ E[R] = E[R_1 + R_2 + \ldots + R_n] $$
By the linearity of expectation
$$ E[R] = E[R_1] + E[R_2] + \ldots + E[R_n] $$
Since each $ R_i $ are equal,
$$ E[R] = (n - 1) E[R_i] = (n - 1) p (1 - p) $$

\subsection*{Part B}

\textit{Find Var(R). Express your answer in terms of p and/or n.}

\bigbreak

From the definition of variance,
$$ \sigma_R^2 = E[R^2] - E[R]^2 $$
From above, this becomes
$$ \sigma_R^2 = E[R^2] - (n - 1) p (1 - p) $$
Let $ R_i $ be the reward at time $ i $ for $ i \in \{ 1, 2, \ldots, n \} $.
The term $ E[R^2] $ is given by
$$ E[R^2] = n E[R_i^2] + 2(n - 1) E[R_i R_{i + 1}] + (n^2 - 3n + 2) E[R_i
R_{i + \ell}] $$
Since the only values of $ R_i $ are either $ 1 $ or $ 0 $, $ E[R_i^2] =
p^2(1 - p)^2 $. For $ R_i R_{i + 1} $, the value must be $ 0 $ as there can
never be two consecutive rewards. Therefore $ E[R_i R_{i + 1}] = 0 $. Lastly,
flips separated by more than one are once again independent. Therefore, $
E[R_i R_{i + \ell}] = p^2(1 - p)^2 $. Variance must be
$$ \sigma_R^2 = (n^2 - 3n + 3) p^2 (1 - p)^2 $$

\section*{Problem 4}

\textit{ For each of the statements that follow, determine whether it is true
(meaning, always true) or false (meaning, not always true), and provide a
very brief explanation (not a complete proof). Here, we assume all random
variables are discrete, and that all expectations are well-defined and
finite. }

\subsection*{Part A}

\textit{Let X and Y be two binomial random variables.}

\begin{enumerate}[i.]
    \item \textit{If X and Y are independent, then $ X + Y $ is also a
    binomial random variable.}
    \item \textit{If X and Y have the same parameters, n and p, then $ X + Y
    $ is a binomial random variable.}
    \item \textit{If X and Y have the same parameter p, and are independent,
    then $ X + Y $ is a binomial random variable.}
\end{enumerate}

\bigbreak

Let $ p_X(x) $ and $ p_Y(y) $ be the PMFs of $ X $ and $ Y $ respectively. If
no assumptions are made about the two, they must be
$$ p_X(k) = \binom{n}{k} p^k (1 - p)^{n - k} $$
$$ p_Y(k) = \binom{m}{k} q^k (1 - q)^{m - k} $$
Then, their sum becomes
$$ p_{X+Y}(k) = \sum\limits_{i = 0}^k p_{X,Y}(i, k - i) $$
If no assumptions are made, this is as far as the expression can reasonably
be simplified. However, if $ X $ and $ Y $ are independent,
$$ p_{X+Y}(k) = \sum\limits_{i = 0}^k p_X(i) p_Y(k - i) $$
Rearranging the terms yields
$$ p_{X+Y}(k) = \binom{m + n}{k} p^i q^{k - i} (1 - p)^{n - i} (1 - q)^{m - k
+ i} $$
Once again, if no more assumptions are made, then this is it and $ X + Y $ is
not binomial. However, if $ p = q $,
$$ p_{X+Y}(k) = \binom{m + n}{k} p^k (1 - p)^{n + m - k} $$
Which is a binomial random variable. These assumptions correspond to the
third statement only.

\subsection*{Part B}

\textit{Suppose that $ E[X] = 0 $. Then, $ X = 0 $ (i.e. $ X(\omega) = 0 $,
for all $ \omega \in \Omega $).}

\bigbreak

This is not always true. Since negative values are perfectly valid for a
random variable $ X $, consider the PMF
$$ p_X(x) = \begin{cases}
    1/2 & x = -1 \\
    1/2 & x = 1
\end{cases} $$
The expected value of this PMF is $ 0 $, yet $ X $ never equals $ 0 $.

\subsection*{Part C}

\textit{Suppose that $ E[X^2] = 0 $. Then, $ P(X = 0) = 1 $.}

\bigbreak

Contrary to the previous part, $ X^2 $ is never negative. Therefore, in order
for the expected value to be exactly zero, the possible values for $ X $ are
limited to only $ 0 $. Thus this is always true.

\end{document}