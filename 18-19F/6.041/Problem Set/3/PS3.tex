\documentclass{article}
\usepackage{tikz}
\usepackage{float}
\usepackage{enumerate}
\usepackage{amsmath}
\usepackage{amsthm}
\usepackage{bm}
\usepackage{indentfirst}
\usepackage{siunitx}
\usepackage[utf8]{inputenc}
\usepackage{graphicx}
\graphicspath{ {Images/} }
\usepackage{float}
\usepackage{mhchem}
\usepackage{chemfig}
\allowdisplaybreaks

\title{6.041 Problem Set 3}
\author{Robert Durfee - R02}
\date{September 25, 2018}

\begin{document}

\maketitle

\section*{Problem 1}

\textit{You are given the set of letters $ \{ A, B, C, D, E \} $. What is the
probability that in a random five-letter string (in which each letter appears
exactly once, and with all such strings equally likely) the letters $ A $, $
B $ and $ D $ are next to each other (in no particular order)?}

\bigbreak

Since, $ A $, $ B $, and $ D $ must stay together, when assembling a sequence,
you have to choose from $ \{ A, B, D \}^* $, $ C $, or $ E $. Let's first
fix the the first element to be $ \{ A, B, D \}^* $:
$$ ( \{ A, B, D \}^*, \star, \star ) $$
The size of sequences of this form is given by
$$ | \{ A, B, D \}^* | \cdot 2 \cdot 1 = (3 \cdot 2 \cdot 1) \cdot 2 \cdot 1
= 12 $$
Now, since we fixed the location of $ \{ A, B, D \}^* $, we must look at how
many other ways there are to position $ \{ A, B, C \}^* $. There are $
\binom{3}{1} = 3 $ different ways to do this. Therefore, the total number of
sequences of this form must be
$$ ((3 \cdot 2 \cdot 1) \cdot 2 \cdot 1) \cdot 3 = 36 $$

Now the total number of sequences $ \{ A, B, C, D, E \}^* $ is given by
$$ 5 \cdot 4 \cdot 3 \cdot 2 \cdot 1 = 5! = 120 $$
Therefore, the probability of getting this sequence, given all are equally
likely, must be
$$ \frac{36}{120} = \frac{3}{10} $$

\section*{Problem 2}

\textit{Alice plays the following game with Bob. First, Alice randomly
chooses a set of 5 cards out of a 52-card deck, memorizes them, and places
them back into the deck. (Any set of 5 cards is equally likely.) Then, Bob
randomly chooses 9 cards out of the same deck. (Any set of 9 cards is equally
likely.) Assume that the choice of 5 cards by Alice and the choice of 9 cards
by Bob are independent.}

\textit{What is the probability that all 5 cards Alice chose were also among
the 9 cards chosen by Bob?}

\bigbreak

The five cards that Alice draws does nothing other than restrict the cards that
Bob can draw. The values of Alice's cards is not important.

First, let's fix the locations of Alice's cards to be the first cards in the
sequence of cards drawn by Bob:
$$ ( \{ A_0, A_1, A_2, A_3, A_4 \}^*, \star, \star, \star, \star ) $$
The number of sequences in this form must then be:
$$ | \{ A_0, A_1, A_2, A_3, A_4 \}^* | \cdot \frac{(52 - 5)!}{(52 - 9)!} = 5!
\cdot \frac{47!}{43!} $$
However, since we fixed the location of $ \{ A_0, A_1, A_2, A_3, A_4 \}^* $,
we must look at how many other ways there are to fix $ \{ A_0, A_1, A_2, A_3,
A_4 \}^* $. There are $ \binom{9}{5} $ ways to do this. Therefore, the total
number of sequences of this form (where Alice's cards were drawn by Bob),
must be
$$ 5! \cdot \frac{47!}{43!} \cdot \binom{9}{5} = 5! \cdot \frac{47!}{43!}
\cdot \frac{9!}{4! \cdot 5!} = \frac{47!}{43!} \cdot \frac{9!}{4!} $$

Now, the total number of sequences of cards Bob could have drawn is given
$$ \frac{52!}{43!} $$
Therefore, the probability of getting all of Alices cards, given card sequences
are equally likely and Alice's draw is independent of Bob's, must be
$$ \frac{(47! \cdot 9!) / (43! \cdot 4!)}{52! / 43!} = \frac{ 47! \cdot 9! /
4! }{ 52! } = \frac{3}{61880} \approx 4.8 \cdot 10^{-5} $$

\section*{Problem 3}

\textit{Each one of $ n $ persons, indexed by $ 1, 2, \ldots, n $, has a
clean hat and throws it into a box. The persons then pick hats from the box,
at random. Every assignment of the hats to the persons is equally likely. In
an equivalent model, each person picks a hat, one at a time, in the order of
their index, with each one of the remaining hats being equally likely to be
picked. Find the probability of the following events.}

\subsection*{Part A}

\textit{Every person gets his or her own hat back.}

\subsection*{Part B}

\textit{Each one of persons $ 1, \ldots, m $ gets his or her own hat back,
where $ 1 \leq m \leq n $.}

\subsection*{Part C}

\textit{Each one of persons $ 1, \ldots, m $ gets back a hat belonging to one
of the last $ m $ persons (persons $ n - m + 1, \ldots, n $), where $ 1 \leq 
m \leq n $.}

\textit{Now assume, in addition, that every hat thrown into the box has
probability $ p $ of getting dirty (independently of what happens to the
other hats or who has dropped or picked it up). Find the probability that:}

\subsection*{Part D}

\textit{Persons $ 1, \ldots, m $ will pick up clean hats.}

\subsection*{Part E}

\textit{Exactly $ m $ persons will pick up clean hats.}

\section*{Problem 4}

\textit{A player is randomly dealt a sequence of 13 cards from a standard
52-card deck. All sequences of 13 cards are equally likely. In an equivalent
model, the cards are chosen and dealt one at a time. When choosing a card,
the dealer is equally likely to pick any of the cards that remain in the
deck.}

\subsection*{Part A}

\textit{What is the probability the 13th card dealt is a King?}

\subsection*{Part B}

\textit{Find the probability of the event that the 13th card dealt is the
first King dealt.}

\section*{Problem 5}

\textit{The newest invention of the 6.041A staff is a three-sided die. On any
roll of this die, the result is 1 with probability 1/2, 2 with probability
1/4, and 3 with probability 1/4. Consider a sequence of six independent rolls
of this die.}

\subsection*{Part A}

\textit{Find the probability that exactly two of the rolls result in a 3.}

\subsection*{Part B}

\textit{Given that exactly two of the six rolls resulted in a 1, find the
probability that the first roll resulted in a 1.}

\subsection*{Part C}

\textit{We are told that exactly three of the rolls resulted in a 1 and
exactly three rolls resulted in a 2. Given this information, find the
probability that the six rolls resulted in the sequence $ (1, 2, 1, 2, 1, 2)
$.}

\subsection*{Part D}

\textit{Find the conditional probability that exactly $ k $ rolls resulted in
a 3, given that at least one roll resulted in a 3.}

\section*{Problem 6}

\textit{We saw in Recitation 5 that the expected value of a random variable $
X $ that takes nonnegative integer values only, can be expressed by the
following formula:}
$$ E [X] = \sum\limits_{k=1}^{\infty} P(X \geq k) $$
\textit{Use this result to find the expectation of a random variable $ Y $
that takes positive integer values and whose $ PMF $ is:}
$$ p_Y(y) = (1 - \alpha) \alpha^{y - 1},\, y = 1, 2, \ldots $$
\textit{where $ 0 < \alpha < 1 $.}

\end{document}