\documentclass{article}
\usepackage{tikz}
\usepackage{float}
\usepackage{enumerate}
\usepackage{amsmath}
\usepackage{amsthm}
\usepackage{bm}
\usepackage{indentfirst}
\usepackage{siunitx}
\usepackage[utf8]{inputenc}
\usepackage{graphicx}
\graphicspath{ {Images/} }
\usepackage{float}
\usepackage{mhchem}
\usepackage{chemfig}
\allowdisplaybreaks

\title{18.02 Problem Set 8}
\author{Robert Durfee}
\date{April 27, 2018}

\begin{document}

\maketitle

\section*{Problem 1}

\subsection*{Part A}

Adding the square individual components of $\vec{r}$ together,
$$ \left(\sqrt{1 - z^2} \cos \theta \right)^2 + \left(\sqrt{1 - z^2} \sin
\theta\right)^2 + z^2 = 1 $$
This is equivalent to,
$$ x^2 + y^2 + z^2 = 1 $$
Therefore, $\vec{r}$ is on the unit sphere.

\subsection*{Part B}

% Insert Figure!

\section*{Problem 2}

\subsection*{Part A}

The partial derivatives of $\vec{r}$,
$$ \frac{\partial \vec{r}}{\partial \theta} = \langle -\sqrt{1 - z^2} \sin
\theta, \sqrt{1 - z^2} \cos \theta, 0 \rangle $$
$$ \frac{\partial \vec{r}}{\partial z} = \left\langle \frac{-z \cos
\theta}{\sqrt{1 - z^2}}, \frac{-z \sin \theta}{\sqrt{1 - z^2}}, 1
\right\rangle $$
Taking the cross product,
$$ \frac{\partial \vec{r}}{\partial \theta} \times \frac{\partial
\vec{r}}{\partial z} = \langle \sqrt{1 - z^2} \cos \theta, \sqrt{1 - z^2} \sin
\theta, z \rangle $$
And the norm of this vector is,
$$ \left\vert \frac{\partial \vec{r}}{\partial \theta} \times \frac{\partial
\vec{r}}{\partial z} \right\vert = 1 $$

\subsection*{Part B}

The limits of integration are the same for both the sphere and the rectangle. If
the difference of the small area on the sphere minus the small area on the
rectangle is zero for all small areas, then the areas are equivalent. Therefore,
$$ \int\limits_0^{2\pi}\int\limits_0^{1} \left( \left\vert \frac{\partial
\vec{r}}{\partial \theta} \times \frac{\partial \vec{r}}{\partial z} \right\vert
- 1\right) dz d\theta $$
And since the norm of the cross product is $1$,
$$ \int\limits_0^{2\pi}\int\limits_0^{1} \left( \left\vert \frac{\partial
\vec{r}}{\partial \theta} \times \frac{\partial \vec{r}}{\partial z} \right\vert
- 1\right) dz d\theta = 0 $$

\subsection*{Part C}

Since the norm of the cross product is $1$ the equation for surface area on the
sphere is simply,
$$ \int\limits_0^{2\pi}\int\limits_{0.1}^{0.2} dz d\theta = 0.628 $$

\section*{Problem 3}

Computing the distance between to points defined by vectors is just taking the
norm of the difference between them,
$$ \vert \vec{r}(0, 0.5) - \vec{r}(0.01, 0.5) \vert = 0.00866 $$
$$ \vert \vec{r}(0, 0.5) - \vec{r}(0,0.51) \vert = 0.01159 $$
The distances are not the same. The second distance, $P'$ to $R'$, is greater
than $P'$ to $Q'$

\section*{Problem 4}

To take the gradient, rearrange the equation into a function of three variables,
$$ f(x,y,z) = x^2 +z^2 - y $$
Then the gradient is just,
$$ \vec{\nabla}f(x, y, z) = \langle 2x, -1, 2z \rangle $$
Substituting values,
$$ \langle 2, -1, 4 \rangle $$

\section*{Problem 5}

A parameterization of this curve could be,
$$ \vec{r}(u, v) = \langle u, u^2 + v^2, v \rangle $$
The partial derivatives of this equation are,
$$ \frac{\partial \vec{r}}{\partial u} = \langle 1, 2u, 0 \rangle $$
$$ \frac{\partial \vec{r}}{\partial v} = \langle 0, 2v, 1 \rangle $$
The cross product of the partial derivatives is,
$$ \frac{\partial \vec{r}}{\partial u} \times \frac{\partial \vec{r}}{\partial
v}  = \langle 2u, -1, 2v \rangle $$
Substituting values,
$$ \langle 2, -1, 4 \rangle $$

\section*{Problem 6}

The cross product from before is,
$$ \frac{\partial \vec{r}}{\partial \theta} \times \frac{\partial
\vec{r}}{\partial z} = \langle \sqrt{1 - z^2} \cos \theta, \sqrt{1 - z^2} \sin
\theta, z \rangle $$
If the point is taken where $z = 1$ and $\theta = 0$, we know the normal vector
at the top of the sphere,
$$ \langle 0, 0, 1 \rangle $$
Since this points upward at the top, the vector points outwards,

\section*{Problem 7}

The equation for a vector field integral is,
$$ \iint_R \vec{V}(\vec{r}(u, v)) \cdot \left(\frac{\partial \vec{r}}{\partial
u}(u, v) \times \frac{\partial \vec{r}}{\partial v}(u, v) \right) dA $$
Substituting the vector field from the problem and the cross product computed
before,
$$ \int\limits_0^{2\pi}\int\limits_{-1}^1 \langle 0, 0, z \rangle \cdot \langle
\sqrt{1 - z^2} \cos \theta, \sqrt{1 - z^2} \sin \theta, z \rangle dz d\theta $$
Taking the dot product, this integral becomes,
$$ \int\limits_0^{2\pi}\int\limits_{-1}^1 z^2 dz d\theta = \frac{4\pi}{3} $$

\section*{Problem 8}

In this case, the normal vector points inward, so we negate the vector field
integral. Substituting vector field and the cross product of the partial
derivatives of the parameterization used above,
$$ -\int\limits_0^1\int\limits_0^1 \langle u, 3(u^2 + v^2), v \rangle \cdot
\langle 2u, -1, 2v \rangle du dv $$
Taking the dot product cancels terms,
$$ \int\limits_0^1\int\limits_0^1 u^2 + v^2 du dv = \frac{u^3}{3} \bigg\vert_0^1
+ \frac{v^3}{3} \bigg\vert_0^1 = \frac{2}{3} $$

\section*{Problem 9}

\subsection*{Part A}

%% Insert Figure

\subsection*{Part B}

The flux through the face $y = B$,
$$ C \int_0^A x dx = \frac{C A^2}{2} $$
And through the flux through the face $y = 0$,
$$ - C\int_0^A x dx = -\frac{C A^2}{2} $$
All other faces contribute no flux because the face and vectors are
perpendicular. Therefore, total flux is zero.

\subsection*{Part C}

The value of zero makes sense because the same amount of fluid is entering that
is leaving. Therefore, the total fluid contained in the box remains constant at
all times.

\section*{Problem 10}

Using the equation for air resistance on a small parallelogram given,
$$ dR = W^2 \cos^3 \phi dA $$
The $W$ is simply the magnitude of the wind (or in this case the speed of the
object within static air).
$$ W = \vert \vec{w} \vert $$
The normal vector of a parallelogram can be determined as the cross product of
the two partial derivatives,
$$ \vec{n} = \frac{\partial \vec{r}}{\partial u} \times \frac{\partial
\vec{r}}{\partial v} $$
The $\cos \phi$ is just the cosine of the angle between the 'wind' vector and
the normal vector of the face of the small parallelogram. This can be determined
by taking the dot product between both unit vectors,
$$ \cos \phi = \frac{\vec{w}}{\vert \vec{w} \vert} \cdot \frac{ \vec{n}
}{\vert\vec{n}\vert} $$
Lastly, $dA$ is given from the magnitude of the cross product,
$$ dA = \vert \vec{n} \vert du dv $$

Combining all this values into the provided equation will given the small air
resistance for the small parallelogram,
$$ dR = \vert \vec{w} \vert^2 \left( \frac{\vec{w}}{\vert \vec{w} \vert} \cdot
\frac{ \vec{n} }{\vert\vec{n}\vert}\right)^3 \vert \vec{n} \vert du dv $$
Substituting values,
$$ dR = \frac{9 du dv}{1 + 4(u^2 + v^2)} $$
Integrating over the disc,
$$ \int_0^{2\pi}\int_0^{1} \frac{9 r dr d\theta}{1 + 4r^2} = \int_0^{2\pi}
\frac{9 \ln (1 + 4r^2)}{8} \bigg\vert_0^1 d\theta = \frac{9 \pi \ln 5}{8} =
11.38 $$

\section*{Problem 11}

\subsection*{Part A}

For this line integral, only the top of the rectangle contributes,
$$ \int_3^0 4 dt = -12 $$

\subsection*{Part B}

For Green's theorem, $P = y^2$ and $Q = 0$. So the integral becomes,
$$ \int\limits_0^3\int\limits_0^2 -2 y dy dx = -12 $$

\section*{Problem 12}

For Green's theorem,
$$ P = y e^{xy} + y $$
$$ Q = x e^{xy} $$
Taking the derivatives of both,
$$ P_y = 1 + e^{xy} + xy e^{ey} $$
$$ Q_x = xy e^{xy} + e^{xy} $$
Then Green's theorem becomes,
$$\iint_R -(1 + e^{xy} + xy e^{ey}) + (xy e^{xy} + e^{xy}) dA $$
Which simplifies substantially to,
$$ \int\limits_0^{2\pi}\int\limits_0^1 -r dr d\theta  = -2\pi $$

\section*{Problem 13}

To make the curve closed, take $c_2$ to be the line going from (1,0) to (-1,0).
Then make use of the fact,
$$ \int_c f(x) ds = \int_{c_1} f(x) ds + \int_{c_2} f(x) ds $$
Therefore, Green's theorem can be used,
$$ -\iint_R dA - \int_{c_2} (y e^{xy} + y)dx + x e^{xy} dy = \int_{c_1} (y
e^{xy} + y)dx + x e^{xy} dy$$
Applying parameterization for $c_2$,
$$ -\int\limits_{\pi}^{2\pi}\int\limits_0^1 r dr d\theta - 0 = -\pi $$

\section*{Problem 14}

\subsection*{Part A}

The integral will be positive because the top and bottom of the rectangle
contribute nothing to the integral because $d\vec{r}$ and the field are
perpendicular. On the right side, $d\vec{r}$ and the vector field point in the
same direction, so the dot product is positive. On the left side, $d\vec{r}$ and
the vector field point in opposite directions, so the dot product is negative.
But the right side has greater magnitude, therefore the integral is positive.

\subsection*{Part B}

For Green's Theorem,
$$ P(x,y) = 0 $$
$$ Q(x,y) = x $$
Therefore, their derivatives are,
$$ P_y(x,y) = 0 $$
$$ Q_x(x,y) = 1 $$
So $P_y$ is zero and $P_x$ is positive.

\subsection*{Part C}

Using Green's Theorem,
$$ \iint_R -P_y + Q_x dA $$
In this case, the integral becomes,
$$ \iint_R Q_x dA $$
Which is positive.

\section*{Problem 15}

Each interior edge of $c_i$, since they are all counterclockwise,
cancel with the adjacent edge of $c_j$. Therefore, the only edges not
canceled make up the entire $c$.

\section*{Problem 16}

A vector field can be constructed from the partial
derivatives provided. Using Green's theorem, if the negative of the $y$ partial
derivative plus the $x$ partial derivative is zero at all points, only then can
the vector field correspond to a gradient of a function.
$$ -2x + y \neq 0\ \forall x,y $$
Since this is not equal to zero for all $x and y$, this doesn't correspond to a
gradient and thus no function can have both these partial derivatives.

\end{document}
