\documentclass{article}
\usepackage{tikz}
\usepackage{float}
\usepackage{enumerate}
\usepackage{amsmath}
\usepackage{bm}
\usepackage{indentfirst}
\usepackage{siunitx}
\usepackage[utf8]{inputenc}
\usepackage{graphicx}
\graphicspath{ {Images/} }
\usepackage{float}
\usepackage{mhchem}
\usepackage{chemfig}
\allowdisplaybreaks

\title{ 18.02 Problem Set 3 }
\author{ Robert Durfee }
\date{ March 9, 2018 }

\begin{document}

\maketitle

\section{ Vectors }

\begin{enumerate}[1.]
  \item Equation for counterclockwise rotation:
    $$ \vec{v}' = \langle v_{1} \cdot \cos \theta - v_{2} \cdot \sin \theta,
    v_{1} \cdot \sin \theta + v_{2} \cdot \cos \theta \rangle $$
    \begin{enumerate}[a.]
      \item Substituting the value of $\vec{v} = \langle 2, 0 \rangle $ into the
        above equation:
        $$ \langle 2 \cdot \cos \theta, 2 \cdot \sin \theta \rangle = \langle
        \sqrt{3}, 1 \rangle$$
      \item Substituting the value of $\vec{v} = \langle 0, 1 \rangle $ into the
        above equation:
        $$ \langle -1 \cdot \sin \theta, 1 \cdot \cos \theta \rangle =
        \left\langle -\frac{ 1 }{ 2 }, \frac{ \sqrt{3} }{ 2 } \right\rangle $$
      \item Summing the two translated vectors together:
        $$\langle \sqrt{3}, 1 \rangle + \left\langle -\frac{ 1 }{ 2 }, \frac{
          \sqrt{3} }{ 2 } \right\rangle = \left\langle \sqrt{3} - \frac{ 1 }{ 2
        }, 1 + \frac{ \sqrt{3} }{ 2 } \right\rangle $$
    \end{enumerate}
  \item The equations $2x + 3y = 2$ and $2x + 3y = 4$ can be rewritten as:
    $$ \langle 2, 3 \rangle \cdot \langle x, y \rangle = 2 $$
    $$ \langle 2, 3 \rangle \cdot \langle x, y \rangle = 4 $$

    The vector $\langle 2, 3 \rangle$ represents a vector perpendicular to the
    line. Let this vector be called $\vec{v}$. Let the perpendicular line
    segment connecting the two lines be $d$.  This line segment is parallel to
    $\vec{v}$ We also know that the two lines are vertically separated by a
    distance of 2/3.  Let the vertical vector of length 2/3 connecting the two
    lines be $\vec{w}$.

    \bigbreak

    From the dot product:
    $$ \vec{v} \cdot \vec{w} = \vert \vec{v} \vert \vert \vec{w} \vert \cos
    \theta $$

    We want the length of the perpendicular line segment, $d$. From right
    triangle trigonometry,
    $$ d = \vert \vec{w} \vert \cos \theta $$

    Solving the dot product for this quantity:
    $$ \frac{ \vec{v} \cdot \vec{w} }{ \vert \vec{v} \vert } = \vert \vec{w}
    \vert \cos \theta$$

    Substituting provided values:
    $$ \frac{ \langle 2, 3 \rangle \cdot \langle 0, 2/3 \rangle }{ \sqrt{2^{2} +
    3^{2}} } = d = 0.5547$$

  \item The normal vector for the plane $x + 2y + 3z = 5$ is $\langle 1, 2, 3
    \rangle$. Let this be $\vec{n}$. We can project the vector for the point (1,
    1, 0), let this vector be $\vec{u}$, onto the normal vector of the plane:
    $$ \frac{ \vec{u} \cdot \vec{n} }{ \vert \vec{n} \vert^{2} }\vec{n} $$

    Now we have one leg and the hypotenuse of the triangle, subtracting the
    vector perpendicular to the plane from the vector $\vec{u}$ will give the
    vector on the plane that represents the point closest to the point, $P$:
    $$ \vec{u} - \frac{ \vec{u} \cdot \vec{n} }{ \vert \vec{n} \vert^{2} }\vec{n} $$

    Substituting values:
    $$ \langle 1, 1, 0 \rangle - \frac{ \langle 1, 1, 0 \rangle \cdot \langle 1,
    2, 3 \rangle }{ \vert \langle 1, 2, 3 \rangle \vert^{2} } \langle 1, 2, 3
    \rangle = \left\langle \frac{ 11 }{ 14 }, \frac{ 4 }{ 7 }, -\frac{ 9 }{ 14 }
      \right\rangle $$
\end{enumerate}

\section{Linear Approximation}

\begin{enumerate}[1.]
  \setcounter{enumi}{3}
  \item For the $x$ component of the robot's finger:
    $$ f(\theta_{1}, \theta_{2}) = 2 \cos \theta_{1} + \cos (\theta_{1} +
    \theta_{2}) $$

    Creating a linear approximation around the point $(\pi / 6, \pi / 3)$:
    $$ L_{f}(\theta_{1}, \theta_{2}) = f(\theta_{1,0}, \theta_{2,0}) +
    f_{\theta_{1}}(\theta_{1,0}, \theta_{2,0})(\theta_{1} - \theta_{1,0}) +
    f_{\theta_{2}}(\theta_{1,0}, \theta_{2,0})(\theta_{2} - \theta_{2,0}) $$

    Partial derivatives:
    $$ f_{\theta_{1}}(\theta_{1}, \theta_{2}) = -2 \sin \theta_{1} - \sin
    (\theta_{1} + \theta_{2}) $$
    $$ f_{\theta_{2}}(\theta_{1}, \theta_{2}) = -\sin (\theta_{1} + \theta_{2}) $$

    Substituting values:
    $$ L_{f}(\theta_{1}, \theta_{2}) = \sqrt{3} - 2(\theta_{1} - \pi / 6) -
    (\theta_{2} - \pi / 3) $$

    For the $y$ component of the robot's finger:
    $$ g(\theta_{1}, \theta_{2}) = 2 \sin \theta_{1} + \sin (\theta_{1} +
    \theta_{2}) $$

    Creating a linear approximation around the point $\left( \pi / 6, \pi / 3
    \right)$:
    $$ L_{g}(\theta_{1}, \theta_{2}) = g(\theta_{1,0}, \theta_{2,0}) +
    g_{\theta_{1}}(\theta_{1,0}, \theta_{2,0})(\theta_{1} - \theta_{1,0}) +
    g_{\theta_{2}}(\theta_{1,0}, \theta_{2,0})(\theta_{2} - \theta_{2,0}) $$

    Partial derivatives:
    $$ g_{\theta_{1}}(\theta_{1}, \theta_{2}) = 2 \cos \theta_{1} + \cos
    (\theta_{1} + \theta_{2}) $$
    $$ g_{\theta_{2}}(\theta_{1}, \theta_{2}) = \cos (\theta_{1} + \theta_{2}) $$

    Substituting values:
    $$ L_{g}(\theta_{1}, \theta_{2}) = 2 + \sqrt{3} (\theta_{1} - \pi / 6)$$

    Solving this system of equations where $L_{f} = \sqrt{3}$ and $L_{f} = 2.01$:
    $$ \theta_{1} = 0.529,\ \theta_{2} = 1.036 $$
\end{enumerate}

\section{Unbounded Minimization}
\begin{enumerate}[1.]
  \setcounter{enumi}{4}
\item Starting with $f(x, y) = x + y + 1/xy$ and finding partial derivatives:
  $$ f_{x}(x, y) = 1 - \frac{ 1 }{ x^{2} y },\ f_{y}(x, y) = 1 - \frac{ 1 }{ x
  y^{2} } $$

  Setting these equal to zero and solving for $x$ and $y$:
  $$ x = 1,\ y = 1 $$

  This is a minimum in the first quadrant as setting $x \geq 10$ will make
  $f(x, y) \geq 10$. Setting $y \geq 10$ with make $f(x, y) \geq 10$. Lastly,
  setting $1/xy \geq 10$ will result in $f(x,y) \geq 10$. These boundaries are
  shown in the figure below. The point (1, 1) is contained within the boundary
  and is less than 10 ($3 < 10$) so it must be a minimum.

  \begin{figure}[H]
    \centering
    \includegraphics[scale=0.70]{"Boundary"}
    \caption{Boundary}
  \end{figure}
\end{enumerate}

\section{Gradients, Directional Derivatives, Lagrange}

\begin{enumerate}[1.]
  \setcounter{enumi}{5}
  \item \begin{enumerate}[a.]
      \item From the function $f(x, y) = xy$, the gradient is:
        $$ \nabla f(x, y) = \langle y, x \rangle $$
      \item Level curves are shown in \textit{Figure 2}.
      \item Computed gradients:
        $$ \nabla f(0,2) = \langle 2, 0 \rangle $$
        $$ \nabla f(1/2, 2) = \langle 2, 1/2 \rangle $$
        $$ \nabla f(1, 2) = \langle 2, 1 \rangle$$

        All vectors are also shown in \textit{Figure 2}.
      \item The vector perpendicular at this point was already computed:
        $$ \nabla f(1, 2) = \langle 2, 1 \rangle $$

        Using a dot product, the tangent vector can be computed:
        $$ \langle 2, 1 \rangle \cdot \vec{v} = 0 $$
        $$ \vec{v} = \langle -1, 2 \rangle $$
      \item The normal vector at this point was already computed:
        $$ \nabla f(1, 2) = \langle 2, 1 \rangle $$

        Dividing by the length will result in the unit vector:
        $$ \hat{v} = \left\langle \frac{ 2 }{ \sqrt{5} }, \frac{ 1 }{ \sqrt{5} }
        \right\rangle $$
    \end{enumerate}

    \begin{figure}[H]
      \centering
      \includegraphics[scale=0.70]{"Gradient"}
      \caption{Gradients}
    \end{figure}
  \item The gradient of $f(x,y) = x^{2} - y^{2}$ can be computed:
    $$ \nabla f(x,y) = \langle 2x, -2y \rangle $$

    To compute the first directional derivative, dot the gradient with $\vec{u}
    = \langle 1/2, \sqrt{3}/2 \rangle$ at the point (2,3):
    $$ D_{\vec{u}}(2,3) = \langle 2 \cdot 2, -2 \cdot 3 \rangle \cdot \langle
    1/2, \sqrt{3}/2 \rangle = 2 - 3 \sqrt{3} $$

    And for the second directional derivative, dot the gradient with $\vec{v} =
    \langle -1/\sqrt{2}, 1/\sqrt{2} \rangle$ at the point (1,1):
    $$ D_{\vec{v}}(1,1) = \langle 2 \cdot 1, -2 \cdot 1 \rangle \cdot \langle
    -1/\sqrt{2}, 1/\sqrt{2} \rangle = -\frac{ 4 }{ \sqrt{2} }$$
  \item The gradient of $h(x, y) = xy$ can be computed:
    $$ \nabla h(x, y) = \langle y, x \rangle $$

    Since Larry starts at (2, 1) and ends at (0, 3), his vector of travel can be
    computed:
    $$ \vec{v} = \langle 0 - 2, 3 - 1 \rangle = \langle -2, 2 \rangle $$

    We can then take the directional derivative of $h(x, y)$ with respect to
    $\vec{v}$:
    $$ D_{\vec{v}}(2, 1) = \langle 1, 2 \rangle \cdot \langle -2, 2 \rangle = 2
    $$

    This is a positive value, therefore Larry must be going uphill.
  \item \begin{enumerate}[a.]
      \item Graph of $C$:

        \begin{figure}[H]
          \centering
          \includegraphics[scale=0.70]{"Hyperbola"}
          \caption{Hyperbola}
        \end{figure}

      \item The point where the normal direction where the gradient is parallel.
        The gradient of $h(x,y) = y^{2} - x^{2}$ is:
        $$ \nabla h(x, y) = \langle -2x, 2y \rangle $$

        Setting the gradient equal to a multiple of the vector $\langle 1, 2 \rangle$:
        $$ \langle 1, 2 \rangle = \lambda \langle -2x, 2y \rangle $$

        Adding additional constraint that the point must be on the level curve
        1:
        $$ y^{2} - x^{2} = 1 $$

        Solving this system of equations for $x$ and $y$:
        $$ x = -\frac{ 1 }{ \sqrt{3} },\ y = \frac{ 2 }{ \sqrt{3} } $$

        The other solution is eliminated because the solution must be within the
        first or second quadrants.
    \end{enumerate}
  \item
  \item
  \item \begin{enumerate}[a.]
      \item
      \item
    \end{enumerate}
  \item
\end{enumerate}

\section{Integral Intuition}

\begin{enumerate}[1.]
  \setcounter{enumi}{13}
  \item \begin{enumerate}[a.]
      \item
      \item
    \end{enumerate}
  \item \begin{enumerate}[a.]
      \item
      \item
      \item
      \item
    \end{enumerate}
\end{enumerate}

\section{Calculating Integrals}

\begin{enumerate}[1.]
  \setcounter{enumi}{15}
  \item
  \item
  \item
\end{enumerate}

\end{document}

