\documentclass{article}
\usepackage{tikz}
\usepackage{float}
\usepackage{enumerate}
\usepackage{amsmath}
\usepackage{bm}
\usepackage{indentfirst}
\usepackage{siunitx}
\usepackage[utf8]{inputenc}
\usepackage{graphicx}
\graphicspath{ {Images/} }
\usepackage{float}
\usepackage{mhchem}
\usepackage{chemfig}
\allowdisplaybreaks

\title{ 18.02 Problem Set 7 }
\author{ Robert Durfee }
\date{ April 20, 2018 }

\begin{document}

\maketitle

\section*{Problem 1}

Let $\vec{v} = \langle v_1, v_2, v_3 \rangle$ and $\vec{w} = \lambda \vec{v} =
\langle \lambda v_1, \lambda v_2, \lambda v_3 \rangle$.

\bigbreak

Computing the cross product of $\vec{v}$ and $\vec{w}$:
$$ \vec{v} \times \vec{w} = 
\begin{vmatrix}
  \hat{i} & \hat{j} & \hat{k} \\
  v_1 & v_2 & v_3 \\
  \lambda v_1 & \lambda v_2 & \lambda v_3 \\
\end{vmatrix} $$

Expanding the determinant:
$$\langle v_2 \lambda v_3 - v_3 \lambda v_2, v_1 \lambda v_3 - v_3 \lambda v_1,
v_1 \lambda v_2 - v_2 \lambda v_1 \rangle = 0 $$

Now let $\vec{v} = \langle v_1, v_2, v_3 \rangle$ and $\vec{w} = \langle w_1,
w_2, w_3 \rangle$. 

\bigbreak

Computing both cross products:
$$ \vec{v} \times \vec{w} = 
\begin{vmatrix}
   \hat{i} & \hat{j} & \hat{k} \\
   v_1 & v_2 & v_3 \\
   w_1 & w_2 & w_3 \\
\end{vmatrix} $$
$$ -\vec{w} \times \vec{v} =
\begin{vmatrix}
   \hat{i} & \hat{j} & \hat{k} \\
   -w_1 & -w_2 & -w_3 \\
   v_1 & v_2 & v_3 \\
\end{vmatrix}$$ 

Expanding both determinants:
$$ \langle v_2 w_3 - v_3 w_2, v_1 w_3 - v_3 w_1, v_1 w_2 - v_2 w_1 \rangle$$
$$ \langle -w_2 v_3 + w_3 v_2, -w_1 v_3 + w_3 v_1, -w_1 v_2 + w_2 v_1 \rangle $$

These two expressions have equal components, therefore
$$ \vec{v} \times \vec{w} = -\vec{w} \times \vec{v} $$

\section*{Problem 2}

Let $\vec{v} = \langle 1, 1, 0 \rangle$ and $\vec{w} = \langle 4, 2, 0 \rangle$. 

\bigbreak

The cross product of these two vectors will yield the area of the parallelogram
defined by those three points. 
$$ \vec{v} \times \vec{w} = 
\begin{vmatrix}
  \hat{i} & \hat{j} & \hat{k} \\
  1 & 1 & 0 \\
  4 & 2 & 0 \\
\end{vmatrix} $$

Expanding the determinant:
$$ \vec{v} \times \vec{w} = \langle 0, 0, -2 \rangle $$

Taking the norm will yield the area:
$$ \vert \langle 0, 0, -2 \rangle \vert = 2 $$

Dividing by $2$ to get the area of one triangle in the parallelogram:
$$ A = 1 $$

\section*{Problem 3}

Let $\vec{v} = \langle -1, 2, 0 \rangle$ and $\vec{w} = \langle -1, 0, 6
\rangle$.

\bigbreak

The cross product of these vectors will yield the area of the parallelogram
defined by those three points:
$$ \vec{v} \times \vec{w} = 
\begin{vmatrix}
  \hat{i} & \hat{j} & \hat{k} \\
  -1 & 2 & 0 \\
  -1 & 0 & 6 \\
\end{vmatrix} $$

Expanding the determinant:
$$ \vec{v} \times \vec{w} = \langle 12, 6, 2 \rangle $$

Taking the norm will yield the area:
$$ \vert \langle 12, 6, 2 \rangle \vert = 2 \sqrt{46} $$

Dividing by $2$ to get the area of one triangle in the parallelogram:
$$ A = \sqrt{46} $$

\section*{Problem 4}

\subsection*{Part A}

Let $\vec{v} = \langle 1, -1, 1 \rangle$ and $\vec{w} = \langle 2, -1, 6
\rangle$.

\bigbreak

The cross product will result in a vector that is perpendicular to both these
vectors and, thus, a normal vector to the plane defined by these points:
$$ \vec{v} \times \vec{w} = 
\begin{vmatrix}
  \hat{i} & \hat{j} & \hat{k} \\
  1 & -1 & 1 \\
  2 & -1 & 6 \\
\end{vmatrix} $$

Expanding the determinant:
$$ \vec{v} \times \vec{w} = \langle -5, -4, 1 \rangle $$

\subsection*{Part B}

The components of the normal vector represent the coefficients of the equation
of the plane $a$, $b$, and $c$;
$$ -5 x -4 y + z = d $$

Substituting for the point (1, 1, 0) to solve for $d$:
$$ d = -9$$

The final equation for the plane is:
$$ -5 x -4 y + z = -9 $$

\subsection*{Part C} 

Testing another point in this equation (2, 0, 1):
$$ -5 \cdot 2 - 4 \cdot 0 + 1 = -9$$

This point also lies on the plane, as expected.

\section*{Problem 5}

Let $\vec{v} = \langle -1, 2, 0 \rangle$ and $\vec{w} = \langle -1, 0, 6
\rangle$.

\bigbreak

From Problem 3, we know the normal vector to the plane is:
$$ \vec{n} = \langle 12, 6, 2 \rangle $$

The components of this vector determine the coefficients of the plane equation:
$$ 12 x + 6 y + 2 z = d $$

Substituting for the point (1, 0, 0) to solve for $d$:
$$ d = 12 $$

The final equation for the plane is:
$$ 12 x + 6 y + 2z = 12 $$

Solving for $z$:
$$ z = 6 - 6 x - 3 y $$

The parameterization of this surface can be:
$$ \vec{r}(u, v) = \langle u, v, 6 - 6u - 3v \rangle $$

Over the region $R$, where $R$ is the triangle on the $u$-$v$ plane with
vertices (0, 0), (1, 0), and (0, 2).

\section*{Problem 6}

The parameterization of the curve $x = 1 + y^2 + z^2$ can be
$$ \vec{r}(u, v) = \langle 1 + u^2 + v^2, u, v \rangle $$

Over the region $R$, where $R$ is the circle on the $u$-$v$ plane centered at
(0, 0) with radius 2. 

\section*{Problem 7}

The parameterization of the entire sphere with $0 \leq \theta \leq 2 \pi$ and $0
\leq \phi \leq \pi$:
$$ \vec{r}(u, v) = \langle \cos \theta \sin \phi, \sin \theta \sin \phi, \cos
\phi \rangle $$

Since $z \geq 1/2$, from the parameterization,
$$ \cos \phi \geq 1/2 $$

This inequality is true over the range $ 0 \leq \phi \leq \pi/3$.

\bigbreak

Since $y \geq x$, from the parameterization,
$$ \sin \theta \sin \phi \geq \cos \theta \sin \phi $$

This inequality is true over the range $ \pi / 4 \leq \theta \leq 5 \pi / 4 $.

\bigbreak

So the new range for the parameterization is
$$ \pi / 4 \leq \theta \leq 5 \pi / 4 $$
$$ 0 \leq \phi \leq \pi / 3 $$

\section*{Problem 8}

The general equation for a sphere in Cartesian coordinates is $x^2 + y^2 + z^2 =
r^2$. This equation can be solved for $z$:
$$ z = \pm \sqrt{r^2 - x^2 - y^2} $$

Since our surface only appears in the region where $z > 0$, this simplifies to 
$$ z = \sqrt{r^2 - x^2 - y^2} $$

Given our radius is 1, the parameterization follows:
$$ \vec{r}(u, v) = \langle u, v, \sqrt{1 - u^2 - v^2} \rangle $$

Since $z \geq 1/2$, 
$$ 1 / 2 \leq \sqrt{1 - u^2 - v^2} $$

Solving for a standard circle:
$$u^2 + v^2 = 1 - \left(1/2)\right)^2 $$

The radius is then $\sqrt{3} / 2$. So the Region $R$ must be part of a circle
centered at (0, 0) with a radius $\sqrt{3} / 2$.

\bigbreak

We also know from the Problem 7 that $\pi / 4 \leq \theta \leq  5 \pi / 4$. This
further defines $R$ as the half-circle between $\pi / 4$ and $5 \pi / 4$ where
$\theta = 0$ at the positive $x$-axis.

\section*{Problem 9}



\end{document}
