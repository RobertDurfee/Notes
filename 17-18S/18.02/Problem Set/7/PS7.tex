\documentclass{article}
\usepackage{tikz}
\usepackage{float}
\usepackage{enumerate}
\usepackage{amsmath}
\usepackage{bm}
\usepackage{indentfirst}
\usepackage{siunitx}
\usepackage[utf8]{inputenc}
\usepackage{graphicx}
\graphicspath{ {Images/} }
\usepackage{float}
\usepackage{mhchem}
\usepackage{chemfig}
\allowdisplaybreaks

\title{ 18.02 Problem Set 7 }
\author{ Robert Durfee }
\date{ April 20, 2018 }

\begin{document}

\maketitle

\section*{Problem 1}

Let $\vec{v} = \langle v_1, v_2, v_3 \rangle$ and $\vec{w} = \lambda \vec{v} =
\langle \lambda v_1, \lambda v_2, \lambda v_3 \rangle$.

\bigbreak

Computing the cross product of $\vec{v}$ and $\vec{w}$:
$$ \vec{v} \times \vec{w} = 
\begin{vmatrix}
  \hat{i} & \hat{j} & \hat{k} \\
  v_1 & v_2 & v_3 \\
  \lambda v_1 & \lambda v_2 & \lambda v_3 \\
\end{vmatrix} $$

Expanding the determinant:
$$\langle v_2 \lambda v_3 - v_3 \lambda v_2, v_1 \lambda v_3 - v_3 \lambda v_1,
v_1 \lambda v_2 - v_2 \lambda v_1 \rangle = 0 $$

Now let $\vec{v} = \langle v_1, v_2, v_3 \rangle$ and $\vec{w} = \langle w_1,
w_2, w_3 \rangle$. 

\bigbreak

Computing both cross products:
$$ \vec{v} \times \vec{w} = 
\begin{vmatrix}
   \hat{i} & \hat{j} & \hat{k} \\
   v_1 & v_2 & v_3 \\
   w_1 & w_2 & w_3 \\
\end{vmatrix} $$
$$ -\vec{w} \times \vec{v} =
\begin{vmatrix}
   \hat{i} & \hat{j} & \hat{k} \\
   -w_1 & -w_2 & -w_3 \\
   v_1 & v_2 & v_3 \\
\end{vmatrix}$$ 

Expanding both determinants:
$$ \langle v_2 w_3 - v_3 w_2, v_1 w_3 - v_3 w_1, v_1 w_2 - v_2 w_1 \rangle$$
$$ \langle -w_2 v_3 + w_3 v_2, -w_1 v_3 + w_3 v_1, -w_1 v_2 + w_2 v_1 \rangle $$

These two expressions have equal components, therefore
$$ \vec{v} \times \vec{w} = -\vec{w} \times \vec{v} $$

\section*{Problem 2}

Let $\vec{v} = \langle 1, 1, 0 \rangle$ and $\vec{w} = \langle 4, 2, 0 \rangle$. 

\bigbreak

The cross product of these two vectors will yield the area of the parallelogram
defined by those three points. 
$$ \vec{v} \times \vec{w} = 
\begin{vmatrix}
  \hat{i} & \hat{j} & \hat{k} \\
  1 & 1 & 0 \\
  4 & 2 & 0 \\
\end{vmatrix} $$

Expanding the determinant:
$$ \vec{v} \times \vec{w} = \langle 0, 0, -2 \rangle $$

Taking the norm will yield the area:
$$ \vert \langle 0, 0, -2 \rangle \vert = 2 $$

Dividing by $2$ to get the area of one triangle in the parallelogram:
$$ A = 1 $$

\section*{Problem 3}

Let $\vec{v} = \langle -1, 2, 0 \rangle$ and $\vec{w} = \langle -1, 0, 6
\rangle$.

\bigbreak

The cross product of these vectors will yield the area of the parallelogram
defined by those three points:
$$ \vec{v} \times \vec{w} = 
\begin{vmatrix}
  \hat{i} & \hat{j} & \hat{k} \\
  -1 & 2 & 0 \\
  -1 & 0 & 6 \\
\end{vmatrix} $$

Expanding the determinant:
$$ \vec{v} \times \vec{w} = \langle 12, 6, 2 \rangle $$

Taking the norm will yield the area:
$$ \vert \langle 12, 6, 2 \rangle \vert = 2 \sqrt{46} $$

Dividing by $2$ to get the area of one triangle in the parallelogram:
$$ A = \sqrt{46} $$

\section*{Problem 4}

\subsection*{Part A}

Let $\vec{v} = \langle 1, -1, 1 \rangle$ and $\vec{w} = \langle 2, -1, 6
\rangle$.

\bigbreak

The cross product will result in a vector that is perpendicular to both these
vectors and, thus, a normal vector to the plane defined by these points:
$$ \vec{v} \times \vec{w} = 
\begin{vmatrix}
  \hat{i} & \hat{j} & \hat{k} \\
  1 & -1 & 1 \\
  2 & -1 & 6 \\
\end{vmatrix} $$

Expanding the determinant:
$$ \vec{v} \times \vec{w} = \langle -5, -4, 1 \rangle $$

\subsection*{Part B}

The components of the normal vector represent the coefficients of the equation
of the plane $a$, $b$, and $c$;
$$ -5 x -4 y + z = d $$

Substituting for the point (1, 1, 0) to solve for $d$:
$$ d = -9$$

The final equation for the plane is:
$$ -5 x -4 y + z = -9 $$

\subsection*{Part C} 

Testing another point in this equation (2, 0, 1):
$$ -5 \cdot 2 - 4 \cdot 0 + 1 = -9$$

This point also lies on the plane, as expected.

\section*{Problem 5}

Let $\vec{v} = \langle -1, 2, 0 \rangle$ and $\vec{w} = \langle -1, 0, 6
\rangle$.

\bigbreak

From Problem 3, we know the normal vector to the plane is:
$$ \vec{n} = \langle 12, 6, 2 \rangle $$

The components of this vector determine the coefficients of the plane equation:
$$ 12 x + 6 y + 2 z = d $$

Substituting for the point (1, 0, 0) to solve for $d$:
$$ d = 12 $$

The final equation for the plane is:
$$ 12 x + 6 y + 2z = 12 $$

Solving for $z$:
$$ z = 6 - 6 x - 3 y $$

The parameterization of this surface can be:
$$ \vec{r}(u, v) = \langle u, v, 6 - 6u - 3v \rangle $$

Over the region $R$, where $R$ is the triangle on the $u$-$v$ plane with
vertices (0, 0), (1, 0), and (0, 2).

\section*{Problem 6}

The parameterization of the curve $x = 1 + y^2 + z^2$ can be
$$ \vec{r}(u, v) = \langle 1 + u^2 + v^2, u, v \rangle $$

Over the region $R$, where $R$ is the circle on the $u$-$v$ plane centered at
(0, 0) with radius 2. 

\section*{Problem 7}

The parameterization of the entire sphere with $0 \leq \theta \leq 2 \pi$ and $0
\leq \phi \leq \pi$:
$$ \vec{r}(u, v) = \langle \cos \theta \sin \phi, \sin \theta \sin \phi, \cos
\phi \rangle $$

Since $z \geq 1/2$, from the parameterization,
$$ \cos \phi \geq 1/2 $$

This inequality is true over the range $ 0 \leq \phi \leq \pi/3$.

\bigbreak

Since $y \geq x$, from the parameterization,
$$ \sin \theta \sin \phi \geq \cos \theta \sin \phi $$

This inequality is true over the range $ \pi / 4 \leq \theta \leq 5 \pi / 4 $.

\bigbreak

So the new range for the parameterization is
$$ \pi / 4 \leq \theta \leq 5 \pi / 4 $$
$$ 0 \leq \phi \leq \pi / 3 $$

\section*{Problem 8}

The general equation for a sphere in Cartesian coordinates is $x^2 + y^2 + z^2 =
r^2$. This equation can be solved for $z$:
$$ z = \pm \sqrt{r^2 - x^2 - y^2} $$

Since our surface only appears in the region where $z > 0$, this simplifies to 
$$ z = \sqrt{r^2 - x^2 - y^2} $$

Given our radius is 1, the parameterization follows:
$$ \vec{r}(u, v) = \langle u, v, \sqrt{1 - u^2 - v^2} \rangle $$

Since $z \geq 1/2$, 
$$ 1 / 2 \leq \sqrt{1 - u^2 - v^2} $$

Solving for a standard circle:
$$u^2 + v^2 = 1 - \left(1/2)\right)^2 $$

The radius is then $\sqrt{3} / 2$. So the Region $R$ must be part of a circle
centered at (0, 0) with a radius $\sqrt{3} / 2$.

\bigbreak

We also know from the Problem 7 that $\pi / 4 \leq \theta \leq  5 \pi / 4$. This
further defines $R$ as the half-circle between $\pi / 4$ and $5 \pi / 4$ where
$\theta = 0$ at the positive $x$-axis.

\section*{Problem 9}

The parameterization of the top hemisphere of the unit circle is:
$$ \vec{r}(\theta, \phi) = \langle \cos \theta \sin \phi, \sin \theta \sin \phi,
\cos \phi \rangle $$

The partial derivatives of this parameterization are:
$$ \frac{\partial \vec{r}}{\partial \theta} = \langle -\sin \theta \sin \phi,
\cos \theta \sin \phi, 0 \rangle $$
$$ \frac{\partial \vec{r}}{\partial \phi} = \langle \cos \theta \cos \phi, \sin
\theta \cos \phi, -\sin \phi \rangle $$

The cross product of these two vectors:
$$ \frac{\partial \vec{r}}{\partial \theta} \times \frac{\partial
\vec{r}}{\partial \phi} =
\begin{vmatrix}
  \hat{i} & \hat{j} & \hat{k} \\
  -\sin \theta \sin \phi & \cos \theta \sin \phi & 0 \\
  \cos \theta \cos \phi & \sin \theta \cos \phi & - \sin \phi \\
\end{vmatrix} $$

Expanding this determinant:
$$ \frac{\partial \vec{r}}{\partial \theta} \times \frac{\partial
\vec{r}}{\partial \phi} = \langle -\cos \theta \sin^2 \phi, -\sin \theta \sin^2
\phi, -\cos \phi \sin \phi \rangle $$

The normal of this vector is:
$$ \sin \phi $$

Computing the integral:
$$ \iint_R z \sin \phi dA$$

Getting $z$ from the parameterization:
$$ z = \cos \phi $$
 
The integral becomes:
$$ \iint_R \cos \phi \sin \phi dA $$

Given the surface is the top hemisphere, $0 \leq \theta \leq \pi$ and $0 \leq
\phi \leq \pi / 2$:
$$ \int\limits_0^{2\pi}\int\limits_0^{\pi/2} \cos \phi \sin \phi d\phi d\theta
$$

Using the double angle identity:
$$ \frac{1}{2} \int\limits_0^{2\pi}\int\limits_0^{\pi/2} \sin(2\phi) d\phi
d\theta $$

This integral evaluates to:
$$ \frac{1}{2} \int\limits_0^{2\pi} -\frac{1}{2} \cos(2\phi) \bigg\vert_0^{\pi/2} d\theta
= \frac{1}{2} \int\limits_0^{2\pi} d\theta = \frac{1}{2} \theta \bigg\vert_0^{2\pi} = \pi $$

\section*{Problem 10}

\subsection*{Part A}

This integral is probably closest to 20 given that the area is approximately 7
and no $y$ or $z$ values are negative so the result should be at least 7.

\subsection*{Part B}

Parameterization for $S$:
$$ \vec{r}(u, v) = \langle u, v, 6 - 6u - 3v \rangle $$

From Problem 3, we already know the norm of the cross product will be the area
of this trangle, which is:
$$ \sqrt{46} $$

The surface integral then becomes:
$$ \iint_R v (6 - 6u - 3v) \sqrt{46} dA $$

Given from the region $R$,
$$ \int\limits_0^1\int\limits_0^{2-2u} v (6 - 6u - 3v) \sqrt{46} dv du $$

This integral evaluates to:
\begin{align*}
\int\limits_0^1 -3\left( (-1 +u)v^2+\frac{v^3}{3} \right) \bigg\vert_0^{2-2u}
  du &= \int\limits_0^1 4 - 12 u +12 u^2 - 4 u ^3 du \\ &= 4u - 6u^2 + 4u^3 - u^4
\bigg\vert_0^1 = 1
\end{align*}

Since I omitted the $\sqrt{46}$ term, the result is:
$$ \sqrt{46} $$

\section*{Problem 11}

The parameterization of $S$:
$$ \vec{r}(u,v) = \langle u, v, u^2 + y^2 \rangle $$

The partial derivatives of this surface:
$$ \frac{\partial \vec{r}}{\partial u} = \langle 1, 0, 2u \rangle $$
$$ \frac{\partial \vec{r}}{\partial v} = \langle 0, 1, 2v \rangle $$

The cross product of these two vectors is:
$$ \langle -2u, -2v, 1 \rangle $$

The norm of the cross product becomes:
$$ \sqrt{1 + 4 (u^2 + v^2)} $$

The integral for the surface area:
$$ \iint \sqrt{1 + 4(u^2 + v^2)} dA $$

This can be easily converted into polar coordinates:
$$ \iint \sqrt{1 + 4r^2} r dr d\theta $$

The region of this integration is where $2 < r < 3$:
$$ \int\limits_0^{2\pi}\int\limits_2^3 \sqrt{1 + 4r^2} r dr d\theta $$

This integral evaluates to:
$$ \int_0^{2\pi} \frac{1}{12} (1 + 4r^2)^{3/2} \bigg\vert_2^3 d\theta =
\int_0^{2\pi} 12.91 d\theta = 12.91 \theta \bigg\vert_0^{2\pi} = 81.14 $$

\section*{Problem 12}

The parameterization of this surface is:
$$ \vec{r}(u,v) = \langle u, v, 2 \rangle $$

The derivatives of this parameterization are:
$$ \frac{\partial \vec{r}}{\partial u} = \langle 1, 0, 0 \rangle $$
$$ \frac{\partial \vec{r}}{\partial v} = \langle 0, 1, 0 \rangle $$

Their cross product is, most clearly,
$$ \langle 0, 0, 1 \rangle $$

This has a norm of:
$$ 1 $$

Thus, the integral becomes:
$$ \iint u^2 + v^2 dA $$

This can be easily converted to polar coordinates:
$$ \iint r^2 r dr d\theta $$

The region of this integration is where $ 0 \leq r \leq 1$:
$$ \int\limits_0^{2\pi}\int\limits_0^1 r^3 dr d\theta $$

This integral evaluates to:
$$ \int\limits_0^{2\pi} \frac{r^4}{4} \bigg\vert_0^1 d\theta =
\int\limits_0^{2\pi} \frac{1}{4} d\theta = \frac{\theta}{4} \bigg\vert_0^{2\pi}
= \frac{\pi}{2} $$

Then, the following integral can also be converted into polar coordinates:
$$ \iint x^2 + y^2 dA = \int\limits_0^{2\pi}\int\limits_0^1 r^2 r dr d\theta $$

Which is clearly the integral computed above.

\section*{Problem 13}

The parameterization of this curve is:
$$ \vec{r}(u,v) = \langle u, v, u^2 - v^2 \rangle $$

The partial derivatives of this parameterization:
$$ \frac{\partial \vec{r}}{\partial u} = \langle 1, 0, 2u \rangle $$
$$ \frac{\partial \vec{r}}{\partial v} = \langle 0, 1, -2v \rangle $$

The cross product of these two partial derivatives will give a normal vector:
$$ \langle -2u, 2v, 1 \rangle $$

To turn this into a normal vector, divide by the norm:
$$ \vert\langle -2u, 2v, 1 \rangle \vert = \sqrt{1 + 4(u^2 + v^2)} $$

And, reversing the parameterization:
$$ \hat{n} = \frac{\langle -2x, 2y, 1 \rangle}{\sqrt{1 +4(x^2 + y^2)}} $$

\section*{Problem 14}

\subsection*{Part A}

The gradient for this curve is:
$$ \vec{\nabla} f(x, y, z) = \langle 2x, 2y, 2z \rangle $$

The vector field is given by:
$$ \vec{V} = \langle -z, z, x-y \rangle $$

Taking the dot product:
$$ \vec{\nabla} f(x, y, z) \cdot \vec{V} = 0 $$

Since the dot product is zero, the vectors are perpendicular. So there cannot be
any flux.

\subsection*{Part B}

This vector field has no $z$ component and the $x$ and $y$ components always
point into the surface. As a result, the flux should be negative out of the
surface.

\section*{Problem 15}

Let $\vec{v} = \langle 10, 0, 0 \rangle$ and $\vec{w} = \langle 0, 20, 0
\rangle$.

The cross product is:
$$ \vec{v} \times \vec{w} = \langle 0, 0, 200 \rangle $$

The equation for this curve is (clearly):
$$ z = 0 $$

The parameterization can be given:
$$ \vec{r}(u, v) = \langle u, v, 0 \rangle $$

The partial derivatives of this parameterization are:
$$ \frac{\partial \vec{r}}{\partial u} = \langle 1, 0, 0 \rangle $$
$$ \frac{\partial \vec{r}}{\partial v} = \langle 0, 1, 0 \rangle $$

Their cross product is:
$$ \langle 0, 0, 1 \rangle $$

The flux through the surface using the parameterized variables:
$$ \vec{V} = \langle v, 0, u \rangle $$

The flux integral becomes:
$$ \iint \langle v, 0, u \rangle \cdot \langle 0, 0, 1 \rangle dA $$

The region is a triangle defined in the first quadrant below the line:
$$ v = 20 - 2u $$

This integral evaluates to:
\begin{align*}
 \int\limits_0^{10} \int\limits_0^{20-2u} u dv du = \int\limits_0^{10} u v
  \bigg\vert_0^{20-2u} du &= \int\limits_0^{10} u(20 - 2u) du\\ &= -2 \left( -5 u^2 +
  \frac{u^3}{3} \right) \bigg\vert_0^{10}\\ &= \frac{1000}{3}
\end{align*}

\section*{Problem 16}
The parameterization of the top hemisphere of the unit circle is:
$$ \vec{r}(\theta, \phi) = \langle \cos \theta \sin \phi, \sin \theta \sin \phi,
\cos \phi \rangle $$

The partial derivatives of this parameterization are:
$$ \frac{\partial \vec{r}}{\partial \theta} = \langle -\sin \theta \sin \phi,
\cos \theta \sin \phi, 0 \rangle $$
$$ \frac{\partial \vec{r}}{\partial \phi} = \langle \cos \theta \cos \phi, \sin
\theta \cos \phi, -\sin \phi \rangle $$

The cross product of these two vectors:
$$ \frac{\partial \vec{r}}{\partial \theta} \times \frac{\partial
\vec{r}}{\partial \phi} =
\begin{vmatrix}
  \hat{i} & \hat{j} & \hat{k} \\
  -\sin \theta \sin \phi & \cos \theta \sin \phi & 0 \\
  \cos \theta \cos \phi & \sin \theta \cos \phi & - \sin \phi \\
\end{vmatrix} $$

Expanding this determinant:
$$ \frac{\partial \vec{r}}{\partial \theta} \times \frac{\partial
\vec{r}}{\partial \phi} = \langle -\cos \theta \sin^2 \phi, -\sin \theta \sin^2
\phi, -\cos \phi \sin \phi \rangle $$

The vector field is given by:
$$ \vec{V} = \langle 0, 0, 1 \rangle $$

The flux integral then becomes:
$$ \iint \langle 0, 0, 1 \rangle \cdot \langle -\cos \theta \sin^2 \phi, -\sin
\theta \sin^2 \phi, -\cos \phi \sin \phi \rangle dA $$

Simplifying after the dot product:
$$ \iint - \cos \phi \sin \phi dA $$

Since the normal vector calculated is pointing inward, we need to add a
negative:
$$ \iint \cos \phi \sin \phi dA $$

Given the surface is the top hemisphere, $0 \leq \theta \leq \pi$ and $0 \leq
\phi \leq \pi / 2$:
$$ \int\limits_0^{2\pi}\int\limits_0^{\pi/2} \cos \phi \sin \phi d\phi d\theta
$$

Using the double angle identity:
$$ \frac{1}{2} \int\limits_0^{2\pi}\int\limits_0^{\pi/2} \sin(2\phi) d\phi
d\theta $$

This integral evaluates to:
$$ \frac{1}{2} \int\limits_0^{2\pi} -\frac{1}{2} \cos(2\phi) \bigg\vert_0^{\pi/2} d\theta
= \frac{1}{2} \int\limits_0^{2\pi} d\theta = \frac{1}{2} \theta \bigg\vert_0^{2\pi} = \pi $$

\section*{Problem 17}

This is the same parameterization as before, except the range of integration is
different:
$$ \int\limits_0^{2\pi}\int\limits_0^{\pi} \cos \phi \sin \phi d\phi d\theta$$

Using the double angle identity:
$$ \frac{1}{2} \int\limits_0^{2\pi}\int\limits_0^{\pi} \sin(2\phi) d\phi
d\theta $$

This integral evaluates to:
$$ \frac{1}{2} \int\limits_0^{2\pi} -\frac{1}{2} \cos(2\phi) \bigg\vert_0^{\pi} d\theta
= 0 $$

This makes intuitive sense because the flux is flowing straight in the $z$
direction with constant speed. If the surface is open, then there will be
positive or negative flux (depending on how we specify into or out of the
surface). If there is a closed surface, then every bit of "fluid" that enters
the surface with have to leave it. Leaving the total "fluid" in the closed
surface constant. 

\section*{Problem 18}

In Problem 16, our surface is defined as the top hemisphere of the unit sphere.
The flux is only pointing in the $z$ direction so the hemisphere can be
simplified to a unit disk where all the vectors are parallel to the normal of
the face. Thus, the flux can be easily computed by multiplying the area times
the constant flow vector: 1. Since the area of the unit sphere is $\pi$, the
flux must be $\pi$.

\end{document}
