\documentclass{article}
\usepackage{tikz}
\usepackage{float}
\usepackage{enumerate}
\usepackage{amsmath}
\usepackage{bm}
\usepackage{indentfirst}
\usepackage{siunitx}
\usepackage[utf8]{inputenc}
\usepackage{graphicx}
\graphicspath{ {Images/} }
\usepackage{float}
\usepackage{mhchem}
\usepackage{chemfig}
\allowdisplaybreaks

\title{ 18.02 Problem Set 5 }
\author{ Robert Durfee }
\date{ March 23, 2018 }

\begin{document}

\maketitle

\section{ Parameterized Curves and Lengths }

\begin{enumerate}[1.]
  \item Let $c$ be defined be $(x(t),y(t)) = (\sin 2t, \sin t)$.
    \begin{enumerate}[a.]
      \item \begin{align*}
          (x(0),y(0)) &= (0,0) \\
          (x(\pi / 4), y(\pi / 4)) &= (1, 1/\sqrt{2}) \\
          (x(\pi / 2), y(\pi / 2)) &= (0,1) \\
          (x(3 \pi / 4), y(3 \pi / 4)) &= (-1, 1/\sqrt{2}) \\
          (x(\pi), y(\pi)) &= (0, 0)
        \end{align*}
      \item The derivatives of the parameterized curve $c$ are $(x'(t), y'(x)) =
        (2\cos2t, \cos t)$.
        \begin{align*}
          (x'(0),y'(0)) &= (2,1) \\
          (x'(\pi / 4), y'(\pi / 4)) &= (0, 1/\sqrt{2}) \\
          (x'(\pi / 2), y'(\pi / 2)) &= (-2,0) \\
          (x'(3 \pi / 4), y'(3 \pi / 4)) &= (0, -1/\sqrt{2}) \\
          (x'(\pi), y'(\pi)) &= (2, -1)
        \end{align*}
      \item Sketch of $c$:

        \begin{figure}[H]
          \centering
          \includegraphics[scale=0.25]{"SketchOfC"}
          \caption{Sketch of $c$}
        \end{figure}
    \end{enumerate}
  \item Using the curve $c$ from the previous problem, the linear approximation
    can be calculated,
    $$ (\sin 2t_{0} + (2 \cos 2 t_{0}) \Delta t, \sin t_{0} + (\cos t_{0})
    \Delta t) $$
    \begin{enumerate}[a.]
      \item When $t=0$,
        $$ (0, 0) $$

        When $t = 0.01$,
        $$ (0.02, 0.01) $$

        The distance between these two points,
        $$ d \approx \sqrt{0.02^{2} + 0.01^{2}} \approx 0.0224 $$

      \item When $t = \pi / 4$,
        $$ (1, 1 / \sqrt{2}) $$

        When $t = \pi / 4 + 0.02$,
        $$ (1, 0.7212) $$

        The distance between these two points,
        $$ d \approx \sqrt{0^{2} + (0.7212 - 1 / \sqrt{2})^{2}} \approx 0.0141 $$
    \end{enumerate}
  \item The length of a small element of the curve,
    $$ d\ell = \sqrt{(x'(t_{0}))^{2} + (y'(t_{0}))^{2}} dt $$

    Integrating over the entire curve,
    $$ L = \int\limits_{a}^{b} \sqrt{(x'(t))^{2} + (y'(t))^{2}} dt $$

    Substituting values,
    $$ L = \int\limits_{0}^{\pi}\sqrt{4 \cos^{2}(2t) + \cos^{2} (t)} dt$$
  \item When $t = 0$,
    $$ g(0) = 2.24 $$

    This is equal to the answer in Problem 2 Part A multiplied by $\Delta t$.

    \bigbreak

    When $t = \pi / 4$,
    $$ g(\pi / 4) = 7.07 $$

    This is equal to the answer in Problem 2 Part B multiplied by $\Delta t$.

  \item \begin{enumerate}[a.]
      \item Setting $x(t) = 1$ and $y(t) = 1$,
        $$ 3t^{2} - 2 = 1,\ t = 1 $$
        $$ 2t^{3} - 1 = 1,\ t = 1 $$

        Thus, (1,1) is on the curve when $t = 1$.

        \bigbreak

        The slope of this curve is
        $$ (6t, 6t) $$

        Therefore, a tangent vector when $t = 0$ is $\langle 1, 1 \rangle$.

        \bigbreak

        A normal vector is perpendicular to the tangent vector, therefore,
        $\langle 1, -1 \rangle$ is a normal vector when $t = 1$.
      \item Setting $x = 1$ and $y = 1$
        $$ 1^{2} = -1^{3} + 1 + 1,\ 1 = 1 $$

        This is a true statement, so (1,1) is on the curve.

        \bigbreak

        The gradient of this curve is
        $$ \langle -3x^{2} + 1, -2y \rangle$$

        Therefore, a normal vector at point (1,1) is $\langle -1, -1 \rangle$.

        \bigbreak

        A tangent vector is perpendicular to the normal vector, therefore,
        $\langle -1, 1 \rangle$ is a tangent vector at (1, 1).

      \item Taking the dot product between the tangent vectors,
        $$ \cos^{-1}\left( \frac{ \langle 1, 1 \rangle \cdot \langle -1, 1
        \rangle }{ 2 }\right) = \frac{ \pi }{ 2 } = 90^{\circ}$$
    \end{enumerate}
  \item \begin{enumerate}[a.]
    \item $(t, \sqrt{4 - (t - 1)^{2}} + 1)$ for $1 \leq t \leq 3$.
    \item $(\sqrt{4 - (2 - t)^{2}} + 1, 3 - t)$ for $0 \leq t \leq 2$.
    \item $(1 + 2 \sin t, 1 + 2 \cos t)$ for $0 \leq t \leq \pi / 2$.
  \end{enumerate}
\end{enumerate}

\section{Line Integrals with Respect To Arc Length}

\begin{enumerate}[1.]
  \setcounter{enumi}{6}
  \item A parameterization of $c$,
    $$ (t, t^{2}) \textrm{ for } 0 \leq t \leq 2 $$

    The slope of $c$,
    $$ (1, 2t) $$

    Computing the integral,
    \begin{align*}
      \int\limits_{c} \sqrt{1 + 4y} ds &= \int\limits_{0}^{2} \sqrt{1 + 4t^{2}}
      \sqrt{1 + 4t^{2}} dt \\
      &= \int\limits_{0}^{2} 1 + 4t^{2} dt \\
      &= t + \frac{ 4 }{ 3 } t^{3} \bigg\vert_{0}^{2} \\
      &= \frac{38}{3}
    \end{align*}
  \item Using parameterization $(1 - t, t)$,
    \begin{align*}
      \int\limits_{c} y ds &= \int\limits_{0}^{1} t \sqrt{2} dt \\
                           &= \frac{ \sqrt{2} t^{2} }{ 2 } \bigg\vert_{0}^{1} \\
                           &= \frac{ \sqrt{2} }{ 2 }
    \end{align*}

    Using parameterization $(1 - 4t^{2}, 4t^{2})$,
    \begin{align*}
      \int\limits_{c} y ds &= \int\limits_{0}^{1/2} 4t^{2} \sqrt{(-8t)^{2} +
      (8t)^{2}} dt \\
      &= \int\limits_{0}^{1/2} 32 \sqrt{2} t^{3} dt \\
      &= 8 \sqrt{2} t^{4} \bigg\vert_{0}^{1/2} \\
      &= \frac{ \sqrt{2} }{ 2 }
    \end{align*}
  \item The rod is most likely to break near the middle. From the function
    provided, the distance between the original rod and the sagging rod is
    largest when $x = 0$. This is the middle of the rod.
\end{enumerate}

\section{Line Integrals with Respect to $dx$ and $dy$}

\begin{enumerate}[1.]
  \setcounter{enumi}{9}
  \item \begin{enumerate}[a.]
    \item Positive. $dy$ is positive throughout the interval and so is $x$.
    \item Zero. $dx$ is zero throughout the interval.
    \item Negative. $dx$ is negative and $y^{2}$ is positive throughout the
      interval.
    \item Positive. $dy$ is positive and $x + y$ is positive throughout the
      interval.
    \item Positive. Although $dx$ is initially positive and becomes negative,
      $e^{y}$ is always positive and decreases in magnitude. Therefore, there
      will be more positives than negatives.
  \end{enumerate}
\item Using the parameterization for $c$, $(t^{2}, t)$ for $1 \geq t \geq -1$,
  \begin{align*}
    \int\limits_{c} y^{2} dx &= \int\limits_{1}^{-1} 2 t^{3} dt \\
                             &= \frac{ t^{4} }{ 2 } \bigg\vert_{1}^{-1} \\
                             &= 0
  \end{align*}
\item Using the parameterization $(-\sqrt{1 - t^{2}}, t)$ for $0 \leq t \leq 1$,
  \begin{align*}
    \int\limits_{c} x^{2} dy &= \int\limits_{0}^{1} (-\sqrt{1 - t^{2}})^{2} dt
    \\
    &= \int\limits_{0}^{1} (1 - t^{2}) dt \\
    &= t - \frac{ t^{3} }{ 3 } \bigg\vert_{0}^{1} \\
    &= \frac{ 2 }{ 3 }
  \end{align*}
\item Parameterizing the curve into four segments,
  $$ c_{1}: (t, b_{1}) \textrm{ for } a_{1} \leq t \leq a_{2} $$
  $$ c_{2}: (a_{2},t) \textrm{ for } b_{1} \leq t \leq b_{2} $$
  $$ c_{3}: (t, b_{2}) \textrm{ for } a_{2} \geq t \geq a_{1} $$
  $$ c_{4}: (a_{1}, t) \textrm{ for } b_{2} \geq t \geq b_{1} $$

  \begin{enumerate}[a.]
    \item Since this integral is with respect to $dx$, $c_{2}$ and $c_{4}$ can
      be ignored as they are vertical lines with no change in $x$. This leaves,
      \begin{align*}
        \int\limits_{c} x dx &= \int\limits_{a_{1}}^{a_{2}} t dt +
        \int\limits_{a_{2}}^{a_{1}} t dt \\
        &= 0
      \end{align*}
    \item Since this integral is with respect to $dy$, $c_{1}$ and $c_{3}$ can
      be ignored as they are horizontal lines with no change in $y$. This
      leaves,
      \begin{align*}
        \int\limits_{c} x dy &= \int\limits_{b_{1}}^{b_{2}} a_{2} dt +
        \int\limits_{b_{2}}^{b_{1}} a_{1} dt \\
        &= a_{1}(b_{1} - b_{2}) + a_{2}(b_{2} - b_{1})
      \end{align*}
  \end{enumerate}

\end{enumerate}

\section{Vector Fields}

\begin{enumerate}[1.]
  \setcounter{enumi}{13}
  \item For $\vec{V} = \langle y, -x \rangle$,
    \begin{center}
      \begin{tabular}{ c c c }
        $\vec{V}(1,0) = \langle 0, -1 \rangle$ & $\vec{V}(0,1) = \langle 1, 0
        \rangle$ & $\vec{V}(0,0) = \langle 0, 0 \rangle$ \\
        $\vec{V}(-1,0) = \langle 0, 1 \rangle$ & $\vec{V}(0,-1) = \langle -1, 0
        \rangle$ & $\vec{V}(1/2,0) = \langle 0, -1/2 \rangle$ \\
        $\vec{V}(1/2,1/2) = \langle 1/2, -1/2 \rangle$ & $\vec{V}(-1/2,1/2) =
        \langle 1/2, 1/2 \rangle$ & $\vec{V}(0,-1/2) = \langle -1/2, 0 \rangle$
      \end{tabular}
    \end{center}

    \begin{figure}[H]
      \centering
      \includegraphics[scale=0.45]{"VectorFieldSketch1"}
      \caption{Vector Field Sketch}
    \end{figure}
  \item For $\vec{V} = \langle 0, x \rangle$,
    \begin{center}
      \begin{tabular}{ c c c }
        $\vec{V}(1,0) = \langle 0, 1 \rangle$ & $\vec{V}(0,1) = \langle 0, 0
        \rangle$ & $\vec{V}(-1,0) = \langle 0, -1 \rangle$ \\
        $\vec{V}(0,-1) = \langle 0, 0 \rangle$ & $\vec{V}(1/2,1/2) = \langle 0,
        1/2 \rangle$ & $\vec{V}(-1/2,-1/2) = \langle 0, -1/2 \rangle$ \\
        $\vec{V}(-1/2,1/2) = \langle 0, -1/2 \rangle$ & $\vec{V}(1/2,-1/2) =
        \langle 0, 1/2 \rangle$ & $\vec{V}(0,0) = \langle 0, 0 \rangle$
      \end{tabular}
    \end{center}

    \begin{figure}[H]
      \centering
      \includegraphics[scale=0.45]{"VectorFieldSketch2"}
      \caption{Vector Field Sketch}
    \end{figure}

\end{enumerate}

\section{Flux Through a Curve}

\begin{enumerate}[1.]
  \setcounter{enumi}{15}
  \item Using the equations determined in problem 17.
    \begin{enumerate}[a.]
      \item For the length of the small segment,
        $$ \ell = \Delta t \sqrt{1 + 4t^{2}} = 0.01 \sqrt{1 +
        4(0.5)^{2}} = 0.01\sqrt{2}$$
      \item For the unit normal vector,
        $$ \hat{n} = \frac{ \langle 2t, -1 \rangle \Delta t}{ \ell } = \frac{
        \langle 2 (0.5), -1 \rangle 0.01 }{ 0.01 \sqrt{2} } = \left\langle
        \frac{ 1 }{ \sqrt{2} }, -\frac{ 1 }{ \sqrt{2} } \right\rangle$$
      \item For flux,
        $$ \Phi = (2 t^{3} + t) \Delta t = (2(0.5)^{3} + 0.5) 0.01 = 0.0075 $$
    \end{enumerate}
  \item \begin{enumerate}[a.]
    \item The curve $c$ can be parameterized $(t, t^{2})$ for $t$ from $0$ to
      $1$. We can take the linear approximation,
      $$ L_{c} = (t + \Delta t, t^{2} + 2t \Delta t) $$

      The distance between two points on this line are
      $$ \sqrt{((t + \Delta t) - t)^{2} + ((t^{2} + 2t\Delta t) - t^{2})^{2}} $$

      Which simplifies to,
      $$ \Delta t \sqrt{1 + 4t^{2}} $$
    \item The tangential vector to the curve is defined through linear
      approximation,
      $$ \langle \Delta t, 2t \Delta t \rangle = \langle 1, 2t \rangle \Delta t $$

      A normal vector is defined perpendicular to the tangent vector. This
      occurs when the dot product is zero,
      $$ \langle 1, 2t \rangle \Delta t \perp \langle 2t, -1 \rangle \Delta t $$

      Converting this vector into a unit vector by dividing by length,
      $$ \hat{n} = \frac{ \langle 2t, -1 \rangle }{ \sqrt{1 + 4t^{2}} } $$

    \item Parameterizing the vector field $\vec{V}(x, y) = \langle y, -x
      \rangle$ as,
      $$ \vec{V}(x(t), y(t)) = \langle t^{2}, -t \rangle = \vec{v}$$

      This will represent the flow at a particular $t$. Now we can calculate
      flux,
      $$ d\Phi = \vec{v} \cdot \hat{n} \vert \ell \vert = \langle t^{2}, t
      \rangle \cdot \langle 2t, -1 \rangle \Delta t = (2t^{3} + t) \Delta t$$
  \end{enumerate}

\item The equation determined in Problem 17 will result in the small amount of
  flux through a small line segment. To calculate the entire flux, integrate
  this over the entire curve,
$$ \int\limits_0^{1} \langle t^{2}, -t \rangle \cdot \langle 2t, -1 \rangle dt =
\int\limits_{0}^{1} (2t^{3} + t)dt = 1$$
\end{enumerate}

\end{document}

