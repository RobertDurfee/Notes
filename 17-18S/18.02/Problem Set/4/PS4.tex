\documentclass{article}
\usepackage{tikz}
\usepackage{float}
\usepackage{enumerate}
\usepackage{amsmath}
\usepackage{bm}
\usepackage{indentfirst}
\usepackage{siunitx}
\usepackage[utf8]{inputenc}
\usepackage{graphicx}
\graphicspath{ {Images/} }
\usepackage{float}
\usepackage{mhchem}
\usepackage{chemfig}
\allowdisplaybreaks

\title{ 18.02 Problem Set 4 }
\author{ Robert Durfee }
\date{ March 16, 2018 }

\begin{document}

\maketitle

\section{ Gradients and Lagrange Multipliers }

\begin{enumerate}[1.]
  \item \begin{enumerate}[a.]
    \item Let $f(x,y) = x^{2} + xy + y^{2}$. Computing the gradient for $f(x,
      y)$:
      $$ \nabla f(x, y) = \langle 2x + y, 2y + x \rangle $$

      Since the gradient points normal to the level curves and $C$ is a level
      curve of $f(x, y)$, the gradient at these points will be normal to the
      curve $C$:
      $$ \nabla f(0, \sqrt{3}) = \langle \sqrt{3}, 2 \sqrt{3} \rangle $$
      $$ \nabla f(0, -\sqrt{3}) = \langle -\sqrt{3}, -2 \sqrt{3} \rangle $$

    \item Tangent vectors can be found by finding vectors perpendicular to
      gradient vectors. Perpendicular vectors result in a zero dot product:
      $$ \langle -2\sqrt{3}, \sqrt{3} \rangle \perp \langle \sqrt{3}, 2\sqrt{3}
      \rangle $$
      $$ \langle 2 \sqrt{3}, -\sqrt{3} \rangle \perp \langle -\sqrt{3}, -2
      \sqrt{3} \rangle $$
    \end{enumerate}
  \item \begin{enumerate}[a.]
      \item When vectors are parallel, they are multiples of each other. Since
        the normal vectors of $C$ are given by the gradient of $f(x, y)$,
        setting the two equal to by a factor $\lambda$ can find the desired
        point:
        $$ \langle 2x + y, 2y + x \rangle = \lambda \langle 1, 0 \rangle $$

        Applying the constraint:
        $$ f(x, y) = 3 $$

        Solving this system of three equations yields:
        $$ (-2, 1) \textrm{ and } (2, -1) $$

      \item Now with $\langle 1, 1 \rangle$:
        $$ \langle 2x + y, 2y + x \rangle = \lambda \langle 1, 1 \rangle $$

        Applying the constraint:
        $$ f(x, y) = 3 $$

        Solving this system of three equations yields:
        $$ (-1, -1) \textrm{ and } (1, 1) $$
    \end{enumerate}
  \item \begin{enumerate}[a.]
      \item
    \end{enumerate}
\end{enumerate}

\end{document}

