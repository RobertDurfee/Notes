\documentclass{article}
\usepackage{tikz}
\usepackage{float}
\usepackage{enumerate}
\usepackage{amsmath}
\usepackage{bm}
\usepackage{indentfirst}
\usepackage{siunitx}
\usepackage[utf8]{inputenc}
\usepackage{graphicx}
\graphicspath{ {Images/} }
\usepackage{float}
\usepackage{mhchem}
\usepackage{chemfig}
\allowdisplaybreaks

\title{ 18.02 Problem Set 4 }
\author{ Robert Durfee }
\date{ March 16, 2018 }

\begin{document}

\maketitle

\section{ Gradients and Lagrange Multipliers }

\begin{enumerate}[1.]
  \item \begin{enumerate}[a.]
    \item Let $f(x,y) = x^{2} + xy + y^{2}$. Computing the gradient for $f(x,
      y)$:
      $$ \nabla f(x, y) = \langle 2x + y, 2y + x \rangle $$

      Since the gradient points normal to the level curves and $C$ is a level
      curve of $f(x, y)$, the gradient at these points will be normal to the
      curve $C$:
      $$ \nabla f(0, \sqrt{3}) = \langle \sqrt{3}, 2 \sqrt{3} \rangle $$
      $$ \nabla f(0, -\sqrt{3}) = \langle -\sqrt{3}, -2 \sqrt{3} \rangle $$

    \item Tangent vectors can be found by finding vectors perpendicular to
      gradient vectors. Perpendicular vectors result in a zero dot product:
      $$ \langle -2\sqrt{3}, \sqrt{3} \rangle \perp \langle \sqrt{3}, 2\sqrt{3}
      \rangle $$
      $$ \langle 2 \sqrt{3}, -\sqrt{3} \rangle \perp \langle -\sqrt{3}, -2
      \sqrt{3} \rangle $$
    \end{enumerate}
  \item \begin{enumerate}[a.]
      \item When vectors are parallel, they are multiples of each other. Since
        the normal vectors of $C$ are given by the gradient of $f(x, y)$,
        setting the two equal to by a factor $\lambda$ can find the desired
        point:
        $$ \langle 2x + y, 2y + x \rangle = \lambda \langle 1, 0 \rangle $$

        Applying the constraint:
        $$ f(x, y) = 3 $$

        Solving this system of three equations yields:
        $$ (-2, 1) \textrm{ and } (2, -1) $$

      \item Now with $\langle 1, 1 \rangle$:
        $$ \langle 2x + y, 2y + x \rangle = \lambda \langle 1, 1 \rangle $$

        Applying the constraint:
        $$ f(x, y) = 3 $$

        Solving this system of three equations yields:
        $$ (-1, -1) \textrm{ and } (1, 1) $$
    \end{enumerate}
  \item \begin{enumerate}[a.]
      \item At the point where $y$ is as large and small as possible, the
        gradient vector will be parallel to $\langle 0, 1 \rangle$.
        $$ \langle 2x + y, 2y + x \rangle = \lambda \langle 0, 1 \rangle $$

        Applying the constraint:
        $$ x^{2} + xy + y^{2} = 3 $$

        Solving this system of three equations yields:
        $$ (-1, 2) \textrm{ and } (1, -2) $$

        Therefore, $y$ is largest at $(-1, 2)$ and smallest at $(1, -2)$.

      \item At the point where $C$ is closest to the origin, the gradient of a
        circle centered at the origin will be parallel to the gradient of $f(x,
        y)$ at that point.
        $$ \langle 2x + y, 2y + x \rangle = \lambda \langle 2x, 2y \rangle$$

        Applying the constraint:
        $$ x^{2} + xy + y^{2} = 3 $$

        Solving this system of three equations yields:
        $$ (-1, -1), (1, 1), (-\sqrt{3}, \sqrt{3}), \textrm{ and } (\sqrt{3},
        -\sqrt{3}) $$

        Of these solutions, $(-1, -1)$ and $(1, 1)$ are closest to the origin and
        $(-\sqrt{3}, \sqrt{3})$ and $(\sqrt{3},-\sqrt{3})$ are farthest from the
        origin.
    \end{enumerate}
  \item $A$ are the vectors from Problem 1. $B$ are the points from Problem 2.
    $C_{1}$ are the points farthest from the origin from Problem 3 and $C_{2}$
    are the points closest to the origin from Problem 3.

    \begin{figure}[H]
      \centering
      \includegraphics[scale=0.50]{"CurveC"}
      \caption{Curve C}
    \end{figure}

  \item \begin{enumerate}[a.]
      \item Using the gradient from previous problem. Dotting this gradient with
        the vector in the derivative:
        $$ \langle 2x + y, 2y + x \rangle \cdot \langle -3/5, 4/5 \rangle =
        -\frac{ 3 }{ 5 }(2x + y) + \frac{ 4 }{ 5 }(x + 2y) $$

        \item Solving this derivative at (1,1):
          $$ -\frac{ 3 }{ 5 }(2 \cdot 1 + 1) + \frac{ 4 }{ 5 }(1 + 2 \cdot 1) =
          -\frac{ 9 }{ 5 } + \frac{ 12 }{ 5 } = \frac{ 3 }{ 5 } $$

          Since this value is positive, the function $f$ will increases as the
          result of our path. Therefore, we will be outside circle $C$.
    \end{enumerate}
\end{enumerate}

\section{Flux in Two Dimensions}

\begin{enumerate}[1.]
  \setcounter{enumi}{5}
  \item \begin{enumerate}[a.]
      \item Flux is defined as the following with $\vec{v}$ perpendicular to
        line segment $\ell$:
        $$ \Phi = \vert \vec{v} \vert \textrm{ length}(\ell) $$

        In the following diagram, to use the previous definition, we need to
        determine the length of $\ell'$ This length is given by:
        $$ \textrm{length}(\ell') = \textrm{length}(\ell) \cos \theta $$

        Taking the dot product of $\vec{n}$ and $\vec{v}$ gives:
        $$ \vec{v} \cdot \vec{n} = \vert \vec{v} \vert \cos \theta $$

        Combining these two definitions together:
        $$ \Phi = \vec{v} \cdot \vec{n} \textrm{ length}(\ell) $$

        \begin{figure}[H]
          \centering
          \includegraphics[scale=0.70]{"Flux"}
          \caption{Flux}
        \end{figure}
      \item The following figure demonstrated negative flux:

        \begin{figure}[H]
          \centering
          \includegraphics[scale=0.70]{"Flux"}
          \caption{Negative Flux}
        \end{figure}
    \end{enumerate}
  \item \begin{enumerate}[a.]
      \item Determine a vector that represents line segment $\ell$:
        $$ \vec{\ell} = \langle 1, -2 \rangle $$

        Determine a perpendicular vector to $\ell$:
        $$ \vec{n} = \langle 2, 1 \rangle $$

        Turn $\vec{n}$ into a unit vector:
        $$ \hat{n} = \left\langle \frac{ 2 }{ \sqrt{5} }, \frac{ 1 }{ \sqrt{5} }
          \right\rangle $$

        Calculate the length of $\ell$:
        $$ \textrm{length}(\ell) = \sqrt{5} $$

        Use the equation:
        $$ \Phi = \vec{v} \cdot \hat{n} \textrm{ length}(\ell) $$

        Substitute values:
        \begin{align*}
          \Phi &= \langle 1, -3 \rangle \cdot \left\langle \frac{ 2 }{ \sqrt{5} },
          \frac{ 1 }{ \sqrt{5} } \right\rangle \sqrt{5} \\
          &= \langle 1, -3 \rangle \cdot \langle 2, 1 \rangle \\
          &= 2 - 3 \\
          &= -1
        \end{align*}
      \item The fluid is passing through $\ell$ from $B$ to $A$.
    \end{enumerate}
\end{enumerate}

\section{Multiple Integrals}

\begin{enumerate}[1.]
  \setcounter{enumi}{7}
  \item \begin{enumerate}[a.]
      \item The double integral over $T'$ will be greater than the double integral
        over $T$ because $T'$ has more area where $x$ is larger and the function
        of integration is based solely on $x$.
      \item For area $T$:
        \begin{align*}
          \iint\limits_{T} x dA &= \int\limits_{0}^{10}\int\limits_{0}^{x/10} x dy
          dx \\
          &= \int\limits_{0}^{10} \frac{ x^{2} }{ 10 } dx \\
          &= \frac{ x^{3} }{ 30 } \bigg\vert_{0}^{10} \\
          &= \frac{ 100 }{ 3 }
        \end{align*}
        For area $T'$:
        \begin{align*}
          \iint\limits_{T'} x dA &= \int \limits_{0}^{10}\int\limits_{x/10}^{1} x
          dy dx \\
          &= \int\limits_{0}^{10} x\left(1 - \frac{ x }{ 10 }\right) dx \\
          &= \frac{ x^{2} }{ 2 } - \frac{ x^{3} }{ 30 } \bigg\vert_{0}^{10} \\
          &= \frac{ 50 }{ 3 }
        \end{align*}
    \end{enumerate}

  \item \begin{align*}
      \int\limits_{0}^{1}\int\limits_{0}^{2} e^{x + y} dx dy &=
      \int\limits_{0}^{1} e^{2+y} - e^{y} dy \\
      &= e^{2 + y} - e^{y} \bigg\vert_{0}^{1} \\
      &= e^{3} - e^{2} - e^{1} + 1
    \end{align*}

  \item Region R is shown below.

    \begin{figure}[H]
      \centering
      \includegraphics[scale=0.60]{"R"}
      \caption{Region R}
    \end{figure}

    \begin{align*}
      \iint\limits_{R} x dA &= \int\limits_{0}^{1}\int\limits_{x^{2}}^{x} x dy
      dx \\
      &= \int \limits_{0}^{1} x^{2} - x^{3} dx \\
      &= \frac{ x^{3} }{ 3 } - \frac{ x^{4} }{ 4 } \bigg\vert_{0}^{1} \\
      &= \frac{ 1 }{ 3 } - \frac{ 1 }{ 4 } \\
      &- \frac{ 1 }{ 12 }
    \end{align*}

  \item \begin{enumerate}[a.]
      \item Region R is shown below.

        \begin{figure}[H]
          \centering
          \includegraphics[scale=0.60]{"R2"}
          \caption{Region R}
        \end{figure}

      \item Switching limits of integration:
        $$ \int\limits_{0}^{1}\int\limits_{2x}^{2} f(x, y) dy dx =
        \int\limits_{0}^{2}\int\limits_{0}^{y/2} f(x, y) dx dy $$
    \end{enumerate}
  \item \begin{align*}
      \iint\limits_{H} y dA &= \int\limits_{-1}^{1}\int\limits_{0}^{\sqrt{1 -
      x^{2}}} y dy dx \\
      &= \int\limits_{-1}^{1} 1 - x^{2} dx \\
      &= x - \frac{ x^{3} }{ 3 } \bigg\vert_{-1}^{1} \\
      &= \frac{ 2 }{ 3 }
    \end{align*}
  \item \begin{align*}
      \iint\limits_{D} \sqrt{x^{2} + y^{2}} dA &=
      \int\limits_{0}^{2\pi}\int\limits_{0}^{10} r^{2} dr d\theta \\
      &= \int\limits_{0}^{2\pi} \frac{ 1000 }{ 3 } d\theta \\
      &= \frac{ 1000 }{ 3 } \theta \bigg\vert_{0}^{2\pi} \\
      &= \frac{ 2000 }{ 3 } \pi
    \end{align*}
  \item \begin{align*}
      \iint\limits_{S} 10 - (x + y) dA &= \int\limits_{0}^{1}\int\limits_{0}^{1}
      (10 - x - y) dx dy \\
      &= \int\limits_{0}^{1} \left( 10 - \frac{ 1 }{ 2 } - y \right) dy \\
      &= 10 - \frac{ 1 }{ 2 } - \frac{ y^{2} }{ 2 } \bigg\vert_{0}^{1} \\
      &= 9
    \end{align*}
  \item \begin{align*}
      \iint\limits_{R} 4 - x^{2}-y^{2}dA &=
      \int\limits_{0}^{2\pi}\int\limits_{0}^{2}(4 -
      r^{2})r dr d\theta \\
      &= \int\limits_{0}^{2\pi} 2r^{2} - \frac{ r^{4} }{ 4 }
      \bigg\vert_{0}^{2}d\theta \\
      &= \int\limits_{0}^{2\pi} 4 d\theta \\
      &= 4\theta \bigg\vert_{0}^{2\pi} \\
      &= 8 \pi
    \end{align*}
  \item \begin{align*}
      \iint\limits_{R} 4 - x^{2}-y^{2}dA &=
      \int\limits_{0}^{2\pi}\int\limits_{0}^{1}(4 -
      r^{2})r dr d\theta \\
      &= \int\limits_{0}^{2\pi} 2r^{2} - \frac{ r^{4} }{ 4 }
      \bigg\vert_{0}^{1}d\theta \\
      &= \int\limits_{0}^{2\pi} \frac{ 7 }{ 4 } d\theta \\
      &= \frac{ 7 }{ 4 }\theta \bigg\vert_{0}^{2\pi} \\
      &= \frac{ 7 }{ 2 } \pi
    \end{align*}
  \item \begin{align*}
      \iint\limits_{R} (4 - x^{2}-y^{2}) - (x^{2} + y^{2}) dA &=
      \int\limits_{0}^{2\pi}\int\limits_{0}^{\sqrt{2}}(4 -
      2 r^{2})r dr d\theta \\
      &= \int\limits_{0}^{2\pi} 2r^{2} - \frac{ r^{4} }{ 2 }
      \bigg\vert_{0}^{\sqrt{2}}d\theta \\
      &= \int\limits_{0}^{2\pi} 2 d\theta \\
      &= 2 \theta \bigg\vert_{0}^{2\pi} \\
      &= 4 \pi
    \end{align*}
\end{enumerate}

\end{document}

