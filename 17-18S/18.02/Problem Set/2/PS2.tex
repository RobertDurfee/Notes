\documentclass{article}
\usepackage{tikz}
\usepackage{float}
\usepackage{enumerate}
\usepackage{amsmath}
\usepackage{bm}
\usepackage{indentfirst}
\usepackage{siunitx}
\usepackage[utf8]{inputenc}
\usepackage{graphicx}
\graphicspath{ {Images/} }
\usepackage{float}
\usepackage{mhchem}
\usepackage{chemfig}
\allowdisplaybreaks

\title{ 18.02 Problem Set 2 }
\author{ Robert Durfee }
\date{ February 23, 2018 }

\begin{document}

\maketitle

\section{ Optimization Problems }

\subsection{Is the maximum on the boundary or in the interior?}

\begin{enumerate}[1.]
  \item Starting with the function $f(x, y) = x + 2y$, the partial derivatives
    can be computed:
    $$ f_{x}(x, y) = 1,\ f_{y}(x, y) = 2 $$

    Setting these equal to zero results in a contradiction. As a result, there
    are no critical points for $f$. Since there are no critical points for the
    whole function, the critical points can only exist along the boundary of
    $D$.

  \item Starting with the function $f(x, y) = xy - x^{4} - y^{4}$, the partial
    derivatives can be computed:
    $$ f_{x}(x, y) = y - 4x^{3},\ f_{y}(x, y) = x - 4y^{3} $$

    Setting these equal to zero and solving for $x$ and $y$ results in two
    critical points within $S$:
    $$ (-0.5, -0.5),\ (0, 0) $$

    The respective values for these critical points are $0.125$ and $0$. These
    critical points are minima, therefore, the maximum has to lie along the
    boundary of $S$ as when either $x$ or $y$ increase, the output increases
    significantly.

\end{enumerate}

\subsection{Turning word problem into equations.}

\begin{enumerate}[1.]
  \setcounter{enumi}{2}
  \item \begin{enumerate}[a.]
      \item From the word problem, we are given two equations:
        $$ x + y + 2z = 12 $$
        $$ xyz = f $$

        These equations can be combined to create a maximization problem in
        terms of two variables:
        $$ f(x, y) = \frac{ x y (12 - x - y) }{ 2 } $$

        The word problem also states that the values must be positive. Thus, the
        region $R$ is confined to greater than or equal to zero. Additionally, a
        little bit of number sense shows that if all numbers are positive, $x$
        and $y$ summed has to be less than or equal to 12.

        \begin{figure}[H]
          \centering
          \includegraphics[scale=0.60]{"RegionR"}
          \caption{Region $R$}
        \end{figure}

      \item Using the function defined in the previous part, the partial
        derivatives can be computed:
        $$ f_{x}(x, y) = -\frac{ y (y + 2(x - 6) }{ 2 },\ f_{y}(x, y) = -\frac{
        x (x + 2 (y - 6))}{ 2 } $$

        Setting these equations equal to zero and solving for $x$ and $y$
        results in:
        $$ (0, 0),\ (0, 12),\ (12, 0),\ (4, 4)  $$

        Using the first equation provided dictating the sum of the variables
        must be $12$, $z$ can be computed for each pair:
        $$ (0, 0, 6),\ (0, 12, 0),\ (12, 0, 0),\ (4, 4, 2) $$

        The product of each respective set of variables becomes:
        $$ 0,\ 0,\ 0, 32 $$

        The set $(4, 4, 2)$ has the highest product and must be the maximum. It
        is also clear that the maximum has to occur within $R$ since moving out
        to the edges of the region lowers the product and along all edges, the
        product is zero.
    \end{enumerate}
  \item This problem gives us two equations:
    $$ y = x^{2},\ y = x - 10 $$

    It also defines two points, each lay on their respective line:
    $$ P_{1} = (x_{1}, y_{1}),\ P_{2} = (x_{2}, y_{2}) $$

    Then, since it is asking for the distance between points, we can bring in
    the distance equation as a constraint:
    $$ D = \sqrt{(x_{2} - x_{1})^{2} + (y_{2} - y_{1})^{2}} $$

    However, this equation has too many variables. However, we can substitute
    for $y_{1}$ and $y_{2}$ as they are constrained by equations according to
    the choices of $x_{1}$ and $x_{2}$. The distance equation then becomes:
    $$ D(x_{1}, x_{2}) = \sqrt{(x_{2} - x_{1})^{2} + \left((x_{2} - 1) -
    x_{1}^{2}\right)^{2}} $$

    This equation can then be minimized using our conventional methods.
\end{enumerate}

\subsection{What if there is no minimum?}

\begin{enumerate}[1.]
  \setcounter{enumi}{4}
  \item \begin{enumerate}[a.]
      \item From Larry's problem, two equations can be extracted:
        $$ x y z = 12,\ 4 xz + 4 yz + xy = C $$

        Using the first equation, the second equation can be reduced to a
        function of two variables:
        $$ C(x, y) = xy + \frac{ 48 }{ xy }(x + y) $$

        Now the partial derivatives can be computed:
        $$ C_{x}(x, y) = y - \frac{ 48 }{ x^{2} },\ C_{y}(x, y) = x - \frac{ 48
        }{ y^{2} } $$

        Setting each equation equal to zero and solving for $x$ and $y$ yields:
        $$ \left(2 \sqrt[3]{6}, 2 \sqrt[3]{6}\right) $$

        Substituting these values into the volume constraint formula:
        $$ \left(2 \sqrt[3]{6}, 2 \sqrt[3]{6}, \sqrt[3]{\frac{ 3 }{ 4 }}\right) $$

        The cost for these dimensions is \$39.62.

      \item Now, Larry's cost function is a little simpler:
        $$ C(x, y) = \frac{ 48 }{ xy }(x + y) $$

        Computing the partial derivatives:
        $$ C_{x}(x, y) = -\frac{ 48 }{ x^{2} },\ C_{x}(x, y) = -\frac{ 48 }{
        y^{2} } $$

        Setting each equation equal to zero is impossible, therefore, there are
        no critical points. If Larry wants to minimize his cost, his fish tank
        should have the smallest possible height and the longest possible length
        and width. This way, his parents would bear the most of the cost.

    \end{enumerate}
\end{enumerate}

\section{Level Curves and Partial Derivatives}

\begin{enumerate}[1.]
  \setcounter{enumi}{5}
  \item The first plot has all positive $x$ partial derivatives as level curves
    are all always increasing from left to right. The second plot has level
    curves on the left side which are decreasing from left to right. These $x$
    partial derivatives must be negative.

    \begin{figure}[H]
      \centering
      \includegraphics[scale=0.60]{"LevelCurvesWithPositiveXPartial"}
      \caption{Level Curves with Positive $x$ Partial Derivatives}
    \end{figure}

  \item The first plot has all $y$ partial derivatives equal to zero as
    traversing in the $y$ direction never results in crossing a level curve,
    only traveling along one.

    \begin{figure}[H]
      \centering
      \includegraphics[scale=0.60]{"LevelCurvesWithZeroYPartial"}
      \caption{Level Curves with Zero $y$ Partial Derivatives}
    \end{figure}

  \item From the function $f(x, y) = x^{2} + 3y^{2}$, the partial derivatives
    can be calculated:
    $$ f_{x}(x, y) = 2x,\ f_{y}(x, y) = 6y $$

    Using these partial derivatives and the linear approximation equation, the
    equation for the tangent plane can be derived:
    $$ L(x, y) = f(x_{0}, y_{0}) + f_{x}(x_{0}, y_{0})(x - x_{0}) + f_{y}(x_{0},
    y_{0})(y - y_{0}) $$
    $$ L(x, y) = 4 + 2(x - 1) + 6(y - 1) $$

\end{enumerate}

\section{Vectors and Dot Products}

\begin{enumerate}[1.]
  \setcounter{enumi}{8}
  \item Two vectors are perpendicular if there dot product is equal to zero.
    $$ \langle 2, 2, 1 \rangle \cdot \vec{v} = 0 $$

    Writing out the dot product explicitly:
    $$ 2 v_{1} + 2 v_{2} + v_{3} = 0 $$

    There are infinite solutions to this equation, so $z$ is chosen to be $0$
    arbitrarily.
    $$ 2 v_{1} + 2 v_{2} = 0 $$

    Solving this equation for the ratio of $v_{1}$ to $v_{2}$:
    $$ \frac{ v_{1} }{ v_{2} } = -1 $$

    As a result, a possible solution for a perpendicular vector:
    $$ \langle -1, 1, 0 \rangle $$

  \item Using the initial vector $\langle 2, 2, 1 \rangle$, compute the length:
    $$ \sqrt{2^{2} + 2^{2} + 1^{2}} = 3 $$

    Dividing the original vector by its length will convert into a unit vector.
    $$ \Big \langle \frac{ 2 }{ 3 }, \frac{ 2 }{ 3 }, \frac{ 1 }{ 3 } \Big \rangle $$

  \item The dot product between two vectors computes the angle between through
    the following formula:
    $$ \vec{a} \cdot \vec{b} = \vert \vec{a} \vert \vert \vec{b} \vert
    \cos \theta $$

    Thus, we can define two vectors that originate at (0, 0):
    $$ \vec{a} = \langle 2, 1 \rangle,\ \vec{b} = \langle 1, 2
    \rangle $$

    The length of each of these vectors are:
    $$ \vert \vec{a} \vert = \sqrt{2^{2} + 1^{2}} = \sqrt{5} $$
    $$ \vert \vec{b} \vert = \sqrt{1^{2} + 2^{2}} = \sqrt{5} $$

    The dot product of these vectors can be computed:
    $$ 2 \cdot 1 + 1 \cdot 2 = 4 $$

    The equation for the angle between then becomes:
    $$ \frac{ 4 }{ 5 } = \cos \theta $$

  \item Planes can be described by a normal vector which is calculated using the
    coefficients. If these normal vectors are parallel, then the planes must be
    parallel as well. Vectors are parallel when they are multiples of each
    other.

    \bigbreak

    The normal vectors for the planes are as follows:
    $$ \langle 1, -1, 2 \rangle,\ \langle 2, -2, 2 \rangle,\ \langle 2, -2, 4
    \rangle $$

    It is easy to see that the third normal vector is twice the first normal
    vectors. As a result, the first and third planes are parallel.

  \item The equation for a plane is:
    $$ A(x - x_{0}) + B(y - y_{0}) + C(z - z_{0}) = 0 $$

    $A$, $B$, and $C$ are the normal vector components and $x_{0}$, $y_{0}$, and
    $z_{0}$ represent a point on the plane.

    \bigbreak

    Substituting provided values:
    $$ x + 2y + 3z = 5 $$

  \item First, define a vector that runs parallel to the line $x + 2y = 2$. This
    line has slope $-0.5$. Therefore, a parallel vector could take the form:
    $$ \langle -2, 1 \rangle $$

    Next, define a vector that runs from (1, 0) to a point along the line, which
    will be called $Q$.
    $$ \langle q_{1} - 1, q_{2} - 0 \rangle $$

    Since $Q$ has to be along the line, $q_{2}$ can be written in terms of
    $q_{1}$:
    $$ \Big \langle q_{1} - 1, \frac{ 2 - q_{1} }{ 2 } \Big \rangle $$

    Now we can use the dot product to solve for $q_{1}$:
    $$ -2 (q_{1} - 1) + \frac{ 2 - q_{1} }{ 2 } = 0$$
    $$ q_{1} = \frac{ 6 }{ 5 } $$

    Substituting $q_{1}$ into the relation to $q_{2}$ yields $q_{2}$:
    $$ q_{2} = \frac{ 2 - q_{1} }{ 2 } = \frac{ 2 }{ 5 } $$

    As a result, the point along the line that is closest to the point (1, 0)
    is:
    $$ \left( \frac{ 6 }{ 5 }, \frac{ 2 }{ 5 } \right) $$

  \item A plane is simply a set of all vectors that are perpendicular to a
    single vector. As a result, we know that the dot product of any of these
    vectors must be zero. Therefore, we can define two vectors, one normal to
    our plane and one that lies on the plane. This vector can only lie on the
    plane if it satisfies the dot product equally zero.

  \item To find the amount of water flowing through a plane, the plane must
    first have a normal vector. To get the normal vector, take the cross product
    of two vectors along the plane.
    $$ \langle 20, 0, 0 \rangle \times \langle 0, 0, -10 \rangle = \langle 0,
    200, 0 \rangle$$

    Now, using the plane's normal vector, take the dot product with the water's
    flow rate.
    $$ \langle 0, 200, 0 \rangle \cdot \langle 2, 2, 1 \rangle = 400$$

    Note that the area of the plane is taken into account through the dot
    product by definition.

  \item We can compute the location of the tip of the finger by adding the two
    vectors together. The first vector has a length of $2$ and is defined only
    by $\theta_{1}$. This vector takes the form:
    $$ \langle 2 \cos \theta_{1}, 2 \sin \theta_{1} \rangle $$

    The second vector has a length of $1$ and is defined by the sum of both
    $\theta_{1}$ and $\theta_{2}$. This vector take the form:
    $$ \langle \cos(\theta_{1} + \theta_{2}), \sin(\theta_{1} + \theta_{2})
    \rangle $$

    This can be simplified to:
    $$ \langle \cos \theta_{1} \cos \theta_{2} - \sin \theta_{1} \sin
    \theta_{2}, \cos \theta_{1} \sin \theta_{2} + \sin \theta_{1} \cos
    \theta_{2} \rangle $$

    Now, summing these two vectors yields:
    $$ \langle - \sin \theta_{1} \sin \theta_{2} + \cos \theta_{1} \cos \theta_{2} + 2
    \cos \theta_{1}, \cos \theta_{1} \sin \theta_{2} + \sin \theta_{1} \cos
    \theta_{2} + 2 \sin \theta_{1} \rangle $$

    This vector defines the tip of the finger's location.
\end{enumerate}

\section{Linear Approximation}

\begin{enumerate}[1.]
  \setcounter{enumi}{17}
  \item \begin{enumerate}[a.]
    \item Starting with function $f(x, y) = \sin (x y) + x$, the partial
      derivatives can be computed:
      $$ f_{x}(x, y) = y \cos (x y) + 1,\ f_{y}(x, y) = x \cos (x y) $$

      Using these partial derivatives, the linear approximation equation can be
      determined:
      $$ L(x, y) = (\sin (x_{0} y_{0}) + x_{0}) + (y_{0} \cos(x_{0} y_{0}) +
      1)(x - x_{0}) + (x_{0} \cos(x_{0} y_{0}))(y - y_{0})$$

      Substituting for point $(0, \pi / 2)$:
      $$ L(x, y) = \frac{ \pi + 2 }{ 2 } x $$

    \item The equation for the line $l$ can be written:
      $$ 0.039 = x $$

      Since this line is completely vertical, the point closest to $(0, \pi /
      2)$ will take the $y$ coordinate from the point and the $x$ coordinate
      from the line $l$:
      $$ (0.039, 1.571) $$

    \item Using the linear approximation around point (0, 0) instead would
      provide a worse answer as the approximation of $f$ would be made further
      away from the point we are looking for.

  \end{enumerate}
\end{enumerate}

\end{document}
