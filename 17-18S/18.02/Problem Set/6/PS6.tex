\documentclass{article}
\usepackage{tikz}
\usepackage{float}
\usepackage{enumerate}
\usepackage{amsmath}
\usepackage{bm}
\usepackage{indentfirst}
\usepackage{siunitx}
\usepackage[utf8]{inputenc}
\usepackage{graphicx}
\graphicspath{ {Images/} }
\usepackage{float}
\usepackage{mhchem}
\usepackage{chemfig}
\allowdisplaybreaks

\title{ 18.02 Problem Set 6 }
\author{ Robert Durfee }
\date{ April 12, 2018 }

\begin{document}

\maketitle

\section*{Problem 1}

\subsection*{Part A}

%Figure

This integral will be negative.

\subsection*{Part B}

$$\vec{r} = \langle t^2 + 1, t \rangle $$
$$\vec{F}(x, y) = \langle -y, 0 \rangle $$

Using these equations, the constructed integral follows:
$$\int\limits_1^2 \langle -t, 0 \rangle \cdot \langle 2t, 1 \rangle dt =
\frac{-14}{3} $$

\section*{Problem 2}

\subsection*{Part A}

Parameterization of curve $c$ with $t$ going from 0 to 1:
$$ \langle t, -t \rangle $$

From the Fundamental Theorem of Calculus:
$$ f(0, 0) - f(1, -1) = \int\limits_0^1 (-2t^2 + t^2) dt + \int\limits_0^1 (2t^2
- t^2) dt $$
$$ 1 - f(1, -1) = 0 $$
$$ 1 = f(1, -1) $$

\subsection*{Part B}

Parameterization of $c_1$ with $t$ going from 0 to 1:
$$ \langle t, 0 \rangle $$

From the Fundamental Theorem of Calculus:
$$ f(0, 0) - f(1, 0) = \int\limits_0^1 0 dt + \int\limits_0^1 0 dt $$
$$ 1 - f(1, 0) = 0 $$
$$ f(1, 0) = 1 $$

Parameterization of $c_2$ with $t$ going from 0 to 1:
$$ \langle 1, -t \rangle $$

From the Fundamental Theorem of Calculus:
$$ f(1, 0) - f(1,-1) = \int\limit_0^1 0 dt + \int\limits_0^1 (2t + 1) dt $$
$$ 1 - f(1, -1) = 0 $$
$$ f(1, -1) = 1 $$ 

\subsection*{Part C}

Parameterization of $c$ with $t$ going from 0 to 1:
$$ \langle x_1 t, y_1 t \rangle $$

From the Fundamental Theorem of Calculus:
$$ f(0, 0) - f(x_1, y_1) = \int\limits_0^1 ( 2 y_1 x_1^2 t^2 + x_1 y_1^2 t^2 )
dt + \int\limits_0^1 (2 x_1 y_1^2 t^2 + y_1 x_1^2 t^2) dt $$
$$ 1 - f(x_1, y_1) = x_1 y_1 (x_1 + y_1) $$
$$ f(x_1, y_1) = 1 - x_1 y_1 (x_1 + y_1) $$

\subsection*{Part D}

Parameterization of $c_1$ with $t$ going from 0 to $x_1$:
$$ \langle t, 0 \rangle $$

From the Fundamental Theorem of Calculus:
$$ f(0, 0) - f(x_1, 0) = \int\limits_0^{x_1} 0 dt + \int\limits_0^{x_1} 0 dt $$
$$ 1 - f(x_1,0) = 0 $$
$$ f(x_1, 0) = 1 $$

Parameterization of $c_2$ with $t$ going from 0 to $y_1$:
$$ \langle x_1, t \rangle $$

From the Fundamental Theorem of Calculus:
$$ f(x_1, 0) - f(x_1, y_1) = \int\limits_0^{y_1} 0 dt + \int\limits_0^{y_1}
(2 x_1 t + x_1^2 ) dt $$
$$ 1 - f(x_1, y_1) = x_1 y_1 (y_1 + x_1) $$
$$ f(x_1, y_1) = 1 - x_1 y_1 (y_1 + x_1) $$

\subsection*{Part E}

Given that $f(0, 0) = 0$, the function $f$ is:
$$ f(x, y) = x y (x + y)$$

The gradient of this function:
$$ \nabla f(x, y) = \langle 2xy + y^2, 2xy + x^2 \rangle $$

\section*{Problem 3}

A function with gradient $\nabla f(x, y) = \langle 0, y \rangle$:
$$ f(x, y) = \frac{y^2}{2} $$

\section*{Problem 4}

\subsection*{Part A}

$$ f(0, 0.5) > f(0,0) $$

\subsection*{Part B} 

The approximate location of the minimum of $f$ is $(-0.5, -0.5)$.

\subsection*{Part C}

The approximate location of the maximum of $f$ is $(0.5, 0.5)$.

\section*{Problem 5}

Picture 2 is not a function because there exists a closed curve such that
$\int_c \nabla f \cdot d\vec{r} \neq 0$. (Namely, an oval centered at (0,0).)

\section*{Problem 6}

\subsection*{Part A}

Curve $c$ is oriented counter clockwise. Thus, $dx$ is negative for the top edge
and positive for the bottom edge. Since the top edge has a higher $y$ value
(which has a negative $dx$), the result is net negative.

\subsection*{Part B}


\end{document}

