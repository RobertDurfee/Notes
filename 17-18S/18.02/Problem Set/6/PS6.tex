\documentclass{article}
\usepackage{tikz}
\usepackage{float}
\usepackage{enumerate}
\usepackage{amsmath}
\usepackage{bm}
\usepackage{indentfirst}
\usepackage{siunitx}
\usepackage[utf8]{inputenc}
\usepackage{graphicx}
\graphicspath{ {Images/} }
\usepackage{float}
\usepackage{mhchem}
\usepackage{chemfig}
\allowdisplaybreaks

\title{ 18.02 Problem Set 6 }
\author{ Robert Durfee }
\date{ April 13, 2018 }

\begin{document}

\maketitle

\section*{Problem 1}

\subsection*{Part A}

%Figure

This integral will be negative.

\subsection*{Part B}

$$\vec{r} = \langle t^2 + 1, t \rangle $$
$$\vec{F}(x, y) = \langle -y, 0 \rangle $$

Using these equations, the constructed integral follows:
$$\int\limits_1^2 \langle -t, 0 \rangle \cdot \langle 2t, 1 \rangle dt =
\frac{-14}{3} $$

\section*{Problem 2}

\subsection*{Part A}

Parameterization of curve $c$ with $t$ going from 0 to 1:
$$ \langle t, -t \rangle $$

From the Fundamental Theorem of Calculus:
$$ f(0, 0) - f(1, -1) = \int\limits_0^1 (-2t^2 + t^2) dt + \int\limits_0^1 (2t^2
- t^2) dt $$
$$ 1 - f(1, -1) = 0 $$
$$ 1 = f(1, -1) $$

\subsection*{Part B}

Parameterization of $c_1$ with $t$ going from 0 to 1:
$$ \langle t, 0 \rangle $$

From the Fundamental Theorem of Calculus:
$$ f(0, 0) - f(1, 0) = \int\limits_0^1 0 dt + \int\limits_0^1 0 dt $$
$$ 1 - f(1, 0) = 0 $$
$$ f(1, 0) = 1 $$

Parameterization of $c_2$ with $t$ going from 0 to 1:
$$ \langle 1, -t \rangle $$

From the Fundamental Theorem of Calculus:
$$ f(1, 0) - f(1,-1) = \int\limits_0^1 0 dt + \int\limits_0^1 (2t + 1) dt $$
$$ 1 - f(1, -1) = 0 $$
$$ f(1, -1) = 1 $$ 

\subsection*{Part C}

Parameterization of $c$ with $t$ going from 0 to 1:
$$ \langle x_1 t, y_1 t \rangle $$

From the Fundamental Theorem of Calculus:
$$ f(0, 0) - f(x_1, y_1) = \int\limits_0^1 ( 2 y_1 x_1^2 t^2 + x_1 y_1^2 t^2 )
dt + \int\limits_0^1 (2 x_1 y_1^2 t^2 + y_1 x_1^2 t^2) dt $$
$$ 1 - f(x_1, y_1) = x_1 y_1 (x_1 + y_1) $$
$$ f(x_1, y_1) = 1 - x_1 y_1 (x_1 + y_1) $$

\subsection*{Part D}

Parameterization of $c_1$ with $t$ going from 0 to $x_1$:
$$ \langle t, 0 \rangle $$

From the Fundamental Theorem of Calculus:
$$ f(0, 0) - f(x_1, 0) = \int\limits_0^{x_1} 0 dt + \int\limits_0^{x_1} 0 dt $$
$$ 1 - f(x_1,0) = 0 $$
$$ f(x_1, 0) = 1 $$

Parameterization of $c_2$ with $t$ going from 0 to $y_1$:
$$ \langle x_1, t \rangle $$

From the Fundamental Theorem of Calculus:
$$ f(x_1, 0) - f(x_1, y_1) = \int\limits_0^{y_1} 0 dt + \int\limits_0^{y_1}
(2 x_1 t + x_1^2 ) dt $$
$$ 1 - f(x_1, y_1) = x_1 y_1 (y_1 + x_1) $$
$$ f(x_1, y_1) = 1 - x_1 y_1 (y_1 + x_1) $$

\subsection*{Part E}

Given that $f(0, 0) = 0$, the function $f$ is:
$$ f(x, y) = x y (x + y)$$

The gradient of this function:
$$ \nabla f(x, y) = \langle 2xy + y^2, 2xy + x^2 \rangle $$

\section*{Problem 3}

A function with gradient $\nabla f(x, y) = \langle 0, y \rangle$:
$$ f(x, y) = \frac{y^2}{2} $$

\section*{Problem 4}

\subsection*{Part A}

$$ f(0, 0.5) > f(0,0) $$

\subsection*{Part B} 

The approximate location of the minimum of $f$ is $(-0.5, -0.5)$.

\subsection*{Part C}

The approximate location of the maximum of $f$ is $(0.5, 0.5)$.

\section*{Problem 5}

Picture 2 is not a function because there exists a closed curve such that
$\int_c \nabla f \cdot d\vec{r} \neq 0$. (Namely, an oval centered at (0,0).)

\section*{Problem 6}

\subsection*{Part A}

Curve $c$ is oriented counter clockwise. Thus, $dx$ is negative for the top edge
and positive for the bottom edge. Since the top edge has a higher $y$ value
(which has a negative $dx$), the result is net negative.

\subsection*{Part B}

The line integral for the top edge:
$$ \int\limits_{a_2}^{a_1} b_2 dx = -b_2(a_2 - a_1) $$

The line integral for the bottom edge:
$$ \int\limits_{a_1}^{a_2} b_1 dx = b_1(a_2 - a_1) $$

\section*{Problem 7}

The linear approximation for $P(x, y)$:
$$ P(x, y) \approx e(x + y - 1) $$

The linear approximation for $Q(x, y)$:
$$ Q(x, y) \approx 2x - 2 + y $$

As a result, the integral now becomes:
$$ e \left[ \int_c x dx + \int_c y dx - \int_c dx \right] + \int_c 2x dy - 2
\int_c dy + \int_c y dy $$

Which simplifies (using the provided table) to:
$$ (2 - e)(\Delta x \Delta y) \approx \int_c x e^{xy} dx + x^2 y dy $$

\section*{Problem 8}

\subsection*{Part A}

Parameterization of $s$ with $t$ going from 0 to 1:
$$ \langle t, 0 \rangle $$

Parameterixation of $w$ with $t$ going from 0 to 1:
$$ \langle t, 1/10 \sin \left( 100 t \right) \rangle $$

The straight integral becomes:
$$ \int\limits_0^1 dt = 1 $$

The wiggly integral becomes:
$$ \int\limits_0^1 (1 + 1/10 \sin \left( 100 t \right) ) dt \approx 1.00012 $$

Note that these two results are very similar.

\subsection*{Part B}

The length of the straight segment is (clearly) 1.

\bigbreak

The length of the wiggly segment is computed using the integral:
$$ \int\limits_0^1 \sqrt{1 + \left( 10 \cos \left( 100 t \right) \right)^2 } dt
\approx 6 $$

Note that these two results are very different!

\section*{Problem 9}

The correct integral for scoring hikers is:
$$ \int_c x dy $$

This is correct because as the hiker moves up ($dy$), the score will increase.
Additionally, as the hiker moves right ($x$, into the smoother cliff side), the
score will also increase.

\section*{Problem 10}

The work for stretching evenly:
$$ W = \frac{L_{new}^2}{L_{old}} - L_{old} $$

The original small section of the rubber band is defined from $t$ to $t + \Delta
t$. The corresponding small section of the stretched rubber band is defined from
$(x(t), y(t))$ to $(x(t + \Delta t), y(t + \Delta t))$. 

The length of the original small section is (clearly) $\Delta t$. The length of
the corresponding small section in the stretched rubber band, through linear
approximation is:
$$ L_{new} = \sqrt{ x'(t)^2 + y'(t)^2 } \Delta t $$

Substituting these values into the work equation:
$$ W = (x'(t)^2 + y'(t)^2 - 1) \Delta t $$

Substituting values of derivatives:
$$ W = ((1 + 2t)^2 + (1 - 2t)^2 - 1) \Delta t $$

Integrating over $t$ (from $0$ to $1$):
$$ W = \int\limits_0^1 ((1 + 2t)^2 + (1 - 2t)^2 - 1) dt = \frac{11}{3} $$

\section*{Problem 11}

Computing the vector between $\langle 2, 3, 4 \rangle$ and $\langle 1, 2, 3
\rangle$:
$$ \langle 2, 3, 4 \rangle - \langle 1, 2, 3 \rangle = \langle 1, 1, 1 \rangle$$

Adding this vector to the vector for the other provided point will determine the
unknown point as the sides must be parallel:
$$ \langle 1, 4, 5 \rangle + \langle 1, 1, 1 \rangle = \langle 2, 5, 6 \rangle
$$

\section*{Problem 12}

\subsection*{Part A}

First, calculate the resulting points after applying parameterization using
linear approximation:
$$ \vec{r}(1, 1) = \langle 1, 1, -1 \rangle $$
$$ \vec{r}(1.01, 1) = \vec{r}(1,1) + \frac{\partial \vec{r}}{\partial
u}(1,1)(0.01) = \langle 1.01, 1, -0.98 \rangle $$
$$ \vec{r}(1, 1.01) = \vec{r}(1, 1) + \frac{\partial \vec{r}}{\partial v}(1, 1)
(0.01) = \langle 1, 1.01, -1.04 \rangle $$
$$ \vec{r}(1.01, 1.01) = \vec{r}(1,1) + \frac{\partial \vec{r}}{\partial
u}(1,1)(0.01) + \frac{\partial \vec{r}}{\partial v}(1,1)(0.01) = \langle 1.01,
1.01, -1.02 \rangle $$

Next, finding the difference between vectors that describe adjacent points:
$$ \vec{r}(1.01, 1) - \vec{r}(1,1) = \langle 0.01, 0, 0.02 \rangle $$
$$ \vec{r}(1.01, 1.01) - \vec{r}(1.01, 1) = \langle 0, 0.01, -0.04 \rangle $$
$$ \vec{r}(1, 1.01) - \vec{r}(1.01, 1.01) = \langle -0.01, 0, -0.02 \rangle $$
$$ \vec{r}(1, 1) - \vec{r}(1, 1.01) = \langle 0, -0.01, 0.04 \rangle $$

Last, calculate the normal of each of these computed vectors:
$$ \vert \langle 0.01, 0, 0.02 \rangle \vert = 0.0224 $$
$$ \vert \langle 0, 0.01, -0.04 \rangle \vert = 0.0412 $$
$$ \vert \langle -0.01, 0, -0.02 \rangle \vert = 0.0224 $$
$$ \vert \langle 0, -0.01, 0.04 \rangle \vert = 0.0412 $$

\subsection*{Part B}

Finding the difference between vectors that describe non-adjacent points:
$$ \vec{r}(1.01, 1) - \vec{r}(1, 1.01) = \langle 0.01, -0.01, 0.06 \rangle $$
$$ \vec{r}(1.01, 1.01) - \vec{r}(1, 1) = \langle 0.01, 0.01, -0.02 \rangle $$

Calculating the normal of each of these computed vectors:
$$ \vert \langle 0.01, -0.01, 0.06 \rangle \vert = 0.0616 $$
$$ \vert \langle 0.01, 0.01, -0.02 \rangle \vert = 0.0245 $$

\subsection*{Part C}

Computing the dot product of the vectors corresponding to the sides adjacent to
point $\vec{r}(1,1)$:
$$ \langle 0.01, 0, 0.02 \rangle \cdot \langle 0, 0.01, -0.04 \rangle = -0.0008
$$

Computing the product of the normals of each vector:
$$ \vert \langle 0.01, 0, 0.02 \rangle \vert \vert \langle 0, 0.01, -0.04
\rangle \vert = 0.000922 $$

Setting the quotient equal to $\cos \theta$ (through dot product):
$$ \cos \theta = \frac{ -0.0008 }{ 0.000922 } $$

Solving for $\theta$:
$$ \theta \approx 150^\circ $$

\subsection*{Part D}

Taking the vector that describes one side:
$$ \vec{r}(1, 1.01) - \vec{r}(1,1) = \langle 0, 0.01, -0.04 \rangle $$

Adding it to the point across from the side to see if resulting point is
approximately equal to the known point:
$$ \langle 0, 0.01, -0.04 \rangle + \vec{r}(1.01, 1) = \langle 1.01, 1.01, -1.02
\rangle $$

This is equal to the point $\vec{r}(1.01, 1.01)$. Therefore, the shape is
approximately a parallelogram. 

\section*{Problem 13}

\subsection*{Part A}

The partial derivatives of the parameterization:
$$ \frac{\partial \vec{r}}{\partial u} = \langle \cos v, \sin v, 0 \rangle $$
$$ \frac{\partial \vec{r}}{\partial v} = \langle - u \sin v, u \cos v, 1 \rangle
$$

Substituting for $u = 3$ and $v = 4$,
$$ \frac{\partial \vec{r}}{\partial u} = \langle -0.654, -0.757, 0 \rangle $$
$$ \frac{\partial \vec{r}}{\partial v} = \langle 2.270, -1.961, 1 \rangle $$

Using linear approximation, the adjacent sides can be calculated:
$$ \vec{r}(3 + \Delta u, 4) - \vec{r}(3, 4) =\langle -0.654, -0.757, 0
\rangle\Delta u $$
$$ \vec{r}(3, 4 + \Delta v) - \vec{r}(3, 4) =  \langle 2.270, -1.961, 1 \rangle
\Delta v $$

Computing the cosine of the angle between using the dot product formula:
$$ \frac{\langle -0.654, -0.757, 0 \rangle \cdot \langle 2.270, -1.961, 1 \rangle}{\vert \langle -0.654, -0.757, 0
\rangle\vert \vert\langle 2.270, -1.961, 1 \rangle \vert} = -0.0000325 $$

This is approximately zero, therefore these sides roughly form a right angle.

\subsection*{Part B}

Taking the normal of the computed sides in Part A:
$$ \vert \langle -0.654, -0.757, 0 \rangle \vert \Delta u = 1.00038 \Delta u $$
$$ \vert \langle 2.270, -1.961, 1 \rangle \vert \Delta v = 3.16203 \Delta v $$

\subsection*{Part C}

Multiplying length of each side:
$$ 1.00038 \Delta u \cdot 3.16203 \Delta v = 3.163 \Delta u \Delta v $$

\subsection*{Part A'}

Taking the dot product of the two partial derivatives (given that they represent
two adjacent sides through linear approximation):
$$ \langle \cos v_0, \sin v_0, 0 \rangle \cdot \langle -u_0 \sin v_0, u_0 \cos
v_0, 1 \rangle = 0 $$

Therefore, these sides roughly form a right angle.

\subsection*{Part B'}

Computing the normal of each of the side vectors:
$$  \langle \cos v_0, \sin v_0, 0 \rangle = \sqrt{ \sin^2 v_0 + \cos^2 v_0 }
\Delta u $$
$$ \langle -u_0 \sin v_0, u_0 \cos v_0, 1 \rangle = \sqrt{ u_0^2 \sin^2 v_0 +
u_0^2 \cos^2 + 1} \Delta v $$

\subsection*{Part C'}

Multiplying these two lengths to find area:
$$ \sqrt{ \sin^2 v_0 + \cos^2 v_0 } \Delta u \cdot \sqrt{ u_0^2 \sin^2 v_0 +
u_0^2 \cos^2 + 1} \Delta v = \sqrt{u_0^2 + 1} \Delta u \Delta v $$

\subsection*{Part D}

Using the result from Part C' to construct the integral for total surface area:
$$ \int\limits_0^{10} \int\limits_{-10}^{10} \sqrt{u^2 + 1} du dv $$

\subsection*{Part E}

The result of this integral is approximately $10^3$.

\section*{Problem 14}

\subsection*{Part A}

The partial derivatives of the parameterization:
$$ \frac{\partial \vec{r}}{\partial \theta} = \langle - \sin \theta (3 + 2 \cos
\phi), \cos \theta (3 + 2 \cos \phi), 0 \rangle $$
$$ \frac{\partial \vec{r}}{\partial \phi} = \langle -2 \cos \theta \sin \phi, -2
\sin \theta \sin \phi, 2 \sin \phi \rangle $$

Computing the dot product of these vectors to confirm the small section is
approximately a rectangle:
$$ \langle - \sin \theta (3 + 2 \cos \phi), \cos \theta (3 + 2 \cos \phi), 0
\rangle \cdot  \langle -2 \cos \theta \sin \phi, -2 \sin \theta \sin \phi, 2 \sin
\phi \rangle = 0 $$

Computing side lengths:
$$  \vert \langle - \sin \theta (3 + 2 \cos \phi), \cos \theta (3 + 2 \cos \phi), 0
\rangle \vert \Delta \theta $$
$$ \vert \langle -2 \cos \theta \sin \phi, -2 \sin \theta \sin \phi, 2 \sin \phi
\rangle \vert \Delta \phi $$

Multiplying to determine area:
\begin{equation*}
  \begin{split}
    \vert \langle - \sin \theta (3 + &2 \cos \phi), \cos \theta (3 + 2 \cos
    \phi), 0 \rangle \vert \Delta \theta \\ \cdot & \vert \langle -2 \cos \theta \sin
    \phi, -2 \sin \theta \sin \phi, 2 \sin \phi \rangle \vert \Delta \phi \\ = & (6 +
    4 \cos \phi) \Delta \theta \Delta phi
  \end{split}
\end{equation*}

Summing this small area over the entire surface:
$$\int\limits_0^{2 \pi} \int\limits_0^{2 \pi} (6 + 4 \cos \phi) d\theta d\phi =
236.871 $$

\subsection*{Part B}

The new curve $C$ is defined with $\phi$ going from $\pi/2$ to $3\pi/2$. It is
then rotated completely around, so $\theta$ remains from $0$ to $2 \pi$. The
integral becomes:
$$ \int\limits_{\pi/2}^{3 \pi/ 2} \int\limits_0^{2 \pi} (6 + 4 \cos \phi)
d\theta d\phi = 68.170 $$

\subsection*{Part C}

Solving for $\phi$ when $z = 1$:
$$ 2 \sin \phi = 1 $$
$$ \phi = \pi/6, 5\pi/6 $$

The new curve $C$ is defined with $\phi$ going from $\pi/6$ to $5\pi/6$. It is
also rotated completely around, so $\theta$ remains from $0$ to $2 \pi$. The
integral becomes:
$$ \int\limits_{\pi/6}^{5\pi/6} \int\limits_0^{2 \pi} (6 + 4 \cos \phi) d\theta
d\phi = 78.957 $$

\end{document}

