\documentclass{article}
\usepackage{tikz}
\usepackage{float}
\usepackage{enumerate}
\usepackage{amsmath}
\usepackage{bm}
\usepackage{indentfirst}
\usepackage{siunitx}
\usepackage[utf8]{inputenc}
\usepackage{graphicx}
\graphicspath{ {Images/} }
\usepackage{float}
\usepackage{mhchem}
\usepackage{chemfig}
\allowdisplaybreaks

\title{ 18.02 Problem Set 1 }
\author{ Robert Durfee }
\date{ February 16, 2018 }

\begin{document}

\maketitle

\section{ Level Curves and Partial Derivatives }

\begin{enumerate}[1.]
  \item \begin{enumerate}[a.]
      \item $f_{x}(0, 1)$ is positive.
      \item $f_{x}(0, -1)$ is negative.
      \item $f_{y}(1, 0)$ is larger that $f_{y}(0, 0)$.
      \item $f_{y}(1, 0) \approx 3$.
    \end{enumerate}
  \item \begin{enumerate}[a.]
      \item $$ f_{x}(x, y) = -e^{y - x} $$
        $$ f_{x}(3, 2) = -\frac{ 1 }{ e } $$
      \item $$ g'(x) = -e^{2 - x} $$
        $$ g'(3) = -\frac{ 1 }{ e }$$
      \item When computing the partial derivative of $f$ with respect to $x$, we
        treat $y$ as a constant. When solving for $f_{x}(3, 2)$, we set $y = 2$
        and $x = 3$.  In $g$, the variable $y$ is replaced with the constant 2.
        As a result, when solving for $g'(3)$, the answer is the same.
    \end{enumerate}
  \item \begin{enumerate}[a.]
      \item $$ f_{x}(x, y) = y^{2} $$
        $$ f_{y}(x, y) = 2xy + 1 $$
      \item $$ f_{x}(x, y) = \frac{ y }{ (x + y)^{2} } $$
        $$ f_{y}(x, y) = -\frac{ x }{ (x + y)^{2} } $$
      \item $$ f_{x}(x, y) = y^{2} \cos(xy^{2}) $$
        $$ f_{y}(x, y) = 2xy \cos(xy^{2}) $$
    \end{enumerate}
  \item Level curves for $f(x,y) = y^{2} - x^{2}$:

    \begin{figure}[H]
      \centering
      \includegraphics[scale=0.60]{"LevelCurveSketch"}
      \caption{Level Curve Sketch}
    \end{figure}

  \item \begin{enumerate}[a.]
      \item $$ f_{x}(x, y) = -2x $$
        $$ f_{y}(x, y) = 2y $$
      \item $f_{y}(1, -1)$ is negative. This can be seen in the level curve
        graph by starting at point $(1, -1)$ and keeping $x$ constant as you
        move one unit in the positive $y$ direction. Since the curves go down in
        height, the partial derivative is negative.
      \item $f_{y}(2, 2)$ is larger than $f_{y}(1, 1)$. This can be seen on the
        picture by starting at $(1, 1)$ and $(2, 2)$ can advancing one unit in
        the $y$ direction while keeping $x$ constant. When the more level curves
        crossed will indicate a larger difference in height and, therefore, the
        larger partial derivative.
      \item The points are shown in \textit{Figure 1}. These points are special
        because they represent a change in direction in the level curve. A
        minimum/maximum, essentially.
    \end{enumerate}
  \item \begin{enumerate}[a.]
    \item $$ f(x, y) = 10 - y^{2} + x $$
      $$ f_{x}(x, y) = 1,\ f_{y}(x, y) = -2y $$
    \item $$ f(x, y) = 10 - xy + y $$
      $$ f_{x}(x, y) = -y,\ f_{y}(x, y) = -x + 1 $$
    \item $$ f(x, y) = 10 - x^{2} + xy $$
      $$ f_{x}(x, y) = -2x + y,\ f_{y}(x, y) = x $$

      \bigbreak

      Using the partial derivatives computed above, we are looking for a partial
      derivative with respect to $y$ that is negative when $x$ is positive and
      is positive when $x$ is negative. The only equation that has this property
      is equation B.
  \end{enumerate}
\end{enumerate}

\section{Linear Approximation}

\begin{enumerate}[1.]
  \setcounter{enumi}{6}
  \item Starting with $f(x, y) = \sin(xy) + 2e^{x}$, the partial derivatives can
    be computed:
    $$ f_{x}(x,y) = y \cos(xy) + 2e^{x}, \ f_{y}(x,y) = x \cos(xy) $$

    These can then be substituted into the general linear approximation
    equation:
    $$ L(x,y) = \sin(x_{0}y_{0}) + 2e^{x_{0}} + (y_{0} \cos(x_{0}y_{0}) +
    2e^{x_{0}})(x - x_{0}) + (x_{0} \cos(x_{0}y_{0}))(y - y_{0}) $$

    Substituting $x_{0}=0$ and $y_{0}=0$:
    $$ L(x,y) = 2 + 2x $$

    Solving for the approximation at (0.1, 0.2):
    $$ L(0.1, 0.2) = 2.2 $$

  \item Starting with $P(W, M) = W \sqrt{M}$, the partial derivatives can be
    computed:
    $$ P_{W}(W, M) = \sqrt{M}, \ P_{M}(W, M) = \frac{ W }{ 2 \sqrt{M} } $$

    These can then be substituted into the general linear approximation
    equation:
    $$ L(W_{0} + \Delta W, M_{0} + \Delta M) = W_{0} \sqrt{M_{0}} + \sqrt{M_{0}}
    \Delta W + \frac{ W_{0} }{ 2 \sqrt{M_{0}}} \Delta M $$

    Substituting $W_{0} = 3$ and $M_{0} = 4$:
    $$ L(3 + \Delta W, 4 + \Delta M) = 6 + 2 \Delta W + 0.75 \Delta M $$

    \bigbreak

    \begin{enumerate}[a.]
      \item Plugging in $\Delta W = 0.1$:
        $$ L(3.1, 4) = 6.2 $$

      \item Plugging in $\Delta M = 0.2$:
        $$ L(3, 4.2) = 6.15 $$

      \item Plugging in $\Delta W = -0.1$ and $\Delta M = 0.4$:
        $$ L(2.9, 4.4) = 6.1 $$
    \end{enumerate}

    \bigbreak

    From this information, we can determine that the best option for the company
    is option A.
  \item \begin{enumerate}[a.]
    \item Starting with $f(x,y) = y^{2} - x^{2}$, the partial derivatives can be
      computed:
      $$ f_{x}(x, y) = -2x, \ f_{y}(x, y) = 2y $$

      These can then be substituted into the general linear approximation
      equation:
      $$ L(x, y) = 4y - 4x $$
    \item Linear approximation level curves for $L(x,y) = 4y - 4x$:

      \begin{figure}[H]
        \centering
        \includegraphics[scale=0.60]{"LinearLevelCurveSketch"}
        \caption{Linear Level Curve Sketch}
      \end{figure}
    \end{enumerate}

  \item \begin{enumerate}[a.]
      \item Starting with the basic linear approximation equation:
        $$ f(x,y) \approx f(x_{0}, y_{0}) + f_{x}(x_{0},y_{0})(x - x_{0}) + f_{y}(x_{0},
        y_{0})(y - y_{0})$$

        Rearranging to get all the $x$, $y$, and $z$ terms together:
        $$ f(x,y) - f(x_{0},y_{0}) \approx f_{x}(x_{0},y_{0})(x - x_{0}) + f_{y}(x_{0},
        y_{0})(y - y_{0}) $$

        We can redefine the differences between terms:
        $$ \Delta z \approx f_{x}(x_{0},y_{0}) \Delta x + f_{y}(x_{0},y_{0}) \Delta
        y $$

        Using $f(x, y) = \frac{ x }{ x + y }$, the partial derivatives can be
        calculated:
        $$ f_{x}(x, y) = \frac{ y }{ (x + y)^{2} }, \ f_{y}(x, y) = -\frac{ x }{ (x
        + y)^{2} } $$

        Now all of the values can be substituted into the equation derived above:
        $$ \Delta z \approx \frac{ 9 }{ 10^{2} }(0.01) - \frac{ 1 }{ 10^{2} }(0.01)
        \approx 0.0008 $$

        This can be rewritten into scientific notation:
        $$ \Delta z = 8 \cdot 10^{-4} $$

        As a result, the magnitude is on the order of $10^{-4}$.

      \item Now, using $\Delta x = 0.01$ and $\Delta y = 0.03$:
        $$ \Delta z \approx \frac{ 9 }{ 10^{2} }(0.01) - \frac{ 1 }{ 10^{2}
        }(0.03) \approx 0.0006 $$

        Using $\Delta x = 0.03$ and $\Delta y = 0.01$
        $$ \Delta z \approx \frac{ 9 }{ 10^{2} }(0.03) - \frac{ 1 }{ 10^{2}
        }(0.01) \approx 0.0026 $$

        As a result, the plan where $\Delta x = 0.01$ and $\Delta y = 0.03$ is
        better because the $\Delta z$ is the lowest.
    \end{enumerate}

  \item \begin{enumerate}[a.]
      \item Sketch of $y = x^{2}$ and $(x-1)^{2} + (y-1)^{2} = 1$:

        \begin{figure}[H]
          \centering
          \includegraphics[scale=0.60]{"PAndCSketch"}
          \caption{Sketch of P and C}
        \end{figure}

      \item Using the sketch, a reasonable choice for the intersection is
        (1.4, 1.9).

      \item Using $f(x,y) = y - x^{2}$ and $g(x,y) = (x-1)^{2} + (y-1)^{2}$, two
        sets of partial derivatives can be derived:
        $$ f_{x}(x,y) = -2x, \ f_{y}(x,y) = 1 $$
        $$ g_{x}(x,y) = 2(x - 1), \ g_{y}(x,y) = 2(y - 1) $$

        These partial derivatives can be substituted into the linear approximation
        equation:
        $$ L_{f}(x,y) = y_{0} - x_{0}^{2} - 2x_{0}(x - x_{0}) + (y - y_{0}) $$
        $$ L_{g}(x,y) = (x_{0} - 1)^{2} + (y_{0} - 1)^{2} + 2(x_{0} - 1)(x -
        x_{0}) + 2(y_{0} - 1)(y - y_{0}) $$

        Substituting the best guess for the intersection at (1.4, 1.9):
        $$ L_{f}(x,y) = -2.8x + y + 1.96 $$
        $$ L_{g}(x,y) = 0.8x + 1.8y - 3.57 $$

      \item Setting these two equal to zero and one, respectively, and solving for
        $x$ and $y$ yields:
        $$ x = 1.39, \ y = 1.92 $$

    \end{enumerate}

\end{enumerate}

\section{Maxima and Minima}

\begin{enumerate}[1.]
  \setcounter{enumi}{11}
  \item \begin{enumerate}[a.]
      \item Starting with $f(x,y) = x(1-x^{2}-y^{2})$, the partial derivatives can
        be calculated:
        $$ f_{x}(x,y) = 1 - 3x^{2} -y^{2}, \ f_{y}(x,y) = -2yx $$

        Setting these partial derivatives equal to zero will find the critical
        points:
        $$ (0, -1),\ (0, 1),\ (-1/\sqrt{3}, 0),\ (1/\sqrt{3}, 0) $$
      \item These points evaluate to:
        $$ 0,\ 0,\ -2/3\sqrt{3},\ 2/3\sqrt{3} $$

        Next, the boundary must be checked for critical points. Solving $x^{2} +
        y^{2} = 1$ for $y^{2}$ can easily be substituted into $f(x,y) = x(1-x^{2}
        - y^{2})$:
        $$ f(x) = 0 $$

        As a result, all points are zero, so there are no critical points along
        the boundary. Also, the value of zero is neither a minimum or a maximum
        within this region so the minimum is $-0.38$ located at (-0.58, 0) and the
        maximum is $0.38$ located at (0.58, 0).

    \end{enumerate}

  \item Starting with $f(x,y) = x -2y$, the partial derivatives can be computed:
    $$ f_{x}(x,y) = 1, \ f_{y}(x,y) = -2 $$

    This shows that there are no critical points along the plane. As a result,
    all functions along the boundary will be linear as well, with no critical
    points. The only point that need to be tested are the corners of the
    triangle:
    $$ f(1, 0) = 1, \ f(1,1) = -1, f(2,1) = 0 $$

    As a result, the corner at (1, 0) is the maximum within the region
    specified.

  \item From the constraints given, two equations can be derived:
    $$ 48 = xyz, \ C = 6xy + 8xz + 8yz $$

    Solving for $z$ and substituting into the cost function results in a two
    variable function which can be minimized:
    $$ C(x, y) = 6xy + 384/y + 384/x $$

    From this function, the partial derivatives follow:
    $$ C_{x}(x,y) = 6(y - 64/x^{2}), \ C_{y}(x,y) = 6(x - 64/y^{2}) $$

    Setting these partial derivatives equal to zero results in the following
    dimensions:
    $$ x = 4, \ y = 4, \ z = 3 $$

  \item At a minimum or maximum within a region, there exists a point that is
    completely flat. In order for a point on a surface to be completely flat,
    that point must have partial derivatives equal to zero in all directions.

\end{enumerate}

\end{document}

