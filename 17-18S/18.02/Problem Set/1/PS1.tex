\documentclass{article}
\usepackage{tikz}
\usepackage{float}
\usepackage{enumerate}
\usepackage{amsmath}
\usepackage{bm}
\usepackage{indentfirst}
\usepackage{siunitx}
\usepackage[utf8]{inputenc}
\usepackage{graphicx}
\graphicspath{ {Images/} }
\usepackage{float}
\usepackage{mhchem}
\usepackage{chemfig}
\allowdisplaybreaks

\title{ 18.02 Problem Set 1 }
\author{ Robert Durfee }
\date{ February 7, 2018 }

\begin{document}

\maketitle

\section{ Level Curves and Partial Derivatives }

\begin{enumerate}[1.]
  \item \begin{enumerate}[a.]
      \item $f_{x}(0, 1)$ is positive.
      \item $f_{x}(0, -1)$ is negative.
      \item $f_{y}(1, 0)$ is larger that $f_{y}(0, 0)$.
      \item $f_{y}(1, 0) \approx 3$.
    \end{enumerate}
  \item \begin{enumerate}[a.]
      \item $$ f_{x}(x, y) = -e^{y - x} $$
        $$ f_{x}(3, 2) = -\frac{ 1 }{ e } $$
      \item $$ g'(x) = -e^{2 - x} $$
        $$ g'(3) = -\frac{ 1 }{ e }$$
      \item When computing the partial derivative of $f$ with respect to $x$, we
        treat $y$ as a constant. When solving for $f_{x}(3, 2)$, we set $y = 2$
        and $x = 3$.  In $g$, the variable $y$ is replaced with the constant 2.
        As a result, when solving for $g'(3)$, the answer is the same.
    \end{enumerate}
  \item \begin{enumerate}[a.]
      \item $$ f_{x}(x, y) = y^{2} $$
        $$ f_{y}(x, y) = 2xy + 1 $$
      \item $$ f_{x}(x, y) = \frac{ y }{ (x + y)^{2} } $$
        $$ f_{y}(x, y) = -\frac{ x }{ (x + y)^{2} } $$
      \item $$ f_{x}(x, y) = y^{2} \cos(xy^{2}) $$
        $$ f_{y}(x, y) = 2xy \cos(xy^{2}) $$
    \end{enumerate}
  \item \begin{enumerate}[a.]
      \item
      \item
    \end{enumerate}
  \item \begin{enumerate}[a.]
      \item $$ f_{x}(x, y) = -2x $$
        $$ f_{y}(x, y) = 2y $$
      \item $f_{y}(1, -1)$ is negative. This can be seen in the level curve
        graph by starting at point $(1, -1)$ and keeping $x$ constant as you
        move one unit in the positive $y$ direction. Since the curves go down in
        height, the partial derivative is negative.
      \item $f_{y}(2, 2)$ is larger than $f_{y}(1, 1)$. This can be seen on the
        picture by starting at $(1, 1)$ and $(2, 2)$ can advancing one unit in
        the $y$ direction while keeping $x$ constant. When the more level curves
        crossed will indicate a larger difference in height and, therefore, the
        larger partial derivative.
      \item
    \end{enumerate}
  \item \begin{enumerate}[a.]
    \item $$ f(x, y) = 10 - y^{2} + x $$
      $$ f_{x}(x, y) = 1,\ f_{y}(x, y) = -2y $$
    \item $$ f(x, y) = 10 - xy + y $$
      $$ f_{x}(x, y) = -y,\ f_{y}(x, y) = -x + 1 $$
    \item $$ f(x, y) = 10 - x^{2} + xy $$
      $$ f_{x}(x, y) = -2x + y,\ f_{y}(x, y) = x $$
  \end{enumerate}
\end{enumerate}

\section{Linear Approximation}

\begin{enumerate}[1.]
  \item
  \item
\end{enumerate}

\end{document}

