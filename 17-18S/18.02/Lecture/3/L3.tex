\documentclass{article}
\usepackage{tikz}
\usepackage{float}
\usepackage{enumerate}
\usepackage{amsmath}
\usepackage{bm}
\usepackage{indentfirst}
\usepackage{siunitx}
\usepackage[utf8]{inputenc}
\usepackage{graphicx}
\graphicspath{ {Images/} }
\usepackage{float}
\usepackage{mhchem}
\usepackage{chemfig}
\allowdisplaybreaks

\title{ 18.02 Lecture 3 }
\author{ Robert Durfee }
\date{ February 9, 2018 }

\begin{document}

\maketitle

\section{ Warm Up }

Let's say that you have two equations:
$$ y = x^{2}, \ 1 = (x - 1)^{2} + (y - 1)^{2}  $$

\begin{figure}[H]
  \centering
  \includegraphics[scale=0.60]{"WarmUp"}
  \caption{Warm Up}
\end{figure}

You need to figure out where these two graphs intersect. You could solve for
this point algebraically, however, the math soon become too difficult to keep
track of. As a result, it would be better to make a linear approximation of the
function close to the point and use that to solve for the intersection.

\bigbreak

Find the linear approximation of $f(x, y) = y - x^{2}$ around the point $(1,
2)$.
$$ L(x, y) \approx f(x_{0}, y_{0}) + f_{x}(x_{0}, y_{0}) (x - x_{0}) +
f_{y}(x_{0}, y_{0}) (y - y_{0})$$
$$ L(x, y) \approx -2x + y + 1 $$

Now we must decide which point to take our linear approximation around. In this
case, $(1.5, 1.9)$ is the best as it is closest to the intersection.

\section{18.01 Minima and Maxima}

An extreme point of a function $g(x)$ within a certain interval takes two forms:

\begin{figure}[H]
  \centering
  \includegraphics[scale=0.60]{"Minimum"}
  \caption{Minimum Within Interval}
\end{figure}

\begin{figure}[H]
  \centering
  \includegraphics[scale=0.60]{"MinimumAtBoundary"}
  \caption{Minimum At Boundary}
\end{figure}

\section{18.02 Minima and Maxima}

Functions of two variables, like $f(x, y)$, are analogous to their one variable
counterparts, however, the interval is a region in two dimensions. As a result,
an extreme point can occur at the edge, a corner, or within the region.

\bigbreak

The minimum and maximum of a graph exist where $f_{x}(x, y)$ and $f_{y}(x, y)$
are equal to zero. This is where the tangent plane is horizontal.

\end{document}

