\documentclass{article}
\usepackage{tikz}
\usepackage{float}
\usepackage{enumerate}
\usepackage{amsmath}
\usepackage{bm}
\usepackage{indentfirst}
\usepackage{siunitx}
\usepackage[utf8]{inputenc}
\usepackage{graphicx}
\graphicspath{ {Images/} }
\usepackage{float}
\usepackage{mhchem}
\usepackage{chemfig}
\allowdisplaybreaks

\title{ 18.02 Lecture 4 }
\author{ Robert Durfee }
\date{ February 13, 2018 }

\begin{document}

\maketitle

\section{ Warm Up }

Find the minimum of $f(x, y) = x^{2} + x + y^{2}$ within the region $S$ defined
between the intervals $-10 \leq x \leq 10$ and $-10 \leq y \leq 10$.

\bigbreak

Find all of the critical points of $f$.
$$ f_{x}(x, y) = 2x + 1, \ f_{y}(x, y) = 2y $$

Setting these partial derivatives equal to zero results in the critical point
$(-1/2, 0)$. This critical point results in the value $-1/4$. This was an easy
function, though, because the partial derivatives only depended on one variable.

\section{More Complicated Warm Up}

We can define a new function $g(x, y) = x^{2} + x + y^{2} + xy$. This function
as the following partial derivatives:
$$ g_{x}(x, y) = 2x + y + 1, \ g_{y}(x, y) = 2y + x $$

These partial derivatives depend on both variables. As a result, they must be
treated as a system of equations.

\section{How to Handle Boundaries}

\subsection{18.01 Methods}

We can determine the minimum along a boundary by setting one variable as a
constant and solving it the same way as a one dimensional extrema problem. This
process must be repeated for each edge and quickly become tedious.

\subsection{Number Sense}

When thinking about $f$, $x^{2} + y^{2}$ will be very large along the
boundaries. At the boundary, $x = \pm 10$ and $y = \pm 10$. As a result, $x^{2}
+ y^{2} \geq 100$. This is way larger than $-1/4$, so the minimum must not be on
an edge.

\bigbreak

What about on the whole plane? The function $f$ is larger outside of the regions
$S$, therefore the global minimum must occur within the region.

\section{Function with No Minima}

There are several examples of functions that do not have global minima. For
example, $f(x) = x$ and $g(x) = e^{-x^{2}}$.

\begin{figure}[H]
  \centering
  \includegraphics[scale=0.60]{"NoMinima"}
  \caption{No Minima}
\end{figure}

\section{Word Problems}

Suppose you have a box that you need to build. It costs a different amount to
build the front/back, top/bottom, and sides. The respective costs are \$1, \$2,
and \$3 per square foot. It is also required that the volume of the box is 48
cubic feet.

\bigbreak

This results in the creation of two equations:
$$ xyz = 48, \ 2xz + 4xy + 6yz = C $$

We can solve for $z$ and substitute into the other equation to create a function
of only two variables:
$$ C(x, y) = 96/y + 4xy + 288/x $$

Then we can solve for the two partial derivatives:
$$ C_{x}(x, y) = 4y + 288/x^{2}, \ C_{y}(x, y) = 96/y^{2} + 4x $$

Solving the system of equations results in the solution $(6, 2, 4)$. This
results in the cost of \$144.

\bigbreak

But is this point actually a minimum? We can test this by applying a boundary
arbitrarily. Along the boundary, we can see that the cost is large no matter
which edge we are using. As a result, the minimum must be within the region.

\end{document}

