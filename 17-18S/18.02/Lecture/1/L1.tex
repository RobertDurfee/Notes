\documentclass{article}
\usepackage{tikz}
\usepackage{float}
\usepackage{enumerate}
\usepackage{amsmath}
\usepackage{bm}
\usepackage{indentfirst}
\usepackage{siunitx}
\usepackage[utf8]{inputenc}
\usepackage{graphicx}
\graphicspath{ {Images/} }
\usepackage{float}
\usepackage{mhchem}
\usepackage{chemfig}
\allowdisplaybreaks

\title{ 18.02 Lecture 1 }
\author{ Robert Durfee }
\date{ February 6, 2018 }

\begin{document}

\maketitle

\section{ Multivariable Calculus in Life }

Most things in life depend on more than one variable. For example, when someone
is studying the weather, they must think in three dimensions, if not four when
thinking about time. When designing an engine, you need to maximize efficiency.
This efficiency does not only depend on a single variable, rather, many.

\section{ 18.01 Graphs }

When we think of the function $ f(x) = x^{2} $, the graph becomes $ y = f(x) $.

\begin{figure}[H]
  \centering
  \includegraphics[scale=0.40]{"Parabola"}
  \caption{Parabola}
\end{figure}

\section{ 18.02 Graphs }

When dealing with two variables, the function $ f(x, y) = x^{2} + y^{2} $ is
graphed as $ z = f(x, y) $. The problem then becomes, how do we visualize these
graphs in three dimensions? For this, we use \textbf{level curves}.

\subsection{ Level Curves }

\textbf{Level curves} are simply cross sections of a three dimensional graph at
certain heights. This can be written as $ f(x, y) = k $, where different height
values are substituted for $k$.

\subsubsection{ Example }

We can use $ f(x, y) = x^{2} + y^{2} $ and plot the different level curves as
follows:

\begin{center}
  \begin{tabular}{ c c }
    $$ x^{2} + y^{2} = 0 $$ & $$ r = 0 $$ &
    $$ x^{2} + y^{2} = 1 $$ & $$ r = 1 $$ &
    $$ x^{2} + y^{2} = 2 $$ & $$ r = \sqrt{2} $$ &
    $$ x^{2} + y^{2} = 3 $$ & $$ r = \sqrt{3} $$ &
    $$ x^{2} + y^{2} = 4 $$ & $$ r = 2 $$ &
    $$ x^{2} + y^{2} = 5 $$ & $$ r = \sqrt{5} $$ &
  \end{tabular}
\end{center}

\begin{figure}[H]
  \centering
  \includegraphics[scale=0.50]{"SphereLevelCurves"}
  \caption{Sphere Level Curves}
\end{figure}

\subsubsection{ Example }

If a hiker starts at (-1, 1) and moves in the positive $x$ direction, he will be
going \textbf{downhill}. If a hiker starts at (-2, 0) and moves in the positive
$x$ direction, he will still be going \textbf{downhill}. However, one hiker
travels across a steeper decline, this is \textbf{hiker 2}.

\begin{figure}[H]
  \centering
  \includegraphics[scale=0.50]{"Sphere"}
  \caption{Sphere}
\end{figure}

\section{ 18.01 Derivatives }

When thinking in only one variable, $ g(x) $ is a function while $ g'(x) $
measures how $g$ changes if $x$ increases by a slight amount.

\begin{figure}[H]
  \centering
  \includegraphics[scale=0.50]{"Derivative"}
  \caption{Derivative}
\end{figure}

We can then see a relationship between the derivative and the function as
follows:
$$ g(x + \Delta x) \approx g(x) + g'(x) \Delta x,\ \Delta x \approx 0 $$

\section{ 18.02 Partial Derivatives }

When thinking in two variables, $ f(x, y) $ is our function and $ f_{x}(x, y) $
measures how $f$ changes if $x$ increases by a slight amount. Additionally, $
f_{y}(x, y) $ measures how $f$ changes if $y$ increases by a slight amount. The
relationship between the partial derivatives and the function is as follows:
$$ f(x + \Delta x, y + \Delta y) \approx f(x, y) + f_{x}(x, y) \Delta x +
f_{y}(x, y) \Delta y,\ \Delta x \approx 0,\ \Delta y \approx 0$$

\subsection{ Computing Partial Derivatives }

Think of one variable as being a constant and different the other the same way
as before in 18.01.

\subsubsection{ 18.01 Example }

$$ g(x) = \cos(7x) $$
$$ g'(x) = -7 \sin(7x) $$

\subsubsection{ 18.02 Example }

$$ h(x, y) = \cos(xy) $$
$$ h_{x}(x, y) = -y \sin(xy) $$

\subsubsection{ Previous Example }

$$ f(x, y) = x^{2} + y^{2} $$
$$ f_{x}(x, y) = 2x,\ f_{y}(x, y) = 2y $$

Using the hiker analogy as before, we can calculate the slope that each one
experiences using partial derivatives.
$$ f_{x}(-1, 1) = -2,\ f_{x}(-2, 0) = -4 $$

The negative values indicate that the slope is downward and the greater value in
the second partial derivative indicates that the slope is greater.

\end{document}

