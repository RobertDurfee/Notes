\documentclass{article}
\usepackage{tikz}
\usepackage{float}
\usepackage{enumerate}
\usepackage{amsmath}
\usepackage{bm}
\usepackage{indentfirst}
\usepackage{siunitx}
\usepackage[utf8]{inputenc}
\usepackage{graphicx}
\graphicspath{ {Images/} }
\usepackage{float}
\usepackage{mhchem}
\usepackage{chemfig}
\allowdisplaybreaks

\title{ 18.02 Recitation 2 }
\author{ Robert Durfee }
\date{ February 12, 2018 }

\begin{document}

\maketitle

\section{ Linear Approximation }

The level curves for the function $f(x, y) = 2(x - 1) + (y - 1)$:

\begin{figure}[H]
  \centering
  \includegraphics[scale=0.60]{"LevelCurves"}
  \caption{Level Curve Warm Up}
\end{figure}

\begin{enumerate}[1.]
  \item Using the function $f(x, y) = y^{2} - x^{3} + xy - x$, $f(1, 1) = 0$.

  \item The partial derivatives for $f$:
    $$ f_{x}(x, y) = -3x^{2} + y - 1, \ f_{y}(x, y) = 2y + x $$
    $$ f_{x}(1, 1) = -3, \ f_{y}(1, 1) = 3 $$

  \item The linear approximation of $f$:
    $$ L(x, y) = -3x + 3y $$

  \item The linear approximation level curves:

    \begin{figure}[H]
      \centering
      \includegraphics[scale=0.60]{"LinearApproximationLevelCurves"}
      \caption{Linear Approximation Level Curves}
    \end{figure}

  \item The function $g(x, y) = x^{2} + 3y^{2}$ has partial derivatives:
    $$ g_{x}(x, y) = 2x, \ g_{y}(x, y) = 6y $$

    The plane $z = 2x + 6y - 6$ has partial derivatives:
    $$ h_{x}(x, y) = 2, \ h_{y}(x, y) = 6 $$

    The function $g$ has the same slope as the plane at the point $(1,1)$. This
    point has a height of 4. The plane, on the other hand, is only at height 2.
    As a result, we must add 2 to the plane to raise it to be tangential to the
    curve $g$.

    \bigbreak

    The new equation for the plane is $z = 2x + 6y - 4$.

  \item The function $f(x, y) = -(x^{2} - 1)^{2} - (x^{2}y - x - 1)^{2}$ has two
    critical points: (-1, 0) and (1, 2).

    \begin{figure}[H]
      \centering
      \includegraphics[scale=0.60]{"NoMinima"}
      \caption{No Minima}
    \end{figure}
\end{enumerate}

\end{document}

