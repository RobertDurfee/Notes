\documentclass{article}
\usepackage{tikz}
\usepackage{float}
\usepackage{enumerate}
\usepackage{amsmath}
\usepackage{bm}
\usepackage{indentfirst}
\usepackage{siunitx}
\usepackage[utf8]{inputenc}
\usepackage{graphicx}
\graphicspath{ {Images/} }
\usepackage{float}
\usepackage{mhchem}
\usepackage{chemfig}
\allowdisplaybreaks

\title{ 8.02 Problem Set 4 }
\author{ Robert Durfee - Lecture 7 - Table 8 }
\date{ March 6, 2018 }

\begin{document}

\maketitle

\section{ Problem 1 }

\subsection*{Part A}

Using $r$ and $\theta$ to calculate $r_{-}$ and $r_{+}$:
$$ r_{+} = \sqrt{(r \cos \theta - a)^{2} + (r \sin \theta)^{2}} $$
$$ r_{-} = \sqrt{(r \cos \theta + a)^{2} + (r \sin \theta)^{2}} $$

Using the radii to calculate potential from each charge:
$$ V_{+} = \frac{ k Q }{ \sqrt{(r \cos \theta - a)^{2} + (r \sin \theta)^{2}} } $$
$$ V_{-} = \frac{ -k Q }{ \sqrt{(r \cos \theta + a)^{2} + (r \sin \theta)^{2}} } $$

Superposition principle can be used to find total potential:
$$ V_{P} = \frac{ k Q }{ \sqrt{r^{2} + a^{2} - 2ar\cos\theta} }
- \frac{ k Q }{ \sqrt{r^{2} + a^{2} + 2ar\cos\theta} }$$

\subsection*{Part B}

The answer from part A can be written to make the Taylor expansion choice more
obvious:
$$ V = \frac{ kQ }{ r } \left( \left( 1 + \left( \frac{ a }{ r }
\right)^{2} - 2 \left( \frac{ a }{ r } \right) \cos \theta \right)^{-1/2}
- \left( 1 + \left( \frac{ a }{ r } \right)^{2} + 2 \left( \frac{ a }{ r
} \right) \cos \theta  \right)^{-1/2}\right) $$

In the above equation, the following Taylor expansion can be applied twice:
$$ (1 + x)^{m} \approx 1 + mx + \ldots $$

Substituting this into the previous equation (ellipsis omitted for simplicity):
$$ V = \frac{ kQ }{ r } \left( \left( 1 - \frac{ 1 }{ 2 } \left( \left(
\frac{ a }{ r } \right)^{2} - 2 \left( \frac{ a }{ r } \right) \cos \theta
\right) \right) - \left( 1 - \frac{ 1 }{ 2 } \left( \left( \frac{ a }{ r}
\right)^{2} + 2 \left( \frac{ a }{ r } \right) \cos \theta \right) \right)
\right) $$

Simplifying this equation:
$$ V = \frac{ 2kQa \cos \theta }{ r^{2} } $$

We also know the values for the following vectors:
$$ \vec{p} = \langle 0, 2aQ \rangle,\ \hat{r} = \langle \sin \theta, \cos \theta
\rangle $$

Using a dot product, this equation can be simplified further:
$$ V = \frac{ k \vec{p} \cdot \hat{r} }{ r^{2} } $$

\subsection*{Part C}

The supplied equation for electric potential:
$$ V(r, \theta) = \frac{ k \vec{p} \cdot \vec{r} }{ r^{3} } $$

And the supplied equation for electric field from electric potential:
$$ E_{\theta}(r, \theta) = -\frac{ 1 }{ r } \frac{ \partial}{ \partial \theta }
V(r, \theta) $$
$$ E_{r}(r, \theta) = -\frac{ \partial }{ \partial r } V(r, \theta) $$

Substituting values for $\vec{p}$ and $\vec{r}$:
$$ E_{\theta}(r, \theta) = -\frac{ 1 }{ r } \frac{ \partial }{ \partial \theta }
\left( \frac{ k \langle p, 0 \rangle \cdot \langle r \cos \theta, r \sin \theta
\rangle }{ r^{3} } \right) = \frac{ kp \sin \theta }{ r^{3} }$$
$$ E_{r}(r, \theta) = -\frac{ \partial }{ \partial r } \left( \frac{ k \langle p, 0
\rangle \cdot \langle r \cos \theta, r \sin \theta \rangle}{ r^{3} } \right) =
\frac{ 2 p k \cos \theta }{ r^{3} }$$

\subsection*{Part D}

Calculating initial electric potential:
$$ V_{i} = kq \left( \frac{ 1 }{ d - a } - \frac{ 1 }{ d + a } \right) $$

And final electric potential:
$$ V_{f} = kq \left( \frac{ 1 }{ d + s - a } - \frac{ 1 }{ d + s + a } \right) $$

Converting change to electric potential energy:
$$ \Delta U = q \Delta V = k q^{2} \left( \left( \frac{ 1 }{ d + s - a } -
\frac{ 1 }{ d + s + a } \right) - \left( \frac{ 1 }{ d - a } - \frac{ 1 }{ d + a
} \right)\right) $$

Given that only conservative forces are involved:
$$ -\Delta U = \Delta KE $$

Setting final kinetic energy equal to negative change in potential energy, given
that the partial started from rest:
$$ \frac{ 1 }{ 2 } m v^{2} = -k q^{2} \left( \left( \frac{ 1 }{ d + s - a } -
\frac{ 1 }{ d + s + a } \right) - \left( \frac{ 1 }{ d - a } - \frac{ 1 }{ d + a
} \right)\right) $$
$$ v = \sqrt{\frac{ -2 k q^{2}}{m} \left( \frac{ 1 }{ d + s - a } -
\frac{ 1 }{ d + s + a } - \frac{ 1 }{ d - a } + \frac{ 1 }{ d + a
} \right)} $$

\section*{Problem 2}

\subsection*{Part A}

The $y$ components of the individual dipole moments will cancel due to symmetry.
Since the dipole moments are the same in the $x$ direction, the formula can be
simplified to:
$$ 2 q d \cos \theta = p $$

Solving for $q$:
$$ q = \frac{ p }{ 2 d \cos \theta } $$

Substituting values:
$$ q = 5.16 \cdot 10^{-20}\ \si{C} $$

\subsection*{Part B}

Using the equation from problem 1:
$$ V = \frac{ k \vec{p} \cdot \vec{r} }{ r^{3} } $$

Substituting values:
$$ V = \frac{ k \langle p, 0 \rangle \cdot \langle r \cos \theta, r \sin \theta
\rangle }{ r^{3} } = \frac{ k p \cos \theta }{ r^{2} }$$

\section*{Problem 3}

\subsection*{Part A}

The positive charge on the insulator will attract the movable electrons within
the conductor. As a result, the electrons from the left conductor will travel
across the wire and to the insulator core of the right conductor.

\subsection*{Part B}

Since there is no electric field within a conductor in static equilibrium, the
inside shell of the right conductor will have $-Q$ charge.

\bigbreak

The outside of each conductor must also have the same electric potential:
$$ V = \frac{ k Q_{L} }{ 2a } = \frac{ k Q_{R} }{ 3a } $$
$$ \frac{ Q_{L} }{ 2 } = \frac{ Q_{R} }{ 3 } $$

\bigbreak

Since the conductor was originally neutral, due to conservation of charge, the
charge on the outside of the left conductor and the outside of the right
conductor will sum to $Q$.
$$ Q = Q_{L} + Q_{R} $$

Solving for $Q_{L}$ and $Q_{R}$:
$$ Q_{L} = \frac{ 2 Q }{ 5 },\ Q_{R} = \frac{ 3 Q }{ 5 } $$

\section*{Problem 4}

\subsection*{Part A}

$$ dU = \frac{ k q dq }{ R } $$

\subsection*{Part B}

Integrating over all $Q$:
$$ U = \int\limits_{0}^{Q} \frac{ k q dq }{ R } = \frac{ k Q^{2} }{ 2R }$$

\subsection*{Part C}

\begin{enumerate}[i.]
  \item Assuming a point charge:
    $$ V = \frac{ k Q }{ R } $$

    Solving for $Q$:
    $$ Q = \frac{ V R }{ k } $$

    Substituting values:
    $$ Q = -5 \cdot 10^{-18}\ \si{C} $$

    Dividing by the elementary charge:
    $$ n_{e} = \frac{ Q }{ 1.602 \cdot 10^{-19} }\approx 31 $$
  \item Once again, assuming point charge:
    $$ E = \frac{ k Q }{ R^{2} } $$

    Substituting values:
    $$ E = 5 \cdot 10^{5}\ \si{N\ C^{-1}} $$
\end{enumerate}

\section*{Problem 5}

\subsection*{Part A}

Calculating the magnitude of a small area of charge:
$$ dq = \sigma r dr d\theta $$

Integrating the electric potential from this small unit of charge:
$$ V = \int\limits_{0}^{2 \pi}\int\limits_{R_{1}}^{R_{2}} k \sigma dr
d\theta = 2 \pi k \sigma (R_{2} - R_{1})$$

Substituting values:
$$ V = -113\ \si{V} $$

\subsection*{Part B}

Converting electric potential to potential energy:
$$ \Delta U = q\Delta V  $$

Since forces are conservative:
$$ -\Delta U = \Delta KE $$

Since the particle starts from rest, change in kinetic is just final:
$$ \frac{ 1 }{ 2 } m v^{2} = -q \Delta V $$

Solving for $v$:
$$ v = \sqrt{\frac{ 2 q \Delta V }{ m }} $$

Substituting values:
$$ v = 6,306,860 \ \si{m\ s^{-1}} $$

\subsection*{Part C}

Using the same small area of charge:
$$ dq = \sigma(r) r dr d\theta $$

Integrating the electric potential from this small unit of charge:
$$ V = \int\limits_{0}^{2 \pi}\int\limits_{0}^{R_{2}} \frac{ k \sigma_{0} R dr
  d\theta}{\sqrt{r^{2} + z^{2}}} = 2 \pi k \sigma_{0} R \left(\ln(R + \sqrt{R^{2} +
z^{2}}) - \ln(z)\right)$$

\section*{Problem 6}

\subsection*{Part A}

The equation for capacitance between two parallel plates:
$$ C = \frac{ \epsilon_{0} A }{ d } $$

Substituting for distances:
$$ C_{1} = \frac{ \epsilon_{0} A }{ z },\ C_{2} = \frac{ \epsilon_{0} A }{ d - z } $$

The total capacitance is superposed:
$$ C =  \frac{ \epsilon_{0} A }{ z } + \frac{ \epsilon_{0} A }{ d - z }$$

\subsection*{Part B}

Equation for stored energy in a capacitor:
$$ E = \frac{ 1 }{ 2 } C \Delta V^{2} $$

Substituting values:
$$ E = \frac{ \epsilon_{0} A }{ 2 } \left( \frac{ 1 }{ z } + \frac{ 1 }{ d - z }
\right) \Delta V^{2} $$

\end{document}

