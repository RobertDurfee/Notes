\documentclass{article}
\usepackage{tikz}
\usepackage{float}
\usepackage{enumerate}
\usepackage{amsmath}
\usepackage{amsthm}
\usepackage{bm}
\usepackage{indentfirst}
\usepackage{siunitx}
\usepackage[utf8]{inputenc}
\usepackage{graphicx}
\graphicspath{ {Images/} }
\usepackage{float}
\usepackage{mhchem}
\usepackage{chemfig}
\allowdisplaybreaks

\title{8.02 Problem Set 10}
\author{Robert Durfee}
\date{May 1, 2018}

\begin{document}

\maketitle

\section*{Problem 1}

\subsection*{Part A}

Construct a Gaussian cylinder of radius $r$ which intersects the left disc. The
disc has a area charge density of $Q/A$. Therefore, Gauss's Law can be written,
$$ \iint \vec{E} \cdot d\vec{A} = \frac{Q_{enc}}{\varepsilon_0} $$
Simplifying, this becomes,
$$ 2 E \cdot \pi r^2 = \frac{Q \pi r^2}{\varepsilon_0 A} $$
Solving for $E$,
$$ E = \frac{Q}{2 \varepsilon_0 A} $$
Since there is another disc on the right, the total electric field is,
$$ E = \frac{Q}{\varepsilon_0 A} $$

\subsection*{Part B}

The change in potential between plates is given,
$$ \Delta V = \int\limits_0^d \frac{Q}{\varepsilon_0 A} ds $$
Computing this integral,
$$ \Delta V = \frac{Q d}{\varepsilon_0 A} $$

\subsection*{Part C}

Capacitance is given,
$$ C = \frac{Q}{\Delta V} $$
Substituting computed value for $\Delta V$,
$$ C = \frac{\varepsilon_0 A}{d} $$

\subsection*{Part D}

Construct an Amperian square of side length $\ell$ containing a few of the wires
of the solenoid on one side. The number of turns per unit length is given by
$N/h$. Therefore, Ampere's Law can be written,
$$ \int \vec{B} \cdot d\vec{s} = \mu_0 I_{enc} $$
Substituting for $I_{enc}$,
$$ B \ell = \mu_0 \frac{N}{h} \ell I $$
Solving for $B$,
$$ B = \mu_0 \frac{N}{h}I $$

\subsection*{Part E}

Inductance is given,
$$ L = \frac{\Phi}{I} $$
Substituting for computed $B$,
$$ L = \mu_0 \pi \frac{N^2}{h} a^2 $$

\subsection*{Part F}

The natural angular frequency for oscillation is given,
$$ \omega = \frac{1}{\sqrt{LC}} $$
Converting to period,
$$ T = 2 \pi \sqrt{LC} $$
Substituting values computed above,
$$ T = 2 \pi \sqrt{\left( \mu_0 \pi \frac{N^2}{h} a^2 \right) \left(
\frac{\varepsilon_0 A}{d} \right)} $$

\subsection*{Part G}

Energy stored in a capacitor is given,
$$ U = \frac{1}{2} \frac{Q^2}{C} $$
Substituting for capacitance from above,
$$ U = \frac{d}{2 \varepsilon_0 A} Q^2 $$

\section*{Problem 2}

\subsection*{Part A}

From the differential equation given from Kirchoff's Loop Rule,
$$ I_0 = \frac{V_0}{\sqrt{R^2 + (\omega L - 1/\omega C)^2}} $$

\subsection*{Part B}

The minimum occurs when,
$$ \omega L = \frac{1}{\omega C} $$
Solving for $\omega$,
$$ \omega = \frac{1}{\sqrt{LC}} $$

\section*{Problem 3}

\subsection*{Part A}

After the switch is closed for a long time, the inductor acts as a wire.
Thus,
$$ I_0 = \frac{\varepsilon_0}{R} $$

\subsection*{Part B}

Taking Kirchoff's Loop Rule clockwise,
$$ 0 = \varepsilon - IR_2 - IR_1 - L\frac{dI}{dt} $$
Solving for the time derivative of current,
$$ \frac{dI}{dt} = \frac{\varepsilon - I(R_1 + R_2)}{L} $$

\subsection*{Part C}

Integrating with respect to current and time results,
$$ \frac{L}{R_1 + R_2} \ln\left(\frac{\varepsilon - I_0 (R_1 + R_2)}{\varepsilon
- I (R_1 + R_2)}\right) = t $$
Solving for $I$,
$$ I(t) = \frac{\varepsilon - (\varepsilon - I_0 (R_1 + R_2)) e^{-t(R_1 +
R_2)/L}}{R_1 + R_2} $$
However, since we are assuming $\varepsilon \approx 0$,
$$ I(t) = I_0 e^{-t (R_1 + R_2) / L} $$

\subsection*{Part D}

Faraday's Law gives,
$$ \varepsilon = -L \frac{dI}{dt} $$
Taking the derivative of current,
$$ \varepsilon(t) = I_0 (R_1 + R_2) e^{-t (R_1 + R_2)/L} $$
Since this is right after the switch is opened, $t = 0$,
$$ \varepsilon = I_0 (R_1 + R_2) $$

\subsection*{Part E}

This assumption was reasonable considering the EMF from the inductor will be
much larger than the battery's at this moment.

\subsection*{Part F}

The change in voltage is given,
$$ \Delta V = I_0 R_2 $$
Substituting for $I_0$,
$$ \Delta V = \frac{\varepsilon_0 R_2}{R_1} $$

\subsection*{Part G}

Substituting the given value of the resistor,
$$ \Delta V = 80 \varepsilon $$

\subsection*{Part H}

This is significantly higher than the voltage supplied by the battery. As a
result, there might be a large arc formed when flipping the switch.

\section*{Problem 4}

\subsection*{Part A}

Inductance is given,
$$ L = \frac{\Phi}{I} $$
Decomposing flux,
$$ L = \frac{B A N}{I} $$
Solving for magnetic field per current,
$$ \frac{B}{I} = \frac{L}{AN} $$
Substituting values,
$$ \frac{78 \cdot 10^{-3}}{\pi (1.7/2 \cdot 10^{-1})^2 \cdot 235} = 1.462\
\si{T\ A^{-1}} $$

\subsection*{Part B}

Mutual inductance is given,
$$ M_{12} = \frac{N_1 \Phi_{12}}{I_2} $$
Decomposing flux,
$$ M_{12} = \frac{B_2}{I_2} N_1 A_2 $$
Substituting values,
$$ \pi 1.462 \cdot 2920 \cdot (1.7/2 \cdot 10^{-2})^2 = 0.969\ \si{H} $$

\subsection*{Part C}

The inductance for both solenoids, assuming equal cross-sectional area,
$$ L_1 = \frac{B_1 N_1 A}{I_1},\ L_2 = \frac{B_2 N_2 A}{I_2} $$
The mutual inductance, computed both ways,
$$ M_{12} = \frac{B_2 N_1 A}{I_2},\ M_{21} = \frac{B_1 N_2 A}{I_1} $$
Multiplying the self inductances together,
$$ L_1 L_2 = \frac{B_1 B_2 N_1 N_2 A}{I_1 I_2} $$
Which is the same as the mutual inductance squared, computed each way.

\subsection*{Part D}

For an ideal transformer,
$$ \frac{N_p}{N_s} = \frac{V_p}{V_s} $$
Therefore, the voltage is given,
$$ \frac{3}{235} \cdot 2920 = 37.28\ \si{V} $$

\section*{Problem 5}

\subsection*{Part A}

Since the current leads and lags the voltage, there must be both an inductor and
a capacitor.

\subsection*{Part B}

From the information provided, two equations are known,
$$ \tan^{-1}(-\pi/6) = \frac{2L - 1/2C}{\sqrt{3}},\ \tan^{-1}(\pi/6) = \frac{6L -
1/6C}{\sqrt{3}} $$
Solving this system of equations yields,
$$ L = 1/4,\ C = 1/3 $$

\subsection*{Part C}

The natural resonant frequency is given,
$$ \omega = \frac{1}{\sqrt{LC}} $$
Substituting values,
$$ \omega = 2 \sqrt{3} $$

\section*{Problem 6}

\subsection*{Part A}

The amplitude of the current through this circuit is given,
$$ I_0 = \frac{V_0}{\sqrt{R^2 + (\omega L)^2}} $$
The time-averaged power through the light bulb is,
$$ P = \frac{1}{2} I_0^2 R $$
Since the power without an inductor is $100\ \si{W}$, the resistance can be
determined,
$$ \frac{(120 \sqrt{2})^2}{2 \cdot 100} = 144\ \si{\Omega} $$
Now, using this resistance and the goal of reducing the power dissipated by the
light bulb by 5 times, the inductance can be determined using the equation,
$$ P = \frac{R V^2}{(R^2 + (\omega L)^2)} $$
Substituting values and solving for $L$,
$$ L = 0.764\ \si{H} $$

\subsection*{Part B}

You can use a resistor instead of an inductor in this circuit. However, a
resistor will consume energy and release it in the form of heat. An inductor,
on the other hand, will only temporarily store energy within its magnetic field.

\end{document}

