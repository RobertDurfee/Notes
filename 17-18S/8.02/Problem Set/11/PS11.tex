\documentclass{article}
\usepackage{tikz}
\usepackage{float}
\usepackage{enumerate}
\usepackage{amsmath}
\usepackage{amsthm}
\usepackage{bm}
\usepackage{indentfirst}
\usepackage{siunitx}
\usepackage[utf8]{inputenc}
\usepackage{graphicx}
\graphicspath{ {Images/} }
\usepackage{float}
\usepackage{mhchem}
\usepackage{chemfig}
\allowdisplaybreaks

\title{8.02 Problem Set 11}
\author{Robert Durfee}
\date{May 11, 2018}

\begin{document}

\maketitle

\section*{Problem 1}

\subsection*{Part A}

Construct a Gaussian cylinder of radius $r$ which intersects the bottom plate of
the capacitor. The charge enclosed in this region, assuming uniform
distribution, is
$$ Q_{enc} = \frac{Q r^2}{a^2} $$
Using this value, Gauss's law can be simplified,
$$ 2 E \pi r^2 = \frac{Q r^2}{\varepsilon_0 a^2} $$
Solving for $E$,
$$ E = \frac{Q}{2 \varepsilon_0 \pi a^2} $$
However, this only accounts for half of the field between the plates. Since the
top plate is symmetric, this value can be doubled to find total field between
plates,
$$ E_{T} = \frac{Q}{\varepsilon_0 \pi a^2} $$
And since field points from positive to negative charge, this field points up.

\subsection*{Part B}

Assuming the current through the resistor is uniform, enclosed current is,
$$ I_{enc} = \frac{I}{\pi a^2} \pi r^2 = \frac{r^2}{a^2} \frac{\partial
Q}{\partial t} $$
This current is flowing downward, thus
$$ \frac{I_{enc}}{\partial Q / \partial t} = - \frac{r^2}{a^2} $$

\subsection*{Part C}

Since the electric field is (nearly) constant between the plates, the flux
through the defined region can be written,
$$ \Phi_E = \left( \frac{Q}{\varepsilon_0 \pi a^2} \right) \left( \pi r^2
\right) = \frac{Q r^2}{\varepsilon_0 a^2} $$
The time derivative of flux,
$$ \frac{\partial \Phi_E}{\partial t} = \frac{r^2}{\varepsilon_0 a^2}
\frac{\partial Q}{\partial t} $$
Dividing by the time derivative of charge,
$$ \frac{\partial \Phi_E / \partial t}{\partial Q / \partial t} =
\frac{r^2}{\varepsilon_0 a^2} $$

\subsection*{Part D}

Using Ampere-Maxwell equation,
$$ \oint \vec{B} \cdot d\vec{s} = \mu_0 \left( I_{enc} + I_{dis} \right) = \mu_0
\left( -\frac{r^2}{a^2} + \frac{r^2}{a^2} \right) = 0 $$

\section*{Problem 2}

\subsection*{Part A}

Using the equation for magnetic field,
$$ \vec{B} = B_0 f(ax + bt) \hat{j} $$
Taking the second derivative with respect to displacement,
$$ \frac{\partial^2 \vec{B}}{\partial x^2} = B_0 a^2 f(ax + bt) \hat{j} $$
And with respect to time,
$$ \frac{\partial^2 \vec{B}}{\partial t^2} = B_0 b^2 f(ax + bt) \hat{j} $$
Setting them equal using the requirement for satisfying Maxwell's equations,
$$ \frac{\partial^2 \vec{B}}{\partial x^2} = \frac{1}{c^2} \frac{\partial^2
\vec{B}}{\partial t^2} $$
Simplifying,
$$ a = \frac{b}{c} $$

\subsection*{Part B}

Given the relation between electric field and magnetic field magnitude,
$$ E_0 = c B_0 $$
Using the cross product for determining direction,
$$ \hat{E} \times \hat{j} = -\hat{i} $$
Thus, the direction of the electric field must be $\hat{k}$.

\subsection*{Part C}

The solution of the Poynting vector only in terms of $B_0$ is,
$$ \vec{S} = -\frac{c B_0^2}{2 \mu_0} \hat{i} $$

\subsection*{Part D}

Pressure on a perfectly reflecting medium is,
$$ P_{ref} = \frac{2}{c} S $$
Substituting for $S$,
$$ P_{ref} = \frac{2 B_0^2}{\mu_0} $$

\section*{Problem 3}

\subsection*{Part A}

The pressure on a perfectly reflecting medium can be written,
$$ P_{ref} = \frac{2}{c} \frac{P}{4 \pi r_{es}^2} $$
Substituting values,
$$ P_{ref} = 9.44 \cdot 10^{-6}\ \si{Pa} $$

\subsection*{Part B}

Determining force from pressure,
$$ F_{rad} = \frac{P}{A} $$
Substituting values,
$$ F_{rad} = 1.89 \cdot 10^{-5}\ \si{N} $$

\subsection*{Part C}

The equation for gravitational force,
$$ F_{grav} = G \frac{m_1 m_2}{r_{es}^2} $$
Substituting values,
$$ F_{grav} = 0.59\ \si{N} $$
Dividing the radiation force by the gravitational force,
$$ \frac{F_{rad}}{F_{grav}} = 3.20 \cdot 10^{-5} $$

\section*{Problem 4}

\subsection*{Part A}

The equation for the electric field,
$$ \vec{E}(y, t) = E_0 \sin\left( \frac{2 \pi}{\lambda} \left(y - ct
\right)\right) \hat{k} $$

\subsection*{Part B}

The corresponding equation for the magnetic field,
$$ \vec{B}(y, t) = \frac{E_0}{c} \sin \left( \frac{2 \pi}{\lambda} \left(y -
ct\right)\right) \hat{i} $$

\subsection*{Part C}

The Poynting vector is defined,
$$ \vec{S} = \frac{\vec{E} \times \vec{B}}{\mu_0} $$
Substituting equations from previous parts,
$$ \vec{S}(y, t) = \frac{E_0^2}{\mu_0 c} \sin^2 \left(\frac{2 \pi}{\lambda}
\left(y - ct \right) \right) \hat{j} $$

\subsection*{Part D}

The force on a perfectly absorbing medium is,
$$ \langle \vert \vec{F}_{abs} \vert \rangle = \frac{A}{c} \langle \vert \vec{S}
\vert \rangle $$
Substituting our equation for $\vec{S}$,
$$ \langle \vert \vec{F}_{abs} \vert \rangle = \frac{E_0^2 b^2}{2 \mu_0 c^2} $$

The force on a perfectly reflecting medium is simply twice that of the absorbing
medium,
$$ \langle \vert \vec{F}_{ref} \vert \rangle = \frac{E_0^2 b^2}{\mu_0 c^2} $$

\section*{Problem 5}

\subsection*{Part A}

The Poynting vector is defined,
$$ \vec{S} = \frac{\vec{E} \times \vec{B}}{\mu_0} $$
Substituting equations provided,
$$ \vec{S} = \frac{E_0^2}{\mu_0 c} \sin^2 \left(kx - \omega t \right) \hat{i} $$

\subsection*{Part B}

Given the time average of a sine function,
$$ \langle \vec{S} \rangle = \frac{E_0^2}{2 \mu_0 c} \hat{i} $$

\subsection*{Part C}

The time average of the energy density in electric and magnetic fields,
$$ \langle u_E \rangle = \frac{\varepsilon_0}{2} \langle E^2 \rangle,\ \langle
u_M \rangle = \frac{1}{2 \mu_0} \langle B^2 \rangle $$
The time average of the electric and magnetic fields are,
$$ \langle E^2 \rangle = \frac{E_0^2}{2},\ \langle B^2 \rangle =
\frac{E_0^2}{2c^2} $$
Substituting,
$$ \langle u_E \rangle = \frac{\varepsilon_0 E_0^2}{4},\ \langle u_M \rangle =
\frac{E_0^2}{4 \mu_0 c^2} $$
Using the relation of $\varepsilon_0 \mu_0 = 1/c^2$,
$$ \langle u_E \rangle = \frac{\varepsilon_0 E_0^2}{4},\ \langle u_M \rangle =
\frac{\varepsilon_0 E_0^2}{4} $$

\subsection*{Part D}

The total energy from energy density is,
$$ \langle U \rangle = \iiint \langle u_E \rangle + \langle u_M \rangle dA $$
Assuming uniformity,
$$ \langle U \rangle = \frac{\varepsilon_0 E_0^2 A c \Delta t}{2} $$

\subsection*{Part E}

The time derivative of the total energy just cancels the $\Delta t$,
$$ \frac{\partial\langle U \rangle}{\partial t} = \frac{\varepsilon_0 E_0^2 A c
}{2} $$

\subsection*{Part F}

The definition of power is the time derivative of energy, thus,
$$ \langle P \rangle = \frac{\varepsilon_0 E_0^2 A c }{2} $$

\section*{Problem 6}

\subsection*{Part A}

From the definition of Poynting vector,
$$ S_{sun}(r) = \frac{P_{sun}}{4 \pi r^2} $$

\subsection*{Part B}

From the definition of pressure on a perfectly reflecting medium,
$$ P_{ref} = \frac{P_{sun}}{2 \pi c r^2} $$

\subsection*{Part C}

From the definition of pressure,
$$ F = \frac{P_{sun} A}{2 \pi c r^2} $$

Equating gravitational force and force from radiation,
$$ G \frac{m_1 m_2}{r^2} = \frac{P_{sun} A}{2 \pi c r^2} $$
Since the sail is circular,
$$ A = \pi \left(\frac{d}{2}\right)^2 $$
Simplifying the expression and solving for $d$,
$$ d = \sqrt{\frac{8 G m_1 m_2 c}{P}} $$
Substituting values for masses,
$$ d \approx 892\ \si{m} $$

\section*{Problem 7}

\subsection*{Part A}

For radii smaller than $a$, there is no field because the charge in a conductor
is located on the surface. Also, for radii larger than $b$, there is in field
because the total enclosed charge is $0$.

For $a < r < b$, construct a Gaussian cylinder of radius $a < r < b$ and length
$\ell$. Gauss's law becomes,
$$ E 2 \pi r \ell = \frac{-Q}{\varepsilon_0} $$
Solving for $E$,
$$ E = \frac{-Q}{2 \pi \varepsilon_0 r \ell} \hat{r} $$

\subsection*{Part B}

For radii smaller than $a$ and larger than $b$, there is no changing field or
current. As a result, there is no magnetic field in these regions.

For $a < r < b$, construct an Amperian loop of radius $a < r < b$. There is no
enclosed current, only displacement current. Therefore, since the electric field
is constant at a certain radius in this region, Ampere-Maxwell equation becomes,
$$ B 2 \pi r = \mu_0 \varepsilon_0 \frac{\partial}{\partial t} \left( \left(
\frac{-Q}{2 \pi \varepsilon_0 r \ell} \right) \left(2 \pi 2 \ell \right) \right)
$$
Simplifying,
$$ B 2 \pi r = - \mu_0 I $$
Solving for $B$,
$$ \vec{B} = \frac{- \mu_0 I}{2 \pi r} \hat{\phi} $$

\subsection*{Part C}

Taking the cross product of electric field and magnetic field give the Poynting
vector,
$$ \vec{S} = \frac{I Q}{4 \varepsilon_0 \pi^2 \ell r^2} \hat{k} $$

\subsection*{Part D}

The surface of integration is the "washer" region between radii $a$ and $b$.
$$ P = \int\limits_0^{2 \pi} \int\limits_a^b \frac{I Q}{4 \varepsilon_0 \pi^2
\ell r^2} r dr d\phi = \frac{I Q}{2 \varepsilon_0 \pi \ell} \int\limits_a^b
\frac{dr}{r} = \frac{I Q}{2 \varepsilon_0 \pi \ell} \ln\left(\frac{b}{a}\right) $$

\subsection*{Part E}

Some of the power that flows through the coaxial will be stored within the
electric and magnetic fields of the cable. Thus, the power dissipated by the
resistor should be less than the power that flows into the cable from the
battery.

\end{document}
