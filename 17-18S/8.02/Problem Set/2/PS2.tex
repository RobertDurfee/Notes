\documentclass{article}
\usepackage{tikz}
\usepackage{float}
\usepackage{enumerate}
\usepackage{amsmath}
\usepackage{bm}
\usepackage{indentfirst}
\usepackage{siunitx}
\usepackage[utf8]{inputenc}
\usepackage{graphicx}
\graphicspath{ {Images/} }
\usepackage{float}
\usepackage{mhchem}
\usepackage{chemfig}
\allowdisplaybreaks

\title{ 8.02 Problem Set 2 }
\author{ Robert Durfee }
\date{ February 20, 2018 }

\begin{document}

\maketitle

\section{ Electric Field and Work Energy }

\subsection*{Part A}

First, define an angular charge density:
$$ \lambda = \frac{ Q }{ 2 \pi } $$

Then the electric field at point $P$ in the $z$ direction due to a small charge
unit can be determined. Note that there is no $x$ or $y$ component due to
symmetry.
$$ dE_{z} = \frac{ k \lambda z d\theta }{ \left( R^{2} + z^{2} \right)^{3/2} } $$

Then integrate this electric field due to the small charge around the whole
ring:
$$ \int\limits_{0}^{2\pi} \frac{ k \lambda z d\theta }{ \left( R^{2} + z^{2}
\right)^{3/2} } = \frac{ 2 \pi k \lambda z }{ \left( R^{2} + z^{2}
\right)^{3/2} } = \frac{ k z Q }{ \left( R^{2} + z^{2}
\right)^{3/2} } $$

Then, take the derivative to maximize:
$$ E_{z}' = \frac{ k Q \left( R^{2} - 2 z^{2} \right) }{ \left( R^{2} + z^{2}
\right)^{5/2} }$$

Setting this equal to zero:
$$ z_{P} = \frac{ R }{ \sqrt{2} } $$

The ratio $z_{P} / R$ then becomes:
$$ \frac{ z_{P} }{ R } = \frac{ 1 }{ \sqrt{2} } $$

\subsection*{Part B}

Using the electric field equation, the force equation can be calculated:
$$ F_{z} = \frac{ k z Q q }{ \left( R^{2} + z^{2} \right)^{3/2} }  $$

Using the work-kinetic energy equation, final kinetic energy can be calculated,
given that initially the particle is at rest and goes from $P$ to $0$:
$$ K_{f} = \int\limits_{\frac{ R }{ \sqrt{2} }}^{0} \frac{ k z Q q dz}{ \left(
R^{2} + z^{2} \right)^{3/2} } = \frac{ \left( \sqrt{6} - 3 \right) k q Q }{ 3 R } $$

Substitute the formula for kinetic energy:
$$ \frac{ 1 }{ 2 } m v^{2} =  \frac{ \left( \sqrt{6} - 3 \right) k q Q }{ 3 R }  $$

Solving for velocity:
$$ v = \sqrt{\frac{ 2 \left( \sqrt{6} - 3 \right) k q Q}{ 3 m R }} $$

Substituting in values provided:
$$ v = 5.75\ \si{m\ s^{-1}} $$

\subsection*{Part C}

The negative of the derivative of the electric field times $q$ is the second
derivative of the potential energy equation. This will become the oscillation
constant:
$$ \frac{ k Q q \left( R^{2} - 2 z^{2} \right) }{ \left( R^{2} + z^{2}
\right)^{5/2} } $$

Now, since $z \ll R$, this can be simplified:
$$ \frac{ k q Q }{ R^{3} } $$

Plugging this value in for the period:
$$ T = 2 \pi \sqrt{\frac{ k q Q }{ m R^{3} } } $$

Substituting in values provided:
$$ T = 0.13\ \si{s} $$

\section{ Charged Slab and Sheets }

\subsection*{Part A}

Define a Gaussian cylinder of radius $a$ that intersects the slab so that one
end is between $-s$ and $-d$ and the other is between $d$ and $s$.

\bigbreak

Calculate the charge enclosed in the cylinder:
$$ Q_{enc} = 2 \rho \pi d a^{2} $$

Define Gauss's Law:
$$ \iint \vec{E} \cdot d\vec{A} = \frac{ 2 \rho \pi d a^{2} }{ \epsilon_{0} } $$

The electric field is different on either side of the cylinder, one is pointing
into the surface and the other is pointing outward:
$$ \frac{ E \pi a^{2}}{ 2 } - E \pi a^{2} = \frac{ 2 \rho \pi d a^{2} }{
\epsilon_{0} } $$

Solving for $\rho / \epsilon_{0}$:
$$ \frac{ \rho }{ \epsilon_{0} } = -\frac{ E }{ 4 d }$$

\subsection*{Part B}

Define a Gaussian cylinder of radius $a$ that intersects the left charged sheet.

\bigbreak

Calculate the charge enclosed in the cylinder:
$$ Q_{enc} = \sigma \pi a^{2} $$

Define Gauss's Law:
$$ \iint \vec{E} \cdot d\vec{A} = \frac{ \sigma \pi a^{2} }{ \epsilon_{0} } $$

The electric field is different on either side of the cylinder, but both are
point outwards:
$$ E \pi a^{2} + 2 E \pi a^{2} = \frac{ \sigma \pi a^{2} }{ \epsilon_{0} } $$

Solving for $\sigma / \epsilon_{0}$:
$$ \frac{ \sigma }{ \epsilon_{0} } = 3 E$$

\subsection*{Part C}

Define a Gaussian cylinder of radius $a$ that intersects the right charged sheet.

\bigbreak

The charge enclosed is the same as before:
$$ Q_{enc} = \sigma \pi a^{2} $$

And so is Gauss's Law:
$$ \iint \vec{E} \cdot d\vec{A} = \frac{ \sigma \pi a^{2} }{ \epsilon_{0} } $$

The electric field on either end of the cylinder, however, is different:
$$ -E \pi a^{2} - \frac{ E \pi a^{2} }{ 2 } = \frac{ \sigma \pi a^{2} }{
\epsilon_{0} } $$

Solving for $\sigma / \epsilon_0$:
$$ \frac{ \sigma }{ \epsilon_0 } = -\frac{ 3 E }{ 2 } $$

\section{Electric Field from a Uniformly Charged Disk}

\subsection*{Part A}

All the electric field vectors cancel as the result of symmetry. Therefore, the
electric field at point $P$ is zero.

\subsection*{Part B}

Using the electric field due to a charged ring calculated in problem
1:
$$ E_{z} = \frac{ 1 }{ 4 \pi \epsilon_0 } \frac{ z dq }{ \left( r^{2} + z^{2}
\right)^{3/2} } $$

The $dq$ can be calculated:
$$ dq = 2 \pi \sigma r \dr $$

Substituting this into the electric field equation:
$$ E_{z} = \frac{ \sigma r dr  }{ 2 \epsilon_0 \left( r^{2} + z^{2} \right)^{3/2} } $$

Integrating this electric field due to a ring across all infinitesimal rings
between $R_{1}$ and $R_{2}$:
$$ \vec{E} = \int\limits_{R_{1}}^{R_{2}} \frac{ \sigma r dr }{ 2 \epsilon_{0}
\left( r^{2} + z^{2} \right)^{3/2} } = \frac{ \sigma z }{ 2 \epsilon_0 } \left(
\frac{ 1 }{ \sqrt{R_{1}^{2} + z^{2}} } - \frac{ 1 }{ \sqrt{R^{2}_{2} + z^{2}}
}\right) \hat{z} $$

\section{Non-Uniformly Charged Cylinder}

\subsection*{Part A}

Define a Gaussian cylinder of radius $r$, where $r < R$, and length $l$.

\bigbreak

Calculated the charge enclosed by the Gaussian surface by integrating
infinitesimal shells multiplied by the charge density function:
$$ Q_{enc} = \int\limits_{0}^{r} \frac{ 2 \pi \rho_{0} l r^{2} dr }{ R } = \frac{ 2
\pi \rho_{0} l r^{3}}{ 3 R }$$

Define Gauss's Law:
$$ \iint \vec{E} \cdot d\vec{A} = \frac{ 2 \pi \rho_{0} l r^{3}}{ 3 R
\epsilon_{0} }$$

Electric field only exists perpendicular to the curved surface of the cylinder:
$$ 2 E \pi l r = \frac{ 2 \pi \rho_{0} l r^{3}}{ 3 R \epsilon_{0} } $$

Solving for $E$:
$$ \vec{E} = \frac{ \rho_{0} r^{2} }{ 3 R \epsilon_0 } \hat{r} $$

\subsection*{Part B}

Define a Gaussian cylinder of radius $r$, where $r > R$, and length $l$.

\bigbreak

Calculate the charge enclosed using the previously calculated formula, except
$r = R$:
$$ Q_{enc} = \frac{ 2 \pi \rho_{0} l R^{2}}{ 3 }$$

Define Gauss's Law:
$$ \iint \vec{E} \cdot d\vec{A} = \frac{ 2 \pi \rho_{0} l R^{2}}{ 3 \epsilon_{0}
}$$

Electric field only exists perpendicular to the curved surface of the cylinder:
$$ 2 E \pi l r = \frac{ 2 \pi \rho_{0} l R^{2}}{ 3 \epsilon_{0} } $$

Solving for $E$:
$$ \vec{E} = \frac{ \rho_{0} R^{2} }{ 3 \epsilon_{0} r } \hat{r} $$

\section{Superposition Principle of Gauss's Law}

\subsection*{Part A}

Define two Gaussian surfaces. One Gaussian cube of sides $2a$ centered at the
origin. One Gaussian sphere of radius $5a$ centered at $x = 6a$.

\bigbreak

Calculate charge enclosed within the cube:
$$ Q_{enc, 1} = \rho_{1} (2a)^{3} $$

Calculate charge enclosed within the sphere:
$$ Q_{enc, 2} = \rho_{2} \frac{ 4 }{ 3 } \pi (2a)^{3} $$

Define Gauss's Law for surface one (note electric field exits only two sides):
$$ 2 \vec{E}_{1} (2a)^{2} = \frac{ \rho_{1} (2a)^{3}}{\epsilon_0} $$

Define Gauss's Law for surface two:
$$ 4 \vec{E}_{2} \pi (5a)^{2} = \frac{ \rho_{2} 4 \pi (2a)^{3} }{3 \epsilon_0}
$$

Solving for $\vec{E}_{1}$:
$$ \vec{E}_{1} = \frac{ \rho_{1} a }{ \epsilon_0 } \hat{i}$$

Solving for $\vec{E}_{2}$:
$$ \vec{E}_{2} = -\frac{ 8 \rho_{2} a }{ 75 \epsilon_0 } \hat{i} $$

Summing both electric field vectors:
$$ \vec{E} = \frac{ a }{ \epsilon_0 }\left( \rho_{1} - \frac{ 8 \rho_{2} }{ 75 }
\right) \hat{i} $$

\subsection*{Part B}

Define a Gaussian cylinder of length $12 a$ and radius $b$ centered at the origin.

\bigbreak

Calculate the charge enclosed:
$$ Q_{enc} = 4 \rho_{1} \pi a b^{2} $$

Define Gauss's Law:
$$ 2 \vec{E} \pi b^{2} = \frac{ 4 \rho_{1} \pi a b^{2} }{ \epsilon_0 } $$

Solving for $E$:
$$ \vec{E} = \frac{ 2 \rho_{1} a }{ \epsilon_0 } \hat{i} $$

\subsection*{Part C}

Define two Gaussian surfaces. One Gaussian cylinder of length $14a$ and radius
$b$ centered around the origin. One Gaussian sphere of radius $a$ centered at $x
= 6a$.

\bigbreak

Charge enclosed in the cylinder is the same as before:
$$ Q_{enc, 1} = 4 \rho_{1} \pi a b^{2} $$

Calculate the charge enclosed in the sphere:
$$ Q_{enc, 2} = \frac{ 4 \rho_{2} \pi a^{3} }{ 3 } $$

Gauss's Law for the cylinder is the same as before:
$$ 2 \vec{E}_{1} \pi b^{2} = \frac{ 4 \rho_{1} \pi a b^{2} }{ \epsilon_0 } $$

Define Gauss's Law for the sphere:
$$ 4 \vec{E}_{2} \pi a^{2} = \frac{ 4 \rho_{2} \pi a^{3} }{ 3 \epsilon_0 } $$

$\vec{E}_{1}$ is the same as it was before:
$$ \vec{E}_{1} = \frac{ 2 \rho_{1} a }{ \epsilon_0 } \hat{i} $$

Solving for $\vec{E}_{2}$:
$$ \vec{E}_{2} = \frac{ \rho_{2} a }{ 3 \epsilon_0 } \hat{i} $$

Summing both electric field vectors:
$$ \vec{E} = \frac{ a }{ \epsilon_0 }\left( 2 \rho_{1} + \frac{ \rho_{2} }{ 3 }
\right) \hat{i} $$

\end{document}

