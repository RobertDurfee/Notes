\documentclass{article}
\usepackage{tikz}
\usepackage{float}
\usepackage{enumerate}
\usepackage{amsmath}
\usepackage{bm}
\usepackage{indentfirst}
\usepackage{siunitx}
\usepackage[utf8]{inputenc}
\usepackage{graphicx}
\graphicspath{ {Images/} }
\usepackage{float}
\usepackage{mhchem}
\usepackage{chemfig}
\allowdisplaybreaks

\title{ 8.02 Problem Set 8 }
\author{ Robert Durfee }
\date{ April 10, 2018 }

\begin{document}

\maketitle

\section{Rotating Coil }

Flux through a rotating coil in a uniform field:
$$ \Phi = N \pi R^2 B \cos \phi $$

Taking the derivative of the flux:
$$ \frac{d\Phi}{dt} = - N \pi R^2 B \sin \phi \frac{d\phi}{dt} $$

Electromotive force generated from a magnetic field:
$$ \varepsilon = - \frac{d\Phi}{dt} $$

Substituting for derivative of flux:
$$ \varepsilon = N \pi R^2 B \sin \phi \frac{d\phi}{dt} $$

Substituting $\omega$ for $d\phi/dt$:
$$ \varepsilon = N \pi R^2 B \omega \sin (\omega t) $$

The maximum electromotive force occurs when $\sin (\omega t) = 1$:
$$ \varepsilon_{max} = N \pi R^2 B \omega $$

Substituting provided values (and converting to radians per second):
$$ \varepsilon_{max} = 1.705\ \si{V} $$

\section{Self Inductance, Energy, Induced Electric Fields, Faraday's Law}

\subsection*{Part A}

Ampere's Law:
$$ \oint \vec{B} \cdot d\vec{s} = \mu_0 \iint \vec{J} \cdot d\vec{A} $$

Choosing a square Amperian loop of length $w$ with one edge inside the solenoid,
one outside, and two between loops, both magnetic field and current are
constant:
$$ B w = \frac{\mu_0 w I(t) N}{\ell} $$

Current after $t = 2$ is $I(2) = 2b$:
$$ B = \frac{2 \mu_0 b N}{\ell} $$

Substituting values:
$$ B = 2.01 \cdot 10^{-4}\ \si{T} $$

\subsection*{Part B}

\begin{figure}[H]
  \centering
  \includegraphics[scale=0.5]{"Solenoid"}
  \caption{Solenoid with Edge Effectts
\end{figure}

\subsection*{Part C}

Magnetic field in solenoid:
$$ B = \frac{ \mu_0 I N}{\ell} $$

Flux through solenoid:
$$ \Phi = N B A $$

Substituting for $B$ and $A$:
$$ \Phi = \frac{ \pi \mu_0 I N^2 a^2}{\ell} $$

Self-inductance is given by:
$$ L = \frac{\Phi}{I} $$

Substituting into inductance equation:
$$ L = \frac{ \pi \mu_0 N^2 a^2}{\ell} $$

Substituting values:
$$ L = 1.42 \cdot 10^{-4}\ \si{H} $$

\subsection*{Part D}

Rate of doing work against the back electromotive force:
$$ \frac{dW}{dt} = - \varepsilon I $$

Electromotive force:
$$ \varepsilon = -L \frac{dI}{dt} $$

Combining equations:
$$ \frac{dW}{dt} = -L I \frac{dI}{dt} $$

Canceling the $dt$ term:
$$ dW = -L I dI $$

Integrating both sides:
$$ \int\limits_0^W dW = L \int\limits_0^I I dI $$
$$ W = \frac{L I^2}{2} $$

\subsection*{Part E}

Stored energy in a magnetic field:
$$ U = \frac{1}{2 \mu_0} \int B^2 dV $$

Since $B$ is constant within the solenoid,
$$ U = \frac{B^2 V}{2 \mu_0} $$

Substituting magentic field and volume of cylinder:
$$ U = \frac{\pi \mu_0 N^2 a^2 I^2}{2 \ell} $$

Separating parts:
$$ U = \frac{LI^2}{2} $$

This result is consistent with Part D as the work required to reach the steady
current should be equal to the energy stored in the magnetic field. This comes
from the conservation of energy (change in potential energy is equal to the work
done for conservative forces).

\subsection*{Part F}

Faraday's Law:
$$ \oint \vec{E} \cdot d\vec{s} = -\frac{d}{dt}\iint \vec{B} \cdot d\vec{A} $$

Electric field is constant along a ring within the solenoid whose face is normal
to the axis. Within this ring, magnetic field is also constant.
$$ 2 \pi r E = -\frac{\pi \mu_0 r^2 N}{\ell} \frac{dI}{dt} $$

Solving for $E$:
$$ \vert E \vert = \frac{\mu_0 N r b}{2 \ell} $$

\section{Mutual Inductance}

\subsection*{Part A}

Magnetic field from outer solenoid:
$$ B = \frac{\mu_0 N_2 I_2}{b_2} $$

Flux from outer solenoid through inner solenoid:
$$ \Phi_{12} = B_2 N_1 A_1 $$

Substituting magnetic field and area:
$$ \Phi_{12} = \frac{\pi \mu_0 N_1 N_2 I_2 a_1^2}{b_2} $$

Mutual inductance:
$$ M_{12} = \frac{\Phi_{12}}{I_2} $$

Substituting flux:
$$ M_{12} = \frac{\pi \mu_0 N_1 N_2 a_1^2}{b_2} $$

\subsection*{Part B}

Multiply mutual inductance by current to get flux:
$$ \Phi_{21} = \frac{\pi \mu_0 N_1 N_2 a_1^2 I_1}{b_2} $$

\section{Self-Inductance of Two Wires}

Magnetic field from current carrying wire:
$$ B = \frac{\mu_0 I}{2 \pi r} $$

Integrate over distance between wires for flux (both wires add fields together):
$$ \Phi = 2 \ell \int_a^{d-a} \frac{\mu_0 I}{2 \pi r} dr = \frac{\mu_0 I
\ell}{\pi} \ln\left(\frac{d - a}{a}\right) $$

Inductance is defined:
$$ L = \frac{\Phi}{I} $$

Substituting for flux:
$$ L = \frac{\mu_0 \ell}{\pi} \ln\left(\frac{d - a}{a}\right) $$

\section{Solenoid}

\subsection*{Part A}

Magnetic field in a solenoid:
$$ B = \mu_0 n I $$

Flux through solenoid:
$$ \Phi = \pi \mu_0 n a^2 I $$

Taking the derivative of flux:
$$ \frac{d\Phi}{dt} = \pi \mu_0 n a^2 \frac{dI}{dt} $$

Electromotive force:
$$ \varepsilon = -\frac{d\Phi}{dt} $$

Substituting derivative of flux:
$$ \varepsilon = -\pi \mu_0 n a^2 \frac{dI}{dt} $$

Ohm's Law:
$$ \varepsilon = IR $$

Substituting for electromotive force and solving for current:
$$ I = -\frac{\pi \mu_0 n h a^2}{R} $$

The current is generated counterclockwise when looking from the right.

\subsection*{Part B}

Faraday's Law:
$$ \oint \vec{E} \cdot d\vec{s} = -\frac{d}{dt}\iint \vec{B} \cdot d\vec{A} $$

Electric field is constant along a ring within the solenoid whose face is normal
to the axis. Within this ring, magnetic field is also constant.
$$ 2 \pi r E = - \pi \mu_0 n r^2 \frac{dI}{dt} $$

Solving for $E$:
$$ E = -\frac{\mu_0 n r h}{2} $$

The electric field is counterclockwise when looking from the right.

\subsection*{Part C}

Change in flux will be equal to twice the constant flux:
$$ \Delta \Phi = 2 \Phi $$

Substituting flux:
$$ \Delta \Phi = 2 \pi \mu_0 n I a^2 $$

Electromotive force:
$$ \varepsilon = -\frac{\Delta \Phi}{\Delta t} $$

Ohm's Law:
$$ \varepsilon = I R $$

Solving for current:
$$ I = \frac{\Delta \Phi}{R \Delta t} $$

Converting to charge:
$$ \frac{\Delta Q}{\Delta t} = I $$

Combining together:
$$ \Delta Q = \frac{2 \pi \mu_0 n I a^2}{R} $$

\section{Rail Gun}

\subsection*{Part A}

Flux through circuit:
$$ \Phi = x(t) w B $$

Induced electromotive force:
$$ \varepsilon_{ind} = - v(t) w B $$

Induced current:
$$ \vert I_{ind} \vert = \frac{v(t) w B}{R} $$

The current flows clockwise.

\subsection*{Part B}

Total current is equal to current from generator minus induced current:
$$ I_{T} = \frac{\varepsilon - v(t) w B}{R} $$

\subsection*{Part C}

Magnetic force acting on a current carrying wire:
$$ F_m = \vec{I} w \times \vec{B} $$

Substituting for current and magnetic field:
$$ F_m = \frac{\varepsilon w B - v(t) w^2 B^2}{R} $$

\subsection*{Part D}

Newton's Second Law differential equation:
$$ \varepsilon w B - v(t) w^2 B^2 = m R \frac{d v(t)}{dt} $$

Solution to differential equation:
$$ v(t) = \frac{\varepsilon}{w B} - \frac{\varepsilon}{w B} e^{-\frac{w^2 B^2}{m
R} t} $$

Terminal velocity ($t \rightarrow \infty$):
$$ v_{term} = \frac{\varepsilon}{w B} $$

\subsection*{Part E}

Substituting values for terminal velocity:
$$ v_{term} = 20\ \si{km\ s^{-1}} $$

\subsection*{Part F}

Power from generator:
$$ P = I \varepsilon $$

Substituting equations from above:
$$ P = \frac{\varepsilon - v w B}{R} \varepsilon $$

Substituting values:
$$ P = 0 $$

Power lost to heat:
$$ P = F v $$

Substituting equations from above:
$$ P = \frac{\varepsilon w B - v w^2 B^2}{R} v $$

Substituting values:
$$ P = 0 $$

This is consistent because the car is no longer accelerating. Therefore no
power is generated. Since no power is generated, none can be lost from heat.

\end{document}

