\documentclass{article}
\usepackage{tikz}
\usepackage{float}
\usepackage{enumerate}
\usepackage{amsmath}
\usepackage{amsthm}
\usepackage{bm}
\usepackage{indentfirst}
\usepackage{siunitx}
\usepackage[utf8]{inputenc}
\usepackage{graphicx}
\graphicspath{ {Images/} }
\usepackage{float}
\usepackage{mhchem}
\usepackage{chemfig}
\allowdisplaybreaks

\title{8.02 Problem Set 9}
\author{Robert Durfee}
\date{April 24, 2018}

\begin{document}

\maketitle

\section{Problem 1}

\subsection*{Part A}

The power (and energy) produced by the battery, due to conservation of energy,
must be consumed by the battery and the capacitor. Therefore,
$$ P_{bat} = P_{res} + P_{cap} $$

The energy stored in a capacitor is given by,
$$ U_{cap} = \frac{1}{2} \frac{Q(t)^2}{C} $$
So the power can be represented as the derivative of this with respect to time,
$$ P_{cap}(t) = \frac{\partial}{\partial t} \left( \frac{1}{2} \frac{Q(t)^2}{C}
\right) $$
where the charge at any given time (since the capacitor is charging) is given
by,
$$ Q(t) = C \varepsilon \left(1 - e^{-t/RC} \right) $$
Therefore, the power absorbed by a capacitor at a certain time $t$ is given by,
$$ P_{cap}(t) = \frac{\varepsilon^2}{R} \left( -e^{-2t/RC} + e^{t/RC} \right) $$

Now, we can solve the conservation of power equation for the power dissipated by
the resistor given that the battery supplies $P_{bat}(t) = I(t) \varepsilon$,
$$ P_{res}(t) = I(t) \varepsilon - \frac{\varepsilon^2}{R} \left( -e^{-2t/RC} +
e^{t/RC} \right) $$
But since current supplied by the battery depends on time as well, it is given,
$$ I(t) = I_0 e^{-t/RC} $$
where the initial current is given as $\varepsilon/R$ as it is like the
capacitor is just a regular wire. So this can be substituted to get the entire
equation for power through resistor:
$$ P_{res}(t) = \frac{\varepsilon^2}{R} \left( e^{-t/RC} + e^{-2t/RC} - e^{t/RC}
\right) $$

Since power is equivalent to the time derivative of energy, 
$$ P_{res}(t) = \frac{\partial E(t)}{\partial t} = \frac{\varepsilon^2}{R} \left(
e^{-t/RC} + e^{-2t/RC} - e^{t/RC} \right) $$
By separating variable, the total energy can be found as $t$ reaches infinity,
$$ U_{res} = \int\limits_0^{\infty} \frac{\varepsilon^2}{R} \left( e^{-t/RC} +
e^{-2t/RC} - e^{t/RC} \right) dt = \frac{C \varepsilon^2}{2} $$ 

\subsection*{Part B}

The energy stored in a capacitor is,
$$ U_{cap} = \frac{1}{2} \frac{Q^2}{C} $$
So at a time $t = T$, the capacitor is fully charged so
$$ Q = C \varepsilon $$
Thus, the total energy stored in the capacitor is,
$$ U_{cap} = \frac{C \varepsilon^2}{2} $$

This value is the same which was calculated in the previous part. Therefore, the
total energy dissipated through the resistor is the same as the stored energy in
the capacitor.

\subsection*{Part C}

In this case, it is simpler as there is no battery, only the capacitor and the
resistor. As a result, all the energy in the capacitor will be dissipated
through the only element, the resistor. From the previous part, the total
energy in the capacitor is,
$$ U_{cap} = \frac{C \varepsilon^2}{2} $$
So the total energy dissipated through the resistor will the be same,
$$ U_{res} = \frac{C \varepsilon^2}{2} $$

\subsection*{Part D}

To calculate the average power dissipated, take the energy in the discharging
cycle and add it to the charging cycle and divide by two times the time span,
$$ P_{avg} = \frac{1}{T} \left( \frac{C \varepsilon^2}{2} \right) $$

\section*{Problem 2}

\subsection*{Part A}

The currents, by Kirchoff's Junction Rule are related by:
$$ I_2 = I_1 + I_3 $$

\subsection*{Part B}

By traversing the left loop counterclockwise, Kirchoff's Loop Rule gives:
$$ 0 = I_3 R - I_1 R + \varepsilon - \frac{Q}{C} $$

\subsection*{Part C}

By traversing the right loop counterclockwise, Kirchoff's Loop Rule gives:
$$ 0 = -I_3 R - I_2 R + \varepsilon $$

\subsection*{Part D}

A capacitor acts like a break in a circuit after a long time, therefore not
current can flow through this part of the circuit,
$$ I_1 = 0 $$

\subsection*{Part E}

Ohm's law gives,
$$ \varepsilon = I R $$
Since the total resistance is $2R$, as the left loop contributes nothing to the
circuit, rearranging this will give us the current for both $I_2$ and $I_3$,
$$ I_2 = I_3 = \frac{\varepsilon}{2R} $$

\subsection*{Part F}

Remembering the loop rule from the previous Part B,
$$ 0 = I_3 R - I_1 R + \varepsilon - \frac{Q}{C} $$
This can be rearranged to solve for voltage across the capacitor,
$$ V_{cap} = I_3 R - I_1 R + \varepsilon $$
And since $I_3$ and $I_1$ were solved for above, this equation becomes,
$$ V_{cap} = \frac{3 \varepsilon}{2} $$

\section*{Problem 3}

\subsection*{Part A}

When the capacitor is fully charged, it acts like a break in the circuit.
Therefore, the top arm of the circuit can be ignored. Ohm's law gives,
$$ I = \frac{V}{R_1 + R_2} $$
Substituting values provided,
$$ I = \frac{5}{3 + 10} = \frac{5}{13} $$

\subsection*{Part B}

When the capacitor is fully charged, there is no current moving though that
entire part of the circuit. Therefore, the potential is constant all along that
arm of the circuit. So the voltage across the capacitor is the same as the
voltage across $R_3$, which is given by,
$$ V_{R2} = \left( \frac{V}{R_1 + R_2} \right) R_2 $$
Substituting values provided,
$$ V_{C} = \left ( \frac{5}{3 + 10} \right) \cdot 10 = \frac{50}{13} $$

\subsection*{Part C}

Since the voltage across the capacitor and its capacitance is known, the
following equation is helpful for energy stored in a capacitor,
$$ U_C = \frac{1}{2} C V^2 $$
Substituting valeus,
$$ U_C = \frac{1}{2} (3 \cdot 10^{-3}) \left(\frac{50}{13}\right)^2 =
\frac{15}{676} $$

\section*{Problem 4}

First, define the current going through $R_1$ as $I_1$ to the left, the
current through $R_2$ as $I_2$ upwards, and the current through $R_3$ as
$I_3$ to the left. Then, using Kirchoff's Junction Rule,
$$ I_3 = I_1 + I_2 $$
As defined in the top, middle junction.

Now, traversing counterclockwise around the left half of the circuit, Kirchoff's
Loop Rule gives,
$$ 0 = \varepsilon_2 - I_2 R_2 + I_1 R_1 - \varepsilon_1 $$
And traversing counterclockwise around the right half of the circuit gives,
$$ 0 = -\varepsilon_3 + I_3 R_3 + I_2 R_2 - \varepsilon_2 $$

Solving this system of three equations yields the current through each battery:
$$ I_1 = 2\ \si{A},\ I_2 = 2\ \si{A},\ I_3 = 4\ \si{A} $$
Where each is flowing from the negative to the positive terminals through the
battery, therefore all batteries are discharging.

Using the values for each batteries voltage respectively,
$$ P_1 = 4\ \si{W},\ P_2 = 8\ \si{W},\ P_3 = 16\ \si{W} $$

\section*{Problem 5}

\begin{figure}[H]
  \centering
  \includegraphics[scale=0.5]{"Ohmmeter"}
  \caption{Ohmmeter}
\end{figure}

Using the displayed design for an ohmmeter, if the two leads are connected,
there is very little resistance in the upper arm of the circuit. Therefore,
nearly all the current will flow through this arm. So the circuit is essentially
two resistors in series, ignoring $R_2$. Since the circuit should display $50\
\si{\mu A}$, the total current in the series circuit must be $50\ \si{\mu A}$.
The total voltage is given as $1.5\ \si{V}$. The total resistance is given,
$$ R_T = R_1 + 20\ \si{\Omega} $$
Using Ohm's Law, $R_1$ can be determined,
$$ R_1 = \frac{V}{I} - 20\ \si{\Omega} = 29980\ \si{\Omega} $$

Now, let the two leads connect to a resistor of $15\ \si{\Omega}$. The total
resistance in the circuit is nearly unchanged. As a result, it can be assumed
that the total current is still $50\ \si{\mu A}$. We want the $15\ \si{\Omega}$
resistor to cause the ammeter to show $25\ \si{\mu A}$. This is half the total
current in the circuit. Therefore, the other half must go through the middle
arm. So the current through the middle arm is roughly $25\ \si{\mu A}$. Using
Ohm's Law, $R_2$ can now be determined,
$$ R_2 = \frac{V - I_T R_1}{I_2} = 40\ \si{\Omega} $$

Let the two leads connect to a resistor of $5\ \si{\Omega}$. The current in the
top arm of the circuit is given by,
$$ I_{5 \si{\Omega}} = \frac{V R_2}{25 R_2 + R_1 (25 + R_2) } = 30.7\ \si{\mu
A} $$

Let the two leads connect to a resistor of $50\ \si{\Omega}$. The current in the
top arm of the circuit is given by,
$$ I_{50 \si{\Omega}} = \frac{V R_2}{70 R_2 + R_1 (70 + R_2) } = 18.2\ \si{\mu
A} $$

This design is pretty feasible as $30\ \si{k\Omega}$ and $40\ \si{\Omega}$
resistors are pretty standard. However, my approximations result in the $15\
\si{\Omega}$ and $0\ \si{\Omega}$ resistors not lining up exactly at $25\
\si{\mu A}$ and $50\ \si{\mu A}$. Rather, they show $26.7\ \si{\mu A}$ and
$33.3\ \si{\mu A}$ respectively. As a result, the scale is a little different
than what was asked for, but should still provide a functional ohmmeter.

\end{document}
