\documentclass{article}
\usepackage{tikz}
\usepackage{float}
\usepackage{enumerate}
\usepackage{amsmath}
\usepackage{bm}
\usepackage{indentfirst}
\usepackage{siunitx}
\usepackage[utf8]{inputenc}
\usepackage{graphicx}
\graphicspath{ {Images/} }
\usepackage{float}
\usepackage{mhchem}
\usepackage{chemfig}
\allowdisplaybreaks

\title{ 8.02 Problem Set 5 }
\author{ Robert Durfee }
\date{ March 13, 2018 }

\begin{document}

\maketitle

\section*{Problem 1 }

The charge on the plates of a capacitor must remain \textbf{constant} after the
battery is removed. Since there is no longer a constant potential difference,
the charge has nowhere to go.

\section*{Problem 2}

The potential difference across the plate of the capacitor \textbf{decreases}
after the battery is removed.  There is an induced electric field that repels
the electric field between the plates. Reducing this electric field reduces the
electric potential through the following relation as distance between the plates
remained the same.
$$ \Delta V = Ed $$

\section*{Problem 3}

The electric field between the plates of the capacitor \textbf{remains the same}
when the battery remains connected. Since there is a constant voltage across the
two plates when a battery is connected, by the following relation, electric
field must also remain constant because distance between the plates remained
the same.
$$ \Delta V = E d $$

\section*{Problem 4}

\subsection*{Part A}

$0 < r < a$: For this radius range, there is no electric field. Due to the
nature of conductors, the charge will try to get as far away from each other as
possible. This will result in no net charge contained within the cable, all will
be on the surface of the cable.

\bigbreak

$a < r < b$: Calculating charge enclosed in a Gaussian cylinder, length $L$:
$$ Q_{enc} = \lambda L $$

Setting up Gauss's Law:
$$ \iint \vec{E} \cdot d\vec{A} = \frac{ Q_{enc} }{ \epsilon_{0} } $$

Solving for electric field:
$$ E = \frac{ \lambda }{ 2 \pi \epsilon_{0} r } $$

\bigbreak

$b < r < c$: The electric field will be reduced by a factor of $kappa$.
Otherwise, the electric field in this region would be the same as the previous
region.
$$ E = \frac{ \lambda }{ 2 \pi \epsilon_{0} \kappa r } $$

\bigbreak

$c < r$: Since the outside shell of the cable has equal and opposite charge of
the inner wire, there is no net charge within this region. Therefore, there is
no electric field.

\subsection*{Part B}

$a \longrightarrow b$: Using the electric field calculated for this region from
Part A:
\begin{align*}
  V(b) - V(a) &= -\int\limits_{a}^{b} \frac{ \lambda }{ 2 \pi \epsilon_{0} r }
  dr \\
              &= -\frac{ \lambda }{ 2 \pi \epsilon_{0} } \int\limits_{a}^{b}
  \frac{ dr }{ r } \\
              &= -\frac{ \lambda }{ 2 \pi \epsilon_{0} } \ln \left( \frac{
              b }{ a } \right)
\end{align*}

\bigbreak

$b \longrightarrow c$: Using the electric field calculated for this region from
Part A:
\begin{align*}
  V(c) - V(b) &= -\int\limits_{b}^{c} \frac{ \lambda }{ 2 \pi \epsilon_{0}
  \kappa r } dr \\
              &= -\frac{ \lambda }{ 2 \pi \epsilon_{0} \kappa }
  \int\limits_{b}^{c} \frac{ dr }{ r } \\
              &= -\frac{ \lambda }{ 2 \pi \epsilon_{0} \kappa } \ln \left(
  \frac{ c }{ b } \right)
\end{align*}

\bigbreak

$a \longrightarrow c$: This change in potential can be written as:
\begin{align*}
  V(c) - V(a) &= \left[V(c) - V(b)\right] + \left[V(b) - V(a)\right] \\
              &= -\frac{ \lambda }{ 2 \pi \epsilon_{0} \kappa } \ln \left(
  \frac{ c }{ b } \right) -\frac{ \lambda }{ 2 \pi \epsilon_{0} } \ln \left( \frac{
              b }{ a } \right)
\end{align*}

Since we only care about magnitude:
$$ \vert V(c) - V(a) \vert = \frac{ \lambda }{ 2 \pi \epsilon_{0}} \ln \left(
\left( \frac{ b }{ a } \right) \left( \frac{ c }{ b } \right)^{1/\kappa} \right)  $$

\subsection*{Part C}

The definition of capacitance:
$$ C = \frac{ Q }{ \Delta V } $$

Converting this to capacitance per unit length:
$$ \frac{ C }{ L } = \frac{ \lambda }{ \Delta V } $$

Substituting value for $\Delta V$ from Part B:
$$ \frac{ C }{ L } = \frac{ 2 \pi \epsilon_{0} \kappa }{ \kappa \ln\left( \frac{
b }{ a } \right) + \ln\left( \frac{ c }{ b } \right) } $$

\section*{Problem 6}

\subsection*{Part A}

Capacitance between two parallel plates is defined as:
$$ C = \frac{ \epsilon_{0} \kappa A }{ d } $$

\subsection*{Part B}

Capacitors in parallel are superposed:
$$ C = C_{1} + C_{2} $$

Capacitors in series are superposed:
$$ \frac{ 1 }{ C } = \frac{ 1 }{ C_{1} } + \frac{ 1 }{ C_{2} } $$

The capacitances for the three regions are as follows:
$$ C_{1} = \frac{ \epsilon_{0} \kappa_{1} A }{ 2 d } $$
$$ C_{2} = \frac{ \epsilon_{0} \kappa_{2} A }{ d } $$
$$ C_{3} = \frac{ \epsilon_{0} \kappa_{3} A }{ d } $$

$\kappa_{2}$ and $\kappa_{3}$ are combined in series and they together are in
parallel with $\kappa _{1}$. Thus, the total capacitance follows the formula:
$$ C = C_{1} + \left( \frac{ 1 }{ C_{2} } + \frac{ 1 }{ C_{3} } \right)^{-1} $$

Substituting values calculated above:
$$ C = \frac{ \epsilon_{0} A }{ d } \left( \frac{ \kappa_{1} }{ 2 } + \left(
\frac{ 1 }{ \kappa_{2} } + \frac{ 1 }{ \kappa_{3} } \right)^{-1} \right) $$

\end{document}

