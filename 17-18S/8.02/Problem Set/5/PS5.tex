\documentclass{article}
\usepackage{tikz}
\usepackage{float}
\usepackage{enumerate}
\usepackage{amsmath}
\usepackage{bm}
\usepackage{indentfirst}
\usepackage{siunitx}
\usepackage[utf8]{inputenc}
\usepackage{graphicx}
\graphicspath{ {Images/} }
\usepackage{float}
\usepackage{mhchem}
\usepackage{chemfig}
\allowdisplaybreaks

\title{ 8.02 Problem Set 5 }
\author{ Robert Durfee }
\date{ March 13, 2018 }

\begin{document}

\maketitle

\section*{Problem 1 }

The charge on the plates of a capacitor must remain \textbf{constant} after the
battery is removed. Since there is no longer a constant potential difference,
the charge has nowhere to go.

\section*{Problem 2}

The potential difference across the plate of the capacitor \textbf{decreases}
after the battery is removed.  There is an induced electric field that repels
the electric field between the plates. Reducing this electric field reduces the
electric potential through the following relation as distance between the plates
remained the same.
$$ \Delta V = Ed $$

\section*{Problem 3}

The electric field between the plates of the capacitor \textbf{remains the same}
when the battery remains connected. Since there is a constant voltage across the
two plates when a battery is connected, by the following relation, electric
field must also remain constant because distance between the plates remained
the same.
$$ \Delta V = E d $$

\section*{Problem 4}

\subsection*{Part A}



\end{document}

