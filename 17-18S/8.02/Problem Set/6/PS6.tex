\documentclass{article}
\usepackage{tikz}
\usepackage{float}
\usepackage{enumerate}
\usepackage{amsmath}
\usepackage{bm}
\usepackage{indentfirst}
\usepackage{siunitx}
\usepackage[utf8]{inputenc}
\usepackage{graphicx}
\graphicspath{ {Images/} }
\usepackage{float}
\usepackage{mhchem}
\usepackage{chemfig}
\allowdisplaybreaks

\title{ 8.02 Problem Set 6 }
\author{ Robert Durfee }
\date{ March 20, 2018 }

\begin{document}

\maketitle

\section*{Problem 1}

The definition of resistance is
$$ R = \frac{ \rho \ell }{ A } $$

Using this definition, we can determine the resistance of our thick shell by
assigning the surface area of a sphere to $A$ and the length $\ell$ to be the
infinitesimal change in radius. Thus,
$$ dR = \frac{ \rho dr }{ 4 \pi r^{2} } $$

Integrating between the two radii,
$$ R = \int\limits_{r_{a}}^{r_{b}} = \frac{ \rho }{ 4 \pi } \left( \frac{ 1 }{
r_{a} } - \frac{ 1 }{ r_{b} } \right) $$

\section*{Problem 2}

\subsection*{Part A}

Given that electric fields are conservative forces and no other forces are
acting on the particle when it is initially accelerated,
$$ \Delta U = -\Delta K $$

We also know, through the definition of electric potential,
$$ \Delta U = - e \Delta V $$

Combining these definition,
$$ e \Delta V = \Delta K $$

Since the particle starts from rest,
$$ \Delta K = \frac{ 1 }{ 2 } m v^2 $$

Solving for final velocity,
$$ v = \sqrt{\frac{ 2 e \Delta V}{ m }} $$

Now, since the particle travels through the magnetic and electric fields in a
straight line,
$$ \vec{F}_{E} + \vec{F}_{M} = 0 $$

Electric force on a particle in a uniform field is
$$ \vec{F}_{E} = q \vec{E} $$

Magnetic force on a moving particle is
$$ \vec{F}_{M} = q \vec{v} \times \vec{B} $$

Combining these results,
$$ q \vec{E} + q \vec{v} \times \vec{B_{1}} = 0 $$

Solving for $B$,
$$ B_{1} = \frac{ E }{ v } $$

Substituting velocity from above,
$$ B_{1} = E \sqrt{\frac{ m }{ 2 e \Delta V }} $$

Using the right hand rule,
$$ \vec{B_{1}} = E \sqrt{\frac{ m }{ 2 e \Delta V }} \hat{k} $$

\subsection*{Part B}

In the second portion of the mass spectrometer, the particle experiences
acceleration only in the angular direction. Therefore,
$$ \vec{F} = \frac{ m v^{2} }{ R } \hat{r} $$

The only force in this portion is magnetic from the second magnetic field,
$$ \vec{F}_{M} = q \vec{v} \times \vec{B_{2}} $$

Setting these two equal,
$$ \frac{ m v^{2} }{ R } = q v B_{2}$$

Solving for velocity and substituting for $R$,
$$ v = \frac{ e R B_{2} }{ m } = \frac{ e x B_{2} }{ 2 m } $$

Using our result for velocity from the previous part,
$$ \sqrt{\frac{ 2 e \Delta V}{ m }} = \frac{ e x B_{2} }{ 2 m } $$

Solving for $m$,
$$ m = \frac{ e x^{2} B_{2}^{2} }{ 8 \Delta V } $$

\section*{Problem 3}

\subsection*{Part A}

Starting with the Biot-Savart Law,
$$ d\vec{B} = \frac{ \mu_{0} }{ 4 \pi } \frac{ I d\vec{s} \times \hat{r} }{ r^{2} } $$

Noticing that $d\vec{s} \perp \hat{r}$ and that all $x$ and $y$ components of
magnetic field cancel by symmetry,
$$ d\vec{B} = \frac{ \mu_{0} }{ 4 \pi } \frac{ I ds }{ r^{2} } \sin \phi \hat{k} $$

Since $ds$ is an arc on a circle,
$$ ds = R d\theta$$

Substituting for $r$, $ds$, and $\sin \phi$,
$$ d\vec{B}_{top} = \frac{ \mu_{0} }{ 4 \pi } \frac{ I R d\theta }{ R^{2} + \left(
\frac{ \ell }{ 2 } - z \right)^{2} } \frac{ R }{ \sqrt{R^{2} + \left( \frac{
\ell }{ 2 } - z \right)^{2}} } \hat{k}$$

Integrating over the entire ring,
$$ \vec{B}_{top} = \int\limits_{0}^{2\pi} \frac{ \mu_{0} R^{2} I d\theta }{ 4 \pi
\left( R^{2} + \left( \frac{ \ell }{ 2 } - z \right)^{2} \right)^{3/2} } \hat{k}
= \frac{ \mu_{0} R^{2} I }{ 2 \left( R^{2} + \left( \frac{ \ell }{ 2
} - z \right)^{2} \right)^{3/2} } \hat{k} $$

For the bottom ring, the magnetic field will go in the reverse direction and
instead of $\ell / 2 - z$, it will be $\ell / 2 + z$,
$$ \vec{B}_{bottom} = - \frac{ \mu_{0} R^{2} I }{ 2 \left( R^{2} + \left( \frac{ \ell }{ 2
} + z \right)^{2} \right)^{3/2} } \hat{k}$$

Combining these two results,
$$ \vec{B} = \frac{ \mu_{0} R^{2} I }{ 2 } \left( \left( R^{2} + \left( \frac{
\ell }{ 2 } - z \right)^{2} \right)^{-3/2} - \left( R^{2} + \left( \frac{ \ell
}{ 2 } + z \right)^{2} \right)^{-3/2} \right) \hat{k}$$

\subsection*{Part B}

The derivative of the magnetic field is,
$$ \frac{ \partial \vec{B} }{ \partial z } = \frac{ 3 \mu_{0}R^{2} I }{ 4 } \left(
-\frac{ -R + 2z }{ \left( \frac{ 5 R^{2} }{ 4 } - Rz + z^{2} \right)^{5/2}}
+ \frac{ R + 2z }{ \left( \frac{ 5R^{2} }{ 4 } + Rz + z^{2} \right)^{5/2}}\right) \hat{k}$$

When $z = 0$, this simplifies to,
$$ \frac{ \partial \vec{B} }{ \partial z }(z = 0) = \frac{ 48 \mu_{0} I }{ 25 \sqrt{5}
R^{2} } \hat{k}$$

\section*{Problem 4}

\subsection*{Part A}

Suppose we have a curve, $\ell$, defined by the parameterization $(x(t), y(t))$
where $t$ ranges from $a$ to $b$ and $(x(a), y(a))$ defines point $A$ and
$(x(b), y(b))$ defines point $B$.

Choose a small segment of curve $\ell$ and call it $d\vec{\ell}$. The tail of
this vector is defined by $\langle x(t_{0}), y(t_{0}) \rangle$. The tip of the
vector is defined by $\langle x(t_{0} + \Delta t), y(t_{0} + \Delta t)\rangle$.
Using vector subtraction, $d\vec{\ell}$ can be expressed as
$$ d\vec{\ell} = \langle x(t_{0} + \Delta t), y(t_{0} + \Delta t) \rangle - \langle x(t_{0}),
y(t_{0}) \rangle $$

Using the equation for magnetic force on a infinitesimal current-carrying wire,
$$ d\vec{F} = I d\vec{\ell} \times \vec{B} $$

Taking the cross product,
$$ d\vec{F} = I B \left( \left( y(t_{0} + \Delta t) - y(t_{0}) \right) \hat{i} -
\left( x(t_{0} + \Delta t) - x(t_{0})\right) \hat{j} \right)$$

Using the linear approximation,
$$ x(t_{0} + \Delta t) \approx x(t_{0}) + x'(t_{0}) \Delta t $$
$$ y(t_{0} + \Delta t) \approx y(t_{0}) + y'(t_{0}) \Delta t $$

The equation simplifies to
$$ d\vec{F} = I B (y'(t) \hat{i} - x'(t) \hat{j}) dt $$

Taking the integral from $A$ to $B$,
$$ \vec{F} = \int\limits_{a}^{b} I B (y'(t) \hat{i} - x'(t) \hat{j}) dt = I B
\big\langle y(b) - y(a), -(x(b) - x(a)) \big\rangle$$

This result is independent of the path taken between $A$ and $B$.

\subsection*{Part B}

Given that the wire moves up $h$ and starts from rest,
$$ \frac{ 1 }{ 2 } m v^{2} = mgh $$

Solving for $v$,
$$ v = \sqrt{2 g h} $$

Using the definition for impulse,
$$ F dt = m dv $$
$$ F dt = m \sqrt{2 g h} $$

From our answer in Part A, we know that we can treat the arc simply as a line
segment of length $L$. Therefore, the magnetic force on the current-carrying wire is
$$ F = I L B $$

Substituting into the above equation,
$$ I L B dt = m \sqrt{2 g h} $$

From the question,
$$ Q = \int I dt $$

Therefore,
$$ Q = \frac{ m \sqrt{2 g h} }{ L B } $$

\section*{Problem 5}

\subsection*{Part A}

Using the Biot-Savart Law:
$$ d\vec{B} = \frac{ \mu_{0} }{ 4 \pi } \frac{ I d\vec{\ell} \times \hat{r} }{
r^{2} } $$

For both sections of the ring, the magnetic field is perpendicular to the radius.
Therefore,
$$ d\vec{B} = -\frac{ \mu_{0} }{ 4 \pi } \frac{ I_{1} d\ell }{ r^{2} } \hat{k} $$

Since $d\ell$ is an arc of a circle,
$$ d\ell = r d\theta $$

Substituting for the radius $R/2$,
$$ d\vec{B} = -\frac{ \mu_{0} I_{1} d\theta }{ 2 \pi R } \hat{k} $$

Integrating over 240 degrees,
$$ \vec{B} = -\int\limits_{0}^{4\pi/3} \frac{ \mu_{0} I_{1} d\theta }{ 2 \pi R }
\hat{k} = -\frac{ 2 \mu_{0} I_{1}  }{ 3 R } \hat{k} $$

Substituting for the radius $R$,
$$ d\vec{B} = -\frac{ \mu_{0} I_{1} d\theta }{ 4 \pi R } \hat{k}$$

Integrating over 120 degrees,
$$ \vec{B} = -\int\limits_{0}^{2\pi/3} \frac{ \mu_{0} I_{1} d\theta }{ 4 \pi R }
\hat{k} = - \frac{ \mu_{0} I_{1} }{ 6 R } \hat{k} $$

Combining these two magnetic fields,
$$ \vec{B} = - \frac{ 5 \mu_{0} I_{1} }{ 6 R } \hat{k} $$

\subsection*{Part B}

The magnetic field at this point is the same as was calculated in Part B. Using
the equation for magnetic force on a current-carrying wire,
$$ \vec{F} = I \vec{\ell} \times \vec{B} $$

Substituting for field found in Part A,
$$ \vec{F} = - \frac{ 5 \mu_{0} s I_{1} I_{2} }{ 6 R } \hat{i} $$

\end{document}

