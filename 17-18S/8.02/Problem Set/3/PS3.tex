\documentclass{article}
\usepackage{tikz}
\usepackage{float}
\usepackage{enumerate}
\usepackage{amsmath}
\usepackage{bm}
\usepackage{indentfirst}
\usepackage{siunitx}
\usepackage[utf8]{inputenc}
\usepackage{graphicx}
\graphicspath{ {Images/} }
\usepackage{float}
\usepackage{mhchem}
\usepackage{chemfig}
\allowdisplaybreaks

\title{ 8.02 Problem Set 3 }
\author{ Robert Durfee - Lecture 7 - Table 8 }
\date{ February 27, 2018 }

\begin{document}

\maketitle

\section*{ Problem 1 }

There are two regions to worry about: within the slab and outside the slab.

\bigbreak

First, define a Gaussian cylinder of radius $a$ and length $2x$ where $x < d$.
This will determine the electric field within the slab.

\bigbreak

Calculating charge enclosed for region one:
$$ Q_{enc,1} = 2 \rho \pi a^{2} x $$

Define Gauss's law for region one:
$$ \iint \vec{E}_{1} \cdot d\vec{A} = \frac{ 2 \rho \pi a^{2} x }{ \epsilon_{0} } $$

Substituting for the surface area of the ends of the Gaussian cylinder:
$$ 2 \pi a^{2} E_{1} = \frac{ 2 \rho \pi a^{2} x }{ \epsilon_{0} } $$

Solving for $E_{1}$:
$$ E_{1} = \frac{ \rho x }{ \epsilon_{0} }$$

Second, define a Gaussian cylinder of radius $a$ and length $L$ where $L > 2d$.
This will determine the electric field outside the slab.

\bigbreak

Calculating charge enclosed for region two:
$$ Q_{enc,2} = 2 \rho \pi a^{2} d$$

Define Gauss's Law for region two:
$$ \iint \vec{E}_{2} \cdot d\vec{A} = \frac{ 2 \rho \pi a^{2} d }{ \epsilon_{0} } $$

Substituting for the surface area of the ends of the Gaussian cylinder:
$$ 2 \pi a^{2} E_{2} = \frac{ 2 \rho \pi a^{2} d }{ \epsilon_{0} } $$

Solving for $E_{2}$:
$$ E_{2} = \frac{ \rho d }{ \epsilon_{0} } $$

For region one, calculate potential:
$$ \Delta V_{1} = V(d) - V(0) = - \int\limits_{0}^{d} \frac{ \rho x }{
\epsilon_{0} } dx = -\frac{ \rho d^{2} }{ 2 \epsilon_{0} } $$

For region two, calculate potential:
$$ \Delta V_{2} = V(x_{B}) - V(d) = - \int\limits_{d}^{x_{B}} \frac{ \rho d }{
\epsilon_{0} } dx = -\frac{ \rho d }{ 2 \epsilon_{0} }\left( x_{B} - d \right)$$

Summing these two electric potentials together:
$$ V(x_{B}) - V(0) = -\frac{\rho d}{ \epsilon_{0} }\left( x_{B} - \frac{ d }{ 2 } \right) $$

\section*{Problem 2}

\subsection*{Part A}

Define a Gaussian cylinder of radius $r$, where $r > R$, and length $L$.

\bigbreak

Calculating charge enclosed in the cylinder:
$$ Q_{enc} = L \lambda $$

Define Gauss's Law for the cylinder:
$$ \iint \vec{E} \cdot d\vec{A} = \frac{ L \lambda }{ \epsilon_{0} } $$

Substituting for the surface area of the curved surface of the Gaussian
cylinder:
$$ 2 \pi L r E = \frac{ L \lambda }{ \epsilon_0 } $$

Solving for $\vec{E}$:
$$ \vec{E}(r) = \frac{ \lambda }{ 2 \pi \epsilon_{0} r } \hat{r} $$

\subsection*{Part B}

Calculating potential from electric field:
$$ \Delta V = - \int\limits_{R_{0}}^{r} \frac{ \lambda }{ 2 \pi \epsilon_{0} r }
dr = -\frac{ \lambda }{ 2 \pi \epsilon_{0} } \ln\left( \frac{ r }{ R_{0} } \right)$$

\section*{Problem 3}

\subsection*{Part A}

$$ V(x, y, z) = \frac{ kq }{ \sqrt{x^{2} + y^{2} + z^{2}} } $$

\subsection*{Part B}

$$ \left\langle \frac{ \partial V }{ \partial x }, \frac{ \partial V }{ \partial
y}, \frac{ \partial V }{ \partial z } \right\rangle = \left\langle -\frac{ kqx
}{ r^{3} }, -\frac{ kqy }{ r^{3} }, -\frac{ kqz }{ r^{3} } \right\rangle$$

\subsection*{Part C}

$$ \vec{\nabla} V = -\frac{ kq\vec{r} }{ r^{3} }$$

\subsection*{Part D}

The $\vec{\nabla} V$ points parallel to the electric field lines, but in the
opposing direction. $\vec{E} = -\vec{\nabla} V$.

\subsection*{Part E}

\begin{enumerate}[i.]
  \item The gradient operator points in the steepest ascent uphill. Therefore,
    $\vec{\nabla} V$ points in the direction of greatest increase of $V$.

  \item The sign of the charge will not impact this conclusion as positive
    charges have positive values for $V$ and negative charges have negative
    values for $V$. Although $\vec{\nabla}$ will point toward and negative
    charge and away from a positive charge, since a positive charge has higher
    $V$ and a negative charge has a lower $V$, then $\vec{\nabla} V$ is still
    pointing in the greatest increasing direction, uphill.

\end{enumerate}

\subsection*{Part F}

No, the choice of zero point for electric potential does not matter in the end
because potential only has meaning based on the difference between points. This
is the same for potential energy.

\section*{Problem 4}

\subsection*{Part A}

Region 1:
$$ \vec{E} = 0 $$

Region 2:
$$ \vec{E} = \frac{ 2 V_{0} (d + x) }{ d^{2} } \hat{i} $$

Region 3:
$$ \vec{E} = \frac{ 2 V_{0} }{ d } \hat{i} $$

Region 4:
$$ \vec{E} = 0 $$

\subsection*{Part B}

The electric field function is plotted below, with $d = 2$ cm and $V_{0} = 2$ V,
the $x$-axis in units of centimeters, and the $y$-axis in units of Volts per
meter.

\begin{figure}[H]
  \centering
  \includegraphics[scale=0.40]{"EFieldPlot"}
  \caption{Electric Field Plot}
\end{figure}

\section*{Problem 5}

\subsection*{Part A}

Once again there are two regions two worry about: within the sphere and outside
the sphere.

\bigbreak

First, define a Gaussian sphere with radius $r$ where $r \leq R$. This will
determine the electric field within the sphere.

\bigbreak

Calculating the charge enclosed by this sphere:
$$ Q_{enc,1} = \frac{ 3 Q }{ 4 \pi R^{3} } \left( \frac{ 4 \pi r^{3} }{ 3 }
\right) = \frac{ Q r^{3} }{ R^{3} } $$

Defining Gauss's Law for the first region:
$$ \iint \vec{E}_{1} \cdot d\vec{A} = \frac{ Q r^{3} }{ \epsilon_{0} R^{3} } $$

Substituting the surface area of the Gaussian sphere:
$$ 4 \pi r^{2} E_{1} = \frac{ Q r^{3} }{ \epsilon_{0} R^{3} } $$

Solving for $E_{1}$:
$$ E_{1} = \frac{ Q r }{ 4 \pi \epsilon_{0} R^{3} } $$

Using this electric field to determine the electric potential within region one:
$$ \Delta V = V_{1}(r) - V(0) = - \int\limits_{0}^{r} \frac{ Q r }{ 4 \pi \epsilon_0
R^{3}} dr $$
$$V_{1}(r) = -\frac{ Q r^{2} }{ 8 \pi \epsilon_{0} R^{3} } $$

\bigbreak

Second, define a Gaussian sphere with radius $r$ where $r > R$. This will
determine the electric field outside the sphere.

\bigbreak

Charge is completely enclosed within this sphere:
$$ Q_{enc,2} = Q $$

Defining Gauss's Law for the second region:
$$ \iint \vec{E}_{2} \cdot d\vec{A} = \frac{ Q }{ \epsilon_{0} } $$

Substituting for the surface area of the Gaussian sphere:
$$ 4 \pi r^{2} E_{2} = \frac{ Q }{ \epsilon_{0} } $$

Solving for $E_{2}$:
$$ E_{2} = \frac{ Q }{ 4 \pi \epsilon_{0} r^{2} } $$

Using this electric field to determine the electric potential within region two:
$$ \Delta V = V_{2}(r) - V(R) = - \int\limits_{R}^{r} \frac{ Q }{ 4 \pi
\epsilon_{0} } dr$$
$$ V_{2}(r) = \frac{ Q }{ 4 \pi \epsilon_0 } \left( \frac{ 1 }{ r } - \frac{ 1
}{ R } \right) + V(R) $$

Now, using the equation for region one to solve for $V(R)$
$$ V(R) = -\frac{ Q }{ 8 \pi \epsilon_0 R } $$

Substituting this value into the equation for potential in region two:
$$ V_{2}(r) = \frac{ Q }{ 4 \pi \epsilon_0 } \left( \frac{ 1 }{ r } - \frac{ 1
}{ R } \right) -\frac{ Q }{ 8 \pi \epsilon_0 R } $$

\subsection*{Part B}

Below is a plot of $V(r)$ vs $r$ where $Q$ was taken to be positive:

\begin{figure}[H]
  \centering
  \includegraphics[scale=0.40]{"ElectricPotentialPlot"}
  \caption{Electric Potential Plot}
\end{figure}

\end{document}

