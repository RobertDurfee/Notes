\documentclass{article}
\usepackage{tikz}
\usepackage{float}
\usepackage{enumerate}
\usepackage{amsmath}
\usepackage{bm}
\usepackage{indentfirst}
\usepackage{siunitx}
\usepackage[utf8]{inputenc}
\usepackage{graphicx}
\graphicspath{ {Images/} }
\usepackage{float}
\usepackage{mhchem}
\usepackage{chemfig}
\allowdisplaybreaks

\title{ 8.02 Problem Set 1 }
\author{ Robert Durfee - Lecture 7 - Table 8 }
\date{ February 12, 2018 }

\begin{document}

\maketitle

\section{ Electric Dipole Torque }

\begin{figure}[H]
  \centering
  \includegraphics[scale=0.5]{"ElectricDipoleTorque"}
  \caption{Electric Dipole Torque}
\end{figure}

\subsection*{Part A}

The magnitude of the dipole moment of this system of charges is defined as:
$$ p = qd $$

\subsection*{Part B}

The force on a charge due to an electric field is defined as:
$$ \vec{F} = q \vec{E} $$

As a result, the force felt on the upper charge is:
$$ \vec{F}_{1} = Eq \hat{i} $$

And for the lower charge:
$$ \vec{F}_{2} = -Eq \hat{i} $$

The resulting net force is the sum of all the components:
$$ \vec{F}_{net} = \vec{F}_{1} + \vec{F}_{2} = 0 $$

\subsection*{Part C}

Torque is defined as:
$$ \vec{\tau} = \vec{F} \times \vec{r} $$

As a result, the torque felt on the upper charge is:
$$ \vec{\tau}_{1} = -\frac{ Eqd \sin(\theta) }{ 2 } \hat{k} $$

And for the lower charge:
$$ \vec{\tau}_{2} = -\frac{ Eqd \sin(\theta) }{ 2 } \hat{k} $$

The resulting net torque is the sum of all the components:
$$ \vec{\tau}_{net} = \vec{\tau}_{1} + \vec{\tau}_{2} = -Eqd \sin{\theta}
\hat{k} $$

Substituting the value for dipole moment, $p$:
$$ \vec{\tau}_{net} = -Ep \sin(\theta) \hat{k} $$

\subsection*{Part D}

\begin{figure}[H]
  \centering
  \includegraphics[scale=0.75]{"GrassSeedRepresentationB"}
  \caption{Grass Seed Representation B}
\end{figure}

Representation B is the correct grass seed diagram because the external electric
field lines must connect with the field lines on the negative charge but go
around the positive charge. This is because field lines start at a positive and
terminate at a negative, or infinity.

\subsection*{Part E}

Since there is only a net torque and no net force, there will be rotational
motion only and no translational motion.

\subsection*{Part F}

\begin{figure}[H]
  \centering
  \includegraphics[scale=0.5]{"NonUniformElectricField"}
  \caption{Non-Uniform Electric Field}
\end{figure}

\begin{figure}[H]
  \centering
  \includegraphics[scale=0.74]{"GrassSeedRepresentationC"}
  \caption{Grass Seed Representation C}
\end{figure}

Representation C is the correct grass seed diagram. Since $Q$ has a stronger
charge, its field lines are more difficult to bend.  Additionally, since the
positive charge $q$ is closest to $Q$, the field lines from $Q$ will not
connect, rather, they will go around the dipole.

\subsection*{Part G}

Since $d \ll x$, we can calculate electric field at both charges in the dipole
using linear approximation.
$$ \vec{E}_{-}\left(x + \frac{ d }{ 2 } \right) = \vec{E}(x) + \frac{ d }{ 2 }
\left( \frac{ d\vec{E} }{ dx } \right) $$
$$ \vec{E}_{+}\left(x - \frac{ d }{ 2 } \right) = \vec{E}(x) - \frac{ d }{ 2 }
\left( \frac{ d\vec{E} }{ dx } \right) $$

Using the definition of force from electric field:
$$ \vec{F} = q(\vec{E}_{+} - \vec{E}_{-}) = -qd \left( \frac{ dE }{ dx } \right)
\hat{i} $$

The electric field due to the large source charge $Q$ is:
$$ \vec{E} = \frac{ k_{e} Q }{ x^{2} } \hat{r} $$

The derivative of the electric field due to $Q$ is:
$$ \frac{ dE }{ dx } = - \frac{ 2 k_{e} Q }{ x^{3} } $$

Substituting the derivative into the force equation yields:
$$ \vec{F} = \frac{ 2qdk_{e}Q }{ x^{3} } \hat{i} $$

Substituting the value for dipole moment, $p$:
$$ \vec{F} = \frac{ 2pk_{e}Q }{ x^{3} } \hat{i} $$

\section{Field of Two Point Charges}

\begin{figure}[H]
  \centering
  \includegraphics[scale=0.50]{"FieldOfTwoPointCharges"}
  \caption{Field of Two Point Charges}
\end{figure}

\subsection*{Part A}

Representation 2 is the correct grass seed diagram because the stronger charge
is on the left and its field lines are more difficult to bend. Also, the charges
have opposite polarity so the field lines will connect between the charges.

\begin{figure}[H]
  \centering
  \includegraphics[scale=0.75]{"GrassSeedRepresentationTwo"}
  \caption{Grass Seed Representation 2}
\end{figure}

\subsection*{Part B}

The electric field at any point along the positive x-axis from the two different
charges can be written as:
$$ \vec{E}_{1} = -\frac{ 2k_{e}q }{ x^{2} } \hat{i} $$
$$ \vec{E}_{2} = \frac{ k_{e}q }{ (x - 1)^{2} } \hat{i} $$

The total electric field at any point becomes:
$$ \vec{E}_{net} = \left[ \frac{ k_{e}q }{ (x - 1)^{2} } - \frac{ 2k_{e}q }{
x^{2} } \right] \hat{i} = 0$$

Setting this equal to zero yields one sensible root:
$$ x = 2 + \sqrt{2} $$

\section{Balancing Forces}

\subsection*{Part A}

Note: we can ignore the positive charge located at the axis because any forces
acting upon it will produce no torque because $r = 0$ and any forces it supplies
will produce no torque because it will be parallel to the radius vector.

\begin{figure}[H]
  \centering
  \includegraphics[scale=0.50]{"BalancingForces"}
  \caption{Balancing Forces}
\end{figure}

The force acting on the negative charge from the positive charge below:
$$ \vec{F}_{q} = \frac{ k_{e} q^{2} }{ d^{2} } \hat{r} $$

$d$ is equal to the distance between the two charges:
$$ d = \sqrt{l^{2} + h^{2}} $$

The force then becomes:
$$ \vec{F}_{q} = \frac{ k_{e} q^{2} }{ l^{2} + h^{2} } \hat{r} $$

Using the definition for torque, the torque supplied on the negative charge
from the positive charge below can be written:
$$ \vec{\tau}_{q} = F_{q} l \sin(\theta) \hat{k} $$

Substituting the value for $\sin(\theta)$:
$$ \vec{\tau}_{q} = \frac{ k_{e}lhq^{2} }{ \left( l^{2} + h^{2} \right)^{3/2} }
\hat{k}$$

Calculating the torque due to the hanging mass:
$$ \vec{\tau}_{m} = -mgx \hat{k} $$

Net torque must be zero since the beam is not rotating:
$$ \frac{ k_{e}lhq^{2} }{ \left( l^{2} + h^{2} \right)^{3/2} } = mgx $$

Solving for $x$:
$$ x = \frac{ k_{e}lhq^{2} }{ mg \left( l^{2} + h^{2} \right)^{3/2} } $$

\subsection*{Part B}

\begin{figure}[H]
  \centering
  \includegraphics[scale=0.75]{"GrassSeedRepresentationD"}
  \caption{Grass Seed Representation D}
\end{figure}

Representation D is the correct grass seed diagram because the two positive
charges' field lines will not connect as they have the same polarity. However,
the field lines from both positive charges will connect to the negative charge
as they have opposite polarity.

\end{document}

