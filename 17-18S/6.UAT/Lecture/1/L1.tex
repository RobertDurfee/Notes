\documentclass{article}
\usepackage{tikz}
\usepackage{float}
\usepackage{enumerate}
\usepackage{amsmath}
\usepackage{bm}
\usepackage{indentfirst}
\usepackage{siunitx}
\usepackage[utf8]{inputenc}
\usepackage{graphicx}
\graphicspath{ {Images/} }
\usepackage{float}
\usepackage{mhchem}
\usepackage{chemfig}
\allowdisplaybreaks

\title{ 6.UAT Lecture 1 }
\author{ Robert Durfee }
\date{ February 7, 2018 }

\begin{document}

\maketitle

\section{ Beginnings }

MIT, in the past, has had difficulty in preparing their students for
communications requirements in the workforce. This class is designed to help
fulfill that requirement.

\section{ Introductions }

There are three main things that are important in creating a good introduction.
First of all, \textbf{content} is important. To make good content, you should
have a \textbf{hook}. This will gain peoples interest. The content should
connect to your audience, establish common ground, and perhaps have humor.
Another tool to use is \textbf{rhetoric}.

The second important part of an introduction is \textbf{delivery}. This can be
strong by using enthusiasm, confidence, eye contact, and even a smile. Also, be
sure to take a \textbf{beat} before you begin your speech.

Lastly, you must have a unique \textbf{storyboard}. This is an easy way to make
your introduction memorable. You don't have to start with your name.

When you complete the assignment for tonight, you should make it memorable, but
not too complicated. For example, you should not use props that you wouldn't
already have with you when you introduce yourself. Also, you can use your
research to describe yourself, that is part of you.

Be careful with making your introduction specific to your audience. There could
be different scenarios where different levels of detail can be used. You can
lose information when simplifying, however, this information wouldn't be
understood by your audience anyway.

\bigbreak

\textit{Read the comics about jargon and narrative.}

\section{ Advice }

The introduction is the most important part of your talk. At that moment, you
have people's attention. You should not waste this moment saying your name and
reading the title of your talk. Do something different.

\bigbreak

\textit{Watch the second case study about using a narrative.}

\bigbreak

It is very important to establish common ground. Try to make sure that people
have a reason to pay attention.

\section{ Class Administratrivia }

There are no laptops allowed in class.

\end{document}

