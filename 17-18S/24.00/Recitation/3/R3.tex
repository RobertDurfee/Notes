\documentclass{article}
\usepackage{tikz}
\usepackage{float}
\usepackage{enumerate}
\usepackage{amsmath}
\usepackage{bm}
\usepackage{indentfirst}
\usepackage{siunitx}
\usepackage[utf8]{inputenc}
\usepackage{graphicx}
\graphicspath{ {Images/} }
\usepackage{float}
\usepackage{mhchem}
\usepackage{chemfig}
\allowdisplaybreaks

\title{ 24.00 Recitation 3 }
\author{ Robert Durfee }
\date{ February 23, 2018 }

\begin{document}

\maketitle

\section{ Writing Philosophy }

There is a section in the textbook about framing philosophy papers. You can also
Google, "Philosophy paper writing with Pryor."

Some important aspects include not spending a lot of time summarizing the works
we studied in class. Essentially, these papers will be four pages long.

The first half page should be introduction. There should be no flowery language.
The first sentence should simply be, "In this essay, I will argue that..." Then
you should state the conclusion found at the termination of the paper. After
this, state, "My argument for $P$ is this..." Then outline the argument. Do not
be afraid of "I".

The next page and a half should be the summary of the previous works studied. No
longer than a page and a half. To do a really good job, paraphrase. Do not use a
lot of block quotes or direct quotes. Still not where in the text you are
paraphrasing.

The next two pages should be your own contribution. Do not use any quotes. These
words should all be original thoughts. If you want to go above and beyond,
supply counter examples. There may not be enough space, but that is okay.

There should be no conclusion. There absolutely is no space for that in a paper
of this length.

\section{What is Knowledge?}

There is a difference between knowing and true belief. You can have true belief
with no reason for belief. There is no good reason. This could come about
through chance. Is there a distinction between thinking you know something
versus actually knowing something? So, clearly, true belief is not the only
requirement. There must be a good reason or justification behind the true
belief.

\section{Gettier Cases}

Take an accurate clock. You read noon, and you have very good reason. The clock
has never been wrong before. Now, consider that the clock breaks at noon the
previous day. You just happen to read that it says noon when it really is noon.
But, one would argue, that you do now really \textit{know} it is noon.

But why not? If would've been really easy for me to have been wrong. I could've
check a moment sooner or later. This adds a new component: safety.

\end{document}

