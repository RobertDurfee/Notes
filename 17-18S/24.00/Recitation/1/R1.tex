\documentclass{article}
\usepackage{tikz}
\usepackage{float}
\usepackage{enumerate}
\usepackage{amsmath}
\usepackage{bm}
\usepackage{indentfirst}
\usepackage{siunitx}
\usepackage[utf8]{inputenc}
\usepackage{graphicx}
\graphicspath{ {Images/} }
\usepackage{float}
\usepackage{mhchem}
\usepackage{chemfig}
\allowdisplaybreaks

\title{ 24.00 Recitation 1 }
\author{ Robert Durfee }
\date{ February 9, 2018 }

\begin{document}

\maketitle

\section{ Philosophy of Religion }

The philosophy of religion is not a big part of professional philosophy.
However, is has significant common ground. As a result, it is a great place to
start studying philosophy for the first time because everyone has been exposed
to the ideas before.

\section{Ontological Argument}

This argument ins interesting. It suggests that people can figure things out
simply by thinking about the logically, not by using the scientific method.
These conclusions can be just as powerful.

The ontological argument can be summarized into one sentence: We can conceive
God, which makes him less than perfect, so he must exists in reality which is
more perfect.

\subsection{The Argument}

\subsubsection*{God exists in the mind.}

We know what God is supposed to be. That is, we can conceive of God's existence.
But by thinking about God, God itself doesn't actually \textit{exist} in the
mind, rather, the concept of God only exists.

\subsubsection*{Either God exists in just mind, or exists in both mind and
reality.}

This follows directly from the first premise. It is also described in Anselm's
view of existence.

\subsubsection*{If God exists in the mind alone, he is imperfect.}

This says that it is better to exist in both rather than only one. A flaw in God
would be that he doesn't exist in reality, it would be a limit. But is this the
case? Is existence better than nonexistence? For example, if there is another
universe in our multiverse that is just empty, is this universe worse than ours?

On another note, how is perfection defined? Can there be objective perfection?

Anselm states that reality includes both being and non-being. However, should
reality really include non-being?

Is existence a quality of an object? Likely not. When you express the qualities
of an object, you don't state that it "exists". As a result, existence doesn't
really add anything to an object.

\section{Parody Arguments}

\textbf{Parody arguments} are useful in that they alter one simple aspect of an
argument and come to an outrageous conclusion. These show something is wrong,
but they don't make it clear what exactly is wrong.

\end{document}

