\documentclass{article}
\usepackage{tikz}
\usepackage{float}
\usepackage{enumerate}
\usepackage{amsmath}
\usepackage{bm}
\usepackage{indentfirst}
\usepackage{siunitx}
\usepackage[utf8]{inputenc}
\usepackage{graphicx}
\graphicspath{ {Images/} }
\usepackage{float}
\usepackage{mhchem}
\usepackage{chemfig}
\allowdisplaybreaks

\title{ 24.00 Lecture 3 }
\author{ Robert Durfee }
\date{ February 13, 2018 }

\begin{document}

\maketitle

\section{ Perfect Island Response }

There is a difference between a perfect \textit{island} verse God. An island can
be made more perfect by adding different qualities and things. God, on the other
hand, can not be made more perfect, he is already perfect in \textit{every} way.
This leads to the idea the perfect is incoherent and, thus, the first premise in
the ontological argument is violated and we cannot even conceive of God in the
mind.

\section{The Design Argument}

This design is set up with the idea of a person crossing a hearth and see and
stone and a watch. The watch must've had an artificer, but the stone, for all
intents and purposes, could've been there forever, not requiring any
explanation.

This argument attempts to prove the existence of the second God: the creator of
the universe and life.

It is important to notice that his argument does not try to be valid. It is
simply trying to say that if there is no better explanation, this must be the
explanation. An example of this type of argument follows:

\begin{enumerate}
  \item Despite years of searching, no one has seen a unicorn.
  \item As a result, unicorns do not exist.
\end{enumerate}

This is not a valid argument. It is possible for the premises to be true and the
conclusion false. This type of argument is called an \textbf{abductive
argument}. This type of argument collects observations and then choose the best
explanation for the observations. In essence, it works backwards from
traditional methods.

An important part of the design argument is that even if we didn't witness
someone creating a watch, when we find it, we would still believe that it had to
have been created by someone.

\section{The Fine Tuning Argument}

There are a specific set of constants that define the world we live in. For
example, the relational strengths of gravity and electromagnetism. If these
constants were any different, life could not exist. As a result, it would seem
that someone must've designed the universe to permit life. This is another
example of abductive argumentation. This is better than the previous, however,
because it implies the creator of the universe completely, rather than simply
life.

A common counterargument is that since we exist, there could've been one
possible outcome: life exists. Therefore, the observation is inevitable and
thus needs no explanation.

\end{document}

