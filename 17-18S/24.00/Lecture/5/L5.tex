\documentclass{article}
\usepackage{tikz}
\usepackage{float}
\usepackage{enumerate}
\usepackage{amsmath}
\usepackage{bm}
\usepackage{indentfirst}
\usepackage{siunitx}
\usepackage[utf8]{inputenc}
\usepackage{graphicx}
\graphicspath{ {Images/} }
\usepackage{float}
\usepackage{mhchem}
\usepackage{chemfig}
\allowdisplaybreaks

\title{ 24.00 Lecture 5 }
\author{ Robert Durfee }
\date{ February 22, 2018 }

\begin{document}

\maketitle

\section{ Plato's Meno }

This is a dialogue between Socrates and Meno, recorded by Plato. The text
concerns whether virtue can be taught. Our excerpt is about the nature of
knowledge.

\section{Types of Knowledge}

To know the Earth is round is considered factual or \textbf{propositional}
knowledge. Who, where, why knowledge is a different form than factual knowledge
presented above. However, it is still considered factual. Think of the answer to
the question when someone asks who, where, or why something is. The answer takes
the form, "I know that..."

On the other hand, knowing how isn't really factual. It is considered "know-how"
knowledge or \textbf{practical} knowledge. You cannot simply tell someone in a
sentence practical knowledge. For example, you can read a book on bassoon
playing, and know everything about playing bassoon, yet still not know how to
play bassoon in real life.

There is also \textbf{acquaintance} knowledge. This is when you know someone or
a place. This is different from knowing a bunch of facts about a place or a
person.

Our focus is on factual knowledge. Factual knowledge is important, this is why
we are here at MIT. It is good to know about everything. Knowing can be useful,
it can even save our lives.

\section{What is Knowledge?}

Knowledge has multiple aspects. The fact that knowledge has to be \textbf{true}
is an undeniable aspect. You simply can't know something if it isn't true. But
you also need to \textbf{believe} the truth in order for you to know it. But are
these two aspects enough? Right know, $S$ knows $P$ if and only if $S$ correctly
believes $P$. No, there needs to be a good reason, otherwise it could be from
dumb luck.

\section{Socrates}

Are true opinions any different from knowledge? Both can get you to the same
place, where you want to go. Socrates argues that knowledge is better because it
is not fleeting, it is more constant and, thus, more valuable.

\section{What is Knowledge?}

Not just true belief, not sufficient, but necessary. But what might be added?
Possibly \textbf{justification}. But is this enough? Still no. You can have
logical reasoning from incorrect facts. Then, your belief is justified, but you
still do not know. Could we simply add that true premises are required?

\end{document}

