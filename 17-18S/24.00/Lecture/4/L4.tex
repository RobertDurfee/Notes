\documentclass{article}
\usepackage{tikz}
\usepackage{float}
\usepackage{enumerate}
\usepackage{amsmath}
\usepackage{bm}
\usepackage{indentfirst}
\usepackage{siunitx}
\usepackage[utf8]{inputenc}
\usepackage{graphicx}
\graphicspath{ {Images/} }
\usepackage{float}
\usepackage{mhchem}
\usepackage{chemfig}
\allowdisplaybreaks

\title{ 24.00 Lecture 4 }
\author{ Robert Durfee }
\date{ February 24, 2018 }

\begin{document}

\maketitle

\section{ Fine Tuning Argument }

Given that we exist, our universe must be fine tuned, so this is the only
possible observation, thus there is no need to try to explain it. This is not
satisfying, though. Consider a firing squad. All bullets miss you. This is the
only possible observation you could've made, but it still doesn't explain why
the bullets missed you in the first place.

An alternative solution to this problem is the existence of a multiverse. This
is where there are infinitely many universes that exists simultaneously. If
multiverses are exist, it removed the need to explain the fine tuning, but it
doesn't directly explain.

\section{Argument from Evil}

This argument is separate from the fine tuning argument because the fine tuning
argument proves the existence of a creator while the argument deals with an all
perfect being.

\section{Types of Arguments}

\textbf{A priori} argumentation comes from logical deduction before evidence and
observation. This is common in mathematics. \textbf{A posteriori} argumentation
comes after observation and evidence. This is common in physics and chemistry.

\section{Argument from Evil}

This is an a posteriori argument.

\subsection{Argument}

\begin{enumerate}
  \item God exists.
  \item If God exists, evil will exist only if there is a good reason for it to
    exist.
  \item There is no good reason for a world with evil.
  \item But evil does exist.
  \item Thus, there cannot be a God.
\end{enumerate}

This is a valid argument, but the conclusion is false. There must be a false
premise within the argument, but which one?

\section{Different Types of Evil}

\begin{enumerate}
  \item Intentional: torture that causes victims pain.
  \item Natural human: disease, out of our control.
  \item Natural non-human: fawn dies in a forest fire.
\end{enumerate}

\section{Freewill Defense}

This argument states that got cannot prevent pain as the result of freewill
because it would restrict freewill. Thus freewill wouldn't be 'free.' The costs
of freewill is not as bad as not having freewill to begin with. The freewill
defense targets the first type of evil: intentional.

\section{Theistic Response}

In general, theists can take one of two different stances. They can try to
combat all types of evil, or they can focus on just one.

\section{Stumpe Argument}

This argument targets the third premise: there is no good reason for evil.

\subsection{Argument}

\begin{enumerate}
  \item Humans made the wrong choice using freewill. This altered the nature of
    freewill for the worse.
  \item Now humans are inclined to make evil choices over good.
  \item This change is inheritable between humans.
  \item God cannot change freewill, unless we choose his help. He uses evil
    things to make this choice easier.
  \item If we ask God's help, we can form union with God (go to heaven).
\end{enumerate}

This argument ignores the third form of evil, the natural non-human evil.

\end{document}

