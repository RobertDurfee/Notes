\documentclass{article}
\usepackage{tikz}
\usepackage{float}
\usepackage{enumerate}
\usepackage{amsmath}
\usepackage{bm}
\usepackage{indentfirst}
\usepackage{siunitx}
\usepackage[utf8]{inputenc}
\usepackage{graphicx}
\graphicspath{ {Images/} }
\usepackage{float}
\usepackage{mhchem}
\usepackage{chemfig}
\allowdisplaybreaks

\title{ 24.00 Lecture 2 }
\author{ Robert Durfee }
\date{ February 8, 2018 }

\begin{document}

\maketitle

\section{ Does God Exist? }

80\% of people have a religious affiliation. 89\% of Americans believe that
there is a God. 60\% of which are absolutely certain. Lots of people conduct
their lives under the assumption that God exists or he doesn't. If you are
wrong, you are going through life severely deluded. But how can you know which
is correct?

\section{Anselm of Canterbury}

Anselm came up with the Ontological Argument for God's existence.
\textbf{Ontology} is the study of being, of what exists.

\section{Disambiguating 'God'}

When we say 'God,' what exactly do we mean? There are many possible meanings of
God and we must be specific when arguing for or against. Here are the different
types:

\begin{enumerate}
  \item Christian, Jewish, Islamic God?
  \item Creator and designer of the universe?
  \item Source of meaning and morality?
  \item Absolutely perfect being: all wise, all powerful, all benevolent?
\end{enumerate}

These are all separate types of God. You can believe in one without the others,
or believe them all exist.

\section{Anselm's Argument}

Anselm is attempting to prove the existence of the fourth type of God: the all
perfect being. His view of existence is that there are things that exist in the
mind and there are things that exist in reality. These two intersect to form a
region of things that exist in both reality and the mind.

For example, Hogwarts and mermaids exist only in the mind, undiscovered species
and undiscovered planets exists only in reality, and you and MIT exist in both
the mind and reality.

Atheists feel that God should be in the left column, only in the mind. However,
it is more perfect to exist in reality \textit{and} the mind rather than only in
the mind or reality, separately. As a result, Anselm says God must exist in
reality.

Essentially, Anselm's argument is:

\begin{enumerate}
  \item God exists in the mind.
  \item God must exist in the mind alone, or both the mind and reality.
  \item If God exists only in the mind, he is less than perfect.
  \item God must be perfect, by definition.
  \item Therefore, God must exists in the reality and the mind.
\end{enumerate}

\section{Arguments}

An \textbf{argument} in philosophy is simply a list of sentences. An argument is
\textbf{valid} if and only if it take the form that makes it impossible for the
premises to be true and the conclusion to be false. If the above is not true,
then the argument is \textbf{invalid}. An argument is \textbf{sound} if all the
premises are actually true. Otherwise, it is \textbf{unsound}.

\end{document}

