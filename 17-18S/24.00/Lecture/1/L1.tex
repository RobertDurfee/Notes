\documentclass{article}
\usepackage{tikz}
\usepackage{float}
\usepackage{enumerate}
\usepackage{amsmath}
\usepackage{bm}
\usepackage{indentfirst}
\usepackage{siunitx}
\usepackage[utf8]{inputenc}
\usepackage{graphicx}
\graphicspath{ {Images/} }
\usepackage{float}
\usepackage{mhchem}
\usepackage{chemfig}
\allowdisplaybreaks

\title{ 24.00 Lecture 1 }
\author{ Robert Durfee }
\date{ February 6, 2018 }

\begin{document}

\maketitle

\section{ Introduction }

This is an introduction to philosophy. Philosophers try to answer deep and
fundamental questions that arise when studying life and, basically, everything.
One main lesson of philosophy is that nothing can be defined. The best way to
introduce is to simply provide examples.

This class is not structured around the history of philosophy, rather, it is
focused on a series of questions.

The readings for this class are taken from a book that is not yet for sale,
which was written by many former or current MIT professors or grad students and
edited by the professor for this class. There are a few guides in the front of
the book. These guides focus on proper argumentation, which keeps philosophy on
track. Another guide focuses on how to write proper philosophy papers.

With the readings, there are typically four questions. You should answer these
questions after you read the text. If you can't answer the questions, you should
reread the section. Some more complicated readings have a readers guide that
goes along with it, explaining the lessons to be extrapolated.

There is also a glossary provided which includes definitions for philosophy
terms.

Do not have laptops or tablets open or on during class.

\end{document}

