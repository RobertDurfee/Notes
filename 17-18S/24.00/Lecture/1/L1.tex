\documentclass{article}
\usepackage{tikz}
\usepackage{float}
\usepackage{enumerate}
\usepackage{amsmath}
\usepackage{bm}
\usepackage{indentfirst}
\usepackage{siunitx}
\usepackage[utf8]{inputenc}
\usepackage{graphicx}
\graphicspath{ {Images/} }
\usepackage{float}
\usepackage{mhchem}
\usepackage{chemfig}
\allowdisplaybreaks

\title{ 24.00 Lecture 1 }
\author{ Robert Durfee }
\date{ February 6, 2018 }

\begin{document}

\maketitle

\section{ Introduction }

This is an introduction to philosophy. Philosophers try to answer deep and
fundamental questions that arise when studying life and, basically, everything.
One main lesson of philosophy is that nothing can be defined. The best way to
introduce is to simply provide examples.

This class is not structured around the history of philosophy, rather, it is
focused on a series of questions.

The readings for this class are taken from a book that is not yet for sale,
which was written by many former or current MIT professors or grad students and
edited by the professor for this class. There are a few guides in the front of
the book. These guides focus on proper argumentation, which keeps philosophy on
track. Another guide focuses on how to write proper philosophy papers.

With the readings, there are typically four questions. You should answer these
questions after you read the text. If you can't answer the questions, you should
reread the section. Some more complicated readings have a readers guide that
goes along with it, explaining the lessons to be extrapolated.

There is also a glossary provided which includes definitions for philosophy
terms.

Do not have laptops or tablets open or on during class.

\section{Bertrand Russell}

Bertrand Russell was on a BBC program and he was asked the most important thing
to impart on the next generation. We said that he had two things to say, one
intellectual and one moral. Then intellectual, "When studying philosophy, ask
only about the facts, nothing more." The moral, "Love is wise. Hatred is
foolish." This is how we will treat this class. Different views can be aired in
disgust, which we would like to be false. But we won't worry about that here.
Distaste for view is no argument against them.

\section{Topics for this Class}

\subsection{Does God Exist?}

This statement is self-explanatory and needs no introduction.

\subsection{What is Knowledge?}

This is a very important topic. After all, this is why we are here at MIT.
Knowledge has been viewed as an important topic since Socrates and Plato. A
tempting answer to this question is, when you know, you believe it to be that
way. However, there is more to it than just that. Knowledge may be an
intersection between truth and belief. However, this is still not enough. True
belief is not all that is required. There must be a reason for that true belief.

\subsection{How Can You Know Your or Another's Mind?}

How do you actually know what someone else is thinking? One way we think we do
this is by observing other's behavior and connecting it to what we are thinking
when we act that way.

What about your own mind? How do you know if a physical object truly exists? How
do you know that you see, to see something white? Are you actually seeing
something white?

\subsection{Is Mind Material?}

In essence, if you copy someone's brain, neuron for neuron, will these two
people have the same mind? Mind is separate from the brain. You are separate and
have an immaterial soul. Or do you?

\subsection{What is Consciousness?}

This is a deep topic that can't be well introduced in a short amount of time.

\subsection{What is the Right Thing To Do?}

If you see someone dying, but you will ruin your \$500 pants if you help. If you
don't help, people will be outraged. However, we can spend money to keep people
from dying all the time. But we don't. How is this viewed differently? Is this
right?

\subsection{Is Morality Objective?}

"If God doesn't exist, everything is permitted." Does God provide morality to
us? Or is morality simply a load of nonsense?

\subsection{Do We Possess Freewill?}

There was a guy and he murdered someone. But it was later found out that he had
a brain disorder that caused him to kill that person. Is he off the hook, then?
If the universe is deterministic, based on a set of initial conditions, are we
responsible for anything? Or is everything already determined for us?

\subsection{Does Justice Require Equality?}

There is a massive disparity of wealth in the world. Is this just? Does justice
require that everyone has equal outcome or simply equal opportunity?

\subsection{What is Race?}

SOme think that race is a type of biological kind. Others think that race does
not actually exist at all.

\subsection{What is Gender?}

Is there a difference between gender and sex? Or are they the same thing?

\subsection{Are Things as they Appear?}

There are optical illusions that show us that things aren't always as they seem.
Is reality always ever as it seems?

\subsection{What is the Meaning of Life?}

42?

\end{document}

