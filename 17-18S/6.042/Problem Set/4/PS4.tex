\documentclass{article}
\usepackage{tikz}
\usepackage{float}
\usepackage{enumerate}
\usepackage{amsmath}
\usepackage{amsfonts}
\usepackage{amsthm}
\usepackage{bm}
\usepackage{indentfirst}
\usepackage{siunitx}
\usepackage[utf8]{inputenc}
\usepackage{graphicx}
\graphicspath{ {Images/} }
\usepackage{float}
\usepackage{mhchem}
\usepackage{chemfig}
\allowdisplaybreaks

\title{ 6.042 Problem Set 4 }
\author{ Robert Durfee }
\date{ March 23, 2018 }

\begin{document}

\maketitle

\section*{Problem 1 }

I spent approximately 45 minutes working on this problem in collaboration with
Daniella Guillen and James Quigley.

\bigbreak

The cardinality of the sets are as follows,

\begin{itemize}
  \item $\mathbb{Z}$: Integers are countably infinite. For example, $\{0, -1, 1, -2, 2,
  -3, 3, \ldots\}$.

  \item $\mathbb{R}$: Reals are uncountably infinite. For $\{0,0.1,0.01,\ldots\}$ there
  are an infinite number of reals between each element.

  \item $\mathbb{C}$: Complexes are uncountably infinite. A complex number is described
  $a + bi$ where $a,b \in \mathbb{R}$ so there is a bijection from $\mathbb{R}$
  which is uncountable.

  \item $\mathbb{Q}$: Rationals are countably infinite. A rational number is described
  $a/b$ where $a,b \in \mathbb{Z}$ so there is a bijection from $\mathbb{Z}$ which
  is countable.

  \item $\wp(\mathbb{Z})$: Power set of integers is uncountably infinite. Use
  Cantor's diagonal argument.

  \item $\wp(\emptyset)$: Power set of the null set is $\{\emptyset\}$ which has
  only one element.

  \item $\wp(\wp(\emptyset))$: Power set of power set of null set is
  $\{\emptyset, \{\emptyset\}\}$ which has only two elements.

  \item $\{0,1\}^*$: Set of finite sequences of binary are countably infinite. For
  example, $\{(0), (1), (0,1), (1,0), (1,1), ...\}$. (Or a bijection from
  $\mathbb{Z}$, which is countable, can be made through decimal-binary conversion.)

  \item $\{0,1\}^{\omega}$: Set of infinite binary sequences are uncountably infinite.
  Use Cantor's diagonal argument.

  \item $\{\mathbf{T}, \mathbf{F}\}$: Set of true and false has only two elements.

  \item $\wp(\{\mathbf{T},\mathbf{F}\})$: Power set of true and false is
  $\{\emptyset, \{\mathbf{T}\}, \{\mathbf{F}\}, \{\mathbf{T}, \mathbf{F}\}\}$
  which has four elements.

  \item $\wp(\{0,1\}^{\omega})$: The power set of the set of infinite binary sequences
  is uncountably infinite because the set of infinite binary sequences are
  uncountable thus, by Cantor, $\{0,1\}^{\omega} \textrm{ strict }
  \wp(\{0,1\}^{\omega})$. So, not only is this set uncountably infinite, but more
  uncountably infinite than the other uncountably infinite sets on this list.
\end{itemize}

\bigbreak

Summarizing the results,

\begin{center}
  \begin{tabular}{ c c c c }
    $\mathbb{Z}$ & Countable & $\mathbb{R}$ & Uncountable \\
    $\mathbb{C}$ & Uncountable & $\mathbb{Q}$ & Countable \\
    $\wp(\mathbb{Z})$ & Uncountable & $\wp(\emptyset)$ & 1 \\
    $\wp(\wp(\emptyset))$ & 2 & $\{0,1\}^{*}$ & Countable \\
    $\{0,1\}^{\omega}$ & Uncountable & $\{\mathbf{T},\mathbf{F}\}$ & 2 \\
    $\wp(\{\mathbf{T},\mathbf{F}\})$ & 4 & $\wp(\{0,1\}^{\omega})$ & Uncountable
  \end{tabular}
\end{center}

Sets with the same cardinality can have bijections between them. Therefore, the
following sets have bijections,

\begin{center}
  \begin{tabular}{ c }
    $\wp(\wp(\emptyset))$ and $\{\mathbf{T},\mathbf{F}\}$ \\
    $\mathbb{Z}$, $\mathbb{Q}$, and $\{0,1\}^{*}$ \\
    $\mathbb{R}$, $\mathbb{C}$, $\wp(\mathbb{Z})$, and $\{0,1\}^{\omega}$
  \end{tabular}
\end{center}

\break

\section*{Problem 2}

I spent approximately 3 hours working on this problem in collaboration with
Daniella Guillen and James Quigley.

\bigbreak

Let $f: A \rightarrow B$ where $A$ is an infinite sequence $(a_{0}, a_{1},
a_{2}, \ldots) \in \mathbb{N}$ and $B$ is an infinite sequence $(b_{0}, b_{1},
b_{2}, \ldots) \in \mathrm{Inc}$. Then $f$ constructs $B$ by setting each
element
$$b_{n} = \sum_{k = 0}^{n} \left( a_{k} + 1\right) \ \forall n \in \mathbb{N}$$.

\subsection*{$f$ is a Total Function}

Using the equation that defines each element of $B$ described above.
Since this equation only sums natural numbers, and summation in naturals is a
function, this equation must be a function. As a result, each element of $A$
results in only one possible element in $B$. Therefore, the entire sequence $B$
can only be generated by a unique $A$, so $f$ must be a function. Also, by the
definition above, since all the elements of sequences $A \in
\mathbb{N}^{\omega}$ are defined by $f$, $f$ must be total.

\subsection*{$f(s) \in \mathrm{Inc}$ for each $s \in \mathbb{N}^{\omega}$}

\begin{proof}

  The proof is by induction on the elements of the sequence. The induction
  hypothesis is,
  \begin{center}
    $ P(n) ::= $ for some $s = (a_{0}, a_{1}, a_{2}, ...) \in \mathbb{N}^{\omega}$,

    $f(s) = (b_{0}, b_{1}, b_{2}, \ldots)$ such that $b_{0} < b_{1} < \ldots < b_{n}$.
  \end{center}

  \textbf{Base case}: When $n = 1$, $b_{0} = a_{0} + 1$ and $b_{1} =
  a_{0} + a_{1} + 2$. Subtracting the first equation from the second, $b_{1} -
  b_{0} = a_{1} + 1$. Since $a_{1} \geq 0$ and $1 > 0$, $b_{1}$ must be greater
  than $b_{0}$. Thus, $P(1)$ holds.

  \bigbreak

  \textbf{Inductive Step}: Assume that $P(n)$ holds. We can represent $b_{n}$
  and $b_{n+1}$,
  $$ b_{n} = \sum\limits_{k=0}^{n} (a_{k} + 1),\ b_{n+1} =
  \sum\limits_{k=0}^{n+1} (a_{k} + 1) $$

  Identifying common terms,
  $$ b_{n} = \sum\limits_{k=0}^{n} a_{k} + n,\ b_{n+1} = \sum\limits_{k=0}^{n}
  a_{k} + a_{n+1} + n + 1 $$

  Subtracting $b_{n}$ from $b_{n+1}$,
  $$ b_{n+1} - b_{n} = a_{n+1} + 1 $$

  Since $a_{n+1} \geq 0$ and $1 > 0$, $b_{n+1}$ must be greater than $b_{n}$.
  Thus, $P(n+1)$ holds.

  \bigbreak

  Because the induction hypothesis holds for the base case and the inductive
  step, by induction, $P(n)$ holds for all $n$ and $f(s) \in \mathrm{Inc}$.

\end{proof}

\subsection*{$f$ is Injective}

\begin{proof}
  Let $f: \mathbb{N}^{\omega} \rightarrow \mathrm{Inc}$. Also let $A = (a_{0},
  a_{1}, a_{2}, \ldots)$ and $B = (b_{0}, b_{1}, b_{2}, \ldots)$. $f$ is
  injective if $f(A) = f(B)$, then $A = B$.

  \bigbreak

  Applying $f$ to $A$,
  $$ f(A) = (a_{0} + 1, a_{0} + a_{1} + 2, a_{0} + a_{1} + a_{2} + 3, ...) $$

  Applying $f$ to $B$,
  $$ f(B) = (b_{0} + 1, b_{0} + b_{1} + 2, b_{0} + b_{1} + b_{2} + 3, ...) $$

  Equating $f(A)$ to $f(B)$,
  \begin{equation*}
    \begin{split}
      (a_{0} + 1, a_{0} + a_{1} + 2, a_{0} + &a_{1} + a_{2} + 3, ...) \\
      & = (b_{0} + 1, b_{0} + b_{1} + 2, b_{0} + b_{1} + b_{2} + 3, ...)
    \end{split}
  \end{equation*}

  Equating the elements of $f(A)$ to $f(B)$,
  \begin{equation*}
    \begin{split}
      (a_{0} = b_{0}, a_{0} + a_{1} = b_{0} + b_{1}, a_{0} + a_{1} + a_{2} =
      b_{0} + b_{1} + b_{2}, ...)
    \end{split}
  \end{equation*}

  From the first element, $a_{0} = b_{0}$, so all $b_{0}$ can be replaced with
  $a_{0}$. This can then be repeated for $b_{1}$, $b_{2}$, $\ldots$
  $$ (a_{0} = b_{0}, a_{1} = b_{1}, a_{2} = b_{2}, \ldots) $$

  Therefore, in order for $f(A) = f(B)$, $A = B$,
  $$ (a_{0}, a_{1}, a_{2}, \ldots) = (b_{0}, b_{1}, b_{2}, \ldots) $$

  Thus, $f$ is injective.
\end{proof}

\subsection*{$f$ is Surjective}

\begin{proof}
  Let $f: \mathbb{N}^{\omega} \rightarrow \mathrm{Inc}$. Also let $A = (a_{0},
  a_{1}, a_{2}, \ldots)$ and $B = (b_{0}, b_{1}, b_{2}, \ldots)$. $f$ is
  surjective if for some $B \in \mathrm{Inc}$, there exists an $A \in
  \mathbb{N}^{\omega}$ such that $f(A) = B$.

  \bigbreak

  Applying $f$ to $A$,
  $$ f(A) = (a_{0} + 1, a_{0} + a_{1} + 2, a_{0} + a_{1} + a_{2} + 3, \ldots) $$

  Equating $f(A)$ to $B$,
  $$ (a_{0} + 1, a_{0} + a_{1} + 2, a_{0} + a_{1} + a_{2} + 3, \ldots) = (b_{0},
  b_{1}, b_{2}, \ldots) $$

  Equating the elements of $f(A)$ to $B$,
  $$ (a_{0} + 1 = b_{0}, a_{0} + a_{1} + 2 = b_{1}, a_{0} + a_{1} + a_{2} + 3 =
  b_{2}, \ldots) $$

  Solving each element for $a$,
  $$ (a_{0} = b_{0} - 1, a_{1} = b_{1} - a_{0} - 2, a_{2} = b_{2} - a_{1} -
  a_{0} - 3, \ldots) $$

  From the first element, $a_{0} = b_{0} - 1$, so all $a_{0}$ can be replaced
  with $b_{0} - 1$. This can then be repeated for $a_{1}, a_{2}, \ldots$
  $$ (a_{0} = b_{0} - 1, a_{1} = b_{1} - b_{0} - 1, a_{2} = b_{2} - b_{1} - 1, \ldots) $$

  Therefore, for some $B \in \mathrm{Inc}$, there exists an $A \in
  \mathbb{N}^{\omega}$ such that $f(A) = B$ when $A$'s elements are defined by
  $$ a_{n} = b_{n} - b_{n-1} - 1 $$

  Thus, $f$ is surjective.

\end{proof}

From all these properties, it follows that $f$ is a bijection.

\break

\section*{Problem 3}

I spent approximately 30 minutes working on this problem in collaboration with
Daniella Guillen and James Quigley.

\subsection*{Part A}

A longest chain in this DAG is $\{1, 6, 7, 11, 14\}$.

\subsection*{Part B}

The longest antichain in this DAG is $\{1, 2, 3, 4, 5, 8, 9, 10, 11, 12, 13\}$.

\begin{proof}
  Proof by contradiction. Assume for contradiction that there is an antichain
  $B$ longer than antichain $A = \{1, 2, 3, 4, 5, 8, 9, 10, 11, 12, 13\}$ in
  this DAG. In order for $B$ to be longer than $A$, $B$ must either

  \begin{itemize}
    \item contain all the nodes of $A$ and some additional node $n$ not
      comparable to any nodes in $A$, or
    \item contain more non-comparable nodes, none of which are in $A$.
  \end{itemize}

  Since the only remaining nodes are comparable to at least one element in $A$,
  there cannot be a node $n$. And, since the remaining number of nodes is less
  than the length of $A$, there is no longer antichain made up of nodes not in
  $A$. Therefore, since $B$ cannot exist, by contradiction, there cannot be a
  longer antichain than $A$ in this DAG.
\end{proof}

\subsection*{Part C}

It would take 14 seconds to complete all the tasks with only one processor. This
is the total number of task in the DAG.

\subsection*{Part D}

It would take 5 seconds to complete all the tasks with unlimited processors.
This is the length of the longest chain.

\subsection*{Part E}

The smallest number of processors that will all completion of all the tasks in 5
seconds would be 5 processors.

\bigbreak

A schedule to achieve this could be $\{1, 2, 3, 4, 5\}, \{6, 8\}, \{7\}, \{9,
10, 11, 12, 13\}, \{14\}$. The maximum number of tasks per iteration is 5.

\end{document}

