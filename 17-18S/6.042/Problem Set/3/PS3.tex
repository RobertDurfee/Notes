\documentclass{article}
\usepackage{tikz}
\usepackage{float}
\usepackage{enumerate}
\usepackage{amsmath}
\usepackage{amsfonts}
\usepackage{amsthm}
\usepackage{bm}
\usepackage{indentfirst}
\usepackage{siunitx}
\usepackage[utf8]{inputenc}
\usepackage{graphicx}
\graphicspath{ {Images/} }
\usepackage{float}
\usepackage{mhchem}
\usepackage{chemfig}
\allowdisplaybreaks

\title{ 6.042 Problem Set 3 }
\author{ Robert Durfee }
\date{ March 9, 2018 }

\begin{document}

\maketitle

\section*{Problem 1 }

I spent approximately 1 hour working on this problem in collaboration with
Daniella Guillen and James Quigley.

\subsection*{Part A}

The state machine for the situation described is the following:
\begin{equation*}
  \begin{split}
    \textrm{states} &::= \{ (c, h, t) \mid c, h, t \in \mathbb{N} \textrm{ and }
    c = h + t \} \\
    \textrm{start} &::= (102, 98, 4) \\
    \textrm{transitions} &::= (c, h, t) \longrightarrow
      \begin{cases}
        \{ (c, 10 + h - 2a, -10 + t + 2a) \mid a \in \mathbb{N} \textrm{ and } a \leq 10\}& \\
        (c + h + 1, h, t + h + 1)&
      \end{cases}
  \end{split}
\end{equation*}

Where $a$ represents the number of heads flipped to tails.

\bigbreak
\bigbreak
\bigbreak
\bigbreak
\bigbreak
\bigbreak
\bigbreak
\bigbreak
\bigbreak
\bigbreak
\bigbreak

\subsection*{Part B}

First, apply the second transition three times:
\begin{equation*}
  \begin{split}
    (102, 98, 4) &\longrightarrow (201, 98, 103) \\
                 &\longrightarrow (300, 98, 202) \\
                 &\longrightarrow (399, 98, 301)
  \end{split}
\end{equation*}

Then, apply the first transition 30 times, flipping tails only:
\begin{equation*}
  \begin{split}
    (399, 98, 301) &\longrightarrow (399, 108, 291) \\
                   &\longrightarrow (399, 118, 281) \\
                   &\longrightarrow \ldots \\
                   &\longrightarrow (399, 398, 1)
  \end{split}
\end{equation*}

\bigbreak
\bigbreak
\bigbreak
\bigbreak
\bigbreak
\bigbreak
\bigbreak
\bigbreak
\bigbreak
\bigbreak
\bigbreak
\bigbreak
\bigbreak
\bigbreak
\bigbreak
\bigbreak
\bigbreak
\bigbreak
\bigbreak
\bigbreak
\bigbreak
\bigbreak
\bigbreak
\bigbreak
\bigbreak
\bigbreak
\bigbreak

\subsection*{Part C}

The following are derived variables:

\begin{center}
  \begin{tabular}{ c c c }
    $C ::= c$ & $H ::= h$ & $T ::= t$
  \end{tabular}
\end{center}
\begin{center}
  \begin{tabular}{ c c c }
    $C_{2} ::= \begin{cases} 0 & c \textrm{ is even.} \\ 1 & c \textrm{ is odd.}
      \end{cases}$ & $H_{2} ::= \begin{cases} 0 & h \textrm{ is even.} \\ 1 & h
      \textrm{ is odd.} \end{cases}$ $T_{2} ::= \begin{cases} 0 & t \textrm{ is
      even.} \\ 1 & t \textrm{ is odd.} \end{cases}$
  \end{tabular}
\end{center}

Of these derived variables, $C$ is weakly increasing and $H_{2}$ is constant.

\bigbreak
\bigbreak
\bigbreak
\bigbreak
\bigbreak
\bigbreak
\bigbreak
\bigbreak
\bigbreak
\bigbreak
\bigbreak
\bigbreak
\bigbreak
\bigbreak
\bigbreak
\bigbreak
\bigbreak
\bigbreak
\bigbreak
\bigbreak
\bigbreak
\bigbreak
\bigbreak
\bigbreak
\bigbreak
\bigbreak
\bigbreak

\subsection*{Part D}

\begin{proof}
  The proof is by induction on the number of transitions. The induction
  hypothesis is
  $$ P(n) ::= \textrm{if } (c, h, t) \textrm{ is a state reachable in } n
  \textrm{ transitions, then } H_{2} = 0. $$

  \noindent
  \textbf{Base case}: P(0) is true since the only state reachable in 0
  transitions is the start state (102, 98, 4) and 98 is even. Therefore, $H_{2}
  = 0$.

  \bigbreak

  \noindent
  \textbf{Inductive step}: Assume that P(n) is true, and let $(c',h',t')$ be any
  state reachable in $n + 1$ transitions. There are two possible transitions to
  worry about.

  \bigbreak

  \textbf{Transition 1}: For the first transition, as defined above, $h' = 10 +
  h - 2a$. We assumed that $H_{2} = 0$ for the previous state, therefore $h$ is
  even. For any value of $a$, $2a$ will be even. Therefore, the sum and
  difference of all even integers will always be even, thus $h'$ is even and
  $H_{2} = 0$. The induction hypothesis holds for transition 1.

  \bigbreak

  \textbf{Transition 2}: For the second transition, as defined above, $h' = h$.
  We assumed that $H_{2} = 0$ for the previous state, therefore $h$ is even.
  Since this is a simple equality statement, $h'$ must also be even, thus $H_{2}
  = 0$, The induction hypothesis holds for transition 2.

  \bigbreak

  Since the induction hypothesis holds for the base case and the inductive step
  for all transitions defined above, by induction, $P(n)$ holds for all states
  and $H_{2} = 0$ is a preserved invariant.
\end{proof}

Since $H_{2} = 0$ is a preserved invariant for the state machine, and when one
head is showing $H_{2} = 1$, it is impossible for one head to be showing as this
contradicts the preserved invariant.

\bigbreak
\bigbreak
\bigbreak
\bigbreak
\bigbreak
\bigbreak
\bigbreak
\bigbreak
\bigbreak
\bigbreak
\bigbreak
\bigbreak
\bigbreak

\section*{Problem 2}

I spent approximately 20 minutes working on this problem in collaboration with
Daniella Guillen and James Quigley.

\bigbreak

Let there be three boys--Alex, James, and Emile--and three girls--Sara, Hannah,
and Betty.

\bigbreak

Let the boys' rankings of the girls be as follows:
\begin{center}
  \begin{tabular}{ c | c c c }
    & \textbf{First} & \textbf{Second} & \textbf{Third} \\
    Alex & Betty & Sara & Hannah \\
    James & Sara & Hannah & Betty \\
    Emile & Hannah & Betty & Sara
  \end{tabular}
\end{center}

Let the girls' ranking of the boys be as follows:
\begin{center}
  \begin{tabular}{ c | c c c }
    & \textbf{First} & \textbf{Second} & \textbf{Third} \\
    Sara & Emile & Alex & James \\
    Hannah & Alex & James & Emile \\
    Betty & James & Emile & Alex
  \end{tabular}
\end{center}

A matching of the boys and girls are as follows:
\begin{center}
  (Alex, Sara), (James, Hannah), (Emile, Betty)
\end{center}

Since everyone is paired with their second choice and everyone's first choice
has them as their last choice, there cannot be any rogue couples. Therefore,
these matchings are stable.

\bigbreak

These results are impossible to be met with by the Mating Ritual. For the mating
ritual favoring the girls' choices, the outcome will be:
\begin{center}
  (Emile, Sara), (Alex, Hannah), (James, Betty)
\end{center}

For the mating ritual favoring the boys' choices, the outcome will be:
\begin{center}
  (Alex, Betty), (James, Sara), (Emile, Hannah)
\end{center}

Each ritual will end on the first day as either the girls will all get their
first choices (and the boys their last) or the boys will all get their first
choices (and the girls their last). Since no one gets their second choice, this
matching in unattainable.

\bigbreak
\bigbreak
\bigbreak
\bigbreak
\bigbreak

\section*{Problem 3}

I spent approximately 1 hour working on this problem in collaboration with
Daniella Guillen and James Quigley.

\subsection*{Part A}

\begin{figure}[H]
  \centering
  \includegraphics[scale=0.70]{"KochSnowflake"}
  \caption{Koch Snowflake}
\end{figure}

This Koch snowflake has 9 edges with three different edge lengths.

\bigbreak
\bigbreak
\bigbreak
\bigbreak
\bigbreak
\bigbreak
\bigbreak
\bigbreak
\bigbreak
\bigbreak
\bigbreak
\bigbreak
\bigbreak
\bigbreak
\bigbreak
\bigbreak
\bigbreak


\subsection*{Part B}

\begin{proof}
  The proof is by structural induction on the edge lengths of a Koch snowflake.
  The induction hypothesis will be
  $$ P(n) ::= \textrm{the lengths of the edges of a Koch snowflake, } n
  \textrm{, are rational.} $$

  \noindent
  \textbf{Base case}: The base case Koch snowflake is an equilateral triangle
  with side length $l \in \mathbb{N}$. Since $\mathbb{N} \subseteq \mathbb{Q}$,
  then $l \in \mathbb{Q}$. Thus the induction hypothesis holds for the base
  case.

  \bigbreak

  \noindent
  \textbf{Constructor case}: Let $n'$ be a Koch snowflake constructed from a
  previous Koch snowflake $n$. Assume that the edge lengths of the Koch
  snowflake $n$ are rational. WLOG, choose an edge, $m$, of Koch snowflake $n$
  of length $l$. Then $l \in \mathbb{Q}$. To construct a new Koch snowflake,
  remove the inner third of $m$ and replace it with two edges of length $l/3$.
  By doing this action, an edge of length $l$ is removed from the Koch snowflake
  $n$ and replaced with four edges of length $l/3$ to create $n'$.

  Since $l \in \mathbb{Q}$ and $\mathbb{Q}$ is closed under division, $l/3 \in
  \mathbb{Q}$. Since all the edges of $n$ are rational and $l/3$, the new edge,
  is rational and no edges of other lengths are created through this action, the
  edges of $n'$ must all be rational. Thus the induction hypothesis holds for
  the constructor case.

  \bigbreak

  Since the induction hypothesis hold for the base case and the constructor
  case, by structural induction, $P(n)$ hold for all $n$.
\end{proof}

\bigbreak
\bigbreak
\bigbreak
\bigbreak
\bigbreak
\bigbreak
\bigbreak
\bigbreak
\bigbreak
\bigbreak
\bigbreak
\bigbreak
\bigbreak
\bigbreak
\bigbreak
\bigbreak
\bigbreak
\bigbreak
\bigbreak

\begin{proof}
  The proof is by structural induction on the area of a Koch snowflake. The
  induction hypothesis will be
  $$ P(n) ::= \textrm{the area of a Koch snowflake, } n \textrm{, is of the form
  } q \sqrt{3} \textrm{ where } q \in \mathbb{Q}. $$

  \noindent
  \textbf{Base case}: The base case Koch snowflake is an equilateral triangle
  with side length $l \in N$. The area of an equilateral triangle is given by
  $$ \frac{ \sqrt{3} }{ 4 }l^{2} $$

  Since the nonnegative integers are close under multiplication, $l^{2} \in
  \mathbb{N}$. Since $\mathbb{N} \subseteq \mathbb{Q}$ and $\mathbb{Q}$ is
  closed under division, $l^{2}/4 \in \mathbb{Q}$. Therefore, $l^{2}/4$ can be
  written as $q$ where $q \in \mathbb{Q}$. And the total area would then be
  $q\sqrt{3}$. Thus the inductive hypothesis is true for the base case.

  \bigbreak

  \noindent
  \textbf{Constructor case}: Let $n'$ be a Koch snowflake constructed from a
  previous Koch snowflake $n$. Assume that the area of the Koch snowflake $n$
  can be written in the form $q\sqrt{3}$.

  To construct $n'$, take an edge, $m$, of Koch snowflake $n$. Let the length of
  this edge be $l$. Remove the inner third of $m$ and replace it with two edges
  of length $l/3$. This adds an area of one equilateral triangle of side length
  $l/3$ to the Koch snowflake $n$.

  The area of $n'$ can then be expressed as
  $$ q\sqrt{3} + \frac{ \sqrt{3} }{ 4 } \left(\frac{ l }{ 3 }\right)^{2} =
  \sqrt{3} \left( q + \frac{ l^{2} }{ 36 } \right) $$

  Since the edges of a Koch snowflake must be rational, as proved above, $l \in
  \mathbb{Q}$. Since $\mathbb{Q}$ is closed under multiplication and division,
  $l^{2}/36 \in \mathbb{Q}$. Therefore, $l^{2}/36$ can be written as $p \in
  \mathbb{Q}$. The area of $n'$ is then expressed as:
  $$ \sqrt{3} \left( q + p \right) $$

  Since both $p,q \in \mathbb{Q}$, and $\mathbb{Q}$ is closed under addition,
  their sum can be written as $q'$ where $q' \in \mathbb{Q}$. Therefore, the
  area of $n'$ can be expressed in the form $q'\sqrt{3}$. Thus the inductive
  hypothesis is true for the constructor case.

  \bigbreak

  Since the induction hypothesis holds for the base case and the constructor
  case, by structural induction, $P(n)$ holds for all $n$.
\end{proof}

\end{document}

