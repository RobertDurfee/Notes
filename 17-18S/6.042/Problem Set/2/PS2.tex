\documentclass{article}
\usepackage{tikz}
\usepackage{float}
\usepackage{enumerate}
\usepackage{amsmath}
\usepackage{amsthm}
\usepackage{amsfonts}
\usepackage{bm}
\usepackage{indentfirst}
\usepackage{siunitx}
\usepackage[utf8]{inputenc}
\usepackage{graphicx}
\graphicspath{ {Images/} }
\usepackage{float}
\usepackage{mhchem}
\usepackage{chemfig}
\allowdisplaybreaks

\title{ 6.042 Problem Set 2 }
\author{ Robert Durfee }
\date{ March 2, 2018 }

\begin{document}

\maketitle

\section*{Problem 1 }

I spent approximately 30 minutes working on this problem alone using only
materials from this term.

\subsubsection*{Claim}

\textit{For all $n \geq 50$, $n$ can be represented as the sum of nonnegative
integer multiples of 7, 11, and 13.}

\subsubsection*{Proof}

The proof is by the Well-Ordering Principle. Let
$$ P(n) ::= \textrm{An integer } n \geq 50 \textrm{ is a sum of integer multiples of
7, 11, and 13.} $$
$$ S ::= \{n \geq 50 \mid \lnot P(n) \} $$

Assume for contradiction that the set $S$ is nonempty. By WOP, the set $S$
must have some smallest element $m$.

\bigbreak

It can be shown that $P(n)$ is true for $n \in [50..56]$,
\begin{align*}
  n &= 50 = 7 \cdot 4 + 11 \cdot 2  \\
  n &= 51 = 7 \cdot 2 + 11 + 13 \cdot 2  \\
  n &= 52 = 13 \cdot 4  \\
  n &= 53 = 7 \cdot 6 + 11  \\
  n &= 54 = 7 \cdot 4 + 13 \cdot 2 \\
  n &= 55 = 11 \cdot 5 \\
  n &= 56 = 7 \cdot 8
\end{align*}

\bigbreak

As a result, $m \geq 57$. Since $m$ is the smallest counterexample, $m - 7$ is
not a counterexample. $P(m - 7)$ must hold because $m - 7 \geq 50$. So $m - 7$
can be written as a multiple of 7, 11, and 13. Thus, $m$ can be represented by
just adding another 7 to $m - 7$ showing that $m$ is not a counterexample which
contradicts the assumption that $S$ is nonempty. Therefore, $P(n)$ must be true.
\qed

\section*{Problem 2}

I spent approximately 30 minutes working on this problem in collaboration with
Daniela Guillen and James Quigley.

\begin{figure}[H]
  \centering
  \includegraphics[scale=0.60]{"Example"}
  \caption{Example}
\end{figure}

For the above example, $R$, depicted as arrows between $A$ and $B$, is not a
\textbf{function} because $[> 1 \textrm{ out}]$. $S$, depicted as arrows between
$B$ and $C$, is also not a \textbf{function} because $[> 1 \textrm{ out}]$.
However, $R$ is \textbf{total}, because $[\geq 1 \textrm{ out}]$, and
\textbf{injective}, because $[\leq \textrm{ in}]$. $S$, on the other hand, is
only \textbf{surjective} because $[\geq \textrm{ in}]$. Overall, $S \circ R$
remains \textbf{bijective} because $[= \textrm{ out}]$ and $[= \textrm{ in}]$.

\bigbreak
\bigbreak
\bigbreak
\bigbreak
\bigbreak
\bigbreak
\bigbreak
\bigbreak
\bigbreak
\bigbreak
\bigbreak
\bigbreak
\bigbreak
\bigbreak
\bigbreak
\bigbreak
\bigbreak
\bigbreak
\bigbreak
\bigbreak
\bigbreak

\section*{Problem 3}

I spent approximately 35 minutes working on this problem in collaboration with
Daniela Guillen and James Quigley.

\subsection*{Part A}

An implies statement such as $A \implies B$ is true when $A$ is false or $B$ is
true. Therefore, to calculate the number of times $A \implies B$ is true, we can
simply sum the number of times $A$ is false with the number of times $B$ is
true.

\bigbreak

Since $F_{n}(P_{1}, P_{2}, ..., P_{n})$ is recursive, it can be rewritten as:
$$ F_{n}(P_{1}, P_{2}, ..., P_{n}) = F_{n - 1}(P_{1}, P_{2}, ..., P_{n - 1})
\implies P_{n} $$

Therefore, to determine when $F_{n}$ is true, we need to determine the number of
times $F_{n-1}$ is false and how many times $P_{n}$ is true.

\bigbreak

We know that $F_{n - 1}$ is true $T_{n - 1}$ times. To calculate the number of
times $F_{n - 1}$ is false, we can take the total number of outputs and subtract
the true outputs and the remaining must be false (as there are no other possible
outputs). The total number of outputs is $2^{n - 1}$. Therefore, the number of
false occurrences is $2^{n - 1} - T_{n - 1}$.

\bigbreak

On the other hand, $P_{n}$ is an input. Since $F_{n}$ has $2^{n}$ outputs,
$P_{n}$ must be assigned $2^{n}$ times in order to result in a complete truth
table. Half of these must be true and the other half false. As a result, $P_{n}$
is true $2^{n}/2$ times.

\bigbreak

Adding these two values together:
\begin{align*}
  T_{n} &= 2^{n - 1} - T_{n - 1} + 2^{n} / 2 \\
        &= 2^{n - 1} - T_{n - 1} + 2^{n - 1} \\
        &= 2 \cdot 2^{n - 1} - T_{n - 1} \\
        &= 2^{n} - T_{n - 1}
\end{align*}

Adjusting indices:
\begin{equation}
  T_{n + 1} = 2^{n + 1} - T_{n}
\end{equation}

\bigbreak
\bigbreak
\bigbreak
\bigbreak
\bigbreak
\bigbreak

\subsection*{Part B}

\subsubsection*{Claim}

\begin{equation}
  T_{n} = \frac{ 2^{n + 1} + (-1)^{n} }{ 3 } \textrm{ for } n \geq 1
\end{equation}

\subsubsection*{Proof}

The proof is by ordinary induction. Let the induction hypothesis $P(n)$ be
equation (2).

\bigbreak

\textbf{Base case}: $P(1)$ is evaluated as follows:
$$ T_{1} = \frac{ 2^{1 + 1} + (-1)^{1} }{ 3 } = 1 $$

And $F_{1}$ by truth table:
\begin{center}
  \begin{tabular}{ c | c }
    $P_{1}$ & $F_{1}(P_{1})$ \\\hline
    T & T \\
    F & F
  \end{tabular}
\end{center}

Therefore, it is clear that $F_{1}$ has only one true output so $T_{1} = 1$.

\bigbreak

\textbf{Inductive step}: Assume that $P(n)$ is true, that is equation (2) holds
for some integer $n$ greater than or equal to 1. Then we can substitute equation
(2) into equation (1) as $T_{n}$:
\begin{align*}
  T_{n + 1} &= 2^{n + 1} - T_{n} \\
            &= 2^{n + 1} - \frac{ 2^{n + 1} + (-1)^{n} }{ 3 } \\
            &= \frac{ 3 \cdot 2^{n + 1} - 2^{n + 1} - (-1)^{n} }{ 3 } \\
            &= \frac{ 2 \cdot 2^{n + 1} - (-1)^{n} }{ 3 } \\
            &= \frac{ 2^{n + 2} - (-1)^{n} }{ 3 } \\
            &= \frac{ 2^{n + 2} + (-1)(-1)^{n} }{ 3 } \\
            &= \frac{ 2^{n + 2} + (-1)^{n + 1} }{ 3 }
\end{align*}

Which shows that $P(n + 1)$ is true:
$$ T_{n + 1} = \frac{ 2^{(n + 1) + 1} + (-1)^{(n + 1)}}{ 3 } = \frac{ 2^{n + 2}
+ (-1)^{n + 1}}{ 3 }$$

So it follows by induction that $P(n)$ is true for all $n \geq 1$. \qed

\end{document}

