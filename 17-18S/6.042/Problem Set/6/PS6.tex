\documentclass{article}
\usepackage{tikz}
\usepackage{float}
\usepackage{enumerate}
\usepackage{amsmath}
\usepackage{amsthm}
\usepackage{bm}
\usepackage{indentfirst}
\usepackage{siunitx}
\usepackage[utf8]{inputenc}
\usepackage{graphicx}
\graphicspath{ {Images/} }
\usepackage{float}
\usepackage{mhchem}
\usepackage{chemfig}
\allowdisplaybreaks

\title{ 6.042 Problem Set 6 }
\author{ Robert Durfee }
\date{ April 20, 2018 }

\begin{document}

\maketitle

\section*{Problem 1}

I spent approximately 1 hours working on this problem in collaboration with
Daniella Guillen.

\subsection*{Part A}

\begin{proof}
  The proof is by induction on the number of plays in the game.
  \begin{center}
    $P(n) ::=$ The number $x$ created during the $n$th play of the game is a
    linear combination of the starting numbers $a$ and $b$.
  \end{center}

  \noindent\textbf{Base case}: In the first play of the game, the only two numbers on the
  board are $a$ and $b$. WLOG, $a > b$, the only play is $x = a - b$.
  $x$ is a linear combination of $a$ and $b$ with the coefficients $1$ and $-1$
  respectively. Thus, $P(1)$ is true.

  \bigbreak

  \noindent\textbf{Inductive step}: Assume all the previous plays in the game 1 through
  $n$ resulted in numbers on the board which can be represented as linear
  combinations of $a$ and $b$. Then we can choose two numbers $x_1$ and $x_2$ on
  the board. These two numbers can be written as linear combinations of $a$ and
  $b$ such as:
  $$ x_1 = s_1 a + t_1 b $$
  $$ x_2 = s_2 a + t_2 b $$
  WLOG, $x_1 > x_2$, the next play can be written $x' = x_1 - x_2$. From the
  assumption about $x_1$ and $x_2$, 
  $$ x' = \left(s_1 a + t_1 b\right) - \left(s_2 a + t_2 b\right) $$
  Which, by combining like terms, can be written as a linear combination of $a$
  and $b$ as follows:
  $$ x' = \left(s_1 - s_2\right)a + \left(t_1 - t_2\right)b $$
  Thus, $P(n + 1)$ is true.

  \bigbreak

  Given the induction hypothesis holds for the base case and the inductive step,
  it holds for all plays during the game. Since all numbers on the board can be
  written as a linear combination of $a$ and $b$, it follows from Corollary
  9.2.3 that every number on the board is a multiple of $\gcd(a, b)$.
 
\end{proof}

\break

\subsection*{Part B}

\begin{proof}

Let the smallest positive number on the board after the game finished be $c$.
This number can be written $a = qc + r$ where $q$ is the quotient and $r$ is the
remainder. 

\bigbreak

Given that $c$ and $a$ are on the board, $r$ could also be computed
and placed on board through repeated subtraction of $c$ from $a$, $q$ times (as
per the definition of division):
$$ a - c - c - \ldots - c = r $$

But $r$ is restricted to the range $[0..c)$. And since we chose $c$ to be the
smallest positive number on the board, $r$ cannot be on the board as it would be
less than $c$. Thus, $r$ must be zero as non-positive numbers are not allowed in
the game.

\bigbreak

Since $r$ is zero, $a = qc$. This shows that $c \vert a$. Repeating this
argument by replacing $a$ with $b$ will lead to the conclusion that $c \vert b$.
From the properties of the greatest common divisor:
$$ (c \vert a \land c \vert b) \iff c \vert \gcd(a, b) $$

It was also shown in Part A that every number on the board (including $c$) must
be a multiple of $\gcd(a, b)$. The only way that $c$ can both be a multiple of
$\gcd(a,b)$ and divide $\gcd(a,b)$ is if it equals $\gcd(a,b)$.

\bigbreak

Since the smallest element on the board must be $\gcd(a, b)$, every other
multiple of $\gcd(a,b)$ up to $\max(a,b)$ can be computed during the game
through repeated subtraction. Since $\max(a,b)$ is a multiple of $\gcd(a,b)$,
every subtraction of $\gcd(a,b)$ from it will incrementally reduce the multiple
of $\gcd(a,b)$ until the smallest number is reached, which is $\gcd(a,b)$. And
since this game only deals with subtraction, no number can be larger than
$\max(a,b)$.

\end{proof}

\break

\subsection*{Part C}

Given that every number on the board is a multiple of the $\gcd(a, b)$ and that
every multiple of the $\gcd(a,b)$ is on the board is in $[\gcd(a,b) ..
\max(a,b)]$, it is possible to count the number of numbers that will be on the
board at the end of the game. If that number is odd, Player 1 will always win.
If that number is even, Player 2 will always win.

\break

\section*{Problem 2}

I spent approximately 20 minutes working on this problem in collaboration with
Daniella Guillen.

\bigbreak

We are given:
$$ y \equiv m_1 x + b_1 \pmod{p} $$
$$ y \equiv m_2 x + b_2 \pmod{p} $$

Using the rule of transitivity, we can combine these two congruences:
$$ m_1 x + b_1 \equiv m_2 x + b_2 \pmod{p} $$

Since there is always an additive inverse in any modulo, we can add the additive
inverse of $b_1$ (which I will denote using the notation $a^{-1(a)}$) to both
sides:
$$ m_1 x + b_1 + b_1^{-1(a)} \equiv m_2 x + b_2 + b_1^{-1(a)} \pmod{p} $$

Then the $b_1$'s will cancel. Now we can take the additive inverse of $m_2$
multiplied by $x$ and add that to both sides:
$$ m_1 x + m_2^{-1(a)} \equiv m_2 x + m_2^{-1(a)} x + b_2 + b_1^{-1(a)} \pmod{p}
$$

Grouping similar $x$ terms on both sides using the rule of distributivity:
$$ \left(m_1 + m_2^{-1(a)}\right) x \equiv \left(m_2 + m_2^{-1(a)}\right) x +
b_2 + b_1^{-1(a)} \pmod{p} $$

Then the $m_2$'s will cancel. Now, provided that $m_1 \neq m_2$ and $p$ is
prime, we can multiply the multiplicative inverse of $m_1 + m_2^{-1(a)}$ (which
I will denote using the notation $a^{-1(m)}$) to both sides:
$$ \left(m_1 + m_2^{-1(a)}\right)^{-1(m)} \left(m_1 + m_2^{-1(a)}\right) x
\equiv \left(m_1 + m_2^{-1(a)}\right) \left(b_2 + b_1^{-1(a)}\right) \pmod{p} $$

Then the $m_1 + m_2^{-1(a)}$ term will cancel leaving:
$$ x \equiv \left(m_1 + m_2^{-1(a)}\right) \left(b_2 + b_1^{-1(a)}\right)
\pmod{p} $$

Substituting $x$ into either equation above yields $y$:
$$ y \equiv m_1 \left(m_1 + m_2^{-1(a)}\right) \left(b_2 + b_1^{-1(a)}\right) +
b_1 \pmod{p} $$

\break

\section*{Problem 3}

I spent approximately 2 hours working on this problem in collaboration with
Daniella Guillen.

\subsection*{Part A}

\begin{proof}
We are given that $p_i$ does not divide $a$ and the expression:
$$ a^{\phi(m)} \pmod{p_i^{k_i}} $$

Since $m = p_1^{k_1} \ldots p_n^{k_n}$ where each $p_i$ is prime, then each
$p_i^{k_i}$ must be relatively prime. Thus, using Theorem 9.10.10.2,
$$ \phi(m) = \phi\left(p_1^{k_1}\right) \ldots \phi\left(p_i^{k_i}\right)
\ldots \phi\left(p_n^{k_n}\right) $$

So the expression is congruent to:
$$ a^{\phi\left(p_1^{k_1}\right) \ldots \phi\left(p_i^{k_i}\right) \ldots
\phi\left(p_n^{k_n}\right)} \pmod{p_i^{k_i}} $$

Pulling the $\phi\left(p_i^{k_i}\right)$ term to the front of the exponent:
$$ \left(a^{\phi\left(p_i^{k_i}\right)}
\right)^{\phi\left(p_1^{k_1}\right) \ldots \phi\left(p_{i-1}^{k_{i-1}}\right)
\phi\left(p_{i+1}^{k_{i+1}}\right)\ldots \phi\left(p_n^{k_n}\right)}
\pmod{p_i^{k_i}} $$

From Euler's Theorem,
$$ a^{\phi\left(p_i^{k_i}\right)} \equiv 1 \pmod{p_i^{k_i}} $$

So the expression is congruent to:
$$ 1^{\phi\left(p_1^{k_1}\right) \ldots \phi\left(p_{i-1}^{k_{i-1}}\right)
\phi\left(p_{i+1}^{k_{i+1}}\right)\ldots \phi\left(p_n^{k_n}\right)} \equiv 1
\pmod{p_i^{k_i}} $$
\end{proof}

\break

\subsection*{Part B}

\begin{proof}
Since $p_i \vert a$, then $p_i^{k_i} \vert a^{m - \phi(m)}$ when $k_i \leq m -
\phi(m)$.

\bigbreak

The meaning of $\phi(m)$ is the number of numbers in $[0..m)$ that are
relatively prime to $m$. Since $m$ includes all numbers in that range, $m -
\phi(m)$ is the number of numbers \textit{not} relatively prime to $m$. So we
have to show that the number of numbers not relatively prime to $m$ is at least
as large as $k_i$.

\bigbreak

All the $p$'s together are the unique prime factors of $m$. With their $k$
exponents, the multiplication of all the terms is the prime factorization of
$m$. As such, for $k_i$, we can know at least $k_i$ numbers that are not
relatively prime to $m$. Namely, $p_i^{1}, p_i^{2}, \ldots, p_i^{k_i}$.
Thus, $k_i \leq m - \phi(m)$.

\bigbreak

From this we can conclude $p_i^{k_i} \vert a^{m - \phi(m)}$ which, in turn,
shows:
$$ a^{m - \phi(m)} \equiv 0 \pmod{p_i^{k_i}} $$
\end{proof}

\break

\subsection*{Part C}



\end{document}

