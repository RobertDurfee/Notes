\documentclass{article}
\usepackage{tikz}
\usepackage{float}
\usepackage{enumerate}
\usepackage{amsmath}
\usepackage{amsthm}
\usepackage{amsfonts}
\usepackage{bm}
\usepackage{indentfirst}
\usepackage{siunitx}
\usepackage[utf8]{inputenc}
\usepackage{graphicx}
\graphicspath{ {Images/} }
\usepackage{float}
\usepackage{mhchem}
\usepackage{chemfig}
\allowdisplaybreaks

\title{6.042 Problem Set 7}
\author{Robert Durfee}
\date{April 27, 2018}

\begin{document}

\maketitle

\section*{Problem 1}

I spent approximately 45 minutes working on this problem in collaboration with
Daniela Guillen.

\subsection*{Part A}

\begin{proof}
  Rearranging the inequality $\log_2 x < x$ to,
  $$ x - \log_2 x > 0 $$
  Letting $f(x) = x - \log_2 x$, take the derivative,
  $$ f'(x) = 1 - \frac{1}{x \ln 2} $$
  This is equal to zero when
  $$ x = \frac{1}{\ln 2} = 1.4427 $$
  Which corresponds to the value
  $$ f(1.4427) = 0.91329 $$
  Since this is the minimum value of $f(x)$, all other values must be greater
  and since $0.91329 > 0$, then $x - \log_2 x > 0$ for all $x > 0$ which in turn
  shows, $$ \log_2 x < x\ \forall x > 1 $$
\end{proof}

\break

\subsection*{Part B}

\begin{proof}
  To prove a strict partial order, one must prove both irreflexivity and
  transitivity. 

  \bigbreak

  \textbf{Irreflexivity}: To show irreflexivity, there cannot be a relation
  $f(x)\ R\ f(x)$. Since the relation $f(x)\ R\ g(x)$ is equivalent to $g(x) =
  o(f(x))$, we can use limits to describe the relation as,
  $$ \lim_{x \to \infty} \frac{f(x)}{g(x)} = 0$$
  But in the case of $f(x)\ R\ f(x)$, the limit takes the form,
  $$ \lim_{x \to \infty} \frac{f(x)}{f(x)} = 1$$
  Since $1 \neq 0$, the relation is not reflexive, therefore it is irreflexive.

  \bigbreak

  \textbf{Transitivity}: To show transitivity, the relation must satisfy,
  $$ f(x)\ R\ g(x) \land g(x)\ R\ h(x) \implies f(x)\ R\ h(x) $$
  Since the relation $f(x)\ R\ g(x)$ is equivalent to $g(x) = o(f(x))$, limits
  can be used once again,
  $$ \lim_{x \to \infty} \frac{g(x)}{f(x)} = 0 $$
  $$ \lim_{x \to \infty} \frac{h(x)}{g(x)} = 0 $$
  We can use these limits to construct $f(x)\ R\ h(x)$,
  $$ \lim_{x \to \infty} \frac{g(x)}{f(x)} \cdot \lim_{x \to \infty}
  \frac{h(x)}{g(x)} = \lim_{x \to \infty} \frac{f(x)}{h(x)} $$
  And we can substitute zero for the limits on the left side to show,
  $$ \lim_{x \to \infty} \frac{f(x)}{h(x)} = 0 $$
  Therefore, $h(x) = o(f(x))$ which further means $f(x)\ R\ h(x)$. So the
  relation is transitive.

  \bigbreak

  Since both irreflexivity and transitivity are satisfied, $R$ is a strict
  partial order.
\end{proof}

\break

\subsection*{Part C}

\begin{proof}
  To prove $f \sim g \iff f = g + h$ for some $h = o(g)$, I will prove the
  implication $ f = g + h \implies f \sim g $ and the contrapositive $ f \neq g
  + h \implies f \not\sim g $.

  \bigbreak

  \textbf{Implication}: From $f(x) = g(x) + h(x)$, it must be,
  $$ \lim_{x \to \infty} \frac{f(x)}{g(x)} = \lim_{x \to \infty} \frac{g(x) +
  h(x)}{g(x)} $$
  Then, the right side can be expanded using the rules of addition of limits,
  $$ \lim_{x \to \infty} \frac{f(x)}{g(x)} = \lim_{x \to \infty}
  \frac{g(x)}{g(x)} + \lim_{x \to \infty} \frac{h(x)}{g(x)} $$
  It is clear that the first term on the left equals $1$ and the second term
  must equal zero because $h(x) = o(g(x))$. Therefore,
  $$ \lim_{x \to \infty} \frac{f(x)}{g(x)} = 1 $$
  This is the definition of asymptotic equivalence, therefore,
  $$ f(x) \sim g(x) $$

  \bigbreak

  \textbf{Contrapositive}: Essentially repeating the proof above, from $f(x)
  \neq g(x) + h(x)$, it must be,
  $$ \lim_{x \to \infty} \frac{f(x)}{g(x)} \neq \lim_{x \to \infty} \frac{g(x) +
  h(x)}{g(x)} $$
  But we just showed that the right side can be written,
  $$ \lim_{x \to \infty} \frac{g(x) + h(x)}{g(x)} = \lim_{x \to \infty}
  \frac{g(x)}{g(x)} + \lim_{x \to \infty} \frac{h(x)}{g(x)} = 1 $$
  Therefore,
  $$ \lim_{x \to \infty} \frac{f(x)}{g(x)} \neq 1 $$
  But this is the requirement for asymptotic equivalence, therefore,
  $$ f(x) \not\sim g(x) $$

  \bigbreak

  Since the implication and the contrapositive hold, both direction of the iff
  hold. Therefore, $f \sim g \iff f = g + h$.

\end{proof}

\break

\section*{Problem 2}

I spent approximately 20 minutes working on this problem in collaboration with
Daniela Guillen.

\bigbreak

From Theorem 14.3.2, if $f: \mathbb{R}^+ \rightarrow \mathbb{R}^+$ is a weakly
decreasing function, then
$$ \int\limits_0^n f(x) dx + f(n) \leq \sum\limits_{i=1}^{n} f(i) \leq
\int\limits_0^n f(x) dx + f(1) $$

For this particular problem,
$$ f(i) = \frac{1}{(2i + 1)^2} $$
Which is clearly a decreasing function because as $i$ increases, $f(i)$ will get
smaller. Thus, Theorem 14.3.2 can be used.

Substituting $f(i)$ and $n = \infty$ into the equation from Theorem 14.3.2,
$$ \int\limits_1^\infty \frac{dx}{(2x + 1)^2} + \lim_{x \to \infty} \frac{1}{(2x
+ 1)^2} \leq \sum\limits_{i=1}^{\infty} \frac{1}{(2i + 1)^2} \leq
\int\limits_1^\infty \frac{dx}{(2x + 1)^2} + \frac{1}{(2 \cdot 1 + 1)^2} $$
Solving the integrals and limits yields,
$$ \frac{1}{6} + 0 \leq \sum\limits_{i=1}^{\infty} \frac{1}{(2i + 1)^2} \leq
\frac{1}{6} + \frac{1}{9} $$
However, this lower and upper bound only restricts the result to a range of
$0.\overline{11}$. Therefore, a tighter bound is required.

Repeating this process but instead starting when $i = 2$,
$$ \int\limits_2^\infty \frac{dx}{(2x + 1)^2} + \lim_{x \to \infty} \frac{1}{(2x
+ 1)^2} \leq \sum\limits_{i=2}^{\infty} \frac{1}{(2i + 1)^2} \leq
\int\limits_2^\infty \frac{dx}{(2x + 1)^2} + \frac{1}{(2 \cdot 2 + 1)^2} $$
Solving the integrals and limits yields,
$$ \frac{1}{10} + 0 \leq \sum\limits_{i=2}^{\infty} \frac{1}{(2i + 1)^2} \leq
\frac{1}{10} + \frac{1}{25} $$
And this lower and upper bound restricts the new result to a range of $0.04$.
Therefore, this bound can be used.

Now, this bound must be converted to support the previous sum which starts at $i
= 1$. This can be done by adding $f(1) = 1/9$ to each of the three terms,
$$ \frac{1}{10} + \frac{1}{9} \leq \sum\limits_{i=2}^{\infty} \frac{1}{(2i +
1)^2} + \frac{1}{9} \leq \frac{1}{10} + \frac{1}{25} + \frac{1}{9} $$
So the center term now becomes the original sum,
$$ \frac{1}{10} + \frac{1}{9} \leq \sum\limits_{i=1}^{\infty} \frac{1}{(2i +
1)^2} \leq \frac{1}{10} + \frac{1}{25} + \frac{1}{9} $$
Evaluating the fractions,
$$ 0.2\overline{11} \leq \sum\limits_{i=1}^{\infty} \frac{1}{(2i + 1)^2} \leq
0.25\overline{1} $$

\break

\section*{Problem 3}

I spent approximately 10 minutes working on this problem in collaboration with
Daniela Guillen.

\subsection*{Part A}

The total, nominal wealth accumulated by a Harvard graduate at the end of $n$
years including a $\$20,000$ raise each year can be represented as,
$$ W_N = \sum\limits_{i = 0}^{n - 1} \$40,000 + \$20,000 i $$
Adjusting for inflation will determine the total, real wealth accumulated by a
Harvard graduate at the end of $n$ years in terms of today's dollars with a
fixed $8\%$ inflation rate,
$$ W_R = \sum\limits_{i = 0}^{n - 1} \frac{\$40,000 + \$20,000 i}{1.08^i} $$

\break

\subsection*{Part B}

The total, nominal wealth accumulated by an MIT graduate at the end of  $n$
years including a $20\%$ raise each year can be represented as,
$$ W_N = \sum\limits_{i = 0}^{n - 1} \$30,000 \cdot 1.20^i $$
Adjusting for inflation will determine the total, real wealth accumulated by an
MIT graduate at the end of $n$ years in terms of today's dollars with a fixed
$8\%$ inflation rate,
$$ W_R = \sum\limits_{i = 0}^{n - 1} \frac{\$30,000 \cdot 1.20^i}{1.08^i} $$

\break

\subsection*{Part C}

Substituting $n = 20$ into the sum for a Harvard graduate's real, total wealth,
$$ W_R = \sum\limits_{i = 0}^{19} \frac{\$40,000 + \$20,000 i}{1.08^i} =
\$1,916,483.45 $$
And substituting $n = 20$ into the sum for an MIT graduate's real total wealth,
$$ W_R = \sum\limits_{i = 0}^{19} \frac{\$30,000 \cdot 1.20^i}{1.08^i} =
\$1,950,821.10 $$

Since the present value of an MIT graduate's total future earnings after 20
years is larger than a Harvard graduate's, the MIT degree is worth more if you
plan to retire in 20 years.

\end{document}

