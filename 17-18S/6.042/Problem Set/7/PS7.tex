\documentclass{article}
\usepackage{tikz}
\usepackage{float}
\usepackage{enumerate}
\usepackage{amsmath}
\usepackage{amsthm}
\usepackage{amsfonts}
\usepackage{bm}
\usepackage{indentfirst}
\usepackage{siunitx}
\usepackage[utf8]{inputenc}
\usepackage{graphicx}
\graphicspath{ {Images/} }
\usepackage{float}
\usepackage{mhchem}
\usepackage{chemfig}
\allowdisplaybreaks
\newtheorem{theorem}{Theorem}
\newtheorem{lemma}[theorem]{Lemma}

\title{6.042 Problem Set 7}
\author{Robert Durfee}
\date{April 27, 2018}

\begin{document}

\maketitle

\section*{Problem 1}

\subsection*{Part A}

\begin{lemma}
  If $f$ is a continuous function on the interval $[a, b]$, differentiable on
  the interval $(a, b)$, and $f'(c) > 0\ \forall c \in (a, b)$, then $\forall
  x_1 < x_2 \in (a, b)$, $f(x_1) < f(x_2)$.
\end{lemma}

\begin{proof}
  Let $f$ be a continuous, differentiable function with $f'(x) > 0$ on the
  interval $[x_1, x_2]$ where $x_1 < x_2$. According to the Mean-Value Theorem,
  there exists some $x_1 < c < x_2$ such that,
  $$ f(x_2) - f(x_1) = f'(c)(x_2 - x_1) $$
  Since $f'(x) > 0\ \forall x \in [x_1, x_2]$, $f'(c) > 0$. Also, since $x_2 >
  x_1$, $x_2 - x_1 > 0$. Therefore, the right term is positive. Since the right
  term is positive, the left term must be as well. Therefore, $f(x_2) - f(x_1) >
  0$.  Shifting terms provides,
  $$ f(x_1) < f(x_2) $$
\end{proof}

Lemma 1 can be used to show that $\log_2 x < x$. Rearrange terms to find,
$$ 0 < x - \log_2 x $$
Now set $f(x) = x - \log_2 x$ and differentiate $f$ to get,
$$ f'(x) = 1 - \frac{1}{x \ln 2} $$
$f'(x) > 0$ when $x > 1 / \ln 2$. Thus, by Lemma 1, 

\break

\subsection*{Part B}

\break

\subsection*{Part C}

\break

\section*{Problem 2}

I spent approximately 20 minutes working on this problem alone using only course
materials from this term.

\bigbreak

From Theorem 14.3.2, if $f: \mathbb{R}^+ \rightarrow \mathbb{R}^+$ is a weakly
decreasing function, then
$$ \int\limits_0^n f(x) dx + f(n) \leq \sum\limits_{i=1}^{n} f(i) \leq
\int\limits_0^n f(x) dx + f(1) $$

For this particular problem,
$$ f(i) = \frac{1}{(2i + 1)^2} $$
Which is clearly a decreasing function because as $i$ increases, $f(i)$ will get
smaller. Thus, Theorem 14.3.2 can be used.

Substituting $f(i)$ and $n = \infty$ into the equation from Theorem 14.3.2,
$$ \int\limits_1^\infty \frac{dx}{(2x + 1)^2} + \lim_{x \to \infty} \frac{1}{(2x
+ 1)^2} \leq \sum\limits_{i=1}^{\infty} \frac{1}{(2i + 1)^2} \leq
\int\limits_1^\infty \frac{dx}{(2x + 1)^2} + \frac{1}{(2 \cdot 1 + 1)^2} $$
Solving the integrals and limits yields,
$$ \frac{1}{6} + 0 \leq \sum\limits_{i=1}^{\infty} \frac{1}{(2i + 1)^2} \leq
\frac{1}{6} + \frac{1}{9} $$
However, this lower and upper bound only restricts the result to a range of
$0.\overline{11}$. Therefore, a tighter bound is required.

Repeating this process but instead starting when $i = 2$,
$$ \int\limits_2^\infty \frac{dx}{(2x + 1)^2} + \lim_{x \to \infty} \frac{1}{(2x
+ 1)^2} \leq \sum\limits_{i=2}^{\infty} \frac{1}{(2i + 1)^2} \leq
\int\limits_2^\infty \frac{dx}{(2x + 1)^2} + \frac{1}{(2 \cdot 2 + 1)^2} $$
Solving the integrals and limits yields,
$$ \frac{1}{10} + 0 \leq \sum\limits_{i=2}^{\infty} \frac{1}{(2i + 1)^2} \leq
\frac{1}{10} + \frac{1}{25} $$
And this lower and upper bound restricts the new result to a range of $0.04$.
Therefore, this bound can be used.

Now, this bound must be converted to support the previous sum which starts at $i
= 1$. This can be done by adding $f(1) = 1/9$ to each of the three terms,
$$ \frac{1}{10} + \frac{1}{9} \leq \sum\limits_{i=2}^{\infty} \frac{1}{(2i +
1)^2} + \frac{1}{9} \leq \frac{1}{10} + \frac{1}{25} + \frac{1}{9} $$
So the center term now becomes the original sum,
$$ \frac{1}{10} + \frac{1}{9} \leq \sum\limits_{i=1}^{\infty} \frac{1}{(2i +
1)^2} \leq \frac{1}{10} + \frac{1}{25} + \frac{1}{9} $$
Evaluating the fractions,
$$ 0.2\overline{11} \leq \sum\limits_{i=1}^{\infty} \frac{1}{(2i + 1)^2} \leq
0.25\overline{1} $$

\break

\section*{Problem 3}

I spent approximately 10 minutes working on this problem alone using only course
materials from this term.

\subsection*{Part A}

The total, nominal wealth accumulated by a Harvard graduate after $n$ years
including a $\$20,000$ raise each year can be represented as,
$$ W_N = \sum\limits_{i = 0}^n \$40,000 + \$20,000 i $$
Adjusting for inflation will determine the total, real wealth accumulated by a
Harvard graduate after $n$ years in terms of today's dollars with a fixed $8\%$
inflation rate,
$$ W_R = \sum\limits_{i = 0}^n \frac{\$40,000 + \$20,000 i}{1.08^i} $$

\break

\subsection*{Part B}

The total, nominal wealth accumulated by an MIT graduate after $n$ years
including a $20\%$ raise each year can be represented as,
$$ W_N = \sum\limits_{i = 0}^n \$30,000 \cdot 1.20^i $$
Adjusting for inflation will determine the total, real wealth accumulated by an
MIT graduate after $n$ years in terms of today's dollars with a fixed $8\%$
inflation rate,
$$ W_R = \sum\limits_{i = 0}^n \frac{\$30,000 \cdot 1.20^i}{1.08^i} $$

\break

\subsection*{Part C}

Substituting $n = 20$ into the sum for a Harvard graduate's real, total wealth,
$$ W_R = \sum\limits_{i = 0}^{20} \frac{\$40,000 + \$20,000 i}{1.08^i} =
\$2,010,884.66 $$
And substituting $n = 20$ into the sum for an MIT graduate's real total wealth,
$$ W_R = \sum\limits_{i = 0}^{20} \frac{\$30,000 \cdot 1.20^i}{1.08^i} =
\$2,198,579.00 $$

Since the present value of an MIT graduate's total future earnings after 20
years is larger than a Harvard graduate's, the MIT degree is worth more if you
plan to retire in 20 years.

\end{document}

