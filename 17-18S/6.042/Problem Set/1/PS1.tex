\documentclass{article}
\usepackage{tikz}
\usepackage{float}
\usepackage{enumerate}
\usepackage{amsmath}
\usepackage{bm}
\usepackage{indentfirst}
\usepackage{siunitx}
\usepackage[utf8]{inputenc}
\usepackage{graphicx}
\graphicspath{ {Images/} }
\usepackage{float}
\usepackage{mhchem}
\usepackage{chemfig}
\allowdisplaybreaks

\title{ 6.042 Problem Set 1 }
\author{ Robert Durfee }
\date{ February 16, 2018 }

\begin{document}

\maketitle

\section*{Problem 1 }

I spent approximately 1 hour working on this problem in collaboration with
Daniela Guillen and James Quigley.

\subsection*{Part A}

The polynomial $ x^{m} - k $ has the root $\sqrt[m]{k}$. It is also in the form
of the polynomial provided in the lemma.  Where $a_{0}$ is $-1$, $a_{1}$ through
$a_{m-1}$ are 0, and $a_{m}$ is 1. As a result, the root of this polynomial, as
stated by the lemma, must me integral or irrational. $\sqrt[m]{k}$ will only be
an integer if $k = n^{m}$, where $n$ is an integer. This is clear because an
integer raised to any other power will not simplify completely, but will contain
an irreducible root and, therefore will be irrational.  And this is the second
stipulation provided by the lemma.

\subsection*{Part B}

\subsubsection*{Lemma}

\textit{Let the coefficients of the polynomial}
$$ a_{0} + a_{1}x + a_{2}x^{2} + \ldots + a_{m-1}x^{m-1} + a_{m}x^{m} $$
\textit{be integers. Then any real root of the polynomial is either integral or
irrational.}

\subsubsection*{Proof}

The proof is by contradiction.

\section*{Problem 2}

I spent approximately 30 minutes working on this problem in collaboration with
Daniella Guillen and James Quigley.

\subsection*{Part A}

\subsubsection*{Theorem}

\textit{Suppose that $a + b + c = d$ where $a$, $b$, $c$, and $d$ are nonnegative
integers. Let $P$ be the assertion that $d$ is even. Let $W$ be the assertion
that exactly one among $a$, $b$, and $c$ are even, and let $T$ be the assertion
that all three are even. Then, $ P \iff (W \lor T) $.}

\subsubsection*{Proof}

The proof is by case analysis. There are three cases:

\begin{enumerate}[\hspace{1cm}1.]
  \item $W$ is true and $T$ is false.
  \item $W$ is false and $T$ is true.
  \item $W$ is false and $T$ is false.
\end{enumerate}

There is no fourth case where both are true as the two assertions $W$ and
$T$ are mutually exclusive.

\subsubsection*{Case 1: $W$ is True and $T$ is False}

Let's assume that $a$ and $b$ are odd and $b$ and $c$ is even. We know that
the sum of two odd numbers is always even. As a result, the sum of $a$ and $b$
is even. Now, the problem is reduced to the sum of two even numbers. We also
know that the sum of two even numbers is also even. As a result, $d$ must be
even.

\subsubsection*{Case 2: $W$ is False and $T$ is True}

Now, $a$, $b$, and $c$ are all even. Since the sum of two evens is even, the sum
of $a$ and $b$ must be even. Now, the problem is reduced to the sum of two even
numbers. As a result, $d$ must be even.

\subsubsection*{Case 3: $W$ is False and $T$ is False}

Now, $a$, $b$, and $c$ are all odd. We know that the sum of two odd numbers is
even. Thus, the sum of $a$ and $b$ is even. Now, the problem is reduced to the
sum of an even and an odd number. We also know that the sum of an even and an
odd number is odd. As a result, $d$ must be odd.

\bigbreak

Given that $d$ is only even when $W \lor T$ evaluates to true, and $P$ is the
assertion that $d$ is even, $ P \iff (W \lor T) $.

\subsection*{Part B}

\subsubsection*{Theorem}

\textit{Suppose that $w^{2} + x^{2} + y^{2} = z^{2}$, where $w$, $x$, $y$, $z$
are nonnegative integers. Let $P$ be the assertion that $z$ is even, and let $R$
be the assertion that all three of $w$, $x$, and $y$ are even. Then, $P \iff R$.}

\subsubsection*{Proof}

The proof is by case analysis. There are two cases:

\begin{enumerate}[\hspace{1cm}1.]
  \item $R$ is true.
  \item $R$ is false.
\end{enumerate}

\subsubsection*{Case 1: $R$ is True}

If $R$ is true, then $w$, $x$, and $y$ must all be even. We know that an even
number times another even number will also be even. As a result, $w^{2}$,
$x^{2}$, and $y^{2}$ will all be even as well. From before, we also know that
the sum of even numbers will be even. Therefore, $z^{2}$ must be even as well.
Once again, we know that the square of a number can only be even if that number
is even itself. Thus, $z$ is even.

\subsubsection*{Case 2: $R$ is False}

If $R$ is false, then at least one of $w$, $x$, and $y$ must be odd. We know
that a square of an odd number takes the form $4(m^{2} + m) + 1$, where $m$ is
some integer. As a result, the sum of one odd and two evens will have a $1$
term, a sum of two odds and one even will have a $2$ term, and a sum of three
odds with have a $3$ term. But, in order for $z$ to be even, $z^{2}$ must be a
multiple of $4$. As a result, none of the previously listed terms is a multiple
of $4$, so $z$ cannot be even.

\bigbreak

Given that $z$ is only even when $R$ evaluates to true, and $P$ is the assertion
that $z$ is even, $P \iff R$.

\section*{Problem 3}

I spent approximately 10 minutes working on this problem in collaboration with
Daniela Guillen and James Quigley.

\subsection*{Part A}

\begin{center}
  \begin{tabular}{ c c c c c c c c }
    $P$ & $Q$ & $R$ & $[(P \implies Q)$ & $\land$ & $(Q \implies R)]$ & $\land$ &
    $(R \implies P)$ \\
    T & T & T & T & T & T & \textbf{T} & T \\
    T & T & F & T & F & F & \textbf{F} & T \\
    T & F & T & F & F & T & \textbf{F} & T \\
    T & F & F & F & F & T & \textbf{F} & T \\
    F & T & T & T & T & T & \textbf{F} & F \\
    F & T & F & T & F & F & \textbf{F} & T \\
    F & F & T & T & T & T & \textbf{F} & F \\
    F & F & F & T & T & T & \textbf{T} & T \\
  \end{tabular}
\end{center}

\begin{center}
  \begin{tabular}{ c c c c c c }
    $P$ & $Q$ & $R$ & $[P \land Q]$ & $\land$ & $R$ \\
    T & T & T & T & \textbf{T} &  \\
    T & T & F & T & \textbf{F} &  \\
    T & F & T & F & \textbf{F} &  \\
    T & F & F & F & \textbf{F} &  \\
    F & T & T & F & \textbf{F} &  \\
    F & T & F & F & \textbf{F} &  \\
    F & F & T & F & \textbf{F} &  \\
    F & F & F & F & \textbf{F} &  \\
  \end{tabular}
\end{center}

The truth assignment that makes the first proposition true and the second
proposition false is $P = \rm{F}$, $Q = \rm{F}$, and $R = \rm{F}$.

\subsection*{Part B}

Sloppy Sam's mistake was in assuming that $Q$ must be true in the first place.
He only proved that $P$, $Q$, and $R$ are all true if one of them is true.
However, this doesn't prove that one has to be true in the first place. If all
$P$, $Q$, and $R$ are false, then all the implications will be true.

\end{document}

