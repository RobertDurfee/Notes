\documentclass{article}
\usepackage{tikz}
\usepackage{float}
\usepackage{enumerate}
\usepackage{amsmath}
\usepackage{amssymb}
\usepackage{bm}
\usepackage{indentfirst}
\usepackage{siunitx}
\usepackage[utf8]{inputenc}
\usepackage{graphicx}
\graphicspath{ {Images/} }
\usepackage{float}
\usepackage{mhchem}
\usepackage{chemfig}
\allowdisplaybreaks

\title{ 6.042 Session 5 }
\author{ Robert Durfee }
\date{ February 16, 2018 }

\begin{document}

\maketitle

\section*{Problem 1 }

\begin{center}
  \begin{tabular}{ c c c c c c }
    Predicate & $\mathbb{N}$ & $\mathbb{Z}$ & $\mathbb{Q}$ & $\mathbb{R}$ & $\mathbb{C}$ \\
    $\exists x . x^{2} = 2 $ & $x = \sqrt{2}$ & $x = \sqrt{2}$ & $x = \sqrt{2}$ & T & T \\
    $\forall x . \exists y . x^{2} = y$ & T & T & T & T & T \\
    $\forall y . \exists x . x^{2} = y$ & $y = 2$ & $y = 2$ & $y = 2$ & $y = -2$ & T \\
    $\forall x \neq 0 . \exists y . xy = 1$ & $x = 2$ & $x = 2$ & T & T & T \\
    $\exists x . \exists y . x + 2y = 2 \land 2x + 4y = 5$ & F & F & F & F & F
  \end{tabular}
\end{center}

\section*{Problem 2}

\begin{enumerate}[a.]
  \item $\exists y . (x = yyy)$
  \item $\exists y . (x = yy) \land (\mathrm{NoOnes}(x))$
  \item $\mathrm{NoOnes}(x) \lor \overline{\mathrm{Substring}(0, x)}$
  \item $(\exists y . ((x = 1y1) \land \mathrm{NoOnes}(y))) \lor (x = 10)$
  \item If $x$ begins with $1$, prefix will fail because $0x$ doesn't begin with
    $1$. If $x$ is $\lambda$, prefix will be true because nothing always comes
    before a binary string. If $x$ begins with $0$, prefix will be true as long
    as a zero follows which will continue recursively terminating only when $x$
    is a string of $0$, evaluating true, or a $1$ is encountered, evaluating to
    false.
\end{enumerate}

\section*{Problem 3}

\begin{enumerate}[1.]
  \item Not a counter model. All rationals have inverses except zero. However,
    since the hypothesis is false, the implies is always true.
  \item This is not a number smaller than every number in the set of reals. As a
    result, this is a counter model.
  \item This is not a number that, when multiplied to every number, makes it 2.
    As a result, this is a counter model.
  \item Both cases, if $y = \lambda$ will be true. As a result, this is not a
    counter model.
\end{enumerate}

\section*{Problem 4}

$$ \exists x . \exists y . \exists z . \forall s . (E(x, s) \implies (s = x \lor
s = y \lor s = z)) $$

\end{document}

