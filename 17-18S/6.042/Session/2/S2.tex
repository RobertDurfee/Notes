\documentclass{article}
\usepackage{tikz}
\usepackage{float}
\usepackage{enumerate}
\usepackage{amsmath}
\usepackage{amsthm}
\usepackage{bm}
\usepackage{indentfirst}
\usepackage{siunitx}
\usepackage[utf8]{inputenc}
\usepackage{graphicx}
\graphicspath{ {Images/} }
\usepackage{float}
\usepackage{mhchem}
\usepackage{chemfig}
\allowdisplaybreaks

\title{ 6.042 Session 2 }
\author{ Robert Durfee }
\date{ February 9, 2018 }

\begin{document}

\maketitle

\section*{Problem 1 }

\subsection*{Proof}

The proof is by contradiction. Assume that $a, b > \sqrt{n}$. Then, $a =
\sqrt{n} + p$ and $b = \sqrt{n} + q$ for some $p, q > 0$. As a result,
$$ ab = n + p \sqrt{n} + q \sqrt{n} + pq $$

Since all the terms are positive, this shows that $ab > n$. This is a
contradiction. Therefore, either $a$ or $b$ must be $ \leq \sqrt{n}$. \qed

\section*{Problem 2}

\subsection*{Proof}

The proof is by contradiction. Assume that $\sqrt{3}$ is rational. As a result,
it can be rewritten as a fraction in lowest terms:
$$ \sqrt{3} = \frac{ n }{ d } $$

By removing the square root:
$$ 3 = \frac{ n^{2} }{ d^{2} } $$

And multiplying each side by $d^{2}$:
$$ 2 d^{2} = n^{2} $$

Since $d$ is an integer, $d^{2}$ must also be an integer. Therefore, $n$ can be
written:
$$ n = 3k $$

This shows that $n$ is a multiple of 3. Therefore, $d^{2}$ can be written as:
$$ 3 d^{2} = \left( 3 k \right)^{2} $$

And simplifying this to:
$$ d^{2} = 3 k^{2} $$

This shows that $d$ must also be divisible by 3. Since both $d$ and $n$ are
multiples of three, the fraction $n/d$ is not written in lowest terms and
therefore is a contradiction. As a result, $\sqrt{3}$ must be irrational.

\bigbreak

It is important to note that this holds for all numbers which aren't perfect
squares which would have integer solutions instead of irrational. \qed

\section*{Problem 3}

\subsection*{Proof}

The proof is by cases. In the following equation, there are three cases $r > s$,
$r < s$, and $r = s$.
$$ \max(r, s) + \min(r, s) = r + s $$

\subsubsection*{Case 1: $r > s$}

If $r > s$, then $\max(r, s) = r$ and $\min(r, s) = s$. Then the right side
simplifies to $r + s$ which is equivalent to the left side.

\subsubsection*{Case 2: $r < s$}

If $r < s$, then $\max(r, s) = s$ and $\min(r, s) = r$. Then the right side
simplifies to $s + r$ which, by commutativity, is equivalent to the left side.

\subsubsection*{Case 3: $r = s$}

If $r = s$, then the minimum and maximum functions will each return either $r$
or $s$. The sum of any combination of $r$ and $s$ will be the same as $r = s$.

\bigbreak

All cases show that the left side of the equation simplifies to $r + s$. Since
these are all the cases, the equation must be true. \qed

\section*{Problem 4}

\subsection*{Proof}

The proof is by cases. There are two cases to consider: $\sqrt{2}^{\sqrt{2}}$ is
rational or irrational.

\subsubsection*{Case 1: $\sqrt{2}^{\sqrt{2}}$ is Irrational}

If $\sqrt{2}^{\sqrt{2}}$ is irrational, then it can be raised to $\sqrt{2}$:
$$ \left( \sqrt{2}^{\sqrt{2}} \right)^{\sqrt{2}} = \sqrt{2}^{\sqrt{2} \cdot
\sqrt{2}} = \sqrt{2}^{(\sqrt{2})^{2}} =
\sqrt{2}^{2} = 2 $$

As a result, an irrational number raised to an irrational power could be
rational since we know that $\sqrt{2}$ is irrational.

\subsubsection*{Case 2: $\sqrt{2}^{\sqrt{2}}$ is Rational }

If $\sqrt{2}^{\sqrt{2}}$ is rational, then we can already see that an irrational
number raised to an irrational power is rational since we know $\sqrt{2}$ is
irrational.

\bigbreak

Since both cases result in an irrational number raised to an irrational power
resulting in a rational output, and the cases are exhaustive, it must be
possible for an irrational number raised to an irrational power to be rational.
\qed

\end{document}

