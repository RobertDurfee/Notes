\documentclass{article}
\usepackage{tikz}
\usepackage{float}
\usepackage{enumerate}
\usepackage{amsmath}
\usepackage{amsthm}
\usepackage{bm}
\usepackage{indentfirst}
\usepackage{siunitx}
\usepackage[utf8]{inputenc}
\usepackage{graphicx}
\graphicspath{ {Images/} }
\usepackage{float}
\usepackage{mhchem}
\usepackage{chemfig}
\allowdisplaybreaks

\title{ 6.042 Session 7 }
\author{ Robert Durfee }
\date{ February 24, 2018 }

\begin{document}

\maketitle

\section*{Problem 1}

Let $S$ be a set of non-makeable cents greater than or equal to 12 and multiples of
3. Assume that the set is non empty. Let the smallest element in this set be
$m$. We know that 12 and 15 are makeable ($6 + 6 = 12$ and $15 = 15$). As a
result, $m$ must be greater than or equal to 18. Since $m$ is the smallest
element in set $S$, $m - 6$ must be makeable. Adding $6$ keeps $m$ makeable so
it can't be in the set. This is a contradiction, therefore the set is empty.
\qed

\section*{Problem 2}

Let $S$ be a set of nonnegative integers such that the following equation
doesn't hold:
$$ \sum\limits_{k=1}^{n} k^{2} = \frac{ n (n + 1) (2n + 1) }{ 6 } $$

Let the smallest integer in $S$ be $m$. Since $1^{2} = 1$, $m$ must be greater
than 1. Since $m$ is the smallest, $m - 1$ must solve the above equation. This
can be written:
$$ 1^{2} + 2^{2} + \ldots + (m - 1)^{2} = \frac{ m (m - 1) (2m - 1) }{ 6 } $$

If we add $m^{2}$ to both sides of the equation, it becomes:
$$ 1^{2} + 2^{2} + \ldots + (m - 1)^{2} + m^{2} = \frac{ m (m + 1) (2m + 1) }{ 6 } $$

This is equivalent to the first equation. As a result, this is a contradiction
and $m$ is not in $S$ so $S$ must be empty. \qed

\section*{Problem 3}

I didn't write out the proof for this problem because it is quite long. However,
the principle behind the proof is in the coefficients. Since $d$ is even, it can
be written as $2 d'^{2}$. This, in turn, proves $c$ is even and so forth. As a
result, a $2$ can be factored out infinitely. As a result, there can be no
solution. \qed

\section*{Problem 4}

\begin{enumerate}[a.]
  \item $n - 1$
  \item For the serial AND circuit, the longest path has $n - 1$ gates. The
    branched circuit only has a longest path of $\log_{2}(n)$ gates.
\end{enumerate}

\end{document}

