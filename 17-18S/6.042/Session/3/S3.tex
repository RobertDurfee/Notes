\documentclass{article}
\usepackage{tikz}
\usepackage{float}
\usepackage{enumerate}
\usepackage{amsmath}
\usepackage{amsthm}
\usepackage{bm}
\usepackage{indentfirst}
\usepackage{siunitx}
\usepackage[utf8]{inputenc}
\usepackage{graphicx}
\graphicspath{ {Images/} }
\usepackage{float}
\usepackage{mhchem}
\usepackage{chemfig}
\allowdisplaybreaks

\title{ 6.042 Session 3 }
\author{ Robert Durfee }
\date{ February 12, 2018 }

\begin{document}

\maketitle

\section*{Problem 1}

\subsection*{Proof}

\begin{center}
  \begin{tabular}{ c c c c c c }
    $P$ & $Q$ & $R$ & $P$ & $\lor$ & $(Q\ \land\ R)$ \\
    T & T & T & & T & T \\
    T & T & F & & T & F \\
    T & F & T & & T & F \\
    T & F & F & & T & F \\
    F & T & T & & T & T \\
    F & T & F & & F & F \\
    F & F & T & & F & F \\
    F & F & F & & F & F \\
  \end{tabular}
\end{center}

\begin{center}
  \begin{tabular}{ c c c c c c }
    $P$ & $Q$ & $R$ & $(P\ \lor\ Q)$ & $\land$ & $(P\ \lor\ R)$ \\
    T & T & T & T & T & T\\
    T & T & F & T & T & T\\
    T & F & T & T & T & T\\
    T & F & F & T & T & T\\
    F & T & T & T & T & T\\
    F & T & F & T & F & F\\
    F & F & T & F & F & T\\
    F & F & F & F & F & F\\
  \end{tabular}
\end{center}

Since second to last column in each table (the final output) are equal for all
truth values, the formulas are equivalent.\qed

\section*{Problem 2}

\begin{enumerate}[a.]
  \item \begin{enumerate}[1.]
      \item \begin{enumerate}[a.]
        \item $\overline{L} \implies Q$
        \item $\overline{L} \implies B$
        \item $\overline{L} \iff N$
        \end{enumerate}
      \item $\overline{Q} \implies B$
      \item $\overline{B}$
    \end{enumerate}
  \item The following formula is satisfiable.
    $$ (\overline{L} \implies Q) \land (\overline{L} \implies B) \land
    (\overline{L} \iff N) \land (\overline{Q} \implies B) \land \overline{B}$$

    An environment that makes this satisfiable is $\langle L, Q, B, N \rangle =
    \langle T, T, F, F \rangle$.

  \item In order for the previous formula to be true, $B$ must be false. Working
    backward from $B$, $Q$ and $L$ have to be true. Given that $L$ is true, $N$
    has to be false.

\end{enumerate}

\section*{Problem 3}

Proposition one and two are equivalent to formula one because a differentiable
function must be continuous, but a continuous function does not need to be
differentiable. For example, the function may have a corner.

Propositions three and four are not reasonably converted to formula two because
although watching TV implies homework is done, not watching TV doesn't imply
that homework is not done. For example, there could be more chores to be done.

\section*{Problem 4}

\begin{enumerate}[a.]
  \item Since the second to last column (final output) of the following truth
    table is always true, the formula is valid.

    \begin{center}
      \begin{tabular}{ c c c c c }
        $P$ & $Q$ & $(P \implies Q)$ & $\lor$ & $(Q \implies P)$ \\
        T & T & T & T & T \\
        T & F & F & T & T \\
        F & T & T & T & F \\
        F & F & T & T & T
      \end{tabular}
    \end{center}

  \item $$ R ::= ((P \land Q) \lor (\overline{P} \land \overline{Q})) $$

  \item If $P$ is always true (valid), then $\overline{P}$ is always false (not
    satisfiable).

    If $\overline{P}$ is always false (not satisfiable), then $P$ is always true
    (valid).

    \bigbreak

    Since both implications are true, the iff must be true.  \qed

  \item $$ S ::= \overline{P_{1} \land P_{2} \land \dots \land P_{k}} $$

\end{enumerate}

\end{document}

