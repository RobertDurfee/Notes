\documentclass{article}
\usepackage{tikz}
\usepackage{float}
\usepackage{enumerate}
\usepackage{amsmath}
\usepackage{bm}
\usepackage{indentfirst}
\usepackage{siunitx}
\usepackage[utf8]{inputenc}
\usepackage{graphicx}
\graphicspath{ {Images/} }
\usepackage{float}
\usepackage{mhchem}
\usepackage{chemfig}
\allowdisplaybreaks

\title{ 6.042 Session 1 }
\author{ Robert Durfee }
\date{ February 7, 2018 }

\begin{document}

\maketitle

\section*{ Problem 1 }

\begin{enumerate}[a.]
  \item Arranging the right triangles into a $c \times c$ square:

    \begin{figure}[H]
      \centering
      \includegraphics[scale=0.80]{"CByCSquare"}
      \caption{$c \times c$ Square}
    \end{figure}

    The smaller interior square is of dimensions $(b-a) \times (b-a)$.

  \item Rearranging the right triangles to form two triangles of dimensions $a
    \times a$ and $b \times b$.

    \bigbreak

    This shows that the area formed by the two
    arrangements are equal. As a result, the following equation always holds.
    $$ a^{2} + b^{2} = c^{2} $$

    \begin{figure}[H]
      \centering
      \includegraphics[scale=0.80]{"BByBAndAByASquare"}
      \caption{$a \times a$ and $b \times b$ Squares}
    \end{figure}

  \item This realization holds for all arbitrary lengths of sides because the
    relations were made using simple properties of triangles.

  \item Some facts used to reach this conclusion include the sum of
    complementary angles add up to be a right angle. Also, lining up two right
    angles results in a straight line. Another is that lengths along a line add
    up.

\end{enumerate}

\section*{Problem 2}

\begin{enumerate}[a.]
  \item $$ 1 = \sqrt{1} = \sqrt{(-1)(-1)} = \sqrt{-1} \sqrt{-1} = \left(
    \sqrt{-1} \right)^{2} = -1 $$

    This proof has an error in the third step. The rule that $ \sqrt{rt} =
    \sqrt{r}\sqrt{t}$ only holds for positive numbers, not negative.

  \item If $1 = -1$, then $2 = 1$ by dividing each side by $2$ and then adding
    $3/2$ to each side.

  \item
    $$ \sqrt{r}\sqrt{s} = \sqrt{\left( \sqrt{r} \sqrt{s} \right)^{2}} =
    \sqrt{\left( \sqrt{r} \right)^{2} \left( \sqrt{s} \right)^{2}} = \sqrt{rs}
    $$

\end{enumerate}

\section*{Problem 3}

\begin{enumerate}[a.]
  \item In the following proof, the error occurs in step 2. The logarithm has a
    negative value. As a result, the inequality sign should flip.

    \begin{align}
      3 &> 2 \\
      3 \log_{10} (1/2) &> 2 \log_{10} (1/2) \\
      \log_{10} (1/2)^{3} &> \log_{10} (1/2)^{2} \\
      (1/2)^{3} &> (1/2)^{2} \\
      1/8 &> 1/4
    \end{align}

  \item In the following proof, the error occurs in step 4. When squaring a
    value with units, the units must also be squared.

    \begin{figure}[H]
      \centering
      \includegraphics[scale=0.60]{"CentsBogusProof"}
    \end{figure}

  \item In the following proof, the error occurs in step 5. Since $a = b$, the
    difference between is $0$ and you cannot divide by $0$.

    \begin{align}
      \setcounter{equation}{0}
      a &= b \\
      a^{2} &= ab \\
      a^{2} - b^{2} &= ab - b^{2} \\
      (a - b)(a + b) &= (a - b)b \\
      a + b &= b \\
      a &= 0
    \end{align}

\end{enumerate}

\end{document}

