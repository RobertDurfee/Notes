\documentclass{article}
\usepackage{tikz}
\usepackage{float}
\usepackage{enumerate}
\usepackage{amsmath}
\usepackage{amsthm}
\usepackage{bm}
\usepackage{indentfirst}
\usepackage{siunitx}
\usepackage[utf8]{inputenc}
\usepackage{graphicx}
\graphicspath{ {Images/} }
\usepackage{float}
\usepackage{mhchem}
\usepackage{chemfig}
\allowdisplaybreaks

\title{ 6.042 Session 8 }
\author{ Robert Durfee }
\date{ February 23, 2018 }

\begin{document}

\maketitle

\section*{ Problem 1 }

\begin{enumerate}[a.]

  \item Proving through a series of if and only if statements:
    $$ (P \land \overline{Q}) \lor (P \land Q) \iff P \land (\overline{Q}
    \lor Q) \iff P \land T \iff P $$ \qed

  \item Note that this problem is identical to problem 1.
    \begin{align*}
      &x \in (A - B) \cup (A \cap B) \\
      &\iff (x \in A \land x \notin B) \lor (A \cap B) \\
      &\iff (x \in A \land x \notin B) \lor (x \in A \land x \in B) \\
      &\iff x \in A \land (x \notin B \lor x \in B) \\
      &\iff x \in A \land T \\
      &\iff x \in A
    \end{align*} \qed

\end{enumerate}

\section*{Problem 2}

The proof is by cases. Let $S$ be a four element set $\{a, b, c, d\}$. There are
three cases in this game:

\begin{enumerate}
  \item P1 picks a one element subset.
  \item P1 picks a two element subset.
  \item P1 picks a three element subset.
\end{enumerate}

\subsection*{Case 1: P1 Picks a One Element Subset}

If P1 picks one of the four elements, WLOG $\{a\}$, P2 picks $\{b, c, d\}$. The
game then becomes a three element game of $\{b, c, d\}$ and P2 wins.

\subsection*{Case 2: P1 Picks a Two Element Subset}

If P1 picks a two element subset, WLOG $\{a, b\}$, P2 picks the complement
subset, $\{c, d\}$. P2 continues in this way if P1 continues to pick two element
subsets. Now, P1 must pick a one element subset, WLOG $\{a\}$. Then P2 picks
$\{b\}$. Now the game is reduced to a two element game of $\{c, d\}$ and P2
wins.

\subsection*{Case 3: P1 Picks a Three Element Subset}

If P1 picks a three element subset, WLOG $\{a, b, c\}$, P2 picks $\{d\}$. The
game becomes a three element game of $\{a, b, c\}$ and P2 wins.

\bigbreak

This exhausts all cases and P2 wins in all cases. \qed

\section*{Problem 3}

\begin{enumerate}[a.]
  \item $(a, b)$ has and order while $\{a, b\}$ does not.
  \item If $a = \{1\}$ and $b = 2$, then $(a, b) = (\{1\}, 2)$, but
    $\{a,{b}\} = \{\{a\},\{b\}\}$. This is ambiguous.
  \item A solution to this problem would be $\{a,\{a, b\}\}$. This can be
    extended to three dimensions, $\{a,\{a,b\},\{\{a,b\},c\}\}$.
\end{enumerate}

\end{document}

