\documentclass{article}
\usepackage{tikz}
\usepackage{float}
\usepackage{enumerate}
\usepackage{amsmath}
\usepackage{bm}
\usepackage{indentfirst}
\usepackage{siunitx}
\usepackage[utf8]{inputenc}
\usepackage{graphicx}
\graphicspath{ {Images/} }
\usepackage{float}
\usepackage{mhchem}
\usepackage{chemfig}
\allowdisplaybreaks

\title{ 8.01 Problem Set 11 }
\author{ Robert Durfee - Lecture 7 - Table 9 }
\date{ November 27, 2017 }

\begin{document}

\maketitle

\section{ Cubical Block Collision with Low Ridge }

\subsection*{ Part A }

Since the axis of rotation is not around the center of mass, the moment of
interia about the edge of the cube must be computed using the paralell axis
theorem.
\begin{align*}
    I &= I_{cm} + md^{2} \\
    I &= \frac{1}{6} ms^{2} + \frac{1}{\sqrt{ 2 }} ms^{2} \\
    I &= \frac{2}{3} ms^{2}
\end{align*}

Using an axis along the horizontal surface, angular momentum is conserved:
\begin{align*}
    \frac{1}{2} mvs &= \frac{2}{3} ms^{2}\omega \\
    \omega &= \frac{3v}{4s}
\end{align*}

\subsection*{ Part B }

After the collision, the mechanical energy of the block is conserved. Thus,
using the angular speed calculated in Part B:
\begin{align*}
    \frac{1}{2} I\omega^{2} + mgh_{cm,0} &= mgh_{cm,f} \\
    \frac{3}{16} msv^{2} + \frac{1}{2} mgs &= \frac{1}{\sqrt{ 2 }} mgs \\
    v &= 2 \sqrt{ \frac{2gs}{3} \left( \sqrt{ 2 } - 1 \right) }
\end{align*}

\section{ Yo-Yo Rolling on Inclined Plane }

Before the yo-yo starts slipping, we can assume that it rolls clockwise, in the
direction of the applied force. As a result, positive angular acceleration is
clockwise and positive linear acceleration is up the plane. Thus, the force and
torque equations are as follows:
$$ R\mu_{s}mg \cos \left( \phi \right) - \frac{ R }{ 3 }F &= \frac{ 1 }{ 2 } mRa $$
$$ F - mg \sin \left( \phi \right) - \mu_{s}mg \cos \left( \phi \right) &= ma $$

Solving this system of equations for $F$ results in:
$$ \frac{ 3 }{ 5 } \left( 3 \mu_{s} \cos \left( \phi \right) - \sin \left( \phi
\right) \right) = F$$

\section{ Billiards }

\subsection*{ Part A }



\subsection*{ Part B }



\end{document}

